\usepackage{amssymb,amsmath,stmaryrd,txfonts,mathrsfs,amsthm}
\usepackage{comment}
\usepackage[neveradjust]{paralist}
\usepackage{mathtools}
\usepackage{multirow}
\usepackage[outline]{contour}
\contourlength{1.2pt}
\usepackage{tikz}
\usetikzlibrary{intersections,arrows.meta,calc,quotes,math,decorations.pathreplacing,decorations.markings,cd,arrows,positioning,fit,matrix,
shapes.geometric,external,decorations.pathmorphing,backgrounds,circuits,circuits.ee.IEC,shapes}
\usepackage[a4paper,top=3cm,bottom=3cm,inner=3cm,outer=3cm]{geometry}
\usepackage[foot]{amsaddr}
\usepackage{graphicx}
\usepackage{stackrel}

% colors
\usepackage{xcolor}
\usepackage{framed,color}

\definecolor{shadecolor}{rgb}{1,0.8,0.3}
\definecolor{myurlcolor}{rgb}{0.5,0,0}
\definecolor{mycitecolor}{rgb}{0,0,0.7}
\definecolor{myrefcolor}{rgb}{0,0,0.7}
\definecolor{hyperrefcolor}{rgb}{0.5,0,0}

% hyperlinks

\usepackage{hyperref}
\hypersetup{
	colorlinks,
	linkcolor={hyperrefcolor},
	citecolor={hyperrefcolor},
	urlcolor={hyperrefcolor}
}

\newcommand{\backref}[1]{(Referred to on page #1.)}

\usepackage[draft]{fixme}
\usepackage[capitalize]{cleveref}
\crefname{equation}{}{}
\crefname{defn}{Definition}{Definitions}
\crefname{thm}{Theorem}{Theorems}
\crefname{lem}{Lemma}{Lemmas}

\definecolor{rewritecolor}{rgb}{0,.9,1}
\tikzset{rewritenode/.style={shape=circle,fill=rewritecolor,scale=0.25,font=\Huge}}
\tikzset{RWopen/.style={shape=circle,draw=black,fill=white,scale=0.5,font=\Huge}}
\tikzset{RWclosed/.style={shape=circle,fill=black,scale=0.5,font=\Huge}}
\tikzset{CDnode/.style={shape=circle,fill=white,scale=.5}}
\makeatletter
\let\ea\expandafter

\pgfdeclarelayer{edgelayer}
\pgfdeclarelayer{nodelayer}
\pgfsetlayers{edgelayer,nodelayer,main}

% Petri nets
\definecolor{lblue}{rgb}{0,250,255}
\tikzstyle{species}=[circle,fill=yellow,draw=black,scale=1.15]
\tikzstyle{transition}=[rectangle,fill=lblue,draw=black,scale=1.15]
\tikzstyle{inarrow}=[->, >=stealth, shorten >=.03cm,line width=1.5]
\tikzstyle{empty}=[circle,fill=none, draw=none]
\tikzstyle{inputdot}=[circle,fill=purple,draw=purple, scale=.25]
\tikzstyle{inputarrow}=[->,draw=purple, shorten >=.05cm]
\tikzstyle{simple}=[-,draw=purple,line width=1.000]
\tikzstyle{none}=[inner sep=0pt]

\tikzset{->-/.style={decoration={
  markings,
  mark=at position .5 with {\arrow{>}}},postaction={decorate}}}

%% Defining commands that are always in math mode.
\def\mdef#1#2{\ea\ea\ea\gdef\ea\ea\noexpand#1\ea{\ea\ensuremath\ea{#2}}}
\def\alwaysmath#1{\ea\ea\ea\global\ea\ea\ea\let\ea\ea\csname your@#1\endcsname\csname #1\endcsname
  \ea\def\csname #1\endcsname{\ensuremath{\csname your@#1\endcsname}}}

\mdef\fahat{\hat{\fa}}

% MISCELLANEOUS SYMBOLS
\newcommand{\inv}{^{-1}}
\newcommand{\op}{^{\mathit{op}}}
\newcommand{\co}{^{\mathit{co}}}
\newcommand{\coop}{^{\mathit{coop}}}
\newcommand{\id}{\mathrm{id}}
\let\adj\dashv
\newcommand{\pullbackcorner}[1][dr]{\save*!/#1-1.2pc/#1:(-1,1)@^{|-}\restore}
\let\iso\cong
\let\eqv\simeq
\let\cng\equiv
\alwaysmath{ell}
\alwaysmath{infty}
\alwaysmath{odot}
\def\frc#1/#2.{\frac{#1}{#2}}   % \frc x^2+1 / x^2-1 .
\mdef\ten{\mathrel{\otimes}}
\newcommand{\simrightarrow}{\xrightarrow{\raisebox{-3pt}[0pt][0pt]{\ensuremath{\sim}}}}

\newcommand{\N}{\mathbb{N}}
\newcommand{\R}{\mathbb{R}}
\newcommand{\Ob}{\mathrm{Ob}}
\newcommand{\Mor}{\mathrm{Mor}}

\newcommand*{\graysquare}{\textcolor{lightgray}{\blacksquare}}


%% OPERATORS
\DeclareMathOperator\colim{colim}
\DeclareMathOperator\eq{eq}
\DeclareMathOperator\Aut{Aut}
\DeclareMathOperator\End{End}
\DeclareMathOperator\Hom{Hom}
\DeclareMathOperator\Map{Map}

%% ARROWS
% \to already exists
\newcommand{\too}[1][]{\ensuremath{\overset{#1}{\longrightarrow}}}
\newcommand{\oot}[1][]{\ensuremath{\overset{#1}{\longleftarrow}}}
\let\toot\rightleftarrows
\let\otto\leftrightarrows
\let\maps\colon

%% EXTENSIBLE ARROWS
\let\xto\xrightarrow
\let\xot\xleftarrow

% THEOREM-TYPE ENVIRONMENTS, hacked to
%% (a) number all with the same numbers, and
%% (b) have the right names for autoref
\def\defthm#1#2{%
  \newtheorem{#1}{#2}[section]%
  \expandafter\def\csname #1autorefname\endcsname{#2}%
  \expandafter\let\csname c@#1\endcsname\c@thm}
\newtheorem{thm}{Theorem}[section]
\newcommand{\thmautorefname}{Theorem}
\defthm{cor}{Corollary}
\defthm{prop}{Proposition}
\defthm{lem}{Lemma}
\defthm{conj}{Conjecture}
\defthm{hyp}{Hypothesis}
\defthm{fact}{Fact}
\theoremstyle{definition}
\defthm{defn}{Definition}
\defthm{notn}{Notation}
\theoremstyle{remark}
\defthm{rmk}{Remark}
\defthm{eg}{Example}

\newcommand{\fhat}{\ensuremath{\hat{f}}}

% Also number formulas with the theorem counter
%\let\c@equation\c@thm
%\numberwithin{equation}{section}

% Only show numbers for equations that are actually referenced (or
% whose tags are specified manually) - uses mathtools. %Christina: for some reason, the command below made it impossible to label any equation!
%\mathtoolsset{showonlyrefs,showmanualtags}

\def\tobar{\mathrel{\mkern3mu  \vcenter{\hbox{$\scriptscriptstyle+$}}%
                    \mkern-12mu{\to}}}

%\input{decls}
%\UseAllTwocells

\mdef\fchk{\check{f}}

\definecolor{purple(x11)}{rgb}{0.5, 0.0, 0.5}
\def\purple{\color{purple(x11)}}
\def\chris{\purple}
\newcommand{\upback}{\mathbin{\rotatebox[origin=c]{135}{$\ulcorner$}}}

% xypic
\usepackage[all,2cell,cmtip]{xy}
%\input{decls}
\UseAllTwocells

% sets

\newcommand{\La}{\mathcal{L}}

% categories

\newcommand{\ca}{\mathsf}
\newcommand{\Set}{\mathsf{Set}}
\newcommand{\Graph}{\mathsf{Graph}}
\newcommand{\Dynam}{\mathsf{Dynam}}
\newcommand{\Mark}{\mathsf{Mark}}
\newcommand{\Petri}{\mathsf{Petri}}
\newcommand{\B}{\mathsf{B}}
\newcommand{\A}{\mathsf{A}}
\newcommand{\C}{\mathsf{C}}
\newcommand{\J}{\mathsf{J}}
\newcommand{\T}{\mathsf{T}}
\newcommand{\D}{\mathsf{D}}
\newcommand{\X}{\mathsf{X}}
\newcommand{\Y}{\mathsf{Y}}
\newcommand{\Z}{\mathsf{Z}}
\newcommand{\Lag}{\mathsf{Lag}}
\newcommand{\SemiAlg}{\mathsf{SemiAlg}}
\newcommand{\Fin}{\mathsf{Fin}}
\newcommand{\Rel}{\mathsf{Rel}}
\newcommand{\Cospan}{\mathsf{Cospan}}
\newcommand{\Csp}{\mathsf{Csp}}
\newcommand{\Circ}{\mathsf{Circ}}
\newcommand{\CMC}{\mathsf{CMC}}
\newcommand{\RxNet}{\mathsf{RxNet}}
\newcommand{\one}{\mathsf{1}}

% bicategories

\newcommand{\bicat}{\mathbf}
\newcommand{\Dbl}{\bicat{Dbl}}
\newcommand{\bCsp}{\bicat{Csp}}
\newcommand{\bA}{\bicat{A}}
\newcommand{\bB}{\bicat{B}}
\newcommand{\bX}{\bicat{X}}
\newcommand{\bY}{\bicat{Y}}
\newcommand{\bD}{\bicat{D}}
\newcommand{\Cat}{\bicat{Cat}}
\newcommand{\MonCat}{\bicat{MonCat}}
\newcommand{\Rex}{\bicat{Rex}}
\newcommand{\SMC}{\bicat{SymMonCat}}
\newcommand{\OpICat}{\bicat{OpICat}}%probably too much to use?
\newcommand{\OpFib}{\bicat{OpFib}}


% double categories

\newcommand{\double}[1]{\mathbf{\mathbb #1}}
\newcommand{\lCsp}{\double{Csp}}
\newcommand{\lCospan}{\double{Cospan}}
\newcommand{\lOpen}{\double{Open}}
\newcommand{\lA}{\double{A}}
\newcommand{\lB}{\double{B}}
\newcommand{\lC}{\double{C}}
\newcommand{\lD}{\double{D}}
\newcommand{\lE}{\double{E}}
\newcommand{\lF}{\double{F}}
\newcommand{\lG}{\double{G}}
\newcommand{\lH}{\double{H}}
\newcommand{\lR}{\double{R}}
\newcommand{\lX}{\double{X}}
\newcommand{\lY}{\double{Y}}

% this puts definitions in boldface

\newcommand{\define}[1]{{\bf \boldmath{#1}}}

% other gunk

\newcommand{\dcsp}[1]{{#1}\mathbb{C}\textnormal{sp}}
\newcommand{\scsp}[2]{_{#1}\mathbb{C}\textnormal{sp}\left(#2\right)}
\tikzset{tick/.style={postaction={decorate,decoration={markings,
mark=at position 0.4 with {\draw[-] (0,.4ex) -- (0,-.4ex);}}}}}
\newcommand{\tickar}{\begin{tikzcd}[baseline=-0.5ex,cramped,sep=small,ampersand replacement=\&]{}\ar[r,tick]\&{}\end{tikzcd}}
\newcommand{\cspn}[5]{\begin{tikzcd}[baseline=-0.5ex,cramped,sep=small,ampersand replacement=\&]{#1}\ar[r,"#4"] \& {#2} \& {#3}\ar[l,"#5"']\end{tikzcd}}

\newcommand{\inta}{\raisebox{.3\depth}{$\smallint\hspace{-.01in}$}}
\newcommand{\pse}{\mathrm{ps}}
\newcommand{\mapseqq}{:=}
\newcommand{\ot}{\otimes}
\newcommand{\To}{\Rightarrow}

