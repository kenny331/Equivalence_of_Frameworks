\usepackage{amssymb,amsmath,stmaryrd,txfonts,mathrsfs,amsthm}

\usepackage[all,2cell]{xy}\UseAllTwocells\SilentMatrices
\usepackage[neveradjust]{paralist}
\usepackage{hyperref}
\usepackage{mathtools}
\usepackage{multirow}
\usepackage[outline]{contour}
\contourlength{1.2pt}
\usepackage{tikz}
\usepackage{xcolor}
\usepackage{framed,color}
\usepackage[draft]{fixme}
\usetikzlibrary{intersections,decorations.markings}
\usetikzlibrary{cd,arrows,positioning,fit,matrix,shapes.geometric,external,decorations.pathmorphing}
\usetikzlibrary{backgrounds,circuits,circuits.ee.IEC,shapes}
\usepackage{comment}
\usepackage[capitalize]{cleveref}
\crefname{equation}{}{}
\definecolor{rewritecolor}{rgb}{0,.9,1}
\tikzset{rewritenode/.style={shape=circle,fill=rewritecolor,scale=0.25,font=\Huge}}
\tikzset{RWopen/.style={shape=circle,draw=black,fill=white,scale=0.5,font=\Huge}}
\tikzset{RWclosed/.style={shape=circle,fill=black,scale=0.5,font=\Huge}}
\tikzset{CDnode/.style={shape=circle,fill=white,scale=.5}}
\makeatletter
\let\ea\expandafter

\pgfdeclarelayer{edgelayer}
\pgfdeclarelayer{nodelayer}
\pgfsetlayers{edgelayer,nodelayer,main}

% Petri nets
\definecolor{lblue}{rgb}{0,250,255}
\tikzstyle{species}=[circle,fill=yellow,draw=black,scale=1.15]
\tikzstyle{transition}=[rectangle,fill=lblue,draw=black,scale=1.15]
\tikzstyle{inarrow}=[->, >=stealth, shorten >=.03cm,line width=1.5]
\tikzstyle{empty}=[circle,fill=none, draw=none]
\tikzstyle{inputdot}=[circle,fill=purple,draw=purple, scale=.25]
\tikzstyle{inputarrow}=[->,draw=purple, shorten >=.05cm]
\tikzstyle{simple}=[-,draw=purple,line width=1.000]
\tikzstyle{none}=[inner sep=0pt]

\definecolor{shadecolor}{rgb}{1,0.8,0.3}
\definecolor{myurlcolor}{rgb}{0.6,0,0}
\definecolor{mycitecolor}{rgb}{0,0,0.8}
\definecolor{myrefcolor}{rgb}{0,0,0.8}
\hypersetup{colorlinks, linkcolor=myrefcolor, citecolor=mycitecolor, urlcolor=myurlcolor}

\tikzset{->-/.style={decoration={
  markings,
  mark=at position .5 with {\arrow{>}}},postaction={decorate}}}

%% Defining commands that are always in math mode.
\def\mdef#1#2{\ea\ea\ea\gdef\ea\ea\noexpand#1\ea{\ea\ensuremath\ea{#2}}}
\def\alwaysmath#1{\ea\ea\ea\global\ea\ea\ea\let\ea\ea\csname your@#1\endcsname\csname #1\endcsname
  \ea\def\csname #1\endcsname{\ensuremath{\csname your@#1\endcsname}}}
\newcommand{\define}[1]{{\bf \boldmath{#1}}}

% blackboard bold letters
\newcommand{\lA}{\ensuremath{\mathbb{A}}}
\newcommand{\lC}{\ensuremath{\mathbb{C}}}
\newcommand{\lD}{\ensuremath{\mathbb{D}}}
\newcommand{\lE}{\ensuremath{\mathbb{E}}}
\newcommand{\lR}{\ensuremath{\mathbb{R}}}
\newcommand{\lX}{\ensuremath{\mathbb{X}}}
\mdef\fahat{\hat{\fa}}

% MISCELLANEOUS SYMBOLS
\newcommand{\inv}{^{-1}}
\newcommand{\op}{^{\mathit{op}}}
\newcommand{\co}{^{\mathit{co}}}
\newcommand{\coop}{^{\mathit{coop}}}
\newcommand{\id}{\mathm{id}}
\let\adj\dashv
\newcommand{\pullbackcorner}[1][dr]{\save*!/#1-1.2pc/#1:(-1,1)@^{|-}\restore}
\let\iso\cong
\let\eqv\simeq
\let\cng\equiv
\mdef\Id{\mathrm{Id}}
\mdef\id{\mathrm{id}}
\alwaysmath{ell}
\alwaysmath{infty}
\alwaysmath{odot}
\def\frc#1/#2.{\frac{#1}{#2}}   % \frc x^2+1 / x^2-1 .
\mdef\ten{\mathrel{\otimes}}

%% OPERATORS
\DeclareMathOperator\colim{colim}
\DeclareMathOperator\eq{eq}
\DeclareMathOperator\Aut{Aut}
\DeclareMathOperator\End{End}
\DeclareMathOperator\Hom{Hom}
\DeclareMathOperator\Map{Map}

%% ARROWS
% \to already exists
\newcommand{\too}[1][]{\ensuremath{\overset{#1}{\longrightarrow}}}
\newcommand{\oot}[1][]{\ensuremath{\overset{#1}{\longleftarrow}}}
\let\toot\rightleftarrows
\let\otto\leftrightarrows
\let\maps\colon

%% EXTENSIBLE ARROWS
\let\xto\xrightarrow
\let\xot\xleftarrow

% THEOREM-TYPE ENVIRONMENTS, hacked to
%% (a) number all with the same numbers, and
%% (b) have the right names for autoref
\def\defthm#1#2{%
  \newtheorem{#1}{#2}[section]%
  \expandafter\def\csname #1autorefname\endcsname{#2}%
  \expandafter\let\csname c@#1\endcsname\c@thm}
\newtheorem{thm}{Theorem}[section]
\newcommand{\thmautorefname}{Theorem}
\defthm{cor}{Corollary}
\defthm{prop}{Proposition}
\defthm{lem}{Lemma}
\defthm{conj}{Conjecture}
\defthm{hyp}{Hypothesis}
\defthm{fact}{Fact}
\theoremstyle{definition}
\defthm{defn}{Definition}
\defthm{notn}{Notation}
\theoremstyle{remark}
\defthm{rmk}{Remark}
\defthm{eg}{Example}

\newcommand{\fhat}{\ensuremath{\hat{f}}}

% Also number formulas with the theorem counter
%\let\c@equation\c@thm
%\numberwithin{equation}{section}

% Only show numbers for equations that are actually referenced (or
% whose tags are specified manually) - uses mathtools. %Christina: for some reason, the command below made it impossible to label any equation!
%\mathtoolsset{showonlyrefs,showmanualtags}

\def\tobar{\mathrel{\mkern3mu  \vcenter{\hbox{$\scriptscriptstyle+$}}%
                    \mkern-12mu{\to}}}

%\input{decls}
\UseAllTwocells

\newcommand{\dblcat}[1]{\mathbb{#1}}
\mdef\fchk{\check{f}}

\definecolor{purple(x11)}{rgb}{0.5, 0.0, 0.5}
\def\purple{\color{purple(x11)}}
\def\chris{\purple}

%Christina: change below accordingly if needed!
\newcommand{\ca}{\mathsf}
\newcommand{\bicat}{\mathbf}
\newcommand{\U}{U}
\newcommand{\D}{\ca{A}}
\newcommand{\C}{\ca{X}} 
\newcommand{\A}{\ca{A}}
\newcommand{\B}{\ca{B}}
\newcommand{\X}{\ca{X}}
\newcommand{\dcsp}[1]{{#1}\mathbb{C}\textnormal{sp}}
\newcommand{\scsp}[2]{_{#1}\mathbb{C}\textnormal{sp}\left(#2\right)}
\tikzset{tick/.style={postaction={decorate,decoration={markings,
mark=at position 0.4 with {\draw[-] (0,.4ex) -- (0,-.4ex);}}}}}
\newcommand{\tickar}{\begin{tikzcd}[baseline=-0.5ex,cramped,sep=small,ampersand replacement=\&]{}\ar[r,tick]\&{}\end{tikzcd}}
\newcommand{\cspn}[5]{\begin{tikzcd}[baseline=-0.5ex,cramped,sep=small,ampersand replacement=\&]{#1}\ar[r,"#4"] \& {#2} \& {#3}\ar[l,"#5"']\end{tikzcd}}
\newcommand{\OpICat}{\bicat{OpICat}}%probably too much to use?
\newcommand{\pse}{\mathrm{ps}}
\newcommand{\OpFib}{\bicat{OpFib}}
\newcommand{\inta}{\raisebox{.3\depth}{$\smallint\hspace{-.02in}$}}
\newcommand{\Cocart}{\mathrm{Cocart}}
