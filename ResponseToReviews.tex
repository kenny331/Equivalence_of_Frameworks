\documentclass[reqno]{amsart}

\usepackage{amssymb,amsmath,stmaryrd,txfonts,mathrsfs,amsthm}%,undertilde}
\usepackage{comment}
\usepackage[neveradjust]{paralist}
\usepackage{mathtools}
\usepackage{multirow}
\usepackage[outline]{contour}
\contourlength{1.2pt}
\usepackage{tikz}
\usetikzlibrary{intersections,arrows.meta,calc,quotes,math,decorations.pathreplacing,decorations.markings,cd,arrows,positioning,fit,matrix,
shapes.geometric,external,decorations.pathmorphing,backgrounds,circuits,circuits.ee.IEC,shapes}
\usepackage[a4paper,top=2cm,bottom=2cm,inner=3cm,outer=3cm]{geometry}
\usepackage[foot]{amsaddr}
\usepackage{graphicx}
\usepackage{stackrel}

% colors
\usepackage{xcolor}
\usepackage{framed,color}

\def\chris{\color{purple} Christina: }
\def\john{\color{red} John: }
\def\kenny{\color{blue} Kenny: }

\definecolor{shadecolor}{rgb}{1,0.8,0.3}
\definecolor{myurlcolor}{rgb}{0.5,0,0}
\definecolor{mycitecolor}{rgb}{0,0,0.7}
\definecolor{myrefcolor}{rgb}{0,0,0.7}
\definecolor{hyperrefcolor}{rgb}{0.5,0,0}

\usepackage{hyperref}
\hypersetup{
	colorlinks,
	linkcolor={hyperrefcolor},
	citecolor={hyperrefcolor},
	urlcolor={hyperrefcolor}
}

\newcommand{\backref}[1]{(Referred to on page #1.)}

\usepackage[draft]{fixme}
\usepackage[capitalize]{cleveref}

\title{Response to reviews for ``Structured versus decorated cospans''}

\author{John\ C.\ Baez, Kenny Courser, and Christina Vasilakopoulou}

\begin{document}

\maketitle

\setcounter{tocdepth}{2}
\tableofcontents

\section{Reviews - to be removed}

\subsection{First Review}

{\footnotesize

Decorated and structured cospans have been used to model open systems in the work of Fong, Baez, Courser, Master and others.

Previous work by Shulman constructed symmetric monoidal double categories functorially from bifibrations. The present manuscript recovers structured 
and decorated cospan categories as instances of this construction and then compares both using the functoriality of that same construction. The crux 
of many of the proofs is in showing the Beck-Chevalley condition required to apply Shulman’s result. All of this could feel technical but the results 
are detailed and clear. Some of the technology used here is very interesting by itself: monoidal fibrations or the construction of (symmetric 
monoidal) double categories, bicategories and categories.

From the applied point of view, they summarize many other papers in structured and decorated cospans and discuss in which cases both approaches 
coincide. It feels that this should not be this difficult to study for the 1-categorical structure (and indeed it seems to have appeared previously on 
the literature), but here it is done for the symmetric monoidal double category structure.

As the only point against, I feel there is a growing number of “cospan” articles. The present work seems like it could serve as a comparison of all of 
these, but perhaps that comparison could be more explicit and detailed: what is it done in each one of these “cospan” articles? is there anything that 
is not already explained in this one? If some of the articles about decorated cospans were found to contain errors, is this the only construction of 
decorated cospans that solves these errors?

Section 1 motivates decorated cospans in terms of systems and interfaces.

Section 2 recovers a generalization of decorated cospans from Shulman’s construction. The most important part is to show the Beck-Chevalley property 
of the fibration at hand. In Section 3, the construction of structured cospans is recovered from Shulman’s double category using a fibration from the 
comma category of a finite colimit preserving functor L : A -> X. Again, the most important part is to show that the fibration is Beck-Chevalley. Both 
constructions (decorated and structured) are treated in a uniform way, which I see as a very positive contribution. This is used in Section 4 to give 
the comparison.

Section 5 maps double categories to bicategories and categories. For this, they use a result by Hansen and Shulman that constructs them functorially 
from fibrant double categories.

Section 6 explains the problems with Set-decoration for open graphs and justifies the need for Cat-decoration. The best argument for switching Set to 
Cat in the definition of decorated cospan is actually detailed in “Structured Cospans” by Baez and Courser (§5), where they show that the notion of 
isomorphism in graph decorated cospans is too restrictive. Section 6.1 explains how, with theorem contibuted by the paper, decorated and structured 
cospans coincide on the right notion of graph isomorphism.

Section 6.2 seems to simply repeat the same procedure after labelling edges, and Section 6.3 repeats again with Petri nets. Finally, Section 6.4 shows 
something different: a category of open dynamical systems that can be described as a decorated cospan category but that cannot be equivalently 
described as a structured cospan category.

The Conclusions section explains how, if we restrict to a suitable class of opfibrations or one-side inverse adjunctions, we would get the same 
construction from decorated and structured cospans.

In my understanding, the value of this work resides in how it systematically compares the approach of decorated and structured cospans. I would 
highlight the care put in each construction, and how it builds upon previous work for most of the heavy lifting.

Suggested changes, corrections, and general comments.

Lemma 2.4: when talking about the functors $F : (A,\otimes)\to(Cat,\times)$, I think sometimes it is specified that they are lax and sometimes it is 
not. Perhaps it would be clearer to always use “lax monoidal” or to just use “monoidal” to mean lax monoidal.

Theorem 4.1: perhaps it would be clearer to say it “factors through R : Rex -> SymMonCat”.

Proof of Lemma 4.4: there is a typo where “L then becomes left adjoint to R” should probably be “L then becomes left adjoint to U”.

After Theorem A.2: In order to be consistent, it would be practical to state explicitly what is a bifibration.

Beck-Chevalley condition, after Theorem A.2: Using $X_a$ instead of $C_a$ for the fibers on the diagram would be more consistent with previous 
notation.

COVID paragraph: I appreciate the effort into explaining the applied aspect. However, claims about the software (it is “easy”, we “easily” modify, we 
“better understand”) sound too vague (is it computationally better? do we assume that the final user understands structured cospans? are we comparing 
it to other Petri net editors?). Perhaps this paragraph could be fused as a comment to the previous one, which is already a very good justification 
for the use of Petri nets with rates.

Section 6: as a reader, I really appreciate the examples: are there more in the literature? I do not think it is necessary to go in detail over each 
one of them, but perhaps it would be useful to point to other places where they are described.}

\subsection{Second review}

{\footnotesize

Overview

This is an important topic and brings clarity to an open question in the ACT field. The fact that structured and decorated cospans are equivalent in 
many cases is a good thing, and the fact that the authors develop a sufficient condition for proving this equivalence is highly relevant to this 
journal. Having the description of decorated and structured cospans in the same place with a consistent presentation is very useful for the community. 
My review focuses on how I think this work fits into the broader context of ACT research in this area. I recommend publication after addressing the 
following concerns.

Suggested changes, corrections, and general comments.

General Thoughts

Given the title and abstract of this paper “Structured vs Decorated Cospans” I would expect this paper to provide a judgment that structured cospans 
are strictly better than decorated cospans in all circumstances and a justification for that claim, or a detailed analysis of the pros and cons of 
structured vs decorated cospans, some case studies in applications that show how the difference between these double categories is felt in the 
application domain, and a some description of the tradeoffs. This paper should come with some practical advice for the applied category theorist on 
how to choose whether to model their application with structured vs decorated cospans. Section 6 does a good job of detailing the various 
constructions of combinatorial structures that get double categories of structured cospans. The graph and Petri net examples are implemented in 
AlgebraicJulia, specifically the Catlab.Graphs module and the AlgebraicPetri.jl package. Given that Petri nets with structured cospans works, and 
Reaction Networks work, but open dynamical systems do not, I would like some more explanation of the differences here. The examples of graph, labeled 
graph, Petri net, reaction net are all examples of attributed C-Sets [1], the structured cospan construction in provided for arbitrary attributed 
C-Sets in Catlab.CategoricalAlgebra [2]. The open dynamical systems are not attributed C-Sets. All the examples where decorated and structured cospans 
agree are combinatorial, and the only example where they are different is not combinatorial. This gives me the impression that for categories of 
presheaves over a finitely presented category, structured cospans and decorated cospans are always equivalent. The introduction of numerical 
attributes (as in the case of reaction networks) does not appear to interfere with that pattern.

The examples in section 6 would be much stronger if the others included a theorem that all (or finitary) categories of presheaves satisfy the 
conditions of theorem 4.1 or a counter example illustrating that the situation is more complex than algebraic vs analytic distinction between open 
presheaves and the category of open dynamical systems. Without this analysis, I am left with the intuition that for discrete structures like finitary 
presheaves over a finitely presented schema, the decorated and structured cospans are going to be equivalent and for continuous mathematical 
structures like dynamical systems they will be different. If that intuition is correct, the paper should say something about it. If the intuition is 
incorrect, then additional explanation should be provided. I think that the implementation in [2] uses the left and right adjoints L and U in Theorem 
4.1, which is why I think that every category of presheaves should satisfy the main theorem. Hopefully it is easy to check and provide some commentary 
on it in this paper.

The interaction of compositional system behavior and the main theorem is also interesting and worth exploring or commenting on in this paper. If we 
take the categorical systems theory perspective advanced by section 6, and the body of the authors’ recent work on functorial semantics for complex 
systems, then we should ask why the decorated and structured cospan double categories are isomorphic for reaction networks, but their semantics as 
dynamical systems are a decorated cospan category that is not equivalent to any structured cospan category. Is there a different construction of 
semantics for reaction networks that uses structured cospans? The current work does not address the possibility of alternative semantics for 
structured cospans of systems. We can only say that DCsp is not equivalent to any structured cospan category, but is there a different decorated 
cospan category that would provide meaningful functorial semantics that is equivalent to some category of structured cospans. The structured cospan 
view is better for implementing the composition of the system specifications (reaction networks) but the current literature only uses decorated 
cospans for providing the systems with behaviors. If the authors can discuss this issue it would greatly enhance the utility of this paper for shaping 
the research discussion around compositional complex systems.

The conclusions refer to a forthcoming paper. The only conclusion that refers to the work presented in this paper is that theorem 4.1 gives conditions 
under which a category of decorated cospans is isomorphic to a category of structured cospans. Given that the audience for this paper is intended to 
be applied category theorists, this is a place where I would expect some guidance on when an ACT practitioner should choose either structured or 
decorated cospans depending on what mathematical structures they are formalizing within ACT.

Specific Concerns

The development of the AlgebraicPetri software system is credited to Halter and Patterson based on a blog post. The correct citation is to the 
software itself and should be credited to the entire AlgebraicPetri team based on a citation to the repository [3]. The additional authors would be 
Andrew Baas and James Fairbanks.

Several points in the text are written in a way style that can be interpreted as the dismissive style common to many discussions of Category Theory. I 
would recommend rewriting at least the following points that should be made in a more collegial tone.

pg. 2: “But this hope is doomed for reasons painstakingly explained in [1, Section 5] … despite all the tricks one might try.” This paragraph provides 
no insight into the reasons that those tricks fail. Referring them to tricks is dismissive. The citation is to a paper in TAC by the current authors.

pg. 3: “Many of the flawed applications of decorated cospans have been fixed using structured cospans.” This dismisses the work of other scholars 
using the decorated cospan approach as flawed. The alleged flaw should be made more precise. What about the decorated cospan approach is inferior to 
the structured cospan approach? The goal of this paper should be to illustrate to an applied category theory audience the differences between 
decorated and structured cospans. Referring back to previous work of the authors to make that point does not provide value.

pg. 27: To show that U does not have a left adjoint, we use the following well-known result. For readers of the TAC paper previously cited, that might 
be a well known result that does not require a citation, but my understanding of the Compositionality audience assumed that the ACT work was targeted 
readers that are more familiar with the applications and thus cannot be assumed to know such results.}

\begin{verbatim}
[1] https://arxiv.org/abs/2106.04703
[2] https://github.com/AlgebraicJulia/Catlab.jl/blob/f06417de29421441c4133c1cfb75b39ae8aa7142/src/categorical_algebra/StructuredCospans.jl
[3] https://github.com/AlgebraicJulia/AlgebraicPetri.jl
\end{verbatim}

\subsection{Third review}

{\footnotesize
Decorated and structured cospans have emerged as a mainstay of the current wave of ACT, the latter correcting certain technical problems with the 
former. This paper poses two questions:

    How can the technical problems with decorated cospans be fixed?

    How does the corrected version of decorated cospans compare with structured cospans?

The paper offers the following answers:

The symmetric lax monoidal functor (A,+) -> (Set, $\times$) defining a category of isomorphism classes of decorated cospans is replaced by a 
symmetric lax monoidal pseudofunctor (A,+) -> (Cat, $\times$) defining a symmetric monoidal double category of decorated cospans.

A sufficient condition, the main Theorem 4.1, is provided for a double category of decorated cospans to be equivalent, indeed isomorphic, to a double 
category of structured cospans. Several examples, as well as a nonexample, of applying the theorem are given.

The paper is a technical one, giving slick but abstract constructions and proofs using machinery developed by Shulman for turning monoidal 
bifibrations into double categories, and it serves more to clarify and compare existing ACT methods than to introduce new ones. The mathematical 
exposition is, however, very strong, and a fairly wide range of readers should be able to take something away from the paper. Given the central status 
of decorated/structured cospans within ACT, it was important that the above questions be addressed, and the authors have done so admirably. Also, the 
constructions and results are “black-boxed” in such a manner that practioners will be able to use them without understanding in detail the proofs that 
support them. I therefore consider this paper to be a strong contribution to the scholarship of ACT and recommend that it be published.

Suggested changes, corrections, and general comments.

Following are some questions and minor suggested changes.

p. 2. “Unfortunately, many applications of decorated cospans were flawed… for reasons painstaking explained in [1]”

Since fixing these problems is a major motivation for the whole paper, it would be good to explain what they are in slightly more detail, rather than 
just giving a reference. This could be done in a one paragraph or less. The gist of it, as I recall, is that in the original approach to decorated 
cospans, the isomorphism classes of open graphs depend on the identities of the edge elements, which is clearly wrong.

p. 4, following statement of Thm 2.1.

The diagram depicting the decoration morphism $\tau$ as a natural transformation, first shown explictly on p. 10, should be moved here since it is 
implicitly used at the beginning of p. 5 to define the decoration of horizontal composites.

p. 9, ingredients for constructing a double functor of decorated cospan double categories.

Does the natural transformation $\theta$: EF => F’ H have to be a natural isomorphism? Quite possibly I missed something, but I don’t see where in 
the following proofs that assumption is used.

p. 21, definition of F: Set -> Cat

The use f, g for vertex/edge maps is opposite of the convention used on the previous page. I suggest that the same convention be used for both the 
decorated and structured approach to constructing open graphs.

p. 27, end of Section 6.

It seems that the left adjoint needed for structured cospans fails to exist because the chosen category of dynamical systems does not have enough 
morphisms. Relatedly, this is the only example where the pseudofunctor D: (FinSet,+) -> (Cat, $\times$) comes from a simple functor D: (FinSet,+) -> 
(Set, $\times$) by viewing sets as discrete categories. In the other examples, this would count as using decorated cospans in the old, flawed way. 
All of this suggests the question: is there a way of constructing a category of dynamical systems that has more morphisms, so that a connection with 
structured cospans can be established?

p. 30, following Thm A.2

If I am not mistaken, “fibration”, “opfibration,” “monoidal (op/bi)fibration” are all defined, but the definition of “bifibration” is omitted. For 
completeness, it should be included.

A last general comment is that the paper unfortunately adopts the strategy, all too common in math papers, of putting all the examples in a section at 
the end of the paper. I think the readability of the paper would be improved, without much change to the organization, if the motivating example of 
open graphs were added to the relevant places in Sections 2 and 3. The remaining examples could be left in Section 6.
}

\subsection{Fourth review}

{\footnotesize
What are the main questions posed by the manuscript, and how does it answer them? Why or why not is this work a significant technical or conceptual 
contribution to scholarship?

Decorated cospans and structured cospans are two different tools for building categories of open systems, both of which have been applied in several 
existing papers (and this reviewer believes they will continue to grow in importance over time). This paper explains the relationship between them, in 
a very elegant way using the equivalence between fibered and indexed categories, to the extent that the two constructions now appear as different 
perspectives on the same fundamental idea. To me this relationship appears obvious in retrospect, and I mean this as a compliment to the paper. I 
think this paper is an important contribution to categorical systems theory, and I recommend acceptance.

Suggested changes, corrections, and general comments.

Composition of morphisms f, g in this paper is usually denoted gf, and occasionally gof. I’d suggest mentioning this in the “conventions” section.

p3: “Using the Grothendieck construction” - maybe worth mentioning this is the “covariant” version which is different to the most common convention 
(at least I think so, but the literature seems to be very split over this)

Theorem 2.1: Maybe worth saying explicitly that s is an object of the category F(m) rather than using $\in$, since this is exactly the distinction to 
Fong’s original definition

Proof of theorem 2.1: You should probably at least handwave that the cartesian liftings really are cartesian liftings. (No doubt it’s easy, but this 
is a classic sort of spot where errors can hide)

On page 9 during the construction of H, I would appreciate a sentence giving some intuition for the functor E, at this point I’m confused about its 
role.

Theorem 2.5: F should be defined on A, not A’

I think it would be worth saying in words a bit more prominently what the hypothesis of theorem 4.1 roughly means, namely that the fibers of the total 
category have finite colimits, reindexing preserves them, and the monoidal product on the fibers is cocartesian.

I wonder if the “problem” where DCsp is not a structured cospan category could be “fixed” by upgrading the functor from Set to Cat in a non-discrete 
way, for example with an ordering with the zero vector field as the bottom element. I suppose this would give you different 2-cells in the resulting 
double category or bicategory, but perhaps it would yield the same 1-category}

\section{Our Answer}


\begin{enumerate}

\item As the only point against, I feel there is a growing number of “cospan” articles. The present work seems like it could serve as a comparison of 
all of 
these, but perhaps that comparison could be more explicit and detailed: what is it done in each one of these “cospan” articles? is there anything 
that 
is not already explained in this one? If some of the articles about decorated cospans were found to contain errors, is this the only construction of 
decorated cospans that solves these errors?

 \item Lemma 2.4: when talking about the functors $F : (A,\otimes)\to(Cat,\times)$, I think sometimes it is specified that they are lax and sometimes 
it is 
not. Perhaps it would be clearer to always use “lax monoidal” or to just use “monoidal” to mean lax monoidal.

{\bf We have added ``lax'' at Lemma 2.4 in two occurences, and have checked the rest of the document to verify that this does not happen 
elsewhere.}

\item Theorem 4.1: perhaps it would be clearer to say it “factors through R : Rex $\to$ SymMonCat”.

{\bf Done, also in Corollary 4.3.}

\item Proof of Lemma 4.4: there is a typo where “L then becomes left adjoint to R” should probably be “L then becomes left adjoint to U”.

{\bf Done.}

\item After Theorem A.2: In order to be consistent, it would be practical to state explicitly what is a bifibration.

{\bf Done.}

\item Beck-Chevalley condition, after Theorem A.2: Using $X_a$ instead of $C_a$ for the fibers on the diagram would be more consistent with previous 
notation.

{\bf Done.}

\item COVID paragraph: I appreciate the effort into explaining the applied aspect. However, claims about the software (it is “easy”, we “easily” 
modify, we 
“better understand”) sound too vague (is it computationally better? do we assume that the final user understands structured cospans? are we comparing 
it to other Petri net editors?). Perhaps this paragraph could be fused as a comment to the previous one, which is already a very good justification 
for the use of Petri nets with rates.

\item Section 6: as a reader, I really appreciate the examples: are there more in the literature? I do not think it is necessary to go in detail over 
each 
one of them, but perhaps it would be useful to point to other places where they are described.

{\chris Second review starts here}
\item Given the title and abstract of this paper “Structured vs Decorated Cospans” I would expect this paper to provide a judgment that structured 
cospans 
are strictly better than decorated cospans in all circumstances and a justification for that claim, or a detailed analysis of the pros and cons of 
structured vs decorated cospans, some case studies in applications that show how the difference between these double categories is felt in the 
application domain, and a some description of the tradeoffs. This paper should come with some practical advice for the applied category theorist on 
how to choose whether to model their application with structured vs decorated cospans (...) [Conclusions section] this is a place where I would expect 
some guidance on when an ACT practitioner should choose either structured or 
decorated cospans depending on what mathematical structures they are formalizing within ACT.

{\john That is their guess about what the paper should be about. I guess they were hoping for something like this:
Theorem. Structured cospans are strictly better than decorated cospans in all circumstances.
Does our abstract actually suggest that we're talking about which formalism is "better", or the "pros and cons"?
I just checked: no, it does not.}

\item Given that Petri nets with structured cospans works, and Reaction Networks work, but open dynamical systems do not, I would like some more 
explanation of the differences here. The examples of graph, labeled 
graph, Petri net, reaction net are all examples of attributed C-Sets [1], the structured cospan construction in provided for arbitrary attributed 
C-Sets in Catlab.CategoricalAlgebra [2]. The open dynamical systems are not attributed C-Sets.(...)The examples in section 6 would be much stronger 
if the others included a theorem that all (or finitary) categories of presheaves satisfy the 
conditions of theorem 4.1 or a counter example (...) Hopefully it is easy to check and provide some 
commentary 
on it in this paper.

{\john I agree that a general result like this would be nice, but I don't have the energy to dream one up and prove it. The most we can do here is say 
that this is worth studying.... maybe after we give the example of a decorated cospan category that's not a structured cospan category.}

\item If we take the categorical systems theory perspective advanced by section 6, and the body of the authors’ recent work on functorial semantics 
for complex 
systems, then we should ask why the decorated and structured cospan double categories are isomorphic for reaction networks, but their semantics as 
dynamical systems are a decorated cospan category that is not equivalent to any structured cospan category. Is there a different construction of 
semantics for reaction networks that uses structured cospans?

\item The development of the AlgebraicPetri software system is credited to Halter and Patterson based on a blog post. The correct citation is to the 
software itself and should be credited to the entire AlgebraicPetri team based on a citation to the repository [3]. The additional authors would be 
Andrew Baas and James Fairbanks.

{\john I wasn't trying to talk about AlgebraicPetri, just the application to epidemiology, and unfortunately that was (and still is???) only explained 
in a blog article. But it'd be good to add a reference to the repository.}

[Several points in the text are written in a way style that can be interpreted as the dismissive style common to many discussions of Category 
Theory. I 
would recommend rewriting at least the following points that should be made in a more collegial tone.]

\item pg. 2: “But this hope is doomed for reasons painstakingly explained in [1, Section 5] … despite all the tricks one might try.” This paragraph 
provides 
no insight into the reasons that those tricks fail. Referring them to tricks is dismissive. The citation is to a paper in TAC by the current authors.

{\chris Perhaps merge this with point 16 and possibly others, and after we do something, write a common response!}

\item pg. 3: “Many of the flawed applications of decorated cospans have been fixed using structured cospans.” This dismisses the work of other 
scholars 
using the decorated cospan approach as flawed. The alleged flaw should be made more precise. What about the decorated cospan approach is inferior to 
the structured cospan approach? The goal of this paper should be to illustrate to an applied category theory audience the differences between 
decorated and structured cospans. Referring back to previous work of the authors to make that point does not provide value.

{\john The "other scholars" are mainly me and Brendan, and the flaws are that some claims are false! I think we describe the flaws in Section 6; if so 
we should point to that.}
{\john I don't like referees who say what the goal of my paper "should be". The goal of this paper is clearly to set up a clear relation between 
structured and decorated cospans, and I hope we make that clear already in the abstract!}

\item pg. 27: To show that U does not have a left adjoint, we use the following well-known result. For readers of the TAC paper previously cited, 
that might 
be a well known result that does not require a citation, but my understanding of the Compositionality audience assumed that the ACT work was targeted 
readers that are more familiar with the applications and thus cannot be assumed to know such results.

{\bf We have added a citation to Emily Riehl's book ``Category Theory in Context'' which includes said result, above Lemma 6.1.}

{\chris Third review starts here}
\item p. 2. “Unfortunately, many applications of decorated cospans were flawed… for reasons painstaking explained in [1]”

Since fixing these problems is a major motivation for the whole paper, it would be good to explain what they are in slightly more detail, rather than 
just giving a reference. This could be done in a one paragraph or less. The gist of it, as I recall, is that in the original approach to decorated 
cospans, the isomorphism classes of open graphs depend on the identities of the edge elements, which is clearly wrong.

{\chris Merge with at least 13.}

\item p. 4, following statement of Thm 2.1.

The diagram depicting the decoration morphism $\tau$ as a natural transformation, first shown explictly on p. 10, should be moved here since it is 
implicitly used at the beginning of p. 5 to define the decoration of horizontal composites.

{\bf We have moved the natural isomorphism triangle as equation (1) in the statement of Theorem 2.1, and we have referred to it at page 10.}

\item p. 9, ingredients for constructing a double functor of decorated cospan double categories.

Does the natural transformation $\theta$: EF $\Rightarrow$ F’ H have to be a natural isomorphism? Quite possibly I missed something, but I don’t see 
where in 
the following proofs that assumption is used.

{\color{blue}{Kenny: I think they are correct in that we don't use this extra assumption.}}

\item p. 21, definition of F: Set $\to$ Cat

The use f, g for vertex/edge maps is opposite of the convention used on the previous page. I suggest that the same convention be used for both the 
decorated and structured approach to constructing open graphs.

{\chris I don't think I understand this, anyone else?}

{\color{blue}{Kenny: I \emph{think} they are complaining that we use $f$ as a map of vertices and $g$ as a map of edges when we're talking about the functor $F \colon \mathsf{Set} \to \mathsf{Cat}$ on page 21, but we use $g$ as a map of vertices and $f$ as a map of edges in the two squares at the start of Section 6.1 on the previous page?}}

\item p. 27, end of Section 6.

It seems that the left adjoint needed for structured cospans fails to exist because the chosen category of dynamical systems does not have enough 
morphisms. Relatedly, this is the only example where the pseudofunctor D: (FinSet,+) $\to$ (Cat, $\times$) comes from a simple functor D: (FinSet,+) 
$\to$ (Set, $\times$) by viewing sets as discrete categories. In the other examples, this would count as using decorated cospans in the old, flawed 
way. 
All of this suggests the question: is there a way of constructing a category of dynamical systems that has more morphisms, so that a connection with 
structured cospans can be established?

\item p. 30, following Thm A.2

If I am not mistaken, “fibration”, “opfibration,” “monoidal (op/bi)fibration” are all defined, but the definition of “bifibration” is omitted. For 
completeness, it should be included.

{\bf Done.}

\item A last general comment is that the paper unfortunately adopts the strategy, all too common in math papers, of putting all the examples in a 
section at 
the end of the paper. I think the readability of the paper would be improved, without much change to the organization, if the motivating example of 
open graphs were added to the relevant places in Sections 2 and 3. The remaining examples could be left in Section 6.

{\john
I don't want to bust up the examples and put some in sections 2 and 3. Anyone who wants examples can easily find them all in Section 6, we mention 
this section in the introduction (and perhaps should advertise it more there), and it's hard to analyze the examples without having all the theory in 
hand.}


{\chris Fourth review starts here}
\item Composition of morphisms f, g in this paper is usually denoted gf, and occasionally gof. I’d suggest mentioning this in the “conventions” 
section.

{\chris I would be happy to add this to conventions, but I cannot seem to find gf anywhere in the paper. Anyone else?}

{\color{blue}{Kenny: They might not mean $gf$ specifically, but there are many instances where we don't use a $\circ$ for functional composition, e.g. when we're pushing forward a graph structure over a map $f$ of vertices towards the top of page 21. Do we want to use a $\circ$ or not?}}

\item p3: “Using the Grothendieck construction” - maybe worth mentioning this is the “covariant” version which is different to the most common 
convention 
(at least I think so, but the literature seems to be very split over this)

{\chris Not sure that `covariant' would make this clearer...both are very well the Grothendieck construction I would think, but I don't mind that 
much.}

\item Theorem 2.1: Maybe worth saying explicitly that s is an object of the category F(m) rather than using $\in$, since this is exactly the 
distinction to 
Fong’s original definition

{\chris Maybe just specify in the statement of 2.1, definitely keep $\in$ sign in all diagrams below!}

\item Proof of theorem 2.1: You should probably at least handwave that the cartesian liftings really are cartesian liftings. (No doubt it’s easy, but 
this 
is a classic sort of spot where errors can hide)

\item On page 9 during the construction of H, I would appreciate a sentence giving some intuition for the functor E, at this point I’m confused about 
its 
role.

{\color{blue}{Kenny: We don't have any examples where $E$ is not an identity, but in my mind it's a functor that possibly modifies decorations, e.g. $F(a)$ is a category of graph structures on $a$, and $E(F(a))$ could be something like a category of possibly labeled graph structures on $a$, if that makes sense. Something potentially different than \emph{just} a graph structure on $a$.}}

\item Theorem 2.5: F should be defined on A, not A’

{\bf Done.}

\item I think it would be worth saying in words a bit more prominently what the hypothesis of theorem 4.1 roughly means, namely that the fibers of 
the total 
category have finite colimits, reindexing preserves them, and the monoidal product on the fibers is cocartesian.

{\bf We have added a paragraph right below the statement of Theorem 4.1, clarifying its assumptions as suggested, which also incorporates an 
explanatory paragraph previously located under Corollary 4.3.}

\item I wonder if the “problem” where DCsp is not a structured cospan category could be “fixed” by upgrading the functor from Set to Cat in a 
non-discrete 
way, for example with an ordering with the zero vector field as the bottom element. I suppose this would give you different 2-cells in the resulting 
double category or bicategory, but perhaps it would yield the same 1-category
\end{enumerate}


\end{document}
