\documentclass[reqno]{amsart}

\usepackage{amssymb,amsmath,stmaryrd,txfonts,mathrsfs,amsthm}%,undertilde}
\usepackage{comment}
\usepackage[neveradjust]{paralist}
\usepackage{mathtools}
\usepackage{multirow}
\usepackage[outline]{contour}
\contourlength{1.2pt}
\usepackage{tikz}
\usetikzlibrary{intersections,arrows.meta,calc,quotes,math,decorations.pathreplacing,decorations.markings,cd,arrows,positioning,fit,matrix,
shapes.geometric,external,decorations.pathmorphing,backgrounds,circuits,circuits.ee.IEC,shapes}
\usepackage[a4paper,top=2cm,bottom=2cm,inner=3cm,outer=3cm]{geometry}
\usepackage[foot]{amsaddr}
\usepackage{graphicx}
\usepackage{stackrel}
\usepackage{soul}

% math symbols

\newcommand*{\graysquare}{\textcolor{lightgray}{\blacksquare}}

\newcommand{\double}[1]{\mathbf{\mathbb #1}}
\newcommand{\lCsp}{\double{Csp}}
\newcommand{\lCospan}{\double{Cospan}}
\newcommand{\lOpen}{\double{Open}}
\newcommand{\Fr}{\double{Fr}}
\newcommand{\lA}{\double{A}}
\newcommand{\lB}{\double{B}}
\newcommand{\lC}{\double{C}}
\newcommand{\lD}{\double{D}}
\newcommand{\lE}{\double{E}}
\newcommand{\lF}{\double{F}}
\newcommand{\lG}{\double{G}}
\newcommand{\lH}{\double{H}}
\newcommand{\lR}{\double{R}}
\newcommand{\lX}{\double{X}}
\newcommand{\lY}{\double{Y}}

% colors
\usepackage{xcolor}
\usepackage{framed,color}

\def\chris{\color{purple} Christina: }
\def\john{\color{red} John: }
\def\kenny{\color{blue} Kenny: }

\definecolor{shadecolor}{rgb}{1,0.8,0.3}
\definecolor{myurlcolor}{rgb}{0.5,0,0}
\definecolor{mycitecolor}{rgb}{0,0,0.7}
\definecolor{myrefcolor}{rgb}{0,0,0.7}
\definecolor{hyperrefcolor}{rgb}{0.5,0,0}

\usepackage{hyperref}
\hypersetup{
	colorlinks,
	linkcolor={hyperrefcolor},
	citecolor={hyperrefcolor},
	urlcolor={hyperrefcolor}
}

\newcommand{\backref}[1]{(Referred to on page #1.)}

\usepackage[draft]{fixme}
\usepackage[capitalize]{cleveref}

\title{Response to reviews for ``Structured versus decorated cospans''}

\author{John\ C.\ Baez, Kenny Courser, and Christina Vasilakopoulou}

\begin{document}

\maketitle

\setcounter{tocdepth}{2}
\tableofcontents

\section{Reviews - to be removed}

\subsection{First Review}

{\footnotesize

Decorated and structured cospans have been used to model open systems in the work of Fong, Baez, Courser, Master and others.

Previous work by Shulman constructed symmetric monoidal double categories functorially from bifibrations. The present manuscript recovers structured 
and decorated cospan categories as instances of this construction and then compares both using the functoriality of that same construction. The crux 
of many of the proofs is in showing the Beck-Chevalley condition required to apply Shulman’s result. All of this could feel technical but the results 
are detailed and clear. Some of the technology used here is very interesting by itself: monoidal fibrations or the construction of (symmetric 
monoidal) double categories, bicategories and categories.

From the applied point of view, they summarize many other papers in structured and decorated cospans and discuss in which cases both approaches 
coincide. It feels that this should not be this difficult to study for the 1-categorical structure (and indeed it seems to have appeared previously on 
the literature), but here it is done for the symmetric monoidal double category structure.

As the only point against, I feel there is a growing number of “cospan” articles. The present work seems like it could serve as a comparison of all of 
these, but perhaps that comparison could be more explicit and detailed: what is it done in each one of these “cospan” articles? is there anything that 
is not already explained in this one? If some of the articles about decorated cospans were found to contain errors, is this the only construction of 
decorated cospans that solves these errors?

Section 1 motivates decorated cospans in terms of systems and interfaces.

Section 2 recovers a generalization of decorated cospans from Shulman’s construction. The most important part is to show the Beck-Chevalley property 
of the fibration at hand. In Section 3, the construction of structured cospans is recovered from Shulman’s double category using a fibration from the 
comma category of a finite colimit preserving functor L : A -> X. Again, the most important part is to show that the fibration is Beck-Chevalley. Both 
constructions (decorated and structured) are treated in a uniform way, which I see as a very positive contribution. This is used in Section 4 to give 
the comparison.

Section 5 maps double categories to bicategories and categories. For this, they use a result by Hansen and Shulman that constructs them functorially 
from fibrant double categories.

Section 6 explains the problems with Set-decoration for open graphs and justifies the need for Cat-decoration. The best argument for switching Set to 
Cat in the definition of decorated cospan is actually detailed in “Structured Cospans” by Baez and Courser (§5), where they show that the notion of 
isomorphism in graph decorated cospans is too restrictive. Section 6.1 explains how, with theorem contibuted by the paper, decorated and structured 
cospans coincide on the right notion of graph isomorphism.

Section 6.2 seems to simply repeat the same procedure after labelling edges, and Section 6.3 repeats again with Petri nets. Finally, Section 6.4 shows 
something different: a category of open dynamical systems that can be described as a decorated cospan category but that cannot be equivalently 
described as a structured cospan category.

The Conclusions section explains how, if we restrict to a suitable class of opfibrations or one-side inverse adjunctions, we would get the same 
construction from decorated and structured cospans.

In my understanding, the value of this work resides in how it systematically compares the approach of decorated and structured cospans. I would 
highlight the care put in each construction, and how it builds upon previous work for most of the heavy lifting.

Suggested changes, corrections, and general comments.

Lemma 2.4: when talking about the functors $F : (A,\otimes)\to(Cat,\times)$, I think sometimes it is specified that they are lax and sometimes it is 
not. Perhaps it would be clearer to always use “lax monoidal” or to just use “monoidal” to mean lax monoidal.

Theorem 4.1: perhaps it would be clearer to say it “factors through R : Rex -> SymMonCat”.

Proof of Lemma 4.4: there is a typo where “L then becomes left adjoint to R” should probably be “L then becomes left adjoint to U”.

After Theorem A.2: In order to be consistent, it would be practical to state explicitly what is a bifibration.

Beck-Chevalley condition, after Theorem A.2: Using $X_a$ instead of $C_a$ for the fibers on the diagram would be more consistent with previous 
notation.

COVID paragraph: I appreciate the effort into explaining the applied aspect. However, claims about the software (it is “easy”, we “easily” modify, we 
“better understand”) sound too vague (is it computationally better? do we assume that the final user understands structured cospans? are we comparing 
it to other Petri net editors?). Perhaps this paragraph could be fused as a comment to the previous one, which is already a very good justification 
for the use of Petri nets with rates.

Section 6: as a reader, I really appreciate the examples: are there more in the literature? I do not think it is necessary to go in detail over each 
one of them, but perhaps it would be useful to point to other places where they are described.}

\subsection{Second review}

{\footnotesize

Overview

This is an important topic and brings clarity to an open question in the ACT field. The fact that structured and decorated cospans are equivalent in 
many cases is a good thing, and the fact that the authors develop a sufficient condition for proving this equivalence is highly relevant to this 
journal. Having the description of decorated and structured cospans in the same place with a consistent presentation is very useful for the community. 
My review focuses on how I think this work fits into the broader context of ACT research in this area. I recommend publication after addressing the 
following concerns.

Suggested changes, corrections, and general comments.

General Thoughts

Given the title and abstract of this paper “Structured vs Decorated Cospans” I would expect this paper to provide a judgment that structured cospans 
are strictly better than decorated cospans in all circumstances and a justification for that claim, or a detailed analysis of the pros and cons of 
structured vs decorated cospans, some case studies in applications that show how the difference between these double categories is felt in the 
application domain, and a some description of the tradeoffs. This paper should come with some practical advice for the applied category theorist on 
how to choose whether to model their application with structured vs decorated cospans. Section 6 does a good job of detailing the various 
constructions of combinatorial structures that get double categories of structured cospans. The graph and Petri net examples are implemented in 
AlgebraicJulia, specifically the Catlab.Graphs module and the AlgebraicPetri.jl package. Given that Petri nets with structured cospans works, and 
Reaction Networks work, but open dynamical systems do not, I would like some more explanation of the differences here. The examples of graph, labeled 
graph, Petri net, reaction net are all examples of attributed C-Sets [1], the structured cospan construction in provided for arbitrary attributed 
C-Sets in Catlab.CategoricalAlgebra [2]. The open dynamical systems are not attributed C-Sets. All the examples where decorated and structured cospans 
agree are combinatorial, and the only example where they are different is not combinatorial. This gives me the impression that for categories of 
presheaves over a finitely presented category, structured cospans and decorated cospans are always equivalent. The introduction of numerical 
attributes (as in the case of reaction networks) does not appear to interfere with that pattern.

The examples in section 6 would be much stronger if the others included a theorem that all (or finitary) categories of presheaves satisfy the 
conditions of theorem 4.1 or a counter example illustrating that the situation is more complex than algebraic vs analytic distinction between open 
presheaves and the category of open dynamical systems. Without this analysis, I am left with the intuition that for discrete structures like finitary 
presheaves over a finitely presented schema, the decorated and structured cospans are going to be equivalent and for continuous mathematical 
structures like dynamical systems they will be different. If that intuition is correct, the paper should say something about it. If the intuition is 
incorrect, then additional explanation should be provided. I think that the implementation in [2] uses the left and right adjoints L and U in Theorem 
4.1, which is why I think that every category of presheaves should satisfy the main theorem. Hopefully it is easy to check and provide some commentary 
on it in this paper.

The interaction of compositional system behavior and the main theorem is also interesting and worth exploring or commenting on in this paper. If we 
take the categorical systems theory perspective advanced by section 6, and the body of the authors’ recent work on functorial semantics for complex 
systems, then we should ask why the decorated and structured cospan double categories are isomorphic for reaction networks, but their semantics as 
dynamical systems are a decorated cospan category that is not equivalent to any structured cospan category. Is there a different construction of 
semantics for reaction networks that uses structured cospans? The current work does not address the possibility of alternative semantics for 
structured cospans of systems. We can only say that DCsp is not equivalent to any structured cospan category, but is there a different decorated 
cospan category that would provide meaningful functorial semantics that is equivalent to some category of structured cospans. The structured cospan 
view is better for implementing the composition of the system specifications (reaction networks) but the current literature only uses decorated 
cospans for providing the systems with behaviors. If the authors can discuss this issue it would greatly enhance the utility of this paper for shaping 
the research discussion around compositional complex systems.

The conclusions refer to a forthcoming paper. The only conclusion that refers to the work presented in this paper is that theorem 4.1 gives conditions 
under which a category of decorated cospans is isomorphic to a category of structured cospans. Given that the audience for this paper is intended to 
be applied category theorists, this is a place where I would expect some guidance on when an ACT practitioner should choose either structured or 
decorated cospans depending on what mathematical structures they are formalizing within ACT.

Specific Concerns

The development of the AlgebraicPetri software system is credited to Halter and Patterson based on a blog post. The correct citation is to the 
software itself and should be credited to the entire AlgebraicPetri team based on a citation to the repository [3]. The additional authors would be 
Andrew Baas and James Fairbanks.

Several points in the text are written in a way style that can be interpreted as the dismissive style common to many discussions of Category Theory. I 
would recommend rewriting at least the following points that should be made in a more collegial tone.

pg. 2: “But this hope is doomed for reasons painstakingly explained in [1, Section 5] … despite all the tricks one might try.” This paragraph provides 
no insight into the reasons that those tricks fail. Referring them to tricks is dismissive. The citation is to a paper in TAC by the current authors.

pg. 3: “Many of the flawed applications of decorated cospans have been fixed using structured cospans.” This dismisses the work of other scholars 
using the decorated cospan approach as flawed. The alleged flaw should be made more precise. What about the decorated cospan approach is inferior to 
the structured cospan approach? The goal of this paper should be to illustrate to an applied category theory audience the differences between 
decorated and structured cospans. Referring back to previous work of the authors to make that point does not provide value.

pg. 27: To show that U does not have a left adjoint, we use the following well-known result. For readers of the TAC paper previously cited, that might 
be a well known result that does not require a citation, but my understanding of the Compositionality audience assumed that the ACT work was targeted 
readers that are more familiar with the applications and thus cannot be assumed to know such results.}

\begin{verbatim}
[1] https://arxiv.org/abs/2106.04703
[2] https://github.com/AlgebraicJulia/Catlab.jl/blob/f06417de29421441c4133c1cfb75b39ae8aa7142/src/categorical_algebra/StructuredCospans.jl
[3] https://github.com/AlgebraicJulia/AlgebraicPetri.jl
\end{verbatim}

\subsection{Third review}

{\footnotesize
Decorated and structured cospans have emerged as a mainstay of the current wave of ACT, the latter correcting certain technical problems with the 
former. This paper poses two questions:

    How can the technical problems with decorated cospans be fixed?

    How does the corrected version of decorated cospans compare with structured cospans?

The paper offers the following answers:

The symmetric lax monoidal functor (A,+) -> (Set, $\times$) defining a category of isomorphism classes of decorated cospans is replaced by a 
symmetric lax monoidal pseudofunctor (A,+) -> (Cat, $\times$) defining a symmetric monoidal double category of decorated cospans.

A sufficient condition, the main Theorem 4.1, is provided for a double category of decorated cospans to be equivalent, indeed isomorphic, to a double 
category of structured cospans. Several examples, as well as a nonexample, of applying the theorem are given.

The paper is a technical one, giving slick but abstract constructions and proofs using machinery developed by Shulman for turning monoidal 
bifibrations into double categories, and it serves more to clarify and compare existing ACT methods than to introduce new ones. The mathematical 
exposition is, however, very strong, and a fairly wide range of readers should be able to take something away from the paper. Given the central status 
of decorated/structured cospans within ACT, it was important that the above questions be addressed, and the authors have done so admirably. Also, the 
constructions and results are “black-boxed” in such a manner that practioners will be able to use them without understanding in detail the proofs that 
support them. I therefore consider this paper to be a strong contribution to the scholarship of ACT and recommend that it be published.

Suggested changes, corrections, and general comments.

Following are some questions and minor suggested changes.

p. 2. “Unfortunately, many applications of decorated cospans were flawed… for reasons painstaking explained in [1]”

Since fixing these problems is a major motivation for the whole paper, it would be good to explain what they are in slightly more detail, rather than 
just giving a reference. This could be done in a one paragraph or less. The gist of it, as I recall, is that in the original approach to decorated 
cospans, the isomorphism classes of open graphs depend on the identities of the edge elements, which is clearly wrong.

p. 4, following statement of Thm 2.1.

The diagram depicting the decoration morphism $\tau$ as a natural transformation, first shown explictly on p. 10, should be moved here since it is 
implicitly used at the beginning of p. 5 to define the decoration of horizontal composites.

p. 9, ingredients for constructing a double functor of decorated cospan double categories.

Does the natural transformation $\theta$: EF => F’ H have to be a natural isomorphism? Quite possibly I missed something, but I don’t see where in 
the following proofs that assumption is used.

p. 21, definition of F: Set -> Cat

The use f, g for vertex/edge maps is opposite of the convention used on the previous page. I suggest that the same convention be used for both the 
decorated and structured approach to constructing open graphs.

p. 27, end of Section 6.

It seems that the left adjoint needed for structured cospans fails to exist because the chosen category of dynamical systems does not have enough 
morphisms. Relatedly, this is the only example where the pseudofunctor D: (FinSet,+) -> (Cat, $\times$) comes from a simple functor D: (FinSet,+) -> 
(Set, $\times$) by viewing sets as discrete categories. In the other examples, this would count as using decorated cospans in the old, flawed way. 
All of this suggests the question: is there a way of constructing a category of dynamical systems that has more morphisms, so that a connection with 
structured cospans can be established?

p. 30, following Thm A.2

If I am not mistaken, “fibration”, “opfibration,” “monoidal (op/bi)fibration” are all defined, but the definition of “bifibration” is omitted. For 
completeness, it should be included.

A last general comment is that the paper unfortunately adopts the strategy, all too common in math papers, of putting all the examples in a section at 
the end of the paper. I think the readability of the paper would be improved, without much change to the organization, if the motivating example of 
open graphs were added to the relevant places in Sections 2 and 3. The remaining examples could be left in Section 6.
}

\subsection{Fourth review}

{\footnotesize
What are the main questions posed by the manuscript, and how does it answer them? Why or why not is this work a significant technical or conceptual 
contribution to scholarship?

Decorated cospans and structured cospans are two different tools for building categories of open systems, both of which have been applied in several 
existing papers (and this reviewer believes they will continue to grow in importance over time). This paper explains the relationship between them, in 
a very elegant way using the equivalence between fibered and indexed categories, to the extent that the two constructions now appear as different 
perspectives on the same fundamental idea. To me this relationship appears obvious in retrospect, and I mean this as a compliment to the paper. I 
think this paper is an important contribution to categorical systems theory, and I recommend acceptance.

Suggested changes, corrections, and general comments.

Composition of morphisms f, g in this paper is usually denoted gf, and occasionally gof. I’d suggest mentioning this in the “conventions” section.

p3: “Using the Grothendieck construction” - maybe worth mentioning this is the “covariant” version which is different to the most common convention 
(at least I think so, but the literature seems to be very split over this)

Theorem 2.1: Maybe worth saying explicitly that s is an object of the category F(m) rather than using $\in$, since this is exactly the distinction to 
Fong’s original definition

Proof of theorem 2.1: You should probably at least handwave that the cartesian liftings really are cartesian liftings. (No doubt it’s easy, but this 
is a classic sort of spot where errors can hide)

On page 9 during the construction of H, I would appreciate a sentence giving some intuition for the functor E, at this point I’m confused about its 
role.

Theorem 2.5: F should be defined on A, not A’

I think it would be worth saying in words a bit more prominently what the hypothesis of theorem 4.1 roughly means, namely that the fibers of the total 
category have finite colimits, reindexing preserves them, and the monoidal product on the fibers is cocartesian.

I wonder if the “problem” where DCsp is not a structured cospan category could be “fixed” by upgrading the functor from Set to Cat in a non-discrete 
way, for example with an ordering with the zero vector field as the bottom element. I suppose this would give you different 2-cells in the resulting 
double category or bicategory, but perhaps it would yield the same 1-category}

\section{Our Answer}


\begin{enumerate}

\item As the only point against, I feel there is a growing number of “cospan” articles. The present work seems like it could serve as a 
comparison of 
all of 
these, but perhaps that comparison could be more explicit and detailed: what is it done in each one of these “cospan” articles? is there anything 
that 
is not already explained in this one? If some of the articles about decorated cospans were found to contain errors, is this the only construction of 
decorated cospans that solves these errors?

{\bf The present work reviews all known previous work on decorated and structured cospans in Section 6, and we now point this out in the end of 
the Introduction.   As we mention, our previous paper ``Structured 
cospans'' fixed the errors in previous work on decorated cospans by using structured cospans instead.   Here we describe how the new improved 
decorated cospans can also fix those errors.     There may be even more ways to correct these errors, but we do not know them.}  

{\kenny $\checkmark$}  {\john $\checkmark$} {\chris $\checkmark$}

\iffalse
{\chris As was suggested by John at some point, would we perhaps want to add a sentence or two in the introduction, saying that indeed there are a 
number of cospan articles etc. and that in the beginning of section 6 we elaborate on those?}

{\kenny I'm fine either way on this.}

{\john I have added a remark in the Introduction, ``This result [Theorem 4.1] shows that under certain conditions, structured and decorated cospans 
provide equivalent ways of describing open systems.  We illustrate this in Section 6 with applications to graphs, electrical circuits, Markov 
processes, Petri nets, Petri nets with rates, and dynamical systems.   This is meant to be a fairly thorough review of the existing literature.'' 

If you're happy with my response above give me a check mark, Christina!}

{\chris Looks good, I only added ``in the end of'' the introduction, if it is OK, John, erase current comment.}
\fi

 \item Lemma 2.4: when talking about the functors $F : (\mathsf{A},\otimes)\to(\mathsf{Cat},\times)$, I think sometimes it is specified that they are lax and sometimes 
it is not. Perhaps it would be clearer to always use “lax monoidal” or to just use “monoidal” to mean lax monoidal.

{\bf We have added ``lax'' at Lemma 2.4 in two occurences, and have checked the rest of the document to verify that this does not happen 
elsewhere.}  

{\john $\checkmark$} {\chris $\checkmark$} {\kenny $\checkmark$}

\item Theorem 4.1: perhaps it would be clearer to say it “factors through R : Rex $\to$ SymMonCat”.

{\bf Done, also in Corollary 4.3.} 

{\chris $\checkmark$} {\kenny $\checkmark$} {\john $\checkmark$}

\iffalse
{\john  I find that this added text confuses me more than it helps me:
\vskip 1em
\begin{quote}
Essentially, the above assumptions state that the fibers of the induced opfibration $U$ have finite colimits which are preserved by the reindexing functors, and the induced fiberwise monoidal structure is cocartesian. Explicitly, regarding the last point: the lax monoidal pseudofunctor structure $(F,\phi,\phi_0)$ gives rise to a specific symmetric monoidal structure on the fibers $F(a)$ given by (3), since $\mathsf{A}$ is cocartesian monoidal, and we ask that the resulting pseudofunctor $F\colon \mathsf{A} \to \mathsf{SMC}$ factors through $\mathsf{Rex}$, so that (3) gives coproducts and an initial object in each $F(a)$.
\end{quote}
\vskip 1em
I think the referee just wanted us to change, in the statement of Theorem 4.1, the phrase
``factors through $\mathsf{Rex}$'' to ``“factors through $R \colon \mathsf{Rex} \to \mathsf{SMC}$, where $R$ was his name for this 2-functor, not ours!   It's not a good name for it since we have another $R$ in our
paper.   So, I've commented out the added text and said ``factors through the above 2-functor $\mathsf{SMC} \to \mathsf{Rex}$'' --- we defined right above the theorem.}


{\chris Indeed this comment refers to the functor $\mathsf{Rex} \to \mathsf{SMC}$, I also chose to disregard R for the same reasons. The paragraph 
below existed in part later in text, and was in part expanded according to comment 29, not this one! I think I agree with that comment in that the 
assumptions of the theorem deserve a bit of explanation, and tried my best there but possibly did not succeed if you say it is confusing. Could we 
revive the paragraph and explain better? I can give it another try!}

{\john  Okay, that makes sense.   I put the paragraph back in and changed it a bunch: basically, I added some more steps so people who don't know what 
we're doing will have an easier time figuring it out.   Does it look okay to you?}

{\chris Yes it does, thanks! Although in my mind, it kinda goes the opposite way: e.g. the already induced monoidal product on the fibers ends up 
having the universal property of a coproduct for example. Possibly just overthinking and getting mixed up with words.}

{\john I changed the paragraph, trying to solve the problem Christina just mentioned.}

{\chris Looks good!}
\fi

\item Proof of Lemma 4.4: there is a typo where “L then becomes left adjoint to R” should probably be “L then becomes left adjoint to U”.

{\bf Done.}  

{\john $\checkmark$} {\chris $\checkmark$} {\kenny $\checkmark$}

\item After Theorem A.2: In order to be consistent, it would be practical to state explicitly what is a bifibration.

{\bf Done.} 

{\john $\checkmark$} {\chris $\checkmark$} {\kenny $\checkmark$}

\item Beck-Chevalley condition, after Theorem A.2: Using $X_a$ instead of $C_a$ for the fibers on the diagram would be more consistent with previous 
notation.

{\bf Done.} 

{\john $\checkmark$} {\chris $\checkmark$} {\kenny $\checkmark$}

\item COVID paragraph: I appreciate the effort into explaining the applied aspect. However, claims about the software (it is “easy”, we “easily”  modify, we “better understand”) sound too vague (is it computationally better? do we assume that the final user understands structured cospans? are we comparing  it to other Petri net editors?). Perhaps this paragraph could be fused as a comment to the previous one, which is already a very good justification  for the use of Petri nets with rates.

{\bf There is now a paper detailing how this software works, so according to this suggestion we have changed our paper so it simply refers to that other paper.}   {\john But we still need to add an arXiv link in reference 1!  It should be on the arXiv soon.} 

{\kenny Can we put ``In preparation." if they don't have this on the arXiv in a timely manner?}

{\chris Sounds good to me...}

{\john Yes, we can do that.  Their paper is written, but they are struggling to get it onto the arXiv... and having trouble for stupid technical reasons.   I'll see if they succeed by the time we submit this.}

%{\chris Great! The 
%hyperref link ??? opens up notes by Kenny for me, I don't know how or why.}

%{\john It's weird that the line ??? opens up anything for you.  It'll go away when we get the arXiv number.} Indeed!

\item Section 6: as a reader, I really appreciate the examples: are there more in the literature? I do not think it is necessary to go in detail over each one of them, but perhaps it would be useful to point to other places where they are described.

{\bf We have added a word about Markov processes in Section 6.2, and with this, all the applications of structured or decorated cospans that we know 
of are discussed here.} 

{\john $\checkmark$} {\chris $\checkmark$} {\kenny $\checkmark$}

\vskip 1em

\item Given the title and abstract of this paper “Structured vs Decorated Cospans” I would expect this paper to provide a judgment that structured  cospans are strictly better than decorated cospans in all circumstances and a justification for that claim, or a detailed analysis of the pros and cons of 
structured vs decorated cospans, some case studies in applications that show how the difference between these double categories is felt in the 
application domain, and a some description of the tradeoffs. This paper should come with some practical advice for the applied category theorist on 
how to choose whether to model their application with structured vs decorated cospans. 

{\bf In fact our goal, as explained in the abstract and introduction, was to prove that the structured and decorated cospan approaches are equivalent under certain conditions, and to illustrate this result with a wide variety of examples.  These examples can serve as a guide for other similar applications, while new applications may require exploring which approach works best.  We do not make a stand that either is strictly better.   However, we have changed the end of the Introduction to read as follows:

\vskip 1em
\begin{quote}
This result shows that under certain conditions, structured and decorated cospans provide equivalent ways of describing open systems.  We illustrate this in Section 6 with applications to graphs, electrical circuits, Markov processes, Petri nets, Petri nets with rates, and dynamical systems.   This is meant to be a fairly thorough review of the existing literature.   It becomes clear that when either structured or decorated cospans can be used, structured cospans are simpler.  However, for open dynamical systems we need decorated cospans, for reasons we explain.   Thus, to describe a map from open Petri nets with rates to open dynamical systems, we use decorated cospans.
\end{quote}
}

{\john $\checkmark$} {\kenny $\checkmark$} {\chris $\checkmark$}
\iffalse
{\kenny I think the reader should come to their own conclusion about which framework they should use if they are trying to decide between the two, assuming both are applicable. I wouldn't say that one is better than the other. Personally, I would rather try structured cospans first, because I would rather think about a left adjoint than a symmetric lax monoidal pseudofunctor, but other people might feel differently.} 

{\john I agree with you, Kenny.  I think this reviewer saw the word ``vs'' and thought we were going to let structured and decorated cospans fight it out and announce who won, like at a boxing match.   Instead of arguing that this is not a good idea for a paper, I decided to just point out that our abstract says what our paper does, and it's something else.   (I also agree with you that structured cospans are simpler when you can use them: that's why we invented them!)}

{\john If you like the response in boldface above, please add a checkmark here.  Otherwise try to
improve it, or something.} 

{\chris I could add ``Those examples should serve as a guide for other similar applications, 
whereas new applications would require exploring which approach works best, for example if a left adjoint (to start off with structured cospans) or a 
lax monoidal pseudofunctor (to start off with decorated cospans) is available. We do not make a stand that either is strictly better.'' In fact, we 
could add something soft along those lines somewhere in the introduction to satisfy the reviewer, if we like.}

{\chris By the way, there is this sentence ''However, it should be clear by now that so
far, in cases where either structured or decorated cospans can be used, structured cospans are simpler. '' in section 6. Perhaps we could even imply 
that somewhere in the intro as well?}

{\kenny The boldface looks good to me, but I also like Christina's suggestions above if we really want to make an effort to appease this referee. I'm fine either way on this.}


{\john See if you like my new remarks in boldface, and give me a check mark if you do!  Kenny had checked this one off already, but I've changed it a 
lot.}

{\chris Looks great to me. Final two minor points, feel free to ignore or not and delete this comment. (1) Is ``apparently'' supposed to sound a bit 
sceptical? If not, would ``actually'' be better or not?}

{\john I said ``apparently'' because two referees suggested that we might be able to create a structured cospan category of open dynamical systems if we could find more morphisms between dynamical systems.  But since the readers of our paper aren't being told about this hope, I guess it will confuse \emph{them} if we say ``apparently''.   So I've deleted ``apparently'' in this sentence: 

\begin{quote}
However, for open dynamical systems we apparently need decorated cospans, for reasons we explain. 
\end{quote}
\noindent
both here and in the actual paper.  The referees may complain, but the readers will be less puzzled.}

{\chris (2) If we were being really accurate, don't we 
actually ONLY use decorated cospans to describe the map from open Petri nets with rates to open dynamical systems? Namely since dynamical systems only 
work as decorated cospans, we are required to use decorated cospans also for Petri nets to then get the map between them. Unless I am mixing something 
there.}

{\john You're right.  So I've also changed this sentence above and in the paper:

\begin{quote}
Thus, to describe a map from open Petri nets with rates to open dynamical systems, we use \emph{both} structured and decorated cospans.
\end{quote}}

{\chris Cool thanks!}

\fi 

\item Given that Petri nets with structured cospans works, and Reaction Networks work, but open dynamical systems do not, I would like some more 
explanation of the differences here. The examples of graph, labeled 
graph, Petri net, reaction net are all examples of attributed C-Sets [1], the structured cospan construction in provided for arbitrary attributed 
C-Sets in Catlab.CategoricalAlgebra [2]. The open dynamical systems are not attributed C-Sets.(...)The examples in section 6 would be much stronger 
if the others included a theorem that all (or finitary) categories of presheaves satisfy the 
conditions of theorem 4.1 or a counter example (...) Without this analysis, I am left with the intuition
that for discrete structures like finitary presheaves over a finitely presented schema, the decorated and structured cospans are going
to be equivalent (...) If that intuition is correct, the paper should say something about it. If the intuition is incorrect, then additional 
explanation should be provided. Hopefully it is easy to check and provide some commentary on it in this paper.

{\bf Unfortunately not all cases of C-sets meet the conditions of Theorem 4.1.   For example, the usual forgetful functor from $\mathsf{Graph}$ to $\mathsf{Set}$ sending any graph to its set of vertices is an opfibration, but not the forgetful functor sending any graph to its set of edges.  So, this gives a structured cospan category to which Theorem 4.1 does not apply.   Finding extra conditions such that it \emph{does} apply is an interesting question, but it would take more thought.}

{\john $\checkmark$} {\kenny $\checkmark$}

{\john Actually it's really easy to prove that presheaf categories obey the conditions of Theorem 4.1:
given any functor $f : \mathsf{C} \to \mathsf{D}$ the left adjoint functor between presheaf categories $f_! : \widehat{\mathsf{C}} \to \widehat{\mathsf{D}}$ obeys the conditions of Theorems 4.1 and 4.2.   So I think we can easily add this as a corollary, maybe right after Theorem 4.2, to get a pile of examples.   Obvious to us, but not to everyone.  Should I do it?}

{\chris Sounds like a great idea to me!! I was also a bit scared to go into this when I first saw it, but I kinda thought morally we should, and now 
it looks like it won't be a disaster, and in fact a very fruitful result!! So yes please, do it and let me know what I can do to help except 
proofread it.}

{\chris Also perhaps we should add those references they provide somewhere.}

{\john Now I think this is a bit harder than I thought.  I started writing stuff:

\vskip 1em
\begin{quote}
Presheaf categories provide many examples of  Theorem 4.1.  Suppose $f \colon C \to D$ is any functor between small categories.  Then precomposition with $f$ gives a functor $f^\ast \colon \widehat{D} \to \widehat{C}$ between their presheaf categories, and this has a left adjoint $f_! \colon \widehat{C} \to \widehat{D}$.   
\end{quote}

\vskip 1em

This is a good enough situation to build a structured cospan category, but the right adjoint $f^\ast$ may not be an opfibration, in which case we can't apply Theorem 4.1 --- right?   

With a lot of help from Todd Trimble's I once figured out when $f^\ast$ \emph{is} an opfibration, but remembering that argument and writing this up would take some work.  And I don't really feel like expanding the paper in such a way --- we've got more than enough already.   So how about if just tell the referee this is a great idea but it would take more thought?   See if you like my boldface comment above.  
}

{\kenny I think I recall you trying to explain this to me once: I believe the condition for $f^*$ to be an opfibration was that each hom-set $\hom(d,d)$ was a singleton, and that it was an if and only if. I remember trying to prove it, but getting lost in some of the details. Regardless, I agree that we have enough. The bold looks good to me.}


\item If we take the categorical systems theory perspective advanced by section 6, and the body of the authors’ recent work on functorial semantics 
for complex 
systems, then we should ask why the decorated and structured cospan double categories are isomorphic for reaction networks, but their semantics as 
dynamical systems are a decorated cospan category that is not equivalent to any structured cospan category. Is there a different construction of 
semantics for reaction networks that uses structured cospans?

{\bf We do not know a way to construct a (double) category of open dynamical systems that uses structured cospans; to find one would require getting around the result at the end of Section 6.  This is an interesting challenge.}

{\john $\checkmark$} {\kenny $\checkmark$} {\chris $\checkmark$}

\iffalse
{\chris Should we refer to which result we are talking about? Also, I am a bit confused about the question and answer. I thought reaction networks 
can be described in both ways?}

{\kenny I'm kinda' confused by the question, too, due to its wording. Based on what Dr. Baez said in response, I think they might have meant dynamical systems instead of reaction networks in their last sentence. They even say earlier in their comment that the frameworks are isomorphic for reaction networks.}

{\john The referee was asking about the \emph{semantics} for reaction networks --- that is, dynamical systems!  Reaction networks are the ``syntax'' and dynamical systems are our chosen ``semantics'' for these.  But my reply in boldface was confused --- you're right.  So I've fixed it. 
The ``result at the end of Section 6'' is the one where we show the functor from dynamical systems to finite sets doesn't have a left adjoint.  For 
some reason we didn't make this into a theorem.   Give me a check mark if you're happy now.   }

{\chris I see, thanks!}
\fi

\item The development of the AlgebraicPetri software system is credited to Halter and Patterson based on a blog post. The correct citation is to the 
software itself and should be credited to the entire AlgebraicPetri team based on a citation to the repository [3]. The additional authors would be 
Andrew Baas and James Fairbanks.

{\bf We have added a reference to the new paper ``An
algebraic framework for structured epidemic modeling'' by A.\ Baas, J.\ Fairbanks, M.\ Halter, S.\ Libkind and E.\ Patterson.  We have also added a reference to the AlgebraicPetri GitHub repository.}

{\john \checkmark} {\kenny \checkmark} {\chris \checkmark}

\iffalse
{\john It'd be good to add a reference to the repository.}

{\chris I added the reference but looks a bit funny, edit at will. Did not add any text.}

{\john I've improved the reference and added a citation to it.  Is this okay now?}

{\chris Yes!}
\fi

%[Several points in the text are written in a way style that can be interpreted as the dismissive style common to many discussions of Category 
%Theory. I would recommend rewriting at least the following points that should be made in a more collegial tone.]

\item pg. 2: ``But this hope is doomed for reasons painstakingly explained in [3, Section 5] … despite all the tricks one might try.'' This paragraph 
provides 
no insight into the reasons that those tricks fail. Referring them to tricks is dismissive. The citation is to a paper in TAC by the current authors.

{\bf We removed the part of the sentence ``despite all the tricks one might try'' according to this suggestion. In [3, Section 5] we spent five pages discussing these attempts and why they fail.    The diligent reader may wish to read this discussion and see if there is some other way out.}

{\chris \checkmark} {\john \checkmark}

\iffalse
{\john He thought we were being dismissive because he didn't know that the ``tricks'' were all things I'd done myself!  If he looked at [1, Section 5] 
he'd see it was me coming up with a bunch of tricks to try to save the decorated cospan category of open graphs, and then Kenny and me explaining why 
this tricks don't work.  So maybe I should just delete ``despite all the tricks one might try''.  Sound good?}

{\chris Sounds good to me! See above, amend at will.}

{\kenny Sounds good.}
\fi

\item pg. 3: ``Many of the flawed applications of decorated cospans have been fixed using structured cospans.” This dismisses the work of other 
scholars 
using the decorated cospan approach as flawed. The alleged flaw should be made more precise. What about the decorated cospan approach is inferior to 
the structured cospan approach?''
%The goal of this paper should be to illustrate to an applied category theory audience the differences between 
%decorated and structured cospans. Referring back to previous work of the authors to make that point does not provide value.

{\bf  The flaws---mathematical errors that render a number of claimed theorems incorrect---are now briefly explained in  in the Introduction, in the paragraph beginning ``Unfortunately, many applications of decorated cospans were flawed.''.  In reference [3], Kenny Courser and John Baez analyzed these mistakes in detail and fixed them using structured cospans.  See also comment (9).}

{\john $\checkmark$} {\kenny $\checkmark$} {\chris \checkmark}

\iffalse
{\john The "other scholars" are mainly me and Brendan, and the flaws are that some claims are false! I think we describe the flaws in Section 6; if so 
we should point to that.}   

{\john I feel like just leaving out this part in our reply: ``The goal of this paper should be to illustrate to an applied category theory audience the differences between 
decorated and structured cospans. Referring back to previous work of the authors to make that point does not provide value.''}

{\chris Right, done. About our answer here, we could politely tell them that the problem is simply that they are wrong! And perhaps, in 
text amend that sentence to ``Many of the flawed applications of decorated cospans, \emph{like the open graph one described above},...'' because we 
do actually say what the flaw is there, correct? I don't see where we describe flaws in Section 6.}

{\kenny It looks like we loosely mention what the problem is on page 2, but we already explained (in detail) what the problems are in \emph{Structured cospans}. I don't see why we can't just point the reader that wants to see what those problems are to that paper. I don't think we explain any flaws in Section 6; only use both frameworks and compare them.}

{\kenny Actually, we do talk about the flaws a bit more in the middle of page 21 when we're setting up the example of open graphs for Theorem 2.2.}

{\john See if you like my boldface response.  I'm acting like we added an explanation of the flaws to our introduction, so we don't have to say ``Hey, 
buster --- we already explained this.''}

{\chris I would potentially skip the first sentence of our response above. Does it sound a bit hostile? And perhaps he also means other people that 
may have used this approach, or perhaps Brendan himself (not a coauthor at the time?) that invented them. Not too important in any case.}

{\john Okay, I deleted it.  Indeed, it \emph{was} a bit hostile.}

{\chris Thanks!}

\fi


\item pg. 27: To show that U does not have a left adjoint, we use the following well-known result. For readers of the TAC paper previously cited, 
that might 
be a well known result that does not require a citation, but my understanding of the Compositionality audience assumed that the ACT work was targeted 
readers that are more familiar with the applications and thus cannot be assumed to know such results.

{\bf We have eliminated the term ``well-known'' and added a citation to Emily Riehl's book \textsl{ Category Theory in Context} which includes said result, above Lemma 6.1.}  

%{\john I have also changed ``well-known'' to ``known'', which is more polite to people who don't know it.} {\chris Good!}

{\john $\checkmark$} {\chris $\checkmark$} {\kenny $\checkmark$}

\vskip 1em

\item p. 2. ``Unfortunately, many applications of decorated cospans were flawed… for reasons painstakingly explained in [1].''

Since fixing these problems is a major motivation for the whole paper, it would be good to explain what they are in slightly more detail, rather than 
just giving a reference. This could be done in a one paragraph or less. The gist of it, as I recall, is that in the original approach to decorated 
cospans, the isomorphism classes of open graphs depend on the identities of the edge elements, which is clearly wrong.

{\bf We have rewritten this paragraph to explain those flaws.}

{\john $\checkmark$} {\kenny $\checkmark$} {\chris $\checkmark$}

\iffalse
{\chris If I am not wrong, the paragraph cited here IS the explanation of the problem, perhaps in application terms rather than in theory terms. 
Consider perhaps adding one sentence along the lines suggested?}

{\kenny This seems similar to comment 14 above. So that's two people that want to know more about what the flaws are with the original decorated cospans but don't feel like opening our other paper...}

{\john You're right, Christina, I already explained why ``many applications of decorated cospans were flawed'' in this paragraph!   See if you like my boldface response.  I sneakily pretend that we rewrote this paragraph to explain the flaws.  We actually have rewritten this paragraph... but the explanation was already there!

And yeah, Kenny, it's annoying: people complain if you repeat already known stuff, but they also complain if you don't.}

\fi

\item p. 4, following statement of Thm 2.1.

The diagram depicting the decoration morphism $\tau$ as a natural transformation, first shown explictly on p. 10, should be moved here since it is 
implicitly used at the beginning of p. 5 to define the decoration of horizontal composites.

{\bf We have moved the natural isomorphism triangle as equation (1) in the statement of Theorem 2.1, and we have referred to it at page 10.} 

{\chris $\checkmark$} {\kenny $\checkmark$} {\john $\checkmark$} 

\item p. 9, ingredients for constructing a double functor of decorated cospan double categories.

Does the natural transformation $\theta \colon EF \Rightarrow F’ H$ have to be a natural isomorphism? Quite possibly I missed something, but I don’t see where in 
the following proofs that assumption is used.

{\bf We thank the referee for catching this important mistake!   It does not need to be an isomorphism, so we changed the construction in p. 
9 accordingly. In fact, in Section 6 we construct a double functor
\[      \graysquare \colon F \lCsp \to D \lCsp \]
sending any open Petri net with rates to its corresponding open dynamical system, and we do this
using a monoidal natural transformation $\theta$ that is \emph{not} an isomorphism.  We had mistakenly written that it was; we have fixed this now.}

{\john $\checkmark$} {\kenny $\checkmark$} {\chris $\checkmark$}

\iffalse
{\kenny I think they are correct in that we don't use this extra assumption.}

{\chris I have a vague feeling that we discussed something similar a long time ago, but I do not remember the context at all. Perhaps it was an 
%identity then? In any case, if we don't actually use it then we should make it into a transformation.}

{\kenny Yes, originally it was an identity. Structured cospans has similar squares for its maps, but those ones are required to be invertible. In my 
thesis, I was going to try to come up with a recipe for turning those squares into these ones, so I made these ones invertible, too, but never 
actually got around to this.}

{\chris I see! So for maps of structured cospans, invertibility is actually essential?}

{\kenny Yep, we needed certain maps $L' F \Rightarrow F' L$ (using structured cospan letters) to be invertible. We also built those maps by hand, 
though, and here, we're using Shulman's mystical machine to build these ones. I'm not sure what happens or is required when building maps of 
structured cospans using Shulman's machine...} {\chris Interesting!}

{\john  As Christina pointed out on Zulip, the $\theta : EF \Rightarrow F'H$ on page 26, where we are
constructing a map from open Petri nets with rates to open dynamical systems, is \emph{not} an
isomorphism.  I changed ``natural isomorphism'' to ``natural transformation'' at three points on page 26 --- 
{\bf please help make sure we catch them all!}

Also, I'd better change the remark about  $\theta: EF \Rightarrow F'H$  on page 9, and
 in the statement of Theorem 2.5 --- these are two cases where we called it a ``natural 
isomorphism''.    I'll do this now: I'll change it to a ``natural transformation''.

I've thanked the referee profusely for catching this problem --- check out my response in boldface
above, and if you're happy, we can check this one off!}

{\chris The response looks great, see my small addition above `` so we changed the construction in p. 
9 accordingly''. I seached ``isomorphism'' the whole document and I think we are fine (but perhaps 
Kenny can double check). Except this one place where I changed it to transformation above Theorem 2.5, check that it doesn't break anything? It looks 
OK to me.}

{\kenny I also checked the entire document for `isomorphism' and didn't see any left over from this situation. Christina's catch on Zulip was a great one, and the response looks good to me.}
\fi

\item p. 21, definition of F: Set $\to$ Cat

The use f, g for vertex/edge maps is opposite of the convention used on the previous page. I suggest that the same convention be used for both the 
decorated and structured approach to constructing open graphs.

{\bf At the beginning of Section 6.1, as well as in Sections 6.2, 6.3 and 6.4, we now use $f$ as a map between vertices and $g$ as a map 
between edges.} 

{\chris $\checkmark$} {\john $\checkmark$} {\kenny  $\checkmark$}

%{\chris Added a few more places where Kenny spotted these mistakes.}

%{\kenny Species and transitions are analogous to vertices and edges, so those maps were also changed to $f$s and $g$s, respectively.}

%{\chris Sure, are those in 6.4? As in, are there sections missing from the response sentence above?}

%{\kenny Yes, one map did get changed in Section 6.4. Good catch.} :)

\iffalse
{\kenny I \emph{think} they are complaining that we use $f$ as a map of vertices and $g$ as a map of edges when we're talking about the functor $F \colon \mathsf{Set} \to \mathsf{Cat}$ on page 21, but we use $g$ as a map of vertices and $f$ as a map of edges in the two squares at the start of Section 6.1 on the previous page?}

{\john That sounds plausible.  Unless there's a damn good reason let's use $f$ as a map for vertices and $g$ as a map for edges, since ``vertices come 
before edges'' - morally the first are 0-cells, the second are 1-cells.}

{\chris I see. Fixed, and see above.}
\fi

\item p. 27, end of Section 6.

It seems that the left adjoint needed for structured cospans fails to exist because the chosen category of dynamical systems does not have enough 
morphisms. Relatedly, this is the only example where the pseudofunctor D: (FinSet,+) $\to$ (Cat, $\times$) comes from a simple functor D: (FinSet,+) 
$\to$ (Set, $\times$) by viewing sets as discrete categories. In the other examples, this would count as using decorated cospans in the old, flawed 
way. 
All of this suggests the question: is there a way of constructing a category of dynamical systems that has more morphisms, so that a connection with 
structured cospans can be established?

{\bf That's a great question.   We do not know how to construct such a category, but the referee
is correct in suggesting that doing it might give a structured cospan category of open dynamical
systems.    

As the referee noticed, the ``old, flawed'' construction of 
decorated cospans from a lax monoidal functor $D \colon (\mathsf{FinSet},+) \to 
(\mathsf{Set}, \times) $ amounts to working with a discrete categories $D(S)$, treated
as sets.  Thus, our use of this construction amounts to working with a
set $D(S)$, rather than a full-fledged category, of dynamical systems on the finite set $S$.

In the example near the end of Section 6, the category $\int \! D$ can be seen as a category
of dynamical systems.   It is not discrete, but it does not have an initial object, as we'd
need to obtain a structured cospan
category that matches our decorated cospan category.   

So, giving the category of dynamical systems more morphisms sounds like a good idea. We will continue thinking about it.}  

{ \john $\checkmark$} {\chris $\checkmark$}

\iffalse
{\chris This looks similar to comment 30.}

{\john Yes, but it's worth a separate answer.   I wrote an answer above.  See if you like it.  }

{\chris Sounds good! However I am a bit stuck on the following. Remember also in 
\href{https://thalis.math.upatras.gr/~cvasilak/documents/MonGroth.pdf}{MonoidalGrothendieck}, we had this issue of a lax monoidal pseudofunctor not 
really being a lax monoidal functor in the split case, whatever I mean here. This also arised for Brendan's graph if I am not wrong. Really what is 
going on is that for a (strictly) lax monoidal functor into Cat, the laxator is strictly natural and associativity is an equality, whereas a (weakly) 
lax monoidal functor into Cat may be strict as a functor, but the laxator is still pseudonatural and associativity still an isomorphism, as per our 
diagram (5). So I am not 100\% sure the starting statement in the second paragraph above is true. Let us think about that for a moment, it may end up 
being true in this case.}

{\john  Oh-ho!  In this response I don't want to get pulled into any subtle math questions that aren't  actually necessary.  I was just trying to give 
the referee a pat on the back.  See if you like my new simplified reply, and bless it with a check mark if you do.}
\fi

\item p. 30, following Thm A.2

If I am not mistaken, “fibration”, “opfibration,” “monoidal (op/bi)fibration” are all defined, but the definition of “bifibration” is omitted. For 
completeness, it should be included.

{\bf Done.} 

{\chris $\checkmark$} {\john $\checkmark$}  {\kenny $\checkmark$}

\item A last general comment is that the paper unfortunately adopts the strategy, all too common in math papers, of putting all the examples in a 
section at 
the end of the paper. I think the readability of the paper would be improved, without much change to the organization, if the motivating example of 
open graphs were added to the relevant places in Sections 2 and 3. The remaining examples could be left in Section 6.

{\bf We believe that it is hard to analyze the examples (even the motivating one of open graphs) without having laid all relevant theory. Also, 
the comparison between the two formalisms---a major feature of Section 6---would then not be possible. In order to 
better direct the reader to the applications, we added one sentence in each introductory paragraph of Sections 2 and 3 that point to Section 6 for 
concrete examples, and also added a paragraph and the end of the Introduction (see comment 9). } 

{\chris $\checkmark$} {\kenny $\checkmark$} {\john $\checkmark$}

\iffalse
{\john
I don't want to bust up the examples and put some in sections 2 and 3. Anyone who wants examples can easily find them all in Section 6, we mention 
this section in the introduction (and perhaps should advertise it more there), and it's hard to analyze the examples without having all the theory in 
hand.}

{\chris I agree. How does the above boldface sound?}

{\kenny Agreed about keeping the examples together. The boldface looks good to me.}

{\john I tweaked the boldface a tiny bit, so you can revoke your checkmarks if you want; 
otherwise just comment out our little conversation here. By the way, you can comment out
a big chunk of text just by putting $\backslash$iffalse in front of it and $\backslash$fi at the end.  This is a special case of a trick for making 
some text appear only if some condition is true!}

{\chris Haha had already started doing it before seeing this!}
\fi

\vskip 1em

\item Composition of morphisms f, g in this paper is usually denoted gf, and occasionally gof. I’d suggest mentioning this in the “conventions” 
section.

{\bf We added the sentence ``For composition of morphisms we use concatenation or occasionally $\circ$.'' in the convention section.} 

{\chris $\checkmark$} {\kenny $\checkmark$} {\john $\checkmark$}

%{\chris I would be happy to add this to conventions, but I cannot seem to find gf anywhere in the paper. Anyone else?}

%{\color{blue}{Kenny: They might not mean $gf$ specifically, but there are many instances where we don't use a $\circ$ for functional composition, 
%e.g. 
%when we're pushing forward a graph structure over a map $f$ of vertices towards the top of page 21. Do we want to use a $\circ$ or not?}}

%{\chris Ah you are right. I don't mind, we could specify that for function composition we sometimes use concatenation!}

%{\john There were a couple places where I put in $\circ$ --- that's what the referee is really complaining
%about here.  Almost always we don't.  I could remove the $\circ$s.  There was a reason I put them in: there are some places where writing a 
%composite 
%using plain old concatenation looks weird to me.   But I don't think it's that big a deal.}

%{\chris By searching our paper, I find $\circ$ in many places that should stay in my opinion, for composition of things that are not functions 
%necessarily, and maybe only a couple for functions namely towards the end of 6.4. I also don't think it's a big deal.}

%{\chris I made the above edit, see if you agree or change at will.}

%{\kenny I think we use concatenation far more than $\circ$, right? I'm gonna' slightly change it to ``concatenation or occasionally $\circ$'' both in 
%the response comment above and the paper. Otherwise, looks good.}

%{\john I changed ``composition of maps'' to ``composition of morphisms'' above because people often use ``map'' to mean a morphism in a category of 
%sets with extra structure, and here we just mean composition of morphisms, or sometimes horizontal 1-cells, which I hope is close enough.  I hope 
%that's okay!}

%{\chris Sure, thanks to both!}

\item p3: “Using the Grothendieck construction” - maybe worth mentioning this is the “covariant” version which is different to the most common 
convention 
(at least I think so, but the literature seems to be very split over this)

{\bf It is indeed the case that we use what some people call ``covariant'' version of the Grothen\-dieck construction in this paper, however we 
believe 
that mentioning this in the introduction could put people off if they haven't heard of it. We added a comment about this in the 
appendix.} 

{\chris $\checkmark$} {\kenny $\checkmark$} {\john $\checkmark$}

\iffalse
{\chris Not sure that `covariant' would make this clearer...both are very well the Grothendieck construction I would think, but I don't mind that 
much.}

{\kenny Agree with Christina.}

{\chris See if you like the answer.}
\fi

%{\kenny Looks good.}

%{\john Great.  Except for one thing: I hate footnotes.   If something is worth saying, just say it, don't stick a little number in the text and make 
%the reader decide if it's worth looking down at the bottom of the page to see what this mysterious number means.  That's more distracting than just 
%saying it!   So I moved the comment into the text and merged it into the previous sentence.}

%{\chris OK!}

\item Theorem 2.1: Maybe worth saying explicitly that $s$ is an object of the category $F(m)$ rather than using $\in$, since this is exactly the  distinction to  Fong's original definition.

{\bf  We have added a remark emphasizing this right after Theorem 2.1.}   

{\john $\checkmark$} {\chris $\checkmark$} {\kenny $\checkmark$}

%{\chris Maybe just specify in the statement of 2.1, definitely keep $\in$ sign in all diagrams below!}

%{\john It's hard to tweak the statement, since we're really trying to introduce the notation of decorations in maps of structured cospans here!    
%So 
%I've added something about this \emph{after} the statement of Theorem 2.1.  See if you like it:
%\begin{quote} 
%Note that the decoration $s \in F(m)$ is now an object in the category $F(m)$, not an element of a set as in Fong's original approach. 
%\end{quote}}

%{\chris Looks good!}

\item Proof of theorem 2.1: You should probably at least handwave that the cartesian liftings really are cartesian liftings. (No doubt it’s easy, but 
this 
is a classic sort of spot where errors can hide.)

\textbf{Kenny Courser carefully checked that these really are co/cartesian liftings, but the argument is straightforward and uninteresting so we will just 
add remarks saying that this can be shown.}   

{\john $\checkmark$} {\chris $\checkmark$} {\kenny $\checkmark$}

\iffalse
{\kenny I think we originally had these details in our proof (of both Theorem 2.1 and Theorem 3.1) but removed them.}

{\john It's your evil advisor who removed those details.   I'll ``handwave it''---see above.}
\fi

\item On page 9 during the construction of $\mathbb{H}$, I would appreciate a sentence giving some intuition for the functor $E$, at this point I’m confused about  its  role.

{\bf  We have added a remark on page 9 discussing this.} 

{\john $\checkmark$} {\chris $\checkmark$} {\kenny $\checkmark$}

\iffalse
{\color{blue}{Kenny: We don't have any examples where $E$ is not an identity, but in my mind it's a functor that possibly modifies decorations, e.g. 
$F(a)$ is a category of graph structures on $a$, and $E(F(a))$ could be something like a category of possibly labeled graph structures on $a$, if 
that 
makes sense. Something potentially different than \emph{just} a graph structure on $a$.}}

{\chris Right! We can perhaps add a sentence to this end, saying that although we only use E as an identity, someone could need to change 
decorations 
for some application?}

{\kenny Sure, maybe sometime before the official statement of the Theorem when we're describing how the double functor $\mathbb{H}$ is defined?}

{\chris Sounds good to me!}

{\kenny I've added a long run-on sentence beneath the square on page 9 trying to say more about the functors $H$ and $E$ from an intuitive standpoint. Edit at will.}

{\john I edited it --- take a looks, folks and see if you're happy. I'm gonna give this one my check mark.  I actually think the referee's implicit point is right: it'd be more natural in some ways to require $E$ to be the identity, so we'd be working in the 2-category of categories over $\mathsf{Cat}$ and lax maps between these.  But I think we can stick with what we've got,
which is more general.}

{\kenny Looks good to me.}
\fi

\item Theorem 2.5: $F$ should be defined on $\mathsf{A}$, not $\mathsf{A}'$.

{\bf Done.}

 {\chris $\checkmark$} {\kenny $\checkmark$} {\john $\checkmark$}

\item I think it would be worth saying in words a bit more prominently what the hypothesis of theorem 4.1 roughly means, namely that the fibers of 
the total 
category have finite colimits, reindexing preserves them, and the monoidal product on the fibers is cocartesian.

{\bf We have added a paragraph right below the statement of Theorem 4.1, clarifying its assumptions as suggested, which also incorporates an 
explanatory paragraph previously located under Corollary 4.3.} 

{\chris $\checkmark$} {\kenny $\checkmark$} {\john $\checkmark$} 

\item I wonder if the “problem” where DCsp is not a structured cospan category could be “fixed” by upgrading the functor from Set to Cat in a 
non-discrete 
way, for example with an ordering with the zero vector field as the bottom element. I suppose this would give you different 2-cells in the resulting 
double category or bicategory, but perhaps it would yield the same 1-category.

{\bf This is an interesting idea but we haven't been able to get it to work yet.  See also comment 20.}

{\john $\checkmark$} {\chris $\checkmark$}
\end{enumerate}


\end{document}
