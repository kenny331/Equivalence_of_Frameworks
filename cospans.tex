% Structured versus decorated cospans
% John Baez, Kenny Courser and Christina Vasilakopoulou
% in LaTeX
% 2020/12/14

\documentclass[reqno]{amsart}
\usepackage{amssymb,amsmath,stmaryrd,txfonts,mathrsfs,amsthm}

\usepackage[all,2cell]{xy}\UseAllTwocells\SilentMatrices
\usepackage[neveradjust]{paralist}
\usepackage{hyperref}
\usepackage{mathtools}
\usepackage{multirow}
\usepackage[outline]{contour}
\contourlength{1.2pt}
\usepackage{tikz}
\usepackage{tikz-cd}
\usepackage{xcolor}
\usepackage{framed,color}
\usepackage[draft]{fixme}
\usetikzlibrary{matrix,arrows,decorations.pathmorphing,positioning}
\usetikzlibrary{intersections,decorations.markings}
\usetikzlibrary{arrows,positioning,fit,matrix,shapes.geometric,external}
\usetikzlibrary{backgrounds,circuits,circuits.ee.IEC,shapes,fit,matrix}
\usepackage{tikz}
\usetikzlibrary{matrix,arrows}
\usepackage{comment}
\usepackage[capitalize]{cleveref}
\definecolor{rewritecolor}{rgb}{0,.9,1}
\tikzset{rewritenode/.style={shape=circle,fill=rewritecolor,scale=0.25,font=\Huge}}
\tikzset{RWopen/.style={shape=circle,draw=black,fill=white,scale=0.5,font=\Huge}}
\tikzset{RWclosed/.style={shape=circle,fill=black,scale=0.5,font=\Huge}}
\tikzset{CDnode/.style={shape=circle,fill=white,scale=.5}}
\makeatletter
\let\ea\expandafter

\pgfdeclarelayer{edgelayer}
\pgfdeclarelayer{nodelayer}
\pgfsetlayers{edgelayer,nodelayer,main}

% Petri nets
\definecolor{lblue}{rgb}{0,250,255}
\tikzstyle{species}=[circle,fill=yellow,draw=black,scale=1.15]
\tikzstyle{transition}=[rectangle,fill=lblue,draw=black,scale=1.15]
\tikzstyle{inarrow}=[->, >=stealth, shorten >=.03cm,line width=1.5]
\tikzstyle{empty}=[circle,fill=none, draw=none]
\tikzstyle{inputdot}=[circle,fill=purple,draw=purple, scale=.25]
\tikzstyle{inputarrow}=[->,draw=purple, shorten >=.05cm]
\tikzstyle{simple}=[-,draw=purple,line width=1.000]
\tikzstyle{none}=[inner sep=0pt]

\definecolor{shadecolor}{rgb}{1,0.8,0.3}
\definecolor{myurlcolor}{rgb}{0.6,0,0}
\definecolor{mycitecolor}{rgb}{0,0,0.8}
\definecolor{myrefcolor}{rgb}{0,0,0.8}
\hypersetup{colorlinks, linkcolor=myrefcolor, citecolor=mycitecolor, urlcolor=myurlcolor}

\tikzset{->-/.style={decoration={
  markings,
  mark=at position .5 with {\arrow{>}}},postaction={decorate}}}

%% Defining commands that are always in math mode.
\def\mdef#1#2{\ea\ea\ea\gdef\ea\ea\noexpand#1\ea{\ea\ensuremath\ea{#2}}}
\def\alwaysmath#1{\ea\ea\ea\global\ea\ea\ea\let\ea\ea\csname your@#1\endcsname\csname #1\endcsname
  \ea\def\csname #1\endcsname{\ensuremath{\csname your@#1\endcsname}}}
\newcommand{\define}[1]{{\bf \boldmath{#1}}}

% blackboard bold letters
\newcommand{\lA}{\ensuremath{\mathbb{A}}}
\newcommand{\lC}{\ensuremath{\mathbb{C}}}
\newcommand{\lD}{\ensuremath{\mathbb{D}}}
\newcommand{\lE}{\ensuremath{\mathbb{E}}}
\newcommand{\lR}{\ensuremath{\mathbb{R}}}
\newcommand{\lX}{\ensuremath{\mathbb{X}}}
\mdef\fahat{\hat{\fa}}

% MISCELLANEOUS SYMBOLS
\newcommand{\inv}{^{-1}}
\newcommand{\op}{^{\mathit{op}}}
\newcommand{\co}{^{\mathit{co}}}
\newcommand{\coop}{^{\mathit{coop}}}
\newcommand{\id}{\mathm{id}}
\let\adj\dashv
\newcommand{\pullbackcorner}[1][dr]{\save*!/#1-1.2pc/#1:(-1,1)@^{|-}\restore}
\let\iso\cong
\let\eqv\simeq
\let\cng\equiv
\mdef\Id{\mathrm{Id}}
\mdef\id{\mathrm{id}}
\alwaysmath{ell}
\alwaysmath{infty}
\alwaysmath{odot}
\def\frc#1/#2.{\frac{#1}{#2}}   % \frc x^2+1 / x^2-1 .
\mdef\ten{\mathrel{\otimes}}

%% OPERATORS
\DeclareMathOperator\colim{colim}
\DeclareMathOperator\eq{eq}
\DeclareMathOperator\Aut{Aut}
\DeclareMathOperator\End{End}
\DeclareMathOperator\Hom{Hom}
\DeclareMathOperator\Map{Map}

%% ARROWS
% \to already exists
\newcommand{\too}[1][]{\ensuremath{\overset{#1}{\longrightarrow}}}
\newcommand{\oot}[1][]{\ensuremath{\overset{#1}{\longleftarrow}}}
\let\toot\rightleftarrows
\let\otto\leftrightarrows
\let\maps\colon

%% EXTENSIBLE ARROWS
\let\xto\xrightarrow
\let\xot\xleftarrow

% THEOREM-TYPE ENVIRONMENTS, hacked to
%% (a) number all with the same numbers, and
%% (b) have the right names for autoref
\def\defthm#1#2{%
  \newtheorem{#1}{#2}[section]%
  \expandafter\def\csname #1autorefname\endcsname{#2}%
  \expandafter\let\csname c@#1\endcsname\c@thm}
\newtheorem{thm}{Theorem}[section]
\newcommand{\thmautorefname}{Theorem}
\defthm{cor}{Corollary}
\defthm{prop}{Proposition}
\defthm{lem}{Lemma}
\defthm{conj}{Conjecture}
\defthm{hyp}{Hypothesis}
\defthm{fact}{Fact}
\theoremstyle{definition}
\defthm{defn}{Definition}
\defthm{notn}{Notation}
\theoremstyle{remark}
\defthm{rmk}{Remark}
\defthm{eg}{Example}

\newcommand{\fhat}{\ensuremath{\hat{f}}}

% Also number formulas with the theorem counter
\let\c@equation\c@thm
\numberwithin{equation}{section}

% Only show numbers for equations that are actually referenced (or
% whose tags are specified manually) - uses mathtools.
%\mathtoolsset{showonlyrefs,showmanualtags}

\def\tobar{\mathrel{\mkern3mu  \vcenter{\hbox{$\scriptscriptstyle+$}}%
                    \mkern-12mu{\to}}}

%\input{decls}
\UseAllTwocells

\newcommand{\dblcat}[1]{\mathbb{#1}}
\mdef\fchk{\check{f}}

\definecolor{purple(x11)}{rgb}{0.5, 0.0, 0.5}
\def\purple{\color{purple(x11)}}
\def\chris{\purple}

%Christina: change below accordingly if needed!
\newcommand{\ca}{\mathsf}
\newcommand{\bicat}{\mathbf}
\newcommand{\U}{U}
\newcommand{\D}{\ca{A}}
\newcommand{\C}{\ca{X}} 
\newcommand{\A}{\ca{A}}
\newcommand{\B}{\ca{B}}
\newcommand{\X}{\ca{X}}
\newcommand{\dcsp}[1]{{#1}\mathbb{C}\textnormal{sp}}
\tikzset{tick/.style={postaction={decorate,decoration={markings,
mark=at position 0.4 with {\draw[-] (0,.4ex) -- (0,-.4ex);}}}}}
\newcommand{\tickar}{\begin{tikzcd}[baseline=-0.5ex,cramped,sep=small,ampersand replacement=\&]{}\ar[r,tick]\&{}\end{tikzcd}}
\newcommand{\cspn}[5]{\begin{tikzcd}[baseline=-0.5ex,cramped,sep=small,ampersand replacement=\&]{#1}\ar[r,"#4"] \& {#2} \& {#3}\ar[l,"#5"']\end{tikzcd}}
\newcommand{\OpICat}{\bicat{OpICat}}%probably too much to use?
\newcommand{\pse}{\mathrm{ps}}
\newcommand{\OpFib}{\bicat{OpFib}}


\title{Structured Versus Decorated Cospans}

\author{John\ C.\ Baez$^{1,2}$, Kenny Courser$^1$, and Christina Vasilakopoulou$^3$}
\address{$^1$Department of Mathematics, University of California, Riverside CA, USA 92521}
\address{$^2$Centre for Quantum Technologies, National University of Singapore, Singapore 117543}
\address{$^3$Department of Mathematics, University of Patras, Greece 265 04}
\email{baez@math.ucr.edu, kcour001@ucr.edu, cvasilak@math.upatras.gr}

\begin{document}

\begin{abstract}
One goal of applied category theory is to understand open systems.  We compare two ways of describing open systems as cospans equipped with extra data.    First, given a functor $L \maps \A \to \X$, a `structured cospan' is a diagram in $\A$ of the form $L(a) \rightarrow x \leftarrow L(b)$.  If $\A$ and $\X$ have finite colimits and $L$ preserves them, it is known that there is a symmetric monoidal double category whose objects are those of $\A$ and whose horizontal 1-cells are structured cospans.   Second, if $\A$ has finite colimits and $F \maps (\A,+) \to (\Cat,\times)$ is a symmetric lax monoidal pseudofunctor , a `decorated cospan' is a diagram in $\A$ of the form $a \rightarrow s \leftarrow b$ together with an object of $F(s)$.   Generalizing the work of Fong, we show that in this situation there is a symmetric monoidal double category whose objects are those of $\A$ and whose horizontal 1-cells are decorated cospans.   We prove that under certain conditions these two constructions become equivalent when we take $\X = \int F$ to be the Grothendieck category of $F$.  We illustrate these ideas with applications to electrical circuits, Petri nets, epidemiology and dynamical systems.
\end{abstract}

\maketitle

\setcounter{tocdepth}{1}
\tableofcontents

\section{Introduction}

An `open system' is any sort of system that can interact with the outside world.  Experience has shown that open systems are nicely modeled using cospans \cite{CourserThesis, FongThesis, PollardThesis}. A cospan in some category $\A$ is a diagram of this form:
\[
\begin{tikzpicture}[scale=1.5]
\node (A) at (0,0) {$a$};
\node (B) at (1,1) {$m$};
\node (C) at (2,0) {$b$};
\path[->,font=\scriptsize,>=angle 90]
(A) edge node[above]{$i$} (B)
(C) edge node[above]{$o$} (B);
\end{tikzpicture}
\]
We call $m$ the \define{apex}, $a$ and $b$ the \define{feet}, and $i$ and $o$ the \define{legs} of the cospan.   The apex describes the system itself.  The feet describe `interfaces'  through which the system can interact with the outside world.  The legs describe how the interfaces are included in the system.   If the category $\A$ has finite colimits, we can compose cospans using pushouts: this describes the operation of attaching two open systems together in series by identifying one interface of the first with one of the second.  We can also `tensor' cospans using coproducts: this describes setting open systems side by side, in parallel.  Via these operations we obtain a symmetric monoidal double category
%%%Christina: add the notation bb{C}sp(A) since it will be central to what follows?
with cospans in $\A$ as its horizontal 1-cells \cite{Niefield}.
%%% Christina: I would be inclined to first cite Niefield and then Courser due to the chronological order.
%%% Kenny: I think when we \cite{...}, we list the references alphabetically, but let's just cite Niefield here and achieve both.
%%% Christina: Sounds good!

However, we often want the system itself to have more structure than its interfaces.   This led Fong to develop a theory of `decorated' cospans \cite{Fong}.  Given a category $\A$ with finite colimits, a symmetric lax monoidal functor $F \maps (\A,+) \to (\textsf{Set},\times)$ can be used to equip the apex $m$ of a cospan in $\A$ with some extra data: an element $x \in F(m)$, which we call a \textbf{decoration}.  Thus a \define{decorated cospan} is a pair:
\[
\begin{tikzpicture}[scale=1.5]
\node (A) at (0,0) {$a$};
\node (B) at (1,0) {$m$};
\node (C) at (2,0) {$b$,};
\node (E) at (4,0) {$x \in F(m)$.};
\path[->,font=\scriptsize,>=angle 90]
(A) edge node[above]{$i$} (B)
(C) edge node[above]{$o$} (B);
\end{tikzpicture}
\]
Fong proved that there is a symmetric monoidal category with objects
of $\A$ as its objects and equivalence classes of decorated cospans as its morphisms.  Such categories were used to describe a variety of open systems: electrical circuits, Markov processes, chemical reaction networks and dynamical systems \cite{BF,BFP,BP}. 

Unfortunately, many applications of decorated cospans were flawed.  The problem is that while Fong's decorated cospans are good for decorating the apex $m$ with an element of a set $F(m)$, they are unable to decorate it with an object of a category.   An example would be equipping a finite set $m$ with edges making its elements into the nodes of a graph.    We would like the following `open graph' to be a decorated cospan where the apex is the finite set $m = \{n_1, n_2, n_3, n_4\}$:  
\[
\scalebox{0.8}{
\begin{tikzpicture}
	\begin{pgfonlayer}{nodelayer}
		\node [contact] (n1) at (-2,0) {$\bullet$};
		\node [style = none] at (-2.1,0.3) {$n_1$};
		\node [contact] (n2) at (0,1) {$\bullet$};
		\node [style = none] at (0,1.3) {$n_2$};
		\node [contact] (n3) at (0,-1) {$\bullet$};
		\node [style = none] at (0,-1.3) {$n_3$};
		\node [contact] (n4) at (2,0) {$\bullet$};
		\node [style = none] at (2.1,0.3) {$n_4$};
		
		\node [style = none] at (-1,1) {$e_1$};
		\node [style = none] at (-1,-1) {$e_2$};
		\node [style = none] at (1,1) {$e_3$};
		\node [style = none] at (1,-1) {$e_4$};
	    \node [style = none] at (0.3,0) {$e_5$};
		
		\node [style=none] (1) at (-3,0) {1};
		\node [style=none] (4) at (3,0) {2};
	
		\node [style=none] (ATL) at (-3.4,1.4) {};
		\node [style=none] (ATR) at (-2.6,1.4) {};
		\node [style=none] (ABR) at (-2.6,-1.4) {};
		\node [style=none] (ABL) at (-3.4,-1.4) {};

		\node [style=none] (X) at (-3,1.8) {$a$};
		\node [style=inputdot] (inI) at (-2.8,0) {};
		
		\node [style=none] (Z) at (3,1.8) {$b$};
	 \node [style=inputdot] (outI') at (2.8,0) {};

		\node [style=none] (MTL) at (2.6,1.4) {};
		\node [style=none] (MTR) at (3.4,1.4) {};
		\node [style=none] (MBR) at (3.4,-1.4) {};
		\node [style=none] (MBL) at (2.6,-1.4) {};
	
	\end{pgfonlayer}
	\begin{pgfonlayer}{edgelayer}
		\draw [style=inarrow, bend left=20, looseness=1.00] (n1) to (n2);
		\draw [style=inarrow, bend right=20, looseness=1.00] (n1) to (n3);
		\draw [style=inarrow, bend left=20, looseness=1.00] (n2) to (n4);
		\draw [style=inarrow, bend right=20, looseness=1.00] (n3) to (n4);
		\draw [style=inarrow] (n2) to (n3);
%		\draw [style=inarrow] (W) to (Water);
%		\draw [style=inarrow, bend left=40, looseness=1.00] (Water2) to (Something);
%		\draw [style=inarrow, bend right=40, looseness=1.00] (Water2) to (Something);
%		\draw [style=inarrow, bend left=40, looseness=1.00] (Something) to (A);
%		\draw [style=inarrow, bend right=40, looseness=1.00] (Something) to (B);
		\draw [style=simple] (ATL.center) to (ATR.center);
		\draw [style=simple] (ATR.center) to (ABR.center);
		\draw [style=simple] (ABR.center) to (ABL.center);
		\draw [style=simple] (ABL.center) to (ATL.center);
%		\draw [style=simple] (BTL.center) to (BTR.center);
%		\draw [style=simple] (BTR.center) to (BBR.center);
%		\draw [style=simple] (BBR.center) to (BBL.center);
%		\draw [style=simple] (BBL.center) to (BTL.center);
		\draw [style=simple] (MTL.center) to (MTR.center);
		\draw [style=simple] (MTR.center) to (MBR.center);
		\draw [style=simple] (MBR.center) to (MBL.center);
		\draw [style=simple] (MBL.center) to (MTL.center);
%		\draw [style=inputarrow] (outI) to (A);
%		\draw [style=inputarrow] (outS) to (B);
		\draw [style=inputarrow] (inI) to (n1);
		\draw [style=inputarrow] (outI') to (n4);
%		\draw [style=inputarrow] (inI') to (Water2);
%		\draw [style=inputarrow] (inS') to (Water2);
	\end{pgfonlayer}
\end{tikzpicture}
}
\]
We might hope to do this using a symmetric lax monoidal functor $F \maps (\Fin\Set, +) \to (\Set, \times)$ assigning to each finite set $m$ the set of all graphs with $m$ as their set of nodes. But this hope is doomed, for reasons painstakingly explained in \cite[Section 5]{BC}. 
There is really a \emph{category} of graphs with $m$ as their set of nodes---and surprisingly, trying to treat this category as a mere set does not work, despite all the tricks one might try.

Here we present a solution to this problem.  Instead of basing the theory of decorated cospans on a symmetric lax monoidal functor $F \maps (\A, +) \to (\Set, \times)$, we use a symmetric lax monoidal pseudofunctor $F \maps (\A, +) \to (\Cat, \times)$.  In \cref{thm:decorated_cospans,DC}, we use this
data to construct a symmetric monoidal double category $F\lCsp$ in which:
\begin{itemize}
\item an object is an object of $\A$,
\item a vertical 1-morphism is a morphism of $\A$,
\item a horizontal 1-cell from $a$ to $b$ is a decorated cospan:
\[
\begin{tikzpicture}[scale=1.5]
\node (A) at (0,0) {$a$};
\node (B) at (1,0) {$m$};
\node (C) at (2,0) {$b,$};
\node (D) at (3,0) {$x \in F(m)$,};
\path[->,font=\scriptsize,>=angle 90]
(A) edge node[above]{$i$} (B)
(C) edge node[above]{$o$} (B);
\end{tikzpicture}
\]
\item a 2-morphism is a \define{map of decorated cospans}: that is, a commutative
diagram
\[
\begin{tikzpicture}[scale=1.5]
\node (A) at (0,0.5) {$a$};
\node (A') at (0,-0.5) {$a'$};
\node (B) at (1,0.5) {$m$};
\node (C) at (2,0.5) {$b$};
\node (C') at (2,-0.5) {$b'$};
\node (D) at (1,-0.5) {$m'$};
\node (E) at (3,0.5) {$x \in F(m)$};
\node (F) at (3,-0.5) {$x' \in F(m')$};
\path[->,font=\scriptsize,>=angle 90]
(A) edge node[above]{$i$} (B)
(C) edge node[above]{$o$} (B)
(A) edge node[left]{$f$} (A')
(C) edge node[right]{$g$} (C')
(A') edge node[above] {$i'$} (D)
(C') edge node[above] {$o'$} (D)
(B) edge node [left] {$h$} (D);
\end{tikzpicture}
\]
together with a morphism $\tau \maps F(h)(x) \to m'$ in $F(x')$.
\end{itemize}

In fact another solution to the problem is already known: the theory of structured cospans \cite{BC,CourserThesis}.  Given a functor $L \maps \A \to \X$, a \define{structured cospan} is a cospan in $\X$ whose feet come from a pair of objects in $\A$:
\[
\begin{tikzpicture}[scale=1.2]
\node (A) at (0,0) {$L(a)$};
\node (B) at (1,1) {$x$};
\node (C) at (2,0) {$L(b).$};
\path[->,font=\scriptsize,>=angle 90]
(A) edge node[above]{$$} (B)
(C)edge node[above]{$$}(B);
\end{tikzpicture}
\]
This is another way of letting the apex have more structure than the feet.   When $\A$ and $\X$ have finite colimits and $L$ preserves them, there is a symmetric monoidal double category ${}_L \lCsp(\X)$ where:
\begin{itemize}
\item an object is an object of $\A$,
\item a vertical 1-morphism is a morphism of $\A$,
\item a horizontal 1-cell from $a$ to $b$ is a diagram in $\X$ of this form:
\[
\begin{tikzpicture}[scale=1.5]
\node (A) at (0,0) {$L(a)$};
\node (B) at (1,0) {$x$};
\node (C) at (2,0) {$L(b)$};
\path[->,font=\scriptsize,>=angle 90]
(A) edge node[above]{$i$} (B)
(C)edge node[above]{$o$}(B);
\end{tikzpicture}
\]
\item a 2-morphism is a commutative diagram in $\X$ of this form:
\[
\begin{tikzpicture}[scale=1.5]
\node (E) at (3,0) {$L(a)$};
\node (F) at (5,0) {$L(b)$};
\node (G) at (4,0) {$x$};
\node (E') at (3,-1) {$L(a')$};
\node (F') at (5,-1) {$L(b')$};
\node (G') at (4,-1) {$x'$};
\path[->,font=\scriptsize,>=angle 90]
(F) edge node[above]{$o$} (G)
(E) edge node[left]{$L(f)$} (E')
(F) edge node[right]{$L(g)$} (F')
(G) edge node[left]{$\alpha$} (G')
(E) edge node[above]{$i$} (G)
(E') edge node[below]{$i'$} (G')
(F') edge node[below]{$o'$} (G');
\end{tikzpicture}
\]
\end{itemize}
Many of the flawed applications of decorated cospans have been fixed using structured cospans \cite[Section 6]{BC}, but not all decorated cospans can handled using structured cospans.

Here we give sufficient conditions for a decorated cospan double category to be equivalent to a structured cospan double category.  Suppose $\A$ has finite colimits and $F \maps (\A , +) \to (\Cat, \times)$ is a symmetric lax monoidal pseudofunctor.   Any such pseudofunctor gives an opfibration $U \maps \X \to \A$ where $\X = \inta F$ is defined by the Grothendieck construction.  Let $\Rex$ be the 2-category of categories with finite colimits, functors preserving finite colimits, and natural transformations.  We show that if $F$ factors through $\mathbf{Rex}$ as a pseudofunctor, the opfibration $U \maps \X \to \A$ is also a right adjoint.  From the accompanying left adjoint $L \maps \A \to \X$, we  construct a symmetric monoidal double category ${}_L \lCsp(\X)$ of structured cospans.  Finally, in \cref{thm:equiv} we prove that this structured cospan double category ${}_L \lCsp(\X)$ is equivalent to the decorated cospan double category $F \lCsp$. 

\subsection*{Outline}

In \cref{DecCospansDoubleCat}, we construct the symmetric monoidal double category $F\lCsp$ and we define maps between decorated cospan double categories. In \cref{EquivDoubleCats}, we briefly review the structured cospans framework and prove that the double categorical versions of decorated cospans and structured cospans are equivalent under suitable conditions. In \cref{spinoffs}, we establish the equivalence between structured and decorated cospans at the level of bicategories and categories (via decategorification).
Finally, in \cref{Applications}, we describe applications to electrical circuits, Petri nets, epidemiology and dynamical systems.

\subsection*{Conventions}

In this paper, we use a sans-serif font like $\C$ for categories, boldface like $\mathbf{B}$ for bicategories or 2-categories, and blackboard bold like $\lD$ for double categories. For double categories with names having more than one letter, like $\lCsp(\X)$, only the first letter is in blackboard bold. In this paper, `double category' means `pseudo double category', as in \cref{defn:double_category}. A double category $\lD$ has a category of objects and a category of arrows, and we call these $\lD_0$ and $\lD_1$ despite the fact that they are categories. Vertical composition in our double categories is strictly associative, while horizontal composition need not be. 

\subsection*{Acknowledgements}

We thank Daniel Cicala, Brendan Fong, and Joe Moeller for helpful conversations. The third author would  like  to  thank  the  General  Secretariat  for  Research  and Technology (GSRT) and the Hellenic Foundation for Research and Innovation (HFRI).

\section{Decorated cospans}\label{DecCospansDoubleCat}

In this section we build symmetric monoidal double categories of decorated cospans, and then study the functoriality of this construction. The definition of a lax monoidal pseudofunctor is recalled in \cref{subsec:bicats}, and the definition of a double category is recalled in \cref{defn:double_category}.  

In all that follows, when we say a category `has finite colimits' we mean it is equipped with a choice of colimit for every finite diagram.   Thus, if $\A$ has finite colimits it gives a cocartesian monoidal category $(\A,+)$: a symmetric monoidal category where the monoidal structure is given by the chosen binary coproducts and initial object.   However, when we say a functor `preserves finite colimits', it need only do this up to canonical isomorphism, unless otherwise specified.

\begin{thm}\label{thm:decorated_cospans}
Let $\A$ be a category with finite colimits and $F \maps (\A,+) \to (\Cat,\times)$ a lax monoidal pseudofunctor. Then there exists a double category $F\lCsp$ in which
\begin{itemize}
\item an object is an object of $\A$,
\item a vertical 1-morphism is a morphism of $\A$,
\item a horizontal 1-cell is an $F$\define{-decorated cospan}, that is, 
a cospan in $\A$:
\[
\begin{tikzpicture}[scale=1.5]
\node (A) at (0,0) {$a$};
\node (B) at (1,0) {$m$};
\node (C) at (2,0) {$b,$};
%\node (D) at (2.6,0) {$x \in F(m))$};
\path[->,font=\scriptsize,>=angle 90]
(A) edge node[above]{$i$} (B)
(C) edge node[above]{$o$} (B);
\end{tikzpicture}
\]
together with a \define{decoration} $x \in F(m)$,
\item a 2-morphism is a \define{map of} $F$\define{-decorated cospans}, that is, 
a map of cospans in $\A$:
\begin{equation}\label{eq:mapofdec}
\begin{tikzpicture}[scale=1.5,baseline=(current bounding box.center)]
\node (A) at (0,0.5) {$a$};
\node (A') at (0,-0.5) {$a'$};
\node (B) at (1,0.5) {$m$};
\node (C) at (2,0.5) {$b$};
\node (C') at (2,-0.5) {$b'$};
\node (D) at (1,-0.5) {$m'$};
\node (E) at (3,0.5) {$x \in F(m)$};
\node (F) at (3,-0.5) {$x' \in F(m')$};
\path[->,font=\scriptsize,>=angle 90]
(A) edge node[above]{$i$} (B)
(C) edge node[above]{$o$} (B)
(A) edge node[left]{$f$} (A')
(C) edge node[right]{$g$} (C')
(A') edge node[above] {$i'$} (D)
(C') edge node[above] {$o'$} (D)
(B) edge node [left] {$h$} (D);
\end{tikzpicture}
\end{equation}
together with a \define{decoration morphism} $\tau \maps F(h)(x) \to x'$ in $F(m')$.
\end{itemize}
\end{thm}

\begin{proof}
Take $\A$ to have finite colimits and let $(F,\phi,\phi_0) \maps (A,+) \to (\Cat,\times)$ be a lax monoidal pseudofunctor.  We shall define the category of objects $F\lCsp_0$, the category of arrows $F\lCsp_1$, as well as the four structure functors which tie them together: the source and target functors $S,T \colon F\lCsp_1 \to F\lCsp_0$, the composition functor $\odot \colon F\lCsp_1 \times_{F\lCsp_0} F\lCsp_1 \to F\lCsp_1$ and the unit functor $U \colon F\lCsp_0 \to F\lCsp_1$.

The category of objects $F\lCsp_0$ is simply $\A$, whereas $F$-decorated cospans and maps between them form the category of arrows $F\lCsp_1$. The (vertical) composition of two $F$-decorated cospans maps is 
\begin{equation}\label{eq:verticalcompo}
 \begin{tikzcd}
a\ar[r,"i"]\ar[d,"f"'] & m\ar[d,"h"] & b\ar[l,"o"']\ar[d,"g"]  & x\in F(m) \\
a'\ar[r,"{i'}"]\ar[d,"{f'}"'] & m'\ar[d,"{h'}"] & b'\ar[l,"{o'}"']\ar[d,"{g'}"] & x'\in F(m') \\
a''\ar[r,"{i''}"] & m'' & b''\ar[l,"{o''}"'] & x''\in F(m'')
 \end{tikzcd}\;=\;
 \begin{tikzcd}[column sep=.3in]
a\ar[r,"i"]\ar[dd,"{f'f}"'] & m\ar[dd,"{h'h}"] & b\ar[l,"o"']\ar[dd,"{g'g}"]  & x\in F(m) \\
\hole \\
a''\ar[r,"{i''}"] & m'' & b''\ar[l,"{o''}"'] & x''\in F(m'')  
 \end{tikzcd}
\end{equation}
where if $\tau \colon F(h)(x) \to x'$ and $\tau' \colon F(h')(x') \to x''$ are the decoration morphisms, 
\begin{equation}\label{eq:tauofverticalcomposite}
F(h'h)(x)\cong F(h')F(h)(x)\xrightarrow{F(h')(\tau)}F(h')(x')\xrightarrow{\tau'}x''
\end{equation}
is the decoration of the composite.
The source and target functors map an $F$-decorated cospan to the source and target of the cospan and a map of decorated cospans \cref{eq:mapofdec} to $f$ and $g$ respectively. The unit functor maps an object $a$ to 
\begin{equation}\label{eq:UFCsp}
\begin{tikzcd}
a\ar[r,"1_a"] & a & a,\ar[l,"1_a"'] & \bot_a:=\one\xrightarrow{\phi_0}F(0)\xrightarrow{F(!_a)}F(a)
\end{tikzcd}
\end{equation}
where $\bot_a \in F(a)$ is called the \define{trivial decoration}, and a vertical 1-morphism $f\maps a\to a'$ to
\[
\begin{tikzpicture}[scale=1.5]
\node (A) at (0,0.5) {$a$};
\node (A') at (0,-0.5) {$a'$};
\node (B) at (1,0.5) {$a$};
\node (C) at (2,0.5) {$a$};
\node (C') at (2,-0.5) {$a'$};
\node (D) at (1,-0.5) {$a'$};
\node (E) at (3,0.5) {$\bot_a \in F(a)$};
\node (F) at (3,-0.5) {$\bot_{a'} \in F(a')$};
\path[->,font=\scriptsize,>=angle 90]
(A) edge node[above]{$1_a$} (B)
(C) edge node[above]{$1_a$} (B)
(A) edge node[left]{$f$} (A')
(C) edge node[right]{$f$} (C')
(A') edge node[above] {$1_{a'}$} (D)
(C') edge node[above] {$1_{a'}$} (D)
(B) edge node [left] {$f$} (D);
\end{tikzpicture}
\]
together with the decoration morphism $F(f)F(!_a)\phi_0(*)\cong F(!_{a'})\phi_0(*)$.
Finally, the (horizontal) composition functor $\odot$ maps two composable $F$-decorated cospans $M=\left(a\to m\leftarrow b,x\in F(m)\right),N=\left(b\to n\leftarrow c,y\in F(n)\right)$ to $N\odot M$, which as a cospan is their pushout in $\A$ over their shared foot
\begin{equation}\label{eq:dcospanscomposition}
\begin{tikzpicture}[scale=1.2,baseline=(current bounding box.center)]
\node (A) at (0,0) {$a$};
\node (B) at (1,1) {$m$};
\node (C) at (2,0) {$b$};
\node (D) at (3,1) {$n$};
\node (E) at (4,0) {$c$};
\node (F) at (2,2) {$m+n$};
\node (G) at (2,3) {$m+_{b} n$};
\path[->,font=\scriptsize,>=angle 90]
(A) edge node[above]{$i$} (B)
(C) edge node[above]{$o$} (B)
(C) edge node [above] {$i'$} (D)
(E) edge node [above] {$o'$} (D)
(B) edge node [below] {$j_m$} (F)
(D) edge node [below] {$j_{n}$} (F)
(B) edge[bend left=10] node [above] {$u_m\;$} (G)
(D) edge[bend right=10] node [above] {$\;u_n$} (G)
(F) edge node [fill=white] {$\psi$} (G)
(A) edge[dashed,bend left] node [left] {} (G)
(E) edge[dashed,bend right] node [right] {} (G);
\end{tikzpicture}
\end{equation}
equipped with the decoration
$$y\odot x:= \one\cong\one \times \one \xrightarrow{x \times y} F(m) \times F(n) \xrightarrow{\phi_{m,n}} F(m+n) \xrightarrow{F(\psi)} F(m+_{b}n)$$
where $\psi \maps m + n \to m+_{b} n$ is the natural map from the coproduct to the pushout and $\phi_{m,n} \maps F(m) \times F(n) \to F(m+n)$ are the lax monoidal structure maps of $F$. Moreover, given two horizontally composable maps of $F$-decorated cospans $\alpha$ and $\beta$ 
\[
\begin{tikzpicture}[scale=1.5]
\node (A) at (0,0.5) {$a$};
\node (A') at (0,-0.5) {$a'$};
\node (B) at (1,0.5) {$m$};
\node (C) at (2,0.5) {$b$};
\node (C') at (2,-0.5) {$b'$};
\node (D) at (1,-0.5) {$m'$};
\node (E) at (3,0.5) {$x \in F(m)$};
\node (F) at (3,-0.5) {$x' \in F(m')$};
\node (G) at (4,0.5) {$b$};
\node (H) at (5,0.5) {$n$};
\node (I) at (6,0.5) {$c$};
\node (G') at (4,-0.5) {$b'$};
\node (H') at (5,-0.5) {$n'$};
\node (I') at (6,-0.5) {$c'$};
\node (J) at (7,0.5) {$y \in F(n)$};
\node (K) at (7,-0.5) {$y' \in F(n')$};
\node (L) at (1,-1) {$\tau_\alpha \maps F(h_1)(x) \to x'$};
\node (M) at (5,-1) {$\tau_\beta \maps F(h_2)(y) \to y'$};
\path[->,font=\scriptsize,>=angle 90]
(A) edge node[above]{$i_1$} (B)
(C) edge node[above]{$o_1$} (B)
(A) edge node[left]{$f$} (A')
(C) edge node[right]{$g$} (C')
(A') edge node[above] {$i_1'$} (D)
(C') edge node[above] {$o_1'$} (D)
(B) edge node [left] {$h_1$} (D)
(G) edge node [above] {$i_2$} (H)
(G) edge node [left] {$g$} (G')
(H) edge node [left] {$h_2$} (H')
(G') edge node [above] {$i_2'$} (H')
(I) edge node [above] {$o_2$} (H)
(I) edge node [right] {$k$} (I')
(I') edge node [above] {$o_2'$} (H');
\end{tikzpicture}
\]
their composite $\beta\odot \alpha$ is horizontal cospan composition in $\A$
\begin{equation}\label{eq:horizontalcompo}
\begin{tikzpicture}[scale=1.5,baseline=(current bounding box.center)]
\node (A) at (0,0.5) {$a$};
\node (A') at (0,-0.5) {$a'$};
\node (B) at (1.5,0.5) {$m+_{b} n$};
\node (C) at (3,0.5) {$c$};
\node (C') at (3,-0.5) {$c'$};
\node (D) at (1.5,-0.5) {$m'+_{b'}n'$};
\node (E) at (4.5,0.5) {$x \odot y \in F(m+_b n)$};
\node (F) at (4.5,-0.5) {$x' \odot y' \in F(m' +_{b'} n')$};
\path[->,font=\scriptsize,>=angle 90]
(A) edge node[above]{} (B)
(C) edge node[above]{} (B)
(A) edge node[left]{$f$} (A')
(C) edge node[right]{$k$} (C')
(A') edge node [above]{} (D)
(C') edge node [above]{} (D)
(B) edge node [left] {$h_1 +_g h_2$} (D);
\end{tikzpicture}
\end{equation}
together with the decoration morphism $\tau_{\beta \odot \alpha} \colon F(h_1 +_g h_2)(y \odot x) \to (y' \odot x')$ given by the diagram:
\begin{equation}\label{eq:decohorizontalcompo}
\begin{tikzpicture}[scale=1.7,baseline=(current bounding box.center)]
\node (A) at (4.75,0) {$\scriptstyle\Downarrow \tau_\alpha \times \tau_\beta$};
\node (D) at (3,0) {$\one \cong \one \times \one$};
\node (E) at (5.5,0.5) {$F(m) \times F(n)$};
\node () at (6.5,0) {$\cong$};
\node () at (8.5,0) {$\cong$};
\node (E') at (5.5,-0.5) {$F(m') \times F(n')$};
\node (B) at (7.5,0.5) {$F(m+n)$};
\node (B') at (7.5,-0.5) {$F(m' + n')$};
\node (C) at (9.5,0.5) {$F(m+_{b} n)$};
\node (C') at (9.5,-0.5) {$F(m' +_{b'} n')$};
\path[->,font=\scriptsize,>=angle 90]
(E) edge node [above] {$\phi_{m,n}$} (B)
(E') edge node [above] {$\phi_{m',n'}$} (B')
(B) edge node [above] {$F(\psi)$} (C)
(B') edge node [above] {$F(\psi)$} (C')
(C) edge node  [fill=white] {$F(h_1 +_g h_2)$} (C')
(B) edge node [fill=white] {$F(h_1 + h_2)$} (B')
(D) edge node [above] {$x \times y$} (E)
(D) edge node [below] {$x' \times y'$} (E')
(E) edge node [fill=white] {$F(h_1) \times F(h_2)$} (E');
\end{tikzpicture}
\end{equation}
where the middle isomorphism is due to pseudonaturality of $\phi$ and the right-hand side isomorphism is due to pseudofunctoriality of $F$.

The associator for horizontal composition
%$\alpha \colon \odot (\odot \times 1) \Rightarrow \odot (1 \times \odot)$ 
is formed as follows: for three composable horizontal 1-cells $M_1=(a\to m_1\leftarrow b,x_1\in F(m_1))$, $M_2=(b\to m_2\leftarrow c,x_2\in F(m_2))$ and $M_3=(c\to m_3\leftarrow d,x_3\in F(m_3)),$
%\[
%\begin{tikzpicture}[scale=1.5]
%\node (A) at (0,0) {$M_1=a$};
%\node (B) at (1,0) {$m_1$};
%\node (C) at (2,0) {$b$};
%\node (D) at (1,-0.5) {$x_1 \in F(m_1)$};
%\node (E) at (3,0) {$M_2=b$};
%\node (F) at (4,0) {$m_2$};
%\node (G) at (5,0) {$c$};
%\node (H) at (4,-0.5) {$x_2 \in F(m_2)$};
%\node (I) at (6,0) {$M_3=c$};
%\node (J) at (7,0) {$m_3$};
%\node (K) at (8,0) {$d$};
%\node (L) at (7,-0.5) {$x_3 \in F(m_3)$};
%\path[->,font=\scriptsize,>=angle 90]
%(A) edge node[above]{$i$} (B)
%(C) edge node[above]{$o$} (B)
%(E) edge node[above]{$i'$} (F)
%(G) edge node[above]{$o'$} (F)
%(I) edge node[above]{$i''$} (J)
%(K) edge node[above]{$o''$} (J);
%\end{tikzpicture}
%\]
the isomorphism $(M_3\odot M_2)\odot M_1\cong M_3\odot (M_2\odot M_1)$ is the isomorphism of cospans 
\[
\begin{tikzpicture}[scale=1.5]
\node (A) at (0,0.5) {$a$};
\node (A') at (0,-0.5) {$a$};
\node (B) at (1.5,0.5) {$(m_1+_{b} m_2)+_{c} m_3$};
\node (C) at (3,0.5) {$d$};
\node (C') at (3,-0.5) {$d$};
\node (D) at (1.5,-0.5) {$m_1+_{b}(m_2 +_{c} m_3)$};
\node (E) at (5.5,0.5) {$(x_3 \odot x_2) \odot x_1 \in F((m_1+_{b} m_2)+_{c} m_3)$};
\node (F) at (5.5,-0.5) {$x_3 \odot (x_2 \odot x_1) \in F(m_1+_{b} (m_2 +_{c} m_3))$};
\path[->,font=\scriptsize,>=angle 90]
(A) edge node[above]{$$} (B)
(C) edge node[above]{$$} (B)
(A) edge node[left]{$1_a$} (A')
(C) edge node[right]{$1_d$} (C')
(A') edge node {$$} (D)
(C') edge node {$$} (D)
(B) edge node [left] {$\stackrel{\kappa}{\cong}$} (D);
\end{tikzpicture}
\]
together with an isomorphism $F(\kappa)((x_3 \odot x_2) \odot x_1) \cong (x_3 \odot (x_2 \odot x_1))$ between the decorations 
%\begin{align*}
%\scalebox{.8}{$\one\xrightarrow{x_1\times x_2}F(m_1)\times F(m_2)\xrightarrow{\phi}F(m_1+m_2)\to F(m_1+_bm_2)\xrightarrow{1\times x_3}F(m_1+_bm_2)\times %F(m_3)\xrightarrow{\phi}F((m_1+_bm_2)+m_3)$}&\scalebox{.8}{$\to F((m_1+_bm_2)+_cm_3)$} \\
%&\scalebox{.8}{$\stackrel{F(\alpha)}{\cong} F(m_1+_b(m_2+_cm_3))$} \\
%\scalebox{.8}{$\one\xrightarrow{x_2\times x_3}F(m_2)\times F(m_3)\xrightarrow{\phi}F(m_2+m_3)\to F(m_2+_cm_3)\xrightarrow{x_1\times1}F(m_1)\times %F(m_2+_cm_3)\xrightarrow{\phi}F(m_1\times(m_2+_cm_3))$}&\scalebox{.8}{$\to F(m_1+_b(m_2+_cm_3))$}
%\end{align*}
\begin{equation}\label{eq:decoiso}
\begin{tikzpicture}[scale=1.7,baseline=(current bounding box.center)]
\node (A) at (3.75,0) {$\scriptstyle{\cong}$};
\node (D) at (3,0) {$\scriptstyle{\one}$};
\node (E) at (6.25,0.5) {$\scriptstyle{F(m_1 + m_2) \times F(m_3)}$};
\node () at (6,0) {$\scriptstyle\stackrel{\cref{eq:omega}}{\cong}$};
\node () at (9.5,0) {$\scriptstyle\cong$};
%\node () at (8.75,0) {$\cong$};
\node (E') at (6.25,-0.5) {$\scriptstyle{F(m_1) \times F(m_2 + m_3)}$};
\node (B) at (8,0.5) {$\scriptstyle{F((m_1 + m_2) + m_3)}$};
\node (B') at (8,-0.5) {$\scriptstyle{F(m_1 + (m_2 + m_3))}$};
\node (C) at (9.5,0.5) {$\scriptstyle{F((m_1+_b m_2)+m_3)}$};
\node (C') at (9.5,-0.5) {$\scriptstyle{F(m_1 + (m_2+_c m_3))}$};
\node (F) at (11,0.5) {$\scriptstyle{F((m_1+_b m_2)+_c m_3)}$};
\node (F') at (11,-0.5) {$\scriptstyle{F(m_1+_b(m_2+_c m_3))}$};
\node (G) at (4.25,0.5) {$\scriptstyle{(F(m_1) \times F(m_2)) \times F(m_3)}$};
\node (G') at (4.25,-0.5) {$\scriptstyle{F(m_1) \times (F(m_2) \times F(m_3))}$};
\node (M) at (3.25,0.3) {$\scriptstyle{(x_1 \times x_2) \times x_3}$};
\node (M') at (3.25,-0.3) {$\scriptstyle{x_1 \times (x_2 \times x_3)}$};
\path[->,font=\scriptsize,>=angle 90]
(B) edge node [above] {} (B')
(C) edge node [above] {} (F)
(C') edge node [above] {} (F')
(F) edge node [right] {$F(\kappa)$} (F')
(E) edge node [above] {$\phi_{m_1+m_2,m_3}$} (B)
(E') edge node [above] {$\phi_{m_1,m_2+m_3}$} (B')
(B) edge node [above] {} (C)
(B') edge node [above] {} (C')
%(C) edge node  [fill=white] {$F(h_1 +_g h_2)$} (C')
%(B) edge node [fill=white] {$F(h_1 + h_2)$} (B')
(G) edge node [above] {$\phi_{m_1,m_2} {\times} 1$} (E)
(G') edge node [above] {$1 {\times} \phi_{m_2,m_3}$} (E')
(G) edge node [fill=white] {$a$} (G')
(D) edge node [left,above] {$$} (G)
(D) edge node [below] {$$} (G');
%(E) edge node [fill=white] {$F(h_1) \times F(h_2)$} (E');
\end{tikzpicture}
\end{equation}
%%% Christina: I erased tau, and also Fpsi because they were not accurate and those arrows are OK to understand - the universal maps from a 'smaller' colimit to a 'bigger' one. If we kept Fpsi's we should write them like F(psi+1) for the first arrow at the top, and that would make the whole thing a bit heavier.
where $a$ is the associator of $\Cat$ and $\kappa$ is the canonical isomorphism between two colimits of the same diagram in $\A$. Similarly the right and left unitors can be constructed, and all data satisfy the axioms of a double category---essentially breaking down to those of the cospan double category of $\A$ and axioms of the lax monoidal pseudofunctor $F$. For full details of these arguments, see \cite[Theorem~4.1.1]{CourserThesis}.
\end{proof}

The following result establishes that under the same assumptions as \cref{thm:decorated_cospans}, the double category of decorated cospans is monoidal; it becomes symmetric when the lax monoidal pseudofunctor is.  Monoidal and symmetric monoidal double categories are recalled in \cref{defn:monoidal_double_category,defn:symmetric_monoidal_double_category}.

\begin{thm}\label{DC}
Let $\A$ be a category with finite colimits and let $(F,\phi,\phi_0)\maps (\A,+) \to (\Cat,\times)$ a symmetric lax monoidal pseudofunctor. Then the double category $F\lCsp$ is symmetric monoidal, where the tensor product
\begin{itemize}
\item of two objects $a$ and $b$ is their coproduct $a+b$ in $\A$,
\item of two vertical 1-morphisms $f \maps a \to b$ and $f' \maps a' \to b'$ is $f+f' \maps a+a' \to b+b'$ in $\A$,
\item of two horizontal 1-cells $(a_1\to m_1\leftarrow b_1,x_1\in F(m_1))$ and $(a_2\to m_2\leftarrow b_2,x_2\in F(m_2))$ is:
\begin{equation}\label{eq:tensordecoration}
\begin{tikzcd}
& m_1+m_2 & \\
a_1+a_2\ar[ur,"i_1+i_2"] && b_1+b_2,\ar[ul,"o_1+o_2"']
\end{tikzcd}\qquad x_1\oplus x_2:=\one \xrightarrow{x_1\times x_2}F(m_1)\times F(m_2)\xrightarrow{\phi_{m_1,m_2}}F(m_1+m_2)
\end{equation}
%$M_1$ and $M_2$:
%\[
%\begin{tikzpicture}[scale=1.5]
%\node (A) at (0,0) {$a_1$};
%\node (B) at (1,0) {$m_1$};
%\node (C) at (2,0) {$b_1$};
%\node (D) at (1,-0.5) {$x_1 \in F(m_1)$};
%\node (E) at (3,0) {$a_2$};
%\node (F) at (4,0) {$m_2$};
%\node (G) at (5,0) {$b_2$};
%\node (H) at (4,-0.5) {$x_2 \in F(m_2)$};
%\path[->,font=\scriptsize,>=angle 90]
%(A) edge node[above]{$i_1$} (B)
%(C) edge node[above]{$o_1$} (B)
%(E) edge node[above]{$i_2$} (F)
%(G) edge node[above]{$o_2$} (F);
%\end{tikzpicture}
%\]
%is the horizontal 1-cell $M_1 \otimes M_2$:
%\[
%\begin{tikzpicture}[scale=1.5]
%\node (A) at (0,0) {$a_1+a_2$};
%\node (B) at (1.5,0) {$m_1+m_2$};
%\node (C) at (3,0) {$b_1+b_2$};
%\node (D) at (4.55,0) {$x_1+x_2 \in F(m_1+m_2)$};
%\path[->,font=\scriptsize,>=angle 90]
%(A) edge node[above]{$i_1+i_2$} (B)
%(C) edge node[above]{$o_1+o_2$} (B);
%\end{tikzpicture}
%\]
%\begin{equation}
%x_1+x_2 \mapseqq 1 \xrightarrow{\lambda^{-1}} 1 \times 1 \xrightarrow{x_1 \times x_2} F(m_1) \times F(m_2) \xrightarrow{\phi_{m_1,m_2}} F(m_1+m_2)
%\end{equation}

\item of two 2-morphisms $\alpha$ and $\beta$ is:
\[
\begin{tikzpicture}[scale=1.2]
\node (A) at (0,0.5) {$a$};
\node (A') at (0,-0.5) {$a'$};
\node (B) at (1,0.5) {$m$};
\node (C) at (2,0.5) {$b$};
\node (C') at (2,-0.5) {$b'$};
\node (D) at (1,-0.5) {$m'$};
%\node (E) at (2.7,0.5) {$\scriptstyle x \in F(m)$};
\node () at (2.7,0) {$\otimes$};
\node () at (6.2,0) {$=$};
%\node (F) at (2.7,-0.5) {$\scriptstyle x' \in F(m')$};
\node (G) at (3.5,0.5) {$c$};
\node (H) at (4.5,0.5) {$n$};
\node (I) at (5.5,0.5) {$d$};
\node (G') at (3.5,-0.5) {$c'$};
\node (H') at (4.5,-0.5) {$n'$};
\node (I') at (5.5,-0.5) {$d'$};
%\node (J) at (6.7,0.5) {$\scriptstyle y \in F(n)$};
%\node (K) at (6.7,-0.5) {$\scriptstyle y' \in F(n')$};
\node (L) at (1,-1.2) {$\scriptstyle\tau_\alpha \maps F(h)(x) \to x'\textrm{ in } F(m')$};
\node (M) at (4.5,-1.2) {$\scriptstyle\tau_\beta \maps F(h')(y) \to y'\textrm{ in } F(n')$};
\path[->,font=\scriptsize,>=angle 90]
(A) edge node[above]{$i_1$} (B)
(C) edge node[above]{$o_1$} (B)
(A) edge node[left]{$f$} (A')
(C) edge node[right]{$g$} (C')
(A') edge node[below] {$i_1'$} (D)
(C') edge node[below] {$o_1'$} (D)
(B) edge node [left] {$h$} (D)
(G) edge node [above] {$i_2$} (H)
(G) edge node [left] {$f'$} (G')
(H) edge node [left] {$h'$} (H')
(G') edge node [below] {$i_2'$} (H')
(I) edge node [above] {$o_2$} (H)
(I) edge node [right] {$g'$} (I')
(I') edge node [below] {$o_2'$} (H');
\end{tikzpicture}
\begin{tikzpicture}[scale=1.1]
\node (A) at (0,0.5) {$a+c$};
\node (A') at (0,-0.5) {$a'+c'$};
\node (B) at (2,0.5) {$m+n$};
\node (C) at (4,0.5) {$b+d$};
\node (C') at (4,-0.5) {$b'+d'$};
\node (D) at (2,-0.5) {$m'+n'$};
%\node (E) at (5.2,0.5) {$\scriptstyle \phi_{m,n}(x,y)$};
%\node (F) at (5.2,-0.5) {$\scriptstyle \phi_{m',n'}(x',y')$};
\node () at (2,-1.3) {$\scriptstyle\tau_{\alpha\ot\beta}\maps F(h+h')(\phi_{m',n'}(x',y'))\to \phi_{m,n}(x,y)\textrm{ in }F(m'+n')$};
\path[->,font=\scriptsize,>=angle 90]
(A) edge node[above]{$i_1+i_2$} (B)
(C) edge node[above]{$o_1+o_2$} (B)
(A) edge node[left]{$f+f'$} (A')
(C) edge node[right]{$g+g'$} (C')
(A') edge node [below]{$i_1'+i_2'$} (D)
(C') edge node [below]{$o_1'+o_2'$} (D)
(B) edge node [left] {$h+h'$} (D);
\end{tikzpicture}
\]
with decoration morphism $\tau_{\alpha\otimes\beta}$ given by the following diagram:
\begin{displaymath}
 \begin{tikzcd}[column sep=.5in, row sep=.1in]
& F(m)\times F(n)\ar[r,"\phi_{m,n}"]\ar[dd,"F(h)\times F(h')"description]\ar[ddr,phantom,"\stackrel{\cref{eq:pseudonaturality}}{\cong}"] & F(m+n)\ar[dd,"F(h+h')"] \\
\one \ar[ur,"x\times y"]\ar[dr,"x'\times y'"']\ar[r,phantom,"\scriptstyle\Downarrow\tau_\alpha\times\tau_\beta"] & \phantom{A} && \\
& F(m')\times F(n')\ar[r,"\phi_{m',n'}"'] & F(m'+n')
 \end{tikzcd}
\end{displaymath}
\end{itemize}
\end{thm}

\begin{proof}
%We must show that both the category of objects $F\lCsp_0$ and the category of arrows $F\lCsp_1$ are symmetric monoidal categories, show that the monoidal unit of $F\lCsp_0$ is preserved by the unit functor $U$, show that the source and target functors $S,T \colon F\lCsp_1 \to F\lCsp_0$ are strict symmetric monoidal functors which preserve the associativity and unit constraints, define the globular isomorphisms $\chi$ and $\mu$ that relate how composition and units interact with the tensor product, and check that the required axioms are satisfied.
We first show that the categories of objects and arrows are symmetric monoidal, and that the source and target functors are symmetric strict monoidal.
Indeed, the category of objects $F\lCsp_0=\A$ is symmetric monoidal under (chosen) binary coproducts in $\A$. The category of arrows
$F\lCsp_1$ is also symmetric monoidal, with tensor product as in \cref{eq:tensordecoration} and monoidal unit
\begin{comment}
The category of arrows $F\lCsp_1$ has:
\begin{enumerate}
\item{objects as $F$-decorated cospans, which are pairs:
\[
\begin{tikzpicture}[scale=1.5]
\node (A) at (0,0) {$(a$};
\node (B) at (1,0) {$m$};
\node (C) at (2,0) {$b,$};
\node (D) at (2.6,0) {$x \in F(m))$};
\path[->,font=\scriptsize,>=angle 90]
(A) edge node[above]{$i$} (B)
(C) edge node[above]{$o$} (B);
\end{tikzpicture}
\]
and}
\item{morphisms as maps of $F$-decorated cospans, which are cospans in $\A$
\[
\begin{tikzpicture}[scale=1.5]
\node (A) at (0,0.5) {$a$};
\node (A') at (0,-0.5) {$a'$};
\node (B) at (1,0.5) {$m$};
\node (C) at (2,0.5) {$b$};
\node (C') at (2,-0.5) {$b'$};
\node (D) at (1,-0.5) {$m'$};
\node (E) at (3,0.5) {$x \in F(m)$};
\node (F) at (3,-0.5) {$x' \in F(m')$};
\path[->,font=\scriptsize,>=angle 90]
(A) edge node[above]{$i$} (B)
(C) edge node[above]{$o$} (B)
(A) edge node[left]{$f$} (A')
(C) edge node[right]{$g$} (C')
(A') edge node [above]{$i'$} (D)
(C') edge node [above]{$o'$} (D)
(B) edge node [left] {$h$} (D);
\end{tikzpicture}
\]
together with a decoration morphism $\tau \maps F(h)(x) \to x'$.
}
\end{enumerate}
First we define the tensor product and monoidal unit of $F\lCsp_1$. Given two objects $M_1$ and $M_2$ of $F\lCsp_1$:
\[
\begin{tikzpicture}[scale=1.5]
\node (A) at (0,0) {$a_1$};
\node (B) at (1,0) {$m_1$};
\node (C) at (2,0) {$b_1$};
\node (D) at (1,-0.5) {$x_1 \in F(m_1)$};
\node (E) at (3,0) {$a_2$};
\node (F) at (4,0) {$m_2$};
\node (G) at (5,0) {$b_2$};
\node (H) at (4,-0.5) {$x_2 \in F(m_2)$};
\path[->,font=\scriptsize,>=angle 90]
(A) edge node[above]{$i_1$} (B)
(C) edge node[above]{$o_1$} (B)
(E) edge node[above]{$i_2$} (F)
(G) edge node[above]{$o_2$} (F);
\end{tikzpicture}
\]
their tensor product $M_1 \otimes M_2$ is given by taking the coproducts of the cospans in $\A$
\[
\begin{tikzpicture}[scale=1.5]
\node (A) at (0,0) {$a_1+a_2$};
\node (B) at (1.5,0) {$m_1+m_2$};
\node (C) at (3,0) {$b_1+b_2$};
\node (D) at (4.55,0) {$x_1+x_2 \in F(m_1+m_2)$};
\path[->,font=\scriptsize,>=angle 90]
(A) edge node[above]{$i_1+i_2$} (B)
(C) edge node[above]{$o_1+o_2$} (B);
\end{tikzpicture}
\]
where the decoration on the apex is obtained using the natural transformation of the symmetric lax monoidal pseudofunctor $F$: $$x_1+x_2 \mapseqq 1 \xrightarrow{\lambda^{-1}} 1 \times 1 \xrightarrow{x_1 \times x_2} F(m_1) \times F(m_2) \xrightarrow{\phi_{m_1,m_2}} F(m_1+m_2).$$
The monoidal unit $0_{F\lCsp_1}$ is given by:
\[
\begin{tikzpicture}[scale=1.5]
\node (A) at (0,0) {$0$};
\node (B) at (1,0) {$0$};
\node (C) at (2,0) {$0,$};
\node (D) at (3,0) {\bot_0 = \phi \in F(0)$};
\path[->,font=\scriptsize,>=angle 90]
(A) edge node[above]{$!$} (B)
(C) edge node[above]{$!$} (B);
\end{tikzpicture}
\]
\end{comment}
\begin{displaymath}
I_1=\left(0\xrightarrow{!}0\xleftarrow{!}0, \one \xrightarrow{\phi_0}F(0)\right)
\end{displaymath}
where $0$ is the initial object in $\A$. Notice that this unit $I_1$ is (isomorphic to) $U_0$ as required. The associator is formed as follows: for three objects
$M_1=(a_1\to m_1\leftarrow b_1,x_1\in F(m_1))$, $M_2=(a_2\to m_2\leftarrow b_2,x_2\in F(m_2))$ and $M_3=(a_3\to m_3\leftarrow b_3,x_3\in F(m_3),$
%\[
%\begin{tikzpicture}[scale=1.5]
%\node (A) at (0,0) {$a_1$};
%\node (B) at (1,0) {$m_1$};
%\node (C) at (2,0) {$b_1$};
%\node (D) at (1,-0.5) {$x_1 \in F(m_1)$};
%\node (E) at (3,0) {$a_2$};
%\node (F) at (4,0) {$m_2$};
%\node (G) at (5,0) {$b_2$};
%\node (H) at (4,-0.5) {$x_2 \in F(m_2)$};
%\node (I) at (6,0) {$a_3$};
%\node (J) at (7,0) {$m_3$};
%\node (K) at (8,0) {$b_3$};
%\node (L) at (7,-0.5) {$x_3 \in F(m_3)$};
%\path[->,font=\scriptsize,>=angle 90]
%(A) edge node[above]{$i_1$} (B)
%(C) edge node[above]{$o_1$} (B)
%(E) edge node[above]{$i_2$} (F)
%(G) edge node[above]{$o_2$} (F)
%(I) edge node[above]{$i_3$} (J)
%(K) edge node[above]{$o_3$} (J);
%\end{tikzpicture}
%\]
the isomorphism $(M_1\ot M_2)\ot M_3\cong M_1\ot(M_2\ot M_3)$ is the isomorphism of cospans
\[
\begin{tikzpicture}[scale=1.5]
\node (A) at (0,0.5) {$(a_1+a_2)+a_3$};
\node (A') at (0,-0.5) {$a_1+(a_2+a_3)$};
\node (B) at (2.25,0.5) {$(m_1+m_2)+m_3$};
\node (C) at (4.5,0.5) {$(b_1+b_2)+b_3$};
\node (C') at (4.5,-0.5) {$b_1+(b_2+b_3)$};
\node (D) at (2.25,-0.5) {$m_1+(m_2+m_3)$};
\node (E) at (7,0.5) {$(x_1\oplus x_2)\oplus x_3 \in F((m_1{+}m_2){+}m_3)$};
%{$\scriptstyle\phi_{m_1{+}m_2,m_3}(\phi_{m_1,m_2}(x_1,x_2),x_3) \in F((m_1{+}m_2){+}m_3)$};
\node (F) at (7,-0.5) {$x_1\oplus (x_2\oplus x_3)\in F(m_1{+}(m_2{+}m_3))$};
%{$\scriptstyle\phi_{m_1,m_2{+}m_3}(x_1,\phi_{m_2,m_3}(x_2,x_3))  \in F(m_1{+}(m_2{+}m_3))$};
\path[->,font=\scriptsize,>=angle 90]
(A) edge node[above]{$(i_1{+}i_2){+}i_3$} (B)
(C) edge node[above]{$(o_1{+}o_2){+}o_3$} (B)
(A) edge node[left]{$\stackrel{\alpha}{\cong}$} (A')
(C) edge node[right]{$\stackrel{\alpha}{\cong}$} (C')
(A') edge node [below]{$i_1{+}(i_2{+}i_3)$} (D)
(C') edge node [below]{$o_1{+}(o_2{+}o_3)$} (D)
(B) edge node [left] {$\stackrel{\alpha}{\cong}$} (D);
\end{tikzpicture}
\]
%%% Christina: I earlier dropped notation x+y for decorations because + is already overused and I find it potentially misleading for the reader. I hope the new symbol is OK?
together with the decoration isomorphism between $F(\alpha)((x_1\oplus x_2)\oplus x_3)\stackrel{\cref{eq:tensordecoration}}{=}F(\alpha)(\phi_{m_1{+}m_2,m_3}(\phi_{m_1,m_2}(x_1,x_2),x_3))$ and $x_1\oplus(x_2\oplus x_3)=\phi_{m_1,m_2{+}m_3}(x_1,\phi_{m_2,m_3}(x_2,x_3))$ formed in the same way as the left two-piece part of \cref{eq:decoiso}. 
\begin{comment}
tensoring the first two and then the third results in $(M_1 \otimes M_2) \otimes M_3$:
\[
\begin{tikzpicture}[scale=1.5]
\node (A) at (0,0) {$(a_1+a_2)+a_3$};
\node (B) at (2.5,0){$(m_1+m_2)+m_3$};
\node (C) at (5,0) {$(b_1+b_2)+b_3$};
\node (D) at (2.5,-0.5) {$(x_1+x_2)+x_3 \in F((m_1+m_2)+m_3)$};
\path[->,font=\scriptsize,>=angle 90]
(A) edge node[above]{$(i_1+i_2)+i_3$} (B)
(C) edge node[above]{$(o_1+o_2)+o_3$} (B);
\end{tikzpicture}
\]
where $(x_1+x_2)+x_3 \maps 1 \to F((m_1+m_2)+m_3)$ is the composite
\[  1 \xrightarrow{(x_1 \times x_2) \times x_3} (F(m_1) \times F(m_2)) \times F(m_3) \xrightarrow{\phi_{m_1,m_2} \times 1} F(m_1+m_2) \times F(m_3) \xrightarrow{\phi_{m_1+m_2,m_3}} F((m_1+m_2)+m_3).\]
The other parenthesization $M_1 \otimes (M_2 \otimes M_3)$ is given by:
\[
\begin{tikzpicture}[scale=1.5]
\node (A) at (0,0) {$a_1+(a_2+a_3)$};
\node (B) at (2.5,0) {$m_1+(m_2+m_3)$};
\node (C) at (5,0) {$b_1+(b_2+b_3)$};
\node (D) at (2.5,-0.5) {$x_1+(x_2+x_3) \in F(m_1+(m_2+m_3))$};
\path[->,font=\scriptsize,>=angle 90]
(A) edge node[above]{$i_1+(i_2+i_3)$} (B)
(C) edge node[above]{$o_1+(o_2+o_3)$} (B);
\end{tikzpicture}
\]
where $x_1+(x_2+x_3) \maps 1 \to F(m_1+(m_2+m_3))$ is the composite
\[  1 \xrightarrow{x_1 \times (x_2 \times x_3)} F(m_1) \times (F(m_2) \times F(m_3)) \xrightarrow{1 \times \phi_{m_2,m_3}} F(m_1) \times F(m_2+m_3) \xrightarrow{\phi_{m_1,m_2+m_3}} F(m_1+(m_2+m_3)).\]
Denoting the associator of $(\A,+)$ by $\alpha$, the associator of $F\lCsp_1$ is then a map of cospans in $\A$ from $(M_1 \otimes M_2) \otimes M_3$ to $M_1 \otimes (M_2 \otimes M_3)$ given by:
\[
\begin{tikzpicture}[scale=1.5]
\node (A) at (0,0.5) {$(a_1+a_2)+a_3$};
\node (A') at (0,-0.5) {$a_1+(a_2+a_3)$};
\node (B) at (2.25,0.5) {$(m_1+m_2)+m_3$};
\node (C) at (4.5,0.5) {$(b_1+b_2)+b_3$};
\node (C') at (4.5,-0.5) {$b_1+(b_2+b_3)$};
\node (D) at (2.25,-0.5) {$m_1+(m_2+m_3)$};
\node (E) at (7,0.5) {$(x_1+x_2)+x_3 \in F((m_1+m_2)+m_3)$};
\node (F) at (7,-0.5) {$x_1+(x_2+x_3) \in F(m_1+(m_2+m_3))$};
\path[->,font=\scriptsize,>=angle 90]
(A) edge node[above]{$(i_1+i_2)+i_3$} (B)
(C) edge node[above]{$(o_1+o_2)+o_3$} (B)
(A) edge node[left]{$\alpha$} (A')
(C) edge node[right]{$\alpha$} (C')
(A') edge node [above]{$i_1+(i_2+i_3)$} (D)
(C') edge node [above]{$o_1+(o_2+o_3)$} (D)
(B) edge node [left] {$\alpha$} (D);
\end{tikzpicture}
\]
together with the decoration isomorphism $\tau_\alpha \maps F(\alpha)((x_1+x_2)+x_3) \to x_1+(x_2+x_3)$.
\end{comment}
For the left and right unitors $I_1\ot M\cong M\cong M\ot I_1$ for any $M=(a\to m\leftarrow b,x\in F(m)$, we have span isomorphisms
\begin{displaymath}
 \begin{tikzcd}[column sep=.3in]
0+a\ar[rr,"1+i"]\ar[d,"\stackrel{\ell}{\cong}"'] && 0+m\ar[d,"\stackrel{\ell}{\cong}"'] && 0+b\ar[ll,"1+o"']\ar[d,"\stackrel{\ell}{\cong}"] &&1\ar[r,"\phi_0\times x"] & F(0)\times F(m)\ar[r,"\phi_{0,m}"] & F(0+m) \\
a\ar[rr,"i"] && m && b\ar[ll,"o"'] && \one \ar[r,"x"] & F(m)\\
a+0\ar[rr,"i+1"']\ar[u,"\stackrel{r}{\cong}"] && m+0\ar[u,"\stackrel{r}{\cong}"] && b+0\ar[ll,"o+1"]\ar[u,"\stackrel{r}{\cong}"']
&&\one \ar[r,"x\times\phi_0"] & F(m)\times F(0)\ar[r,"\phi_{m,0}"] & F(m+0)
 \end{tikzcd}
\end{displaymath}
with isomorphisms between decorations precisely given by \cref{eq:unitality}.
\begin{comment}
For the left unitor, we first compute $0_{F\lCsp_1} \otimes M$:
\[
\begin{tikzpicture}[scale=1.5]
\node (A) at (0,0) {$0$};
\node (B) at (1,0) {$0$};
\node (C) at (2,0) {$0$};
\node (D) at (1,-0.5) {$\bot_0 \in F(0)$};
\node (E) at (3,0) {$a$};
\node (F) at (4,0) {$m$};
\node (G) at (5,0) {$b$};
\node (H) at (4,-0.5) {$x \in F(m)$};
\node (I) at (2.5,0) {$\otimes$};
\path[->,font=\scriptsize,>=angle 90]
(A) edge node[above]{$!$} (B)
(C) edge node[above]{$!$} (B)
(E) edge node[above]{$i$} (F)
(G) edge node[above]{$o$} (F);
\end{tikzpicture}
\]
which results in:
\[
\begin{tikzpicture}[scale=1.5]
\node (A) at (0,0) {$0+a$};
\node (B) at (1,0) {$0+m$};
\node (C) at (2,0) {$0+b$};
\node (D) at (3.5,0) {$\bot_0 + x \in F(0+m)$};
\path[->,font=\scriptsize,>=angle 90]
(A) edge node[above]{$!+i$} (B)
(C) edge node[above]{$!+o$} (B);
\end{tikzpicture}
\]
where $\bot_0 + x \in F(0+m)$ is given by $$1 \xrightarrow{\lambda^{-1}} 1 \times 1 \xrightarrow{\bot_0 \times x} F(0) \times F(m) \xrightarrow{\phi_{0,m}} F(0+m).$$Denoting the left unitor of $(\A,+,0)$ by $\ell$, the left unitor of $F\lCsp_1$ is then an isomorphism in $F\lCsp_1$ given by the following map of cospans in $\A$:
\[
\begin{tikzpicture}[scale=1.5]
\node (A) at (0,0.5) {$0+a$};
\node (A') at (0,-0.5) {$a$};
\node (B) at (1,0.5) {$0+m$};
\node (C) at (2,0.5) {$0+b$};
\node (C') at (2,-0.5) {$b$};
\node (D) at (1,-0.5) {$m$};
\node (E) at (3.5,0.5) {$\bot_0 + x \in F(0+m)$};
\node (F) at (3.5,-0.5) {$x \in F(m)$};
\path[->,font=\scriptsize,>=angle 90]
(A) edge node[above]{$!+i$} (B)
(C) edge node[above]{$!+o$} (B)
(A) edge node[left]{$\ell$} (A')
(C) edge node[right]{$\ell$} (C')
(A') edge node [above]{$i$} (D)
(C') edge node [above]{$o$} (D)
(B) edge node [left] {$\ell$} (D);
\end{tikzpicture}
\]
together with the decoration isomorphism $\tau_{\ell} \maps F(\ell)(\bot_0 + x) \to x$. The right unitor is similar, and one can check that the associator and left and right unitors together satisfy the pentagon and triangle identities of a monoidal category.  See the second author's thesis \cite{CourserThesis} for more details.
\end{comment}
Finally, the symmetric monoidal structure of $F\lCsp_1$ is inherited from that in $\A$ and $F$, with braiding $\beta\maps M_1\ot M_2\to M_2\ot M_1$ the cospan isomorphism
%To finish showing that $F\lCsp_1$ is symmetric monoidal, we next define the braiding of $F\lCsp_1$. Given two objects $M_1$ and $M_2$ of $F\lCsp_1$ and denoting the braiding of $(\A,+,0)$ by $\beta$, the braiding $\beta_{M_1,M_2}$ of $F\lCsp_1$ is then a map of cospans in $\A$ from $M_1 \otimes M_2$ to $M_2 \otimes M_1$:
%\[
%\begin{tikzpicture}[scale=1.5]
%\node (A) at (0,0.5) {$a_1+a_2$};
%\node (A') at (0,-0.5) {$a_2+a_1$};
%\node (B) at (2,0.5) {$m_1+m_2$};
%\node (C) at (4,0.5) {$b_1+b_2$};
%\node (C') at (4,-0.5) {$b_2+b_1$};
%\node (D) at (2,-0.5) {$m_2+m_1$};
%\node (E) at (6,0.5) {$x_1\oplus x_2 \in F(m_1+m_2)$};
%\node (F) at (6,-0.5) {$x_2\oplus x_1 \in F(m_2+m_1)$};
%\path[->,font=\scriptsize,>=angle 90]
%(A) edge node[above]{$i_1+i_2$} (B)
%(C) edge node[above]{$o_1+o_2$} (B)
%(A) edge node[left]{$\stackrel{\beta}{\cong}$} (A')
%(C) edge node[left]{$\stackrel{\beta}{\cong}$} (C')
%(A') edge node [above]{$i_2+i_1$} (D)
%(C') edge node [above]{$o_2+o_1$} (D)
%(B) edge node [right] {$\stackrel{\beta}{\cong}$} (D);
%\end{tikzpicture}
%\]
\begin{equation}\label{eq:braidingFCsp1}
 \begin{tikzcd}[column sep=.25in]
a_1+a_2\ar[rr,"i_1+i_2"]\ar[d,"\stackrel{\beta}{\cong}"'] && m_1+m_2\ar[d,"\stackrel{\beta}{\cong}"'] && b_1+b_2\ar[ll,"o_1+o_2"']\ar[d,"\stackrel{\beta}{\cong}"] &\one\ar[r,"x_1\times x_2"] & F(m_1)\times F(m_2)\ar[r,"\phi_{m_1,m_2}"] & F(m_1+m_2) \\
a_2+a_1\ar[rr,"i_2+i_1"'] && m_2+m_1 && b_2+b_1\ar[ll,"o_2+o_1"] &\one \ar[r,"x_2\times x_1"] & F(m_2)\times F(m_1)\ar[r,"\phi_{m_2,m_1}"] & F(m_2+m_1)
 \end{tikzcd}
\end{equation}
together with the decoration isomorphism 
\begin{equation}\label{eq:decobraiding}
 \begin{tikzcd}
  1\ar[r,"x_1\times x_2"]\ar[dr,"x_2\times x_1"'] & F(m_1)\times F(m_2)\ar[d,"\beta"']\ar[r,"\phi_{m_1,m_2}"]\ar[dr,phantom,"\stackrel{\cref{eq:braidedpseudofun}}{\cong}"] & F(m_1+m_2)\ar[d,"F(\beta)"] \\
  & F(m_2)\times F(m_1)\ar[r,"\phi_{m_2,m_1}"'] & F(m_2+m_1)
 \end{tikzcd}
\end{equation}
%$F(\beta)(x_1+x_2) \to x_2+x_1$. Note that obtaining the decoration isomorphism $\tau_\beta$ makes use of $F$ being symmetric. One can check that the braiding together with the associator satisfy the hexagon equations, and it is clear that $\beta_{M_1,M_2}=\beta_{M_2,M_1}^{-1}$. This completes the description of $F\lCsp_1$ as a symmetric monoidal category.
All axioms for a symmetric monoidal category can be verified, %see \cite[Theorem 4.1.3]{CourserThesis}, 
and it is easy to see that with these structures, the source and target functors $S,T\colon F\lCsp_1\to F\lCsp_0$ are symmetric strict monoidal.

Lastly, according to \cref{defn:monoidal_double_category} we construct two globular 2-isomorphisms $(M_2 \otimes N_2)\odot (M_1 \otimes N_1) \simrightarrow(M_2\odot M_1) \otimes (N_2\odot N_1)$ and $U_{a + b} \simrightarrow U_a \otimes U_b$ that express compatibility of the tensor with horizontal composition and units. For any appropriate four decorated cospans
\begin{gather}\label{eq:4deccospans}
M_1=\left(a\xrightarrow{i_1}m_1\xleftarrow{o_1}b,x_1\in F(m_1)\right),\;\; M_2=\left(b\xrightarrow{i_2}m_2\xleftarrow{o_2}c,x_2\in F(m_2)\right) \\
N_1=\left(a'\xrightarrow{i'_1}n_1\xleftarrow{o'_1}b',y_1\in F(n_1)\right),\;\; N_2=\left(b'\xrightarrow{i'_2}n_2\xleftarrow{o'_2}c',y_2\in F(n_2)\right)\nonumber
\end{gather}
first tensoring $M_1\ot N_1$, $M_2\ot N_2$ by \cref{eq:tensordecoration} and then horizontally composing them by \cref{eq:dcospanscomposition} gives
\begin{equation}\label{eq:bigeq1}
\scalebox{.8}{\begin{tikzcd}[column sep={.6in,between origins},ampersand replacement=\&]
\&\& (m_1+n_1)+_{b{+}b'}(m_2+n_2)\ar[dd,phantom,near start,"\upback"] \&\& \\
\& m_1+n_1\ar[ur] \&\& m_2+n_2\ar[ul] \& \\
a+a'\ar[ur,"{i_1+i'_1}"] \&\& b+b'\ar[ul,"{o_1+o'_1}"']\ar[ur,"{i_2+i'_2}"] \&\& c+c'\ar[ul,"{o_2+o'_2}"']
\end{tikzcd}}\hspace{-.2in}
\scalebox{.8}{\begin{tikzcd}[ampersand replacement=\&,column sep=.5in]
\one \ar[ddrr,dashed,"{(x_2\oplus y_2)\odot(x_1\oplus y_1)}"']\ar[r,"x_1{\times}y_1{\times}x_2{\times}y_2"] \& F(m_1){\times}F(n_1){\times}F(m_2){\times}F(n_2)\ar[r,"\phi_{m_1,n_1}\times\phi_{m_2,n_2}"] \& F(m_1{+}n_1){\times}F(m_2{+}n_2) \ar[d,"\phi_{m_1+n_1,m_2+n_2}"] \\
\&\& F(m_1+n_1+m_2+n_2)\ar[d,"F\psi"] \\
\&\& F((m_1+n_1)+_{b+b'}(m_2+n_2))
\end{tikzcd}}
\end{equation}
If we first horizontally compose $M_2\odot M_1$, $N_2\odot N_1$ and then tensor the result, we obtain
\begin{equation}\label{eq:bigeq2}
  \scalebox{.8}{\begin{tikzcd}[column sep=.3in,ampersand replacement=\&]
 \& (m_1+_b m_2)+(n_1+_{b'}n_2) \& \\
 \hole \\
 a+a'\ar[uur]%,"{u_{m_1}i_1+u_{n_1}i'_1}"] 
 \&\& c+c'\ar[uul]%,"{u_{m_2}o_2+u_{n_2}o'_2}"']  
 \end{tikzcd}}\hspace{-.2in}
 \scalebox{.8}{\begin{tikzcd}[ampersand replacement=\&,column sep=.4in]
\one \ar[ddrr,dashed,"{(x_2\odot x_1)\oplus(y_2\odot y_1)}"']\ar[r,"x_1{\times}x_2{\times}y_1{\times}y_2"] \& F(m_1){\times}F(m_2){\times}F(n_1){\times}F(n_2)\ar[r,"\phi_{m_1,m_2}\times\phi_{n_1,n_2}"] \& F(m_1{+}m_2){\times}F(n_1{+}n_2) \ar[d, "F\psi\times F\psi"] \\
\&\& F(m_1+_bm_2)\times F(n_1+_{b'}n_2)\ar[d,"\phi_{m_1+_bm_2,n_1+_{b'}n_2}"] \\
\&\& F((m_1+_bm_2)+(n_1+_{b'}n_2))
\end{tikzcd}}
\end{equation}
where the legs of the cospan are the sums of the dashed ones of the appropriate diagrams \cref{eq:dcospanscomposition}.
Thus the (globular) cospan isomorphism is the canonical universal map between the colimit of the same diagram obtained in two different ways $\hat{\chi}\colon(m_1+n_1)+_{b+b'}(m_2+n_2)\cong(m_1+_b m_2)+(n_1+_{b'}n_2)$, and the decoration isomorphism can be built in a similar way using coherent isomorphisms as follows: 
\begin{equation}\label{eq:interchangedeco}
 \scalebox{.75}{\begin{tikzcd}[ampersand replacement=\&,column sep=.3in,row sep=.1in]
\& F(m_1)\times F(n_1)\times F(m_2)\times F(n_2)\ar[dd,"1\times\beta\times1"']\ar[ddrr,phantom,"\stackrel{\cref{eq:thetaa}}{\cong}"]\ar[rr,"\phi_{m_1,n_1}\times\phi_{m_2,n_2}"] \&\& F(m_1+n_1)\times F(m_2+n_2)\ar[rr,"\phi_{m_1+n_1,m_2+n_2}"] \&\& F(m_1+n_1+m_2+n_2)\ar[dd,"F\psi"]\ar[ddll,"F(1+\beta+1)"'] \\
\one \ar[ur,"{x_1\times y_1\times x_2\times y_2}"]\ar[dr,"{x_1\times x_2\times y_1\times y_2}"'] \&\&\&\& \\
\& F(m_1)\times F(m_2)\times F(n_1)\times F(n_2)\ar[dd,"\phi_{m_1,m_2}\times\phi_{n_1,n_2}"']\&\& F(m_1+m_2+n_1+n_2)\ar[ddrr,"F(\psi+\psi)"']\ar[dd,phantom,"\stackrel{\cref{eq:pseudonaturality}}{\cong}"] \ar[rr,phantom,"\cong"]\&\& F((m_1+n_1)+_{b+b'}(m_2+n_2))\ar[dd,"F(\hat{\chi})"] \\
\hole \\
\& F(m_1+m_2)\times F(n_1+n_2)\ar[uurr,"\phi_{m_1+m_2,n_1+n_2}"']\ar[rr,"F\psi\times F\psi"'] \&\& F(m_1+_bm_2)\times F(n_1+_{b'}n_2)\ar[rr,"\phi_{m_1+_b m_2,n_1+_{b'}n_2}"'] \&\& F((m_1+_bm_2)+(n_1+_{b'}n_2))
\end{tikzcd}}
\end{equation}
where the left-hand side isomorphism combines pseudoassociativity \cref{eq:omega} and braided monoidal structure \cref{eq:braidedpseudofun} of the pseudofunctor $F$ as seen in detail in the appendix, and the right hand side follows from the universal property of colimits and pseudofunctoriality of $F$.
Similarly, for units $U_{a+b}$ and $U_a\ot U_b$ it is easy to see that as cospans they are both the identity $1_{a+b}:a+b\to a+b\leftarrow a+b:1_{a+b}$, and there is a canonical isomorphism between their trivial decorations \cref{eq:UFCsp}, namely $F(!_{a+b})\phi_0\cong\phi_{a,b}(F(!_a)\times F(!_b))(\phi_0\times\phi_0)$ via the structure isomorphisms for $(\phi,\phi_0)$.
\begin{comment}
, the first of which relates the double functor $\otimes$ with the functor $\odot$ and the second of which relates the double functor $\otimes$ with the functor $U$. 
The latter of these is easy to see: given two objects $a$ and $b$ in $F\lCsp_0=\A$, $\mu$ is given by the identity map of cospans in $\A$:
\[
\begin{tikzpicture}[scale=1.5]
\node (A) at (0,0.5) {$a+b$};
\node (A') at (0,-0.5) {$a+b$};
\node (B) at (2,0.5) {$a+b$};
\node (C) at (4,0.5) {$a+b$};
\node (C') at (4,-0.5) {$a+b$};
\node (D) at (2,-0.5) {$a+b$};
\node (E) at (5.5,0.5) {$\bot_{a+b} \in F(a+b)$};
\node (F) at (5.5,-0.5) {$\bot_a + \bot_b \in F(a+b)$};
\path[->,font=\scriptsize,>=angle 90]
(A) edge node[above]{$1_{a+b}$} (B)
(C) edge node[above]{$1_{a+b}$} (B)
(A) edge node[left]{$1$} (A')
(C) edge node[right]{$1$} (C')
(A') edge node [above]{$1_a + 1_b$} (D)
(C') edge node [above]{$1_a + 1_b$} (D)
(B) edge node [left] {$1$} (D);
\end{tikzpicture}
\]
together with the decoration isomorphism $\tau_{a,b} \maps \bot_{a+b} \xrightarrow{\sim} \bot_a + \bot_b$ given by the unique map between the two initial objects $\bot_{a+b}$ and $\bot_a + \bot_b$ in $F(a+b)$. 
%%%Christina: we do not have any such information at this point! We have not talked about the categories F(a) having finite colimits, and it's not deduced from the assumptions. We will show this is the case much later!
For the former globular 2-morphism $\chi$, we first introduce some notation for objects $M_1,M_2,N_1$ and $N_2$ of $F\lCsp_1$:
\[
\begin{tikzpicture}[scale=1.5]
\node (A) at (0,0) {$a$};
\node (B) at (1,0) {$m_1$};
\node (C) at (2,0) {$b$};
\node (D) at (1,-0.5) {$x_1 \in F(m_1)$};
\node (E) at (3,0) {$b$};
\node (F) at (4,0) {$m_2$};
\node (G) at (5,0) {$c$};
\node (H) at (4,-0.5) {$x_2 \in F(m_2)$};
\node (I) at (0,-1.5) {$a'$};
\node (J) at (1,-1.5) {$n_1$};
\node (K) at (2,-1.5) {$b'$};
\node (L) at (1,-2) {$y_1 \in F(n_1)$};
\node (M) at (3,-1.5) {$b'$};
\node (N) at (4,-1.5) {$n_2$};
\node (O) at (5,-1.5) {$c'$};
\node (P) at (4,-2) {$y_2 \in F(n_2)$};
\path[->,font=\scriptsize,>=angle 90]
(A) edge node[above]{$i_1$} (B)
(C) edge node[above]{$o_1$} (B)
(E) edge node[above]{$i_2$} (F)
(G) edge node[above]{$o_2$} (F)
(I) edge node[above]{$i_1'$} (J)
(K) edge node[above]{$o_1'$} (J)
(M) edge node[above]{$i_2'$} (N)
(O) edge node[above]{$o_2'$} (N);
\end{tikzpicture}
\]
If we first tensor $M_1$ and $N_1$ as well as $M_2$ and $N_2$ and then compose these two, we obtain $(M_1 \otimes N_1) \odot (M_2 \otimes N_2)$, which is given by:
\[
\begin{tikzpicture}[scale=1.5]
\node (A) at (0,0) {$a+a'$};
\node (B) at (2.5,0) {$(m_1+n_1)+_{b+b'}(m_2+n_2)$};
\node (C) at (5,0) {$c+c'$};
\node (D) at (2.5,-0.5) {$(x_1+y_1)\odot(x_2+y_2) \in F((m_1+n_1)+_{b+b'}(m_2+n_2))$};
\path[->,font=\scriptsize,>=angle 90]
(A) edge node[above]{$\psi j (i_1+i_1')$} (B)
(C) edge node[above]{$\psi j (o_2 + o_2')$} (B);
\end{tikzpicture}
\]
where the decoration $(x_1+y_1) \odot (x_2+y_2) \in F((m_1+n_1)+_{b+b'}(m_2+n_2))$ is given by
%$$\scriptscriptstyle{1 \xrightarrow{(\lambda^{-1} \times \lambda^{-1})\lambda^{-1}} (1 \times 1) \times (1 \times 1) \xrightarrow{(x_1 \times y_1) \times (x_2 \times y_2)} (F(m_1) \times F(n_1)) \times (F(m_2) \times F(n_2)) \xrightarrow{\phi_{m_1,n_1} \times \phi_{m_2,n_2}} F(m_1+n_1) \times F(m_2+n_2)} \xrightarrow{\phi_{m_1+n_1,m_2+n_2}} F((m_1+n_1)+(m_2+n_2)) \xrightarrow{F(\psi)} F((m_1+n_1)+_{b+b'}(m_2+n_2))$$
\[ \scriptstyle{1 \xrightarrow{} (F(m_1) \times F(n_1)) \times (F(m_2) \times F(n_2)) \xrightarrow{\phi_{m_1,n_1} \times \phi_{m_2,n_2}} F(m_1+n_1) \times F(m_2+n_2)} \xrightarrow{\phi_{m_1+n_1,m_2+n_2}} F((m_1+n_1)+(m_2+n_2)) \xrightarrow{F(\psi)} F((m_1+n_1)+_{b+b'}(m_2+n_2)) \]
where the first arrow is the composite
\[1 \xrightarrow{\lambda^{-1}} 1 \times 1 \xrightarrow{\lambda^{-1} \times \lambda^{-1}} (1 \times 1) \times (1 \times 1) \xrightarrow{(x_1 \times y_1) \times (x_2 \times y_2)} (F(m_1) \times F(n_1)) \times (F(m_2) \times F(n_2)). \]
%\[
%\begin{tikzpicture}[scale=1.5]
%\node (A) at (0,0) {$1$};
%\node (B) at (0,-1) {$1 \times 1$};
%\node (C) at (0,-2) {$(1 \times 1) \times (1 \times 1)$};
%\node (D) at (0,-3) {$(F(m_1) \times F(n_1)) \times (F(m_2) \times F(n_2))$};
%\node (E) at (0,-4) {$F(m_1+n_1) \times F(m_2+n_2)$};
%\node (F) at (0,-5) {$F((m_1+n_1)+(m_2+n_2))$};
%\node (G) at (0,-6) {$F((m_1+n_1)+_{b+b'}(m_2+n_2))$};
%\path[->,font=\scriptsize,>=angle 90]
%(A) edge node[left]{$\lambda^{-1}$} (B)
%(B) edge node[left]{$\lambda^{-1} \times \lambda^{-1}$} (C)
%(C) edge node[left]{$(x_1 \times y_1) \times (x_2 \times y_2)$} (D)
%(D) edge node[left]{$\phi_{m_1,n_1} \times \phi_{m_2,n_2}$} (E)
%(E) edge node[left]{$\phi_{m_1+n_1,m_2+n_2}$} (F)
%(F) edge node[left]{$F(j_{m_1+n_1,m_2+n_2})$} (G);
%\end{tikzpicture}
%\]
Interchanging the order in which we tensor and compose, we obtain $(M_1 \odot M_2) \otimes (N_1 \odot N_2)$, which is given by:
\[
\begin{tikzpicture}[scale=1.5]
\node (A) at (0,0) {$a+a'$};
\node (B) at (2.5,0) {$(m_1+_{b} m_2) + (n_1 +_{b'} n_2)$};
\node (C) at (5,0) {$c+c'$};
\node (D) at (2.5,-0.5) {$(x_1 \odot x_2) + (y_1 \odot y_2) \in F((m_1+_{b}m_2)+(n_1+_{b'}n_2))$};
\path[->,font=\scriptsize,>=angle 90]
(A) edge node[above]{$(\psi j i_1)+(\psi j i_1')$} (B)
(C) edge node[above]{$(\psi j o_2)+(\psi j o_2')$} (B);
\end{tikzpicture}
\]
where the decoration $(x_1 \odot x_2) + (y_1 \odot y_2) \in F((m_1+_{b}m_2)+(n_1+_{b'}n_2))$ is given by:
%$$\scriptscriptstyle{1 \xrightarrow{(\lambda^{-1} \times \lambda^{-1})\lambda^{-1}} (1 \times 1) \times (1 \times 1) \xrightarrow{(x_1 \times y_1) \times (x_2 \times y_2)} (F(m_1) \times F(n_1)) \times (F(m_2) \times F(n_2)) \xrightarrow{\phi_{m_1,n_1} \times \phi_{m_2,n_2}} F(m_1+n_1) \times F(m_2+n_2) \xrightarrow{F(\psi) \times F(\psi)} F(m_1 +_b m_2) \times F(n_1+_{b'}n_2) \xrightarrow{\phi_{m_1+_b m_2,n_1+_{b'}n_2}} F((m_1+_b m_2)+(n_1+_{b'}n_2))}$$
$$\scriptstyle{1 \xrightarrow{} (F(m_1) \times F(n_1)) \times (F(m_2) \times F(n_2)) \xrightarrow{\phi_{m_1,n_1} \times \phi_{m_2,n_2}} F(m_1+n_1) \times F(m_2+n_2) \xrightarrow{F(\psi) \times F(\psi)} F(m_1 +_b m_2) \times F(n_1+_{b'}n_2) \xrightarrow{\phi_{m_1+_b m_2,n_1+_{b'}n_2}} F((m_1+_b m_2)+(n_1+_{b'}n_2))}$$
with the first arrow being the same as the first arrow in the previous composite.
%\[
%\begin{tikzpicture}[scale=1.5]
%\node (A) at (0,0) {$1$};
%\node (B) at (0,-1) {$1 \times 1$};
%\node (C) at (0,-2) {$(1 \times 1) \times (1 \times 1)$};
%\node (D) at (0,-3) {$(F(m_1) \times F(m_2)) \times (F(n_1) \times F(n_2))$};
%\node (E) at (0,-4) {$F(m_1+m_2) \times F(n_1+n_2)$};
%\node (F) at (0,-5) {$F(m_1+_{b}m_2) \times F(n_1+_{b'}n_2)$};
%\node (G) at (0,-6) {$F((m_1+_{b}m_2)+(n_1+_{b'}n_2))$};
%\path[->,font=\scriptsize,>=angle 90]
%(A) edge node[left]{$\lambda^{-1}$} (B)
%(B) edge node[left]{$\lambda^{-1} \times \lambda^{-1}$} (C)
%(C) edge node[left]{$(x_1 \times x_2) \times (y_1 \times y_2)$} (D)
%(D) edge node[left]{$\phi_{m_1,m_2} \times \phi_{n_1,n_2}$} (E)
%(E) edge node[left]{$F(j_{m_1,m_2}) \times F(j_{n_1,n_2})$} (F)
%(F) edge node[left]{$\phi_{m_1+_{b}m_2,n_1+_{b'}n_2}$} (G);
%\end{tikzpicture}
%\]
The globular 2-morphism $\chi \colon (M_1 \otimes N_1) \odot (M_2 \otimes N_2) \to (M_1 \odot M_2) \otimes (N_1 \odot N_2)$ is then given by the map of cospans in $\A$:
\[
\begin{tikzpicture}[scale=1.5]
\node (A) at (0,0.5) {$a+a'$};
\node (A') at (0,-0.5) {$a+a'$};
\node (B) at (2.5,0.5) {$(m_1+n_1)+_{b+b'}(m_2+n_2)$};
\node (C) at (5,0.5) {$c+c'$};
\node (C') at (5,-0.5) {$c+c'$};
\node (D) at (2.5,-0.5) {$(m_1+_{b}m_2)+(n_1+_{b'}n_2)$};
\node (E) at (7.5,0.5) {$\scriptstyle{(x_1+y_1) \odot (x_2+y_2) \in F((m_1+n_1)+_{b+b'}(m_2+n_2))}$};
\node (F) at (7.5,-0.5) {$\scriptstyle{(x_1 \odot x_2)+(y_1 \odot y_2) \in F((m_1+_{b}m_2)+(n_1+_{b'}n_2))}$};
\path[->,font=\scriptsize,>=angle 90]
(A) edge node[above]{$\psi j(i_1+i_1')$} (B)
(C) edge node[above]{$\psi j(o_2 + o_2')$} (B)
(A) edge node[left]{$1$} (A')
(C) edge node[right]{$1$} (C')
(A') edge node [above]{$(\psi j i_1')+(\psi j i_1)$} (D)
(C') edge node [above]{$(\psi j o_2)+(\psi j o_2')$} (D)
(B) edge node [left] {$\hat{\chi}$} (D);
\end{tikzpicture}
\]
together with decoration isomorphism $\tau_{\chi} \maps F(\hat{\chi})((x_1+y_1)\odot(x_2+y_2)) \to (x_1 \odot x_2)+(y_1 \odot y_2)$
where $\hat{\chi}$ is the same interchange map $\chi$ for the symmetric double category $\lCsp(\A)$ given by the universal map between the colimit of the same diagram obtained in two different ways. This completes the description of the globular 2-morphisms $\chi$ and $\mu$, and the description of all of the structure of the symmetric monoidal double category $F\lCsp$.
\end{comment}

It can be verified with this structure, $F\lCsp$ is indeed a symmetric monoidal double category, namely all coherence laws are satisfied. The details can be found in \cite[Theorem 4.1.3]{CourserThesis}, but in order to give the reader a sense of what these checks entail, we provide an explicit proof of the fact that the braiding is a transformation of double categories in \cref{eq:proofofaxiom}.  \end{proof}

%The remaining properties are coherence laws that make use of these maps $\chi$ and $\mu$, the previously defined symmetric monoidal structure maps $\alpha,\lambda,\rho$ and $\beta$ and the double category structure functors $U$ and $\odot$. We exhibit one of these laws to give the reader a sense of how they are verified, with more details available in the second author's thesis \cite{CourserThesis}.

Our construction gives not only decorated cospan double categories, but  also maps between these: that is, symmetric monoidal double functors, as in \cref{def:doublefun}. Suppose we have two categories $\A,A'$ with finite colimits and two symmetric lax monoidal pseudofunctors $F \maps \A \to \Cat$ and $F' \maps \A' \to \Cat$.    Then we can obtain a map between their decorated cospan double categories, $\lH \maps F\lCsp \to F' \lCsp$, from:
\begin{itemize}
\item a functor $H \maps \A \to \A'$ that preserves finite limits,
\item a symmetric lax monoidal pseudofunctor $(E,\phi,\phi_0) \maps \Cat \to \Cat$, 
\item a natural isomorphism $\theta \maps EF \Rightarrow F'H$: 
\[
\begin{tikzpicture}[scale=1.5]
\node (A) at (0,0) {$\A$};
\node (B) at (1,0) {$\Cat$};
\node (C) at (0,-1) {$\A'$};
\node (D) at (1,-1) {$\Cat$.};
\node (E) at (0.5,-0.5) {$\Downarrow\theta$};
\path[->,font=\scriptsize,>=angle 90]
(A) edge node[above]{$F$} (B)
(A) edge node[left]{$H$} (C)
(B) edge node[right]{$E$} (D)
(C) edge node[above]{$F'$} (D);
\end{tikzpicture}
\]
\end{itemize}
The induced double functor $\lH \colon F\lCsp \to F'\lCsp$ is defined as follows:
\begin{itemize}
\item The image of an object $a \in F\lCsp_0=\A$ is the object $H(a) \in F'\lCsp_0=\A'$.
\item The image of a vertical 1-morphism $f \colon a \to b$ is the vertical 1-morphism $H(f) \colon H(a) \to H(b)$. 
\end{itemize}
In other words, the object component $\lH_0$ of the double functor $\lH$ is the functor $H$.
\begin{itemize}
\item The image of an $F$-decorated cospan $M$ in $F\lCsp$ is the $F'$-decorated cospan $\lH(M)$ in $F'\lCsp$
\begin{align*}
 M=\left(a\xrightarrow{i}m\xleftarrow{o}b,x\in F(m)\right)\mapsto \lH(M)=\left(H(a)\xrightarrow{H(i)}H(m)\xleftarrow{H(o)}H(b),\bar{x}\in F'(H(m))\right)& \textrm{ where } \\
\bar{x}:=\one \xrightarrow{\phi_0} E(\one) \xrightarrow{E(x)} E(F(m)) \xrightarrow{\theta_m} F'(H(m)) &
\end{align*}
%\[
%\begin{tikzpicture}[scale=1.5]
%\node (A) at (0,0) {$a$};
%\node (B) at (1,0) {$m$};
%\node (C) at (2,0) {$b$};
%\node (D) at (1,-0.5) {$x \in F(m)$};
%\node (E) at (2.75,0) {$\mapsto$};
%\node (A') at (3.5,0) {$H(a)$};
%\node (B') at (4.5,0) {$H(m)$};
%\node (C') at (5.5,0) {$H(b)$};
%\node (D') at (4.5,-0.5) {$\theta_m E(x) \phi \in F'(H(m))$};
%\node (F) at (1,0.5) {$M$};
%\node (G) at (4.5,0.5) {$\lH(M)$};
%\path[->,font=\scriptsize,>=angle 90]
%(A) edge node[above]{$i$} (B)
%(C) edge node[above]{$o$} (B)
%(A') edge node[above]{$H(i)$} (B')
%(C') edge node[above]{$H(o)$} (B');
%\end{tikzpicture}
%\]
%where given a decoration $x \colon 1 \to F(m)$ on $m$, we obtain a decoration on $H(m)$ via the composite 
%\[ 1 \xrightarrow{\phi} E(1) \xrightarrow{E(x)} E(F(m)) \xrightarrow{\theta_m} F'(H(m)).\]
\item The image of a map of decorated cospans $\alpha \colon M \to N$ in $F\lCsp$ is the map of $F'$-decorated cospans $\lH(\alpha)$ in $F'\lCsp$
\[
\begin{tikzpicture}[scale=1.5]
\node (A) at (0,0.5) {$a$};
\node (A') at (0,-0.5) {$a'$};
\node (B) at (1,0.5) {$m$};
\node (C) at (2,0.5) {$b$};
\node (C') at (2,-0.5) {$b'$};
\node (D) at (1,-0.5) {$n$};
\node (E) at (3,0.5) {$x \in F(m)$};
\node (F) at (3,-0.5) {$x' \in F(n)$};
%\node (G) at (1,-1) {$\tau \maps F(h)(x) \to x'$};
\node (A'') at (4.75,0.5) {$H(a)$};
\node (A''') at (4.75,-0.5) {$H(a')$};
\node (B'') at (5.75,0.5) {$H(m)$};
\node (C'') at (6.75,0.5) {$H(b)$};
\node (C''') at (6.75,-0.5) {$H(b')$};
\node (D'') at (5.75,-0.5) {$H(n)$};
\node (E'') at (8.5,0.5) {$\theta_m E(x) \phi_0 \in F'(H(m))$};
\node (F'') at (8.5,-0.5) {$\theta_n E(x') \phi_0 \in F'(H(n))$};
%\node (G'') at (5.75,-1) {$E(\tau) \maps F'(H(h))(\theta_m E(x)\phi) \to (\theta_n E(x')\phi)$}; %%%Christina: E(tau) is a bad name since it already means something!
\node (H) at (3.75,0) {$\mapsto$};
\path[->,font=\scriptsize,>=angle 90]
(A) edge node[above]{$i$} (B)
(C) edge node[above]{$o$} (B)
(A) edge node[left]{$f$} (A')
(C) edge node[right]{$g$} (C')
(A') edge node[above] {$i'$} (D)
(C') edge node[above] {$o'$} (D)
(B) edge node [left] {$h$} (D)
(A'') edge node[above]{$H(i)$} (B'')
(C'') edge node[above]{$H(o)$} (B'')
(A'') edge node[left]{$H(f)$} (A''')
(C'') edge node[right]{$H(g)$} (C''')
(A''') edge node[above] {$H(i')$} (D'')
(C''') edge node[above] {$H(o')$} (D'')
(B'') edge node [left] {$H(h)$} (D'');
\end{tikzpicture}
\]
where a decoration morphism $\tau \colon F(h)(x) \to x'$ is mapped to a 
decoration morphism as follows:
\[
\begin{tikzpicture}[scale=1.5]
\node (G) at (-3.5,-0.5) {$1$};
\node (H) at (-2,0) {$F(m)$};
\node (I) at (-2,-1) {$F(n)$};
\node (J) at (-2.5,-0.5) {$\scriptstyle\Downarrow\tau$};
\node (K) at (-1,-0.5) {$\mapsto$};
\node (A) at (-0.25,-0.5) {$1$};
\node (L) at (0.5,-0.5) {$E(1)$};
\node (B) at (2,0) {$E(F(m))$};
\node (D) at (2,-1) {$E(F(n))$};
\node (C) at (1.5,-0.5) {$\scriptstyle\Downarrow E(\tau)$};
\node (E) at (4,0) {$F^\prime(H(m))$};
\node (F) at (4,-1) {$F^\prime(H(n))$};
\path[->,font=\scriptsize,>=angle 90]
(A) edge node [above] {$\phi$} (L)
(G) edge node [above] {$x$} (H)
(G) edge node [below] {$x'$} (I)
(H) edge node [right] {$F(h)$} (I)
(B) edge node [above] {$\theta_m$} (E)
(D) edge node [above] {$\theta_n$} (F)
(E) edge node [right] {$F'(H(h))$} (F)
(L) edge node[above]{$E(x)$} (B)
(L) edge node[below]{$E(x')$} (D)
(B) edge node[right]{$E(F((h))$} (D);
\end{tikzpicture}
\]
\end{itemize}


\begin{thm}
\label{thm:functoriality}
Given two categories $\A$ and $\A'$ with finite colimits, two symmetric lax monoidal pseudofunctors $F \maps \A \to \Cat$ and $F' \maps \A' \to \Cat$, a finite colimit preserving functor $H \maps \A \to \A'$, a symmetric lax monoidal pseudofunctor $E \maps \Cat \to \Cat$ and a natural isomorphism $\theta \maps EF \Rightarrow F' H$ as in the following diagram:
\[
\begin{tikzpicture}[scale=1.5]
\node (A) at (0,0) {$\A$};
\node (B) at (1,0) {$\Cat$};
\node (C) at (0,-1) {$\A'$};
\node (D) at (1,-1) {$\Cat$};
\node (E) at (0.5,-0.5) {$\Downarrow \theta$};
\path[->,font=\scriptsize,>=angle 90]
(A) edge node[above]{$F$} (B)
(A) edge node[left]{$H$} (C)
(B) edge node[right]{$E$} (D)
(C) edge node[above]{$F'$} (D);
\end{tikzpicture}
\] 
the triple $(H,E,\theta)$ induces a symmetric monoidal double functor $\lH \maps F\lCsp \to F'\lCsp$ as defined above.
\end{thm}

\begin{proof}  See \cite[Theorem 4.2.1]{CourserThesis}. \end{proof}

This theorem should generalize to the case when $\theta$ is a pseudonatural equivalence, but the weaker version suffices for our application in \cref{subsec:petrirates}.

\section{Structured versus decorated cospans} \label{EquivDoubleCats}

In \cite{BC}, the first two authors introduce the symmetric monoidal double category of \emph{structured cospans} as a formalism to capture open networks. One of the main goals of this paper is to provide a monoidal double isomorphism between this double category and that of \emph{decorated cospans}, described in detail in \cref{DecCospansDoubleCat}. We first recall the double category of structured cospans.

\begin{thm}\label{thm:SC}
Given categories $\A$ and $\X$ with finite colimits and $L \maps \A \to \X$ a functor preserving finite colimits, there is a symmetric monoidal double category $_L\lCsp(\X)$ in which
\begin{itemize}
\item an object is an object of $\A$,
\item a vertical 1-morphism is a morphism of $\A$,
\item a horizontal 1-cell from $a$ to $b$ is a cospan in $\X$ of the form
\begin{displaymath}
\begin{tikzpicture}[scale=1.5]
\node (A) at (0,0) {$L(a)$};
\node (B) at (1,0) {$x$};
\node (C) at (2,0) {$L(b)$};
\path[->,font=\scriptsize,>=angle 90]
(A) edge node[above]{$i$} (B)
(C) edge node[above]{$o$} (B);
\end{tikzpicture}
\end{displaymath}
\item a 2-morphism is a map of cospans in $\X$ of the form
\begin{displaymath}
\begin{tikzpicture}[scale=1.5]
\node (A) at (0,0) {$L(a)$};
\node (B) at (1,0) {$x$};
\node (C) at (2,0) {$L(b)$};
\node (A') at (0,-1) {$L(a')$};
\node (B') at (1,-1) {$x'$};
\node (C') at (2,-1) {$L(b')$};
\path[->,font=\scriptsize,>=angle 90]
(A) edge node[above]{$i$} (B)
(C) edge node[above]{$o$} (B)
(A') edge node[above]{$i'$} (B')
(C') edge node[above]{$o'$} (B')
(A) edge node [left]{$L(f)$} (A')
(B) edge node [left]{$\alpha$} (B')
(C) edge node [right]{$L(g)$} (C');
\end{tikzpicture}
\end{displaymath}
\end{itemize}
Composition of horizontal 1-cells and 2-cells is done using pushouts in $\X$, and the
symmetric monoidal structure is defined using finite coproducts in $\A$ and $\X$: 
the tensor of horizontal 1-cells is
\begin{equation}\label{eq:tensorstuctured}
\begin{tikzcd}[column sep={.4in,between origins}, row sep=.1in]
& x &&&& x' &&&& x+x' & \\
&&&\ot&&&& = &&&& \\
L(a)\ar[uur,"i"]&& L(b)\ar[uul,"o"'] && L(a')\ar[uur,"i'"] && L(b')\ar[uul,"o'"'] && L(a+a')\ar[uur,"i+i'"] && L(b+b')\ar[uul,"o+o'"'] 
\end{tikzcd}
\end{equation}
using that $L$ preserves coproducts.
\end{thm}

\begin{proof}
This was proved in \cite[Theorems~2.3 \& 3.9]{BC}, where all the structures are specified in detail.  In fact, the double category structure only requires that $\X$ have pushouts, whereas the symmetric monoidal structure also requires that $\X$ and $\A$ have finite colimits and that $L$ preserves finite coproducts \cite[Theorem~3.2.3]{CourserThesis}.
\end{proof}

The following theorem establishes an isomorphism between structured and decorated cospan double categories under certain conditions.  Let $\Rex$ be the 2-category of categories with finite colimits, functors preserving finite colimits, and natural transformations.  Let $\SMC$ be the 2-category of symmetric monoidal categories, strong symmetric monoidal functors and natural transformations.   Recall that for us a category $\C \in \Rex$ comes with a choice of finite colimits, so it gives a specific cocartesian monoidal category $(\C,+)$, and this induces a 2-functor $\Rex \to \SMC$.

\begin{thm} \label{thm:equiv}
Suppose $\A$ has finite colimits and $F \maps (\A,+) \to (\Cat,\times)$ is a symmetric lax monoidal pseudofunctor. If the corresponding pseudofunctor $F \maps \A \to \SMC$ factors through $\Rex$, then the symmetric monoidal double categories $F\lCsp$ of decorated cospans and $_L\lCsp(\inta F)$ of structured cospans are isomorphic, where $L \maps \A \to \inta F$ is a left adjoint of the induced Grothendieck opfibration $U\maps \inta F \to \A$.
\end{thm}

The ``corresponding pseudofunctor'' comes from the so-called monoidal Grothendieck construction, and the conditions of this theorem related to the existence of colimits as well as left adjoints for opfibrations. We first sketch the relevant underlying framework in detail, and then we proceed to the proof of the theorem. The basics of fibration theory needed for our purposes are recalled in \cref{sec:fibrations}. 

In \cite{MV}, the classical Grothendieck construction is generalized to the monoidal setting: given a monoidal category $\A$ there is a bijection between \define{monoidal opindexed categories}, which are lax monoidal pseudofunctors $F\maps (\A, \otimes_A) \to (\Cat, \times)$, and \define{monoidal opfibrations}, which are opfibrations $U \maps \X \to \A$ where $\X$, $\A$ are monoidal, $U$ is strict monoidal, and $\otimes_\X$ preserves cocartesian liftings.   This bijection sends a monoidal opindexed category $F \maps (\A,\otimes_A) \to (\Cat, \times)$ to the monoidal opfibration $U \maps \inta F \to \A$.  If the monoidal structure on $\A$ is in fact cocartesian, there is a further correspondence between these structures and ordinary pseudofunctors $F\maps \A \to \MonCat$, where $\MonCat$ is 2-category of monoidal categories, strong monoidal functors and monoidal natural transformations.

In fact, all these correspondences lift to 2-equivalences, and symmetry can also be incorporated.

\begin{lem}\label{lem:MonGroth}
There is a 2-equivalence between the 2-categories of monoidal opfibrations and monoidal opindexed categories with a fixed base, and if the base is cocartesian monoidal, there is a 2-equivalence between these and pseudofunctors into $\MonCat$.   Similarly there is a 2-equivalence between symmetric monoidal opfibrations and symmetric monoidal opindexed categories with a fixed base, and if the base is cocartesian monoidal also a 2-equivalence between these and pseudofunctors into $\SMC$.   
\end{lem}

\begin{proof}
This was shown by Moeller and the third author \cite[Theorems~3.13 \& 4.2]{MV}. In summary, for a cocartesian base $\A$ we have correspondences
\begin{gather}
\textrm{lax monoidal pseudofunctors }F\maps(\A,+)\to(\Cat,\times) \notag\\
\Updownarrow \notag\\
\textrm{monoidal opfibrations }U\maps(\X,\otimes_\X)\to(\A,+) \notag\\
\Updownarrow \notag\\
\textrm{pseudofunctors } \tilde{F} \maps \A\to \MonCat \notag
\end{gather}
The second equivalence was observed earlier by Shulman \cite{Shulman2008}. Moreover, symmetric lax monoidal pseudofunctors correspond to symmetric monoidal opfibrations, and those to pseudofunctors into $\SMC$.

In more detail, if $(\phi,\phi_0)$ is the lax monoidal structure of the pseudofunctor $F$ as recalled in \cref{subsec:bicats}, the induced monoidal structure on the total Grothendieck category $\X=\inta F$ (\cref{def:GrothCat}) is given by
\begin{equation}\label{eq:explicitstructure2}
\Big(a,x\in F(a)\Big)\ot_{\X}\Big(b,y\in F(b)\Big)=\Big(a+b,\phi_{a,b}(x,y)\in F(a+b)\Big), \qquad I_{\X}=\Big(0_A,\phi_0\Big)
\end{equation}
If $F$ is a symmetric lax monoidal pseudofunctor, the induced monoidal structure in $\inta F$ is symmetric via
$$\left(\beta_{a,b},(u_{a,b})_{x,y}\right)\colon \left(a+b,\phi_{a,b}(x,y)\right)\simrightarrow\left(b+a,\phi_{b,a}(y,x)\right)$$ where $\beta$ is the canonical symmetry for $\A$ and $u$ is the natural isomorphism of \cref{eq:braidedpseudofun}.

Moreover, each fiber $\X_a=F(a)$ obtains a monoidal structure via 
\begin{equation}\label{eq:explicitstructure1}
\otimes_a\maps F(a)\times F(a)\xrightarrow{\phi_{a,a}}F(a+a)\xrightarrow{F(\nabla)}F(a),\quad
I_a\maps\one\xrightarrow{\phi_0}F(0)\xrightarrow{F(!)}F(a)
\end{equation}
where $\nabla$ is the fold map, which is symmetric when $F$ is again via the components of $u_{a,a}$.
\end{proof}

\begin{comment} %%%John's version with symmetries
In \cite{MV}, the classical Grothendieck construction is generalized to the monoidal setting: given a symmetric monoidal category $\A$ there is a bijection between \define{symmetric monoidal opindexed categories}, which are symmetric lax monoidal pseudofunctors $F\maps (\A, \otimes_A) \to (\Cat, \times)$, and \define{symmetric monoidal opfibrations}, which are opfibrations $U \maps \X \to \A$ where $\X$, $\A$ are monoidal, $U$ is strict symmetric monoidal, %%% must it preserve the symmetries strictly too? 
and $\otimes_\X$ preserves cocartesian liftings.   This bijection sends a symmetric monoidal opindexed category $F \maps (\A,\otimes_A) \to (\Cat, \times)$ to the symmetric monoidal opfibration $U \maps \inta F \to \A$.  If the symmetric monoidal structure on $\A$ is in fact cocartesian, there is a further correspondence between these structures and ordinary pseudofunctors $F\maps \A \to \SMC$, where $\SMC$ is 2-category of symmetric monoidal categories, strong symmetric monoidal functors and monoidal natural transformations.

 In fact, all these correspondences lift to 2-equivalences.

\begin{lem}\label{lem:MonGroth}
There is a 2-equivalence between the 2-categories of symmetric monoidal opfibrations and symmetric monoidal opindexed categories. If the base is cocartesian monoidal, there is a 2-equivalence between symmetric monoidal opfibrations and pseudofunctors into $\SMC$.
\end{lem}

\begin{proof}
This was shown by Moeller and the third author \cite[Theorems~3.13\&4.2]{MV}. In summary, for a cocartesian base $\A$ we have correspondences
\begin{gather}
\textrm{symmetric lax monoidal pseudofunctors }F\maps(\A,+)\to(\Cat,\times) \notag\\
\Updownarrow \notag\\
\textrm{symmetric monoidal opfibrations }U\maps(\X,\otimes_\X)\to(\A,+) \notag\\
\Updownarrow \notag\\
\textrm{pseudofunctors } \tilde{F} \maps \A\to \SMC \notag
\end{gather}
The second equivalence was observed earlier by Shulman \cite{Shulman2008}.

In more detail, if $(\phi,\phi_0)$ is the lax monoidal structure of the pseudofunctor $F$ as recalled in \cref{subsec:bicats}, the induced monoidal structure on the total Grothendieck category $\X=\inta F$ (\cref{def:GrothCat}) is given by
\begin{equation}\label{eq:explicitstructure2}
\Big(a,x\in F(a)\Big)\ot_{\X}\Big(b,y\in F(b)\Big)=\Big(a+b,\phi_{a,b}(x,y)\in F(a+b)\Big), \qquad I_{\X}=\Big(0_A,\phi_0\Big)
\end{equation}
Moreover, each fiber $\X_a=F(a)$ obtains a monoidal structure via 
\begin{equation}\label{eq:explicitstructure1}
\otimes_a\maps F(a)\times F(a)\xrightarrow{\phi_{a,a}}F(a+a)\xrightarrow{F(\nabla)}F(a),\quad
I_a\maps\one\xrightarrow{\phi_0}F(0)\xrightarrow{F(!)}F(a)
\end{equation}
where $\nabla$ is the fold map.   SYMMETRIC MONOIDAL STRUCTURE??? %%%!!!
\end{proof}
\end{comment}

These 2-equivalences further restrict to the case when the Grothendieck category $(X,\otimes_\X)$ is specifically cocartesian monoidal itself, with coproducts built up from \cref{eq:explicitstructure2}. In that case, opfibrations $(\X,+) \to (\A,+)$ that strictly preserve coproducts and initial object bijectively correspond to pseudofunctors into the 2-category of cocartesian categories.  For more details, see \cite[Corollary 4.7]{MV} and the related discussion.

Finally, we can further restrict to the setting of opfibrations that also preserve pushouts, and thus all finite colimits, thanks to the following more general result.

\begin{lem}[\textbf{Hermida}] \label{lem:fibrewiselimits}
Suppose $\J$ is a small category and $U \maps \X \to \A$ is an opfibration where the base 
$\A$ has $\J$-colimits.  Then the following are equivalent:
\begin{enumerate}
 \item all fibers have $\J$-colimits, and the reindexing functors preserve them;
 \item the total category $\X$ has $\J$-colimits and $U$ strictly preserves them.
\end{enumerate}
Moreover, if $\X$ has $\J$-colimits and $U$ preserves them, they can be chosen so that $U$ strictly preserves them.   
\end{lem}

\begin{proof}
See \cite[Corollary~4.9]{Hermida1999}, and for the final statement \cite[Remark~4.11]{Hermida1999}.
\end{proof}

The first part formulates the existence of colimits \emph{locally} in each fiber, and if we let $\J$ range over all finite categories it says that the corresponding pseudofunctor $F \maps \A \to \Cat$ lands in the sub-2-category $\Rex$. The second part formulates the existence of colimits \emph{globally} in the total category $\inta F$, and if we let $\J$ range over all finite categories it says that $\X$ has finite colimits and $U$ preserves all finite colimits. As an example, suppose that $\A$ has pushouts and a Grothendieck opfibration $U \colon \inta F \to\A$ has fiberwise pushouts preserved by its reindexing functors. We can construct pushouts in $\inta F$ as follows: 
\begin{equation}\label{eq:globalpushout}
 \begin{tikzcd}[ampersand replacement=\&,column sep=1in,row sep=.5in]
\& (a+_bc,w)\ar[dd,phantom, very near start,"\upback"description] \& \\
(a,x)\ar[ur] \&\& (c,z)\ar[ul] \\
\& (b,y)\ar[ul,"{\begin{cases}f\maps b\to a &\textrm{in }\A \\k\maps F(f)y\to x &\textrm{in }F(a)\end{cases}}"description]\ar[ur,"{\begin{cases}g\maps b\to c &\textrm{in }\A \\\ell\maps F(g)y\to z &\textrm{in }F(c)\end{cases}}"description] \&
 \end{tikzcd}
\end{equation}
where $a+_b c$ is defined using the pushout in $\A$ shown at left below, and $w$ is formed as the pushout in the fiber $F(a+_bc)$ at the right below:
\begin{displaymath}
\begin{tikzcd}
& a+_bc\ar[dd,phantom, very near start,"\upback"description] & \\
a\ar[ur,"u_a"] && c\ar[ul,"u_c"'] \\
& b\ar[ul,"f"]\ar[ur,"g"'] &
\end{tikzcd}\qquad
\begin{tikzcd}
 & w \ar[dd,phantom, very near start,"\upback"description] & \\
 F(u_a)x\ar[ur] & & F(u_c)z\ar[ul] \\
 & F(u_a)(F(f)(y))\cong F(u_c)(F(g)(y))\ar[ul,"F(u_a)k"]\ar[ur,"F(u_c)\ell"'] &
 \end{tikzcd}
\end{displaymath}
In a similar way we can construct the initial object and coproducts in $\int F$ from those in the fibers, namely
\begin{equation}\label{eq:fibrewisecoprod}
(a,x)+(b,y)=\left(a+ b,F(\iota_a)x+F(\iota_b)y\right), \qquad 
 0_{\inta F}=\left(0_A,0_{F(0_A)}\right)
\end{equation}
where $\iota_a \maps a\to a+b$ and $\iota_b \maps a \to a+b$ are the sum inclusions in $\A$ and the sum on the second variable is in the fiber $F(a+b)$.

\begin{cor}\label{cor:fcocMonGroth}
 Suppose $\A$ has finite colimits and $F \maps (\A,+) \to (\Cat,\times)$ is a symmetric lax monoidal pseudofunctor. If the corresponding pseudofunctor $F  \maps \A \to \SMC$ from \cref{lem:MonGroth} factors through $\Rex$, then $\inta F$ has all finite colimits and the induced opfibration $U\maps \inta F \to \A$ preserves them.
 %Moreover, for any choice of finite colimits in $\A$, finite colimits in $\inta F$ can be chosen so that that $U$ strictly preserves them.
\end{cor}

Regarding the assumptions of the above corollary, notice that since $\A$ is cocartesian monoidal, the initial lax monoidal pseudofunc\-tor structure $(F,\phi,\phi_0)$ gives rise to a specific symmetric monoidal structure on the fibers $F(a)$ in terms of $\phi,\phi_0$, given explicitly by \cref{eq:explicitstructure1}. Since we now ask that the pseudofunctor $F\colon \A\to\SMC$ factors through $\Rex$, this fiberwise monoidal structure is required to be cocartesian, namely \cref{eq:explicitstructure1} gives coproducts and an initial object in each $F(a)$.
%,  those are clearly isomorphic to the arbitrary choice of finite coproducts in $F(a)$ coming from the above assumption that as a pseudofunctor, $F\colon \A\to\Rex$. For example, in that case, the coproducts induced in the Grothendieck category from the fiberwise ones as in \cref{eq:fibrewisecoprod} are equivalently given by \cref{eq:explicitstructure2}.

%%% Initial version
%In the case we are interested in, the pseudofunctor $F$ is lax monoidal; combining \cref{lem:MonGroth,lem:fibrewiselimits}, we deduce the following. 
%\begin{cor}\label{cor:fcocMonGroth}
% Suppose $\A$ has finite colimits and $F\maps(\A,+)\to(\Cat,\times)$ is a lax monoidal pseudofunctor. If the corresponding pseudofunctor $\tilde{F} \maps \A\to\bicat{MonCat}$ given in \cref{lem:MonGroth} factors through $\Rex$, then $\inta F$ has all finite colimits and the induced opfibration $U\maps\inta F\to\A$ preserves them.   Moreover, for any choice of finite colimits in $\A$, finite colimits in $\inta F$ can be chosen so that that $U$ strictly preserves them.
%\end{cor}

%Notice a subtlety in the above formulation. \cref{lem:MonGroth} provides a specific monoidal structure on the fibers which becomes \cref{eq:explicitstructure1} when the base $\A$ is cocartesian monoidal.   If this fiberwise monoidal structure happens to itself be cocartesian, then it is necessarily isomorphic to any other cocartesian monoidal structure on the fibers, by the universal property of colimits. Therefore, we can simplify the above assumption to that in \cref{thm:equiv} without having to specify that the structures came from monoidality of the pseudofunctor, although we \emph{will} use the formulation of coproducts and initial object in terms of the monoidal structure of $F$. For example, in that case, the coproducts induced in the Grothendieck category from the fiberwise ones as in \cref{eq:fibrewisecoprod} are equivalently given by \cref{eq:explicitstructure2}.

On a highly related note, in what follows we are also interested in the existence of a left adjoint $L$ to an induced monoidal opfibration.  The following result provides sufficient conditions for that.  Following Gray \cite{Gray}, we say a functor has a `left adjoint right inverse' or \define{lari} if it has a left adjoint where the unit of the adjunction is the identity.

\begin{lem}[\textbf{Gray}] \label{prop:opfibtolari}
Let $U \maps\X \to \A$ be an opfibration.   Then $U$ has a lari if its fibers have initial objects
that are preserved by the reindexing functors.
\end{lem}

\begin{proof}
 This is \cite[Proposition~4.4]{Gray}.    Suppose each fiber $\X_a$ of the opfibration $U$ has an initial object $\bot_a$ and these objects are preserved (up to isomorphism) by the reindexing functors.   Define $L \maps \A \to \X$ on objects $a \in \A$ by $L(a) = \bot_a$.   Given a morphism $f\maps a\to a'$ in $\A$, define $L(f)$ to be the composite
 \begin{equation}\label{eq:Lonarrows}
  \bot_a\xrightarrow{\mathrm{Cocart}(f,\bot_a)}f_!(\bot_a)\xrightarrow{\;\chi_a\;}\bot_{a'}
 \end{equation}
where $\mathrm{Cocart}(f,\bot_a)$ is the cocartesian lifting of $f$ to $\bot_a$ and $\chi_a$ is the unique isomorphism between two initial objects in the fiber above $a'$.    The functor $L$ then becomes left adjoint to $R$ with unit $\iota_a \maps a \to U(L(a))$ being the identity, using the fact that $U(L(a)) = U(\bot_a) = a$.
\end{proof}

%Notice that under \cref{lem:fibrewiselimits}, if the base category $\A$ has an initial object denoted $0_\A$, the above lemma is equivalent to $\X$ having an initial object $0_\X$ above $0_\A$.

%Work in progress \cite{CV} includes a related discussion as well as a result in the opposite direction.

%Christina: I don't think the following is needed.
%Notice that under \cref{lem:fibrewiselimits}, if the base category $\A$ has an initial object denoted $0_\A$, the above lemma is equivalent to $\X$ having an initial object $0_\X$ above $0_\A$. The way in which these two conditions give one another in this case is as follows. 
%The fibrewise initial object $\bot_a$ is precisely the cocartesian lifting of $0_\X$ along the unique map $u_a\maps0_\A\to a$ in the base category
%\begin{displaymath}
%\xymatrix @C=.4in @R=.2in
%{0_\X \ar @{.>}[d]\ar[rr]^-{\mathrm{Cocart}(0_\X,u_a)} && (u_a)_!(0_X)=:\bot_a\ar @{.>}[d] & \textrm{in }\X  \\
%0_\A\ar[rr]^-{\exists!u_a} && a & \textrm{in }\A}
%\end{displaymath}
%Moreover, if the opfibration $U$ comes from the Grothendieck construction on a pseudofunctor $F\maps\A\to\Cat$, the reindexing functors $(u_a)_!$ of the %opfibration are precisely $F(u_a)$ therefore $\bot_a=(a,F(u_a)(0_\X))$ in the Grothendieck category notation (\cref{def:GrothCat}).

%Finally, if we start with a lax monoidal pseudofunctor $(F,\phi,\phi_0)$, the monoidal Grothendieck construction of \cref{lem:MonGroth} in the cocartesian case expresses $\bot_a$ as follows by \cref{eq:explicitstructure1}:
%\begin{equation}\label{eq:initialinfibre}
%\mathbf{1}\xrightarrow{\phi_0}F(0_\A)\xrightarrow{F(u_a)}F(a) 
%\end{equation}

We now have all the necessary background to formally construct an isomorphism between the double category of decorated cospans and the double category of structured cospans, starting from a symmetric lax monoidal pseudofunctor $F \maps (\A, +) \to (\Cat, \times)$ whose corresponding pseudofunctor $F\colon \A\to\SMC$ factors through $\Rex$.

\begin{proof}[Proof of \cref{thm:equiv}]
Recall from \cref{thm:decorated_cospans} that the double category of decorated cospans $F\lCsp$ has objects and vertical 1-morphisms as in $\A$, while horizontal 1-cells are cospans $\cspn{a}{m}{b}{}{}$ in $\A$ decorated by an $x\in F(m)$ and 2-morphisms are maps of cospans $k\maps m\to m'$ together with $\tau\maps(Fk)(x)\to x'$ in $F(m')$.

By \cref{cor:fcocMonGroth}, when $F$ as an ordinary pseudofunctor factors through $\Rex$, the Gro\-the\-ndieck construction gives rise to a category $\inta F$ with finite colimits such that the corresponding opfibration $U\maps\inta{F}\to\A$ preserves all finite colimits. Moreover, by \cref{prop:opfibtolari}, $U$ has a left adjoint $L\maps\A\to\inta F$ with $UL=1_\A$. Diagrammatically,
\begin{equation}\label{eq:FtoL}
 F\maps\A\to\Cat\quad\mapsto\quad\begin{tikzcd}[baseline=.3]\inta F\ar[d,"U"'] \\ \A \end{tikzcd}\quad\mapsto\quad\begin{tikzcd}\A\ar[r,bend left,pos=.55,"L"]\ar[r,phantom,"\bot"description] & \inta F\ar[l,bend left,pos=.45,"U"]\end{tikzcd}
\end{equation}
describes the construction of the adjunction from the original $F$. Explicitly, the left adjoint picks the initial object in the fiber $L(a)=(a,\bot_a)$, also expressed as $F(!_a)\circ\phi_0(*)$ according to \cref{eq:explicitstructure1} for $(\phi,\phi_0)$ the monoidal structure of $F$. %Christina: If each adjoint picks one isomorphic initial object, can we choose the one with that choice? Well they are all isomorphic and that could help with the monoidal structure later...

As a left adjoint, this induced $L$ preserves all colimits that exist between the categories $\A$ and $\inta F$, which have finite colimits, so we can construct the double category of structured cospans $_L\lCsp(\inta F)$ of \cref{thm:SC}. Its objects and vertical morphisms are those of the category $\A$ (just as for $F\lCsp$), whereas its horizontal 1-cells are now cospans of the form $\cspn{L(a)}{(m,x)}{L(b)}{}{}$ in the Grothendieck category $\inta F$ (see \cref{def:GrothCat}). Explicitly, they consist of two pairs of morphisms
\begin{equation}\label{eq:scsphor1cell}
 (a,\bot_a)\xrightarrow{\scalebox{0.7}{\(\begin{cases}i\maps a\to m &\textrm{in }\A \\!\maps F(i)(\bot_a)\to x &\textrm{in }F(m)\end{cases}\)}}(m,x)\xleftarrow{\scalebox{0.7}{\(\begin{cases}o\maps b\to m &\textrm{in }\A \\!\maps F(o)(\bot_b)\to x &\textrm{in }F(m)\end{cases}\)}}(b,\bot_b)
\end{equation}
with $x\in F(m)$; recall that the reindexing functors $F(i),F(o)$ preserve all finite colimits across the fibers.
Finally, the 2-morphisms of this double category in this context unravel as follows:
\begin{equation}\label{eq:2morphismLCspintaF}
 \begin{tikzcd}[row sep=1in,column sep=1.7in,ampersand replacement=\&]
 (a,\bot_a)\ar[r,"{\begin{cases}i\maps a\to m &\textrm{in }\A \\!\maps F(i)(\bot_a)\to x &\textrm{in }F(m)\end{cases}}"]\ar[d,"{\begin{cases}f\maps a\to a' &\textrm{in }\A \\\chi_a\maps F(f)(\bot_a)\cong \bot_{a'} &\textrm{in }F(a')\end{cases}}"description] \& (m,x) \ar[d,"{\begin{cases}k\maps m\to m' &\textrm{in }\A \\ \tau\maps F(k)(x)\to x' &\textrm{in }F(m')\end{cases}}"description] \& (b,\bot_b)\ar[l,"{\begin{cases}o\maps b\to m &\textrm{in }\A \\!\maps F(o)(\bot_b)\to x &\textrm{in }F(m)\end{cases}}"']\ar[d,"{\begin{cases}g\maps b\to b' &\textrm{in }\A \\\chi_b\maps F(g)(\bot_b)\cong \bot_{b'} &\textrm{in }F(b')\end{cases}}"description] \\
 (a',\bot_{a'})\ar[r,"{\begin{cases}i'\maps a'\to m' &\textrm{in }\A \\!\maps F(i')(\bot_{a'})\to x' &\textrm{in }F(m')\end{cases}}"'] \& (m',x') \& (b',\bot_{b'})\ar[l,"{\begin{cases}o'\maps b'\to m' &\textrm{in }\A \\!\maps F(o')(\bot_{b'})\to x' &\textrm{in }F(m')\end{cases}}"]
 \end{tikzcd}
\end{equation}
where the outside vertical legs come from the definition of $L$ on arrows \cref{eq:Lonarrows}. 
Explicitly, the above square commutativities translate to the conditions $k\circ i=i'\circ f$ and $k\circ o=o'\circ g$ in $\A$, and also to
\begin{gather}\label{eq:Grothcommutativity}
 F(k\circ i)(\bot_a)\cong Fk(Fi(\bot_a))\xrightarrow{Fk(!)}Fk(x)\xrightarrow{\tau}x'= \\
 F(i'\circ f)(\bot_a)\cong Fi'(Ff(\bot_a))\xrightarrow{Fi'(\chi_a)}Fi'(\bot_{a'})\xrightarrow{!}x'\nonumber
\end{gather}
in the fiber $F(m')$, by composition in the Grothendieck category \cref{eq:compGrothcat}. Since both morphisms emanate from the mapped initial object in $F(m')$ under the finite-colimit-preserving reindexing functors, they are the unique such into $x'$ therefore this gives no extra conditions; similarly for the equality including $o,o'$.

Building up to an isomorphism of the above described double categories, we now define a double functor
\begin{displaymath}
\lE=(\lE_0,\lE_1)\maps _L\lCsp(\inta F) \longrightarrow F\lCsp
\end{displaymath}
which on vertical categories acts as the identity $\lE_0 = \id_{\A}$. Given a horizontal 1-cell in $_L\lCsp(\inta F)$ as in \cref{eq:scsphor1cell}, $\lE_1$ maps it to the decorated cospan $i\maps a\to m\leftarrow b:o$ with decoration $x\in F(m)$,
whereas a 2-morphism of $L$-structured cospans as in \cref{eq:2morphismLCspintaF} is mapped to the underlying map of cospans in $\A$
\begin{displaymath}
\begin{tikzcd}
a\ar[r,"i"]\ar[d,"f"'] & m\ar[d,"k"] & b\ar[l,"o"']\ar[d,"g"] \\
a'\ar[r,"i'"']& m' & b'\ar[l,"o'"]
\end{tikzcd}
\end{displaymath}
in $\A$ along with the map $\tau\maps Fk(x)\to x'$ in $F(m')$ coming from the middle arrow above, namely an $F$-decorated cospan 2-morphism. 
With these definitions, $\lE$ is indeed a double functor as in \cref{def:doublefun}: both $\lE_0$ and $\lE_1$ are functors that appropriately commute with sources and targets, and there exist globular 2-isomorphisms $\lE(N) \odot \lE(M) \simrightarrow\lE(N \odot M)$, $U_{\lE m} \simrightarrow\lE(U_m)$
for any pair of composable horizontal 1-cells $M=\cspn{(a,\bot_a)}{(m,x)}{(b,\bot_b)}{}{}$ and $N=\cspn{(b,\bot_b)}{(n,y)}{(c,\bot_c)}{}{}$ and any object $m$ of $_L\lCsp(\inta F)$. Explicitly, $\mathbb{E}(N)\odot \mathbb{E}(M)$ is given as in \cref{eq:dcospanscomposition} via the pushout cospan and decoration
\begin{equation}\label{eq:1}
 \begin{tikzcd}
  & m+_bn & \\
  a\ar[ur] && b,\ar[ul]
 \end{tikzcd}\qquad
  \one\xrightarrow{x\times y}F(m)\times F(n)\xrightarrow{\phi_{m,n}}F(m+n)\xrightarrow{F(\psi)}F(m+_bn)
\end{equation}
whereas first composing $M$ and $N$ in $_L\lCsp(\inta F)$ using pushouts in $\inta F$ as in \cref{eq:globalpushout} produces 
\begin{displaymath}
 \begin{tikzcd}[column sep={.7in,between origins}]
 && (m+_b n,F(u_m)x+_{(\bot_{m+_bn})}F(u_n)y) && \\
 & (m,x) \ar[ur] && (n,y)\ar[ul] \\
 (a,\bot_a)\ar[ur] && (b,\bot_b)\ar[ur]\ar[ul] && (c,\bot_c)\ar[ul]
 \end{tikzcd}
\end{displaymath}
where $u$ are the canonical maps into the pushout.
Under $\lE$ this is mapped to the underlying $\A$-cospan $\cspn{a}{m+_bn}{b}{}{}$ with decoration $F(u_m)x+F(u_n)y$, since a pushout over an initial object is really a coproduct. That decoration is isomorphic to that of \cref{eq:1} due to the following diagram 
\begin{equation}\label{eq:isomorphismdiag}
 \begin{tikzcd}
F(m)\times F(n)\ar[rr,"\phi"]\ar[d,"F(u_m)\times F(u_n)"']\ar[dr,phantom,"\cong"] && F(m+n)\ar[d,"F(\psi)"]\ar[d,bend right=50, phantom,"\cong"']\ar[dl,"F(u_m+u_n)"description] \\
F(m+_bn)\times F(m+_bn)\ar[r,"\phi"'] & F\left((m+_bn)+(m+_bn)\right)\ar[r,"F(\nabla)"'] & F(m+_bn)
 \end{tikzcd}
\end{equation}
where the bottom composite is the fiberwise coproduct in $F(m+_bn)$ as in \cref{eq:explicitstructure1}. The left-hand side isomorphism is due to pseudonaturality of $\phi$ whereas the right-hand side isomorphism follows from universal properties and pseudofunctoriality of $F$.
Regarding identities, the identity horizontal 1-cell $U_m$ in $_L\lCsp(\inta F)$ is $\cspn{(m,\bot_m)}{(m,\bot_m)}{(m,\bot_m)}{}{}$ with $1_m$ as the $A$-component and isomorphisms between initial objects in the fibers. Thus $\lE(U_m)$ is the identity cospan in $A$ together with the `initial decoration' $\bot_m\in F(m)$, and so is $U_{\mathbb{E}(m)}$ as in \cref{eq:UFCsp}. %Do we want to check that E is a double functor one last time?

In fact, the double functor $\lE\maps _L\lCsp(\inta F)\to F\lCsp$ as defined above is bijective on objects and vertical 1-cells (trivially as an identity functor), and is also bijective on horizontal 1-cells and on 2-morphisms. Indeed, for \cref{eq:scsphor1cell}, the unique maps from the initial objects in the fibers provide no actual extra information, and similarly for 2-cells all extra data in \cref{eq:2morphismLCspintaF} is uniquely determined and \cref{eq:Grothcommutativity} holds automatically as discussed above. As a result, we have a double isomorphism $$_L\lCsp(\inta F) \cong F\lCsp.$$

{\chris Below, if the fiberwise coproducts are directly from the monoidal Grothendieck construction, we may end up with stricter monoidal double functor.}
Since both double categories are symmetric monoidal under the running assumptions (\cref{DC,thm:SC}), we will now show that this double isomorphism is a symmetric monoidal one, namely $\lE=(\lE_0,\lE_1)$ is a symmetric strong monoidal double functor as per \cref{defn:monoidal_double_functor}. Indeed, $\lE_0=\mathrm{id}_\A$ is a symmetric monoidal functor trivially. For $\lE_1$, given two structured cospans $M=(a,\bot_a)\to(m,x)\leftarrow(b,\bot_b)$ and $M'=(a',\bot_{a'})\to(m',x')\leftarrow(b',\bot_{b'})$ in $_L\lCsp(\inta F)$, we compute the appropriate decorated cospans
\begin{align*}
\lE(M)\ot\lE(M')&\;\stackrel{\cref{eq:tensordecoration}}{=}\;a+a'\xrightarrow{i+i'}m+m'\xleftarrow{o+o'}b+b' \textrm{ with }\one\xrightarrow{x\times x'}F(m)\times F(m')\xrightarrow{\phi_{m,m'}}F(m+m') \\
\lE(M\ot M')&\;\stackrel{\cref{eq:tensorstuctured}}{=}\;\lE\Big((a+a',\bot_{a+a'})\to\underbrace{(m+m',F(\iota_m)x+F(\iota_{m'})x')}_{\cref{eq:fibrewisecoprod}}\leftarrow(b+b',\bot_{b+b'})\Big)%=a+a'\xrightarrow{i+i'}m+m'\xleftarrow{o+o'}b+b' \textrm{ with }\phi_{m,m'}(x,x')
\end{align*}
which are uniquely isomorphic $F(\iota_m)x+F(\iota_{m'})\cong \phi_{m,m'}(x,x')$ in an analogous way to \cref{eq:isomorphismdiag}, as also discussed after \cref{cor:fcocMonGroth}.
Moreover, the monoidal unit for the category of arrows of $F\lCsp$ is $0_\A\xrightarrow{\raisebox{-3pt}[0pt][0pt]{$=$}}0_\A\xleftarrow{\raisebox{-3pt}[0pt][0pt]{$=$}}0_\A$ with the trivial decoration $\phi_0\maps \one\to F(0_A)$,
which is an initial object and as such isomorphic to $0_{F(0)}$
%=\lE((0_A,)\xrightarrow{\raisebox{-3pt}[0pt][0pt]{$=$}}(0_A,\phi_0)\xleftarrow{\raisebox{-3pt}[0pt][0pt]{$=$}}(0_A,\phi_0))$
again by \cref{eq:explicitstructure2} and the discussion after \cref{cor:fcocMonGroth}.
It can be verified that with the above structure, $\lE_1$ is also a symmetric monoidal functor, and the rest of the axioms of \cref{defn:monoidal_double_functor} hold.
\end{proof}

\section{Bicategorical and categorical aspects}
\label{spinoffs}

While double categories are a natural context for studying cospans, bicategories are more 
familiar---and of course, \emph{categories} are even more so!   Luckily, all our results 
phrased in the language of double categories have analogues for bicategories and categories.  
We explain those here.

As discussed for example by Shulman \cite{Shulman2010}, any double category $\lD$ has a 
\define{horizontal bicategory}, denoted $\bD$, in which:
\begin{itemize}
\item objects are objects of $\lD$,
\item morphisms are horizontal 1-cells of $\lD$,
\item 2-morphisms are \define{globular} 2-morphisms of $\lD$, meaning 2-morphisms whose source and target vertical 1-morphisms are identities,
\item composition of morphisms is given by horizontal composition of horizontal 1-cells in $\lD$,
\item vertical and horizontal composition of 2-morphisms are given by vertical and horizontal
composition of 2-morphisms in $\lD$.
\end{itemize}
The bicategory $\bD$ has a \define{decategorification}, a category $\D$ in which:
\begin{itemize}
\item objects are objects of $\bD$,
\item morphisms are isomorphism classes of 1-morphisms of $\bD$.   
\end{itemize}
Thus, the double category $F\lCsp$ of structured cospans constructed in \cref{thm:decorated_cospans} automatically gives rise to a bicategory $F\bCsp$, and a category $F\Csp$.   In \cref{DC} we gave conditions under which the double category $F\lCsp$ becomes symmetric monoidal.   We would like the bicategory $F\bCsp$ and the category $F\Csp$ to become symmetric monoidal under the same conditions, and indeed this is true.   

A double category is `fibrant' if every vertical 1-morphism has a `companion' and a `conjoint'---concepts explained in \cref{def:companion}. Shulman (\cref{Shulhorizontalbicat}) proved that when a double category $\lD$ is fibrant, any symmetrical monoidal structure on $\lD$ gives one on $\bD$.    We can apply this to decorated cospans as follows:

\begin{lem}
The double category $F\lCsp$ is fibrant.
\end{lem}

\begin{proof}
We show that any vertical 1-morphism $f \maps a \to b'$ in $F\lCsp$ has a companion and a conjoint.  This horizontal 1-cell $\hat{f}$:
\begin{displaymath}
 a\xrightarrow{f}b\xleftarrow{1}b,\quad \bot_{b} \in F(b)
\end{displaymath}
%\[
%\begin{tikzpicture}[scale=1.5]
%\node (A) at (0,0) {$a$};
%\node (B) at (1,0) {$b$};
%\node (C) at (2,0) {$b$};
%\node (D) at (1,-0.5) {$\bot_{b} \in F(b)$};
%\path[->,font=\scriptsize,>=angle 90]
%(A) edge node[above]{$f$} (B)
%(C) edge node[above]{$1$} (B);
%\end{tikzpicture}
%\]
where $\bot_b$ is the trivial decoration as in \cref{eq:UFCsp}, is a companion of $f$ because the following two 2-morphisms, having $\hat{f}$ as
source and target:
\[
\begin{tikzpicture}[scale=1.5]
\node (A) at (0,0.5) {$a$};
\node (A') at (0,-0.5) {$b$};
\node (B) at (1,0.5) {$b$};
\node (C) at (2,0.5) {$b$};
\node (C') at (2,-0.5) {$b$};
\node (D) at (1,-0.5) {$b$};
\node (E) at (3,0.5) {$\bot_{b} \in F(b)$};
\node (F) at (3,-0.5) {$\bot_{b} \in F(b)$};
\node (G) at (5,0.5) {$a$};
\node (H) at (6,0.5) {$a$};
\node (I) at (7,0.5) {$a$};
\node (G') at (5,-0.5) {$a$};
\node (H') at (6,-0.5) {$b$};
\node (I') at (7,-0.5) {$b$};
\node (J) at (8,0.5) {$\bot_{a} \in F(a)$};
\node (K) at (8,-0.5) {$\bot_{b} \in F(b)$};
\node (L) at (1,-1) {$\tau_{1_{b}} = 1_{\bot_{b}}$};
\node (M) at (6,-1) {$\tau_f \maps F(f)(\bot_a) \to \bot_{b}$};
\path[->,font=\scriptsize,>=angle 90]
(A) edge node[above]{$f$} (B)
(C) edge node[above]{$1$} (B)
(A) edge node[left]{$f$} (A')
(C) edge node[left]{$1$} (C')
(A') edge node[above] {$1$} (D)
(C') edge node[above] {$1$} (D)
(B) edge node [left] {$1$} (D)
(G) edge node [above] {$1$} (H)
(G) edge node [left] {$1$} (G')
(H) edge node [left] {$f$} (H')
(G') edge node [above] {$f$} (H')
(I) edge node [above] {$1$} (H)
(I) edge node [left] {$f$} (I')
(I') edge node [above] {$1$} (H');
\end{tikzpicture}
\]
{\chris The decoration morphisms should be both isomorphisms due to pseudofunctoriality of F...}
satisfy the equations required of a companion \cref{eq:CompanionEq}, using vertical \cref{eq:verticalcompo} and horizontal \cref{eq:horizontalcompo} composition of 2-cells in this double category:
\[
\begin{tikzpicture}[scale=1.5]
\node (N) at (0,1.5) {$a$};
\node (O) at (1,1.5) {$a$};
\node (P) at (2,1.5) {$a$};
\node (Q) at (-1,1.5) {$\bot_a \in F(a)$};
\node (A) at (0,0.5) {$a$};
\node (A') at (0,-0.5) {$b$};
\node (B) at (1,0.5) {$b$};
\node (C) at (2,0.5) {$b$};
\node (C') at (2,-0.5) {$b$};
\node (D) at (1,-0.5) {$b$};
\node (E) at (-1,0.5) {$\bot_{b} \in F(b)$};
\node (F) at (-1,-0.5) {$\bot_{b} \in F(b)$};
\node (G) at (4,1) {$a$};
\node (H) at (5,1) {$a$};
\node (I) at (6,1) {$a$};
\node (G') at (4,0) {$b$};
\node (H') at (5,0) {$b$};
\node (I') at (6,0) {$b$};
\node (J) at (7,1) {$\bot_{a} \in F(a)$};
\node (K) at (7,0) {$\bot_{b} \in F(b)$};
\node (Q) at (1,-1) {$\tau_f \maps F(f)(\bot_a) \to \bot_{b}$};
\node (L) at (1,-1.5) {$\tau_{b} = 1_{\bot_{b}}$};
\node (M) at (5,-0.5) {$\tau_f \maps F(f)(\bot_a) \to \bot_{b}$};
\node (R) at (3,0.5) {$=$};
\path[->,font=\scriptsize,>=angle 90]
(N) edge node[above]{$1$} (O)
(P) edge node[above]{$1$} (O)
(N) edge node[left]{$1$} (A)
(O) edge node[left]{$f$} (B)
(P) edge node[left]{$f$} (C)
(A) edge node[above]{$f$} (B)
(C) edge node[above]{$1$} (B)
(A) edge node[left]{$f$} (A')
(C) edge node[left]{$1$} (C')
(A') edge node[above] {$1$} (D)
(C') edge node[above] {$1$} (D)
(B) edge node [left] {$1$} (D)
(G) edge node [above] {$1$} (H)
(G) edge node [left] {$f$} (G')
(H) edge node [left] {$f$} (H')
(G') edge node [above] {$1$} (H')
(I) edge node [above] {$1$} (H)
(I) edge node [left] {$f$} (I')
(I') edge node [above] {$1$} (H');
\end{tikzpicture}
\]
\[
\begin{tikzpicture}[scale=1.5]
\node (G) at (-1,0.5) {$a$};
\node (H) at (-1,-0.5)  {$b$};
\node (I) at (-2,0.5) {$a$};
\node (J) at (-2,-0.5) {$a$};
\node (A) at (0,0.5) {$a$};
\node (A') at (0,-0.5) {$b$};
\node (B) at (1,0.5) {$b$};
\node (C) at (2,0.5) {$b$};
\node (C') at (2,-0.5) {$b$};
\node (D) at (1,-0.5) {$b$};
\node (E) at (1,1) {$\bot_{b} \in F(b)$};
\node (F) at (1,-1) {$\bot_{b} \in F(b)$};
\node (L) at (1,-1.5) {$\tau_{b} = 1_{\bot_{b}}$};
\node (E') at (-1,1) {$\bot_{a} \in F(a)$};
\node (F') at (-1,-1) {$\bot_{b} \in F(b)$};
\node (L') at (-1,-1.5) {$\tau_{f} \maps F(f)(\bot_a) \to \bot_{b}$};
\node (M) at (2.5,0) {$=$};
\node (N) at (3,0.5) {$a$};
\node (O) at (3,-0.5) {$a$};
\node (P) at (4,0.5) {$b$};
\node (Q) at (4,-0.5) {$b$};
\node (R) at (5,0.5) {$b$};
\node (S) at (5,-0.5) {$b$};
\node (T) at (4,1) {$\bot_{b} \in F(b)$};
\node (U) at (4,-1) {$\bot_{b} \in F(b)$};
\node (V) at (4,-1.5) {$\tau_{b} = 1_{\bot_{b}}$};
\path[->,font=\scriptsize,>=angle 90]
(N) edge node[left]{$1$} (O)
(P) edge node[left]{$1$} (Q)
(R) edge node[left]{$1$} (S)
(N) edge node[above]{$f$} (P)
(O) edge node[above]{$f$} (Q)
(R) edge node[above]{$1$} (P)
(S) edge node[above]{$1$} (Q)
(A) edge node[above]{$f$} (B)
(C) edge node[above]{$1$} (B)
(A) edge node[left]{$f$} (A')
(C) edge node[left]{$1$} (C')
(A') edge node[above] {$1$} (D)
(C') edge node[above] {$1$} (D)
(B) edge node [left] {$1$} (D)
(A) edge node[above]{$1$} (G)
(G) edge node[left]{$f$} (H)
(A') edge node[above]{$1$} (H)
(J) edge node[above] {$f$} (H)
(I) edge node[left] {$1$} (J)
(I) edge node [above] {$1$} (G);
\end{tikzpicture}
\]
and right hand sides of the above two equations are given respectively by the 2-morphisms $U_f$ and $1_{\hat{f}}$. The conjoint of $f$ is given by this horizontal 1-cell $\check{f}$, which is just the opposite of the companion above:
\begin{displaymath}
 b\xrightarrow{1}b\xleftarrow{f}b,\quad \bot_{b} \in F(b)
\end{displaymath}
%\[
%\begin{tikzpicture}[scale=1.5]
%\node (A) at (0,0) {$b$};
%\node (B) at (1,0) {$b$};
%\node (C) at (2,0) {$a$};
%\node (D) at (4,0) {$\bot_{b} \in F(b)$};
%\path[->,font=\scriptsize,>=angle 90]
%(A) edge node[above]{$1$} (B)
%(C) edge node[above]{$f$} (B);
%\end{tikzpicture}
%\]
Just as $\hat{f}$ obeys the equations required of a companion, $\check{f}$ obeys the equations required of a conjoint.
\end{proof}


\begin{thm}
\label{thm:bicat}
Let $\A$ be a category with finite colimits and $F \maps (\A, +) \to (\Cat,\times)$ a symmetric lax monoidal pseudofunctor. Then there exists a symmetric monoidal bicategory $F \mathbf{Csp}$ in which:
\begin{enumerate}
\item objects are those of $\A$,
\item morphisms are $F$-decorated cospans:
\[
\begin{tikzpicture}[scale=1.5]
\node (A) at (0,0) {$a$};
\node (B) at (1,0) {$m$};
\node (C) at (2,0) {$b$};
\node (D) at (4,0) {$x \in F(m)$,};
\path[->,font=\scriptsize,>=angle 90]
(A) edge node[above]{$i$} (B)
(C) edge node[above]{$o$} (B);
\end{tikzpicture}
\]
\item a 2-morphism is a map of cospans in $\A$ 
\[
\begin{tikzpicture}[scale=1.5]
\node (A) at (0,0) {$a$};
\node (B) at (1,0.5) {$m$};
\node (C) at (2,0) {$b$};
\node (E) at (1,-0.5) {$m'$};
\node (D) at (3,0.5) {$x \in F(m)$};
\node (F) at (3,-0.5) {$x' \in F(m')$};
\path[->,font=\scriptsize,>=angle 90]
(A) edge node[above]{$i$} (B)
(C) edge node[above]{$o$} (B)
(A) edge node[below]{$i'$} (E)
(B) edge node[left]{$h$} (E)
(C) edge node[below]{$o'$} (E);
\end{tikzpicture}
\]
together with a morphism $\tau \maps F(h)(x) \to x'$ in $F(m')$.
\end{enumerate}
\end{thm}

\begin{proof}
This follows by applying Shulman's result \cite[Theorem 1.2]{Shulman2010} to the fibrant symmetric monoidal double category $F\lCsp$.
\end{proof}

This symmetric monoidal bicategory $F\bCsp$ generalizes the one constructed by the second author \cite{Courser}.    We can decategorify $F\bCsp$ to obtain a symmetric monoidal category generalizing the kind considered by Fong \cite{Fong}:

\begin{cor}
Let $\A$ be a category with finite colimits and $F \maps (\A, +) \to (\Cat, \times)$ a symmetric lax monoidal pseudofunctor.  Then there exists a symmetric monoidal category $F\Csp$ in which:
\begin{enumerate}
\item objects are those of $\A$
\item morphisms as isomorphism classes of $F$-decorated cospans of $\A$, where two
$F$-decorated cospans
\[
\begin{tikzpicture}[scale=1.5]
\node (A) at (0,0) {$a$};
\node (B) at (1,0) {$m$};
\node (C) at (2,0) {$b$};
\node (D) at (4,0) {$x \in F(m)$};
\path[->,font=\scriptsize,>=angle 90]
(A) edge node[above]{$i$} (B)
(C) edge node[above]{$o$} (B);
\end{tikzpicture}
\]
\[
\begin{tikzpicture}[scale=1.5]
\node (A) at (0,0) {$a$};
\node (B) at (1,0) {$m'$};
\node (C) at (2,0) {$b$};
\node (D) at (4,0) {$x' \in F(m')$};
\path[->,font=\scriptsize,>=angle 90]
(A) edge node[below]{$i'$} (B)
(C) edge node[below]{$o'$} (B);
\end{tikzpicture}
\]
are isomorphic if and only if there exists an isomorphism $f \maps m \to m'$ in $\A$ such that following diagram commutes:
\[
\begin{tikzpicture}[scale=1.5]
\node (A) at (0,0) {$a$};
\node (B') at (1,0.5) {$m$};
\node (B) at (1,-0.5) {$m'$};
\node (C) at (2,0) {$b$};
\path[->,font=\scriptsize,>=angle 90]
(A) edge node[below]{$i'$} (B)
(C) edge node[below]{$o'$} (B)
(A) edge node[above]{$i$} (B')
(C) edge node[above]{$o$} (B')
(B') edge node[left]{$f$} (B);
\end{tikzpicture}
\]
and there exists an isomorphism $\tau \maps F(f)(x) \to x'$ in $F(m')$.
\end{enumerate}
\end{cor}

In \cref{thm:equiv} we gave conditions under which the symmetric monoidal double category of \emph{decorated} cospans $F\lCsp$ is isomorphic to the  symmetric monoidal double category of \emph{structured} cospans ${}_L \lCsp(\int F)$.   We now show that under the same conditions we get an isomorphism of symmetric monoidal bicategories, and of categories.

\begin{thm} \label{thm:bicat_equiv}
Suppose $\A$ has finite colimits and $F \maps(\A,+) \to (\Cat,\times)$ is a symmetric lax monoidal pseudofunctor that factors through $\Rex$ as an ordinary pseudofunctor.    Define the symmetric monoidal bicategory $_L\bCsp(\inta F)$ as in \cref{thm:equiv}.   Then there is an equivalence of symmetric monoidal bicategories
\[      F\bCsp \simeq {}_L \bCsp(\inta F)   \]
and of symmetric monoidal categories
\[      F\Csp \simeq {}_L \Csp(\inta F)  . \]
\end{thm}

\begin{proof} 
Hansen and Shulman \cite{HS} showed that the passage from symmetric monoidal double categories to symmetric monoidal bicategories is  functorial in a suitable sense, implying that an equivalence of symmetric monoidal double categories $\lD \simeq \lD'$ gives an equivalence of symmetric monoidal bicategories $\bD \simeq \bD'$.    Since the process of decategorifying a bicategory merely discards 2-morphisms and takes isomorphism classes of 1-morphisms, the equivalence of symmetric monoidal bicategories $\bD \simeq \bD'$ in turn induces an equivalence of symmetric monoidal categories $\D \simeq \D'$.   Thus, the theorem follows from \cref{thm:equiv}. \end{proof}

\section{Applications}\label{Applications}

Thinking about systems and processes categorically dates back to early works by Lawvere \cite{Lawvere}, Bunge--Fiore \cite{BungeFiore}, Joyal--Nielsen--Winskel \cite{JNW}, Katis--Sabadini--Walters \cite{KSW} and others.   Spivak and others have used wiring diagrams and sheaves to capture compositional features of dynamical systems, \cite{BFV,SSV,VSL}.  Another approach uses signal flow diagrams and other string diagrams \cite{BE,BSZ,FRS} to understand systems behaviorally, following ideas of Willems \cite{Willems}.  
 
Decorated cospans were introduced by Fong \cite{Fong,FongThesis} to describe open systems as cospans equipped with extra data.  They were then applied to open electrical circuits \cite{BF}, Markov processes \cite{BFP}, and chemical reaction networks \cite{BP}.  Unfortunately, some of these applications were marred by technical flaws, which were later fixed using structured cospans \cite{BC}. Here we explain how to also fix them using our new decorated cospans, since they provide another another solution to these problems. Below we compare the two approaches in applications to graphs, electrical circuits, Petri nets, reaction networks and dynamical systems. In many cases, \cref{thm:equiv} shows that the structured and decorated cospan approaches are equivalent: \cref{subsec:graphs,subsec:circuits,subsec:petri} illustrate this. However, in \cref{subsec:petrirates} we show that open dynamical systems can only be treated using decorated cospans.

\subsection{Graphs}
\label{subsec:graphs}

One of the simplest kinds of network is a graph.  For us a \define{graph} will be a pair of functions $s,t\maps E \to N$ where $E$ and $N$ are finite sets.   We call elements of $E$ \define{edges} and elements of $N$ \define{nodes}.  There is a category $\Graph$ where the objects are graphs and a morphism from the graph $s,t\maps E \to N$ to the graph $s',t' \maps E' \to N'$ is a pair of functions $f \maps E \to E', g \maps N \to N'$ such that these diagrams commute:
\[
\begin{tikzpicture}[scale=1.5]
\node (A) at (0,0) {$E$};
\node (A') at (0,-1) {$E'$};
\node (B) at (1,0) {$N$};
\node (B') at (1,-1) {$N'$};
\path[->,font=\scriptsize,>=angle 90]
(A) edge node[above]{$s$} (B)
(A') edge node[above]{$s'$} (B')
(A) edge node[left]{$f$} (A')
(B) edge node[right]{$g$} (B');

\node (C) at (2,0) {$E$};
\node (C') at (2,-1) {$E'$};
\node (D) at (3,0) {$N$};
\node (D') at (3,-1) {$N'$.};
\path[->,font=\scriptsize,>=angle 90]
(C) edge node[above]{$t$} (D)
(C') edge node[above]{$t'$} (D')
(C) edge node[left]{$f$} (C')
(D) edge node[right]{$g$} (D');
\end{tikzpicture}
\]

We can easily build a double category with `open graphs' as horizontal 1-cells using the machinery of structured cospans, see \cite[Section 5]{BC}.  Let $L \maps \Fin\Set \to \Graph$ be the functor that assigns to a finite set $N$ the \define{discrete graph} on $N$: the graph with no edges and $N$ as its set of vertices. Both $\Fin\Set$ and $\Graph$ have finite colimits, and the functor $L \maps \Fin\Set \to \Graph$ is left adjoint to the forgetful functor $R \maps \Graph \to \Fin\Set$ that assigns to a graph $G$ its underlying set of vertices $R(G)$. Thus, using structured cospans and appealing to \cref{thm:SC}, we get a symmetric monoidal double category $_L \lCsp(\Graph)$ in which:
\begin{itemize}
\item objects are finite sets,
\item a vertical 1-morphism from $X$ to $Y$ is a function $f \maps X \to Y$,
\item a horizontal 1-cell from $X$ to $Y$ is an \define{open graph} from $X$ to $Y$, meaning a cospan in $\Graph$ of this form:
\[
\begin{tikzpicture}[scale=1.5]
\node (A) at (0,0) {$L(X)$};
\node (B) at (1,0) {$G$};
\node (C) at (2,0) {$L(Y)$,};
\path[->,font=\scriptsize,>=angle 90]
(A) edge node[above]{$i$} (B)
(C) edge node[above]{$o$} (B);
\end{tikzpicture}
\]
\item a 2-morphism is a commuting diagram in $\Graph$ of this form:
\[
\begin{tikzpicture}[scale=1.5]
\node (A) at (0,0) {$L(X_1)$};
\node (B) at (1,0) {$G_1$};
\node (C) at (2,0) {$L(Y_1)$};
\node (A') at (0,-1) {$L(X_2)$};
\node (B') at (1,-1) {$G_2$};
\node (C') at (2,-1) {$L(Y_2)$.};
\path[->,font=\scriptsize,>=angle 90]
(A) edge node[above]{$I_1$} (B)
(C) edge node[above]{$O_1$} (B)
(A') edge node[above]{$I_2$} (B')
(C') edge node[above]{$O_2$} (B')
(A) edge node [left]{$L(f)$} (A')
(B) edge node [left]{$\alpha$} (B')
(C) edge node [left]{$L(g)$} (C');
\end{tikzpicture}
\]
\end{itemize}
Here is an example of an open graph:
\[
\scalebox{0.8}{
\begin{tikzpicture}
	\begin{pgfonlayer}{nodelayer}
		\node [contact] (n1) at (-2,0) {$\bullet$};
		\node [style = none] at (-2.1,0.3) {$n_1$};
		\node [contact] (n2) at (0,1) {$\bullet$};
		\node [style = none] at (0,1.3) {$n_2$};
		\node [contact] (n3) at (0,-1) {$\bullet$};
		\node [style = none] at (0,-1.3) {$n_3$};
		\node [contact] (n4) at (2,0) {$\bullet$};
		\node [style = none] at (2.1,0.3) {$n_4$};
		
		\node [style = none] at (-1,1) {$e_1$};
		\node [style = none] at (-1,-1) {$e_2$};
		\node [style = none] at (1,1) {$e_3$};
		\node [style = none] at (1,-1) {$e_4$};
	    \node [style = none] at (0.3,0) {$e_5$};
		
		\node [style=none] (1) at (-3,0) {1};
		\node [style=none] (4) at (3,0) {2};
	
		\node [style=none] (ATL) at (-3.4,1.4) {};
		\node [style=none] (ATR) at (-2.6,1.4) {};
		\node [style=none] (ABR) at (-2.6,-1.4) {};
		\node [style=none] (ABL) at (-3.4,-1.4) {};

		\node [style=none] (X) at (-3,1.8) {$X$};
		\node [style=inputdot] (inI) at (-2.8,0) {};
		
		\node [style=none] (Z) at (3,1.8) {$Y$};
	 \node [style=inputdot] (outI') at (2.8,0) {};

		\node [style=none] (MTL) at (2.6,1.4) {};
		\node [style=none] (MTR) at (3.4,1.4) {};
		\node [style=none] (MBR) at (3.4,-1.4) {};
		\node [style=none] (MBL) at (2.6,-1.4) {};
	
	\end{pgfonlayer}
	\begin{pgfonlayer}{edgelayer}
		\draw [style=inarrow, bend left=20, looseness=1.00] (n1) to (n2);
		\draw [style=inarrow, bend right=20, looseness=1.00] (n1) to (n3);
		\draw [style=inarrow, bend left=20, looseness=1.00] (n2) to (n4);
		\draw [style=inarrow, bend right=20, looseness=1.00] (n3) to (n4);
		\draw [style=inarrow] (n2) to (n3);
%		\draw [style=inarrow] (W) to (Water);
%		\draw [style=inarrow, bend left=40, looseness=1.00] (Water2) to (Something);
%		\draw [style=inarrow, bend right=40, looseness=1.00] (Water2) to (Something);
%		\draw [style=inarrow, bend left=40, looseness=1.00] (Something) to (A);
%		\draw [style=inarrow, bend right=40, looseness=1.00] (Something) to (B);
		\draw [style=simple] (ATL.center) to (ATR.center);
		\draw [style=simple] (ATR.center) to (ABR.center);
		\draw [style=simple] (ABR.center) to (ABL.center);
		\draw [style=simple] (ABL.center) to (ATL.center);
%		\draw [style=simple] (BTL.center) to (BTR.center);
%		\draw [style=simple] (BTR.center) to (BBR.center);
%		\draw [style=simple] (BBR.center) to (BBL.center);
%		\draw [style=simple] (BBL.center) to (BTL.center);
		\draw [style=simple] (MTL.center) to (MTR.center);
		\draw [style=simple] (MTR.center) to (MBR.center);
		\draw [style=simple] (MBR.center) to (MBL.center);
		\draw [style=simple] (MBL.center) to (MTL.center);
%		\draw [style=inputarrow] (outI) to (A);
%		\draw [style=inputarrow] (outS) to (B);
		\draw [style=inputarrow] (inI) to (n1);
		\draw [style=inputarrow] (outI') to (n4);
%		\draw [style=inputarrow] (inI') to (Water2);
%		\draw [style=inputarrow] (inS') to (Water2);
	\end{pgfonlayer}
\end{tikzpicture}
}
\]

We can also build a double category with open graphs as horizontal 1-cells using decorated cospans.    For any finite set $N$, there is a category $F(N)$ where:
\begin{itemize}
\item An object is a \define{graph structure} on $N$: that is, a graph $s,t \maps E \to N$.
\item A morphism from $s,t \maps E \to N$ to $s',t' \maps E' \to N$ is a morphism of graphs that is
the identity on $N$: that is, a function $f \maps E \to E'$ such that these diagrams commute:
\[
\begin{tikzpicture}[scale=1.5]
\node (A) at (0,0) {$E$};
\node (A') at (0,-1) {$E'$};
\node (B) at (1,-0.5) {$N$};
\path[->,font=\scriptsize,>=angle 90]
(A) edge node[above]{$s$} (B)
(A') edge node[below]{$s'$} (B)
(A) edge node[left]{$f$} (A');
\node (C) at (2,0) {$E$};
\node (C') at (2,-1) {$E'$};
\node (D) at (3,-0.5) {$N$.};
\path[->,font=\scriptsize,>=angle 90]
(C) edge node[above]{$t$} (D)
(C') edge node[below]{$t'$} (D)
(C) edge node[left]{$f$} (C');
\end{tikzpicture}
\]
\end{itemize}

In general, decorated cospans involve a pseudofunctor to $\Cat$, but in this example there is actually an honest functor $F \maps \Set \to \Cat$ that assigns to a set $N$ the above category $F(N)$.   Given a function $f \maps M \to N$, we define $F(f) \maps F(M) \to F(N)$ to any graph structure $s,t \maps E \to M$ to the graph structure $f s, f t \maps E \to N$.   

We can make $F$ into a symmetric lax monoidal pseudofunctor $F \maps (\Fin\Set, +) \to (\Cat,\times)$ by equipping it with suitable functors 
\[   \phi_{N,N'} \maps F(N) \times F(N') \to F(N+N'), \qquad \phi_0 \maps  \one \to F(\emptyset) . \]  

The functor $\phi_0$ is uniquely determined since $F(\emptyset)$ is the terminal category.   More interesting is $\phi_{N,N'}$.   This functor maps a pair of graph structures $s, t \maps E \to N$ and $s',t' \maps E' \to N'$ to the graph structure $s+s', t+t' \maps E+E' \to N+N'$.  In other words, it sends a pair of graph structures to their `disjoint union'.   Surprisingly, though $F$ is a functor, this choice of $\phi_{N,N'}$ does not make $F$ into a symmetric lax monoidal functor, but only a symmetric lax monoidal pseudofunctor, since it obeys the required laws only up to natural isomorphism, as in \cref{eq:omega}.   See \cite[Section 5]{BC} for a proof that these laws fail to hold on the nose.   This fact is what necessitated a generalization of Fong's original approach to decorated cospans.

It is well known, and easy to check, that the Grothendieck category $\inta F$ is isomorphic to the category $\Graph$.   The other side of this observation is that the opfibration $U \maps \inta F \to \Fin\Set$ is isomorphic to the forgetful functor $R \maps \Graph \to \Fin\Set$.    In fact one can check that $U \maps \inta F \to \Fin\Set$ and $R \maps \Graph \to \Fin\Set$ are isomorphic as symmetric monoidal opfibrations, where all the categories involved are given cocartesian monoidal structures.

Starting from the symmetric lax monoidal pseudofunctor $F \maps (\Fin\Set, +) \to (\Cat,\times)$, \cref{DC} gives us a symmetric monoidal double category $F\lCsp$ in which:
\begin{itemize}
\item objects are sets,
\item a vertical 1-morphism from $X$ to $X'$ is a function $f \maps X \to X'$,
\item a horizontal 1-cell from $X$ to $Y$ is a pair
\[
\begin{tikzpicture}[scale=1.5]
\node (A) at (0,0) {$X$};
\node (B) at (1,0) {$N$};
\node (C) at (2,0) {$Y$};
\node (D) at (3.25,0) {$G \in F(N)$};
\path[->,font=\scriptsize,>=angle 90]
(A) edge node[above]{$i$} (B)
(C) edge node[above]{$o$} (B);
\end{tikzpicture}
\]
which can also be thought of as an open graph from $X$ to $Y$,
\item a 2-morphism
\[
\begin{tikzpicture}[scale=1.5]
\node (A) at (0,0) {$X$};
\node (B) at (1,0) {$N$};
\node (C) at (2,0) {$Y$};
\node (A') at (0,-1) {$X'$};
\node (C') at (2,-1) {$Y'$};
\node (D) at (1,-1) {$N'$};
\node (E) at (3,0) {$G \in F(N)$};
\node (F) at (3,-1) {$G' \in F(N')$};
\path[->,font=\scriptsize,>=angle 90]
(A) edge node[above]{$i$} (B)
(C) edge node[above]{$o$} (B)
(A) edge node[left]{$f$} (A')
(C) edge node[right]{$g$} (C')
(C') edge node [above] {$o'$} (D)
(A') edge node [above] {$i'$} (D)
(B) edge node [left] {$h$} (D);
\end{tikzpicture}
\]
is a commuting diagram in $\Fin\Set$ together with a morphism $\tau \maps F(h)(G) \to G'$ in $F(N')$.
\end{itemize}

We thus have two symmetric monoidal double categories: ${}_L \lCsp(\Graph)$ obtained from structured cospans and $F\lCsp$ obtained from decorated cospans. Each of these double categories has $\Fin\Set$ as its category of objects, open graphs as horizontal 1-cells, and maps of open graphs as 2-morphisms.   This suggests that ${}_L \lCsp(\Graph)$  and $F\lCsp$ are isomorphic as symmetric monoidal double categories---and indeed this is true.   This follows from \cref{thm:equiv} together with the fact that $U \maps \inta F \to \Fin\Set$ and $R \maps \Graph \to \Fin\Set$ are isomorphic as symmetric monoidal opfibrations.  

\subsection{Circuits}
\label{subsec:circuits}

Structured and decorated cospans are a powerful tool for studying categories where the morphisms are electrical circuits---see \cite[Section 6.1]{BC} and \cite{BCR,BF}.  The key idea is to use open graphs with labeled edges to describe circuits, where the labels can stand for resistors with any chosen resistance, capacitors with any chosen capacitance, or other circuit elements.   The whole theory of open graphs discussed in the previous section can be recapitulated for labeled graphs.  Since the abstract formalism works the same way, we can be brief.   Concrete applications of this formalism are discussed in the references.

Fix a set $\La$ to serve as edge labels.  Define an $\La$-\define{graph} to be a graph $s,t\maps E\to N$ equipped with a function $\ell \maps E \to \La$.  There is a category $\Graph_\La$ where the objects are $\La$-graphs and a morphism from the $\La$-graph 
 \[ \xymatrix{\La & E \ar@<2.5pt>[r]^{s} \ar@<-2.5pt>[r]_{t} \ar[l]_{\ell} & N} \]
 to the $\La$-graph 
\[ \xymatrix{\La & E' \ar@<2.5pt>[r]^{s'} \ar@<-2.5pt>[r]_{t'} \ar[l]_{\ell'} & N'} \]
is a pair of functions $f \maps E \to E', g \maps N \to N'$ such that these diagrams commute:
\[
\begin{tikzpicture}[scale=1.5]
\node (A) at (0,0) {$E$};
\node (A') at (0,-1) {$E'$};
\node (B) at (1,0) {$N$};
\node (B') at (1,-1) {$N'$};
\path[->,font=\scriptsize,>=angle 90]
(A) edge node[above]{$s$} (B)
(A') edge node[above]{$s'$} (B')
(A) edge node[left]{$f$} (A')
(B) edge node[right]{$g$} (B');

\node (C) at (2,0) {$E$};
\node (C') at (2,-1) {$E'$};
\node (D) at (3,0) {$N$};
\node (D') at (3,-1) {$N'$};
\path[->,font=\scriptsize,>=angle 90]
(C) edge node[above]{$t$} (D)
(C') edge node[above]{$t'$} (D')
(C) edge node[left]{$f$} (C')
(D) edge node[right]{$g$} (D');

\node (E) at (4,-0.5) {$\La$};
\node (G) at (5,0) {$E$};
\node (G') at (5,-1) {$E'$.};
%\node[rotate=-90] at (4.1,-0.5) {$\sim$};
\path[->,font=\scriptsize,>=angle 90]
(G) edge node[above]{$\ell$} (E)
(G) edge node[right]{$f$} (G')
(G') edge node[below]{$\ell'$} (E);
\end{tikzpicture}
\]
There is a functor $U \maps \Graph_\La \to \Fin\Set$ that takes an $\La$-graph to its underlying set of nodes. This has a left adjoint $L \maps \Fin\Set \to \Graph_\La$ sending any set to the $\La$-graph with that set of nodes and no edges.  Both $\Fin\Set$ and $\Graph_\La$ have colimits, and $L$ preserves them.  

This sets the stage for structured cospans: \cref{thm:SC} gives us a symmetric monoidal double category ${}_L \lCsp(\Graph_\La)$ where a horizontal 1-cell is an \define{open} $\La$-\define{graph}, also called an $\La$-\define{circuit}: that is, a cospan in $\Graph_\La$ of this form:
\[
\begin{tikzpicture}[scale=1.5]
\node (A) at (0,0) {$L(X)$};
\node (B) at (1,0) {$G$};
\node (C) at (2,0) {$L(Y)$.};
\path[->,font=\scriptsize,>=angle 90]
(A) edge node[above]{$i$} (B)
(C) edge node[above]{$o$} (B);
\end{tikzpicture}
\]
For example, here is a $\La$-circuit with $\La = (0,\infty)$:
\[
\scalebox{1}{
\begin{tikzpicture}
	\begin{pgfonlayer}{nodelayer}
		\node [contact] (n1) at (-2,0) {$\bullet$};
		\node [style = none] at (-2.1,0.3) {$$};
		\node [contact] (n2) at (0,1) {$\bullet$};
		\node [style = none] at (0,1.3) {$$};
		\node [contact] (n3) at (0,-1) {$\bullet$};
		\node [style = none] at (0,-1.3) {$$};
		\node [contact] (n4) at (2,1) {$\bullet$};
		\node [style = none] at (2.1,0.3) {$$};
		\node [contact] (n5) at (2,-1) {$\bullet$};
		\node [style = none] at (2.1,0.3) {$$};
		
		\node [style = none] at (-1,1.1) {$2.53$};
		\node [style = none] at (-1,-1.1) {$0.71$};
		\node [style = none] at (1,1.3) {$9.6$};
		\node [style = none] at (1,-1.3) {$1.02$};
	     \node [style = none] at (-0.4,0) {$12.4$};
	     \node [style = none] at (1.6,0) {$6.3$};
		
		\node [style=none] (1) at (-3,0) {};
		\node [style=none] (4) at (3,0) {};
	
		\node [style=none] (ATL) at (-3.4,1.4) {};
		\node [style=none] (ATR) at (-2.6,1.4) {};
		\node [style=none] (ABR) at (-2.6,-1.4) {};
		\node [style=none] (ABL) at (-3.4,-1.4) {};

		\node [style=none] (X) at (-3,1.8) {$X$};
		\node [style=inputdot] (inI) at (-2.8,0) {};
		
		\node [style=none] (Z) at (3,1.8) {$Y$};
	 \node [style=inputdot] (outI') at (2.8,1) {};
	 \node [style=inputdot] (outI'') at (2.8,0) {};
	 \node [style=inputdot] (outI''') at (2.8,-1) {};

		\node [style=none] (MTL) at (2.6,1.4) {};
		\node [style=none] (MTR) at (3.4,1.4) {};
		\node [style=none] (MBR) at (3.4,-1.4) {};
		\node [style=none] (MBL) at (2.6,-1.4) {};
	
	\end{pgfonlayer}
	\begin{pgfonlayer}{edgelayer}
		\draw [style=inarrow, bend left=20, looseness=1.00] (n1) to (n2);
		\draw [style=inarrow, bend right=20, looseness=1.00] (n1) to (n3);
		\draw [style=inarrow, bend left=0, looseness=1.00] (n2) to (n4);
		\draw [style=inarrow, bend right=0, looseness=1.00] (n3) to (n4);
		\draw [style=inarrow, bend right=0, looseness=1.00] (n3) to (n5);
		\draw [style=inarrow] (n2) to (n3);
		\draw [style=simple] (ATL.center) to (ATR.center);
		\draw [style=simple] (ATR.center) to (ABR.center);
		\draw [style=simple] (ABR.center) to (ABL.center);
		\draw [style=simple] (ABL.center) to (ATL.center);
		\draw [style=simple] (MTL.center) to (MTR.center);
		\draw [style=simple] (MTR.center) to (MBR.center);
		\draw [style=simple] (MBR.center) to (MBL.center);
		\draw [style=simple] (MBL.center) to (MTL.center);
		\draw [style=inputarrow] (inI) to (n1);
		\draw [style=inputarrow] (outI') to (n4);
		\draw [style=inputarrow] (outI'') to (n5);
		\draw [style=inputarrow] (outI''') to (n5);
	\end{pgfonlayer}
\end{tikzpicture}
}
\]
The edges here represent wires, with the positive real numbers labeling them serving to describe the resistance of resistors on the wires.  The elements of the sets $X$ and $Y$ represent `terminals': that is, points where we allow ourselves to attach a wire from another circuit.

We can now also describe $\La$-circuits using our new approach to decorated cospans.   There is a symmetric lax monoidal pseudofunctor $F \maps (\Fin\Set, +) \to (\Cat, \times)$ such that for any finite set $N$, the category $F(N)$ has:
\begin{itemize}
\item objects being $\La$-\define{graph structures} on $N$: that is, $\La$-graphs where the set of nodes is $N$,
\item morphisms being morphisms of $\La$-graphs that are the identity on the set of nodes.
\end{itemize}
This gives a symmetric monoidal double category $F \lCsp$, and using \cref{thm:equiv} we can show that this is equivalent, as a symmetric monoidal double category, to ${}_L \lCsp(\Graph_\La)$.

\subsection{Petri nets}
\label{subsec:petri}

Petri nets are widely used as models of systems in engineering and computer science \cite{GiraultValk, Peterson}.   Structured cospans have been used to define a symmetric monoidal double category of `open Petri nets' \cite{BM}, which lets us build large Petri nets out of smaller pieces.  We can also use decorated cospans to create a double category of open Petri nets.  Again this example is very similar to the example of open graphs.

A \define{Petri net} is a pair of sets $S$ and $T$ and functions $s,t \maps T \to \N[S]$.  Here $S$ is the set of \define{places}, $T$ is the set of \define{transitions}, and $\N[S]$ is the underlying set of the free commutative monoid on $S$.  Each transition thus has a formal sum of places as its source and target as prescribed by the functions $s$ and $t$, respectively.  Here is an example:
\[
\begin{tikzpicture}
	\begin{pgfonlayer}{nodelayer}
		\node [style=species] (I) at (0,1) {H};
		\node [style=species] (T) at (0,-1) {O};
		\node [style=transition] (W) at (2,0) {$\phantom{\Big|}\alpha$\phantom{\Big|}};
		\node [style=species] (Water) at (4,0) {$\textnormal{H}_2$O};
%		\node [style=transition] (Something) at (6,0) {\tiny{Something}};
%		\node [style=species] (A) at (8,1) {O$\textnormal{H}^{-}$};
%		\node [style=species] (B) at (8,-1) {$\textnormal{H}_3 \textnormal{O}^{+}$};
	\end{pgfonlayer}
	\begin{pgfonlayer}{edgelayer}
		\draw [style=inarrow, bend right=40, looseness=1.00] (I) to (W);
		\draw [style=inarrow, bend left=40, looseness=1.00] (I) to (W);
		\draw [style=inarrow, bend right=40, looseness=1.00] (T) to (W);
		\draw [style=inarrow] (W) to (Water);
%		\draw [style=inarrow, bend left=40, looseness=1.00] (Water) to (Something);
%		\draw [style=inarrow, bend right=40, looseness=1.00] (Water) to (Something);
%		\draw [style=inarrow, bend left=40, looseness=1.00] (Something) to (A);
%		\draw [style=inarrow, bend right=40, looseness=1.00] (Something) to (B);
	\end{pgfonlayer}
\end{tikzpicture}
\]
This Petri net has a single transition $\alpha$ with $2\textnormal{H}+\textnormal{O}$ as its source and $\textnormal{H}_2 \textnormal{O}$ as its target. 

There is a category $\Petri$ with Petri nets as objects, where a morphism from the Petri net 
$s, t \maps T \to \N[S]$ to the Petri net $s', t' \maps T' \to \N[S']$ is a pair of functions $f \maps T \to T', g \maps S \to S'$ such that the following diagrams commute:
	\[
	\xymatrix{ 
		T \ar[d]_f  \ar[r]^-{s} & \N[S] \ar[d]^-{\N[g]} \\	
		T' \ar[r]^-{s'} & \N[S'] 
	}
	\qquad
	\xymatrix{ 
		T \ar[d]_f  \ar[r]^-{t} & \N[S] \ar[d]^-{\N[g]} \\	
		T' \ar[r]^-{t'} & \N[S'] . 
	}
	\]
There is a functor $R \maps \Petri \to \Set$ sending any Petri net to its set of places, and this has a left adjoint $L \maps \Set \to \Petri$ sending any set $S$ to the Petri net with $S$ as its set of species and no transitions \cite[Lemma 11]{BM}.   Since both $\Set$ and $\Petri$ have finite colimits and $L$ preserves them, \cref{thm:SC} yields a symmetric monoidal double category ${}_L \lCsp(\Petri)$ in which:
\begin{itemize}
\item objects are finite sets,
\item vertical 1-morphisms are functions,
\item horizontal 1-cells are \define{open Petri nets}, which are cospans in $\Petri$ of the form:
\[
\begin{tikzpicture}[scale=1.5]
\node (D) at (-3,0) {$L(X)$};
\node (E) at (-2,0) {$P$};
\node (F) at (-1,0) {$L(Y)$};
\path[->,font=\scriptsize,>=angle 90]
(D) edge node [above] {$i$} (E)
(F) edge node [above] {$o$} (E);
\end{tikzpicture}
\]
\item 2-morphisms are diagrams in $\Petri$ of the form:
\[
\begin{tikzpicture}[scale=1.5]
\node (E) at (3,0) {$L(X_1)$};
\node (F) at (5,0) {$L(Y_1)$};
\node (G) at (4,0) {$P_1$};
\node (E') at (3,-1) {$L(X_2)$};
\node (F') at (5,-1) {$L(Y_2)$.};
\node (G') at (4,-1) {$P_2$};
\path[->,font=\scriptsize,>=angle 90]
(F) edge node[above]{$O_1$} (G)
(E) edge node[left]{$L(f)$} (E')
(F) edge node[right]{$L(g)$} (F')
(G) edge node[left]{$\alpha$} (G')
(E) edge node[above]{$I_1$} (G)
(E') edge node[above]{$I_2$} (G')
(F') edge node[above]{$O_2$} (G');
\end{tikzpicture}
\]
\end{itemize}

We can equivalently describe open Petri nets using decorated cospans.  This works very much like the previous examples.  There is a symmetric lax monoidal pseudofunctor $F \maps (\Fin\Set, +) \to (\Cat, \times)$ such that for any finite set $S$, the category $F(S)$ has:
\begin{itemize}
\item objects given by Petri nets whose set of places is $S$,
\item morphisms given by morphisms of Petri nets that are the identity on the set of places.
\end{itemize}
This gives a symmetric monoidal double category $F \lCsp$, and using \cref{thm:equiv} we can show that this is equivalent, as a symmetric monoidal double category, to ${}_L \lCsp(\Petri)$.

The machinery of structured cospans has been used to provide a semantics for open Petri nets \cite{BM}: a symmetric monoidal double functor from ${}_L \lCsp(\Petri)$ to  a symmetric monoidal double category of `open commutative monoidal categories'.  Presumably this double functor can equivalently be obtained using the machinery of decorated cospans, with the help of \cref{thm:functoriality}.  However, it should be clear by now that so far, in cases where either structured or decorated cospans can be used, structured cospans are simpler.   We next turn to an example where decorated cospans are necessary.

\subsection{Petri nets with rates}
\label{subsec:petrirates}

In chemistry, population biology, epidemiology and other fields, modelers use `Petri nets with rates', where the transitions are labelled with positive real numbers called `rate constants' \cite{Haas,Koch,Wilkinson}.   From any Petri net with rates one can systematically construct a dynamical system.  Mathematical chemists have proved deep theorems relating the topology of Petri nets with rates to the qualitative behavior of their dynamical systems \cite{CTF}.

Pollard and the first author showed how to construct an `open' dynamical system from any `open' Petri net with rates,  thus defining a functor from a category with open Petri nets with rates as morphisms to one with open dynamical systems as morphisms \cite{BP}.  They used Fong's original decorated cospans to do this.   Later, structured cospans were used to promote the first of these categories to a double category \cite[Section 6.16]{BC}. Here we show that the second of these categories, with open dynamical systems as morphisms, \emph{cannot} be made into a double category using structured cospans.  However, we \emph{can} do it using decorated cospans.
 
First, to briefly illustrate these ideas, here is an open Petri net with rates:
\[
\begin{tikzpicture}
	\begin{pgfonlayer}{nodelayer}
\node [style=inputdot] (1a) at (-5.3, 0.5) {};
\node [style=inputdot] (1b) at (-5.3, 0) {};
\node [style=inputdot] (2) at (-5.3, -0.5) {};
\node [style=inputdot] (3) at (1.3, 0) {};
\node [style=none] (1'a) at (-4.4, 0.5) {};
\node [style=none] (1'b) at (-4.4, 0.5) {};
\node [style=none] (2') at (-4.4, -0.5) {};
\node [style=none] (3') at (0.4, 0) {};
\node [style=species] (S) at (-4, 0.5) {$S$};
\node [style=species] (I) at (-4, -0.5) {$I$};
\node [style=species] (R) at (0, 0) {$R$};
		\node [style=transition] (tau1) at (-1.5, 0.7) {$r_1$};
            \node [style=transition] (tau2) at (-1.5,-1) {$r_2$};
		\node [style=none] (I1a) at (-5.6, 0.5) {$i_1$};
		 \node [style=none] (I1b) at (-5.6, 0) {$i_2$};
           \node [style=none] (I2) at (-5.6, -0.5) {$i_3$};
		\node [style=none] (O3) at (1.65, 0) {$o_1$};
		\node [style=none] (ATL) at (-5.8,1.4) {};
		\node [style=none] (ATR) at (-5,1.4) {};
		\node [style=none] (ABR) at (-5,-1.4) {};
		\node [style=none] (ABL) at (-5.8,-1.4) {};
		\node [style=none] (MTL) at (1,1.4) {};
		\node [style=none] (MTR) at (1.8,1.4) {};
		\node [style=none] (MBR) at (1.8,-1.4) {};
		\node [style=none] (MBL) at (1,-1.4) {};
		\node [style=none] (X) at (-5.4,1.8) {$X$};		
		\node [style=none] (Z) at (1.4,1.8) {$Y$};
	
	\end{pgfonlayer}
	\begin{pgfonlayer}{edgelayer}
		\draw [style=inarrow, bend left =10, looseness=1.00] (S) to (tau1);
		\draw [style=inarrow, bend left=15, looseness=1.00] (I) to (tau1);
		\draw [style=inarrow, bend left=25, looseness=1.00] (tau1) to (I);
		\draw [style=inarrow, bend left=40, looseness=1.00] (tau1) to (I);
	       \draw [style=inarrow, bend right=25, looseness=1.00] (I) to (tau2);
             \draw [style=inarrow, bend right=15, looseness=1.00] (tau2) to (R); 
\path[color=purple, very thick, shorten >=2pt, shorten <=2pt, ->, >=stealth] (1a) edge (1'a);
\path[color=purple, very thick, shorten >=2pt, shorten <=2pt, ->, >=stealth] (1b) edge (1'b);
\path[color=purple, very thick, shorten >=2pt, shorten <=2pt, ->, >=stealth] (2) edge (2');	
\path[color=purple, very thick, shorten >=2pt, shorten <=2pt, ->, >=stealth] (3) edge (3');
		\draw [style=simple] (ATL.center) to (ATR.center);
		\draw [style=simple] (ATR.center) to (ABR.center);
		\draw [style=simple] (ABR.center) to (ABL.center);
		\draw [style=simple] (ABL.center) to (ATL.center);
		\draw [style=simple] (MTL.center) to (MTR.center);
		\draw [style=simple] (MTR.center) to (MBR.center);
		\draw [style=simple] (MBR.center) to (MBL.center);
		\draw [style=simple] (MBL.center) to (MTL.center);
	\end{pgfonlayer}
\end{tikzpicture}
\]
It is an open Petri net where the transitions are labelled with rate constants $r_1, r_2 > 0$.
Here is the corresponding open dynamical system:
\begin{equation}
\label{eq:openPetrir}
  \begin{array}{ccl} \displaystyle{\frac{dS(t)}{dt}} &=& -r_1 \, S(t)I(t)  + I_1(t) + I_2(t) \\ \\
\displaystyle{\frac{dI(t)}{dt}}  &=& r_1\, S(t)I(t) - r_2 \, I(t) + I_3(t)  \\  \\
\displaystyle{\frac{dR(t)}{dt}}  &=&   r_2 \, I(t)  - O_1(t).
\end{array}
\end{equation}
Here $I_1(t),I_2(t),I_3(t)$ and $O_1(t)$ are arbitrary smooth functions of time, which describe inflows and outflows at the points $i_1,i_2,i_3 \in X$ and $o_1 \in Y$.
If we drop these inflow and outflow terms, we obtain an autonomous system of coupled nonlinear first-order ordinary differential equations.   In fact these equations are a famous model of infectious disease, the `SIR model', where $S(t)$, $I(t)$ and $R(t)$ describe the populations of susceptible, infected and recovered individuals, respectively.   The inflow and outflow terms allow individuals to enter or leave the population.   This in turn lets us couple the SIR model to other models, and build complex models from smaller pieces.  
Recently Halter and Patterson \cite{HP} implemented this idea in software using structured cospans.  They illustrated the use of this tool by rebuilding part of the COVID-19 model that the UK has been using to make policy decisions.

Now we turn to the details.  A \define{Petri net with rates} is a Petri net $s,t \maps T \to \N[S]$ together with a function $r \maps T \to (0,\infty)$ assigning to each transition $\tau \in T$ a positive real number called its \define{rate constant}.  There is a category $\Petri_r$ whose objects are Petri nets with rates, where a morphism from 
\[   \xymatrix{ (0,\infty) & T \ar[l]_-r \ar@<-.5ex>[r]_-t \ar@<.5ex>[r]^-s & \N[S] }\]
 to 
 \[   \xymatrix{ (0,\infty) & T' \ar[l]_-r \ar@<-.5ex>[r]_-t \ar@<.5ex>[r]^-s & \N[S'] }\]
is a morphism of the underlying Petri nets such that the following diagram also commutes:
\[
\begin{tikzpicture}[scale=1.5]
\node (E) at (3,-0.5) {$(0,\infty)$};
\node (G) at (4,0) {$T$};
\node (G') at (4,-1) {$T'$};
\path[->,font=\scriptsize,>=angle 90]
(G) edge node[right]{$f$} (G')
(G) edge node[above]{$r$} (E)
(G') edge node[below]{$r'$} (E);
\end{tikzpicture}
\]
There is a functor $R \maps \Petri_r \to \Set$ sending any Petri net with rates to its set of places, and this has a left adjoint $L \maps \Set \to \Petri_r$ sending any set $S$ to the Petri net with $S$ as its set of species and no transitions \cite[Lemma 6.18]{BC}.   Since $\Petri_r$ has finite colimits \cite[Lemma 6.19]{BC}, it follows that there is a symmetric monoidal double category ${}_L \lCsp(\Petri_r)$ where:
\begin{itemize}
\item objects are finite sets,
\item vertical 1-morphisms are functions,
\item horizontal 1-cells are \define{open Petri nets with rates}, namely diagrams in $\Petri_r$ of the form
\[
\begin{tikzpicture}[scale=1.5]
\node (D) at (-3,0) {$L(X)$};
\node (E) at (-2,0) {$P$};
\node (F) at (-1,0) {$L(Y)$,};
\path[->,font=\scriptsize,>=angle 90]
(D) edge node [above] {$i$} (E)
(F) edge node [above] {$o$} (E);
\end{tikzpicture}
\]
\item 2-morphisms are diagrams in $\Petri_r$ of the form
\[
\begin{tikzpicture}[scale=1.5]
\node (E) at (3,0) {$L(X_1)$};
\node (F) at (5,0) {$L(Y_1)$};
\node (G) at (4,0) {$P_1$};
\node (E') at (3,-1) {$L(X_2)$};
\node (F') at (5,-1) {$L(Y_2)$.};
\node (G') at (4,-1) {$P_2$};
\path[->,font=\scriptsize,>=angle 90]
(F) edge node[above]{$O_1$} (G)
(E) edge node[left]{$L(f)$} (E')
(F) edge node[right]{$L(g)$} (F')
(G) edge node[left]{$\alpha$} (G')
(E) edge node[above]{$I_1$} (G)
(E') edge node[above]{$I_2$} (G')
(F') edge node[above]{$O_2$} (G');
\end{tikzpicture}
\]
\end{itemize}

We can equivalently describe open Petri nets with rates using decorated cospans.  There is a symmetric lax monoidal pseudofunctor $F \maps (\Fin\Set, +) \to (\Cat, \times)$ such that for any finite set $S$, the category $F(S)$ has:
\begin{itemize}
\item objects given by Petri nets with rates whose set of places is $S$,
\item morphisms given by morphisms of Petri nets with rates that are the identity on the set of places.
\end{itemize}
This gives a symmetric monoidal double category $F \lCsp$, and using \cref{thm:equiv} we can show that this is equivalent, as a symmetric monoidal double category, to ${}_L \lCsp(\Petri_r)$.

All this so far is very similar to the previous examples.  More interesting is the symmetric monoidal double category of open dynamical systems.   A dynamical system is a vector field from the perspective of a system of first-order ordinary differential equation.  An Petri net with rates gives a special sort of dynamical system: an \define{algebraic} vector field on $\R^n$, meaning one where the components of the vector field are polynomials in the coordinates. We shall think of such a vector field as a special sort of function $v \maps \R^n \to \R^n$.

Using Fong's original approach to decorated cospans, Pollard and the first author constructed a symmetric monoidal category for which the morphisms are open dynamical systems \cite[Theorem 17]{BP}.  This category is constructed from a symmetric lax monoidal functor 
$D \maps \Fin\Set \to \Set$ such that:
\begin{itemize}
\item $D$ maps any finite set $S$ to 
\[ D(S) = \{ v \maps \R^S \to \R^S | \; v \textrm{ is algebraic}  \}. \]
\item $D$ maps any function $f \maps S \to S'$ between finite sets to the function $D(f) \maps D(S) \to D(S')$ given as follows:
\[ D(f)(v) = f_* \circ v \circ f^* \]
where the pullback $ f^* \maps \R^{S'} \to \R^S $ is given by
\[ f^*(c)(\sigma) = c(f(\sigma)) \] 
while the pushforward $ f_* \maps \R^{S} \to \R^{S'} $ is given by
\[ f_*(c)(\sigma') = \sum_{ \{ \sigma \in S : f(\sigma) = \sigma' \} } c(\sigma). \]
\end{itemize}
The functorality of $D$ is proved in \cite[Lemma 15]{BP} while the lax symmetric monoidal stucture is given in Lemma 16 of that paper.

Since every set gives a discrete category with that set of objects, we can reinterpret $D$ as a symmetric lax monoidal pseudofunctor $D \maps (\Fin\Set, +) \to (\Cat, \times)$ which happens to actually be a functor.  Applying \cref{DC} we obtain a symmetric monoidal double category $D \lCsp$ where:
\begin{itemize}
\item objects are finite sets,
\item vertical 1-morphisms are functions,
\item a horizontal 1-cell from $X$ to $Y$ is an \define{open dynamical system}, that is, a cospan
\[
\begin{tikzpicture}[scale=1.5]
\node (D) at (-3,0) {$X$};
\node (E) at (-2,0) {$S$};
\node (F) at (-1,0) {$Y$};
\path[->,font=\scriptsize,>=angle 90]
(D) edge node [above] {$i$} (E)
(F) edge node [above] {$o$} (E);
\end{tikzpicture}
\]
in $\Fin\Set$ together with an algebraic vector field $v \in D(S)$,
\item a 2-morphism from
\[
\begin{tikzpicture}[scale=1.5]
\node (D) at (-3,0) {$X$};
\node (E) at (-2,0) {$S$};
\node (F) at (-1,0) {$Y,$};
\node (G) at (0,0) {$v \in D(S)$};
\path[->,font=\scriptsize,>=angle 90]
(D) edge node [above] {$i$} (E)
(F) edge node [above] {$o$} (E);
\end{tikzpicture}
\]
to
\[
\begin{tikzpicture}[scale=1.5]
\node (D) at (-3,0) {$X'$};
\node (E) at (-2,0) {$S'$};
\node (F) at (-1,0) {$Y'$};
\node (G) at (0,0) {$v' \in D(S')$};
\path[->,font=\scriptsize,>=angle 90]
(D) edge node [above] {$i'$} (E)
(F) edge node [above] {$o'$} (E);
\end{tikzpicture}
\]
is a diagram
\[
\begin{tikzpicture}[scale=1.5]
\node (E) at (3,0) {$X$};
\node (F) at (5,0) {$Y$};
\node (G) at (4,0) {$S$};
\node (E') at (3,-1) {$X'$};
\node (F') at (5,-1) {$Y'$};
\node (G') at (4,-1) {$S'$};
\path[->,font=\scriptsize,>=angle 90]
(F) edge node[above]{$o$} (G)
(E) edge node[left]{$f$} (E')
(F) edge node[right]{$g$} (F')
(G) edge node[left]{$h$} (G')
(E) edge node[above]{$i$} (G)
(E') edge node[above]{$i'$} (G')
(F') edge node[above]{$o'$} (G');
\end{tikzpicture}
\]
in $\Fin\Set$ such that $D(h)(v) = v'$.
\end{itemize}

Next, we can define a symmetric monoidal double functor
\[      \graysquare \maps F \lCsp \to D \lCsp \]
sending any open Petri net with rates to its corresponding open dynamical system.  This was already defined at the level of categories by Pollard and the first author \cite[Section 7]{BP}, who called it `gray-boxing'.   To boost this result to the double category level we use \cref{thm:functoriality}, taking the square in that theorem to be
\[
\begin{tikzpicture}[scale=1.5]
\node (A) at (0,0) {$\Fin\Set$};
\node (B) at (1,0) {$\Cat$};
\node (C) at (0,-1) {$\Fin\Set$};
\node (D) at (1,-1) {$\Cat$.};
\node (E) at (0.5,-0.5) {$\Downarrow \theta$};
\path[->,font=\scriptsize,>=angle 90]
(A) edge node[above]{$F$} (B)
(A) edge node[left]{$1$} (C)
(B) edge node[right]{$1$} (D)
(C) edge node[above]{$D$} (D);
\end{tikzpicture}
\] 
Here $\theta$ is a monoidal natural isomorphism given as follows.  For any finite set $S$, $\theta_S \maps F(S) \to D(S)$ maps any Petri nets with rates 
\[   \xymatrix{ (0,\infty) & T \ar[l]_-r \ar@<-.5ex>[r]_-t \ar@<.5ex>[r]^-s & \N[S] }\]
to an algebraic vector field on $\R^S$, say $v$.   This vector field is defined using a standard prescription taken from chemistry, called `the law of mass action'.   Namely, for any $c \in \R^s$, we set
\[  
v(c) = \sum_{\tau \in T} r(\tau) \, ( t(\tau) - s(\tau) ) c^{s(\tau)} 
\]
where 
\[     c^{s(\tau)} = \prod_{i \in S} {c_i}^{s(\tau)_i}  \]
and we think of $t(\tau), s(\tau) \in \N[S]$ as vectors in $\R^S$.   This formula is explained in the paper with Pollard, where it is also shown that $\theta$ defines a monoidal natural isomorphism between functors to $(\Set,\times)$ \cite[Theorem 18]{BP}.   As such, it automatically becomes a monoidal natural isomorphism between the functors $F, D \maps (\Fin\Set,+) \to (\Cat,\times)$.

Given a Petri net with rates $P$ and defining $v$ as above, this system of first-order ordinary differential equations for a function $c \maps \R \to \R^S$:
\[    \frac{d}{dt} c(t) = v(c(t))   \]
is called the \define{rate equation}.    When $P$ is part of an open Petri net with rates
\[
\begin{tikzpicture}[scale=1.5]
\node (D) at (-3,0) {$L(X)$};
\node (E) at (-2,0) {$P$};
\node (F) at (-1,0) {$L(Y)$,};
\path[->,font=\scriptsize,>=angle 90]
(D) edge node [above] {$i$} (E)
(F) edge node [above] {$o$} (E);
\end{tikzpicture}
\]
we get an open dynamical system called the \define{open rate equation}:
\[     \frac{d}{dt} c(t) = v(c(t))  + i_*(I(t)) - o_*(O(t)) \]
where $I \maps \R \to \R^X$ and $O \maps \R \to \R^X$ are arbitrary smooth functions describing \define{inflows} and \define{outflows}, respectively.   Applying this prescription to the open Petri net with rates shown at the start of this section one gets the differential equations \cref{eq:openPetrir}.  Other examples are worked out in \cite{BP}.

We now show that decorated cospan double category $D \lCsp$ of open dynamical systems is not equivalent to a structured cospan double category via \cref{thm:equiv}.  Recall that in that theorem we start with the data required to build a decorated cospan category, namely a symmetric lax monoidal pseudofunctor $F \maps (\A,+) \to (\Cat,\times)$, and show that under certain conditions the resulting opfibration $U \maps \X = \inta F \to \A$ has a left adjoint $L \maps \A \to \X$.   We then obtain an equivalence between decorated and structured cospan double categories, $F \lCsp \simeq {}_L \Csp(\X)$.   We now show that in the case at hand, where $F = D$,  the opfibration $U$ does \emph{not} have a left adjoint.  Thus, the conditions of \cref{thm:equiv} do not hold in this case.

Taking $D$ as above, it is easy to see that in the category $\inta D$
\begin{itemize}
\item an object is a pair $(S,v)$ where $S$ is a finite set and $v$ is an algebraic vector field $v \maps \R^S \to \R^S$,
\item a morphism from $(S,v)$ to $(S',v')$ is a function $f \maps S \to S'$ such that $v' = f_* \circ v \circ f^*$.
\end{itemize}
with the usual composition of functions.   The forgetful functor $U \maps \inta D \to \Fin\Set$ acts as follows:
\begin{itemize}
\item on objects, $D(S,v) = S$,
\item on morphisms, $D(f) = f$.
\end{itemize}

To show that $U$ does not have a left adjoint, we use the following well-known result:
\begin{lem} \label{lem:initial}
A functor $U \maps \A \to \X$ admits a left adjoint if and only if for every $x \in \X$, the comma category $x \downarrow U$ has an initial object.
\end{lem}
Because the empty set is initial in $\Fin\Set$, the comma category 
$\emptyset \downarrow U$ is just $\inta D$.  This contains an object $(\emptyset, v_\emptyset)$, where $v_\emptyset$ is the only possible vector field on $\R^\emptyset$, namely, the zero vector field.   The only object in $\inta D$ with any morphisms to $(\emptyset, v_\emptyset)$ is $(\emptyset, v_\emptyset)$ itself, so no other object can be initial.  However $(\emptyset, v_\emptyset)$ is not initial either, because it has no morphisms to an object $(S,v)$ unless $v$ is the zero vector field on $\R^S$.  Thus by \cref{lem:initial}, $U$ does not have a left adjoint.

\section{Conclusions}\label{sec:conclusions}

We have given conditions under which a decorated cospan double category is isomorphic to a structured cospan double category, in \cref{thm:equiv}. The converse question is also interesting: is every structured cospan double category equivalent to a decorated cospan double category? The answer is similar to the previous one: yes, under certain conditions that let us pass from an appropriate functor $L \maps \A \to \X$ to an appropriate pseudofunctor $F \maps \A \to \Cat$.

Let us now sketch the story; details will appear in a forthcoming paper \cite{CV}.  Suppose the conditions hold for constructing the double category of structured cospans ${}_L \lCsp(\X)$ as in \cref{thm:SC}.  That is, suppose $\A$ and $\X$ have finite colimits and $L \maps \A \to \X$ preserves them.    If $L$ also has a right adjoint `left inverse' (meaning the unit is the identity) $U \maps \X \to \A$, which moreover strictly preserves the chosen pushouts, it can be shown that $U$ is an opfibration.  Consequently, $U$ corresponds to a pseudofunctor $F \maps \A \to \Cat$ by the inverse Grothendieck construction, as in the first part of \cref{thm:Grothendieck}. Furthermore, if $U$ preserves finite coproducts, $F$ acquires the structure of a lax monoidal pseudofunctor $F \maps (\A,+) \to (\Cat,\times)$ by the special case of the cocartesian monoidal Grothendieck construction discussed under \cref{lem:MonGroth}.  As a result, $F$ has now enough structure to induce a double category of decorated cospans $F\lCsp$  as in \cref{DC}.  Finally, it can be shown that the structured and decorated cospan double categories are equivalent as symmetric monoidal double categories: ${}_L \lCsp(\X) \simeq F\lCsp$.

To give a better sense of how the pseudofunctor $F \maps \A \to\Cat$ is constructed: for each object $a \in A$, $F(a)$ is defined to be the fiber of $U$ over $a$, namely the category of all objects in $x \in \X$ such that $U(x)=a$ and morphisms $k\maps x\to y$ such that $U(k)=1_a$.  Given a morphism $f \maps a \to b$, there is a functor $F(f) \maps F(a)\to F(b)$ that maps $x \in F(a)$ to the following pushout:
\begin{displaymath}
 \begin{tikzcd}
La\ar[r,"Lf"]\ar[d,"\varepsilon_x"']\ar[dr,phantom,near end,"\ulcorner"] & Lb\ar[d] & \\
x\ar[d,dotted]\ar[r] & x+_{La}Lb\ar[d,dotted] & \textrm{in }\X \\
a\ar[r,"f"'] & b & \textrm{in }\A
 \end{tikzcd}
\end{displaymath}
where $\varepsilon_x\colon LU(x)=L(a)\to x$ is the counit of the adjunction $L\dashv U$. The fact that $U$ strictly preserves pushouts is necessary to show that the pushout is mapped, via $U$, directly down to $b$.

Even though for both \cref{thm:equiv} and the above result the conditions stated are only sufficient, they suggest that with work we could establish this functorial picture:
\begin{displaymath}
\begin{tikzpicture}
\node [align=center,draw] (A) { Lax monoidal pseudofunctors \\ $(\A,+)\to(\Cat,\times)$};
\node[below right=1 and 1.5 of A,align=center,draw] (C)  { Symmetric monoidal \\double categories };
\node[below left=1.05 and .5  of A,draw] (D)  {Special opfibrations};
\node[below left=2 and 1 of A,draw] (E)  {Special laris};
\node[below=3 of A,align=center,draw] (B)  {Finite colimit preserving \\ functors $A\to\X$};
\node[below=0 of D] {\rotatebox{90}{$\simeq$}};
\path[->,font=\scriptsize]
(A) edge node[above,sloped, midway]{$\quad F\mapsto F\lCsp$}  (C)
(D) edge (A)
(E) edge (B)
(B) edge node[below,sloped, midway]{$\quad L\mapsto {}_{L}\lCsp(\X)$} (C);
\end{tikzpicture}
\end{displaymath}
with a natural isomorphism in the middle.  The connection between opfibrations and laris goes back to Gray's \cref{prop:opfibtolari}, but we need to specialize it to a class suitable for both the structured and decorated cospan constructions.  This would imply that starting from an appropriate middle ground, these two constructions are essentially the same.   We leave such considerations for future work.

\appendix

\section{Definitions}
In this appendix, we gather some well-known concepts required to make the material self-contained, as well as references to more detailed expositions.

\subsection{Bicategories}
\label{subsec:bicats}

For standard 2-categorical material, we refer the reader to \cite{KS}.  For monoidal 2-categories see \cite{DS}, and for detailed definitions concerning monoidal bicategories see \cite{GPS,McCrudden,Stay}.  Briefly, a \define{monoidal} bicategory $\bA$ comes with a pseudofunctor $\otimes\maps\bA\times\bA\to\bA$ and a unit object $I$ that are associative and unital up to coherent equivalence. A \define{braided} monoidal bicategory also comes with a pseudonatural equivalence $\beta_{a,b}\maps a\ot b\to b\ot a$ and appropriate invertible modifications obeying certain equations; it is \define{sylleptic} if there is an invertible modification $1_{a\ot b}\Rrightarrow\beta_{b,a}\circ\beta_{a,b}$ its own equation, \define{symmetric} if one further axiom holds. 

A \define{lax monoidal} pseudofunctor (called \emph{weak monoidal} homomorphism in earlier references) between monoidal bicategories $F\maps\bA\to\bB$ is a pseudofunctor equipped with pseudonatural transformations with components $\phi_{a,b}\maps Fa\otimes Fb\to F(a\otimes b)$ and $\phi_0\maps I\to FI$ along with invertible modifications for associativity and unitality with components 
\begin{equation}\label{eq:omega}
\begin{tikzcd}[column sep=.8in, row sep=.3in]
(F a\ot F b)\ot F c\ar[ddr,phantom,"{\cong}"]\ar[r,"{\phi_{a,b}\ot1}"]\ar[d,sloped,"\sim"'] & F (a\ot b)\ot F 
c\ar[d,"{\phi_{a\ot b,c}}"] \\
F a\ot( F b\ot F c)\ar[d,"{1\ot\phi_{b,c}}"'] & 
F ((a\ot b)\ot c)\ar[d,sloped,"\sim"] \\
F a\ot F(b\ot c)\ar[r,"{\phi_{a,b\ot c}}"'] &
F (a\ot(b\ot c))
\end{tikzcd}
\end{equation}
\begin{equation}\label{eq:unitality}
\begin{tikzcd}[column sep=.3in,row sep=.3in]
F a\ar[r,"\sim"]\ar[drr,sloped,"\sim"'] & F a\otimes I\ar[r,"1\otimes\phi_0"]\ar[dr,phantom,"\cong"] & F 
a\otimes F I\ar[d,"\phi_{a,I}"] \\
&& F(a\ot I)
\end{tikzcd}\qquad
\begin{tikzcd}[column sep=.3in,row sep=.3in]
F a\ar[r,"\sim"]\ar[drr,sloped,"\sim"'] & I\otimes F a \ar[r,"\phi_0\otimes1"]\ar[dr,phantom,"\cong"] & F 
I\otimes F a\ar[d,"\phi_{I,a}"] \\
&& F(I\ot a)
\end{tikzcd}
\end{equation}
subject to coherence conditions listed in \cite[Definition 2]{DS}.
In particular, pseudonaturality of the monoidal structure means that it comes with isomorphisms of this form:
\begin{equation}\label{eq:pseudonaturality}
\begin{tikzcd}[column sep=.5in]
 Fa\ot Fb\ar[r,"Ff\ot Fg"]\ar[d,"\phi_{a,b}"']\ar[dr,phantom,"\stackrel{\phi_{f,g}}{\cong}"] & Fa'\ot Fb'\ar[d,"\phi_{a',b'}"] \\
 F(a\ot b)\ar[r,"F(f\ot g)"'] & F(a'\ot b')
 \end{tikzcd}
\end{equation}
 natural in $f$ and $g$.
A \define{braided lax monoidal} pseudofunctor between braided monoidal bicategories comes with an invertible modification with components
\begin{equation}\label{eq:braidedpseudofun}
 \begin{tikzcd}
Fa\otimes Fb\ar[r,"\phi_{a,b}"]\ar[d,"\beta_{Fa,Fb}"']\ar[dr,phantom,"\stackrel{u_{a,b}}{\cong}"] & F(a\ot b)\ar[d,"F(\beta_{a,b})"] \\
Fb\otimes Fa\ar[r,"\phi_{b,a}"'] & F(b\ot a)
 \end{tikzcd}
\end{equation}
subject to two axioms found e.g. in \cite[Definition 14]{DS}.
A \define{sylleptic lax monoidal} pseudofunctor satisfies one extra condition and a \define{symmetric lax monoidal} pseudofunctor between symmetric monoidal bicategories is just a sylleptic one. 


\subsection{Fibrations}\label{sec:fibrations}

Basic material regarding the theory of fibrations can be found, for example, in \cite{Borc,Gray}. Recall that a functor $U \maps \X \to \A$ is an \textbf{opfibration} if for every $x\in\X$ with $U(x)=a$ and $f \maps a \to b$ in $\A$, there exists a \textbf{cocartesian lifting} of $f$ to $x$, namely a morphism $\beta$ in $\X$ with domain $x$ with $U(\beta) = f$ and the following universal property: for any $g\maps b\to b'$ in $\A$ and $\gamma\maps x\to y'$ in $\X$ above the composite $g\circ f$, there exists a unique $\delta\maps y\to y'$ such that $U(\delta)=g$ and $\gamma=\delta\circ\beta$ as shown below
\begin{displaymath}
\xymatrix @R=.1in @C=.6in
{&& y'\ar @{.>}@/_/[dd] &&\\
x\ar[r]_-{\beta} \ar @{.>}@/_/[dd]
\ar[urr]^-{\gamma} & 
y \ar @{.>}@/_/[dd] \ar @{-->}[ur]_-{\exists! \delta}
&& \textrm{in }\X\\
&& b' &&\\
a\ar[r]_-{f=U(\beta)} \ar[urr]^-{g\circ f=U(\gamma)}
 & b \ar[ur]_-g && \textrm{in }\A}
\end{displaymath}
The category $\X$ is called the \textbf{total} category and $\A$ is called the \textbf{base} category of the opfibration. For any $a\in\A$, the \textbf{fiber} above $a$ is the category $\X_a$ consisting of all objects that map to $a$ and \define{vertical} morphisms between them, i.e., morphisms mapping to $1_a$.   

Assuming the axiom of choice, we may select a cocartesian lifting of each morphism $f\maps a\to b$ in $\A$ to each $x\in\X _a$, denoted by $\mathrm{Cocart}(f,x)\maps x\to f_!(x)$, rendering $U$ a so-called \textbf{cloven} opfibration. This choice induces \textbf{reindexing functors} $f_!\maps\X _a\to\X _b$ between the fibers, which by the lifting's universal property come equipped with natural isomorphisms $(1_a)_!\cong 1_{\X _a}$ and $(f\circ g)_!\cong f_!\circ g_!$.   With the help of these, any cloven opfibration $U \maps \X \to \A$ gives a pseudofunctor $F \maps \A \to \Cat$, where $\A$ viewed as a 2-category with trivial 2-cells, $F(a) = \X_a$ for each object $a \in \A$, and $F(f) = f_!$ for each morphism $f$ in $\A$.

In fact, there is a 2-equivalence between opfibrations and pseudofunctors induced by the so-called `Grothendieck construction'.  Let $\OpFib(\A)$ denote the 2-subcategory of the slice 2-category $\Cat/ \A$ of opfibrations over $\A$, functors that preserve cocartesian liftings, and natural transformations with vertical components.  
\begin{defn}\label{def:GrothCat}
For any pseudofunctor $F\maps\A\to\Cat$ where $\A$ is a category viewed as a 2-category with trivial 2-cells, the \textbf{Grothendieck category}
$\inta F$ has
\begin{itemize}
\item objects pairs $(a, x \in F(a))$ and
\item a morphism from $(a, x \in F(a))$ to $(b, y\in F(b))$ is a pair $(f \maps a \to b,k \maps F(f)(x) \to y)\in\A\times F(b)$.
\end{itemize}
The unit is $(1_a\maps a\to a,F(1_a)(x)\cong x)$ and composition of $(f,k)\maps(a,x)\to(b,y)$ and $(g,\ell)\maps(b,y)\to(c,z)$ is
\begin{equation}\label{eq:compGrothcat}
\left(a\xrightarrow{f}b\xrightarrow{g}c, F(g\circ f)x\cong Fg(Ff(x))\xrightarrow{Fg(k)}Fg(y)\xrightarrow{\ell}z\right) 
\end{equation}
This is an opfibered category over $\A$ via the obvious forgetful functor, with fibers $(\inta F)_a=F(a)$ and reindexing functors $f_!=F(f)$.
\end{defn}
The constructions sketched so far---the Grothendieck construction and the construction of a pseudofunctor into $\Cat$ from a cloven opfibration---are the two halves of the following equivalence.

\begin{thm}\label{thm:Grothendieck}\hfill
\begin{enumerate}
\item Every opfibration $\X \to \A$ gives rise to a pseudofunctor $\A \to \Cat$.
\item Every pseudofunctor $\A \to \Cat$ gives rise to an opfibration $\inta F \to\A$.
\item The above correspondences yield an equivalence of 2-categories 
\begin{displaymath}
[\A,\Cat]_\pse \simeq \OpFib(\A)
\end{displaymath}
where $[\A,\Cat]_\pse$ is the 2-category of pseudofunctors from $\A$ to $\Cat$, pseudonatural transformations, and modifications.
\end{enumerate}
\end{thm}

\begin{proof}  
The idea goes back to Grothendieck; a proof can be found in, for example, \cite[Section 1.10]{Jacobs}.
\end{proof}

\subsection{Double categories}\label{sec:doublecats}

For double categories we follow the notation of our paper on structured cospans \cite{BC}, which in turn follows that of Hansen and Shulman \cite{HS,Shulman2010}. 

\begin{defn}\label{defn:double_category}
A \define{double category} $\lD$ consists of a \define{category of objects}
$\lD_0$, a \define{category of arrows} $\lD_1$, functors 
\[    U\maps \lD_0 \to \lD_1, \; S,T \maps \lD_1 \to \lD_0 , \textrm{ and }
   \odot \maps \lD_1 \times_{\lD_0} \lD_1 \to \lD_1\]
such that 
\[  S(U_{A})=A=T(U_{A}),  \quad S(M \odot N)=S(N), \quad T(M \odot N)=T(M), \]
and natural isomorphisms called the \define{associator}
\[ \alpha_{L,M,N} \maps (L \odot M) \odot N \to L \odot (M \odot N) , \]
and \define{left and right unitors}
\[		\lambda_N \maps U_{T(N)} \odot N \to N, \qquad
     \rho_N \maps N \odot U_{S(N)} \to N \]
such that $S(\alpha), S(\lambda), S(\rho), T(\alpha), T(\lambda)$ and $T(\rho)$ are all identities,
such that the standard coherence laws hold: the pentagon identity for the 
associator and the triangle identity for the left and right unitor.
\end{defn}

Objects of $\lD_0$ are called \define{objects} and morphisms of $\lD_0$ are called \define{vertical 1-morphisms}. Objects of $\lD_1$ are called \define{horizontal 1-cells} and morphisms of $\lD_1$ are called \define{2-morphisms}.   We can draw a 2-morphism $a \maps M \to N$ with $S(a)=f,T(a)=g$ as follows:
\[
\begin{tikzcd}
A\ar[r,"M"]\ar[d,"f"']\ar[dr,phantom,"\scriptstyle\Downarrow\alpha"] & B\ar[d,"g"] \\
C\ar[r,"N"'] & D
\end{tikzcd}
\]
%Christina: changed the code above to match the squares later better
We call $M$ and $N$ the \define{horizontal source and target} of $a$ respectively, and call $f$ and $g$ its \define{vertical source and target}.   A 2-morphism where $f$ and $g$ are identities is called \textbf{globular}.   For example, the associator and unitors in a double category are globular 2-morphisms.


\begin{defn}\label{def:doublefun}
Given double categories $\lD$ and $\lE$, a \define{double functor} $\lF \maps \lD \to \lE$ consists of:
\begin{itemize}
\item{functors $\lF_0 \maps \lD_0 \to \lE_0$ and $\lF_1 \maps \lD_1 \to \lE_1$ such that $S \lF_1 = \lF_0 S$ and $T \lF_1 = \lF_0 T$, and}
\item{for every composable pair of horizontal 1-cells $M$ and $N$ in $\lD$, a natural transformation $\phi_{M,N} \maps \lF(N) \odot \lF(M) \to \lF(N \odot M)$ called the \define{composite comparison} and for every object $a$ in $\lD$, a natural transformation $\phi_a \maps U_{\lF_0(a)} \to \lF_1(U_a)$ called the \define{unit comparison}. The components of each of these natural transformations are globular isomorphisms which satisfy coherence axioms analogous to those of a monoidal functor.}
\end{itemize}
\end{defn}

\begin{defn}
Given double functors $(\lF,\phi),(\mathbb{G},\psi) \maps \lD \to \lE$, a \define{double natural transformation} $\alpha \maps \lF \Rightarrow \mathbb{G}$ consists of natural transformations $\alpha_0 \maps \lF_0 \Rightarrow \mathbb{G}_0$ and $\alpha_1 \maps \lF_1 \Rightarrow \mathbb{G}_1$ such that:
\begin{itemize}
\item{$S(\alpha_M) = \alpha_{S(M)}$ and $T(\alpha_M) = \alpha_{T(M)}$ for all horizontal 1-cells $M$ of $\lD$,}
\item{$\alpha \circ \phi_{M,N} = \psi \circ (\alpha_M \odot \alpha_N)$ for all composable pairs $M$ and $N$ of horizontal 1-cells in $\lD$, and}
\item{$\alpha \circ \phi_a = \psi_a \circ \alpha$ for all objects $a$ of $\lD$.}
\end{itemize} 
%$S(\alpha_M) = \alpha_{S(M)}$ and $T(\alpha_M) = \alpha_{T(M)}$, and for two composable horizontal 1-cells $M$ and $N$ of $\lD$, $\alpha \circ \phi_{M,N} = \psi \circ (\alpha_M \odot \alpha_N)$ and $\alpha \circ \phi_a = \psi_a \circ \alpha$. 
The double natural transformation $\alpha$ is a \define{double natural isomorphism} if both $\alpha_0$ and $\alpha_1$ are natural isomorphisms.
\end{defn}


%%% Christina: Pretty sure the def below can be incorporated above, however got a bit lost in different notation. Kenny Fix?
%\begin{defn}
%A double functor $\lF \maps \lD \to \lE$ is \define{strong} if the comparison and unit constraints are globular isomorphisms, meaning that for each composable pair of horizontal 1-cells $M$ and $N$ we have a natural isomorphism $$\lF_{M,N} \maps \lF(M) \odot \lF(N) \xrightarrow{\sim} \lF(M \odot N)$$ and for each object $a \in \lD$ a natural isomorphism $$\lF_a \maps \hat{U}_{\lF(a)} \xrightarrow{\sim} \lF(U_a).$$
%\end{defn}

%Following the notation of Shulman \cite{Shulman2010}, given a double category $\lD$, we write $_f \lD_g(M,N)$ for the set of 2-morphisms in $\lD$ of the form:
%\[
% \xymatrix@-.5pc{
%   A \ar[r]|{|}^{M}  \ar[d]_f \ar@{}[dr]|{\Downarrow a}&
%    B\ar[d]^g\\
%    C \ar[r]|{|}_N & D
%  }
%\]

\begin{defn}\label{def:fullfaithful}
A double functor $\lF \maps \lD \to \lE$ is \define{full} (resp.\ \define{faithful}) if $\lF_0 \maps \lD_0 \to \lE_0$ is full (resp.\ faithful) and each map 
\[   lF_1 \maps _f \lD_g(M,N) \to {_{\lF(f)} \lE_{\lF(g)}}(\lF(M),\lF(N))\] 
is surjective (respectively, injective), where $_f \lD_g(M,N)$ is set of 2-morphisms in $\A$ whose horizontal source is $M$ and whose horizontal target is $N$, and similarly for ${_{\lF(f)} \lE_{\lF(g)}}(\lF(M),\lF(N))$.
\end{defn}

\begin{defn}\label{def:essentiallysurj}
A double functor $\lF \maps \lD \to \lE$ is \define{essentially surjective} if we can simultaneously make the following choices:
\begin{itemize}
\item For each object $x \in \lE$, we can find an object $a \in \lD$ together with a vertical 1-isomorphism $\alpha_x \maps \lF(a) \to x$, and
\item For each horizontal 1-cell $N \maps x_1 \to x_2$  of $\lE$, we can find a horizontal 1-cell $M \maps a_1 \to a_2$ of $\lD$ and a 2-isomorphism $a_{N} \colon \lF(M) \to N$ in $\lE$ as in the following diagram:
\[
 \begin{tikzcd}
\lF(a_1) \ar[r,"\lF(M)"]  \ar[d,"\alpha_{x_1}"'] \ar[dr,phantom,"\scriptstyle\Downarrow a_N"] & \lF(a_1) \ar[d,"\alpha_{x_2}"]\\
 x_1 \ar[r,"N"'] & x_2
 \end{tikzcd}
\]
\end{itemize}
\end{defn}

\begin{defn}
Two double functors $\lF \colon \lD \to \lE$ and $\lG \colon \lE \to \lD$ form a \define{double equivalence} if there exist double natural transformations $\mathbb{G} \lF \cong 1_\lD$ and $\lF \mathbb{G} \cong 1_\lE$.
\end{defn}

Like equivalence of categories, equivalence of double categories has an alternative characterization:

\begin{thm}{\cite[Theorem 7.8]{Shulman2008}}
\label{ShulDubEquiv}
A pseudo double functor $\lF \maps \lD \to \lE$ is part of a double equivalence if and only if it is full, faithful and essentially surjective.
\end{thm}

Let $\Dbl$ denote the 2-category of double categories, double functors and double transformations. One can check that $\Dbl$ has finite products, and in any 2-category with finite products we can define a `pseudomonoid', which is a categorified analogue of a monoid \cite{DS}. For example, a pseudomonoid in $\Cat$ is a monoidal category.   We can also define symmetric pseudomonoids, which in $\Cat$ are symmetric monoidal categories.

\begin{defn}
\label{defn:monoidal_double_category}
A \define{monoidal double category} is a pseudomonoid in $\Dbl$. 
\end{defn} 
\noindent
Explicitly, a monoidal double category is a double category $\lD$ with:
\begin{itemize}
\item monoidal structures on both $\lD_0$ and $\lD_1$, each with tensor product denoted $\otimes$, associator $a$, left unitor $\ell$ and right unitor $r$, such that if $I$ is the unit object of $\lD_0$ then $U_I$ is\footnote{\chris In reality, just isomorphic is OK!} the unit of $\lD_1$, and such that the source and target functors $S,T \maps \lD_1 \to \lD_0$ are strict monoidal, 
\item the structure of a double functor on $\otimes$:
that is, invertible globular 2-morphisms
\[ \chi \maps (M_2\otimes N_2)\odot(M_1\otimes N_1)  \xrightarrow{\sim}
(M_2\odot M_1)\otimes (N_2\odot N_1)\]
\[ \mu \maps U_{A\otimes B} \xrightarrow{\sim} U_A \otimes U_B\]
obeying a list of equations that can be found after \cite[Definition 2.10]{HS} and also \cite[Definition A.5]{BC}.
\end{itemize}

\begin{defn}
\label{defn:symmetric_monoidal_double_category}
A \define{symmetric monoidal double category} is a symmetric pseudomonoid in $\Dbl$. 
\end{defn}
\noindent
Explicitly, a symmetric monoidal double category is a monoidal double category $\lD$ such that:
\begin{itemize}
		\item $\lD_0$ and $\lD_1$ are symmetric monoidal categories, with braidings both denoted $\beta$.
		\item The functors $S$ and $T$ are symmetric strict monoidal functors.
		\item The following diagrams commute, expressing that the braiding is a transformation of double categories:
		\begin{equation}
		\begin{tikzpicture}\label{eq:braidingaxiom}
			\node (A) at (0,1.5) {\footnotesize{$ (M_2\otimes N_2)\odot (M_1\otimes N_1)$}};
			\node (A') at (0,0) {\footnotesize{$(M_2 \odot M_1) \otimes (N_2 \odot N_1)$}};
			\node (B) at (5,1.5) {\footnotesize{$ (N_2 \otimes M_2)\odot (N_1 \otimes M_1)$}};
			\node (B') at (5,0) {\footnotesize{$(N_2\odot N_1) \otimes (M_2\odot M_1)$}};
			%
			\path[->,font=\scriptsize]
				(A) edge node[left]{$\chi$} (A')
				(A) edge node[above]{$\beta \odot \beta$} (B)
				(B) edge node[right]{$\chi$} (B')
				(A') edge node[above]{$\beta$} (B');
		\end{tikzpicture}
		%
		\quad
		%
		\begin{tikzpicture}
			\node (A) at (0,1.5) {\footnotesize{$U_A \otimes U_B$}};
			\node (A') at (0,0) {\footnotesize{$U_B\otimes U_A$}};
			\node (B) at (2,1.5) {\footnotesize{$U_{A\otimes B} $}};
			\node (B') at (2,0) {\footnotesize{$U_{B\otimes A}$}};
			%
			\path[->,font=\scriptsize]
				(A) edge node[left]{$\beta$} (A')
				(B) edge node[above]{$\mu$} (A)
				(B) edge node[right]{$U_\beta$} (B')
				(B') edge node[above]{$\mu$} (A');
		\end{tikzpicture}
		\end{equation}
\end{itemize}


\begin{defn}\label{defn:monoidal_double_functor}
Given symmetric monoidal double categories $\lD$ and $\lE$, a \define{symmetric monoidal double functor} $\lF \colon \lD \to \lE$ is a double functor $\lF$ together with transformations $\theta \colon {\otimes \circ (\lF,\lF)} \to \lF \circ \otimes$ and $\theta_U \colon 1_\lE \to \lF \circ 1_\lD$ that satisfy the usual coherence axioms for a symmetric monoidal functor with respect to $\otimes$.
\end{defn}
\noindent
Unpacking this definition, one obtains a number of diagrams which can be found after \cite[Definition 2.4]{HS}.

WE NEED TO SAY BRIEFLY WHAT A SYMMETRIC MONOIDAL DOUBLE ISOMORPHISM, IF WE USE THIS CONCEPT IN OUR BIG RESULT, \cref{thm:equiv}.
CLAIM: $\lD$ and $\lE$ be symmetric monoidal double categories and let $\lF \maps \lD \to \lE$ be a symmetric monoidal double functor. If $\lF$ is part of a double equivalence, then $\lF$ is in fact part of a symmetric monoidal double equivalence, and $\lD$ and $\lE$ are equivalent as symmetric monoidal double categories.

\begin{defn}\label{def:companion}
  Let $\lD$ be a double category and $f\maps A\to B$ a vertical
  1-morphism.  A \textbf{companion} of $f$ is a horizontal 1-cell
  $\fhat\maps A\to B$ together with 2-morphisms
	\[
	\raisebox{-0.5\height}{
	\begin{tikzpicture}
		\node (A) at (0,1) {$A$};
		\node (B) at (1,1) {$B$};
		\node (A') at (0,0) {$B$};
		\node (B') at (1,0) {$B$};
		%
		\path[->,font=\scriptsize,>=angle 90]
			(A) edge node[above]{$\widehat{f}$} (B)
			(A) edge node[left]{$f$} (A')
			(B) edge node[right]{$1$} (B')
			(A') edge node[below]{$U_B$} (B');
		%
	%	\draw (0.5,.925) -- (0.5,1.075);
	%	\draw (0.5,-.075) -- (0.5,.075);
		\node () at (0.5,0.5) {\scriptsize{$\Downarrow$}};
	\end{tikzpicture}
	}
	%
	\quad \text{ and } \quad
	%
	\raisebox{-0.5\height}{
	\begin{tikzpicture}
		\node (A) at (0,1) {$A$};
		\node (B) at (1,1) {$A$};
		\node (A') at (0,0) {$A$};
		\node (B') at (1,0) {$B$};
		%
		\path[->,font=\scriptsize,>=angle 90]
			(A) edge node[above]{$U_A$} (B)
			(A) edge node[left]{$1$} (A')
			(B) edge node[right]{$f$} (B')
			(A') edge node[below]{ $\widehat{f}$} (B');
		%
	%	\draw (0.5,.925) -- (0.5,1.075);
	%	\draw (0.5,-.075) -- (0.5,.075);
		\node () at (0.5,0.5) {\scriptsize{$\Downarrow$}};
	\end{tikzpicture}
	}
	\]
  such that the following equations hold.
	\begin{equation}
	\label{eq:CompanionEq}
	\raisebox{-0.5\height}{
	\begin{tikzpicture}
		\node (A) at (0,2) {$A$};
		\node (B) at (1.1,2) {$A$};
		\node (A') at (0,1) {$A$};
		\node (B') at (1.1,1) {$B$};
		\node (A'') at (0,0) {$B$};
		\node (B'') at (1.1,0) {$B$};
		%
		\path[->,font=\scriptsize,>=angle 90]
			(A) edge node[left]{$1$} (A')
			(A') edge node[left]{$f$} (A'')
			(B) edge node[right]{$f$} (B')
			(B') edge node[right]{$1$} (B'')
			(A) edge node[above]{$U_A$} (B)
			(A') edge  (B')
			(A'') edge node[below]{$U_B$} (B'');
		%
	%	\draw (0.5,1.925) -- (0.5,2.075);
		\draw[line width=2mm,white] (0.5,.925) -- (0.5,1.075);
	%	\draw (0.5,-.075) -- (0.5,.075);
		\node () at (0.5,0.5) {\scriptsize{$\Downarrow$}};
		\node () at (0.5,1.5) {\scriptsize{$\Downarrow$}};
		\node () at (0.5,1) {\scriptsize $\widehat{f}$};
	\end{tikzpicture}
	}
	%
	\raisebox{-0.5\height}{=}
	%
	\raisebox{-0.5\height}{
	\begin{tikzpicture}
		\node (A) at (0,1) {$A$};
		\node (B) at (1,1) {$A$};
		\node (A') at (0,0) {$B$};
		\node (B') at (1,0) {$B$};
		%
		\path[->,font=\scriptsize,>=angle 90]
		(A) edge node[left]{$f$} (A')
		(B) edge node[right]{$f$} (B')
		(A) edge node[above]{$U_A$} (B)
		(A') edge node[below]{$U_B$} (B');
		%
		%\draw (0.5,.925) -- (0.5,1.075);
		%\draw (0.5,-.075) -- (0.5,.075);
		\node () at (0.5,0.5) {\scriptsize{$\Downarrow U_f$}};
	\end{tikzpicture}
	}
	%
	\raisebox{-0.5\height}{\text{   and   }}
	%
	\raisebox{-0.5\height}{
	\begin{tikzpicture}
		\node (A) at (0,1) {$A$};
		\node (A') at (0,0) {$A$};
		\node (B) at (1.4,1) {$A$};
		\node (B') at (1.4,0) {$B$};
		\node (C) at (2.8,1) {$B$};
		\node (C') at (2.8,0) {$B$};
		\node (A'') at (0,-1) {$A$};
		\node (C'') at (2.8,-1) {$B$};
		%
		\path[->,font=\scriptsize,>=angle 90]
			(A) edge node[left]{$1$} (A')
			(B) edge node [left] {$f$} (B')
			(C) edge node[right]{$1$} (C')
			(A) edge node[above]{$U_A$} (B)
			(B) edge node[above]{$\hat{f}$} (C)
			(A') edge (B')
			(B') edge (C')
			(A'') edge node[below]{$\hat{f}$} (C'')
			(A') edge node[left]{$1$} (A'')
			(C') edge node[right]{$1$} (C'');
		%
	%	\draw (1.5,0.925) -- (1.5,1.075);
	%	\draw (1.5,0.925) -- (1.5,1.075);
	%	\draw (0.5,.925) -- (0.5,1.075);
	%	\draw (0.5,-.075) -- (0.5,.075);
	\draw[line width=2mm,white] (0.7,-.05) -- (0.7,.05);
	\draw[line width=4.4mm,white] (2.1,-.05) -- (2.1,.05);
%	\draw[line width=2mm,white] (1.4,.35) -- (1.4,.65);
		\node () at (0.7,0.5) {\scriptsize{$\Downarrow$}};
		\node () at (2.1,0.5) {\scriptsize{$\Downarrow$}};
		\node () at (1.4,-0.6) {\scriptsize{$\Downarrow \lambda_{\hat{f}}$}};
		\node () at (0.7,0) {$\hat{f}$};
%		\node () at (1.4,.5) {$\scriptstyle{f}$};
		\node () at (2.1,0) {\scriptsize{$U_B$}};
		
	\end{tikzpicture}
	}
	%
	\raisebox{-0.5\height}{=}
	%
	\raisebox{-0.5\height}{
	\begin{tikzpicture}
	     \node (A0) at (0,2) {$A$};
	     \node (B0) at (1,2) {$A$};
		\node (C0) at (2,2) {$B$};
		\node (A) at (0,1) {$A$};
		\node (C) at (2,1) {$B$};
	%	\node (A') at (0,0) {$a$};
	%	\node (C') at (2,0) {$b$};
		%
		\path[->,font=\scriptsize,>=angle 90]
			(A0) edge node[above]{$U_A$} (B0)
			(B0) edge node[above]{$\hat{f}$} (C0)
			(A0) edge node[left]{$1$} (A)
			(C0) edge node[right]{$1$} (C)
		%	(A) edge node[left]{$1$} (A')
		%	(C) edge node[right]{$1$} (C')
			(A) edge node[below]{$\hat{f}$} (C);
		%	(A') edge node[below]{$\hat{f}$} (C');
		%
	%	\draw (0.5,.925) -- (0.5,1.075);
	%	\draw (0.5,-.075) -- (0.5,.075);
	%	\node () at (1,0.5) {\scriptsize{$\Downarrow 1$}};
		\node () at (1,1.4) {\scriptsize{$\Downarrow \rho_f$}};
	\end{tikzpicture}
	}
	\end{equation}
  A \textbf{conjoint} of $f$, denoted $\fchk \maps B\to A$, is a
  companion of $f$ in the double category 
  obtained by reversing the horizontal 1-cells, but not the vertical
  1-morphisms, of $\lD$.
\end{defn}
\noindent

\begin{defn}
\label{defn:fibrant}
We say that a double category is \textbf{fibrant} if every vertical
1-morphism has both a companion and a conjoint.  
\end{defn}

\begin{thm}{\cite[Theorem 1.1]{HS}}
\label{Shulhorizontalbicat}
If $\lD$ is a fibrant monoidal double category, then its horizontal bicategory $\bD$ is a monoidal bicategory. If $\lD$ is braided or symmetric, then so is $\bD$. 
\end{thm}

\section{Checking a coherence law} \label{eq:proofofaxiom}

Here we show a sample check of a coherence law for the proof of \cref{DC}.
All relevant structure for decorated cospans is as described in \cref{thm:decorated_cospans,DC}. The coherence law that we check says that this diagram commutes:
\begin{equation}\label{eq:thisdiagram}
\begin{tikzpicture}[scale=1.5]
\node (A) at (0,0) {$(M_2 \otimes N_2) \odot (M_1 \otimes N_1)$};
\node (B) at (3,0) {$(N_2 \otimes M_2) \odot (N_1 \otimes M_1)$};
\node (C) at (0,-.75) {$(M_2 \odot M_1) \otimes (N_2 \odot N_1)$};
\node (D) at (3,-.75) {$(N_2 \odot N_1) \otimes (M_2 \odot M_1)$};
\path[->,font=\scriptsize,>=angle 90]
(A) edge node[above]{$\beta \odot \beta$} (B)
(B) edge node[right]{$\chi$} (D)
(A) edge node[left]{$\chi$} (C)
(C) edge node[above]{$\beta$} (D);
\end{tikzpicture}
\end{equation}
where $M_1,M_2,N_1,N_2$ are as in \cref{eq:4deccospans}, $\chi$ is as described right below therein and $\beta$ is the braiding defined as in \cref{eq:braidingFCsp1} -- although the letter $\beta$ is also used for the braiding in $\A$.
First of all, if we only consider the underlying cospans without their decorations, we obtain the following `flattened' commutative diagram in $\A$ -- which verifies the corresponding axiom for the symmetric monoidal double category $\lCsp(\A)$:
%%% Kenny any good reference for that? Not just cospans, which I can find in Niefield, but the interchange law as well for cospans only. Is it in your thesis? Somewhere in Shulman? Had a look but couldn't dig up anything...I know it is easy but would look nicer if we had some reference here
\[
		\begin{tikzpicture}
			\node (a) at (-4,0) {$a+a'$};
			\node (b) at (1,0) {$(m_1+n_1) +_{(b+b')} (m_2+n_2)$};
			\node (c) at (6,0) {$c+c'$};
			\node (a2) at (-4,1) {$a'+a$};
			\node (b2) at (1,1) {$(n_1+m_1) +_{(b'+b)} (n_2+m_2)$};
			\node (c2) at (6,1) {$c'+c$};
            \node (a3) at (-4,2) {$a'+a$};
			\node (b3) at (1,2) {$(n_1 +_{b'} n_2) + (m_1 +_b m_2)$};
			\node (c3) at (6,2) {$c'+c$};
            \node (a5) at (-4,-1) {$a+a'$};
			\node (b5) at (1,-1) {$(m_1+_b m_2) + (n_1+_{b'} n_2)$};
			\node (c5) at (6,-1) {$c+c'$};
            \node (a6) at (-4,-2) {$a'+a$};
			\node (b6) at (1,-2) {$(n_1 +_{b'} n_2) + (m_1 +_b m_2)$};
			\node (c6) at (6,-2) {$c'+c$};
			\path[->,font=\scriptsize,>=angle 90]
			(a) edge node[above]{$$} (b)
			(c) edge node[above]{$$} (b)
            (a2) edge node[above]{$$} (b2)
			(c2) edge node[above]{$$} (b2)
            (a) edge node[left]{$\beta$} (a2)
            (b) edge node[left]{$\beta +_\beta \beta$} (b2)
%(b) edge node[right]{$\tau_1$} (b2)
			(c) edge node[right]{$\beta$} (c2)
            (a3) edge node[above]{$$} (b3)
			(c3) edge node[above]{$$} (b3)
            (a2) edge node[left]{$1$} (a3)
            (b2) edge node[left]{$\hat{\chi}$} (b3)
%(b2) edge node[right]{$\tau_2$} (b3)
			(c2) edge node[right]{$1$} (c3)
            (a5) edge node[above]{$$} (b5)
			(c5) edge node[above]{$$} (b5)
            (a) edge node[left]{$1$} (a5)
            (b) edge node[left]{$\hat{\chi}$} (b5)
%(b) edge node[right]{$\tau_4$} (b5)
			(c) edge node[right]{$1$} (c5)
            (a6) edge node[above]{$$} (b6)
			(c6) edge node[above]{$$} (b6)
            (a5) edge node[left]{$\beta$} (a6)
            (b5) edge node[left]{$\beta$} (b6)
 %(b5) edge node[right]{$\tau_5$} (b6)
			(c5) edge node[right]{$\beta$} (c6);
		\end{tikzpicture}
	\]
The top and the bottom cospans coincide, and $\hat{\chi}$ is the canonical isomorphism by the universal property of colimits
%globular 2-morphism interchanging the tensor product and horizontal composition of the symmetric monoidal double category $\lCsp(\A)$,
and the inwards pointing morphisms are natural maps from the each cospan's feet to its apex. It follows that $\beta\hat{\chi}=\hat{\chi}(\beta+_\beta\beta)$ as the unique map between the involved colimits.

Regarding decorations, each of the above four maps of cospans has an associated map, which we label with $\tau_i$ for $i=1,2,3,4$, between the corresponding decorations:
\[
		\begin{tikzpicture}
%			\node (a) at (-4,0) {$a+a'$};
			\node (b) at (1,0) {$(m_1+n_1) +_{(b+b')} (m_2+n_2)$};
			\node (c) at (8,0) {$ (x_1\oplus y_1)\odot (x_2\oplus y_2)  \in F((m_1+n_1)+_{b+b'}(m_2+n_2))$};
%			\node (a2) at (-4,1) {$a'+a$};
			\node (b2) at (1,1) {$(n_1+m_1) +_{(b'+b)} (n_2+m_2)$};
			\node (c2) at (8,1) {$(y_1\oplus x_1)\odot(y_2\oplus x_2) \in F((n_1+m_1)+_{b'+b}(n_2+m_2))$};
%                            \node (a3) at (-4,2) {$a'+a$};
			\node (b3) at (1,2) {$(n_1 +_{b'} n_2) + (m_1 +_b m_2)$};
			\node (c3) at (8,2) {$(y_1 \odot y_2) \oplus (x_1 \odot x_2) \in F((n_1+_{b'} n_2)+(m_1+_b m_2))$};
%                                \node (a5) at (-4,-1) {$a+a'$};
			\node (b5) at (1,-1) {$(m_1+_b m_2) + (n_1+_{b'} n_2)$};
			\node (c5) at (8,-1) {$(x_1 \odot x_2) \oplus (y_1 \odot y_2) \in F((m_1+_b m_2)+(n_1+_{b'} n_2))$};
%                                \node (a6) at (-4,-2) {$a'+a$};
			\node (b6) at (1,-2) {$(n_1 +_{b'} n_2) + (m_1 +_b m_2)$};
			\node (c6) at (8,-2) {$(y_1 \odot y_2) \oplus (x_1 \odot x_2) \in F((n_1+_{b'} n_2)+(m_1+_b m_2))$};
			\path[->,font=\scriptsize,>=angle 90]
%			(a) edge node[above]{$$} (b)
%			(c) edge node[above]{$$} (b)
%                                (a2) edge node[above]{$$} (b2)
%			(c2) edge node[above]{$$} (b2)
%                                (a) edge node[left]{$\beta$} (a2)
                                (b) edge node[left]{$\beta +_\beta \beta$} (b2)
(b) edge node[right]{$\tau_1$} (b2)
%			(c) edge node[right]{$\beta$} (c2)
%                                (a3) edge node[above]{$$} (b3)
%			(c3) edge node[above]{$$} (b3)
%                               (a2) edge node[above]{$$} (a3)
                                (b2) edge node[left]{$\hat{\chi}$} (b3)
(b2) edge node[right]{$\tau_2$} (b3)
%			(c2) edge node[above]{$$} (c3)
%                                (a5) edge node[above]{$$} (b5)
%			(c5) edge node[above]{$$} (b5)
%                                (a) edge node[above]{$$} (a5)
                                (b) edge node[left]{$\hat{\chi}$} (b5)
(b) edge node[right]{$\tau_3$} (b5)
%			(c) edge node[above]{$$} (c5)
%                               (a6) edge node[above]{$$} (b6)
%			(c6) edge node[above]{$$} (b6)
%                                (a5) edge node[left]{$\beta$} (a6)
                                (b5) edge node[left]{$\beta$} (b6)
 (b5) edge node[right]{$\tau_4$} (b6);
%			(c5) edge node[left]{$\beta$} (c6);
		\end{tikzpicture}
	\]
and are explicitly appropriate isomorphisms of the form:
\[
\begin{tikzpicture}[scale=1.5]
\node (A) at (0,0) {$\tau_1 \colon F(\beta +_\beta \beta)((x_1\oplus y_1)\odot(x_2\oplus y_2)) \to (y_1\oplus x_1)\odot(y_2\oplus x_2)$};
\node (B) at (0,-.5) {$\tau_2 \colon F(\hat{\chi})((y_1 \oplus x_1)\odot(y_2 \oplus x_2)) \to (y_1\odot y_2) \oplus (x_1\odot x_2)$};
\node (C) at (0,-1) {$\tau_3 \colon F(\hat{\chi})((x_1 \oplus y_1)\odot(x_2 \oplus y_2)) \to (x_1\odot x_2) \oplus (y_1\odot y_2)$};
\node (D) at (0,-1.5) {$\tau_4 \colon F(\beta)((x_1\odot x_2)\oplus(y_1\odot y_2)) \to (y_1\odot y_2)\oplus(x_1\odot x_2)$};
\path[->,font=\scriptsize,>=angle 90]
(A) edge[color=white] node[above]{$$} (A);
%(B) edge node[right]{$\chi$} (D)
%(A) edge node[left]{$\chi$} (C)
%(C) edge node[above]{$\beta \odot \beta$} (D);
\end{tikzpicture}
\]
Firstly, performing the (vertical) composition of $\chi$ and $\beta$ of $\cref{eq:thisdiagram}$ gives the composite decoration $\tau_4\tau_3$ computed by the formula \cref{eq:tauofverticalcomposite} to be 
\begin{displaymath}
\scalebox{.8}{$F(\beta\hat{\chi})((x_1 \oplus y_1)\odot(x_2 \oplus y_2))\cong F(\beta)(F\hat{\chi})((x_1 \oplus y_1)\odot(x_2 \oplus y_2))\xrightarrow{F(\beta)(\tau_3)}F(\beta)((x_1\odot x_2) \oplus (y_1\odot y_2))\xrightarrow{\tau_4}(y_1 \odot y_2) \oplus (x_1 \odot x_2)$}
\end{displaymath}
Using the formulas for the involved decorations \cref{eq:bigeq1,eq:bigeq2}, as well as the decoration morphisms of $\chi$ \cref{eq:interchangedeco} and $\beta$ \cref{eq:decobraiding}, we form:
\[
\begin{tikzpicture}[scale=1.6]
%\node (A) at (3.85,0) {$\scriptstyle{\cong}$};
%\node (A') at (3.85,-1) {$\scriptstyle{\cong}$};
\node () at (12,-1) {$\scriptstyle\cong$};
\node (D) at (3,-0.5) {$\scriptstyle{\one}$};
\node (E) at (6.75,0.5) {$\scriptstyle{F(m_1 + n_1) \times F(m_2+n_2)}$};
\node () at (10.25,-1) {$\scriptstyle\stackrel[\cref{eq:braidedpseudofun}]{u}{\cong}$};
\node () at (7.875,0) {$\scriptstyle\stackrel[\cref{eq:interchangedeco}]{}{\cong}$};
\node (E') at (6.75,-0.5) {$\scriptstyle{F(m_1+m_2) \times F(n_1+n_2)}$};
\node (B) at (9,0.5) {$\scriptstyle{F((m_1+n_1)+(m_2+n_2))}$};
\node (B') at (9,-0.5) {$\scriptstyle{F(m_1+_b m_2)\times F(n_1+_{b'}n_2)}$};
\node (C) at (11.5,0.5) {$\scriptstyle{F((m_1+n_1)+_{b+b'}(m_2+n_2))}$};
\node (C') at (11.5,-0.5) {$\scriptstyle{F((m_1+_b m_2)+ (n_1+_{b'} n_2))}$};
\node (G) at (4.25,0.5) {$\scriptstyle{F(m_1) \times F(n_1) \times F(m_2) \times F(n_2)}$};
\node (G') at (4.25,-0.5) {$\scriptstyle{F(m_1) \times F(m_2) \times F(n_1) \times F(n_2)}$};
\node (G'') at (4.25,-1.5) {$\scriptstyle{F(n_1) \times F(n_2) \times F(m_1) \times F(m_2)}$};
\node (E'') at (6.75,-1.5) {$\scriptstyle{F(n_1+n_2) \times F(m_1+m_2)}$};
\node (B'') at (9,-1.5) {$\scriptstyle{F(n_1+_{b'}n_2) \times F(m_1+_b m_2)}$};
\node (C'') at (11.5,-1.5) {$\scriptstyle{F((n_1+_{b'}n_2)+(m_1+_b m_2))}$};
\node (X) at (3.4,0.2) {$\scriptscriptstyle{x_1 \times y_1 \times x_2 \times y_2}$};
\node (X') at (3.4,-0.35) {$\scriptscriptstyle{x_1 \times x_2 \times y_1 \times y_2}$};
\node (X'') at (3.4,-1.2) {$\scriptscriptstyle{y_1 \times y_2 \times x_1 \times x_2}$};
\path[->,font=\scriptsize,>=angle 90]
(B') edge node [fill=white] {$\beta$} (B'')
(E) edge node [above] {$\phi_{m_1+n_1,m_2+n_2}$} (B)
(E'') edge node [below] {$F\psi\times F\psi$} (B'')
(E') edge node [above] {$F\psi\times F\psi$} (B')
(B) edge node [above] {$F(\psi)$} (C)
(B'') edge node [below] {$\phi_{n_1+_{b'}n_2,m_1+_b m_2}$} (C'')
(B') edge node [above] {$\phi_{m_1+_b m_2,n_1+_{b'}n_2}$} (C')
(C) edge node [fill=white] {$F(\hat{\chi})$} (C')
(C') edge node [fill=white] {$F(\beta)$} (C'')
(G) edge node [above] {$\phi_{m_1,n_1} {\times} \phi_{m_2,n_2}$} (E)
(G'') edge node [below] {$\phi_{n_1,n_2} {\times} \phi_{m_1,m_2}$} (E'')
(G') edge node [above] {$\phi_{m_1,m_2} \times \phi_{n_1,n_2}$} (E')
(G) edge node [fill=white] {$1 \times \beta \times 1$} (G')
(G') edge node [fill=white] {$\beta$} (G'')
(D) edge[out=90,in=180] node [left,above] {$$} (G)
(D) edge[out=270,in=180] node [left,above] {$$} (G'')
(D) edge node [below] {$$} (G')
(C) edge[bend left=80] node [fill=white] {$F(\beta\hat{\chi})$} (C'')
;
\end{tikzpicture}
\]
%{\chris Kenny please see difference in diagrams (the rightmost part is not important, just to make the two decoration isomorphism have the same border). The blue is the old one. The new diagram \cref{eq:interchangedeco} should be the one we disagreed, namely its codomain. But also on the down part of the diagram, coming from the decoration morphism of the braiding \cref{eq:decobraiding}, mine is much ``easier''.}
%{\color{blue}
%\[
%\begin{tikzpicture}[scale=1.6]
%%\node (A) at (3.85,0) {$\scriptstyle{\cong}$};
%%\node (A') at (3.85,-1) {$\scriptstyle{\cong}$};
%\node (D) at (3,-0.5) {$\scriptstyle{\one}$};
%\node (E) at (6.75,0.5) {$\scriptstyle{F(m_1 + n_1) \times F(m_2+n_2)}$};
%\node () at (5.5,-1) {$\scriptstyle{\cong}$};
%\node () at (7.875,-1) {$\scriptstyle\stackrel{\cref{eq:braidedpseudofun}}{\cong}$};
%\node () at (10.25,-1) {$\scriptstyle{\cong}$};
%\node () at (7.875,0) {$\scriptstyle\stackrel{\cref{eq:interchangedeco}}{\cong}$};
%\node (E') at (6.75,-0.5) {$\scriptstyle{F(m_1+m_2) \times F(n_1+n_2)}$};
%\node (B) at (9,0.5) {$\scriptstyle{F((m_1+n_1)+(m_2+n_2))}$};
%\node (B') at (9,-0.5) {$\scriptstyle{F((m_1+ m_2)+(n_1+n_2))}$};
%\node (C) at (11.5,0.5) {$\scriptstyle{F((m_1+n_1)+_{b+b'}(m_2+n_2))}$};
%\node (C') at (11.5,-0.5) {$\scriptstyle{F((m_1+_b m_2)+ (n_1+_{b'} n_2))}$};
%\node (G) at (4.25,0.5) {$\scriptstyle{F(m_1) \times F(n_1) \times F(m_2) \times F(n_2)}$};
%\node (G') at (4.25,-0.5) {$\scriptstyle{F(m_1) \times F(m_2) \times F(n_1) \times F(n_2)}$};
%\node (G'') at (4.25,-1.5) {$\scriptstyle{F(n_1) \times F(n_2) \times F(m_1) \times F(m_2)}$};
%\node (E'') at (6.75,-1.5) {$\scriptstyle{F(n_1+n_2) \times F(m_1+m_2)}$};
%\node (B'') at (9,-1.5) {$\scriptstyle{F((n_1+n_2)+(m_1+ m_2))}$};
%\node (C'') at (11.5,-1.5) {$\scriptstyle{F((n_1+_{b'}n_2)+(m_1+_b m_2))}$};
%\node (X) at (3.4,0.2) {$\scriptscriptstyle{x_1 \times y_1 \times x_2 \times y_2}$};
%\node (X') at (3.4,-0.35) {$\scriptscriptstyle{x_1 \times x_2 \times y_1 \times y_2}$};
%\node (X'') at (3.4,-1.2) {$\scriptscriptstyle{y_1 \times y_2 \times x_1 \times x_2}$};
%\path[->,font=\scriptsize,>=angle 90]
%(B') edge node [fill=white] {$F(\beta)$} (B'')
%(E') edge node [fill=white] {$\beta$} (E'')
%(E) edge node [above] {$\phi_{m_1+n_1,m_2+n_2}$} (B)
%(E'') edge node [above] {$\phi_{n_1+n_2,m_1+m_2}$} (B'')
%(E') edge node [above] {$\phi_{m_1+m_2,n_1+n_2}$} (B')
%(B) edge node [above] {$F(\psi)$} (C)
%(B'') edge node [above] {$F(\psi+\psi)$} (C'')
%(B') edge node [above] {$F(\psi+\psi)$} (C')
%(C) edge node [fill=white] {$F(\hat{\chi})$} (C')
%(C') edge node [fill=white] {$F(\beta)$} (C'')
%(G) edge node [above] {$\phi_{m_1,n_1} {\times} \phi_{m_2,n_2}$} (E)
%(G'') edge node [above] {$\phi_{n_1,n_2} {\times} \phi_{m_1,m_2}$} (E'')
%(G') edge node [above] {$\phi_{m_1,m_2} \times \phi_{n_1,n_2}$} (E')
%(G) edge node [fill=white] {$1 \times \beta \times 1$} (G')
%(G') edge node [fill=white] {$\beta$} (G'')
%(D) edge[out=90,in=180] node [left,above] {$$} (G)
%(D) edge[out=270,in=180] node [left,above] {$$} (G'')
%(D) edge node [below] {$$} (G');
%\end{tikzpicture}
%\]
%}
For the other (vertical) composition of \cref{eq:thisdiagram}, that of $\beta\odot\beta$ followed by $\chi$, the composite decoration $\tau_2\tau_1$ is
\begin{displaymath}
\scalebox{.8}{$F(\hat{\chi}(\beta +_\beta \beta))( (x_1 \oplus y_1)\odot (x_2 \oplus y_2))\cong F(\hat{\chi})F(\beta +_\beta \beta)( (x_1 \oplus y_1)\odot (x_2 \oplus y_2)) \xrightarrow{F(\hat{\chi})(\tau_1)}F(\hat{\chi})(y_1\oplus x_1)\odot(y_2\oplus x_2)\xrightarrow{\tau_2}(y_1 \odot y_2) \oplus (x_1 \odot x_2)$}
\end{displaymath}
Once again, using the tensor and composite formulas for decorations, as well the horizontal composite formulas \cref{eq:decohorizontalcompo}
for $\beta\odot\beta$, the above is explicitly given by the pasted isomorphism:
\[
\begin{tikzpicture}[scale=1.6]
%\node (A) at (3.85,0) {$\scriptstyle{\cong}$};
%\node (A') at (3.85,-1) {$\scriptstyle{\cong}$};
\node (D) at (3,-0.5) {$\scriptstyle{\one}$};
\node (E) at (6.75,0.5) {$\scriptstyle{F(m_1 + n_1) \times F(m_2+n_2)}$};
\node () at (5.5,0) {$\scriptstyle\stackrel[\cref{eq:braidedpseudofun}]{u\times u}{\cong}$};
\node () at (7.875,0) {$\scriptstyle\stackrel[\cref{eq:pseudonaturality}]{\phi_{\beta,\beta}}{\cong}$};
\node () at (10.25,0) {$\scriptstyle{\cong}$};
\node () at (7.875,-1) {$\scriptstyle\stackrel[\cref{eq:interchangedeco}]{}{\cong}$};
\node () at (12,-1) {$\scriptstyle\cong$};
\node (E') at (6.75,-0.5) {$\scriptstyle{F(n_1+m_1) \times F(n_2+m_2)}$};
\node (B) at (9,0.5) {$\scriptstyle{F((m_1+n_1)+(m_2+n_2))}$};
\node (B') at (9,-0.5) {$\scriptstyle{F((n_1+m_1)+(n_2+m_2))}$};
\node (C) at (11.5,0.5) {$\scriptstyle{F((m_1+n_1)+_{b+b'}(m_2+n_2))}$};
\node (C') at (11.5,-0.5) {$\scriptstyle{F((n_1+m_1)+_{b'+b}(n_2+m_2))}$};
\node (G) at (4.25,0.5) {$\scriptstyle{F(m_1) \times F(n_1) \times F(m_2) \times F(n_2)}$};
\node (G') at (4.25,-0.5) {$\scriptstyle{F(n_1) \times F(m_1) \times F(n_2) \times F(m_2)}$};
\node (G'') at (4.25,-1.5) {$\scriptstyle{F(n_1) \times F(n_2) \times F(m_1) \times F(m_2)}$};
\node (E'') at (6.75,-1.5) {$\scriptstyle{F(n_1+n_2) \times F(m_1+m_2)}$};
\node (B'') at (9,-1.5) {$\scriptstyle{F(n_1+_{b'}n_2)\times F(m_1+_b m_2)}$};
\node (C'') at (11.5,-1.5) {$\scriptstyle{F((n_1+_{b'}n_2)+(m_1+_b m_2))}$};
\node (X) at (3.4,0.2) {$\scriptscriptstyle{x_1 \times y_1 \times x_2 \times y_2}$};
\node (X') at (3.4,-0.35) {$\scriptscriptstyle{y_1 \times x_1 \times y_2 \times x_2}$};
\node (X'') at (3.4,-1.2) {$\scriptscriptstyle{y_1 \times y_2 \times x_1 \times x_2}$};
\path[->,font=\scriptsize,>=angle 90]
(B) edge node [fill=white] {$F(\beta+\beta)$} (B')
(E) edge node [fill=white] {$F(\beta) \times F(\beta)$} (E')
(E) edge node [above] {$\phi_{m_1+n_1,m_2+n_2}$} (B)
(E'') edge node [below] {$F\psi\times F\psi$} (B'')
(E') edge node [above] {$\phi_{n_1+m_1,n_2+m_2}$} (B')
(B) edge node [above] {$F(\psi)$} (C)
(B'') edge node [below] {$\phi_{n_1+_{b'}n_2,m_1+_{b}m_2}$} (C'')
(B') edge node [above] {$F(\psi)$} (C')
(C) edge node [fill=white] {$F(\beta +_\beta \beta)$} (C')
(C') edge node [fill=white] {$F(\hat{\chi})$} (C'')
(G) edge node [above] {$\phi_{m_1,n_1} {\times} \phi_{m_2,n_2}$} (E)
(G'') edge node [below] {$\phi_{n_1,n_2} {\times} \phi_{m_1,m_2}$} (E'')
(G') edge node [above] {$\phi_{n_1,m_1} \times \phi_{n_2,m_2}$} (E')
(G) edge node [fill=white] {$\beta \times \beta$} (G')
(G') edge node [fill=white] {$1 \times \beta \times 1$} (G'')
(D) edge[out=90,in=180] node [left,above] {$$} (G)
(D) edge[out=270,in=180] node [left,above] {$$} (G'')
(D) edge node [below] {$$} (G')
(C) edge[bend left=80] node [fill=white] {$F(\hat{\chi}(\beta+_\beta\beta))$} (C'')
;
\end{tikzpicture}
\]
%{\chris The closing sentence for the proof to be complete should be that these two decoration maps are EQUAL for the diagram \cref{eq:thisdiagram} to commute! Right? Is the sentence below satisfactory Kenny? Do you see the logic?}
In order to verify that these two pasted isomorphisms are equal, we first of all need to expand the isomorphism \cref{eq:interchangedeco}, whose left half part is explicitly written as follows (also found in \cite[Page 125]{DS}):
\begin{equation}\label{eq:thetaa}
\scalebox{.8}{
\begin{tikzcd}[column sep=.05in,ampersand replacement=\&]
Fm_1{\times} Fn_1{\times} Fm_2{\times} Fn_2\ar[ddr,phantom,bend right,"{\scriptstyle\stackrel[\cref{eq:braidedpseudofun}]{1\times u\times1}{\cong}}"]\ar[dr,"1\times\phi\times1"]\ar[rr,"\phi\times\phi"]\ar[ddd,"1\times\beta\times1"'] \&\& F(m_1{+}n_1){\times} F(m_2{+}n_2)\ar[d,phantom,"\scriptstyle{\cong}"]\ar[ddd,phantom,"{\scriptstyle\stackrel[\cref{eq:pseudonaturality}]{1\times\phi_{\beta,1}}{\cong}}"]\ar[rr,"\phi"] \&\& F(m_1{+}n_1{+}m_2{+}n_2)\ar[ddd,"F(1+\beta+1)"]\ar[ddl,bend left,phantom,"{\scriptstyle\stackrel[\cref{eq:pseudonaturality}]{\phi_{1,\beta+1}}{\cong}}"] \\
\& Fm_1{\times} F(n_1 {+} m_2){\times} Fn_2\ar[rr,"{1\times\phi}"]\ar[d,"{1\times F\beta\times1}"'] \&\phantom{A}\& Fm_1{\times} F(n_1 {+} m_2 {+} n_2)\ar[d,"{1\times F(\beta+1)}"]\ar[ur,"\phi"] \& \\
\& Fm_1{\times} F(m_2{+}n_1){\times} Fn_2\ar[rr,"1\times\phi"'] \&\phantom{A}\ar[d,phantom,"\scriptstyle\cong"]\& Fm_1{\times} F(m_2{+}n_1{+}n_2)\ar[dr,"\phi"'] \& \\
Fm_1{\times} Fm_2{\times} Fn_1{\times} Fn_2\ar[ur,"1\times\phi\times1"']\ar[rr,"\phi\times\phi"'] \&\& F(m_1{+}m_2){\times} F(n_1{+}n_2)\ar[rr,"\phi"'] \&\& F(m_1{+}m_2{+}n_1{+}n_2)
\end{tikzcd}}
\end{equation}
where the unnamed isomorphisms are appropriate composites of the pseudoassociativity of $F$ \cref{eq:omega}. The result now follows from lengthy pasted diagram calculations using the axioms of a sylleptic lax monoidal pseudofunctor.
\begin{comment}
These decoration isomorphisms then form the following commutative diagram in the category $F({(n_1+_{b'}n_2)}+{(m_1+_b m_2)})$:
\[
\begin{tikzpicture}[scale=1.5]
\node (A) at (0,0) {$F(\hat{\chi} (\beta \odot \beta))((x_1 \oplus y_1)\odot(x_2 \oplus y_2))$};
\node (B) at (4,0) {$F(\hat{\chi})((y_1 \oplus x_1)\odot(y_2 \oplus x_2))$};
\node (C) at (0,-1) {$F(\beta)((x_1\odot x_2) \oplus (y_1 \odot y_2))$};
\node (D) at (4,-1) {$(y_1\odot y_2) \oplus (x_1\odot x_2)$};
\path[->,font=\scriptsize,>=angle 90]
(A) edge node[above]{$F(\hat{\chi})(\tau_1)$} (B)
(B) edge node[right]{$\tau_2$} (D)
(A) edge node[left]{$F(\beta)(\tau_3)$} (C)
(C) edge node[above]{$\tau_4$} (D);
\end{tikzpicture}
\]
where in the top left corner, we note that $$F(\hat{\chi} (\beta \odot \beta))((x_1 \oplus y_1)\odot(x_2 \oplus y_2)) = F(\beta\hat{\chi})((x_1 \oplus y_1)\odot (x_2 \oplus y_2))$$
as the above diagram of maps of cospans commutes in $\A$ and then applying the pseudofunctor $F$ to this diagram yields a commutative diagram in $\Cat$. 
\end{comment}

\begin{thebibliography}{100}

\bibitem{BC} J.\ C.\ Baez and K.\ Courser, Structured cospans,  \textsl{Theory Appl.\ Categ.\ }\textbf{35} (2020), 1771--1822.   Available as \href{http://arxiv.org/abs/1911.04630}{arXiv:1911.04630}.

\bibitem{BCR} J.\ C.\ Baez, B.\ Coya and F.\ Rebro, Props in circuit theory, \textsl{Theory Appl.\ Categ.\ }\textbf{33} (2018), 727--783.    Available as \href{https://arxiv.org/abs/1707.08321}{arXiv:1707.08321}. 

\bibitem{BE} J.\ C.\ Baez and J.\ Erbele, Categories in control, \textsl{Theory Appl.\ Categ.} {\bf 30} (2015), 836--881. Available as \href{http://arxiv.org/abs/1405.6881}{arXiv:1405.6881}.

\bibitem{BF}  J.\ C.\ Baez and B.\ Fong, A compositional framework for passive linear networks, \textsl{Theory Appl.\ Categ.\ }\textbf{33} (2018), 1158--1222.  Available as \href{http://arxiv.org/abs/1504.05625}{arXiv:1504.05625}.

\bibitem{BFP} J.\ C.\ Baez, B.\ Fong and B.\ S.\ Pollard, A compositional framework for Markov processes, \textsl{Jour. Math. Phys.} \textbf{57} (2016), 033301. Available as \href{http://arxiv.org/abs/1508.06448}{arXiv:1508.06448}.

\bibitem{BM}  J.\ C.\ Baez and J.\ Master, Open Petri nets, \textsl{Math.\ Struct.\ Comput.\ Sci.\ }\textbf{30} (2020), 314--341. Available as 
\href{https://arxiv.org/abs/1808.05415}{arXiv:1808.05415}. 

\bibitem{BP} J.\ C.\ Baez and B.\ S.\ Pollard, A compositional framework for chemical reaction networks, \textsl{Rev.\ Math.\ Phys.\ }\textbf{29} (2017), 1750028.  Available as \href{http://arxiv.org/abs/1704.02051}{arXiv:1704.02051}.

\bibitem{BFV} G.\ Bakirtzis, C.\ H.\ Fleming and C.\ Vasilakopoulou, Categorical semantics of cyber-physical systems theory. Available as 
\href{https://arxiv.org/abs/2010.08003}{arXiv:2010.08003}.

\bibitem{BSZ} Filippo Bonchi, Pawe\l~Soboci\'nski and Fabio Zanasi, A categorical semantics of signal flow graphs, in \textsl{CONCUR 2014--Concurrency Theory}, eds.\ P.\ Baldan and D.\ Gorla, \textsl{Lecture Notes in Computer Science} vol.\ 8704, Springer, Berlin, 2014, pp.\ 435--450.  Also available at \href{http://users.ecs.soton.ac.uk/ps/papers/sfg.pdf}{http://users.ecs.soton.ac.uk/ps/papers/sfg.pdf}.

\bibitem{Borc} F. \ Borceux, \textsl{Handbook of Categorical Algebra}, vol.\ 2, 
Cambridge University Press, Cambridge, 1994

%\bibitem{Brown1} R.\ Brown and C.\ B.\ Spencer, Double groupoids and crossed modules, 
%\textsl{Cah.\ Top.\ G\'eom.\ Diff.} \textbf{17} (1976), 343--362.

%\bibitem{Brown2} R.\ Brown, K.\ Hardie, H.\ Kamps and T.\ Porter, The homotopy double groupoid of a Hausdorff space, \textsl{Theory Appl.\ Categ.} \textbf{10} (2002), 71--93.

%\bibitem{Be} J.\ B\'enabou, Introduction to bicategories, in {\sl Reports
%of the Midwest Category Seminar}, Lecture Notes in Mathematics, vol.\ \textbf{47}, Springer, Berlin, 1967, pp.\ 1--77.

\bibitem{BungeFiore} M.\ Bunge and M.\ Fiore, Unique factorisation lifting functors and categories of linearly-controlled processes, \textsl{Math.\ Struct.\ Comput.\ Sci.\ } \textbf{10} (2000), 137--163.

%\bibitem{CC} D.\ Cicala and K.\ Courser, Spans of cospans in a topos. Available as \href{https://arxiv.org/abs/1707.02098}{arXiv:1707.02098}.

\bibitem{CV} D.\ Cicala and C.\ Vasilakopoulou, On adjoints and fibrations. In preparation.

\bibitem{Courser} K.\ Courser, A bicategory of decorated cospans, \textsl{Theory Appl.\ Categ.} \textbf{32} (2017), 995--1027. Available as \href{https://arxiv.org/abs/1605.08100}{arXiv:1605.08100}.

\bibitem{CourserThesis} K.\ Courser, \textsl{Open Systems: a Double Categorical Perspective}, Ph.D.\ thesis, Department of Mathematics, U.\ C.\ Riverside, 2020.  Available as \href{https://arxiv.org/abs/2008.02394}{arXiv:2008.02394}.

\bibitem{CTF} Gheorghe Craciun, Yangzhong Tang and Martin Feinberg, Understanding bistability in complex enzyme-driven reaction networks, \textsl{PNAS} \textbf{103} (2006), 8697--8702.  

\bibitem{DS} B.\ Day and R.\ Street, Monoidal Bicategories and Hopf algebroids, \textsl{Adv.\ Math.} \textbf{129} (1997), 99--157.

%\bibitem{Ehresmann63} C.\ Ehresmann, Cat\'egories structur\'ees III: Quintettes et applications covariantes,  \textsl{Cah.\ Top.\ G\'eom.\ Diff.} \textbf{5} (1963), 1--22.

%\bibitem{Ehresmann65} C.\ Ehresmann, {\sl Cat\'egories et Structures,} Dunod, Paris, 1965.

\bibitem{Fong} B.\ Fong, Decorated cospans, \emph{Theory Appl.\ Categ.} \textbf{30} (2015), 1096--1120.  Available as \href{http://arxiv.org/abs/1502.00872}{arXiv:1502.00872}.

\bibitem{FongThesis} B.\ Fong, \textsl{The Algebra of Open and Interconnected Systems},
Ph.D.\ thesis, Computer Science Department, University of Oxford, 2016.
Available as \href{https://arxiv.org/abs/1609.05382}{arXiv:1609.05382}.

\bibitem{FRS} B.\ Fong, P.\ Rapisarda and P.\ Sobocinski, A categorical approach to open and interconnected dynamical systems, in \textsl{Proceedings of the 31st Annual ACM/IEEE Symposium on Logic in Computer Science (LICS)}, IEEE, New York, 2016, pp.\ 1--10.  Available as \href{http://arxiv.org/abs/510.05076}{arXiv:1510.05076}.

\bibitem{GiraultValk} C.\ Girault and R.\ Valk, \textsl{Petri Nets for Systems Engineering: a Guide to Modeling, Verification, and Applications}, Springer, Berlin, 2013.

\bibitem{GPS} R.\ Gordon, A.\ J.\ Power and R.\ Street, Coherence for tricategories, \textsl{Mem.\ Amer.\ Math.\ Soc.\ }\textbf{558}, 1995.

%\bibitem{GP1} M.\ Grandis and R.\ Par\'e, Limits in double categories, \textsl{Cah.\ Top.\ G\'eom.\ Diff.} \textbf{40} (1999), 162--220.

%\bibitem{GP2} M.\ Grandis and R.\ Par\'e, Adjoints for double categories, 
% \textsl{Cah.\ Top.\ G\'eom.\ Diff.} \textbf{45} (2004), 193--240.

\bibitem{Gray} J. \ Gray, Fibred and cofibred categories, in \textsl{Proceedings of the Conference on Categorical Algebra: La Jolla 1965}, eds.\ S.\ Eilenberg \textit{et al}, Springer, Berlin, 1966, pp.\ 21--83.

\bibitem{Haas} P.\ J.\ Haas, \textsl{Stochastic Petri Nets: Modelling, Stability, Simulation},
Springer, Berlin, 2002.

\bibitem{HP} M.\ Halter and E.\ Patterson, Compositional epidemiological modeling using structured cospans, 2020.  Available at \href{https://www.algebraicjulia.org/blog/post/2020/10/structured-cospans}{https://www.algebraicjulia.org/blog/post/2020/10/structured-cospans}.

\bibitem{HS}  L.\ W.\ Hansen and M.\ Shulman, Constructing symmetric monoidal bicategories functorially.  Available as \href{https://arxiv.org/abs/1910.09240}{arXiv:1910.09240}.

\bibitem{Hermida1999} C. \ Hermida, Some properties of Fib as a fibred 2-category, \textsl{J.\ Pure Appl.\ Alg.\ } \textbf{134} (1999), 83--109.

\bibitem{Jacobs} B. \ Jacobs, \textsl{Categorical Logic and Type Theory}, Elsevier, Amsterdam, 1999.

\bibitem{JNW} A.\ Joyal, M.\ Nielsen and G.\ Winskel, Bisimulation from open maps, \textsl{Inf.\ Comput.\ }\textbf{127} (1996), 164--185.

\bibitem{KSW} P.\ Katis, N.\ Sabadini and R.\ F.\ C.\ Walters, On the algebra of systems with feedback and boundary, \textsl{Rendiconti del Circolo Matematico di Palermo Serie II} \textbf{63} (2000), 123--156.

\bibitem{KS} G.\ Kelly and R.\ Street, Review of the elements of 2-categories, in \textsl{Category Seminar}, Lecture Notes in Mathematics 40, Springer, Berlin, 1974, 
pp.\ 75--103.

\bibitem{Koch} I.\ Koch, Petri nets---a mathematical formalism to analyze chemical reaction
networks, \textsl{Mol.\ Inform.\  }\textbf{29} (2010), 838--843.

\bibitem{Lawvere} F.\ W.\ Lawvere, State categories and response functors, unpublished manuscript, 1986.  Available at \href{https://tinyurl.com/state}{https://tinyurl.com/state-}  \href{https://tinyurl.com/state-categories}{categories}.

%\bibitem{LS} E.\ Lerman and D.\ Spivak, An algebra of open continuous time dynamical systems and networks. Available as \href{http://arxiv.org/abs/1602.01017}{arXiv:1602.01017}.

%\bibitem{Lack} S.\ Lack, Limits for lax morphisms, \textsl{Applied Categorical Structures} $\mathbf{30}$ (2005), 189--203. Available %at \href{http://maths.mq.edu.au/~slack/papers/talgl.pdf}{http://maths.mq.edu.au/$\sim$slack/papers/talgl.pdf}.

%\bibitem{Lerm} E.\ Lerman and D.\ Spivak, An algebra of open continuous time dynamical systems and networks. Available as %%\href{http://arxiv.org/abs/1602.01017}{arXiv:1602.01017}.

%   \bibitem{ML} S.\ Mac Lane, {\sl Categories for the Working Mathematician},
%     Springer, Berlin, 1998.

\bibitem{McCrudden} P.\ McCrudden, Balanced coalgebroids, \textsl{Theory Appl.\ Categ.\ } \textbf{7} (2000), 71--147.

\bibitem{MV} J.\ Moeller and C.\ Vasilakopoulou, Monoidal Grothendieck construction, \textsl{Theory Appl.\ Categ.\ }\textbf{35} (2020), 1159--1207. Available as \href{https://arxiv.org/abs/1809.00727}{arXiv:1809.00727}.

\bibitem{Niefield} S.~Niefield, Span, cospan, and other double categories, \textsl{Theory Appl.\ Categ.} \textbf{26} (2012), 729--742. Available as \href{https://arxiv.org/abs/1201.3789}{arXiv:1201.3789}.

%\bibitem{Panan} F.\ Clerc, H.\ Humphrey and P.\ Panangaden, Bicategories of Markov processes,  to appear.

\bibitem{Peterson} J.\ L.\ Peterson, \textit{Petri Net Theory and the Modeling of Systems}, Prentice-Hall, New Jersey, 1981.  

\bibitem{PollardThesis} B.\ S.\ Pollard, \textsl{Open Markov Processes and Reaction Networks}, Ph.D. thesis, U. C. Riverside, 2017.  Available as \href{https://arxiv.org/abs/1709.09743}{arXiv:1709.09743}.

%\bibitem{RSW} R.\ Rosebrugh, N.\ Sabadini and R.\ F.\ C.\ Walters, Generic commutative separable algebras and cospans of graphs, \textsl{Theory Appl.\ Categ.\ }\textbf{15} (2005), 164--177. 

%\bibitem{Reb} F.\ Rebro, Constructing the bicategory Span$_{2}(\mathrm{A})$. Available as \href{https://arxiv.org/abs/1501.00792}{arXiv:1501.00792}.

\bibitem{SSV} P.\ Schultz, D.\ Spivak and C.\ Vasilakopoulou, Dynamical systems and sheaves, \textsl{Appl.\ Cat.\ Struct.\ }\textbf{28} (2020), 1--57. Available as \href{https://arxiv.org/abs/1609.08086}{arXiv:1609.08086}.

\bibitem{Shulman2008} M.\ Shulman, Framed bicategories and monoidal fibrations, \textsl{Theory Appl.\ Categ.\ }\textbf{20} (2008), 650--738. Available as \href{https://arxiv.org/abs/0706.1286}{arXiv:0706.1286}.

\bibitem{Shulman2010} M.\ Shulman, Constructing symmetric monoidal bicategories. Available as \href{http://arxiv.org/abs/1004.0993}{arXiv:1004.0993}.

\bibitem{Stay} M.\ Stay, Compact closed bicategories, \textsl{Theory Appl.\ Categ.\ }\textbf{31} (2016), 755--798.   Available as \href{http://arxiv.org/abs/1301.1053}{arXiv:1301.1053}.

\bibitem{VSL} D.\ Vagner, D.\ Spivak and E.\ Lerman, Algebras of open dynamical systems on the operad of wiring diagrams, \textsl{Theory Appl.\ Categ.\ } \textbf{30} (2015), 1793--1822. Available as \href{https://arxiv.org/abs/1408.1598}{arXiv:1408.1598}.

\bibitem{Wilkinson} D.\ J.\ Wilkinson, \textsl{Stochastic Modelling for Systems Biology},
Taylor and Francis, New York, 2006.

\bibitem{Willems} J.\ C.\ Willems, The behavioral approach to open and interconnected systems, \textsl{IEEE Control Systems Magazine} (2007)

\end{thebibliography}
\end{document}
