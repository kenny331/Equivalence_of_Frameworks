% Structured and decorated cospans
% John Baez, Kenny Courser and Christina Vasilakopoulou
% 2020/11/12 - JB

\documentclass[reqno]{amsart}
\usepackage{amssymb,amsmath,stmaryrd,txfonts,mathrsfs,amsthm}

\usepackage[all,2cell]{xy}\UseAllTwocells\SilentMatrices
\usepackage[neveradjust]{paralist}
\usepackage{hyperref}
\usepackage{mathtools}
\usepackage{multirow}
\usepackage[outline]{contour}
\contourlength{1.2pt}
\usepackage{tikz}
\usepackage{tikz-cd}
\usepackage{xcolor}
\usepackage{framed,color}
\usepackage[draft]{fixme}
\usetikzlibrary{matrix,arrows,decorations.pathmorphing,positioning}
\usetikzlibrary{intersections,decorations.markings}
\usetikzlibrary{arrows,positioning,fit,matrix,shapes.geometric,external}
\usetikzlibrary{backgrounds,circuits,circuits.ee.IEC,shapes,fit,matrix}
\usepackage{tikz}
\usetikzlibrary{matrix,arrows}
\usepackage{comment}
\usepackage[capitalize]{cleveref}
\definecolor{rewritecolor}{rgb}{0,.9,1}
\tikzset{rewritenode/.style={shape=circle,fill=rewritecolor,scale=0.25,font=\Huge}}
\tikzset{RWopen/.style={shape=circle,draw=black,fill=white,scale=0.5,font=\Huge}}
\tikzset{RWclosed/.style={shape=circle,fill=black,scale=0.5,font=\Huge}}
\tikzset{CDnode/.style={shape=circle,fill=white,scale=.5}}
\makeatletter
\let\ea\expandafter

\pgfdeclarelayer{edgelayer}
\pgfdeclarelayer{nodelayer}
\pgfsetlayers{edgelayer,nodelayer,main}

% Petri nets
\definecolor{lblue}{rgb}{0,250,255}
\tikzstyle{species}=[circle,fill=yellow,draw=black,scale=1.15]
\tikzstyle{transition}=[rectangle,fill=lblue,draw=black,scale=1.15]
\tikzstyle{inarrow}=[->, >=stealth, shorten >=.03cm,line width=1.5]
\tikzstyle{empty}=[circle,fill=none, draw=none]
\tikzstyle{inputdot}=[circle,fill=purple,draw=purple, scale=.25]
\tikzstyle{inputarrow}=[->,draw=purple, shorten >=.05cm]
\tikzstyle{simple}=[-,draw=purple,line width=1.000]
\tikzstyle{none}=[inner sep=0pt]

\definecolor{shadecolor}{rgb}{1,0.8,0.3}
\definecolor{myurlcolor}{rgb}{0.6,0,0}
\definecolor{mycitecolor}{rgb}{0,0,0.8}
\definecolor{myrefcolor}{rgb}{0,0,0.8}
\hypersetup{colorlinks, linkcolor=myrefcolor, citecolor=mycitecolor, urlcolor=myurlcolor}

\tikzset{->-/.style={decoration={
  markings,
  mark=at position .5 with {\arrow{>}}},postaction={decorate}}}

%% Defining commands that are always in math mode.
\def\mdef#1#2{\ea\ea\ea\gdef\ea\ea\noexpand#1\ea{\ea\ensuremath\ea{#2}}}
\def\alwaysmath#1{\ea\ea\ea\global\ea\ea\ea\let\ea\ea\csname your@#1\endcsname\csname #1\endcsname
  \ea\def\csname #1\endcsname{\ensuremath{\csname your@#1\endcsname}}}
\newcommand{\define}[1]{{\bf \boldmath{#1}}}

% blackboard bold letters
\newcommand{\lA}{\ensuremath{\mathbb{A}}}
\newcommand{\lC}{\ensuremath{\mathbb{C}}}
\newcommand{\lD}{\ensuremath{\mathbb{D}}}
\newcommand{\lE}{\ensuremath{\mathbb{E}}}
\newcommand{\lR}{\ensuremath{\mathbb{R}}}
\newcommand{\lX}{\ensuremath{\mathbb{X}}}
\mdef\fahat{\hat{\fa}}

% MISCELLANEOUS SYMBOLS
\newcommand{\inv}{^{-1}}
\newcommand{\op}{^{\mathit{op}}}
\newcommand{\co}{^{\mathit{co}}}
\newcommand{\coop}{^{\mathit{coop}}}
\newcommand{\id}{\mathm{id}}
\let\adj\dashv
\newcommand{\pullbackcorner}[1][dr]{\save*!/#1-1.2pc/#1:(-1,1)@^{|-}\restore}
\let\iso\cong
\let\eqv\simeq
\let\cng\equiv
\mdef\Id{\mathrm{Id}}
\mdef\id{\mathrm{id}}
\alwaysmath{ell}
\alwaysmath{infty}
\alwaysmath{odot}
\def\frc#1/#2.{\frac{#1}{#2}}   % \frc x^2+1 / x^2-1 .
\mdef\ten{\mathrel{\otimes}}

%% OPERATORS
\DeclareMathOperator\colim{colim}
\DeclareMathOperator\eq{eq}
\DeclareMathOperator\Aut{Aut}
\DeclareMathOperator\End{End}
\DeclareMathOperator\Hom{Hom}
\DeclareMathOperator\Map{Map}

%% ARROWS
% \to already exists
\newcommand{\too}[1][]{\ensuremath{\overset{#1}{\longrightarrow}}}
\newcommand{\oot}[1][]{\ensuremath{\overset{#1}{\longleftarrow}}}
\let\toot\rightleftarrows
\let\otto\leftrightarrows
\let\maps\colon

%% EXTENSIBLE ARROWS
\let\xto\xrightarrow
\let\xot\xleftarrow

% THEOREM-TYPE ENVIRONMENTS, hacked to
%% (a) number all with the same numbers, and
%% (b) have the right names for autoref
\def\defthm#1#2{%
  \newtheorem{#1}{#2}[section]%
  \expandafter\def\csname #1autorefname\endcsname{#2}%
  \expandafter\let\csname c@#1\endcsname\c@thm}
\newtheorem{thm}{Theorem}[section]
\newcommand{\thmautorefname}{Theorem}
\defthm{cor}{Corollary}
\defthm{prop}{Proposition}
\defthm{lem}{Lemma}
\defthm{conj}{Conjecture}
\defthm{hyp}{Hypothesis}
\defthm{fact}{Fact}
\theoremstyle{definition}
\defthm{defn}{Definition}
\defthm{notn}{Notation}
\theoremstyle{remark}
\defthm{rmk}{Remark}
\defthm{eg}{Example}

\newcommand{\fhat}{\ensuremath{\hat{f}}}

% Also number formulas with the theorem counter
\let\c@equation\c@thm
\numberwithin{equation}{section}

% Only show numbers for equations that are actually referenced (or
% whose tags are specified manually) - uses mathtools.
%\mathtoolsset{showonlyrefs,showmanualtags}

\def\tobar{\mathrel{\mkern3mu  \vcenter{\hbox{$\scriptscriptstyle+$}}%
                    \mkern-12mu{\to}}}

%\input{decls}
\UseAllTwocells

\newcommand{\dblcat}[1]{\mathbb{#1}}
\mdef\fchk{\check{f}}

\definecolor{purple(x11)}{rgb}{0.5, 0.0, 0.5}
\def\purple{\color{purple(x11)}}
\def\chris{\purple}

%Christina: change below accordingly if needed!
\newcommand{\ca}{\mathsf}
\newcommand{\bicat}{\mathbf}
\newcommand{\U}{U}
\newcommand{\D}{\ca{A}}
\newcommand{\C}{\ca{X}} 
\newcommand{\A}{\ca{A}}
\newcommand{\B}{\ca{B}}
\newcommand{\X}{\ca{X}}
\newcommand{\dcsp}[1]{{#1}\mathbb{C}\textnormal{sp}}
\tikzset{tick/.style={postaction={decorate,decoration={markings,
mark=at position 0.4 with {\draw[-] (0,.4ex) -- (0,-.4ex);}}}}}
\newcommand{\tickar}{\begin{tikzcd}[baseline=-0.5ex,cramped,sep=small,ampersand replacement=\&]{}\ar[r,tick]\&{}\end{tikzcd}}
\newcommand{\cspn}[5]{\begin{tikzcd}[baseline=-0.5ex,cramped,sep=small,ampersand replacement=\&]{#1}\ar[r,"#4"] \& {#2} \& {#3}\ar[l,"#5"']\end{tikzcd}}
\newcommand{\OpICat}{\bicat{OpICat}}%probably too much to use?
\newcommand{\pse}{\mathrm{ps}}
\newcommand{\OpFib}{\bicat{OpFib}}


\title{Structured Versus Decorated Cospans}

\author{John\ C.\ Baez$^{1,2}$, Kenny Courser$^1$, and Christina Vasilakopoulou$^3$}
\address{$^1$Department of Mathematics, University of California, Riverside CA, USA 92521}
\address{$^2$Centre for Quantum Technologies, National University of Singapore, Singapore 117543}
\address{$^3$Department of Mathematics, University of Patras, Greece 265 04}
\email{baez@math.ucr.edu, kcour001@ucr.edu, cvasilak@math.upatrs.gr}


\begin{document}
\begin{abstract}
\noindent
The first two authors have developed a compositional framework well-suited for studying networks that are built out of finite sets equipped 
with extra stuff. This framework, which goes by the name of `structured cospans', utilizes double categories where the objects are seen as 
inputs and outputs, horizontal 1-cells are `open networks', and 2-morphisms are maps between open networks. In this setup a functor $L \maps 
\A \to \textsf{X}$, which is typically a left adjoint, is used to replace the objects and vertical 1-morphisms of a given double category 
$\mathbb{X}$ with the objects and morphisms, respectively, of the category $\A$. Horizontal 1-cells are then cospans in $\textsf{X}$ of a 
particular form with 2-morphisms given by maps of these cospans. Fong has also developed a similar framework utilizing cospans to study open 
networks which goes by the name of `decorated cospans'. In this setup, a lax monoidal functor $F \maps \A \to \textsf{Set}$ is used to 
`decorate' the apices of cospans in $\A$ with elements of $\textsf{Set}$ giving the cospans extra structure. Using a slight variation of 
Fong's framework, we prove that these two frameworks are equivalent in the situation where a left adjoint can be obtained from a lax 
monoidal pseudofunctor using a well known construction of Grothendieck.
\end{abstract}

\maketitle

\setcounter{tocdepth}{1}
\tableofcontents

\section{Introduction}

An `open system' is any sort of system that can interact with the outside world.  Experience has shown that open systems are nicely modeled using cospans \cite{CourserThesis, FongThesis, PollardThesis}. A cospan in some category $\A$ is a diagram of this form:
\[
\begin{tikzpicture}[scale=1.5]
\node (A) at (0,0) {$a$};
\node (B) at (1,1) {$m$};
\node (C) at (2,0) {$b$};
\path[->,font=\scriptsize,>=angle 90]
(A) edge node[above]{$i$} (B)
(C) edge node[above]{$o$} (B);
\end{tikzpicture}
\]
We call $m$ the \define{apex}, $a$ and $b$ the \define{feet}, and $i$ and $o$ the \define{legs} of the cospan.   The apex describes the system itself.  The feet describe `interfaces'  through which the system can interact with the outside world.  The legs describe how the interfaces are included in the system.   If the category $\A$ has finite colimits, we can compose cospans using pushouts: this describes the operation of attaching two open systems together in series by identifying one interface of the first with one of the second.  We can also `tensor' cospans using coproducts: this describes setting open systems side by side, in parallel.  Via these operations we obtain a symmetric monoidal double category with cospans in $\A$ as its horizontal 1-cells \cite{Courser,Niefield}.

However, we often want the system itself to have more structure than its interfaces.   This led Fong to develop a theory of `decorated' cospans \cite{Fong}.  Given a category $\A$ with finite colimits, a symmetric lax monoidal functor $F \maps (\A,+) \to (\textsf{Set},\times)$ can be used to equip the apex $m$ of a cospan in $\A$ with some extra data: an element $d \in F(m)$, which we call a \textbf{decoration}.  Thus a \define{decorated cospan} is a pair:
\[
\begin{tikzpicture}[scale=1.5]
\node (A) at (0,0) {$a$};
\node (B) at (1,0) {$m$};
\node (C) at (2,0) {$b$,};
\node (E) at (4,0) {$d \in F(m)$.};
\path[->,font=\scriptsize,>=angle 90]
(A) edge node[above]{$i$} (B)
(C) edge node[above]{$o$} (B);
\end{tikzpicture}
\]
Fong proved that there is a symmetric monoidal category with objects
of $\A$ as its objects and equivalence classes of decorated cospans as its morphisms.  Such categories were used to describe a variety of open systems: electrical circuits, Markov processes, chemical reaction networks and dynamical systems \cite{BF,BFP,BP}.  Later the second author improved Fong's categories to symmetric monoidal double categories, eliminating the need to work with equivalence classes \cite{Courser}.

Unfortunately, many applications of decorated cospans were flawed.  The problem is that while Fong's decorated cospans are good for decorating the apex $m$ with an element of a set $F(m)$, they are unable to decorate it with an object of a category.   An example would be equipping a finite set $m$ with edges making its elements into the nodes of a graph.    We would like the following `open graph' to be a decorated cospan with $m = \{n_1, n_2, n_3, n_4\}$:  
\[
\scalebox{0.8}{
\begin{tikzpicture}
	\begin{pgfonlayer}{nodelayer}
		\node [contact] (n1) at (-2,0) {$\bullet$};
		\node [style = none] at (-2.1,0.3) {$n_1$};
		\node [contact] (n2) at (0,1) {$\bullet$};
		\node [style = none] at (0,1.3) {$n_2$};
		\node [contact] (n3) at (0,-1) {$\bullet$};
		\node [style = none] at (0,-1.3) {$n_3$};
		\node [contact] (n4) at (2,0) {$\bullet$};
		\node [style = none] at (2.1,0.3) {$n_4$};
		
		\node [style = none] at (-1,1) {$e_1$};
		\node [style = none] at (-1,-1) {$e_2$};
		\node [style = none] at (1,1) {$e_3$};
		\node [style = none] at (1,-1) {$e_4$};
	    \node [style = none] at (0.3,0) {$e_5$};
		
		\node [style=none] (1) at (-3,0) {1};
		\node [style=none] (4) at (3,0) {2};
	
		\node [style=none] (ATL) at (-3.4,1.4) {};
		\node [style=none] (ATR) at (-2.6,1.4) {};
		\node [style=none] (ABR) at (-2.6,-1.4) {};
		\node [style=none] (ABL) at (-3.4,-1.4) {};

		\node [style=none] (X) at (-3,1.8) {$a$};
		\node [style=inputdot] (inI) at (-2.8,0) {};
		
		\node [style=none] (Z) at (3,1.8) {$b$};
	 \node [style=inputdot] (outI') at (2.8,0) {};

		\node [style=none] (MTL) at (2.6,1.4) {};
		\node [style=none] (MTR) at (3.4,1.4) {};
		\node [style=none] (MBR) at (3.4,-1.4) {};
		\node [style=none] (MBL) at (2.6,-1.4) {};
	
	\end{pgfonlayer}
	\begin{pgfonlayer}{edgelayer}
		\draw [style=inarrow, bend left=20, looseness=1.00] (n1) to (n2);
		\draw [style=inarrow, bend right=20, looseness=1.00] (n1) to (n3);
		\draw [style=inarrow, bend left=20, looseness=1.00] (n2) to (n4);
		\draw [style=inarrow, bend right=20, looseness=1.00] (n3) to (n4);
		\draw [style=inarrow] (n2) to (n3);
%		\draw [style=inarrow] (W) to (Water);
%		\draw [style=inarrow, bend left=40, looseness=1.00] (Water2) to (Something);
%		\draw [style=inarrow, bend right=40, looseness=1.00] (Water2) to (Something);
%		\draw [style=inarrow, bend left=40, looseness=1.00] (Something) to (A);
%		\draw [style=inarrow, bend right=40, looseness=1.00] (Something) to (B);
		\draw [style=simple] (ATL.center) to (ATR.center);
		\draw [style=simple] (ATR.center) to (ABR.center);
		\draw [style=simple] (ABR.center) to (ABL.center);
		\draw [style=simple] (ABL.center) to (ATL.center);
%		\draw [style=simple] (BTL.center) to (BTR.center);
%		\draw [style=simple] (BTR.center) to (BBR.center);
%		\draw [style=simple] (BBR.center) to (BBL.center);
%		\draw [style=simple] (BBL.center) to (BTL.center);
		\draw [style=simple] (MTL.center) to (MTR.center);
		\draw [style=simple] (MTR.center) to (MBR.center);
		\draw [style=simple] (MBR.center) to (MBL.center);
		\draw [style=simple] (MBL.center) to (MTL.center);
%		\draw [style=inputarrow] (outI) to (A);
%		\draw [style=inputarrow] (outS) to (B);
		\draw [style=inputarrow] (inI) to (n1);
		\draw [style=inputarrow] (outI') to (n4);
%		\draw [style=inputarrow] (inI') to (Water2);
%		\draw [style=inputarrow] (inS') to (Water2);
	\end{pgfonlayer}
\end{tikzpicture}
}
\]
We might hope to do this using symmetric lax monoidal functor $F \maps (\Fin\Set, +) \to (\Set, \times)$ assigning to each finite set $m$ the set of all graphs with $m$ as their set of nodes    But this hope is doomed, for reasons painstakingly explained in \cite[Sec.\ 5]{BC}.  There is really a \emph{category} of graphs with $m$ as their set of nodes, and trying to treat it as a mere set does not work.

Here we present a solution to this problem.  Instead of basing the theory of decorated cospans on a symmetric lax monoidal functor $F \maps (\A, +) \to (\Set, \times)$, we use a symmetric lax monoidal pseudofunctor $F \maps (\A, +) \to (\Cat, \times)$.  In \cref{DC} we use this
data to construct a symmetric monoidal double category $F\lCsp$ in which:
\begin{itemize}
\item an object is an object of $\A$,
\item a vertical 1-morphism is a morphism of $\A$,
\item a horizontal 1-cell from $a$ to $b$ is a decorated cospan:
\[
\begin{tikzpicture}[scale=1.5]
\node (A) at (0,0) {$a$};
\node (B) at (1,0) {$m$};
\node (C) at (2,0) {$b,$};
\node (D) at (3,0) {$x \in F(m)$,};
\path[->,font=\scriptsize,>=angle 90]
(A) edge node[above]{$i$} (B)
(C) edge node[above]{$o$} (B);
\end{tikzpicture}
\]
\item a 2-morphism is a \define{map of decorated cospans}: that is, a commutative
diagram
\[
\begin{tikzpicture}[scale=1.5]
\node (A) at (0,0.5) {$a$};
\node (A') at (0,-0.5) {$a'$};
\node (B) at (1,0.5) {$m$};
\node (C) at (2,0.5) {$b$};
\node (C') at (2,-0.5) {$b'$};
\node (D) at (1,-0.5) {$m'$};
\node (E) at (3,0.5) {$x \in F(m)$};
\node (F) at (3,-0.5) {$x' \in F(m')$};
\path[->,font=\scriptsize,>=angle 90]
(A) edge node[above]{$i$} (B)
(C) edge node[above]{$o$} (B)
(A) edge node[left]{$f$} (A')
(C) edge node[right]{$g$} (C')
(A') edge node[above] {$i'$} (D)
(C') edge node[above] {$o'$} (D)
(B) edge node [left] {$h$} (D);
\end{tikzpicture}
\]
together with a morphism $\tau \maps F(h)(m) \to m'$ in $F(m')$.
\end{itemize}

In fact another solution to the problem is already known: the theory of structured cospans \cite{BC,CourserThesis}.  Given a functor $L \maps \A \to \X$, a \define{structured cospan} is a cospan in $\X$ whose feet come from a pair of objects in $\A$:
\[
\begin{tikzpicture}[scale=1.2]
\node (A) at (0,0) {$L(a)$};
\node (B) at (1,1) {$x$};
\node (C) at (2,0) {$L(b).$};
\path[->,font=\scriptsize,>=angle 90]
(A) edge node[above]{$$} (B)
(C)edge node[above]{$$}(B);
\end{tikzpicture}
\]
This is another way of letting the apex have more structure than the feet.   When $\A$ and $\X$ have finite colimits and $L$ preserves them, there is a symmetric monoidal double category ${}_L \lCsp(\X)$ where:
\begin{itemize}
\item an object is an object of $\A$,
\item a vertical 1-morphism is a morphism of $\A$,
\item a horizontal 1-cell from $a$ to $b$ is a diagram in $\X$ of this form:
\[
\begin{tikzpicture}[scale=1.5]
\node (A) at (0,0) {$L(a)$};
\node (B) at (1,0) {$x$};
\node (C) at (2,0) {$L(b)$};
\path[->,font=\scriptsize,>=angle 90]
(A) edge node[above]{$i$} (B)
(C)edge node[above]{$o$}(B);
\end{tikzpicture}
\]
\item a 2-morphism is a commutative diagram in $\X$ of this form:
\[
\begin{tikzpicture}[scale=1.5]
\node (E) at (3,0) {$L(a)$};
\node (F) at (5,0) {$L(b)$};
\node (G) at (4,0) {$x$};
\node (E') at (3,-1) {$L(a')$};
\node (F') at (5,-1) {$L(b')$};
\node (G') at (4,-1) {$x'$};
\path[->,font=\scriptsize,>=angle 90]
(F) edge node[above]{$o$} (G)
(E) edge node[left]{$L(\alpha)$} (E')
(F) edge node[right]{$L(\beta)$} (F')
(G) edge node[left]{$f$} (G')
(E) edge node[above]{$i$} (G)
(E') edge node[below]{$i'$} (G')
(F') edge node[below]{$o'$} (G');
\end{tikzpicture}
\]
\end{itemize}
Many of the flawed applications of decorated cospans have been fixed using structured cospans
\cite[Sec.\ 6]{BC}.

We show that our new solution subsumes the old one: structured cospan double categories are always equivalent to decorated cospan double categories.  (REALLY????)   

We also give sufficient conditions for a decorated cospan double category to be equivalent to a structured cospan double category. 
Suppose $\A$ has finite colimits and $F \maps (\A , +) \to (\Cat, \times)$ is 
symmetric lax monoidal pseudofunctor.   Any such pseudofunctor gives an opfibration 
$R \maps \X \to \A$ where $\X = \int F$ is defined by the Grothendieck construction.
Let $\Rex$ be the 2-category of finitely cocomplete categories, finite-colimit-preserving
functors and natural transformations.  We show that if $F$ factors through $\mathbf{Rex}$ as
a pseudofunctor, the opfibration $R \maps \X \to \A$ is also a right adjoint.  From the accompanying left adjoint $L \maps \A \to \X$, we  construct a symmetric monoidal double category ${}_L \lCsp(\X)$ of structured cospans.  Finally, in \cref{thm:equiv} we prove that this structured cospan double category ${}_L \lCsp(\X)$ is equivalent to the decorated cospan double category $F \lCospan$. 

\iffalse
To obtain a decoration on the composition of two composable decorated cospans:
\[
\begin{tikzpicture}[scale=1.5]
\node (A) at (0,0) {$a_1$};
\node (B) at (1,0) {$b$};
\node (C) at (2,0) {$a_2$};
\node (F) at (3,0) {$a_2$};
\node (G) at (4,0) {$b'$};
\node (H) at (5,0) {$a_3$};
\node (D) at (0.5,-0.5) {$1$};
\node (E) at (1.5,-0.5) {$F(b)$};
\node (D') at (3.5,-0.5) {$1$};
\node (E') at (4.5,-0.5) {$F(b')$};
\path[->,font=\scriptsize,>=angle 90]
(A) edge node[above]{$i$} (B)
(C) edge node[above]{$o$} (B)
(F) edge node[above]{$i'$} (G)
(H) edge node[above]{$o'$} (G)
(D) edge node [above] {$d_1$} (E)
(D') edge node [above] {$d_2$} (E');
\end{tikzpicture}
\]
we use the natural map from a coproduct to a pushout as well as the structure maps, or `laxators', that come as part of the structure of a lax monoidal functor:
\[
1 \xrightarrow{\lambda^{-1}} 1 \times 1 \xrightarrow{d_1 \times d_2} F(b) \times F(b') \xrightarrow{\phi_{b,b'}} F(b+ b') \xrightarrow{F(j_{b,b'})} F(b+_{a_2} b').
\]
Using this idea, Fong constructed a symmetric monoidal category $F\textnormal{Cospan}(\A)$ which has:
\begin{enumerate}
\item{objects as those of $\A$ and}
\item{morphisms as isomorphism classes of $F$-decorated cospans, where an $F$-decorated cospan is given as above, and two $F$-decorated cospans are in the same isomorphism class if there exists an isomorphism $f \maps b \to b'$ between the apices such that the following diagrams commute:
\[
\begin{tikzpicture}[scale=1.5]
\node (A) at (0,0) {$a_1$};
\node (B) at (1,1) {$b$};
\node (B') at (1,-1) {$b'$};
\node (C) at (2,0) {$a_2$};
\node (D) at (3,0) {$1$};
\node (E) at (4,0.5) {$F(b)$};
\node (E') at (4,-0.5) {$F(b')$};
\path[->,font=\scriptsize,>=angle 90]
(A) edge node[above]{$i$} (B)
(C) edge node[above]{$o$} (B)
(A) edge node[below]{$i'$} (B')
(C) edge node[below]{$o'$} (B')
(B) edge node [left] {$f$} (B')
(B) edge node [right] {$\sim$} (B')
(D) edge node [above] {$d$} (E)
(D) edge node [below] {$d'$} (E')
(E) edge node [right] {$F(f)$} (E');
\end{tikzpicture}
\]
}
\end{enumerate}


As an example, let $F \maps \textsf{FinSet} \to \textsf{Set}$ be the symmetric lax monoidal functor that assigns to a finite set $b$ the (large) set of all possible graph structures on the finite set $b$, where a graph structure on $b$ is given by a diagram in $\textsf{Set}$ of the form:
\[
\begin{tikzpicture}[scale=1.5]
\node (A) at (0,0) {$E$};
\node (B) at (1,0) {$b.$};
\path[->,font=\scriptsize,>=angle 90]
(A) edge[bend left] node[above]{$s$} (B)
(A) edge[bend right] node[below]{$t$} (B);
\end{tikzpicture}
\]
Let $b=\{ v_1,v_2 \}$ be a two element set. Then one element of the (large) set $F(b)$, which is the collection of all graph structures on the finite set $b$, is given by a single edge $e$ whose source and target are $v_1$ and $v_2$, respectively.
\[
\begin{tikzpicture}
  [scale=.8,auto=left]
  \node [style={circle,fill=teal}] (n1) at (1,10) {$v_1$};
  \node[style={circle,fill=teal}] (n2) at (4,10)  {$v_2$};
\path[->,font=\scriptsize,>=angle 90]
(n1) edge node[above]{$e$} (n2);
\end{tikzpicture}
\]
%\[
%\begin{tikzpicture}[scale=1.5]
%\node (A) at (0,0) {$v_1$};
%\node (B) at (1,0) {$v_2$};
%\path[->,font=\scriptsize,>=angle 90]
%(A) edge node[above]{$e$} (B);
%\end{tikzpicture}
%\]
Denote this element of $F(b)$ as $d \maps 1 \to F(b)$. Let $a_1=\{ 1 \}$ and $a_2=\{2,3\}$ and define functions $i \maps a_1 \to b$ and $o \maps a_2 \to b$ by $i(1)=v_1$ and $o(2)=o(3)=v_2$. Then we have an $F$-decorated cospan: 
\[
\begin{tikzpicture}[scale=1.5]
\node (A) at (0,0) {$a_1$};
\node (B) at (1,0) {$b$};
\node (C) at (2,0) {$a_2$};
\node (D) at (3,0) {$1$};
\node (E) at (4,0) {$F(b)$};
\path[->,font=\scriptsize,>=angle 90]
(A) edge node[above]{$i$} (B)
(C) edge node[above]{$o$} (B)
(D) edge node [above] {$d$} (E);
\end{tikzpicture}
\]
which is given by the open graph:
\[
\begin{tikzpicture}
  [scale=.8,auto=left]
\node (a) at (-2,10) {$1$};
\node (b) at (7,10.5) {$2$};
\node (c) at (7,9.5) {$3$};
\node (i) at (-0.5,11) {$i$};
\node (o) at (5.5,11) {$o$};
  \node [style={circle,fill=teal}] (n1) at (1,10) {$v_1$};
  \node[style={circle,fill=teal}] (n2) at (4,10)  {$v_2$};
\path[->,font=\scriptsize,>=angle 90]
(a) edge[dashed] node[above] {$$} (n1)
(b) edge[dashed] node[below] {$$} (n2)
(c) edge[dashed] node[above] {$$} (n2)
(n1) edge node[above]{$e$} (n2);
\end{tikzpicture}
\]
We can compose this open graph with another whose set inputs coincide with the set of outputs of the above open graph, namely the set $a_2=\{2,3\}$. For example:
\[
\begin{tikzpicture}
  [scale=.8,auto=left]
%\node (a) at (-2,10) {$1$};
\node (b) at (7,10.5) {$2$};
\node (c) at (7,9.5) {$3$};
%\node (i) at (-0.5,11) {$i$};
\node (o) at (8.5,11.5) {$i'$};
\node (d) at (16,10) {$4$};
\node (e) at (14.5,11.5) {$o'$};
  \node [style={circle,fill=teal}] (n1) at (10,11.5) {$v_4$};
  \node[style={circle,fill=teal}] (n2) at (10,8.5)  {$v_3$};
  \node[style={circle,fill=teal}] (n3) at (13,10)  {$v_5$};
\path[->,font=\scriptsize,>=angle 90]
%(a) edge[dashed] node[above] {$$} (n1)
(b) edge[dashed] node[below] {$$} (n1)
(c) edge[dashed] node[above] {$$} (n2)
(n1) edge[bend left] node[above]{$e_2$} (n3)
(n3) edge[bend left] node[below]{$e_3$} (n2)
(n2) edge[bend left] node[left]{$e_1$} (n1)
(d) edge[dashed] node[above]{$$}(n3);
\end{tikzpicture}
\]
This second open graph can be represented by the $F$-decorated cospan:
\[
\begin{tikzpicture}[scale=1.5]
\node (A) at (0,0) {$a_2$};
\node (B) at (1,0) {$b'$};
\node (C) at (2,0) {$a_3$};
\node (D) at (3,0) {$1$};
\node (E) at (4,0) {$F(b')$};
\path[->,font=\scriptsize,>=angle 90]
(A) edge node[above]{$i'$} (B)
(C) edge node[above]{$o'$} (B)
(D) edge node [above] {$d'$} (E);
\end{tikzpicture}
\]
The composite of these two open graphs is then given by:
\[
\begin{tikzpicture}
  [scale=.8,auto=left]
\node (a) at (-2,10) {$1$};
%\node (b) at (7,10.5) {$2$};
\node (c) at (10,10) {$4$};
\node (i) at (-0.5,11) {$i$};
\node (o) at (8.5,11) {$o'$};
  \node [style={circle,fill=teal}] (n1) at (1,10) {$v_1$};
  \node[style={circle,fill=teal}] (n2) at (4,10)  {$v_2$};
  \node[style={circle,fill=teal}] (n3) at (7,10)  {$v_5$};
\path[->,font=\scriptsize,>=angle 90]
(a) edge[dashed] node[above] {$$} (n1)
%(b) edge[dashed] node[below] {$$} (n2)
(c) edge[dashed] node[above] {$$} (n3)
(n1) edge node[above]{$e$} (n2)
(n2) edge[bend left] node[above]{$e_2$} (n3)
(n2) edge[loop below]node{$e_1$} (n2)
(n3) edge[bend left] node[below]{$e_3$} (n2);
\end{tikzpicture}
\]
where we identify $v_2=v_3=v_4$ in taking the pushout of the two cospans of finite sets. The composite open graph is represented by the $F$-decorated cospan:
\[
\begin{tikzpicture}[scale=1.5]
\node (A) at (0,0) {$a_1$};
\node (B) at (1.25,0) {$b+_{a_2}b'$};
\node (C) at (2.5,0) {$a_3$};
\node (D) at (3,0) {$1$};
\node (E) at (4,0) {$F(b+_{a_2}b')$};
\path[->,font=\scriptsize,>=angle 90]
(A) edge node[above]{$\psi j_b i$} (B)
(C) edge node[above]{$\psi j_{b'} o'$} (B)
(D) edge node [above] {$d''$} (E);
\end{tikzpicture}
\]
where $j$ is the natural map into a coproduct and $\psi$ the natural map from a coproduct into a pushout. 

We can also consider the above two open graphs in parallel:
\[
\begin{tikzpicture}
  [scale=.8,auto=left]
\node (a') at (7,12.5) {$1$};
\node (b') at (16,12.5) {$2$};
\node (c') at (16,11.5) {$3$};
\node (i') at (8.5,13.5) {$i+i'$};
\node (o') at (14.5,13.5) {$o+o'$};
  \node [style={circle,fill=teal}] (n1') at (10,13) {$v_1$};
  \node[style={circle,fill=teal}] (n2') at (13,13)  {$v_2$};
\node (b) at (7,11.5) {$2$};
\node (c) at (7,10.5) {$3$};
\node (d) at (16,10.5) {$4$};
  \node [style={circle,fill=teal}] (n1) at (10,11.5) {$v_4$};
  \node[style={circle,fill=teal}] (n2) at (10,8.5)  {$v_3$};
  \node[style={circle,fill=teal}] (n3) at (13,10)  {$v_5$};
\path[->,font=\scriptsize,>=angle 90]
(a') edge[dashed] node[above] {$$} (n1')
(b') edge[dashed] node[below] {$$} (n2')
(c') edge[dashed] node[above] {$$} (n2')
(n1') edge node[above]{$e$} (n2')
(b) edge[dashed] node[below] {$$} (n1)
(c) edge[dashed] node[above] {$$} (n2)
(n1) edge[bend left] node[above]{$e_2$} (n3)
(n3) edge[bend left] node[below]{$e_3$} (n2)
(n2) edge[bend left] node[left]{$e_1$} (n1)
(d) edge[dashed] node[above]{$$}(n3);
\end{tikzpicture}
\]
which is represented by the tensor product of the above two $F$-decorated cospans:
\[
\begin{tikzpicture}[scale=1.5]
\node (A) at (0,0) {$a_1+a_2$};
\node (B) at (1.25,0) {$b+b'$};
\node (C) at (2.5,0) {$a_2+a_3$};
\node (D) at (3.5,0) {$1$};
\node (E) at (4.5,0) {$F(b+b')$};
\path[->,font=\scriptsize,>=angle 90]
(A) edge node[above]{$i+i'$} (B)
(C) edge node[above]{$o+ o'$} (B)
(D) edge node [above] {$d+d'$} (E);
\end{tikzpicture}
\]

There are some subtleties to this framework; consider two decorated cospans with the same inputs and outputs.
\[
\begin{tikzpicture}[scale=1.5]
\node (A) at (0,0) {$a_1$};
\node (B) at (1,0) {$b$};
\node (C) at (2,0) {$a_2$};
\node (F) at (3,0) {$a_1$};
\node (G) at (4,0) {$b'$};
\node (H) at (5,0) {$a_2$};
\node (D) at (0.5,-0.5) {$1$};
\node (E) at (1.5,-0.5) {$F(b)$};
\node (D') at (3.5,-0.5) {$1$};
\node (E') at (4.5,-0.5) {$F(b')$};
\path[->,font=\scriptsize,>=angle 90]
(A) edge node[above]{$i$} (B)
(C) edge node[above]{$o$} (B)
(F) edge node[above]{$i'$} (G)
(H) edge node[above]{$o'$} (G)
(D) edge node [above] {$d$} (E)
(D') edge node [above] {$d'$} (E');
\end{tikzpicture}
\]
For these two $F$-decorated cospans to be in the same isomorphism class, the following triangle is to commute:
\[
\begin{tikzpicture}[scale=1.5]
\node (D) at (3,0) {$1$};
\node (E) at (4,0.5) {$F(b)$};
\node (E') at (4,-0.5) {$F(b')$};
\path[->,font=\scriptsize,>=angle 90]
(D) edge node [above] {$d$} (E)
(D) edge node [below] {$d'$} (E')
(E) edge node [right] {$F(f)$} (E');
\end{tikzpicture}
\]
This commutative triangle in $\mathsf{Set}$ in the context of the symmetric lax monoidal functor $F \maps \mathsf{FinSet} \to \mathsf{Set}$ says the following: given a decoration $d \in F(b)$, which is a graph structure with underlying set of vertices $b$, the function $F(f)$ pushes forward the graph structure $d$ to the graph structure $d' \in F(b')$ with underlying set of vertices $b'$, and \emph{precisely} this graph structure. The graph structure is given by set of edges of $d$. For example, if we take $b = \{v_1 ,v_2\}$ as before and let $d \in F(b)$ be given by:
\[
\begin{tikzpicture}
  [scale=.8,auto=left]
\node (a) at (-2,10) {$1$};
\node (b) at (7,10) {$2$};
\node (i) at (-0.5,11) {$i$};
\node (o) at (5.5,11) {$o$};
  \node [style={circle,fill=teal}] (n1) at (1,10) {$v_1$};
  \node[style={circle,fill=teal}] (n2) at (4,10)  {$v_2$};
\path[->,font=\scriptsize,>=angle 90]
(a) edge[dashed] node[above] {$$} (n1)
(b) edge[dashed] node[below] {$$} (n2)
(n1) edge node[above]{$e$} (n2);
\end{tikzpicture}
\]
Let $b' = \{w_1,w_2\}$ and a define bijection $f \maps b \to b'$ by $f(v_i)=w_i$ for $i=1,2$. Then the requirement $F(f)(d)=d'$ says that $d' \in F(b')$ must be given by:
\[
\begin{tikzpicture}
  [scale=.8,auto=left]
\node (a) at (-2,10) {$1$};
\node (b) at (7,10) {$2$};
\node (i) at (-0.5,11) {$i'$};
\node (o) at (5.5,11) {$o'$};
  \node [style={circle,fill=teal}] (n1) at (1,10) {$w_1$};
  \node[style={circle,fill=teal}] (n2) at (4,10)  {$w_2$};
\path[->,font=\scriptsize,>=angle 90]
(a) edge[dashed] node[above] {$$} (n1)
(b) edge[dashed] node[below] {$$} (n2)
(n1) edge node[above]{$e$} (n2);
\end{tikzpicture}
\]
The point to be made here is that the single edge of $d'$ must also be $e$. If we were to label it say, $e'$, there is no bijection $f \maps b \to b'$ such that the triangle on the right commutes, and hence no isomorphism between these two $F$-decorated cospans.
\[
\begin{tikzpicture}\label{distinctisoclasses}
  [scale=.8,auto=left]
\node (D) at (9,11.5) {$1$};
\node (E) at (11,12.5) {$F(b)$};
\node (E') at (11,10.5) {$F(b')$};
\node (a) at (-2,11.5) {$1$};
\node (b) at (7,11.5) {$2$};
\node (i) at (-0.5,12.5) {$i$};
\node (o) at (5.5,12.5) {$o$};
\node (i') at (-0.5,10.5) {$i'$};
\node (o') at (5.5,10.5) {$o'$};
\node (A) at (2.5,11.5) {$\nexists F(f) \Downarrow$};
  \node [style={circle,fill=teal}] (n1) at (1,13) {$v_1$};
  \node[style={circle,fill=teal}] (n2) at (4,13)  {$v_2$};
  \node [style={circle,fill=red}] (n1') at (1,10) {$w_1$};
  \node[style={circle,fill=red}] (n2') at (4,10)  {$w_2$};
\path[->,font=\scriptsize,>=angle 90]
(D) edge node [above] {$d$} (E)
(D) edge node [below] {$d'$} (E')
(E) edge[dashed] node [right] {$\nexists F(f)$} (E')
(a) edge[dashed] node[above] {$$} (n1)
(b) edge[dashed] node[above] {$$} (n2)
(a) edge[dashed] node[below] {$$} (n1')
(b) edge[dashed] node[below] {$$} (n2')
(n1) edge node[above]{$e$} (n2)
(n1') edge node[above]{$e'$} (n2');
\end{tikzpicture}
\]
Thus these two $F$-decorated cospans constitute distinct isomorphism classes. This nuisance is amplified when viewed from a higher categorical perspective.

In a previous work, the second author extended Fong's theory of decorated cospans to a symmetric monoidal bicategory of decorated cospans \cite{Courser}. Namely, under the same hypotheses used by Fong amounting to a category $\A$ with finite colimits and a symmetric lax monoidal functor $F \maps (\A,+,0) \to (\mathsf{Set}, \times, 1)$, there exists a symmetric monoidal bicategory $F\mathbf{Cospan}(\A)$ which has:
\begin{enumerate}
\item objects as those of $\A$,
\item morphisms as $F$-decorated cospans in $\A$, which again are pairs
\[
\begin{tikzpicture}[scale=1.5]
\node (A) at (0,0) {$a_1$};
\node (B) at (1,0) {$b$};
\node (C) at (2,0) {$a_2$};
\node (D) at (3,0) {$1$};
\node (E) at (4,0) {$F(b)$};
\path[->,font=\scriptsize,>=angle 90]
(A) edge node[above]{$i$} (B)
(C) edge node[above]{$o$} (B)
(D) edge node [above] {$d$} (E);
\end{tikzpicture}
\]
and
\item 2-morphisms as pairs of commuting diagrams where the left diagram is a map of cospans in $\A$ and the right diagram is a commutative triangle in $\mathsf{Set}$.
\[
\begin{tikzpicture}[scale=1.5]
\node (A) at (0,0) {$a_1$};
\node (B) at (1,1) {$b$};
\node (B') at (1,-1) {$b'$};
\node (C) at (2,0) {$a_2$};
\node (D) at (3,0) {$1$};
\node (E) at (4,0.5) {$F(b)$};
\node (E') at (4,-0.5) {$F(b')$};
\path[->,font=\scriptsize,>=angle 90]
(A) edge node[above]{$i$} (B)
(C) edge node[above]{$o$} (B)
(A) edge node[below]{$i'$} (B')
(C) edge node[below]{$o'$} (B')
(B) edge node [left] {$f$} (B')
%(B) edge node [right] {$\sim$} (B')
(D) edge node [above] {$d$} (E)
(D) edge node [below] {$d'$} (E')
(E) edge node [right] {$F(f)$} (E');
\end{tikzpicture}
\]
\end{enumerate}
Returning to the previous example where we take the functor $F$ to be the functor $F \maps \mathsf{FinSet} \to \mathsf{Set}$, the situation becomes more dire. Previously, in the ordinary symmetric monoidal category $F$Cospan$(\mathsf{FinSet})$ the following two $F$-decorated cospans resided in distinct isomorphism classes:
\[
\begin{tikzpicture}\label{no2morphism}
  [scale=.8,auto=left]
%\node (D) at (9,11.5) {$1$};
\node (E) at (9,13) {$d \in F(b)$};
\node (E') at (9,10) {$d' \in F(b')$};
\node (a) at (-2,13) {$1$};
\node (a') at (-2,10) {$1$};
\node (b) at (7,13) {$2$};
\node (b') at (7,10) {$2$};
\node (i) at (-0.5,13.5) {$i$};
\node (o) at (5.5,13.5) {$o$};
\node (i') at (-0.5,10.5) {$i'$};
\node (o') at (5.5,10.5) {$o'$};
%\node (A) at (2.5,11.5) {$\nexists F(f) \Downarrow$};
  \node [style={circle,fill=teal}] (n1) at (1,13) {$v_1$};
  \node[style={circle,fill=teal}] (n2) at (4,13)  {$v_2$};
  \node [style={circle,fill=red}] (n1') at (1,10) {$w_1$};
  \node[style={circle,fill=red}] (n2') at (4,10)  {$w_2$};
\path[->,font=\scriptsize,>=angle 90]
%(D) edge node [above] {$d$} (E)
%(D) edge node [below] {$d'$} (E')
%(E) edge[dashed] node [right] {$\nexists F(f)$} (E')
(a) edge[dashed] node[above] {$$} (n1)
(b) edge[dashed] node[above] {$$} (n2)
(a') edge[dashed] node[below] {$$} (n1')
(b') edge[dashed] node[below] {$$} (n2')
(n1) edge node[above]{$e$} (n2)
(n1') edge node[above]{$e'$} (n2');
\end{tikzpicture}
\]
In the symmetric monoidal bicategory $F\mathbf{Cospan}(\mathsf{FinSet})$, the above two $F$-decorated cospans are two morphisms - with \emph{no 2-morphism between them!} The culprit for this is the same as before - the requirement that the triangle of decorations in $\mathsf{Set}$ commute `on the nose'. In this paper we construct a similar symmetric monoidal bicategory $F$Csp which will remedy this situation, and whose decategorification will remedy the analogous problem for decorated cospan categories above. 

It is worth noting that in both examples involving sets and graphs above, we are decorating each finite set with extra \emph{stuff}. This phenomenon does not occur when the decorations only involve extra structure, and this has been utilized in other works \cite{BFP,BP,Yass}.

Our approach is to instead view the functor $F$ as a pseudofunctor $F \maps \textrm{A} \to \Cat$ and take advantage of the 2-categorical structure of $\Cat$. Given an object $a \in \A$, this then allows us to view $F(a)$ not as a \emph{set} of decorations on the object $a$ but as a \emph{category} of decorations on the object $a$. From this pseudofunctor, we construct a symmetric monoidal double category $F \mathbb{C}$sp. Double categories were first introduced by Ehresmann \cite{Ehresmann63, Ehresmann65} and symmetric monoidal double categories by Shulman \cite{Shul}. They have long been used in topology and other branches of pure mathematics \cite{Brown1,Brown2}.  More recently they have been used to study open dynamical systems \cite{LS} and open Markov processes \cite{BC}. While a mere category has only objects and morphisms, a double category has a few more types of entities:
\[
\begin{tikzpicture}[scale=1]
\node (D) at (-4,0.5) {$A$};
\node (E) at (-2,0.5) {$B$};
\node (F) at (-4,-1) {$C$};
\node (A) at (-2,-1) {$D$};
\node (B) at (-3,-0.25) {$\Downarrow a$};
\path[->,font=\scriptsize,>=angle 90]
(D) edge node [above]{$M$}(E)
(E) edge node [right]{$g$}(A)
(D) edge node [left]{$f$}(F)
(F) edge node [above]{$N$} (A);
\end{tikzpicture}
\]
We call $A, B, C$ and $D$ `objects', $f$ and $g$ `vertical 1-morphisms', $M$ and $N$ `horizontal 1-cells', and $a$ a `2-morphism'.   We can compose vertical 1-morphisms to get new vertical 1-morphisms and compose horizontal 1-cells to get new horizontal 1-cells.  We can compose the 2-morphisms in two ways: horizontally by setting squares side by side, and vertically by setting one on top of the other.   In a `strict' double category all these forms of composition are associative.  In a `pseudo' double category, horizontal 1-cells compose in a weakly associative manner: that is, the associative law holds only up to an invertible 2-morphism, called the `associator', which obeys a coherence law.

The symmetric monoidal double category $F \mathbb{C}$sp we construct has:
\begin{enumerate}
\item{objects as those of $\mathrm{A}$,}
\item{vertical 1-morphisms as morphisms of $\mathrm{A}$,}
\item{horizontal 1-cells as $F$-decorated cospans in $\mathrm{A}$
\[
\begin{tikzpicture}[scale=1.5]
\node (A) at (0,0) {$a_1$};
\node (B) at (1,0) {$b$};
\node (C) at (2,0) {$a_2$};
\node (D) at (3,0) {$d \in F(b)$};
\path[->,font=\scriptsize,>=angle 90]
(A) edge node[above]{$i$} (B)
(C) edge node[above]{$o$} (B);
\end{tikzpicture}
\]
and}
\item{2-morphisms as maps of cospans together with a natural transformation.
\[
\begin{tikzpicture}[scale=1.5]
\node (A) at (0,0.5) {$a_1$};
\node (A') at (0,-0.5) {$a_1'$};
\node (B) at (1,0.5) {$b$};
\node (B') at (1,-0.5) {$b'$};
\node (C) at (2,0.5) {$a_2$};
\node (C') at (2,-0.5) {$a_2;$};
\node (D) at (3,0) {$1$};
\node (E) at (4,0.5) {$F(b)$};
\node (E') at (4,-0.5) {$F(b')$};
\node (F) at (3.65,0) {$\Swarrow \tau$};
\path[->,font=\scriptsize,>=angle 90]
(A) edge node[above]{$i$} (B)
(C) edge node[above]{$o$} (B)
(A) edge node {$$} (A')
(C) edge node {$$} (C')
(A') edge node[above]{$i'$} (B')
(C') edge node[above]{$o'$} (B')
(B) edge node [left] {$f$} (B')
(D) edge node [above] {$d$} (E)
(D) edge node [below] {$d'$} (E')
(E) edge node [right] {$F(f)$} (E');
\end{tikzpicture}
\]
}
\end{enumerate}
Note that the natural transformation is a family of morphisms in the category $F(b')$ indexed by $1$, and so is the same as just a morphism $\tau \maps F(f)(d) \to d'$ in $F(b')$. We will typically write the transformation $\tau$ in later sections as a morphism in $F(b')$ to conserve space. In the context of the example above involving graphs, this morphism $\tau$ will be a map between two possibly distinct sets of edges and thus already solves the problem.

This double category is in fact fibrant and so a result of Shulman \cite{Shul} allows us to extract from this symmetric monoidal double category $F \mathbb{C}$sp a symmetric monoidal bicategory $F \mathbf{Csp} \mapseqq H(F \lCsp)$ as the `horizontal edge bicategory' of the double category $F \mathbb{C}$sp. This bicategory $F \mathbf{Csp}$ has:
\begin{enumerate}
\item{objects as those of $\mathrm{A}$,}
\item{morphisms as the horizontal 1-cells of $F\mathbb{C}$sp, which are $F$-decorated cospans of $\A$
\[
\begin{tikzpicture}[scale=1.5]
\node (A) at (0,0) {$a_1$};
\node (B) at (1,0) {$b$};
\node (C) at (2,0) {$a_2$};
\node (D) at (3,0) {$d \in F(b)$};
\path[->,font=\scriptsize,>=angle 90]
(A) edge node[above]{$i$} (B)
(C) edge node[above]{$o$} (B);
\end{tikzpicture}
\]
and}
\item{2-morphisms as maps of $F$-decorated cospans in $\A$ with the same feet together with a natrual transformation.
\[
\begin{tikzpicture}[scale=1.5]
\node (A) at (0,0) {$a_1$};
\node (B) at (1,1) {$b$};
\node (B') at (1,-1) {$b'$};
\node (C) at (2,0) {$a_2$};
\node (D) at (3,0) {$1$};
\node (E) at (4,0.5) {$F(b)$};
\node (E') at (4,-0.5) {$F(b')$};
\node (F) at (3.65,0) {$\Swarrow \tau$};
\path[->,font=\scriptsize,>=angle 90]
(A) edge node[above]{$i$} (B)
(C) edge node[above]{$o$} (B)
(A) edge node[below]{$i'$} (B')
(C) edge node[below]{$o'$} (B')
(B) edge node [left] {$f$} (B')
(D) edge node [above] {$d$} (E)
(D) edge node [below] {$d'$} (E')
(E) edge node [right] {$F(f)$} (E');
\end{tikzpicture}
\]
}
\end{enumerate}
This bicategory $F\mathbf{Csp}$ is superior to the original bicategory $F$Cospan$(A)$ constructed by the second author \cite{Courser} in that it has the 2-morphisms that one would expect such a bicategory to have, such as (\ref{no2morphism}). 

Finally, we can decategorify the symmetric monoidal bicategory $F \mathbf{Csp}$ to obtain a symmetric monoidal category $F \textnormal{Csp} \mapseqq D(F\mathbf{Csp})$. This symmetric monoidal category $F$Csp has:

\begin{enumerate}
\item{objects as those of $\A$ and}
\item{morphisms as isomorphism classes of $F$-decorated cospans in $\A$ with the same feet together with a natural isomorphism, where two $F$-decorated cospans are in the same isomorphism class if $f$ and $\tau$ are isomorphisms.
\[
\begin{tikzpicture}[scale=1.5]
\node (A) at (0,0) {$a_1$};
\node (B) at (1,1) {$b$};
\node (B') at (1,-1) {$b'$};
\node (C) at (2,0) {$a_2$};
\node (D) at (3,0) {$1$};
\node (E) at (4,0.5) {$F(b)$};
\node (E') at (4,-0.5) {$F(b')$};
\node (F) at (3.5,0) {$\Swarrow \tau$};
\path[->,font=\scriptsize,>=angle 90]
(A) edge node[above]{$i$} (B)
(C) edge node[above]{$o$} (B)
(A) edge node[below]{$i'$} (B')
(C) edge node[below]{$o'$} (B')
(B) edge node [left] {$f$} (B')
(B) edge node [right] {$\sim$} (B')
(D) edge node [above] {$d$} (E)
(D) edge node [below] {$d'$} (E')
(E) edge node [right] {$F(f)$} (E')
(E) edge node [left] {$\sim$} (E');
\end{tikzpicture}
\]
}
\end{enumerate}
This symmetric monoidal category $F$Csp has isomorphism classes as one would expect, and in the context of the example with finite sets and graphs above solves the problem in (\ref{distinctisoclasses}). In this particular example, $\tau \maps F(f)(d) \xrightarrow{\sim} d'$ is a bijection between possibly different sets of edges which respects the source and target of each edge.
\fi

\subsection*{Outline}

%{\chris Amend at the end}
%The outline for the paper is as follows: In \cref{DecCospansDoubleCat}, we construct the symmetric monoidal double category $F\lCsp$, obtain the underlying symmetric monoidal bicategory $F \mathbf{Csp}$ via a result of Shulman, and decategorify this symmetric monoidal bicategory to obtain the symmetric monoidal category $F$Csp. In \cref{MapsDecCospansDoubleCat} we define maps between decorated cospan double categories which are double functors of an appropriate type. In \cref{some other section} we investigate when a symmetric lax monoidal pseudofunctor gives rise to a left adjoint via the Grothendieck construction, which naturally leads to another compositional framework known as `Structured cospans'. In \cref{EquivDoubleCats} we briefly review the framework Structured cospans and prove that the double categorical versions of decorated cospans and structured cospans are equivalent when both are appropriate. Finally, in \cref{Applications}, we present a few examples that can be realized with either framework and discuss the realized equivalences.

\subsection*{Conventions}

In this paper, `double category' means `pseudo double category', as in \cref{defn:double_category}.   Following Shulman \cite{Shulman2008}, vertical composition in our double categories is strictly associative, while horizontal composition need not be. We use sans-serif font like $\C$ for categories, boldface like $\B$ for bicategories or 2-categories, and blackboard bold like $\lD$ for double categories. We also use blackboard
bold for weak category objects in any 2-category.  For double categories with names having more than one letter, like $\lCsp(\X)$, only the first letter is in blackboard bold.  A double category $\lD$ has a category of objects and a category of arrows, and we call these $\lD_0$ and $\lD_1$ despite the fact that they are categories.


\subsection*{Acknowledgements}
Daniel Cicala, Brendan Fong...



\section{Decorated cospans}\label{DecCospansDoubleCat}

In this section we build the symmetric monoidal double category $F \lCsp$ mentioned in the introduction, and then study the functoriality of this construction.   The definition of a lax monoidal pseudofunctor is recalled in \cref{sec:2cats}, and the basics of double categories are recalled in \cref{sec:doublecats}.

\begin{thm}\label{thm:decorated_cospans}
Let $\A$ be a category with chosen finite colimits and $F \maps \A \to \Cat$ a lax monoidal pseudofunctor. Then there exists a double category $F\lCsp$ in which
\begin{itemize}
\item an object is an object of $\A$,
\item a vertical 1-morphism is a morphism of $\A$,
\item a horizontal 1-cell is an $F$\define{-decorated cospan}: that is, 
a cospan in $\A$:
\[
\begin{tikzpicture}[scale=1.5]
\node (A) at (0,0) {$a$};
\node (B) at (1,0) {$m$};
\node (C) at (2,0) {$b,$};
%\node (D) at (2.6,0) {$x \in F(m))$};
\path[->,font=\scriptsize,>=angle 90]
(A) edge node[above]{$i$} (B)
(C) edge node[above]{$o$} (B);
\end{tikzpicture}
\]
together with a \define{decoration} $x \in F(m)$,
\item a 2-morphism is a \define{map of} $F$\define{-decorated cospans}: that is, 
a map of cospans in $\A$:
\begin{equation}\label{eq:FCsp2morph}
\begin{tikzpicture}[scale=1.5]
\node (A) at (0,0.5) {$a$};
\node (A') at (0,-0.5) {$a'$};
\node (B) at (1,0.5) {$m$};
\node (C) at (2,0.5) {$b$};
\node (C') at (2,-0.5) {$b'$};
\node (D) at (1,-0.5) {$m'$};
\node (E) at (3,0.5) {$x \in F(m)$};
\node (F) at (3,-0.5) {$x' \in F(m')$};
\path[->,font=\scriptsize,>=angle 90]
(A) edge node[above]{$i$} (B)
(C) edge node[above]{$o$} (B)
(A) edge node[left]{$f$} (A')
(C) edge node[right]{$g$} (C')
(A') edge node[above] {$i'$} (D)
(C') edge node[above] {$o'$} (D)
(B) edge node [left] {$h$} (D);
\end{tikzpicture}
\end{equation}
together with a \define{decoration morphism} $\tau \maps F(h)(x) \to x'$ in $F(m')$.
\end{itemize}
\end{thm}

\begin{proof}
The definition of double category may be found in Definition \ref{defn:double_category}. To show that the pseudofunctor $F$ in the theorem statement induces a double category $F\lCsp$, we must first define the category of objects $F\lCsp_0$, the category of arrows $F\lCsp_1$, as well as the four structure functors which tie them together: the source and target functors $S,T \colon F\lCsp_1 \to F\lCsp_0$, the unit functor $U \colon F\lCsp_0 \to F\lCsp_1$ and the composition functor $\odot$ which says how to compose two composable horizontal 1-cells or two composable 2-morphisms horizontally: $$\odot \colon F\lCsp_1 \times_{F\lCsp_0} F\lCsp_1 \to F\lCsp_1.$$

The category of objects $F\lCsp_0$ is simply the category $\A$. The category of arrows $F\lCsp_1$ has $F$-decorated cospans for objects and maps of $F$-decorated cospans as morphisms. Given two vertically composable $F$-decorated cospans:
\[
\begin{tikzpicture}[scale=1.5]
\node (A) at (0,0.5) {$a$};
\node (A') at (0,-0.5) {$a'$};
\node (B) at (1,0.5) {$m$};
\node (C) at (2,0.5) {$b$};
\node (C') at (2,-0.5) {$b'$};
\node (D) at (1,-0.5) {$m'$};
\node (E) at (3,0.5) {$x \in F(m)$};
\node (F) at (3,-0.5) {$x' \in F(m')$};
\node (G) at (1.5,-1) {$\tau \colon F(h)(x) \to x'$};
\node (A'') at (0,-1.5) {$a'$};
\node (A''') at (0,-2.5) {$a''$};
\node (B'') at (1,-1.5) {$m'$};
\node (C'') at (2,-1.5) {$b'$};
\node (C''') at (2,-2.5) {$b''$};
\node (D'') at (1,-2.5) {$m''$};
\node (E'') at (3,-1.5) {$x' \in F(m')$};
\node (F'') at (3,-2.5) {$x'' \in F(m'')$};
\node (G'') at (1.5,-3) {$\tau' \colon F(h')(x') \to x''$};
\path[->,font=\scriptsize,>=angle 90]
(A) edge node[above]{$i$} (B)
(C) edge node[above]{$o$} (B)
(A) edge node[left]{$f$} (A')
(C) edge node[right]{$g$} (C')
(A') edge node[above] {$i'$} (D)
(C') edge node[above] {$o'$} (D)
(B) edge node [left] {$h$} (D)
(A'') edge node[above]{$i'$} (B'')
(C'') edge node[above]{$o'$} (B'')
(A'') edge node[left]{$f'$} (A''')
(C'') edge node[right]{$g'$} (C''')
(A''') edge node[above] {$i''$} (D'')
(C''') edge node[above] {$o''$} (D'')
(B'') edge node [left] {$h'$} (D'');
\end{tikzpicture}
\]
their composite is given by the composite of the two maps of cospans in $\A$
\[
\begin{tikzpicture}[scale=1.5]
\node (A) at (0,0.5) {$a$};
\node (A') at (0,-0.5) {$a''$};
\node (B) at (1,0.5) {$m$};
\node (C) at (2,0.5) {$b$};
\node (C') at (2,-0.5) {$b''$};
\node (D) at (1,-0.5) {$m''$};
\node (E) at (3,0.5) {$x \in F(m)$};
\node (F) at (3,-0.5) {$x'' \in F(m'')$};
\path[->,font=\scriptsize,>=angle 90]
(A) edge node[above]{$i$} (B)
(C) edge node[above]{$o$} (B)
(A) edge node[left]{$f'f$} (A')
(C) edge node[right]{$g'g$} (C')
(A') edge node[above] {$i''$} (D)
(C') edge node[above] {$o''$} (D)
(B) edge node [left] {$h'h$} (D);
\end{tikzpicture}
\]
together with the composite decoration morphism $\tau' F(\tau) \colon F(h'h)(x) \to x'$ in $F(m'')$.
% this was:
% together with the composite morphism of decorations $\tau' \tau \colon F(h'h)(x) \to x'$ in $F(m'')$
%% \tau' \tau was the notation that I was using, but I see why this was misleading. I believe it should be called \tau'(F(h')(\tau)) \instead of \tau' F(\tau), and also with target x''.
 This completes the descriptions of the category of objects $F\lCsp_0 = \A$ and the category of arrows $F\lCsp_1$.

Next, we define the four structure functors $U,S,T$ and $\odot$. The unit functor $U \maps F\lCsp_0 \to F\lCsp_1$ is defined on objects as: 
\[
\begin{tikzpicture}[scale=1.5]
\node (E) at (-1,0) {$a$};
\node (F) at (-.5,0) {$\mapsto$};
\node (A) at (0,0) {$a$};
\node (B) at (1,0) {$a$};
\node (C) at (2,0) {$a$};
\node (D) at (3,0) {$!_a \in F(a)$};
\path[->,font=\scriptsize,>=angle 90]
(A) edge node[above]{$1$} (B)
(C) edge node[above]{$1$} (B);
\end{tikzpicture}
\]
where $!_a \in F(a)$ is the \define{trivial decoration} on $a$ given by the composite of the unique map $F(!) \maps F(0) \to F(a)$ and the morphism $\phi \maps 1 \to F(0)$  which comes from $F \maps \A \to \Cat$ being lax monoidal.  For morphisms, the functor $U$ is defined as:
\[
\begin{tikzpicture}[scale=1.5]
\node (G) at (-1,0.5) {$a$};
\node (G') at (-1,-0.5) {$a'$};
\node (A) at (0,0.5) {$a$};
\node (A') at (0,-0.5) {$a'$};
\node (B) at (1,0.5) {$a$};
\node (C) at (2,0.5) {$a$};
\node (C') at (2,-0.5) {$a'$};
\node (D) at (1,-0.5) {$a'$};
\node (E) at (3,0.5) {$!_a \in F(a)$};
\node (F) at (3,-0.5) {$!_{a'} \in F(a')$};
\node (H) at (-0.5,0) {$\mapsto$};
\path[->,font=\scriptsize,>=angle 90]
(A) edge node[above]{$1$} (B)
(C) edge node[above]{$1$} (B)
(A) edge node[left]{$f$} (A')
(C) edge node[right]{$f$} (C')
(A') edge node[above] {$1$} (D)
(C') edge node[above] {$1$} (D)
(B) edge node [left] {$f$} (D)
(G) edge node [left] {$f$} (G');
\end{tikzpicture}
\]
together with the morphism $\tau_{!_f} = F(f) F(!) \phi \maps 1 \to F(a')$. For the source and target functors $S, T \maps F\lCsp_1 \to F\lCsp_0$, given an object $F\lCsp_1$, which is an $F$-decorated cospan,
\[
\begin{tikzpicture}[scale=1.5]
\node (A) at (0,0) {$a$};
\node (B) at (1,0) {$m$};
\node (C) at (2,0) {$b$};
\node (D) at (3,0) {$x \in F(m)$};
\path[->,font=\scriptsize,>=angle 90]
(A) edge node[above]{$i$} (B)
(C) edge node[above]{$o$} (B);
\end{tikzpicture}
\]
the source is the object $a$ and the target is the object $b$. Given a morphism of $F\lCsp_1$, which is a map of $F$-decorated cospans,
\[
\begin{tikzpicture}[scale=1.5]
\node (A) at (0,0.5) {$a$};
\node (A') at (0,-0.5) {$a'$};
\node (B) at (1,0.5) {$m$};
\node (C) at (2,0.5) {$b$};
\node (C') at (2,-0.5) {$b'$};
\node (D) at (1,-0.5) {$m'$};
\node (E) at (3,0.5) {$x \in F(m)$};
\node (F) at (3,-0.5) {$x' \in F(m')$};
\path[->,font=\scriptsize,>=angle 90]
(A) edge node[above]{$i$} (B)
(C) edge node[above]{$o$} (B)
(A) edge node[left]{$f$} (A')
(C) edge node[right]{$g$} (C')
(A') edge node[above] {$i'$} (D)
(C') edge node[above] {$o'$} (D)
(B) edge node [left] {$h$} (D);
\end{tikzpicture}
\]
$$\tau \maps F(h)(x) \to x'$$
the source is the morphism $f$ and the target is the morphism $g$. The three functors $S,T$ and $U$ satisfy the equations $SU=\id_{\A}=TU$.

Next we define the composite functor $\odot \colon F\lCsp_1 \times_{F\lCsp_0} F\lCsp_1 \to F\lCsp_1$, which amounts to defining how we compose two composable $F$-decorated cospans or horizontally compose two maps of $F$-decorated cospans with a common source and target. Given two composable $F$-decorated cospans which we respectively denote by $M$ and $N$, the capitals of their apices:
\[
\begin{tikzpicture}[scale=1.5]
\node (A) at (0,0) {$a$};
\node (B) at (1,0) {$m$};
\node (C) at (2,0) {$b$};
\node (D) at (1,-0.5) {$x \in F(m)$};
\node (E) at (3,0) {$b$};
\node (F) at (4,0) {$n$};
\node (G) at (5,0) {$c$};
\node (H) at (4,-0.5) {$y \in F(n)$};
\path[->,font=\scriptsize,>=angle 90]
(A) edge node[above]{$i$} (B)
(C) edge node[above]{$o$} (B)
(E) edge node[above]{$i'$} (F)
(G) edge node[above]{$o'$} (F);
\end{tikzpicture}
\]
the composite $M \odot N$ is given by taking the pushout of the two cospans in $\A$ over the shared foot which is the object $b$:
\begin{equation}\label{eq:dcospanscomposition}
\begin{tikzpicture}[scale=1.5]
\node (A) at (0,0) {$a$};
\node (B) at (1,1) {$m$};
\node (C) at (2,0) {$b$};
\node (D) at (3,1) {$n$};
\node (E) at (4,0) {$c$};
\node (F) at (2,2) {$m+n$};
\node (G) at (2,3) {$m+_{b} n$};
\path[->,font=\scriptsize,>=angle 90]
(A) edge node[above]{$i$} (B)
(C) edge node[above]{$o$} (B)
(C) edge node [above] {$i'$} (D)
(E) edge node [above] {$o'$} (D)
(B) edge node [above] {$j_m$} (F)
(D) edge node [above] {$j_{n}'$} (F)
(F) edge node [left] {$\psi$} (G)
(A) edge[bend left] node [left] {$\psi j_m i$} (G)
(E) edge[bend right] node [right] {$\psi j_{n}' o'$} (G);
\end{tikzpicture}
\end{equation}
The composite decoration $x \odot y$ on the apex $m+_b n$ is given by:
$$x \odot y = 1 \xrightarrow{\lambda^{-1}} 1 \times 1 \xrightarrow{x \times y} F(m) \times F(n) \xrightarrow{\phi_{m,n}} F(m+n) \xrightarrow{F(\psi)} F(m+_{b}n)$$
where $\psi \maps m + n \to m+_{b} n$ is the natural map from the coproduct to the pushout and $\phi_{m,n} \maps F(m) \times F(n) \to F(m+n)$ is the natural transformation coming from the structure of the lax monoidal pseudofunctor $F \maps \A \to \Cat$. Given two horizontally composable maps of $F$-decorated cospans which we respectively denote by $\alpha$ and $\beta$:
\[
\begin{tikzpicture}[scale=1.5]
\node (A) at (0,0.5) {$a$};
\node (A') at (0,-0.5) {$a'$};
\node (B) at (1,0.5) {$m$};
\node (C) at (2,0.5) {$b$};
\node (C') at (2,-0.5) {$b'$};
\node (D) at (1,-0.5) {$m'$};
\node (E) at (3,0.5) {$x \in F(m)$};
\node (F) at (3,-0.5) {$x' \in F(m')$};
\node (G) at (4,0.5) {$b$};
\node (H) at (5,0.5) {$n$};
\node (I) at (6,0.5) {$c$};
\node (G') at (4,-0.5) {$b'$};
\node (H') at (5,-0.5) {$n'$};
\node (I') at (6,-0.5) {$c'$};
\node (J) at (7,0.5) {$y \in F(n)$};
\node (K) at (7,-0.5) {$y' \in F(n')$};
\node (L) at (1,-1) {$\tau_\alpha \maps F(h_1)(x) \to x'$};
\node (M) at (5,-1) {$\tau_\beta \maps F(h_2)(y) \to y'$};
\path[->,font=\scriptsize,>=angle 90]
(A) edge node[above]{$i_1$} (B)
(C) edge node[above]{$o_1$} (B)
(A) edge node[left]{$f$} (A')
(C) edge node[right]{$g$} (C')
(A') edge node[above] {$i_1'$} (D)
(C') edge node[above] {$o_1'$} (D)
(B) edge node [left] {$h_1$} (D)
(G) edge node [above] {$i_2$} (H)
(G) edge node [left] {$g$} (G')
(H) edge node [left] {$h_2$} (H')
(G') edge node [above] {$i_2'$} (H')
(I) edge node [above] {$o_2$} (H)
(I) edge node [right] {$k$} (I')
(I') edge node [above] {$o_2'$} (H');
\end{tikzpicture}
\]
their horizontal composite $\alpha \odot \beta$ is the pair given by first horizontally composing the two maps of cospans in $\A$:
\[
\begin{tikzpicture}[scale=1.5]
\node (A) at (0,0.5) {$a$};
\node (A') at (0,-0.5) {$a'$};
\node (B) at (1.5,0.5) {$m+_{b} n$};
\node (C) at (3,0.5) {$c$};
\node (C') at (3,-0.5) {$c'$};
\node (D) at (1.5,-0.5) {$m'+_{b'}n'$};
\node (E) at (4.5,0.5) {$x \odot y \in F(m+_b n)$};
\node (F) at (4.5,-0.5) {$x' \odot y' \in F(m' +_{b'} n')$};
\path[->,font=\scriptsize,>=angle 90]
(A) edge node[above]{$\psi j_m i_1$} (B)
(C) edge node[above]{$\psi j_n o_2$} (B)
(A) edge node[left]{$f$} (A')
(C) edge node[right]{$k$} (C')
(A') edge node [above]{$\psi j_{m'} i_1'$} (D)
(C') edge node [above]{$\psi j_{n'} o_2'$} (D)
(B) edge node [left] {$h_1 +_g h_2$} (D);
\end{tikzpicture}
\]
together with the decoration morphism $\tau_{\alpha \odot \beta} \colon F(h_1 +_g h_2)(x \odot y) \to (x' \odot y')$ given by the following commutative diagram:
\[
\begin{tikzpicture}[scale=1.5]
\node (A) at (4.75,0) {$\tau_\alpha \times \tau_\beta \Swarrow$};
\node (D) at (3,0) {$1 \xrightarrow{\lambda^{-1}} 1 \times 1$};
\node (E) at (5.5,0.5) {$F(m) \times F(n)$};
\node (E') at (5.5,-0.5) {$F(m') \times F(n')$};
\node (B) at (7.5,0.5) {$F(m+n)$};
\node (B') at (7.5,-0.5) {$F(m' + n')$};
\node (C) at (9.25,0.5) {$F(m+_{b} n)$};
\node (C') at (9.25,-0.5) {$F(m' +_{b'} n')$};
\path[->,font=\scriptsize,>=angle 90]
(E) edge node [above] {$\phi_{m,n}$} (B)
(E') edge node [above] {$\phi_{m',n'}$} (B')
(B) edge node [above] {$F(\psi)$} (C)
(B') edge node [above] {$F(\psi)$} (C')
(C) edge node [right] {$F(h_1 +_g h_2)$} (C')
(B) edge node [right] {$F(h_1 + h_2)$} (B')
(D) edge node [above] {$x \times y$} (E)
(D) edge node [below] {$x' \times y'$} (E')
(E) edge node [right] {$F(h_1) \times F(h_2)$} (E');
\end{tikzpicture}
\]
The middle square commutes since $F$ is a lax monoidal pseudofunctor and the right square commutes as the underlying square commutes. The decorations $x \odot y$ and $x' \odot y'$ are given respectively by the top and bottom composite of arrows and the decoration morphism $\tau_{\alpha \odot \beta}$ is given by composing $\tau_\alpha \times \tau_\beta$ with the two commuting squares. It is clear that the equations $S(M \odot N)=S(M)$ and $T(M \odot N)=T(N)$ hold, and this finishes defining the four structure functors $U,S,T$ and $\odot$.

Lastly, our double category is required to have an associator $\alpha$ (\textbf{overusing alpha, ignoring for now}) and left and right unitors $\lambda$ and $\rho$ which are natural isomorphisms governing the composition of horizontal 1-cells that are required to satisfy the standard coherence axioms for a monoidal category, as well as $S(\alpha),S(\lambda),S(\rho),T(\alpha),T(\lambda)$ and $T(\rho)$ all being identities. We will only show the associator $\alpha$, left unitor $\lambda$ and the pentagon identity, as the remaining details will be similar to these.

First we will build the associator $\alpha \colon \odot (\odot \times 1) \Rightarrow \odot (1 \times \odot)$, which will be a globular 2-morphism; in other words, a map of $F$-decorated cospans with identities on the sides. Given three composable horizontal 1-cells $M_1, M_2$ and $M_3$:
\[
\begin{tikzpicture}[scale=1.5]
\node (A) at (0,0) {$a$};
\node (B) at (1,0) {$m_1$};
\node (C) at (2,0) {$b$};
\node (D) at (1,-0.5) {$x_1 \in F(m_1)$};
\node (E) at (3,0) {$b$};
\node (F) at (4,0) {$m_2$};
\node (G) at (5,0) {$c$};
\node (H) at (4,-0.5) {$x_2 \in F(m_2)$};
\node (I) at (6,0) {$c$};
\node (J) at (7,0) {$m_3$};
\node (K) at (8,0) {$d$};
\node (L) at (7,-0.5) {$x_3 \in F(m_3)$};
\path[->,font=\scriptsize,>=angle 90]
(A) edge node[above]{$i$} (B)
(C) edge node[above]{$o$} (B)
(E) edge node[above]{$i'$} (F)
(G) edge node[above]{$o'$} (F)
(I) edge node[above]{$i''$} (J)
(K) edge node[above]{$o''$} (J);
\end{tikzpicture}
\]
we first compute $(M_1 \odot M_2) \odot M_3)$ by pushing out the first two cospans followed by the third:
\[
\begin{tikzpicture}[scale=1.5]
\node (A) at (0,0) {$a$};
\node (B) at (2,0) {$(m_1 +_b m_2) +_c m_3$};
\node (C) at (4,0) {$d$};
%\node (D) at (3,0) {$x \in F(m)$};
\path[->,font=\scriptsize,>=angle 90]
(A) edge node[above]{$$} (B)
(C) edge node[above]{$$} (B);
\end{tikzpicture}
\]
as well as the decoration $(x_1 \odot x_2) \odot x_3$ on the apex obtained as explained above. Explicitly, the decoration $(x_1 \odot x_2) \odot x_3) \in F((m_1+_b m_2) +_c m_3)$ is obtained as the composite $$ (x_1 \odot x_2) \odot x_3 \mapseqq 1 \xrightarrow{\zeta_1} F(m_1+_{b} m_2) \times F(m_3) \xrightarrow{\phi_{m_1+_{b} m_2, m_3}} F((m_1+_{b}m_2) +m_3) \xrightarrow{F(\psi)} F((m_1+_{b} m_2)+_{c} m_3)$$where$$\scriptstyle{\zeta_1 \mapseqq 1 \xrightarrow{\lambda^{-1}} 1 \times 1 \xrightarrow{x_1 \times x_2} F(m_1) \times F(m_2) \xrightarrow{\phi_{m_1,m_2}} F(m_1+m_2) \xrightarrow{F(j)} F(m_1+_b m_2) \xrightarrow {\rho^{-1}} F(m_1+_b m_2) \times 1 \xrightarrow{1 \times x_3} F(m_1+_b m_2) \times F(m_3).}$$
The composite $M_1 \odot (M_2 \odot M_3)$ is obtained similarly:
\[
\begin{tikzpicture}[scale=1.5]
\node (A) at (0,0) {$a$};
\node (B) at (2,0) {$m_1 +_b (m_2 +_c m_3)$};
\node (C) at (4,0) {$d$};
%\node (D) at (3,0) {$x \in F(m)$};
\path[->,font=\scriptsize,>=angle 90]
(A) edge node[above]{$$} (B)
(C) edge node[above]{$$} (B);
\end{tikzpicture}
\]
as well as the decoration $x_1 \odot (x_2 \odot x_3)$ on the apex:
$$x_1 \odot (x_2 \odot x_3) \mapseqq 1 \xrightarrow{\zeta_2} F(m_1) \times F(m_2 +_c m_3) \xrightarrow{\phi_{m_1, m_2 +_c m_3}} F(m_1+(m_2 +_c +m_3)) \xrightarrow{F(\psi)} F(m_1+_{b} (m_2+_{c} m_3))$$with $\zeta_2$ being similar to $\zeta_1$ above. The associator $\alpha$ is then given by the map of $F$-decorated cospans given by the map of cospans
\[
\begin{tikzpicture}[scale=1.5]
\node (A) at (0,0.5) {$a$};
\node (A') at (0,-0.5) {$a$};
\node (B) at (1.5,0.5) {$(m_1+_{b} m_2)+_{c} m_3$};
\node (C) at (3,0.5) {$d$};
\node (C') at (3,-0.5) {$d$};
\node (D) at (1.5,-0.5) {$m_1+_{b}(m_2 +_{c} m_3)$};
\node (E) at (5.5,0.5) {$(x_1 \odot x_2) \odot x_3 \in F((m_1+_{b} m_2)+_{c} m_3)$};
\node (F) at (5.5,-0.5) {$x_1 \odot (x_2 \odot x_3) \in F(m_1+_{b} (m_2 +_{c} m_3))$};
\path[->,font=\scriptsize,>=angle 90]
(A) edge node[above]{$$} (B)
(C) edge node[above]{$$} (B)
(A) edge node[left]{$\id_a$} (A')
(C) edge node[right]{$\id_d$} (C')
(A') edge node {$$} (D)
(C') edge node {$$} (D)
(B) edge node [left] {$\alpha$} (D);
\end{tikzpicture}
\]
together with the decoration morphism $\tau_\alpha \colon F(\alpha)((x_1 \odot x_2) \odot x_3) \to (x_1 \odot (x_2 \odot x_3))$ that is obtained by pasting various commutative squares together, as we did for a horizontal composite above. The associator $\alpha$ will then satisfy the pentagon identity:
\[
\begin{tikzpicture}[scale=1.5]
\node (A) at (-0.5,0) {$((M_1 \odot M_2) \odot M_3) \odot M_4$};
\node (B) at (3,1.05) {$(M_1 \odot (M_2 \odot M_3)) \odot M_4$};
\node (C) at (6.5,0) {$M_1 \odot ((M_2 \odot M_3) \odot M_4)$};
\node (D) at (1.5,-1.05) {$(M_1 \odot M_2) \odot (M_3 \odot M_4)$};
\node (E) at (4.5,-1.05) {$M_1 \odot (M_2 \odot (M_3 \odot M_4))$};
\path[->,font=\scriptsize,>=angle 90]
(A) edge node [above] {$\alpha \times 1$}(B)
(B) edge node [above] {$\alpha$}(C)
(A) edge node [below] {$\alpha$}(D)
(D) edge node [above] {$\alpha$}(E)
(C) edge node [below] {$1 \times \alpha$}(E);
\end{tikzpicture}
\]
which we may realize in two stages. First, for the underlying pentagon of maps of cospans in $\A$ commuting due to the fact that $\lCsp(\A)$ is a double category with its own associator that satisfies the pentagon identity, and second, for the pentagon of morphisms of decorations due to the pasting of various commutative squares in $\Cat$.

We next construct the left unitor. Given a horizontal 1-cell $M$:
\[
\begin{tikzpicture}[scale=1.5]
\node (A) at (0,0) {$a$};
\node (B) at (1,0) {$m$};
\node (C) at (2,0) {$b$};
\node (D) at (3,0) {$x \in F(m)$};
\path[->,font=\scriptsize,>=angle 90]
(A) edge node[above]{$i$} (B)
(C) edge node[above]{$o$} (B);
\end{tikzpicture}
\]
the left unitor will be a globular 2-morphism $\lambda \colon 1_a \odot M \to M$, where the identity horizontal 1-cell on the object $a$ will once again make use of the trivial decoration on an object introduced earlier.
\[
\begin{tikzpicture}[scale=1.5]
\node (A) at (0,0) {$a$};
\node (B) at (1,0) {$a$};
\node (C) at (2,0) {$a$};
\node (D) at (3,0) {$!_a \in F(a)$};
\path[->,font=\scriptsize,>=angle 90]
(A) edge node[above]{$\id_a$} (B)
(C) edge node[above]{$\id_a$} (B);
\end{tikzpicture}
\]
To compose $1_a$ and $M$, we first compose the two cospans in $\A$ by pushing out over their shared object $a$:
\[
\begin{tikzpicture}[scale=1.5]
\node (A) at (0,0) {$a$};
\node (B) at (1,1) {$a$};
\node (C) at (2,0) {$a$};
\node (D) at (3,1) {$m$};
\node (E) at (4,0) {$b$};
\node (F) at (2,2) {$a+m$};
\node (G) at (2,3) {$a+_{a} m$};
\path[->,font=\scriptsize,>=angle 90]
(A) edge node[above]{$\id_a$} (B)
(C) edge node[above]{$\id_a$} (B)
(C) edge node [above] {$i$} (D)
(E) edge node [above] {$o$} (D)
(B) edge node [above] {$j_a$} (F)
(D) edge node [above] {$j_m$} (F)
(F) edge node [left] {$\psi$} (G)
(A) edge[bend left] node [left] {$\psi j_a$} (G)
(E) edge[bend right] node [right] {$\psi j_{m} o$} (G);
\end{tikzpicture}
\]
The composite decoration $!_a \odot x$ will be given by the composite
$$1 \xrightarrow{\lambda^{-1}} 1 \times 1 \xrightarrow{!_a \times x} F(a) \times F(m) \xrightarrow{\phi_{a,m}} F(a+m) \xrightarrow{F(\psi)} F(a+_{a} m).$$
The left unitor is then the map of $F$-decorated cospans determined by the following map of cospans in $\A$:
\[
\begin{tikzpicture}[scale=1.5]
\node (A) at (0,0.5) {$a$};
\node (A') at (0,-0.5) {$a$};
\node (B) at (1.5,0.5) {$a+_{a} m$};
\node (C) at (3,0.5) {$b$};
\node (C') at (3,-0.5) {$b$};
\node (D) at (1.5,-0.5) {$m$};
\node (E) at (4.5,0.5) {$!_a \odot x \in F(a+_{a} m)$};
\node (F) at (4.5,-0.5) {$x \in F(m)$};
\path[->,font=\scriptsize,>=angle 90]
(A) edge node[above]{$\psi j_a$} (B)
(C) edge node[above]{$\psi j_m o$} (B)
(A) edge node[left]{$\id_{a}$} (A')
(C) edge node[right]{$\id_{b}$} (C')
(A') edge node [above]{$i$} (D)
(C') edge node [above]{$o$} (D)
(B) edge node [left] {$\ell$} (D);
\end{tikzpicture}
\]
together with the decoration morphism $\tau_\lambda \colon F(\ell)(!_a \odot x) \to x$; note, that $\ell \colon a+_a m \to m$ is a canonical isomorphism as both $a+_a m$ and $m$ are colimits of the span $a \xleftarrow{\id_a} a \xrightarrow{i} m$. The right unitor $\rho$ is similar, the two unitality squares are similar to the associator shown above, and it is clear that the source and target functors applied to the associator and left and right unitors yield identities.

The final property that we will check is the interchange law for 2-morphisms. Suppose that we have four 2-morphisms $\alpha, \beta, \alpha'$ and $\beta'$ as shown below:
\[
\begin{tikzpicture}[scale=1.5]
\node (A) at (0,0.5) {$a$};
\node (A') at (0,-0.5) {$a'$};
\node (B) at (1,0.5) {$m$};
\node (C) at (2,0.5) {$b$};
\node (C') at (2,-0.5) {$b'$};
\node (D) at (1,-0.5) {$m'$};
\node (E) at (3,0.5) {$x \in F(m)$};
\node (F) at (3,-0.5) {$x' \in F(m')$};
\node (G) at (4,0.5) {$b$};
\node (H) at (5,0.5) {$n$};
\node (I) at (6,0.5) {$c$};
\node (G') at (4,-0.5) {$b'$};
\node (H') at (5,-0.5) {$n'$};
\node (I') at (6,-0.5) {$c'$};
\node (J) at (7,0.5) {$y \in F(n)$};
\node (K) at (7,-0.5) {$y' \in F(n')$};
\node (L) at (1,-1) {$\tau_\alpha \maps F(h_1)(x) \to x'$};
\node (M) at (5,-1) {$\tau_\beta \maps F(h_2)(y) \to y'$};
\node (A'') at (0,-1.5) {$a'$};
\node (A''') at (0,-2.5) {$a''$};
\node (B'') at (1,-1.5) {$m'$};
\node (C'') at (2,-1.5) {$b'$};
\node (C''') at (2,-2.5) {$b''$};
\node (D'') at (1,-2.5) {$m''$};
\node (E'') at (3,-1.5) {$x' \in F(m')$};
\node (F'') at (3,-2.5) {$x'' \in F(m'')$};
\node (G'') at (4,-1.5) {$b'$};
\node (H'') at (5,-1.5) {$n'$};
\node (I'') at (6,-1.5) {$c'$};
\node (G''') at (4,-2.5) {$b''$};
\node (H''') at (5,-2.5) {$n''$};
\node (I''') at (6,-2.5) {$c''$};
\node (J'') at (7,-1.5) {$y' \in F(n')$};
\node (K'') at (7,-2.5) {$y'' \in F(n'')$};
\node (L'') at (1,-3) {$\tau_{\alpha'} \maps F(h_1')(x') \to x''$};
\node (M'') at (5,-3) {$\tau_{\beta'} \maps F(h_2')(y') \to y''$};
\path[->,font=\scriptsize,>=angle 90]
(A) edge node[above]{$i_1$} (B)
(C) edge node[above]{$o_1$} (B)
(A) edge node[left]{$f$} (A')
(C) edge node[right]{$g$} (C')
(A') edge node[above] {$i_1'$} (D)
(C') edge node[above] {$o_1'$} (D)
(B) edge node [left] {$h_1$} (D)
(G) edge node [above] {$i_2$} (H)
(G) edge node [left] {$g$} (G')
(H) edge node [left] {$h_2$} (H')
(G') edge node [above] {$i_2'$} (H')
(I) edge node [above] {$o_2$} (H)
(I) edge node [right] {$k$} (I')
(I') edge node [above] {$o_2'$} (H')
(A'') edge node[above]{$i_1'$} (B'')
(C'') edge node[above]{$o_1'$} (B'')
(A'') edge node[left]{$f'$} (A''')
(C'') edge node[right]{$g'$} (C''')
(A''') edge node[above] {$i_1''$} (D'')
(C''') edge node[above] {$o_1''$} (D'')
(B'') edge node [left] {$h_1'$} (D'')
(G'') edge node [above] {$i_2'$} (H'')
(G'') edge node [left] {$g'$} (G''')
(H'') edge node [left] {$h_2'$} (H''')
(G''') edge node [above] {$i_2''$} (H''')
(I'') edge node [above] {$o_2'$} (H'')
(I'') edge node [right] {$k'$} (I''')
(I''') edge node [above] {$o_2''$} (H''');
\end{tikzpicture}
\]
We may compose these first vertically and then horizontally, or first horizontally and then vertically, and we must show that these two ways of composing yield the same result. Composing first vertically and then horizontally results in:
\[
\begin{tikzpicture}[scale=1.5]
\node (A) at (0,0.5) {$a$};
\node (A') at (0,-0.5) {$a''$};
\node (B) at (2,0.5) {$m+_{b} n$};
\node (C) at (4,0.5) {$c$};
\node (C') at (4,-0.5) {$c''$};
\node (D) at (2,-0.5) {$m'' +_{b''} n''$};
\node (E) at (5.5,0.5) {$x \odot y \in F(m+_{b} n)$};
\node (F) at (5.5,-0.5) {$x'' \odot y'' \in F(m'' +_{b''} n'')$};
\node (G) at (2,-1) {$\tau_{(\alpha' \alpha) \odot (\beta' \beta)} \maps F((h_1' h_1)+_{g' g} (h_2' h_2))(x \odot y) \to x'' \odot y''$};
\path[->,font=\scriptsize,>=angle 90]
(A) edge node[above]{$ \psi j_m i_1$} (B)
(C) edge node[above]{$ \psi j_n o_2$} (B)
(A) edge node[left]{$f' f$} (A')
(C) edge node[right]{$k' k$} (C')
(A') edge node [above]{$\psi j_{m''} i_1''$} (D)
(C') edge node [above]{$\psi j_{n''} o_2''$} (D)
(B) edge node [left] {$(h_1' h_1) +_{g' g} (h_2' h_2)$} (D);
\end{tikzpicture}
\]
whereas first composing horizontally and then vertically results in:
\[
\begin{tikzpicture}[scale=1.5]
\node (A) at (0,0.5) {$a$};
\node (A') at (0,-0.5) {$a''$};
\node (B) at (2,0.5) {$m+_{b} n$};
\node (C) at (4,0.5) {$c$};
\node (C') at (4,-0.5) {$c''$};
\node (D) at (2,-0.5) {$m'' +_{b''} n''$};
\node (E) at (5.5,0.5) {$x \odot y \in F(m+_{b} n)$};
\node (F) at (5.5,-0.5) {$x'' \odot y'' \in F(m'' +_{b''} n'')$};
\node (G) at (2,-1) {$\tau_{(a' \odot \beta')(\alpha \odot \beta)} \maps F((h_1' +_{g'} h_2')(h_1 +_g h_2))(x \odot y) \to x'' \odot y''.$};
\path[->,font=\scriptsize,>=angle 90]
(A) edge node[above]{$\psi j_{m} i_1$} (B)
(C) edge node[above]{$\psi j_{n} o_2$} (B)
(A) edge node[left]{$f' f$} (A')
(C) edge node[right]{$k' k$} (C')
(A') edge node [above]{$\psi j_{m''} i_1''$} (D)
(C') edge node [above]{$\psi j_{n''} o_2''$} (D)
(B) edge node [left] {$(h_1' +_{g'} h_2')(h_1 +_g h_2)$} (D);
\end{tikzpicture}
\]
Everything about these two composites appears identical except for the two middle morphisms and the two morphisms of decorations. However, the two maps $(h_1' h_1) +_{g' g} (h_2' h_2) \maps m+_{b} n \to m'' +_{b''}n''$ and $(h_1' +_{g'} h_2')(h_1 +_g h_2) \maps m+_{b} n \to m'' +_{b''} n''$ are the same universal map realized in two different ways. The two morphisms of decorations $\tau_{(\alpha' \odot \beta')(\alpha \odot \beta)}$ and $\tau_{(\alpha' \alpha) \odot (\beta' \beta)}$ are also the same as they are obtained as two different compositions of four 2-morphisms in $\Cat$, namely first vertically and then horizontally and first horizontally and then vertically. The interchange law for $\Cat$ says that these two composites must be the same as well, and thus the interchange law for our double category $F\lCsp$ is also satisfied.
 %\noindent
%\textbf{In progress...}
%we get a natural isomorphism $a_{M_1,M_2,M_3} \maps (M_1 \odot M_2) \odot M_3 \to M_1 \odot (M_2 \odot M_3)$ which is a globular 2-morphism given by a map of cospans $(\id_{a_1},\sigma,\id_{a_4})$:
%with the decorations on the cospan's apices given by:
%$$ (x \odot y) \odot z \mapseqq 1 \xrightarrow{\zeta_1} F(m_1+_{b} m_2) \times F(m_3) \xrightarrow{\phi_{m_1+_{b} m_2, m_3}} F((m_1+_{b}m_2) +m_3) \xrightarrow{F(j_{m_1+_{b} m_2,m_3})} F((m_1+_{b} m_2)+_{c} m_3)$$ $$\zeta_1 = (1 \times z) \rho^{-1} F(j_{m_1,m_2}) \phi_{m_1,m_2} (x \times y) \lambda^{-1}$$
%and
%$$ x \odot (y \odot z) \mapseqq 1 \xrightarrow{\zeta_2} F(m_1) \times F(m_2 +_{c} m_3) \xrightarrow{\phi_{m_1, m_2 +_{c} m_3}} F(m_1+(m_2 +_{c} m_3)) \xrightarrow{F(j_{m_1,m_2 +_{c} m_3})} F(m_1+_{b} (m_2+_{c} m_3))$$ $$\zeta_2 = (x \times 1) \lambda^{-1} F(j_{m_2,m_3}) \phi_{m_2,m_3} (y \odot z) \lambda^{-1}$$
%together with the isomorphism $\tau_\sigma \maps F(\sigma)((x \odot y) \odot z) \to x \odot (y \odot z)$. Note that the map $\sigma \maps (m_1+_{b} m_2)+_{c}m_3 \to m_1+_{b}(m_2+_{c}m_3)$ is the universal map between two colimits of the same diagram. 
%\noindent
%We also have left and right unitors where given a horizontal 1-cell $M$:
%\[
%\begin{tikzpicture}[scale=1.5]
%\node (A) at (0,0) {$a$};
%\node (B) at (1,0) {$m$};
%\node (C) at (2,0) {$b$};
%\node (D) at (3,0) {$x \in F(m)$};
%\path[->,font=\scriptsize,>=angle 90]
%(A) edge node[above]{$i$} (B)
%(C) edge node[above]{$o$} (B);
%\end{tikzpicture}
%\]
%if we, say, compose with the identity horizontal 1-cell of $b$ on the right:
%\[
%\begin{tikzpicture}[scale=1.5]
%\node (A) at (0,0) {$a$};
%\node (B) at (1,0) {$m$};
%\node (C) at (2,0) {$b$};
%\node (D) at (1,-0.5) {$x \in F(m)$};
%\node (E) at (3,0) {$b$};
%\node (F) at (4,0) {$b$};
%\node (G) at (5,0) {$b$};
%\node (H) at (4,-0.5) {$!_{b} \in F(b)$};
%\path[->,font=\scriptsize,>=angle 90]
%(A) edge node[above]{$i$} (B)
%(C) edge node[above]{$o$} (B)
%(E) edge node[above]{$1$} (F)
%(G) edge node[above]{$1$} (F);
%\end{tikzpicture}
%\]
%where $!_{b} = F(!)  \phi \maps 1 \to F(b)$ is the trivial decoration on $b$, composing these then gives:
%\[
%\begin{tikzpicture}[scale=1.5]
%\node (A) at (0,0) {$a$};
%\node (B) at (1,0) {$m+_{b} b$};
%\node (C) at (2,0) {$b$};
%\node (D) at (3.5,0) {$x \odot !_b \in F(m +_{b} b)$};
%\path[->,font=\scriptsize,>=angle 90]
%(A) edge node[above]{$j \psi_m i$} (B)
%(C) edge node[above]{$j \psi_{b}$} (B);
%\end{tikzpicture}
%\]
%where $\psi_m \maps m \to m+b$ is the natural map into the coproduct and likewise for $\psi_{b}$ and $j \maps m+b \to m+_{b} b$ is the natural map from the coproduct to the pushout. The decoration $x \odot !_b \maps 1 \to F(m+_{b} b)$ is given by: $$1 \xrightarrow{\lambda^{-1}} 1 \times 1 \xrightarrow{x \times !_{b}} F(m) \times F(b) \xrightarrow{\phi_{m,b}} F(m+b) \xrightarrow{F(j_{m,b})} F(m+_{b} b).$$ We then have that the right unitor $R \maps M \odot 1_{b} \xrightarrow{\sim} M$ is given by the globular 2-morphism $(\id_{a},r,\id_{b})$ from the above composite to $M$:
%\[
%\begin{tikzpicture}[scale=1.5]
%\node (A) at (0,0.5) {$a$};
%\node (A') at (0,-0.5) {$a$};
%\node (B) at (1.5,0.5) {$m+_{b} b$};
%\node (C) at (3,0.5) {$b$};
%\node (C') at (3,-0.5) {$b$};
%\node (D) at (1.5,-0.5) {$m$};
%\node (E) at (4.5,0.5) {$x \odot !_b \in F(m+_{b} b)$};
%\node (F) at (4.5,-0.5) {$x \in F(m)$};
%\path[->,font=\scriptsize,>=angle 90]
%(A) edge node[above]{$j \psi_m i$} (B)
%(C) edge node[above]{$j \psi_{b}$} (B)
%(A) edge node[left]{$\id_{a}$} (A')
%(C) edge node[right]{$\id_{b}$} (C')
%(A') edge node [above]{$i$} (D)
%(C') edge node [above]{$o$} (D)
%(B) edge node [left] {$r$} (D);
%\end{tikzpicture}
%\]
%where $r \maps m+_{b} b \xrightarrow{\sim} m$ is a universal map together with the isomorphism $\tau_r \maps F(r)(x \odot !_b) \to x$. The left unitor is similar. The source and target functor applied to the left and right unitors and associators yield identities, and the left and right unitors together with the associator satisfy the standard pentagon and triangle identities of a monoidal category or bicategory. Finally, for the interchange law, given four 2-morphisms $\alpha, \beta, \alpha'$ and $\beta'$:
%if we first compose horizontally we obtain:
%\[
%\begin{tikzpicture}[scale=1.5]
%\node (A) at (0,0.5) {$a$};
%\node (A') at (0,-0.5) {$a'$};
%\node (B) at (1.5,0.5) {$m+_{b} n$};
%\node (C) at (3,0.5) {$c$};
%\node (C') at (3,-0.5) {$c'$};
%\node (D) at (1.5,-0.5) {$m' +_{b'} n'$};
%\node (E) at (4.5,0.5) {$x \odot y \in F(m+_{b} n)$};
%\node (F) at (4.5,-0.5) {$x' \odot y' \in F(m' +_{b'} n')$};
%\node (G) at (1.5,-1) {$\tau_{\alpha \odot \beta} \maps F(h_1 +_g h_2)(x \odot y) \to x' \odot y'$};
%\node (A'') at (0,-1.5) {$a'$};
%\node (A''') at (0,-2.5) {$a''$};
%\node (B'') at (1.5,-1.5) {$m' +_{b\prime} n'$};
%\node (C'') at (3,-1.5) {$c'$};
%\node (C''') at (3,-2.5) {$c''$};
%\node (D'') at (1.5,-2.5) {$m'' +_{b''} n''$};
%\node (E'') at (4.5,-1.5) {$x' \odot y' \in F(m'+_{b'} n')$};
%\node (F'') at (4.5,-2.5) {$x'' \odot y'' \in F(m'' +_{b''} n'')$};
%\node (G'') at (1.5,-3) {$\tau_{\alpha' \odot \beta'} \maps F(h_1' +_{g'} h_2')(x' \odot y') \to x'' \odot y''.$};
%\path[->,font=\scriptsize,>=angle 90]
%(A) edge node[above]{$j \psi_{m} i_1$} (B)
%(C) edge node[above]{$j \psi_{n} o_2$} (B)
%(A) edge node[left]{$f$} (A')
%(C) edge node[right]{$k$} (C')
%(A') edge node [above]{$j \psi_{m'} i_1'$} (D)
%(C') edge node [above]{$j \psi_{n'} o_2'$} (D)
%(B) edge node [left] {$h_1 +_g h_2$} (D)
%(A'') edge node[above]{$j \psi_{m'} i_1'$} (B'')
%(C'') edge node[above]{$j \psi_{n'} o_2'$} (B'')
%(A'') edge node[left]{$f'$} (A''')
%(C'') edge node[right]{$k'$} (C''')
%(A''') edge node [above]{$j \psi_{m''} i_1''$} (D'')
%(C''') edge node [above]{$j \psi_{n''} o_2''$} (D'')
%(B'') edge node [left] {$h_1' +_{g'} h_2'$} (D'');
%\end{tikzpicture}
%\]
%To obtain the morphism of decorations for a horizontal composite, we have as initial data:
%\[
%\begin{tikzpicture}[scale=1.5]
%\node (A) at (4,0) {$\tau_\alpha \Swarrow$};
%\node (D) at (3,0) {$1$};
%\node (E) at (4.5,0.5) {$F(m)$};
%\node (E') at (4.5,-0.5) {$F(m')$};
%\node (A') at (6.5,0) {$\tau_\beta \Swarrow$};
%\node (D') at (5.5,0) {$1$};
%\node (E'') at (7,0.5) {$F(n)$};
%\node (E''') at (7,-0.5) {$F(n')$};
%\path[->,font=\scriptsize,>=angle 90]
%(D) edge node [above] {$x$} (E)
%(D) edge node [below] {$x'$} (E')
%(E) edge node [right] {$F(h_1)$} (E')
%(D') edge node [above] {$y$} (E'')
%(D') edge node [below] {$y'$} (E''')
%(E'') edge node [right] {$F(h_2)$} (E''');
%\end{tikzpicture}
%\]
%These two 2-morphisms $\tau_\alpha$ and $\tau_\beta$ are two 2-morphisms in the monoidal 2-category $(\Cat,\times,1)$ and so we can tensor them which results in:
%Returning to the interchange law, composing the two horizontal compositions above vertically then results in:
%\[
%\begin{tikzpicture}[scale=1.5]
%\node (A) at (0,0.5) {$a$};
%\node (A') at (0,-0.5) {$a''$};
%\node (B) at (2,0.5) {$m+_{b} n$};
%\node (C) at (4,0.5) {$c$};
%\node (C') at (4,-0.5) {$c''$};
%\node (D) at (2,-0.5) {$m'' +_{b''} n''$};
%\node (E) at (5.5,0.5) {$x \odot y \in F(m+_{b} n)$};
%\node (F) at (5.5,-0.5) {$x'' \odot y'' \in F(m'' +_{b''} n'')$};
%\node (G) at (2,-1) {$\tau_{(a' \odot \beta')(\alpha \odot \beta)} \maps F((h_1' +_{g'} h_2')(h_1 +_g h_2))(x \odot y) \to x'' \odot y''.$};
%\path[->,font=\scriptsize,>=angle 90]
%(A) edge node[above]{$j \psi_{m} i_1$} (B)
%(C) edge node[above]{$j \psi_{n} o_2$} (B)
%(A) edge node[left]{$f' f$} (A')
%(C) edge node[right]{$k' k$} (C')
%(A') edge node [above]{$j \psi_{m''} i_1''$} (D)
%(C') edge node [above]{$j \psi_{n''} o_2''$} (D)
%(B) edge node [left] {$(h_1' +_{g'} h_2')(h_1 +_g h_2)$} (D);
%\end{tikzpicture}
%\]
%The vertical composite of two morphisms of decorations is straightforward. On the other hand, if we first compose vertically we obtain:
%\[
%\begin{tikzpicture}[scale=1.5]
%\node (A) at (0,0.5) {$a$};
%\node (A') at (0,-0.5) {$a''$};
%\node (B) at (1,0.5) {$m$};
%\node (C) at (2,0.5) {$b$};
%\node (C') at (2,-0.5) {$b''$};
%\node (D) at (1,-0.5) {$m''$};
%\node (E) at (3,0.5) {$x \in F(m)$};
%\node (F) at (3,-0.5) {$x'' \in F(m'')$};
%\node (G) at (4,0.5) {$b$};
%\node (H) at (5,0.5) {$n$};
%\node (I) at (6,0.5) {$c$};
%\node (G') at (4,-0.5) {$b''$};
%\node (H') at (5,-0.5) {$n''$};
%\node (I') at (6,-0.5) {$c''$};
%\node (J) at (7,0.5) {$y \in F(n)$};
%\node (K) at (7,-0.5) {$y'' \in F(n'')$};
%\node (L) at (1,-1) {$\tau_{\alpha' \alpha} \maps F(h_1' h_1)(x) \to x''$};
%\node (M) at (5,-1) {$\tau_{\beta' \beta} \maps F(h_2' h_2)(y) \to y''$};
%\path[->,font=\scriptsize,>=angle 90]
%(A) edge node[above]{$i_1$} (B)
%(C) edge node[above]{$o_1$} (B)
%(A) edge node[left]{$f' f$} (A')
%(C) edge node[right]{$g' g$} (C')
%(A') edge node[above] {$i_1''$} (D)
%(C') edge node[above] {$o_1''$} (D)
%(B) edge node [left] {$h_1' h_1$} (D)
%(G) edge node [above] {$i_2$} (H)
%(G) edge node [left] {$g' g$} (G')
%(H) edge node [left] {$h_2' h_2$} (H')
%(G') edge node [above] {$i_2''$} (H')
%(I) edge node [above] {$o_2$} (H)
%(I) edge node [right] {$k' k$} (I')
%(I') edge node [above] {$o_2''$} (H');
%\end{tikzpicture}
%\]
%and then composing horizontally results in:
%\[
%\begin{tikzpicture}[scale=1.5]
%\node (A) at (0,0.5) {$a$};
%\node (A') at (0,-0.5) {$a''$};
%\node (B) at (2,0.5) {$m+_{b} n$};
%\node (C) at (4,0.5) {$c$};
%\node (C') at (4,-0.5) {$c''$};
%\node (D) at (2,-0.5) {$m'' +_{b''} n''$};
%\node (E) at (5.5,0.5) {$x \odot y \in F(m+_{b} n)$};
%\node (F) at (5.5,-0.5) {$x'' \odot y'' \in F(m'' +_{b''} n'')$};
%\node (G) at (2,-1) {$\tau_{(\alpha' \alpha) \odot (\beta' \beta)} \maps F((h_1' h_1)+_{g' g} (h_2' h_2))(x \odot y) \to x'' \odot y''.$};
%\path[->,font=\scriptsize,>=angle 90]
%(A) edge node[above]{$j \psi_{m} i_1$} (B)
%(C) edge node[above]{$j \psi_{n} o_2$} (B)
%(A) edge node[left]{$f' f$} (A')
%(C) edge node[right]{$k' k$} (C')
%(A') edge node [above]{$j \psi_{m''} i_1''$} (D)
%(C') edge node [above]{$j \psi_{n''} o_2''$} (D)
%(B) edge node [left] {$(h_1' h_1) +_{g' g} (h_2' h_2)$} (D);
%\end{tikzpicture}
%\]
%The decorations of 
%\[
%\begin{tikzpicture}[scale=1.5]
%\node (A) at (0,0) {$1$};
%\node (B) at (2,0) {$F(c) \times F(e)$};
%\node (C) at (4,0) {$F(c+e)$};
%\node (D) at (4,-1) {$F(c+_b e)$};
%\node (E) at (4,-2) {$F(c'' +_{b''} e'')$};
%\node (F) at (0,-1) {$F(c) \times F(e)$};
%\node (G) at (0,-2) {$F(c'') \times F(e'')$};
%\node (H) at (2,-2) {$F(c'' + e'')$};
%\path[->,font=\scriptsize,>=angle 90]
%(A) edge node [above]{$d_1 \times d_2$} (B)
%(B) edge node [above] {$\phi_{c,e}$} (C)
%(C) edge node [right]{$F(j_{c,e})$} (D)
%(D) edge node [right] {$F((h_1' +_{g'} h_2')(h_1 +_g h_2))$} (E)
%(A) edge node [left] {$d_1 \times d_2$} (F)
%(F) edge node [left] {$F((h_1' h_1) \times (h_2' h_2))$} (G)
%(G) edge node [above] {$\phi_{c'',e''}$} (H)
%(H) edge node [above] {$F(j_{c'',e''})$} (E);
%\end{tikzpicture}
%\]
%As is usual concerning the interchange law of double categories of this nature, only the `interior' of the two composites appears different, but the two morphisms $(h_1' +_{g'} h_2')(h_1 +_g h_2) \maps m+_{b} n \to m'' +_{b''} n''$ and $(h_1' h_1) +_{g' g} (h_2' h_2) \maps m+_{b} n \to m'' +_{b''}n''$ are the same universal map realized in two different ways. The two morphisms of decorations $\tau_{(\alpha' \odot \beta')(\alpha \odot \beta)}$ and $\tau_{(\alpha' \alpha) \odot (\beta' \beta)}$ are obtained as two different compositions of four 2-morphisms in $\Cat$, namely horizontally then vertically and vertically then horizontally. As $\Cat$ is a 2-category, the interchange law for these 2-morphisms already holds, and as a result, the morphisms $$\tau_{(\alpha' \odot \beta')(\alpha \odot \beta)} \maps F((h_1' +_{g'} h_2')(h_1 +_g h_2))(x \odot y) \to x'' \odot y''$$ and $$\tau_{(\alpha' \alpha)\odot(\beta' \beta)} \maps F((h_1' h_1)+_{g' g} (h_2' h_2))(x \odot y) \to x'' \odot y''$$ are also the same. Thus the interchange law for 2-morphisms holds and $F\mathbb{C}$sp is a double category.
\end{proof}

\begin{thm}\label{DC}
Let $\A$ be a category with chosen finite colimits and $F \maps \A \to \Cat$ a symmetric lax monoidal pseudofunctor. Then the double category $F\lCsp$ of Theorem \ref{thm:decorated_cospans} is symmetric monoidal.
\end{thm}
%!!!Kenny: We should try to make this proof shorter.
%Christina: Yes! Again, I think that it can be simplified and easier for the reader to follow. Should tackle after we do the previous theorem. Also notice that a big part of the following proof is commented out below. 
\begin{proof}
The definition of symmetric monoidal double category may be found in Definition \ref{defn:smdc}. We must show that both the category of objects $F\lCsp_0$ and the category of arrows $F\lCsp_1$ are symmetric monoidal categories, show that the monoidal unit of $F\lCsp_0$ is preserved by the unit functor $U$, show that the source and target functors $S,T \colon F\lCsp_1 \to F\lCsp_0$ are strict symmetric monoidal functors which preserve the associativity and unit constraints, define the globular isomorphisms $\chi$ and $\mu$ that relate how composition and units interact with the tensor product, and check that the required axioms are satisfied.

First we note that the category of objects $F\lCsp_0=\A$ is symmetric monoidal under chosen binary coproducts in $\A$ and the left and right unitors, associators and braidings are given as natural maps. Next we show that $F\lCsp_1$ is symmetric monoidal. The category of arrows $F\lCsp_1$ has:
\begin{enumerate}
\item{objects as $F$-decorated cospans, which are pairs:
\[
\begin{tikzpicture}[scale=1.5]
\node (A) at (0,0) {$(a$};
\node (B) at (1,0) {$m$};
\node (C) at (2,0) {$b,$};
\node (D) at (2.6,0) {$x \in F(m))$};
\path[->,font=\scriptsize,>=angle 90]
(A) edge node[above]{$i$} (B)
(C) edge node[above]{$o$} (B);
\end{tikzpicture}
\]
and}
\item{morphisms as maps of $F$-decorated cospans, which are cospans in $\A$
\[
\begin{tikzpicture}[scale=1.5]
\node (A) at (0,0.5) {$a$};
\node (A') at (0,-0.5) {$a'$};
\node (B) at (1,0.5) {$m$};
\node (C) at (2,0.5) {$b$};
\node (C') at (2,-0.5) {$b'$};
\node (D) at (1,-0.5) {$m'$};
\node (E) at (3,0.5) {$x \in F(m)$};
\node (F) at (3,-0.5) {$x' \in F(m')$};
\path[->,font=\scriptsize,>=angle 90]
(A) edge node[above]{$i$} (B)
(C) edge node[above]{$o$} (B)
(A) edge node[left]{$f$} (A')
(C) edge node[right]{$g$} (C')
(A') edge node [above]{$i'$} (D)
(C') edge node [above]{$o'$} (D)
(B) edge node [left] {$h$} (D);
\end{tikzpicture}
\]
together with a decoration morphism $\tau \maps F(h)(x) \to x'$.
}
\end{enumerate}

First we define the tensor product and monoidal unit of $F\lCsp_1$. Given two objects $M_1$ and $M_2$ of $F\lCsp_1$:
\[
\begin{tikzpicture}[scale=1.5]
\node (A) at (0,0) {$a_1$};
\node (B) at (1,0) {$m_1$};
\node (C) at (2,0) {$b_1$};
\node (D) at (1,-0.5) {$x_1 \in F(m_1)$};
\node (E) at (3,0) {$a_2$};
\node (F) at (4,0) {$m_2$};
\node (G) at (5,0) {$b_2$};
\node (H) at (4,-0.5) {$x_2 \in F(m_2)$};
\path[->,font=\scriptsize,>=angle 90]
(A) edge node[above]{$i_1$} (B)
(C) edge node[above]{$o_1$} (B)
(E) edge node[above]{$i_2$} (F)
(G) edge node[above]{$o_2$} (F);
\end{tikzpicture}
\]
their tensor product $M_1 \otimes M_2$ is given by taking the coproducts of the cospans in $\A$
\[
\begin{tikzpicture}[scale=1.5]
\node (A) at (0,0) {$a_1+a_2$};
\node (B) at (1.5,0) {$m_1+m_2$};
\node (C) at (3,0) {$b_1+b_2$};
\node (D) at (4.55,0) {$x_1+x_2 \in F(m_1+m_2)$};
\path[->,font=\scriptsize,>=angle 90]
(A) edge node[above]{$i_1+i_2$} (B)
(C) edge node[above]{$o_1+o_2$} (B);
\end{tikzpicture}
\]
and where the decoration on the apex is obtained using the natural transformation of the symmetric lax monoidal pseudofunctor $F$: $$x_1+x_2 \mapseqq 1 \xrightarrow{\lambda^{-1}} 1 \times 1 \xrightarrow{x_1 \times x_2} F(m_1) \times F(m_2) \xrightarrow{\phi_{m_1,m_2}} F(m_1+m_2).$$The monoidal unit $0_{F\lCsp_1}$ is given by:
\[
\begin{tikzpicture}[scale=1.5]
\node (A) at (0,0) {$0$};
\node (B) at (1,0) {$0$};
\node (C) at (2,0) {$0$};
\node (D) at (3,0) {$!_0 \in F(0)$};
\path[->,font=\scriptsize,>=angle 90]
(A) edge node[above]{$!$} (B)
(C) edge node[above]{$!$} (B);
\end{tikzpicture}
\]
where $0$ is the monoidal unit of $\A$ and $!_0 \maps 1 \to F(0)$ is the morphism which is part of the structure of the symmetric lax monoidal pseudofunctor $F \maps \A \to \Cat$. It is also clear that $U(0)=0_{F\lCsp_1}$.  

Next we define the associator and left unitor of $F\lCsp_1$, with the right unitor being similar to the left. Given three objects $M_1, M_2$ and $M_3$ of $F\lCsp_1$:
\[
\begin{tikzpicture}[scale=1.5]
\node (A) at (0,0) {$a_1$};
\node (B) at (1,0) {$m_1$};
\node (C) at (2,0) {$b_1$};
\node (D) at (1,-0.5) {$x_1 \in F(m_1)$};
\node (E) at (3,0) {$a_2$};
\node (F) at (4,0) {$m_2$};
\node (G) at (5,0) {$b_2$};
\node (H) at (4,-0.5) {$x_2 \in F(m_2)$};
\node (I) at (6,0) {$a_3$};
\node (J) at (7,0) {$m_3$};
\node (K) at (8,0) {$b_3$};
\node (L) at (7,-0.5) {$x_3 \in F(m_3)$};
\path[->,font=\scriptsize,>=angle 90]
(A) edge node[above]{$i_1$} (B)
(C) edge node[above]{$o_1$} (B)
(E) edge node[above]{$i_2$} (F)
(G) edge node[above]{$o_2$} (F)
(I) edge node[above]{$i_3$} (J)
(K) edge node[above]{$o_3$} (J);
\end{tikzpicture}
\]
tensoring the first two and then the third results in $(M_1 \otimes M_2) \otimes M_3$:
\[
\begin{tikzpicture}[scale=1.5]
\node (A) at (0,0) {$(a_1+a_2)+a_3$};
\node (B) at (2.5,0){$(m_1+m_2)+m_3$};
\node (C) at (5,0) {$(b_1+b_2)+b_3$};
\node (D) at (2.5,-0.5) {$(x_1+x_2)+x_3 \in F((m_1+m_2)+m_3)$};
\path[->,font=\scriptsize,>=angle 90]
(A) edge node[above]{$(i_1+i_2)+i_3$} (B)
(C) edge node[above]{$(o_1+o_2)+o_3$} (B);
\end{tikzpicture}
\]
where $(x_1+x_2)+x_3 \maps 1 \to F((m_1+m_2)+m_3)$ is given by: $$1 \xrightarrow{(x_1 \times x_2) \times x_3} (F(m_1) \times F(m_2)) \times F(m_3) \xrightarrow{\phi_{m_1,m_2} \times 1} F(m_1+m_2) \times F(m_3) \xrightarrow{\phi_{m_1+m_2,m_3}} F((m_1+m_2)+m_3).$$ The other parenthesization $M_1 \otimes (M_2 \otimes M_3)$ is given by:
\[
\begin{tikzpicture}[scale=1.5]
\node (A) at (0,0) {$a_1+(a_2+a_3)$};
\node (B) at (2.5,0) {$m_1+(m_2+m_3)$};
\node (C) at (5,0) {$b_1+(b_2+b_3)$};
\node (D) at (2.5,-0.5) {$x_1+(x_2+x_3) \in F(m_1+(m_2+m_3))$};
\path[->,font=\scriptsize,>=angle 90]
(A) edge node[above]{$i_1+(i_2+i_3)$} (B)
(C) edge node[above]{$o_1+(o_2+o_3)$} (B);
\end{tikzpicture}
\]
where $x_1+(x_2+x_3) \maps 1 \to F(m_1+(m_2+m_3))$ is given by: $$1 \xrightarrow{x_1 \times (x_2 \times x_3)} F(m_1) \times (F(m_2) \times F(m_3)) \xrightarrow{1 \times \phi_{m_2,m_3}} F(m_1) \times F(m_2+m_3) \xrightarrow{\phi_{m_1,m_2+m_3}} F(m_1+(m_2+m_3)).$$
\noindent
\textbf{In progress}

\noindent
If we let $a$ denote the associator of $(\A,+,0)$, the associator of $F\lCsp_1$ is then a map of cospans in $\A$ from $(M_1 \otimes M_2) \otimes M_3$ to $M_1 \otimes (M_2 \otimes M_3)$ given by:
\[
\begin{tikzpicture}[scale=1.5]
\node (A) at (0,0.5) {$(a_1+a_2)+a_3$};
\node (A') at (0,-0.5) {$a_1+(a_2+a_3)$};
\node (B) at (2.25,0.5) {$(m_1+m_2)+m_3$};
\node (C) at (4.5,0.5) {$(b_1+b_2)+b_3$};
\node (C') at (4.5,-0.5) {$b_1+(b_2+b_3)$};
\node (D) at (2.25,-0.5) {$m_1+(m_2+m_3)$};
\node (E) at (7,0.5) {$(x_1+x_2)+x_3 \in F((m_1+m_2)+m_3)$};
\node (F) at (7,-0.5) {$x_1+(x_2+x_3) \in F(m_1+(m_2+m_3))$};
\path[->,font=\scriptsize,>=angle 90]
(A) edge node[above]{$(i_1+i_2)+i_3$} (B)
(C) edge node[above]{$(o_1+o_2)+o_3$} (B)
(A) edge node[left]{$a$} (A')
(C) edge node[right]{$a$} (C')
(A') edge node [above]{$i_1+(i_2+i_3)$} (D)
(C') edge node [above]{$o_1+(o_2+o_3)$} (D)
(B) edge node [left] {$a$} (D);
\end{tikzpicture}
\]
together with the isomorphism $\tau_a \maps F(a)((x_1+x_2)+x_3) \to x_1+(x_2+x_3)$.




Tensoring an object with the monoidal unit, say, on the left:
\[
\begin{tikzpicture}[scale=1.5]
\node (A) at (0,0) {$0$};
\node (B) at (1,0) {$0$};
\node (C) at (2,0) {$0$};
\node (D) at (1,-0.5) {$!_0 \in F(0)$};
\node (E) at (3,0) {$a$};
\node (F) at (4,0) {$m$};
\node (G) at (5,0) {$b$};
\node (H) at (4,-0.5) {$x \in F(m)$};
\node (I) at (2.5,0) {$\otimes$};
\path[->,font=\scriptsize,>=angle 90]
(A) edge node[above]{$!$} (B)
(C) edge node[above]{$!$} (B)
(E) edge node[above]{$i$} (F)
(G) edge node[above]{$o$} (F);
\end{tikzpicture}
\]
results in:
\[
\begin{tikzpicture}[scale=1.5]
\node (A) at (0,0) {$0+a$};
\node (B) at (1,0) {$0+m$};
\node (C) at (2,0) {$0+b$};
\node (D) at (3.5,0) {$!_0 + x \in F(0+m)$};
\path[->,font=\scriptsize,>=angle 90]
(A) edge node[above]{$!+i$} (B)
(C) edge node[above]{$!+o$} (B);
\end{tikzpicture}
\]
where $!_0 + x \in F(0+m)$ is given by $$1 \xrightarrow{\lambda^{-1}} 1 \times 1 \xrightarrow{!_0 \times x} F(0) \times F(m) \xrightarrow{\phi_{0,m}} F(0+m).$$The left unitor is then an isomorphism in $F\lCsp_1$ given by:
\[
\begin{tikzpicture}[scale=1.5]
\node (A) at (0,0.5) {$0+a$};
\node (A') at (0,-0.5) {$a$};
\node (B) at (1,0.5) {$0+m$};
\node (C) at (2,0.5) {$0+b$};
\node (C') at (2,-0.5) {$b$};
\node (D) at (1,-0.5) {$m$};
\node (E) at (3.5,0.5) {$!_0 + x \in F(0+m)$};
\node (F) at (3.5,-0.5) {$x \in F(m)$};
\path[->,font=\scriptsize,>=angle 90]
(A) edge node[above]{$!+i$} (B)
(C) edge node[above]{$!+o$} (B)
(A) edge node[left]{$\ell$} (A')
(C) edge node[right]{$\ell$} (C')
(A') edge node [above]{$i$} (D)
(C') edge node [above]{$o$} (D)
(B) edge node [left] {$\ell$} (D);
\end{tikzpicture}
\]
where $\ell$ is the left unitor of $(\A,+,0)$, together with the isomorphism $\tau_{\lambda} \maps F(\ell)(!_0 + x) \to x$. The right unitor is similar.

 These associators and left and right unitors together satisfy the pentagon and triangle identities of a monoidal category. 
%!!! Kenny: Probably cut stuff out below this.
% Christina: Commented them out for now, but should definitely have a close look to see if anything is needed when we go over and remake this proof.
\begin{comment}
If we denote the above associator simply as $a$ and the left and right unitors as $\lambda$ and $\rho$, respectively, then given four objects in $F\lCsp_1$, say $M_1, M_2, M_3$ and $M_4$:
\[
\begin{tikzpicture}[scale=1.5]
\node (A) at (0,0) {$a_1$};
\node (B) at (1,0) {$m_1$};
\node (C) at (2,0) {$b_1$};
\node (D) at (1,-0.5) {$x_1 \in F(m_1)$};
\node (E) at (3,0) {$a_2$};
\node (F) at (4,0) {$m_2$};
\node (G) at (5,0) {$b_2$};
\node (H) at (4,-0.5) {$x_2 \in F(m_2)$};
\node (I) at (0,-1.5) {$a_3$};
\node (J) at (1,-1.5) {$m_3$};
\node (K) at (2,-1.5) {$b_3$};
\node (L) at (1,-2) {$x_3 \in F(m_3)$};
\node (M) at (3,-1.5) {$a_4$};
\node (N) at (4,-1.5) {$m_4$};
\node (O) at (5,-1.5) {$b_4$};
\node (P) at (4,-2) {$x_4 \in F(m_4)$};
\path[->,font=\scriptsize,>=angle 90]
(A) edge node[above]{$i_1$} (B)
(C) edge node[above]{$o_1$} (B)
(E) edge node[above]{$i_2$} (F)
(G) edge node[above]{$o_2$} (F)
(I) edge node[above]{$i_3$} (J)
(K) edge node[above]{$o_3$} (J)
(M) edge node[above]{$i_4$} (N)
(O) edge node[above]{$o_4$} (N);
\end{tikzpicture}
\]
then as $\lCsp(\A)$ is a symmetric monoidal double category, the following pentagon of underlying cospans and maps of cospans commutes:
\[
\begin{tikzpicture}[scale=1.5]
\node (A) at (0,0) {$((M_1 \otimes M_2) \otimes M_3) \otimes M_4$};
\node (B) at (2.25,1) {$(M_1 \otimes M_2) \otimes (M_3 \otimes M_4)$};
\node (C) at (4.5,0) {$M_1 \otimes (M_2 \otimes (M_3 \otimes M_4))$};
\node (D) at (.75,-1.5) {$(M_1 \otimes (M_2 \otimes M_3)) \otimes M_4$};
\node (E) at (3.75,-1.5) {$M_1 \otimes ((M_2 \otimes M_3) \otimes M_4)$};
\path[->,font=\scriptsize,>=angle 90]
(A) edge node[above]{$a$} (B)
(B) edge node[above]{$a$} (C)
(A) edge node[left]{$a \otimes 1$} (D)
(D) edge node[above]{$a$} (E)
(E) edge node[right]{$1 \otimes a$} (C);
\end{tikzpicture}
\]
as well as the following pentagon of corresponding decorations in the category $F(m_1 +(m_2 +(m_3+m_4)))$:
\[
\begin{tikzpicture}[scale=1.5]
\node (A) at (-.5,0) {$F(aa)(((x_1+x_2)+x_3)+x_4) $};
\node (B) at (2.25,1) {$F(a)((x_1+x_2)+(x_3+x_4))$};
\node (C) at (5,0) {$x_1+(x_2+(x_3+x_4))$};
\node (D) at (-.5,-1) {$F((1 \otimes a)a)((x_1+(x_2+x_3))+x_4) $};
\node (E) at (5,-1) {$F(1 \otimes a)(x_1+((x_2+x_3)+x_4)) $};
\path[->,font=\scriptsize,>=angle 90]
(A) edge node[above]{$F(a)(\tau_a)$} (B)
(B) edge node[above]{$\tau_a$} (C)
(A) edge node[left]{$F((1 \otimes a)a)(\tau_{a \otimes 1})$} (D)
(D) edge node[above]{$F(1 \otimes a)(\tau_a)$} (E)
(E) edge node[right]{$\tau_{1 \otimes a}$} (C);
\end{tikzpicture}
\]
since $$F(aa)(((x_1+x_2)+x_3)+x_4)=F((1 \otimes a)a(a \otimes 1))(((x_1+x_2)+x_3)+x_4)$$ as the corresponding pentagon in the symmetric monoidal category $(\A,+,0)$ commutes, and then appliying the pseudofunctor $F$ to said pentagon yields a commutative pentagon in $\Cat$.

Similarly, if we denote the left and right unitors as $\lambda$ and $\rho$, respectively, then the following triangle of cospans and underlying maps of cospans commutes:
\[
\begin{tikzpicture}[scale=1.5]
\node (A) at (0,0) {$(M_1 \otimes 0) \otimes M_2$};
\node (B) at (2.25,1) {$M_1 \otimes M_2$};
\node (C) at (4.5,0) {$M_1 \otimes (0 \otimes M_2)$};
\path[->,font=\scriptsize,>=angle 90]
(A) edge node[above]{$\rho \otimes 1$} (B)
(C) edge node[above]{$1 \otimes \lambda$} (B)
(A) edge node[above]{$a$} (C);
\end{tikzpicture}
\]
as well as the following triangle of corresponding decorations in the category $F(m_1+m_2)$:
\[
\begin{tikzpicture}[scale=1.5]
\node (A) at (0,0) {$F(\rho \otimes 1)((x_1+0)+x_2)$};
\node (B) at (2.25,1) {$x_1+x_2$};
\node (C) at (4.5,0) {$F(1 \otimes \lambda)(x_1+(0+x_2)) $};
\path[->,font=\scriptsize,>=angle 90]
(A) edge node[left]{$\tau_{\rho \otimes 1}$} (B)
(C) edge node[right]{$\tau_{1 \otimes \lambda}$} (B)
(A) edge node[above]{$F(1 \otimes \lambda)(\tau_a)$} (C);
\end{tikzpicture}
\]
since $$F(\rho \otimes 1)((x_1+0)+x_2)=F((1 \otimes \lambda)a)((x_1+0)+x_2)$$ as the corresponding triangle in the symmetric monoidal category $(\A,+,0)$ commutes and applying the pseudofunctor $F$ to this commutative triangle results in a commutative triangle in $\Cat$.

For a tensor product of objects $M_1 \otimes M_2$ in $F\lCsp_1$, the source and target structure functors $S,T \maps F\lCsp_1 \to F\lCsp_0$ satisfy the following equations: $$S(M_1 \otimes M_2)=S(M_1) \otimes S(M_2)$$ $$T(M_1 \otimes M_2)=T(M_1) \otimes T(M_2).$$
For two objects $M_1$ and $M_2$ in $F\lCsp_1$, we have a braiding $\beta_{M_1,M_2} \maps M_1 \otimes M_2 \to M_2 \otimes M_1$ given by:
\[
\begin{tikzpicture}[scale=1.5]
\node (A) at (0,0.5) {$a_1+a_2$};
\node (A') at (0,-0.5) {$a_2+a_1$};
\node (B) at (2,0.5) {$m_1+m_2$};
\node (C) at (4,0.5) {$b_1+b_2$};
\node (C') at (4,-0.5) {$b_2+b_1$};
\node (D) at (2,-0.5) {$m_2+m_1$};
\node (E) at (5.5,0.5) {$x_1+x_2 \in F(m_1+m_2)$};
\node (F) at (5.5,-0.5) {$x_2+x_1 \in F(m_2+m_1)$};
\path[->,font=\scriptsize,>=angle 90]
(A) edge node[above]{$i_1+i_2$} (B)
(C) edge node[above]{$o_1+o_2$} (B)
(A) edge node[left]{$\beta_{a_1,a_2}$} (A')
(C) edge node[left]{$\beta_{b_1,b_2}$} (C')
(A') edge node [above]{$i_2+i_1$} (D)
(C') edge node [above]{$o_2+o_1$} (D)
(B) edge node [left] {$\beta_{m_1,m_2}$} (D);
\end{tikzpicture}
\]
$$\tau_{\beta_{M_1,M_2}} \maps F(\beta_{m_1,m_2})(x_1+x_2) \xrightarrow{\sim} x_2+x_1$$ where the vertical 1-morphisms are given by braidings in $(\A,+,0)$. This braiding makes the following triangle of underlying cospans commute:
\[
\begin{tikzpicture}[scale=1.5]
\node (A) at (0,0) {$M_1 \otimes M_2$};
\node (B) at (2.25,1) {$M_1 \otimes M_2$};
\node (C) at (4.5,0) {$M_2 \otimes M_1$};
\path[->,font=\scriptsize,>=angle 90]
(A) edge node[above]{$1$} (B)
(C) edge node[above]{$\beta_{M_2,M_1}$} (B)
(A) edge node[above]{$\beta_{M_1,M_2}$} (C);
\end{tikzpicture}
\]
as well as the following diagram of corresponding decorations in the category $F(m_1+m_2)$:
\[
\begin{tikzpicture}[scale=1.5]
\node (A) at (0,0) {$x_1+x_2$};
\node (B) at (2.25,1) {$x_1+x_2$};
\node (C) at (4.5,0) {$F(\beta_{m_2,m_1})(x_2+x_1)$};
\path[->,font=\scriptsize,>=angle 90]
(A) edge node[above]{$1$} (B)
(C) edge node[right]{$\tau_{\beta_{M_2,M_1}}$} (B)
(A) edge node[above]{$F(\beta_{m_2,m_1})(\tau_{\beta_{M_1,M_2}})$} (C);
\end{tikzpicture}
\]
since $F(\beta_{m_2,m_1} \beta_{m_1,m_2}) (x_1+x_2) = x_1+x_2$. Thus $F\lCsp_1$ is also symmetric monoidal.

Next we derive the globular isomorphisms required in the definition of a symmetric monoidal double category relating horizontal composition and the tensor product. Given four horizontal 1-cells $M_1, M_2, N_1$ and $N_2$ respectively by:
\[
\begin{tikzpicture}[scale=1.5]
\node (A) at (0,0) {$a$};
\node (B) at (1,0) {$m_1$};
\node (C) at (2,0) {$b$};
\node (D) at (1,-0.5) {$x_1 \in F(m_1)$};
\node (E) at (3,0) {$b$};
\node (F) at (4,0) {$m_2$};
\node (G) at (5,0) {$c$};
\node (H) at (4,-0.5) {$x_2 \in F(m_2)$};
\node (I) at (0,-1.5) {$a'$};
\node (J) at (1,-1.5) {$n_1$};
\node (K) at (2,-1.5) {$b'$};
\node (L) at (1,-2) {$y_1 \in F(n_1)$};
\node (M) at (3,-1.5) {$b'$};
\node (N) at (4,-1.5) {$n_2$};
\node (O) at (5,-1.5) {$c'$};
\node (P) at (4,-2) {$y_2 \in F(n_2)$};
\path[->,font=\scriptsize,>=angle 90]
(A) edge node[above]{$i_1$} (B)
(C) edge node[above]{$o_1$} (B)
(E) edge node[above]{$i_2$} (F)
(G) edge node[above]{$o_2$} (F)
(I) edge node[above]{$i_1'$} (J)
(K) edge node[above]{$o_1'$} (J)
(M) edge node[above]{$i_2'$} (N)
(O) edge node[above]{$o_2'$} (N);
\end{tikzpicture}
\]
we have that $(M_1 \otimes N_1) \odot (M_2 \otimes N_2)$ is given by:
\[
\begin{tikzpicture}[scale=1.5]
\node (A) at (0,0) {$a+a'$};
\node (B) at (2.5,0) {$(m_1+n_1)+_{b+b'}(m_2+n_2)$};
\node (C) at (5,0) {$c+c'$};
\node (D) at (2.5,-0.5) {$(x_1+y_1)\odot(x_2+y_2) \in F((m_1+n_1)+_{b+b'}(m_2+n_2))$};
\path[->,font=\scriptsize,>=angle 90]
(A) edge node[above]{$j \psi(i_1+i_1')$} (B)
(C) edge node[above]{$j \psi(o_2 + o_2')$} (B);
\end{tikzpicture}
\]
where the decoration $(x_1+y_1) \odot (x_2+y_2) \in F((m_1+n_1)+_{b+b'}(m_2+n_2))$ is given by:
\[
\begin{tikzpicture}[scale=1.5]
\node (A) at (0,0) {$1$};
\node (B) at (0,-1) {$1 \times 1$};
\node (C) at (0,-2) {$(1 \times 1) \times (1 \times 1)$};
\node (D) at (0,-3) {$(F(m_1) \times F(n_1)) \times (F(m_2) \times F(n_2))$};
\node (E) at (0,-4) {$F(m_1+n_1) \times F(m_2+n_2)$};
\node (F) at (0,-5) {$F((m_1+n_1)+(m_2+n_2))$};
\node (G) at (0,-6) {$F((m_1+n_1)+_{b+b'}(m_2+n_2))$};
\path[->,font=\scriptsize,>=angle 90]
(A) edge node[left]{$\lambda^{-1}$} (B)
(B) edge node[left]{$\lambda^{-1} \times \lambda^{-1}$} (C)
(C) edge node[left]{$(x_1 \times y_1) \times (x_2 \times y_2)$} (D)
(D) edge node[left]{$\phi_{m_1,n_1} \times \phi_{m_2,n_2}$} (E)
(E) edge node[left]{$\phi_{m_1+n_1,m_2+n_2}$} (F)
(F) edge node[left]{$F(j_{m_1+n_1,m_2+n_2})$} (G);
\end{tikzpicture}
\]
and $(M_1 \odot M_2) \otimes (N_1 \odot N_2)$ is given by:
\[
\begin{tikzpicture}[scale=1.5]
\node (A) at (0,0) {$a+a'$};
\node (B) at (2.5,0) {$(m_1+_{b} m_2) + (n_1 +_{b'} n_2)$};
\node (C) at (5,0) {$c+c'$};
\node (D) at (2.5,-0.5) {$(x_1 \odot x_2) + (y_1 \odot y_2) \in F((m_1+_{b}m_2)+(n_1+_{b'}n_2))$};
\path[->,font=\scriptsize,>=angle 90]
(A) edge node[above]{$(j \psi i_1)+(j \psi i_1')$} (B)
(C) edge node[above]{$(j \psi o_2)+(j \psi o_2')$} (B);
\end{tikzpicture}
\]
where the decoration $(x_1 \odot x_2) + (y_1 \odot y_2) \in F((m_1+_{b}m_2)+(n_1+_{b'}n_2))$ is given by:
\[
\begin{tikzpicture}[scale=1.5]
\node (A) at (0,0) {$1$};
\node (B) at (0,-1) {$1 \times 1$};
\node (C) at (0,-2) {$(1 \times 1) \times (1 \times 1)$};
\node (D) at (0,-3) {$(F(m_1) \times F(m_2)) \times (F(n_1) \times F(n_2))$};
\node (E) at (0,-4) {$F(m_1+m_2) \times F(n_1+n_2)$};
\node (F) at (0,-5) {$F(m_1+_{b}m_2) \times F(n_1+_{b'}n_2)$};
\node (G) at (0,-6) {$F((m_1+_{b}m_2)+(n_1+_{b'}n_2))$};
\path[->,font=\scriptsize,>=angle 90]
(A) edge node[left]{$\lambda^{-1}$} (B)
(B) edge node[left]{$\lambda^{-1} \times \lambda^{-1}$} (C)
(C) edge node[left]{$(x_1 \times x_2) \times (y_1 \times y_2)$} (D)
(D) edge node[left]{$\phi_{m_1,m_2} \times \phi_{n_1,n_2}$} (E)
(E) edge node[left]{$F(j_{m_1,m_2}) \times F(j_{n_1,n_2})$} (F)
(F) edge node[left]{$\phi_{m_1+_{b}m_2,n_1+_{b'}n_2}$} (G);
\end{tikzpicture}
\]
and where $\psi$ and $j$ are the natural maps into a coproduct and from a coproduct into a pushout, respectively. We then get a globular 2-morphism $$\chi \maps (M_1 \otimes N_1) \odot (M_2 \otimes N_2) \to (M_1 \odot M_2) \otimes (N_1 \odot N_2)$$ given by:
\[
\begin{tikzpicture}[scale=1.5]
\node (A) at (0,0.5) {$a+a'$};
\node (A') at (0,-0.5) {$a+a'$};
\node (B) at (2.5,0.5) {$(m_1+n_1)+_{b+b'}(m_2+n_2)$};
\node (C) at (5,0.5) {$c+c'$};
\node (C') at (5,-0.5) {$c+c'$};
\node (D) at (2.5,-0.5) {$(m_1+_{b}m_2)+(n_1+_{b'}n_2)$};
\node (E) at (2.5,1) {$(x_1+y_1) \odot (x_2+y_2) \in F((m_1+n_1)+_{b+b'}(m_2+n_2))$};
\node (F) at (2.5,-1) {$(x_1 \odot x_2)+(y_1 \odot y_2) \in F((m_1+_{b}m_2)+(n_1+_{b'}n_2))$};
\path[->,font=\scriptsize,>=angle 90]
(A) edge node[above]{$j \psi (i_1+i_1')$} (B)
(C) edge node[above]{$j \psi (o_2 + o_2')$} (B)
(A) edge node[left]{$1$} (A')
(C) edge node[right]{$1$} (C')
(A') edge node [above]{$(j \psi i_1')+(j \psi i_1)$} (D)
(C') edge node [above]{$(j \psi o_2)+(j \psi o_2')$} (D)
(B) edge node [left] {$\hat{\chi}$} (D);
\end{tikzpicture}
\]
$$\tau_{\hat{\chi}} \maps F(\hat{\chi})((x_1+y_1)\odot(x_2+y_2)) \to (x_1 \odot x_2)+(y_1 \odot y_2)$$
where $\hat{\chi}$ is the universal map between two colimits of the same diagram.
For two objects $a,b \in \mathrm{A}$, $U_{a+b}$ is given by:
\[
\begin{tikzpicture}[scale=1.5]
\node (A) at (0,0) {$a+b$};
\node (B) at (1,0) {$a+b$};
\node (C) at (2,0) {$a+b$};
\node (D) at (1,-0.5) {$!_{a+b} \in F(a+b)$};
\path[->,font=\scriptsize,>=angle 90]
(A) edge node[above]{$1_{a+b}$} (B)
(C) edge node[above]{$1_{a+b}$} (B);
\end{tikzpicture}
\]
where $$!_{a+b} \maps 1 \xrightarrow{\phi} F(0) \xrightarrow{F(!_{a+b})}  F(a+b).$$
Similarly, we have $U_a$ and $U_b$ given respectively by:
\[
\begin{tikzpicture}[scale=1.5]
\node (A) at (0,0) {$a$};
\node (B) at (1,0) {$a$};
\node (C) at (2,0) {$a$};
\node (D) at (1,-0.5) {$!_a \in F(a)$};
\node (E) at (3,0) {$b$};
\node (F) at (4,0) {$b$};
\node (G) at (5,0) {$b$};
\node (H) at (4,-0.5) {$!_b \in F(b)$};
\path[->,font=\scriptsize,>=angle 90]
(A) edge node[above]{$1_a$} (B)
(C) edge node[above]{$1_a$} (B)
(E) edge node[above]{$1_b$} (F)
(G) edge node[above]{$1_b$} (F);
\end{tikzpicture}
\]
and then $U_a + U_b$ is given by:
\[
\begin{tikzpicture}[scale=1.5]
\node (A) at (0,0) {$a+b$};
\node (B) at (1.25,0) {$a+b$};
\node (C) at (2.5,0) {$a+b$};
\node (D) at (1.25,-0.5) {$!_a + !_b \in F(a+b)$};
\path[->,font=\scriptsize,>=angle 90]
(A) edge node[above]{$1_a + 1_b$} (B)
(C) edge node[above]{$1_a + 1_b$} (B);
\end{tikzpicture}
\]
where
$$!_a + !_b \maps 1 \xrightarrow{\lambda^{-1}} 1 \times 1 \xrightarrow{\phi \times \phi} F(0) \times F(0) \xrightarrow{F(!_a) \times F(!_b)} F(a) \times F(b) \xrightarrow{\phi_{a,b}} F(a+b).$$
We then have another globular isomorphism $$\mu_{a,b} \maps U_{a+b} \to U_a + U_b$$ given by the identity 2-morphism:
\[
\begin{tikzpicture}[scale=1.5]
\node (A) at (0,0.5) {$a+b$};
\node (A') at (0,-0.5) {$a+b$};
\node (B) at (2,0.5) {$a+b$};
\node (C) at (4,0.5) {$a+b$};
\node (C') at (4,-0.5) {$a+b$};
\node (D) at (2,-0.5) {$a+b$};
\node (E) at (5.5,0.5) {$!_{a+b} \in F(a+b)$};
\node (F) at (5.5,-0.5) {$!_a + !_b \in F(a+b)$};
\path[->,font=\scriptsize,>=angle 90]
(A) edge node[above]{$1_{a+b}$} (B)
(C) edge node[above]{$1_{a+b}$} (B)
(A) edge node[left]{$1$} (A')
(C) edge node[right]{$1$} (C')
(A') edge node [above]{$1_a + 1_b$} (D)
(C') edge node [above]{$1_a + 1_b$} (D)
(B) edge node [left] {$1$} (D);
\end{tikzpicture}
\]
$$\tau_{a,b} \maps !_{a+b} \xrightarrow{\sim} !_a + !_b$$
where $!_{a+b}$ and $!_a + !_b$ are both initial objects in $F(a+b)$, hence isomorphic.

There are a fair amount of coherence diagrams to verify, many of which are similar in flavor and make use of the two above globular ismorphisms. We check a few to give a sense of what these are like. For example,  given horizontal 1-cells $M_i,N_i,P_i$ for $i=1,2$, the following commutative diagram expresses the associativity isomorphism as a transformation of double categories.
\[
\begin{tikzpicture}[scale=1.5]
\node (A) at (0,0.5) {$((M_1 \otimes N_1) \otimes P_1) \odot ((M_2 \otimes N_2) \otimes P_2)$};
\node (A') at (4.5,0.5) {$(M_1 \otimes (N_1 \otimes P_1)) \odot (M_2 \otimes (N_2 \otimes P_2))$};
\node (B) at (0,-0.25) {$((M_1 \otimes N_1) \odot (M_2 \otimes N_2)) \otimes (P_1 \odot P_2)$};
\node (C) at (4.5,-0.25) {$(M_1 \odot M_2) \otimes ((N_1 \otimes P_1) \odot (N_2 \otimes P_2))$};
\node (C') at (0,-1) {$((M_1 \odot M_2) \otimes (N_1 \odot N_2)) \otimes (P_1 \odot P_2)$};
\node (D) at (4.5,-1) {$(M_1 \odot M_2) \otimes ((N_1 \odot N_2) \otimes (P_1 \odot P_2))$};
\path[->,font=\scriptsize,>=angle 90]
(A) edge node[above]{$a \odot a$} (A')
(A) edge node[left]{$\chi$} (B)
(A') edge node[right]{$\chi$} (C)
(B) edge node[left]{$\chi \otimes 1$} (C')
(C) edge node [right] {$1 \otimes \chi$} (D)
(C') edge node [above] {$a$} (D);
\end{tikzpicture}
\]
Here, $a$ is the associator of $F\lCsp_1$ and $\chi$ is the first globular isomorphism above. To see that this diagram does indeed commute, we first consider this diagram with respect to only the underlying cospans of each horizontal 1-cell. For notation:
	\[
		\begin{tikzpicture}
			\node (k) at (0,0) {$a$};
			\node (l) at (1,0) {$m_1$};
			\node (m) at (2,0) {$b$};
			\node (M1) at (-1,0) {$M_1 =$};
			\node (N1) at (3,0) {$N_1 =$};
			\node (q) at (4,0) {$a'$};
			\node (r) at (5,0) {$n_1$};
			\node (s) at (6,0) {$b'$};
			\node (P1) at (7,0) {$P_1 =$};
			\node (v) at (8,0) {$a''$};
			\node (w) at (9,0) {$p_1$};
			\node (x) at (10,0) {$b''$};
			\node (m2) at (0,-1.5) {$b$};
			\node (n) at (1,-1.5) {$m_2$};
			\node (p) at (2,-1.5) {$c$};
			\node (M2) at (-1,-1.5) {$M_2 =$};
			\node (N2) at (3,-1.5) {$N_2 =$};
			\node (s2) at (4,-1.5) {$b'$};
			\node (t) at (5,-1.5) {$n_2$};
			\node (u) at (6,-1.5) {$c'$};
			\node (P2) at (7,-1.5) {$P_2 =$};
			\node (x2) at (8,-1.5) {$b''$};
			\node (y) at (9,-1.5) {$p_2$};
			\node (z) at (10,-1.5) {$c''$};
\node (dM1) at (1,-0.5) {$x_1 \in F(m_1)$};
\node (dM2) at (1,-2) {$x_2 \in F(m_2)$};
\node (dN1) at (5,-0.5) {$y_1 \in F(n_1)$};
\node (dN2) at (5,-2) {$y_2 \in F(n_2)$};
\node (dP1) at (9,-0.5) {$z_1 \in F(p_1)$};
\node (dP2) at (9,-2) {$z_2 \in F(p_2)$};
			\path[->,font=\scriptsize,>=angle 90]
			(k) edge node[above]{$$} (l)
			(m) edge node[above]{$$} (l)
			(q) edge node[above]{$$} (r)
			(s) edge node[above]{$$} (r)
			(v) edge node[above]{$$} (w)
			(x) edge node[above]{$$} (w)
			(m2) edge node[above]{$$} (n)
			(p) edge node[above]{$$} (n)
			(s2) edge node[above]{$$} (t)
			(u) edge node[above]{$$} (t)
			(x2) edge node[above]{$$} (y)
			(z) edge node[above]{$$} (y);
		\end{tikzpicture}
	\]
The above diagram then becomes:
\[
		\begin{tikzpicture}
			\node (a) at (-4,0) {$(a+a')+a''$};
			\node (b) at (1,0) {$((m_1+n_1)+p_1) +_{((b+b')+b'')}((m_2+n_2)+p_2)$};
			\node (c) at (6,0) {$(c+c')+c''$};
			\node (a2) at (-4,1) {$a+(a'+a'')$};
			\node (b2) at (1,1) {$(m_1+(n_1+p_1)) +_{(b+(b'+b''))}(m_2+(n_2+p_2))$};
			\node (c2) at (6,1) {$c+(c'+c'')$};
                                \node (a3) at (-4,2) {$a+(a'+a'')$};
			\node (b3) at (1,2) {$(m_1+_b m_2)+((n_1+p_1)+_{(b'+b'')}(n_2+p_2))$};
			\node (c3) at (6,2) {$c+(c'+c'')$};
                                \node (a4) at (-4,3) {$a+(a'+a'')$};
			\node (b4) at (1,3) {$(m_1+_b m_2)+((n_1+_{b'} n_2)+(p_1+_{b''} p_2))$};
			\node (c4) at (6,3) {$c+(c'+c'')$};
                                \node (a5) at (-4,-1) {$(a+a')+a''$};
			\node (b5) at (1,-1) {$((m_1+n_1)+_{(b+b')}(m_2+n_2))+(p_1+_{b''} p_2)$};
			\node (c5) at (6,-1) {$(c+c')+c''$};
                                \node (a6) at (-4,-2) {$(a+a')+a''$};
			\node (b6) at (1,-2) {$((m_1+_b m_2)+(n_1+_{b'} n_2))+(p_1+_{b''} p_2)$};
			\node (c6) at (6,-2) {$(c+c')+c''$};
                                \node (a7) at (-4,-3) {$a+(a'+a'')$};
			\node (b7) at (1,-3) {$(m_1+_b m_2)+((n_1+_{b'} n_2)+(p_1+_{b''} p_2))$};
			\node (c7) at (6,-3) {$c+(c'+c'')$};
			\path[->,font=\scriptsize,>=angle 90]
			(a) edge node[above]{$$} (b)
			(c) edge node[above]{$$} (b)
                                (a2) edge node[above]{$$} (b2)
			(c2) edge node[above]{$$} (b2)
                                (a) edge node[left]{$$} (a2)
                                (b) edge node[left]{$a \odot a$} (b2)
(b) edge node[right]{$\tau_1$} (b2)
			(c) edge node[above]{$$} (c2)
                                (a3) edge node[above]{$$} (b3)
			(c3) edge node[above]{$$} (b3)
                                (a2) edge node[above]{$$} (a3)
                                (b2) edge node[left]{$\chi$} (b3)
(b2) edge node[right]{$\tau_2$} (b3)
			(c2) edge node[above]{$$} (c3)
                                (a4) edge node[above]{$$} (b4)
			(c4) edge node[above]{$$} (b4)
                                (a3) edge node[above]{$$} (a4)
                                (b3) edge node[left]{$1 \otimes \chi$} (b4)
(b3) edge node[right]{$\tau_3$} (b4)
			(c3) edge node[above]{$$} (c4)
                                (a5) edge node[above]{$$} (b5)
			(c5) edge node[above]{$$} (b5)
                                (a) edge node[above]{$$} (a5)
                                (b) edge node[left]{$\chi$} (b5)
(b) edge node[right]{$\tau_4$} (b5)
			(c) edge node[above]{$$} (c5)
                                (a6) edge node[above]{$$} (b6)
			(c6) edge node[above]{$$} (b6)
                                (a5) edge node[above]{$$} (a6)
                                (b5) edge node[left]{$\chi \otimes 1$} (b6)
 (b5) edge node[right]{$\tau_5$} (b6)
			(c5) edge node[above]{$$} (c6)
                                (a7) edge node[above]{$$} (b7)
			(c7) edge node[above]{$$} (b7)
                                (a6) edge node[above]{$$} (a7)
                                (b6) edge node[left]{$a$} (b7)
(b6) edge node[right]{$\tau_6$} (b7)
			(c6) edge node[above]{$$} (c7);
		\end{tikzpicture}
	\]
Here all of the vertical 1-morphisms on the left and right are associators or identities, the middle vertical 1-morphisms labeled on the left are the 2-morphisms from the previous commutative diagram, and the horizontal vertical 1-morphisms pointing towards the middle are natural maps into each colimit, all of which are naturally isomorphic to each other as all the middle objects are colimits of the same diagram, namely the previous collection of cospans, taken in various ways. The above diagram of maps of cospans can then be visualized as a hexagonal prism in which all the faces commute by identifying the top and the bottom as the same. As for the morphisms of decorations, which are labeled on the right of the interior vertical 1-morphisms, each isomorphism $\tau_n$ goes from the domain under the image of the functor $F$ applied to natural isomorphism adjacent to it to the codomain as written, meaning that, for example: $$\tau_1 \maps F(a \odot a)(((x_1+y_1)+z_1) \odot ((x_2+y_2)+z_2)) \to (x_1+(y_1+z_1)) \odot (x_2+(y_2+z_2)).$$  The following diagram commutes in the category $F((m_1+_b m_2) + ((n_1+_{b'} n_2) + (p_1+_{b''} p_2)))$:
\[
\begin{tikzpicture}[scale=1.5]
\node (A) at (0,0.5) {$F( a (\chi \otimes 1)\chi)(((x_1+y_1)+z_1) \odot ((x_2+y_2)+z_2))$};
\node (A') at (6,0.5) {$F((1 \otimes \chi) \chi)((x_1+(y_1+z_1)) \odot (x_2+(y_2+z_2)))$};
\node (B) at (0,-0.5) {$F(a(\chi  \otimes 1))(((x_1+y_1) \odot (x_2+y_2)) + (z_1 \odot z_2))$};
\node (C) at (6,-0.5) {$F(1 \otimes \chi)((x_1 \odot x_2) + ((y_1+z_1) \odot (y_2+z_2)))$};
\node (C') at (0,-1.5) {$F(a)(((x_1 \odot x_2)+(y_1 \odot y_2)) + (z_1 \odot z_2))$};
\node (D) at (6,-1.5) {$(x_1 \odot x_2) + ((y_1 \odot y_2) + (z_1 \odot z_2))$};
\path[->,font=\scriptsize,>=angle 90]
(A) edge node[above]{$F((1 \otimes \chi) \chi)(\tau_1)$} (A')
(A) edge node[left]{$F(a(\chi \otimes 1))(\tau_4)$} (B)
(A') edge node[right]{$F(1 \otimes \chi)(\tau_2)$} (C)
(B) edge node[left]{$F(a)(\tau_5)$} (C')
(C) edge node [right] {$\tau_3$} (D)
(C') edge node [above] {$\tau_6$} (D);
\end{tikzpicture}
\]
since $$F(a(\chi \otimes 1)\chi)(((x_1+y_1)+z_1) \odot ((x_2+y_2)+z_2)) = F((1 \otimes \chi) \chi (a \odot a))(((x_1+y_1)+z_1) \odot ((x_2+y_2)+z_2))$$
as the above underlying diagram of maps of cospans commutes and then applying the pseudofunctor $F$ yields a commutative diagram in $\Cat$.

Another requirement for a double category to be symmetric monoidal is that the braiding $$\beta_{ ( \_, \_ ) } \maps F\lCsp_1 \times F\lCsp_1 \to F\lCsp_1 \times F\lCsp_1$$ be a transformation of double categories, and one of the diagrams that is required to commute is the following:
\[
\begin{tikzpicture}[scale=1.5]
\node (A) at (0,0) {$(M_1 \odot M_2) \otimes (N_1 \odot N_2)$};
\node (B) at (3,0) {$(N_1 \odot N_2) \otimes (M_1 \odot M_2)$};
\node (C) at (0,-.75) {$(M_1 \otimes N_1) \odot (M_2 \otimes N_2)$};
\node (D) at (3,-.75) {$(N_1 \otimes M_1) \odot (N_2 \otimes M_2)$};
\path[->,font=\scriptsize,>=angle 90]
(A) edge node[above]{$\beta$} (B)
(B) edge node[right]{$\chi$} (D)
(A) edge node[left]{$\chi$} (C)
(C) edge node[above]{$\beta \odot \beta$} (D);
\end{tikzpicture}
\]
Using the same notation as the previous coherence diagram, the diagram for the underlying maps of cospans becomes:
\[
		\begin{tikzpicture}
			\node (a) at (-4,0) {$a+a'$};
			\node (b) at (1,0) {$(m_1+_b m_2) + (n_1+_{b'} n_2)$};
			\node (c) at (6,0) {$c+c'$};
			\node (a2) at (-4,1) {$a'+a$};
			\node (b2) at (1,1) {$(n_1 +_{b'} n_2) + (m_1 +_b m_2)$};
			\node (c2) at (6,1) {$c'+c$};
                                \node (a3) at (-4,2) {$a'+a$};
			\node (b3) at (1,2) {$(n_1+m_1) +_{(b'+b)} (n_2+m_2)$};
			\node (c3) at (6,2) {$c'+c$};
                                \node (a5) at (-4,-1) {$a+a'$};
			\node (b5) at (1,-1) {$(m_1+n_1) +_{(b+b')} (m_2+n_2)$};
			\node (c5) at (6,-1) {$c+c'$};
                                \node (a6) at (-4,-2) {$a'+a$};
			\node (b6) at (1,-2) {$(n_1+m_1) +_{(b'+b)} (n_2+m_2)$};
			\node (c6) at (6,-2) {$c'+c$};
			\path[->,font=\scriptsize,>=angle 90]
			(a) edge node[above]{$$} (b)
			(c) edge node[above]{$$} (b)
                                (a2) edge node[above]{$$} (b2)
			(c2) edge node[above]{$$} (b2)
                                (a) edge node[above]{$$} (a2)
                                (b) edge node[left]{$\beta$} (b2)
(b) edge node[right]{$\tau_1$} (b2)
			(c) edge node[above]{$$} (c2)
                                (a3) edge node[above]{$$} (b3)
			(c3) edge node[above]{$$} (b3)
                                (a2) edge node[above]{$$} (a3)
                                (b2) edge node[left]{$\chi$} (b3)
(b2) edge node[right]{$\tau_2$} (b3)
			(c2) edge node[above]{$$} (c3)
                                (a5) edge node[above]{$$} (b5)
			(c5) edge node[above]{$$} (b5)
                                (a) edge node[above]{$$} (a5)
                                (b) edge node[left]{$\chi$} (b5)
(b) edge node[right]{$\tau_4$} (b5)
			(c) edge node[above]{$$} (c5)
                                (a6) edge node[above]{$$} (b6)
			(c6) edge node[above]{$$} (b6)
                                (a5) edge node[above]{$$} (a6)
                                (b5) edge node[left]{$\beta \odot \beta$} (b6)
 (b5) edge node[right]{$\tau_5$} (b6)
			(c5) edge node[above]{$$} (c6);
		\end{tikzpicture}
	\]
All the comments about the previous underlying coherence diagram of maps of cospans apply to this one. As for the decorations, the following diagram commutes in the category $F((y_1+x_1)+_{(b'+b)}(y_2+x_2))$:
\[
\begin{tikzpicture}[scale=1.5]
\node (A) at (0,0) {$F(\chi \beta)((x_1 \odot x_2)+(y_1 \odot y_2))$};
\node (B) at (4,0) {$F(\chi)((y_1 \odot y_2)+(x_1 \odot x_2))$};
\node (C) at (0,-1) {$F(\beta \odot \beta)((x_1+y_1) \odot (x_2 + y_2))$};
\node (D) at (4,-1) {$(y_1+x_1) \odot (y_2+x_2)$};
\path[->,font=\scriptsize,>=angle 90]
(A) edge node[above]{$F(\chi)(\tau_1)$} (B)
(B) edge node[right]{$\tau_2$} (D)
(A) edge node[left]{$F(\beta \odot \beta)(\tau_3)$} (C)
(C) edge node[above]{$\tau_4$} (D);
\end{tikzpicture}
\]
since $$F(\chi \beta)((x_1 \odot x_2)+(y_1 \odot y_2)) = F((\beta \odot \beta)\chi)((x_1 \odot x_2)+(y_1 \odot y_2))$$
as the above underlying diagram of maps of cospans commutes and then applying the pseudofunctor $F$ to this diagram yields a commutative diagram in $\Cat$. The other diagrams are shown to commute similarly.
\end{comment}
\end{proof}


Regarding maps between symmetric monoidal double categories, consider another symmetric lax monoidal pseudofunctor $F' \maps \mathsf{A'} \to \Cat$%, we can obtain another symmetric monoidal double category $F' \lCsp$.
a morphism from $F\lCsp$ to $F' \lCsp$ will then be a double functor $\mathbb{H} \maps F\lCsp \to F' \lCsp$ whose object component is given by a finite colimit preserving functor $\mathbb{H}_0 = H \maps \A \to \mathsf{A'}$ and whose arrow component is given by a functor $\mathbb{H}_1$ defined on horizontal 1-cells by:
\[
\begin{tikzpicture}[scale=1.5]
\node (A) at (0,0) {$a$};
\node (B) at (1,0) {$c$};
\node (C) at (2,0) {$b$};
\node (D) at (1,-0.5) {$d \in F(c)$};
\node (E) at (2.75,0) {$\mapsto$};
\node (A') at (3.5,0) {$H(a)$};
\node (B') at (4.5,0) {$H(c)$};
\node (C') at (5.5,0) {$H(b)$};
\node (D') at (4.5,-0.5) {$\theta_c E(d) \phi \in F'(H(c))$};
\path[->,font=\scriptsize,>=angle 90]
(A) edge node[above]{$i$} (B)
(C) edge node[above]{$o$} (B)
(A') edge node[above]{$H(i)$} (B')
(C') edge node[above]{$H(o)$} (B');
\end{tikzpicture}
\]
and on 2-morphisms by:
\[
\begin{tikzpicture}[scale=1.5]
\node (A) at (0,0.5) {$a$};
\node (A') at (0,-0.5) {$a'$};
\node (B) at (1,0.5) {$c$};
\node (C) at (2,0.5) {$b$};
\node (C') at (2,-0.5) {$b'$};
\node (D) at (1,-0.5) {$c'$};
\node (E) at (3,0.5) {$d \in F(c)$};
\node (F) at (3,-0.5) {$d' \in F(c')$};
\node (G) at (1,-1) {$\tau \maps F(h)(d) \to d'$};
\node (A'') at (4.25,0.5) {$H(a)$};
\node (A''') at (4.25,-0.5) {$H(a')$};
\node (B'') at (5.25,0.5) {$H(c)$};
\node (C'') at (6.25,0.5) {$H(b)$};
\node (C''') at (6.25,-0.5) {$H(b')$};
\node (D'') at (5.25,-0.5) {$H(c')$};
\node (E'') at (8,0.5) {$\theta_c E(d) \phi \in F'(H(c))$};
\node (F'') at (8,-0.5) {$\theta_{c'} E(d') \phi \in F'(H(c'))$};
\node (G'') at (5.25,-1) {$E(\tau) \maps F'(H(h))(\theta_c E(d)\phi) \to (\theta_{c'}E(d')\phi)$};
\node (H) at (3.5,0) {$\mapsto$};
\path[->,font=\scriptsize,>=angle 90]
(A) edge node[above]{$$} (B)
(C) edge node[above]{$$} (B)
(A) edge node[left]{$f$} (A')
(C) edge node[right]{$g$} (C')
(A') edge node {$$} (D)
(C') edge node {$$} (D)
(B) edge node [left] {$h$} (D)
(A'') edge node[above]{$$} (B'')
(C'') edge node[above]{$$} (B'')
(A'') edge node[left]{$H(f)$} (A''')
(C'') edge node[right]{$H(g)$} (C''')
(A''') edge node {$$} (D'')
(C''') edge node {$$} (D'')
(B'') edge node [left] {$H(h)$} (D'');
\end{tikzpicture}
\]
where $E \maps \Cat \to \Cat$ is a symmetric lax monoidal pseudofuctor such that the following diagram commutes up to isomorphism $\theta \maps EF \Rightarrow F'H$:
\[
\begin{tikzpicture}[scale=1.5]
\node (A) at (0,0) {$\A$};
\node (B) at (1,0) {$\Cat$};
\node (C) at (0,-1) {$\mathsf{A'}$};
\node (D) at (1,-1) {$\Cat$};
\node (E) at (0.5,-0.5) {$\Swarrow \theta$};
\path[->,font=\scriptsize,>=angle 90]
(A) edge node[above]{$F$} (B)
(A) edge node[left]{$H$} (C)
(B) edge node[right]{$E$} (D)
(C) edge node[above]{$F'$} (D);
\end{tikzpicture}
\]

\begin{comment}
Recall that we can think of the object $d \in F(c)$ as a morphism $d \maps 1 \to F(c)$ and the functor $\tau \maps F(h)(d) \to d'$ of $F(c')$ as a natural transformation  in $\Cat$:
\[
\begin{tikzpicture}[scale=1.5]
\node (A) at (0,-0.5) {$1$};
\node (B) at (1,0) {$F(c)$};
\node (D) at (1,-1) {$F(c')$};
\node (C) at (0.75,-0.5) {$\Swarrow \tau$};
\path[->,font=\scriptsize,>=angle 90]
(A) edge node[above]{$d$} (B)
(A) edge node[below]{$d'$} (D)
(B) edge node[right]{$F(h)$} (D);
\end{tikzpicture}
\]
Applying the symmetric lax monoidal pseudofunctor $E \maps \Cat \to \Cat$ to this diagram yields:
\[
\begin{tikzpicture}[scale=1.5]
\node (A) at (0,-0.5) {$1 \xrightarrow{\phi} E(1)$};
\node (B) at (2,0) {$E(F(c))$};
\node (D) at (2,-1) {$E(F(c'))$};
\node (C) at (1.5,-0.5) {$\Swarrow E(\tau)$};
\path[->,font=\scriptsize,>=angle 90]
(A) edge node[above]{$E(d)$} (B)
(A) edge node[below]{$E(d')$} (D)
(B) edge node[right]{$E(F(h))$} (D);
\end{tikzpicture}
\]
Then because the above square commutes up to the isomorphism $\theta \maps EF \Rightarrow F'H$, we get:
\[
\begin{tikzpicture}[scale=1.5]
\node (A) at (0,-0.5) {$1 \xrightarrow{\phi} E(1)$};
\node (B) at (2,0) {$E(F(c))$};
\node (D) at (2,-1) {$E(F(c'))$};
\node (C) at (1.5,-0.5) {$\Swarrow E(\tau)$};
\node (E) at (4,0) {$F'(H(c))$};
\node (F) at (4,-1) {$F'(H(c'))$};
\path[->,font=\scriptsize,>=angle 90]
(B) edge node [above] {$\theta_c$} (E)
(D) edge node [above] {$\theta_{c'}$} (F)
(E) edge node [right] {$F'(H(h))$} (F)
(A) edge node[above]{$E(d)$} (B)
(A) edge node[below]{$E(d')$} (D)
(B) edge node[right]{$E(F((h))$} (D);
\end{tikzpicture}
\]
which results in a 2-morphism $E(\tau) \maps F'(H(h))(\theta_c E(d)\phi) \to (\theta_{c'} E(d')\phi)$ in $F'(H(c'))$. To check that the above recipe is functorial, given two vertically composable 2-morphisms in $F\lCsp$:
\[
\begin{tikzpicture}[scale=1.5]
\node (A) at (0,0.5) {$a$};
\node (A') at (0,-0.5) {$a'$};
\node (B) at (1.5,0.5) {$c$};
\node (C) at (3,0.5) {$b$};
\node (C') at (3,-0.5) {$b'$};
\node (D) at (1.5,-0.5) {$c'$};
\node (E) at (4.5,0.5) {$d \in F(c)$};
\node (F) at (4.5,-0.5) {$d' \in F(c')$};
\node (G) at (1.5,-1) {$\tau \maps F(h)(d) \to d'$};
\node (A'') at (0,-1.5) {$a'$};
\node (A''') at (0,-2.5) {$a''$};
\node (B'') at (1.5,-1.5) {$c'$};
\node (C'') at (3,-1.5) {$b'$};
\node (C''') at (3,-2.5) {$b''$};
\node (D'') at (1.5,-2.5) {$c''$};
\node (E'') at (4.5,-1.5) {$d' \in F(c')$};
\node (F'') at (4.5,-2.5) {$d'' \in F(c'')$};
\node (G'') at (1.5,-3) {$\tau' \maps F(h')(d') \to d''$};
\path[->,font=\scriptsize,>=angle 90]
(A) edge node[above]{$$} (B)
(C) edge node[above]{$$} (B)
(A) edge node[left]{$f$} (A')
(C) edge node[right]{$g$} (C')
(A') edge node [above]{$$} (D)
(C') edge node [above]{$$} (D)
(B) edge node [left] {$h$} (D)
(A'') edge node[above]{$$} (B'')
(C'') edge node[above]{$$} (B'')
(A'') edge node[left]{$f'$} (A''')
(C'') edge node[right]{$g'$} (C''')
(A''') edge node [above]{$$} (D'')
(C''') edge node [above]{$$} (D'')
(B'') edge node [left] {$h'$} (D'');
\end{tikzpicture}
\]
if we first compose these, the result is:
\[
\begin{tikzpicture}[scale=1.5]
\node (A) at (0,0.5) {$a$};
\node (A') at (0,-0.5) {$a''$};
\node (B) at (1.5,0.5) {$c$};
\node (C) at (3,0.5) {$b$};
\node (C') at (3,-0.5) {$b''$};
\node (D) at (1.5,-0.5) {$c''$};
\node (E) at (4.5,0.5) {$d \in F(c)$};
\node (F) at (4.5,-0.5) {$d'' \in F(c'')$};
\node (G) at (1.5,-1) {$\tau ' \tau \maps F(h' h)(d) \to d''$};
\path[->,font=\scriptsize,>=angle 90]
(A) edge node[above]{$$} (B)
(C) edge node[above]{$$} (B)
(A) edge node[left]{$f' f$} (A')
(C) edge node[right]{$g' g$} (C')
(A') edge node [above]{$$} (D)
(C') edge node [above]{$$} (D)
(B) edge node [left] {$h' h$} (D);
\end{tikzpicture}
\]
and then the image of this 2-morphism under the double functor $\mathbb{H}$ is given by:
\[
\begin{tikzpicture}[scale=1.5]
\node (A) at (0,0.5) {$H(a)$};
\node (A') at (0,-0.5) {$H(a'')$};
\node (B) at (1.5,0.5) {$H(c)$};
\node (C) at (3,0.5) {$H(b)$};
\node (C') at (3,-0.5) {$H(b'')$};
\node (D) at (1.5,-0.5) {$H(c'')$};
\node (E) at (5,0.5) {$\theta_c E(d) \phi \in F'(H(c))$};
\node (F) at (5,-0.5) {$\theta_{c''} E(d'') \phi \in F'(H(c''))$};
\node (G) at (1.5,-1) {$E(\tau ' \tau) \maps F'(H(h' h))(\theta_c E(d)\phi) \to (\theta_{c''}E(d'')\phi).$};
\path[->,font=\scriptsize,>=angle 90]
(A) edge node[above]{$$} (B)
(C) edge node[above]{$$} (B)
(A) edge node[left]{$H(f' f)$} (A')
(C) edge node[right]{$H(g' g)$} (C')
(A') edge node [above]{$$} (D)
(C') edge node [above]{$$} (D)
(B) edge node [left] {$H(h' h)$} (D);
\end{tikzpicture}
\]
On the other hand, applying the double functor $\mathbb{H}$ first gives:
\[
\begin{tikzpicture}[scale=1.5]
\node (A) at (0,0.5) {$H(a)$};
\node (A') at (0,-0.5) {$H(a')$};
\node (B) at (1.5,0.5) {$H(c)$};
\node (C) at (3,0.5) {$H(b)$};
\node (C') at (3,-0.5) {$H(b')$};
\node (D) at (1.5,-0.5) {$H(c')$};
\node (E) at (5,0.5) {$\theta_c E(d) \phi \in F'(H(c))$};
\node (F) at (5,-0.5) {$\theta_{c'} E(d') \phi \in F'(H(c'))$};
\node (G) at (1.5,-1) {$E(\tau) \maps F'(H(h))(\theta_c E(d)\phi) \to (\theta_{c'} E(d')\phi)$};
\node (A'') at (0,-1.5) {$H(a')$};
\node (A''') at (0,-2.5) {$H(a'')$};
\node (B'') at (1.5,-1.5) {$H(c')$};
\node (C'') at (3,-1.5) {$H(b')$};
\node (C''') at (3,-2.5) {$H(b'')$};
\node (D'') at (1.5,-2.5) {$H(c'')$};
\node (E'') at (5,-1.5) {$\theta_{c'} E(d') \phi \in F'(H(c'))$};
\node (F'') at (5,-2.5) {$\theta_{c''} E(d'') \phi \in F'(H(c''))$};
\node (G'') at (1.5,-3) {$E(\tau') \maps F'(H(h'))(\theta_{c'} E(d')\phi) \to (\theta_{c''}E(d'')\phi)$};
\path[->,font=\scriptsize,>=angle 90]
(A) edge node[above]{$$} (B)
(C) edge node[above]{$$} (B)
(A) edge node[left]{$H(f)$} (A')
(C) edge node[right]{$H(g)$} (C')
(A') edge node [above]{$$} (D)
(C') edge node [above]{$$} (D)
(B) edge node [left] {$H(h)$} (D)
(A'') edge node[above]{$$} (B'')
(C'') edge node[above]{$$} (B'')
(A'') edge node[left]{$H(f')$} (A''')
(C'') edge node[right]{$H(g')$} (C''')
(A''') edge node [above]{$$} (D'')
(C''') edge node [above]{$$} (D'')
(B'') edge node [left] {$H(h')$} (D'');
\end{tikzpicture}
\]
and then composing these gives:
\[
\begin{tikzpicture}[scale=1.5]
\node (A) at (0,0.5) {$H(a)$};
\node (A') at (0,-0.5) {$H(a'')$};
\node (B) at (1.5,0.5) {$H(c)$};
\node (C) at (3,0.5) {$H(b)$};
\node (C') at (3,-0.5) {$H(b'')$};
\node (D) at (1.5,-0.5) {$H(c'')$};
\node (E) at (5,0.5) {$\theta_c E(d) \phi \in F'(H(c))$};
\node (F) at (5,-0.5) {$\theta_{c''} E(d'') \phi \in F'(H(c''))$};
\node (G) at (1.5,-1) {$E(\tau ' \tau) \maps F'(H(h' h))(\theta_c E(d)\phi) \to (\theta_{c''}E(d'')\phi).$};
\path[->,font=\scriptsize,>=angle 90]
(A) edge node[above]{$$} (B)
(C) edge node[above]{$$} (B)
(A) edge node[left]{$H(f' f)$} (A')
(C) edge node[right]{$H(g' g)$} (C')
(A') edge node [above]{$$} (D)
(C') edge node [above]{$$} (D)
(B) edge node [left] {$H(h' h)$} (D);
\end{tikzpicture}
\]
This double functor $\mathbb{H}$ satisfies the equations $S \mathbb{H}_1 = HS$ and $T \mathbb{H}_1=HT$.

Given two composable horizontal 1-cells $M$ and $N$ in $F\lCsp$:
\[
\begin{tikzpicture}[scale=1.5]
\node (A) at (0,0) {$a_1$};
\node (B) at (1,0) {$c_1$};
\node (C) at (2,0) {$b$};
\node (D) at (1,-0.5) {$d_M \in F(c_1)$};
\node (E) at (3,0) {$b$};
\node (F) at (4,0) {$c_2$};
\node (G) at (5,0) {$a_2$};
\node (H) at (4,-0.5) {$d_N \in F(c_2)$};
\path[->,font=\scriptsize,>=angle 90]
(A) edge node[above]{$i_1$} (B)
(C) edge node[above]{$o_1$} (B)
(E) edge node[above]{$i_2$} (F)
(G) edge node[above]{$o_2$} (F);
\end{tikzpicture}
\]
composing first gives $M \odot N$:
\[
\begin{tikzpicture}[scale=1.5]
\node (A) at (0,0) {$a_1$};
\node (B) at (1.5,0) {$c_1+_b c_2$};
\node (C) at (3,0) {$a_2$};
\node (D) at (1.5,-0.5) {$d_{M \odot N} \in F(c_1 +_b c_2)$};
\path[->,font=\scriptsize,>=angle 90]
(A) edge node[above]{$\psi j_{c_1} i_1$} (B)
(C) edge node[above]{$\psi j_{c_2} o_2$} (B);
\end{tikzpicture}
\]
where $$d \maps 1 \xrightarrow{\lambda^{-1}} 1 \times 1 \xrightarrow{d_1 \times d_2} F(c_1) \times F(c_2) \xrightarrow{\phi_{c_1,c_2}} F(c_1+c_2) \xrightarrow{F(j)}F(c_1 +_b c_2).$$ The image of this horizontal 1-cell is then given by $\mathbb{H}(M \odot N)$:
\[
\begin{tikzpicture}[scale=1.5]
\node (A) at (0,0) {$H(a_1)$};
\node (B) at (2,0) {$H(c_1+_b c_2)$};
\node (C) at (4,0) {$H(a_2)$};
\node (D) at (2,-0.5) {$\mathbb{H}(M \odot N) = \theta_{c_1 +_b c_2}E(d_{M \odot N}) \phi \in F'(H(c_1 +_b c_2))$};
\path[->,font=\scriptsize,>=angle 90]
(A) edge node[above]{$H(\psi j_{c_1} i_1)$} (B)
(C) edge node[above]{$H(\psi j_{c_2} o_2)$} (B);
\end{tikzpicture}
\]
where $$\mathbb{H}(M \odot N) = \theta_{c_1 +_b c_2} E(d_{M \odot N}) \phi \maps 1 \xrightarrow{\phi} E(1) \xrightarrow{E(d)} E(F(c_1 +_b c_2)) \xrightarrow{\theta_{c_1 +_b c_2}} F'(H(c_1 +_b c_2)).$$ On the other hand, the image of each horizontal 1-cell under the double functor $\mathbb{H}$ is given respectively by $\mathbb{H}(M)$ and $\mathbb{H}(N)$:
\[
\begin{tikzpicture}[scale=1.5]
\node (A) at (0,0) {$H(a_1)$};
\node (B) at (1,0) {$H(c_1)$};
\node (C) at (2,0) {$H(b)$};
\node (D) at (1,-0.5) {$\theta_{c_1} E(d_M)\phi \in F'(H(c_1))$};
\node (E) at (3,0) {$H(b)$};
\node (F) at (4,0) {$H(c_2)$};
\node (G) at (5,0) {$H(a_2)$};
\node (H) at (4,-0.5) {$\theta_{c_2} E(d_N) \phi \in F'(H(c_2))$};
\path[->,font=\scriptsize,>=angle 90]
(A) edge node[above]{$H(i_1)$} (B)
(C) edge node[above]{$H(o_1)$} (B)
(E) edge node[above]{$H(i_2)$} (F)
(G) edge node[above]{$H(o_2)$} (F);
\end{tikzpicture}
\]
Composing these then gives $\mathbb{H}(M) \odot \mathbb{H}(N)$:
\[
\begin{tikzpicture}[scale=1.5]
\node (A) at (0,0) {$H(a_1)$};
\node (B) at (2.25,0) {$H(c_1)+_{H(b)} H(c_2)$};
\node (C) at (4.5,0) {$H(a_2)$};
\node (D) at (2.25,-0.5) {$d_{\mathbb{H}(M) \odot \mathbb{H}(N)} \in F'(H(c_1) +_{H(b)} H(c_2))$};
\path[->,font=\scriptsize,>=angle 90]
(A) edge node[above]{$\Psi j_{H(c_1)} H(i_1)$} (B)
(C) edge node[above]{$\Psi j_{H(c_2)} H(o_2)$} (B);
\end{tikzpicture}
\]
where $$\hspace{-.4in}\scriptstyle{d_{\mathbb{H}(M) \odot \mathbb{H}(N)} \maps 1 \xrightarrow{(\theta_{c_1} \times \theta_{c_2})(E(d_M) \times E(d_N)) \phi} F'(H(c_1)) \times F'(H(c_2)) \xrightarrow{\Phi_{H(c_1),H(c_2)}} F'(H(c_1)+ H(c_2)) \xrightarrow{F' (J)} F'(H(c_1) +_{H(b)} H(c_2)).}$$
We then have a comparison constraint: $$\mathbb{H}_{M,N} \maps \mathbb{H}(M) \odot \mathbb{H}(N) \xrightarrow{\sim} \mathbb{H}(M \odot N)$$given by the globular 2-isomorphism:
\[
\begin{tikzpicture}[scale=1.5]
\node (A) at (0,0.5) {$H(a_1)$};
\node (A') at (0,-0.5) {$H(a_1)$};
\node (B) at (2.5,0.5) {$H(c_1)+_{H(b)} H(c_2)$};
\node (C) at (5,0.5) {$H(a_2)$};
\node (C') at (5,-0.5) {$H(a_2)$};
\node (D) at (2.5,-0.5) {$H(c_1 +_b c_2)$};
\node (E) at (7.5,0.5) {$d_{\mathbb{H}(M) \odot \mathbb{H}(N)} \in F'(H(c_1)+_{H(b_1)}H(c_2))$};
\node (F) at (7.5,-0.5) {$d_{\mathbb{H}(M \odot N)} \in F'(H(c_1 +_b c_2))$};
\node (G) at (2.5,-1) {$\tau_{\kappa^{-1}} \maps F'(\kappa^{-1})(d_{\mathbb{H}(M) \odot \mathbb{H}(N)}) \to d_{\mathbb{H}(M \odot N)}.$};
\path[->,font=\scriptsize,>=angle 90]
(A) edge node[above]{$\Psi j_{H(c_1)} H(i_1)$} (B)
(C) edge node[above]{$\Psi j_{H(c_2)} H(o_2)$} (B)
(A) edge node[left]{$1$} (A')
(C) edge node[right]{$1$} (C')
(A') edge node [above]{$H(\psi j_{c_1} i_1)$} (D)
(C') edge node [above]{$H(\psi j_{c_2} o_2)$} (D)
(B) edge node [left] {$\kappa^{-1}$} (D);
\end{tikzpicture}
\]
where $\kappa$ is the isomorphism $$\kappa \maps H(c_1 +_b c_2) \xrightarrow{\sim} H(c_1) +_{H(b)} H(c_2)$$ which comes from the finite colimit preserving functor $H \maps \A \to \mathsf{A'}$. The above diagram commutes by a similar argument to the one used in \cref{thm:equiv}. Similarly, for every object $c \in \A$, we have a unit comparison constraint $$\mathbb{H}_U \maps U_{\mathbb{H}(c)} \to \mathbb{H}(U_c)$$ given by the globular 2-isomorphism:
\[
\begin{tikzpicture}[scale=1.5]
\node (A) at (0,0.5) {$H(c)$};
\node (A') at (0,-0.5) {$H(c)$};
\node (B) at (1.5,0.5) {$H(c)$};
\node (C) at (3,0.5) {$H(c)$};
\node (C') at (3,-0.5) {$H(c)$};
\node (D) at (1.5,-0.5) {$H(c)$};
\node (E) at (4.5,0.5) {$!_{H(c)} \in F'(H(c))$};
\node (F) at (4.5,-0.5) {$\theta_c E(!_c) \phi \in F'(H(c))$};
\path[->,font=\scriptsize,>=angle 90]
(A) edge node[above]{$1$} (B)
(C) edge node[above]{$1$} (B)
(A) edge node[left]{$1$} (A')
(C) edge node[right]{$1$} (C')
(A') edge node [above]{$1$} (D)
(C') edge node [above]{$1$} (D)
(B) edge node [left] {$1$} (D);
\end{tikzpicture}
\]
where the morphism of decorations is the morphism $\tau \maps !_{H(c)} \to (\theta_c E(!_c) \phi)$ in $F'(H(c))$. These comparison constrains satisfy the coherence axioms of a monoidal category, namely:
\[
\begin{tikzpicture}[scale=1.5]
\node (A) at (0,0.5) {$(\mathbb{H}(M) \odot \mathbb{H}(N)) \odot \mathbb{H}(P)$};
\node (B) at (0,-0.5) {$\mathbb{H}(M \odot N) \odot \mathbb{H}(P)$};
\node (C) at (0,-1.5) {$\mathbb{H}((M \odot N) \odot P)$};
\node (A') at (3,0.5) {$\mathbb{H}(M) \odot (\mathbb{H}(N) \odot \mathbb{H}(P))$};
\node (B') at (3,-0.5) {$\mathbb{H}(M) \odot \mathbb{H}(N \odot P)$};
\node (C') at (3,-1.5) {$\mathbb{H}(M \odot (N \odot P))$};
\path[->,font=\scriptsize,>=angle 90]
(A) edge node[left]{$\mathbb{H}_{M,N} \odot 1$} (B)
(B) edge node[left]{$\mathbb{H}_{M \odot N,P}$} (C)
(A) edge node[above]{$a$} (A')
(C) edge node [above] {$\mathbb{H}(a')$} (C')
(B') edge node [right] {$\mathbb{H}_{M,N \odot P}$} (C')
(A') edge node [right]{$1 \odot \mathbb{H}_{N,P}$} (B');
\end{tikzpicture}
\]
\[
\begin{tikzpicture}[scale=1.5]
\node (A) at (0,0) {$U_{\mathbb{H}(a)} \odot \mathbb{H}(M)$};
\node (B) at (2.5,0) {$\mathbb{H}(U_a) \odot \mathbb{H}(M)$};
\node (C) at (0,-1) {$\mathbb{H}(M)$};
\node (D) at (2.5,-1) {$\mathbb{H}(U_a \odot M)$};
\node (A') at (5,0) {$\mathbb{H}(M) \odot U_{\mathbb{H}(b)}$};
\node (B') at (7.5,0) {$\mathbb{H}(M) \odot \mathbb{H}(U_b)$};
\node (C') at (5,-1) {$\mathbb{H}(M)$};
\node (D') at (7.5,-1) {$\mathbb{H}(M \odot U_b)$};
\path[->,font=\scriptsize,>=angle 90]
(A) edge node[above]{$\mathbb{H}_U \odot 1$} (B)
(A) edge node[left]{$\lambda$} (C)
(B) edge node[right]{$\mathbb{H}_{U_a,M}$} (D)
(D) edge node[above]{$\mathbb{H}(\lambda')$} (C)
(A') edge node[above]{$1 \odot \mathbb{H}_U$} (B')
(A') edge node[left]{$\rho$} (C')
(B') edge node[right]{$\mathbb{H}_{M,U_b}$} (D')
(D') edge node[above]{$\mathbb{H}(\rho')$} (C');
\end{tikzpicture}
\]
The diagrams involving the morphisms of decorations are similar to those in Theorem \cref{DC}. This shows that $\mathbb{H}=(H,E,\theta)$ is a double functor. Next we show that this double functor is symmetric monoidal. First, that the object component $\mathbb{H}_0=H$ is symmetric monoidal is clear as $H \maps \A \to \A'$ preserves finite colimits. As for the arrow component $\mathbb{H}_1$, given two horizontal 1-cells $M_1$ and $M_2$ in $F\lCsp$:
\[
\begin{tikzpicture}[scale=1.5]
\node (A) at (0,0) {$a_1$};
\node (B) at (1,0) {$c_1$};
\node (C) at (2,0) {$b_1$};
\node (D) at (1,-0.5) {$d_{M_1}\in F(c_1)$};
\node (E) at (3,0) {$a_2$};
\node (F) at (4,0) {$c_2$};
\node (G) at (5,0) {$b_2$};
\node (H) at (4,-0.5) {$d_{M_2} \in F(c_2)$};
\path[->,font=\scriptsize,>=angle 90]
(A) edge node[above]{$i_1$} (B)
(C) edge node[above]{$o_1$} (B)
(E) edge node[above]{$i_2$} (F)
(G) edge node[above]{$o_2$} (F);
\end{tikzpicture}
\]
their tensor product $M_1 \otimes M_2$ in $F\lCsp$ is given by:
\[
\begin{tikzpicture}[scale=1.5]
\node (A) at (0,0) {$a_1+a_2$};
\node (B) at (1.5,0) {$c_1+c_2$};
\node (C) at (3,0) {$b_1+b_2$};
\node (D) at (1.5,-0.5) {$d_{M_1 \otimes M_2} \in F(c_1+c_2)$};
\path[->,font=\scriptsize,>=angle 90]
(A) edge node[above]{$i_1+i_2$} (B)
(C) edge node[above]{$o_1+o_2$} (B);
\end{tikzpicture}
\]
$$d_{M_1 \otimes M_2} \maps 1 \xrightarrow{d_1 \times d_2} F(c_1) \times F(c_2) \xrightarrow{\phi_{c_1,c_2}} F(c_1+c_2)$$
and the image of this horizontal 1-cell under the double functor $\mathbb{H}$ is $\mathbb{H}(M_1 \otimes M_2)$ given by:
\[
\begin{tikzpicture}[scale=1.5]
\node (A) at (0,0) {$H(a_1+a_2)$};
\node (B) at (2,0) {$H(c_1+c_2)$};
\node (C) at (4,0) {$H(b_1+b_2)$};
\node (D) at (2,-0.5) {$d_{\mathbb{H}(M_1 \otimes M_2)} = \theta_{c_1+c_2} E(d_{M_1 \otimes M_2})\phi \in F'(H(c_1+c_2))$};
\path[->,font=\scriptsize,>=angle 90]
(A) edge node[above]{$H(i_1+i_2)$} (B)
(C) edge node[above]{$H(o_1+o_2)$} (B);
\end{tikzpicture}
\]
On the other hand, the image of $M_1$ and $M_2$ is given by $\mathbb{H}(M_1)$ and $\mathbb{H}(M_2)$:
\[
\begin{tikzpicture}[scale=1.5]
\node (A) at (0,0) {$H(a_1)$};
\node (B) at (1.5,0) {$H(c_1)$};
\node (C) at (3,0) {$H(b_1)$};
\node (D) at (1.5,-0.5) {$d_{\mathbb{H}(M_1)} = \theta_{c_1}E(d_{M_1})\phi \in F'(H(c_1))$};
\node (E) at (4.5,0) {$H(a_2)$};
\node (F) at (6,0) {$H(c_2)$};
\node (G) at (7.5,0) {$H(b_2)$};
\node (H) at (6,-0.5) {$d_{\mathbb{H}(M_2)} = \theta_{c_2}E(d_{M_2})\phi \in F'(H(c_2))$};
\path[->,font=\scriptsize,>=angle 90]
(A) edge node[above]{$H(i_1)$} (B)
(C) edge node[above]{$H(o_1)$} (B)
(E) edge node[above]{$H(i_2)$} (F)
(G) edge node[above]{$H(o_2)$} (F);
\end{tikzpicture}
\]
and their tensor product $\mathbb{H}(M_1) \otimes \mathbb{H}(M_2)$ is given by:
\[
\begin{tikzpicture}[scale=1.5]
\node (A) at (0,0) {$H(a_1)+H(a_2)$};
\node (B) at (2.5,0) {$H(c_1)+H(c_2)$};
\node (C) at (5,0) {$H(b_1)+H(b_2)$};
\node (D) at (2.5,-0.5) {$d_{\mathbb{H}(M_1) \otimes \mathbb{H}(M_2)} \in F'(H(c_1)+H(c_2))$};
\path[->,font=\scriptsize,>=angle 90]
(A) edge node[above]{$H(i_1)+H(i_2)$} (B)
(C) edge node[above]{$H(o_1)+H(o_2)$} (B);
\end{tikzpicture}
\]
$$\hspace{-.4in} \scriptstyle{d_{\mathbb{H}(M_1) \otimes \mathbb{H}(M_2)} \maps 1 \xrightarrow{(\phi \times \phi)(\lambda^{-1} \times \lambda^{-1})} E(1) \times E(1) \xrightarrow{(\theta_{c_1} \times \theta_{c_2})(E(d_{M_1}) \times E(d_{M_2}))} F'(H(c_1)) \times F'(H(c_2)) \xrightarrow{\Phi_{H(c_1),H(c_2)}} F' (H(c_1)+H(c_2)).}$$  We then have a natural 2-isomorphism $\mu_{M_1,M_2} \maps \mathbb{H}(M_1) \otimes \mathbb{H}(M_2) \to \mathbb{H}(M_1 \otimes M_2)$ in $F'\lCsp$ given by:
\[
\begin{tikzpicture}[scale=1.5]
\node (A) at (0,0.5) {$H(a_1)+H(a_2)$};
\node (A') at (0,-0.5) {$H(a_1+a_2)$};
\node (B) at (2.5,0.5) {$H(c_1)+H(c_2)$};
\node (C) at (5,0.5) {$H(b_1)+H(b_2)$};
\node (C') at (5,-0.5) {$H(b_1+b_2)$};
\node (D) at (2.5,-0.5) {$H(c_1+c_2)$};
\node (E) at (2.5,1) {$d_{\mathbb{H}(M_1) \otimes \mathbb{H}(M_2)} \in F'(H(c_1)+H(c_2))$};
\node (F) at (2.5,-1) {$d_{\mathbb{H}(M_1 \otimes M_2)} \in F'(H(c_1+c_2))$};
\path[->,font=\scriptsize,>=angle 90]
(A) edge node[above]{$H(i_1)+H(i_2)$} (B)
(C) edge node[above]{$H(o_1)+H(o_2)$} (B)
(A) edge node[left]{$\kappa$} (A')
(C) edge node[left]{$\kappa$} (C')
(A') edge node [above]{$H(i_1+i_2)$} (D)
(C') edge node [above]{$H(o_1+o_2)$} (D)
(B) edge node [left] {$\kappa$} (D);
\end{tikzpicture}
\]
$$\tau_\kappa \maps F'(\kappa)(d_{\mathbb{H}(M_1) \otimes \mathbb{H}(M_2)}) \to d_{\mathbb{H}(M_1 \otimes M_2)}$$
where $\kappa$ denotes the isomorphism arising from $H$ preserving finite colimits. This natural 2-isomorphism together with the associators of $F\lCsp$ and $F'\lCsp$, respectively $\alpha$ and $\alpha'$, make the following diagram commute:
\[
\begin{tikzpicture}[scale=1.5]
\node (A) at (0,0.5) {$(\mathbb{H}(M_1) \otimes \mathbb{H}(M_2)) \otimes \mathbb{H}(M_3)$};
\node (B) at (0,-0.5) {$\mathbb{H}(M_1 \otimes M_2) \otimes \mathbb{H}(M_3)$};
\node (C) at (0,-1.5) {$\mathbb{H}((M_1 \otimes M_2) \otimes M_3)$};
\node (A') at (4,0.5) {$\mathbb{H}(M_1) \otimes (\mathbb{H}(M_2) \otimes \mathbb{H}(M_3))$};
\node (B') at (4,-0.5) {$\mathbb{H}(M_1) \otimes \mathbb{H}(M_2 \otimes M_3)$};
\node (C') at (4,-1.5) {$\mathbb{H}(M_1 \otimes (M_2 \otimes M_3))$};
\path[->,font=\scriptsize,>=angle 90]
(A) edge node[left]{$\mu_{M_1,M_2} \otimes 1$} (B)
(B) edge node[left]{$\mu_{M_1 \otimes M_2,M_3}$} (C)
(A) edge node[above]{$\alpha'$} (A')
(C) edge node [above] {$\mathbb{H}(\alpha)$} (C')
(B') edge node [right] {$\mu_{M_1,M_2 \otimes M_3}$} (C')
(A') edge node [right]{$1 \otimes \mu_{M_2 \otimes M_3}$} (B');
\end{tikzpicture}
\]
with the corresponding diagram of decorations in $F'(H(c_1+(c_2+c_3)))$:
\[
\begin{tikzpicture}[scale=1.5]
\node (A) at (0,0.5) {$F'(\alpha \kappa \kappa)(d_{(\mathbb{H}(M_1) \otimes \mathbb{H}(M_2)) \otimes \mathbb{H}(M_3)})$};
\node (B) at (0,-0.5) {$F'(\alpha \kappa)(d_{\mathbb{H}(M_1 \otimes M_2) \otimes \mathbb{H}(M_3)})$};
\node (C) at (0,-1.5) {$F'(\alpha)(d_{\mathbb{H}((M_1 \otimes M_2) \otimes M_3)})$};
\node (A') at (5,0.5) {$F'(\kappa \kappa)(d_{\mathbb{H}(M_1) \otimes (\mathbb{H}(M_2) \otimes \mathbb{H}(M_3))})$};
\node (B') at (5,-0.5) {$F'(\kappa)(d_{\mathbb{H}(M_1) \otimes \mathbb{H}(M_2 \otimes M_3)})$};
\node (C') at (5,-1.5) {$d_{\mathbb{H}(M_1 \otimes (M_2 \otimes M_3))}$};
\path[->,font=\scriptsize,>=angle 90]
(A) edge node[left]{$F'(\alpha \kappa)(\tau_\kappa + 1)$} (B)
(B) edge node[left]{$F'(\alpha)(\tau_\kappa)$} (C)
(A) edge node[above]{$F'(\kappa \kappa)(\tau_{\alpha'})$} (A')
(C) edge node [above] {$\tau_\alpha$} (C')
(B') edge node [right] {$\tau_\kappa$} (C')
(A') edge node [right]{$F'(\kappa)(1+\tau_\kappa)$} (B');
\end{tikzpicture}
\]
where $$F'(\alpha \kappa \kappa)(d_{(\mathbb{H}(M_1) \otimes \mathbb{H}(M_2)) \otimes \mathbb{H}(M_3)}) = F'(\kappa \kappa \alpha')(d_{(\mathbb{H}(M_1) \otimes \mathbb{H}(M_2)) \otimes \mathbb{H}(M_3)})$$
as the corresponding hexagon for the finite colimit preserving functor $H \maps \A \to \mathsf{A'}$ commutes. The map $\mu_{M_1,M_2}$ is also compatible with the braidings $\beta$ and $\beta'$ of $F \lCsp_1$ and $F' \lCsp_1$, respectively, and make the necessary square commute as a consequence of the corresponding commutative square involving braidings from the finite colimit preserving functor $H \maps \A \to \mathsf{A'}$.

We also have that the monoidal unit of $F\lCsp_1$ is given by:
\[
\begin{tikzpicture}[scale=1.5]
\node (A) at (0,0) {$1_\A$};
\node (B) at (1,0) {$1_\A$};
\node (C) at (2,0) {$1_\A$};
\node (D) at (1,-0.5) {$!_{1_\A} \in F(1_\A)$};
\path[->,font=\scriptsize,>=angle 90]
(A) edge node[above]{$1$} (B)
(C) edge node[above]{$1$} (B);
\end{tikzpicture}
\]
where $1_\A$ is the monoidal unit of the cocartesian category $\A$. The image of this horizontal 1-cell under $\mathbb{H}$ is given by:
\[
\begin{tikzpicture}[scale=1.5]
\node (A) at (0,0) {$H(1_\A)$};
\node (B) at (1,0) {$H(1_\A)$};
\node (C) at (2,0) {$H(1_\A)$};
%\node (E) at (3,0) {$=$};
%\node (F) at (4,0) {$1_\mathsf{A'}$};
%\node (G) at (5,0) {$1_\mathsf{A'}$};
%\node (H) at (6,0) {$1_\mathsf{A'}$};
%\node (I) at (5,-0.5) {$!_{1_{\mathsf{A'}}} \in F'(1_{\mathsf{A'}})$};
\node (D) at (1,-0.5) {$\theta_{1_\A}E(!_{1_\A}) \phi \in F'(H(1_\A))$};
\path[->,font=\scriptsize,>=angle 90]
(A) edge node[above]{$1$} (B)
(C) edge node[above]{$1$} (B);
%(F) edge node[above]{$1$} (G)
%(H) edge node[above]{$1$} (G);
\end{tikzpicture}
\]
as $H$ preserves finite colimits. We then have a 2-isomorphism in $F'\lCsp$ given by: $$\mu \maps 1_{F'\lCsp_1} \to \mathbb{H}(1_{F\lCsp_1})$$ 
\[
\begin{tikzpicture}[scale=1.5]
\node (A) at (0,0.5) {$1_{\mathsf{A'}}$};
\node (A') at (0,-0.5) {$H(1_\A)$};
\node (B) at (1.5,0.5) {$1_{\mathsf{A'}}$};
\node (C) at (3,0.5) {$1_{\mathsf{A'}}$};
\node (C') at (3,-0.5) {$H(1_\A)$};
\node (D) at (1.5,-0.5) {$H(1_\A)$};
\node (E) at (5,0.5) {$!_{1_\mathsf{A'}} \in F'(1_\mathsf{A'})$};
\node (F) at (5,-0.5) {$\theta_{1_\A} E(!_{1_\A}) \phi \in F'(H(1_\A))$};
\path[->,font=\scriptsize,>=angle 90]
(A) edge node[above]{$1$} (B)
(C) edge node[above]{$1$} (B)
(A) edge node[left]{$\kappa$} (A')
(C) edge node[right]{$\kappa$} (C')
(A') edge node [above]{$1$} (D)
(C') edge node [above]{$1$} (D)
(B) edge node [left] {$\kappa$} (D);
\end{tikzpicture}
\]
together with the morphism $\tau_\mu \maps F'(\kappa)(!_{1_\mathsf{A'}}) \to (\theta_{1_\A} E(!_{1_\A})\phi)$ in $F'(H(1_\A))$. The following square then commutes for any horizontal 1-cell $M$ of $F \lCsp$: 
\[
\begin{tikzpicture}[scale=1.5]
\node (A) at (0,0) {$1_\mathsf{A'} \otimes \mathbb{H}(M)$};
\node (B) at (2.5,0) {$\mathbb{H}(1_\A) \otimes \mathbb{H}(M)$};
\node (C) at (0,-1) {$\mathbb{H}(M)$};
\node (D) at (2.5,-1) {$\mathbb{H}(1_\A \otimes M)$};
\path[->,font=\scriptsize,>=angle 90]
(A) edge node[above]{$\mu \otimes 1$} (B)
(A) edge node[left]{$\ell$} (C)
(B) edge node[right]{$\mu_{1_\mathrm{A},M}$} (D)
(D) edge node[above]{$\mathbb{H}(\ell')$} (C);
\end{tikzpicture}
\]
where we have abbreviated the monoidal units of $F\lCsp_1$ and $F'\lCsp_1$ as $1_\A$ and $1_\mathsf{A'}$, respectively. The diagram of corresponding decorations is given by:
\[
\begin{tikzpicture}[scale=1.5]
\node (A) at (0,0.5) {$F'(\ell)(d_{!_{1_\mathsf{A'}}} \otimes d_{\mathbb{H}(M)})$};
\node (B) at (0,-0.5) {$d_{\mathbb{H}(M)}$};
\node (A') at (4,0.5) {$F'(H(\ell') \kappa)(d_{!_{\mathbb{H}(1_\A)}}\otimes d_{\mathbb{H}(M)})$};
\node (B') at (4,-0.5) {$F'(H(\ell'))(d_{\mathbb{H}(1_\A \otimes M)})$};
\path[->,font=\scriptsize,>=angle 90]
(A) edge node[left]{$\tau_\ell$} (B)
(A) edge node[above]{$F'(H(\ell')\kappa)(\tau_{\mu \otimes 1})$} (A')
(A') edge node [right]{$F'(H(\ell'))(\tau_\kappa)$} (B')
(B') edge node [above] {$\tau_{H(\ell')}$} (B);
\end{tikzpicture}
\]
where $$F'(\ell)(d_{!_{1_\mathsf{A'}}} \otimes d_{\mathbb{H}(M)})=F'(H(\ell')\kappa(\mu \otimes 1))(d_{!_{1_\mathsf{A'}}} \otimes d_{\mathbb{H}(M)})$$ since the corresponding square involving left unitors for the finite colimit preserving functor $H \maps \A \to \mathsf{A'}$ commutes. The other square involving the right unitors $r$ and $r'$ is similar. Note that because the comparison constraints $\mu$ and $\mu_{(\_ , \_)}$ are both isomorphisms, the symmetric monoidal double functor $\mathbb{H}$ is strong.

\end{comment}

\begin{thm}
Given two finitely cocomplete categories $\A$ and $\A'$, two symmetric lax monoidal pseudofunctors $F \maps \A \to \Cat$ and $F' \maps \A' \to \Cat$, a finite colimit preserving functor $H \maps \A \to \A'$, a symmetric lax monoidal pseudofunctor $E \maps \Cat \to \Cat$ and a 2-isomorphism $\theta \maps EF \Rightarrow F' H$ as in the following diagram, the triple $(H,E,\theta)$ induces a symmetric monoidal (strong) double functor $\mathbb{H} \maps F\lCsp \to F'\lCsp$ as defined above.
\[
\begin{tikzpicture}[scale=1.5]
\node (A) at (0,0) {$\A$};
\node (B) at (1,0) {$\Cat$};
\node (C) at (0,-1) {$\mathsf{A'}$};
\node (D) at (1,-1) {$\Cat$};
\node (E) at (0.5,-0.5) {$\Swarrow \theta$};
\path[->,font=\scriptsize,>=angle 90]
(A) edge node[above]{$F$} (B)
(A) edge node[left]{$H$} (C)
(B) edge node[right]{$E$} (D)
(C) edge node[above]{$F'$} (D);
\end{tikzpicture}
\]
\end{thm}

Details of the constructions and the proof of the above theorem can be found at \cite{CourserThesis}.

%\newpage
%Scratch work for Petri net problem
%\newline
%Let $P = (s_1,t_1 \maps T_1 \to \mathbb{N}(S_1))$ and $Q = (s_2,t_2 \maps T_2 \to \mathbb{N}(S_2))$. We want to show that the functor $F \maps \textrm{Petri} \to \textrm{CMC}$ preserves pushouts of cospans of the form $$P \xleftarrow{} L(Y) \xrightarrow{} Q$$ meaning that $F(P+_{L(Y)} Q) \cong F(P) +_{F(L(Y))} F(Q)$. Here, $L \maps \textrm{Set} \to \textrm{Petri}$ is the left adjoint which sends a set $Y$ to the discrete Petri net with $Y$ as its set of species and no transitions. Thus no transitions are identified in the pushout $P+_{L(Y)} Q$, and so the set of transitions of $P+_{L(Y)} Q$ is given by $$\textrm{Trans}(P)+\textrm{Trans}(Q).$$ Thus the morphisms \emph{contributed from transitions} in both $F(P+_{L(Y)}Q)$ and $F(P)+_{F(L(Y)} F(Q)$ are the same. If we can show that $$\textrm{Ob}(F(P+_{L(Y)}Q) \cong \textrm{Ob}(F(P)+_{F(L(Y))} F(Q)$$ then the remaining morphisms in each commutative monoidal category will be canonically isomorphic as sets.

%Now, what can we say about $\textrm{Ob}(F(P+_{L(Y)}Q))$ vs. $\textrm{Ob}(F(P)+_{F(L(Y))} F(Q))$? Well, $\textrm{Ob}(F(P+_{L(Y)}Q)) = \mathbb{N}(S_1+_Y S_2)$ and $\textrm{Ob}(F(P)+_{F(L(Y))} F(Q))= \mathbb{N}(S_1) +_{\mathbb{N}(Y)} \mathbb{N}(S_2)$.

%\newpage


%is given by $$1 \to F(c_1) \times F(c_2) \xrightarrow{} F(c_1 + c_2) \xrightarrow{} F(c_1 + _b c_2)$$ and then applying the functor $E$ results in $1 \to E(F(c_1+_b c_2))$, which by the comutative square is the same as $1 \to F'(H(c_1+_b c_2))$. On the other hand, applying the functor $E$ to each decoration results in $$1 \to E(F(c_1)) \times E(F(c_2)) \xrightarrow{} E(F(c_1) \times F(c_2)) \xrightarrow{} E(F(c_1+c_2)) \xrightarrow E(F(c_1+_b c_2)) \xrightarrow{} F'(H(c_1 +_b c_2))$$



\section{Structured versus decorated cospans} \label{EquivDoubleCats}

In \cite{BC2}, the first two authors introduce the symmetric monoidal double category of \emph{structured cospans} as a formalism to capture open networks. One of the main goals of this paper is to provide a monoidal double equivalence between this double category and that of \emph{decorated cospans}, described in detail in \cref{DecCospansDoubleCat}. We first recall the double category structured cospans form.

\begin{thm}\label{SC}
Given categories $\X$ and $\A$ with finite colimits and $L \maps \A \to \X$ a finitely cocontinuous functor, there exists a symmetric monoidal double category $_L\lCsp(\X)$ in which
\begin{itemize}
\item an object is an object of $\A$,
\item a vertical morphism is a morphism of $\A$,
\item a horizontal 1-cell from $a$ to $b$ is a cospan in $\X$ of the form
\begin{equation}\label{eq:structuredcospan}
\begin{tikzpicture}[scale=1.5]
\node (A) at (0,0) {$L(a)$};
\node (B) at (1,0) {$x$};
\node (C) at (2,0) {$L(b)$};
\path[->,font=\scriptsize,>=angle 90]
(A) edge node[above]{$$} (B)
(C) edge node[above]{$$} (B);
\end{tikzpicture}
\end{equation}
\item a 2-morphism is a map of cospans in $\X$ of the form
\begin{equation}\label{eq:2cellsStrCsp}
\begin{tikzpicture}[scale=1.5]
\node (A) at (0,0) {$L(a)$};
\node (B) at (1,0) {$x$};
\node (C) at (2,0) {$L(b)$};
\node (A') at (0,-1) {$L(a')$};
\node (B') at (1,-1) {$x'$};
\node (C') at (2,-1) {$L(b')$};
\path[->,font=\scriptsize,>=angle 90]
(A) edge node[above]{$$} (B)
(C) edge node[above]{$$} (B)
(A') edge node[above]{$$} (B')
(C') edge node[above]{$$} (B')
(A) edge node [left]{$L(f)$} (A')
(B) edge node [left]{$\alpha$} (B')
(C) edge node [right]{$L(g)$} (C');
\end{tikzpicture}
\end{equation}
\end{itemize}
The symmetric monoidal structure is formed by using binary coproducts and the initial object.
\end{thm}

\begin{proof}
All relevant details can be found in \cite[Theorems~2.3\&3.9]{BC2}. In particular, the double category structure only requires that $\X$ have pushouts, since this is how composition is formed. Moreover, the symmetric monoidal structure requires that $\X$ also have binary coproducts and an initial object (thus all finite colimits), and that $\A$ have and $L$ preserve finite coproducts, see \cite[Theorem~3.2.3]{CourserThesis}.
\end{proof}

The following theorem establishes the desired equivalence between these two different double categorical network frameworks of structured and decorated cospans in a formal way. We denote by $\bicat{Rex}$ the 2-category of finitely cocomplete categories and finitely cocontinuous functors, and the symmetric monoidal double structure of decorated cospans is worked out in \cref{thm:decorated_cospans,DC}.

\begin{thm} \label{thm:equiv}
Suppose $\ca{A}$ has finite colimits and $(F,\phi,\phi_0)\maps(\ca{A},+) \to (\bicat{Cat},\times)$ is a symmetric lax monoidal pseudofunctor that factors through $\bicat{Rex}$ as an ordinary pseudofunctor.   Then the symmetric monoidal double category $F\lCsp$ of decorated cospans is equivalent to the symmetric monoidal double category $_L\lCsp(\inta F)$ of structured cospans, where $L \maps \A \to \inta F$ is the left adjoint of the induced Grothendieck opfibration $U \maps \inta F \to \A$.
\end{thm}

The conditions of the above theorem relate to the existence of left adjoints for opfibrations; we first sketch this underlying framework, and then we proceed to the proof of the theorem. The basics of fibration theory needed for our purposes are recalled in \cref{sec:fibrations}. 

In \cite{MV}, the classical Grothendieck construction is generalized to the monoidal setting: lax monoidal structures on a pseudofunctor $F\maps\A\to\bicat{Cat}$ (called \emph{monoidal opindexed categories}) bijectively correspond to monoidal structures on the total category $\inta F$ such that the induced opfibration is a strict monoidal functor and $\otimes_{\inta F}$ preserves cocartesian liftings (called \emph{monoidal fibrations}). If the monoidal category $\A$ is in fact cocartesian, there is a further correspondence of the two structures with an ordinary pseudofunctor $F\maps\A\to\bicat{MonCat}$ into the 2-category of monoidal categories, strong monoidal functors and monoidal natural transformations.

\begin{lem}\label{lem:MonGroth}
 There is a 2-equivalence between the 2-categories of monoidal opfibrations and monoidal opindexed categories. If the base is cocartesian monoidal, there is a 2-equivalence between monoidal opfibrations and pseudofunctors into monoidal categories. 
\end{lem}

\begin{proof}
This was shown by Moeller and the third author \cite[Theorems~3.13\&4.1]{MV}. In summary, for a cocartesian base we have correspondences
\begin{gather}
\textrm{lax monoidal pseudofunctors }F\maps(\A,+)\to(\bicat{Cat},\times) \notag\\
\Updownarrow \notag\\
\textrm{monoidal opfibrations }U\maps(\X,\otimes)\to(\A,+) \label{monGroth}\\
\Updownarrow \notag\\
\textrm{pseudofunctors }F\maps\A\to\bicat{MonCat} \notag
\end{gather}
The second equivalence was also earlier observed by Shulman \cite{Shulman2008}. In more detail, if $(\mu,\mu_0)$ is the lax monoidal structure of the pseudofunctor $F$ as recalled in \cref{sec:2cats}, each fibre category $\X_a=F(a)$ obtains a monoidal structure via
\begin{equation}\label{eq:explicitstructure1}
\otimes_a\maps F(a)\times F(a)\xrightarrow{\mu_{a,a}}F(a+a)\xrightarrow{F(\nabla)}F(a),\quad
I_a\maps\mathbf{1}\xrightarrow{\mu_0}F(0)\xrightarrow{F(!)}F(a)
\end{equation}
where $\nabla$ is the fold map. 
\end{proof}

These correspondences further restrict to the case when the Grothendieck category is specifically cocartesian monoidal itself. In that case, opfibrations
$(\X,+)\to(\A,+)$ that strictly preserve chosen coproducts and initial object
%, and such that $+$ preserves cocartesian liftings, Christina: this should ALWAYS be the case, by universal properties!
bijectively correspond to pseudofunctors into the 2-category of cocartesian categories; for more details, see \cite[Corollary 4.7]{MV} and the related discussion.

Finally, we can further restrict to the setting of opfibrations that also preserve pushouts, i.e. all finite colimits, which is important for our purposes. The following result is in fact more general since it relates the existence of any class of colimits in the total category of an opfibration to their existence in the fibres; for a more detailed exposition, see \cite[Corollary~4.9]{Hermida1999}.
%as well as Remark~4.11 therein regarding the non-necessary strictness condition. Is this obscuring or clarifying?

\begin{lem}\label{lem:fibrewiselimits}
Suppose $\ca{J}$ is a small category and $\U\maps\X\to\A$ is an opfibration. If the base $\A$ has $\ca{J}$-colimits,
the following are equivalent:
\begin{enumerate}
 \item all fibres have $\ca{J}$-colimits, and the reindexing functors preserve them;
 \item the total $\X$ has $\ca{J}$-colimits, and $\U$ (strictly) preserves them.
\end{enumerate}
\end{lem}

The first part formulates the existence of colimits \emph{locally} in each fibre, which for $\ca{J}$ a finite category can equivalently be expressed by the corresponding pseudofunctor $F\maps\A\to\bicat{Cat}$ landing on the sub-2-category $\bicat{Rex}$ of finitely cocomplete categories and functors that preserve colimits. The second part formulates the existence of colimits \emph{globally} in the total category $\inta F$. Combining this with \cref{lem:MonGroth}, we deduce the following.

\begin{cor}\label{cor:fcocMonGroth}
 Suppose $\A$ is a finitely cocomplete category and $F\maps(\A,+)\to(\bicat{Cat},\times)$ is a lax monoidal pseudofunctor. If its corresponding pseudofunctor $\A\to\bicat{MonCat}$ via \cref{monGroth} 
 factors through $\bicat{Rex}$, then the Grothendieck category $\inta F$ has all finite colimits and the induced opfibration $\U_F\maps\inta F\to\A$ preserves them. 
\end{cor}

On a highly related note, in what follows we are also interested in the existence of a left adjoint $L_F$ to an induced monoidal opfibration; the following result provides sufficient conditions for that.
%, which go hand-in-hand with \cref{cor:fcocMonGroth} and \cref{lem:fibrewiselimits}.

\begin{lem}\label{prop:opfibtolari}
  Let $\U \maps\X \to \A$ be an opfibration. Then
  $\U$ is a right adjoint `left inverse', a.k.a the unit is the identity, if (and only if) its fibers have
  initial objects which are preserved by the
  reindexing functors.
\end{lem}

\begin{proof}\emph{(Sketch)}
 This is \cite[Proposition~4.4]{Gray}. The left adjoint $L\maps\A\to\X$ maps an object $a$ to the initial object in its fibre $\X_a$, denoted by $\bot_a$. By construction, $U(L(a))=U(\bot_a)=a$. On morphisms $f\maps a\to a'$, $L(f)$ is defined by
 \begin{equation}\label{eq:Lonarrows}
  \bot_a\xrightarrow{\mathrm{Cocart}(f,\bot_a)}f_!(\bot_a)\stackrel{\chi_a}{\cong}\bot_{a'}
 \end{equation}
where $\mathrm{Cocart}(f,\bot_a)$ is the cocartesian lifting of $\bot_a$ along $f$ and $\chi_a$ is the unique isomorphism between the initial objects in the fibre above $a'$ since $f_!$ preserves them.
\end{proof}

Ongoing work \cite{CV} includes a related discussion as well as a result in the opposite direction.
%For a discussion on the unusual strict cocontinuity condition...
%\begin{lem}\label{prop:CV}\cite[Proposition~3.3]{CV}
%Suppose $\U\maps\X\to\A$ is a right adjoint `left inverse'. If $\X$ and $\A$ have chosen pushouts and initial objects and $\U$ strictly preserves them, %then $\U$ is an opfibration.
%\end{lem}

\begin{rmk}\label{rmk:important}
Notice that under \cref{lem:fibrewiselimits}, if the base category $\A$ has an initial object denoted $0_\A$, the above is equivalent to $\X$ having an initial object $0_\X$ above $0_\A$. The way in which these two conditions give one another in this case is as follows. 
The fibrewise initial object $\bot_a$ is precisely the cocartesian lifting of $0_\X$ along the unique map $u_a\maps0_\A\to a$ in the base category
\begin{displaymath}
\xymatrix @C=.4in @R=.2in
{0_\X \ar @{.>}[d]\ar[rr]^-{\mathrm{Cocart}(0_\X,u_a)} && (u_a)_!(0_X)=:\bot_a\ar @{.>}[d] & \textrm{in }\X  \\
0_\A\ar[rr]^-{\exists!u_a} && a & \textrm{in }\A}
\end{displaymath}
Moreover, if the opfibration $U$ comes from the Grothendieck construction on a pseudofunctor $F\maps\A\to\bicat{Cat}$, the reindexing functors $(u_a)_!$ of the opfibration are precisely $F(u_a)$ therefore $\bot_a=(a,F(u_a)(0_\X))$ in the Grothendieck category notation (\cref{def:GrothCat}).

Finally, if the original pseudofunctor is in fact lax monoidal $(F,\mu,\mu_0)$, the monoidal Grothendieck construction of \cref{lem:MonGroth} in the cocartesian case expresses $\bot_a$ as follows by \cref{eq:explicitstructure1}:
\begin{equation}\label{eq:initialinfibre}
\mathbf{1}\xrightarrow{\phi_0}F(0_\A)\xrightarrow{F(u_a)}F(a) 
\end{equation}
\end{rmk}

We have now laid all the necessary background to formally construct an equivalence between the double category of decorated cospans and the double category of structured cospans, with starting point a symmetric lax monoidal pseudofunctor into $\bicat{Cat}$.

{\chris In progress...}

\begin{proof}[Proof of \cref{thm:equiv}]
Recall from \cref{thm:decorated_cospans} that the double category of decorated cospans $\dcsp{F}$ has objects and vertical 1-morphisms as in $\A$, while a horizontal 1-cell $a\tickar b$ is cospan $\cspn{a}{m}{b}{}{}$ in $\A$ decorated by an object $x\in F(m)$ and a 2-morphism is a map $k\maps m\to m'$ together with a morphism $(Fk)(x)\to x'$.

By \cref{cor:fcocMonGroth}, when $F$ as an ordinary pseudofunctor factors through $\bicat{Rex}$, the (monoidal) Gro\-the\-ndieck construction gives rise to a finitely cocomplete $\inta F$ such that the corresponding opfibration $U_F\maps(\inta{F},+)\to(\A,+)$ preserves all finite colimits. %Christina: maybe the cocartesian structures are redundant to write here, since U_F really preserves all finite colimits anyways.
Since in particular it preserves the initial object, \cref{prop:opfibtolari} through \cref{lem:fibrewiselimits} applies 
%which equivalently means that all fibres have initial objects and the reindexing functors preserve them by , %Christina: there is a little bit of circularity here. We knew fibres have and reindexing functors preserve from the factorization already! Change Gray accordingly perhaps?
to construct a left adjoint $L_F\maps\A\to\inta F$ which is also a right inverse, namely $U_FL_F=\mathrm{id}_\A$. Explicitly, this left adjoint is given by $L(a)=(a,\bot_a)$, picking the initial object in the finitely cocomplete fibre also expressed as $F(u_a)\circ\phi_0(*)$ in \cref{eq:initialinfibre}. Diagrammatically,
\begin{displaymath}
 F\maps\X\to\bicat{Cat}\quad\mapsto\quad\begin{tikzcd}[baseline=.3]\inta F\ar[d,"U_F"'] \\ \X \end{tikzcd}\quad\mapsto\quad\begin{tikzcd}\X\ar[r,bend left,pos=.55,"L_F"]\ar[r,phantom,"\bot"description] & \inta F\ar[l,bend left,pos=.45,"U_F"]\end{tikzcd}
\end{displaymath}
roughly describes the processes between the original $F$ and the resulting $L_F$.

Using this functor $L_F$ between finitely cocomplete categories that preserves all colimits that exist (as a left adjoint), we can construct the double category of structured cospans $\scsp{L_F}{\inta F}$. Its objects and vertical morphisms are again those of the category $\A$, whereas horizontal 1-cells $a\tickar b$ as in \cref{eq:structuredcospan} are now cospans of the form $\cspn{L_F(a)}{v}{L_F(b)}{}{}$ in the Grothendieck category $\inta F$. Explicitly, they consist of two pairs of morphisms 
\begin{equation}\label{eq:scsphor1cell}
 (a,\bot_a)\xrightarrow{\scalebox{0.7}{\(\begin{cases}i\maps a\to m &\textrm{in }\A \\!\maps F(i)(\bot_a)\to x &\textrm{in }F(m)\end{cases}\)}}(m,x)\xleftarrow{\scalebox{0.7}{\(\begin{cases}o\maps b\to m &\textrm{in }\A \\!\maps F(o)(\bot_b)\to x &\textrm{in }F(m)\end{cases}\)}}(b,\bot_b)
\end{equation}
where $x\in F(m)$, according to \cref{def:GrothCat}. Finally, the 2-morphisms in this double category are as described in \cref{eq:2cellsStrCsp}, in this context fully unravelled below:
\begin{displaymath}
 \begin{tikzcd}[sep=1.5in,ampersand replacement=\&]
 (a,\bot_a)\ar[r,"{\begin{cases}i\maps a\to m &\textrm{in }\A \\!\maps F(i)(\bot_a)\to x &\textrm{in }F(m)\end{cases}}"]\ar[d,"{\begin{cases}f\maps a\to a' &\textrm{in }\A \\\chi_a\maps F(f)(\bot_a)\cong \bot_{a'} &\textrm{in }F(a')\end{cases}}"description] \& (m,x) \ar[d,"{\begin{cases}k\maps m\to m' &\textrm{in }\A \\ \tau\maps F(k)(x)\to x' &\textrm{in }F(m')\end{cases}}"description] \& (b,\bot_b)\ar[l,"{\begin{cases}o\maps b\to m &\textrm{in }\A \\!\maps F(o)(\bot_b)\to x &\textrm{in }F(m)\end{cases}}"']\ar[d,"{\begin{cases}g\maps b\to b' &\textrm{in }\A \\\chi_b\maps F(g)(\bot_b)\cong \bot_{b'} &\textrm{in }F(b')\end{cases}}"description] \\
 (a',\bot_{a'})\ar[r,"{\begin{cases}i'\maps a'\to m' &\textrm{in }\A \\!\maps F(i')(\bot_{a'})\to x' &\textrm{in }F(m')\end{cases}}"'] \& (m',x') \& (b',\bot_{b'})\ar[l,"{\begin{cases}o'\maps b'\to m' &\textrm{in }\A \\!\maps F(o')(\bot_{b'})\to x' &\textrm{in }F(m')\end{cases}}"]
 \end{tikzcd}
\end{displaymath}
where the outside vertical legs come from the definition of $L_F$ on arrows, \cref{eq:Lonarrows}. 
%!!! The above diagram might be a bit overwhelming, but all info is there. Choose which parts are required for a smooth proof!
The square commutativities translate to $k\circ i=i'\circ f$ and $k\circ o=o'\circ g$ in $\A$, and also by composition in the Grothendieck category, to
\begin{gather}\label{eq:Grothcommutativity}
 F(k\circ i)(\bot_a)\cong Fk(Fi(\bot_a))\xrightarrow{Fk(!)}Fk(x)\xrightarrow{\tau}x'= \\
 F(i'\circ f)(\bot_a)\cong Fi'(Ff(\bot_a))\xrightarrow{Fi'(\chi_a)}Fi'(\bot_{a'})\xrightarrow{!}x'\nonumber
\end{gather}
in the category $F(m')$. However, since all maps are unique in the above equality, emanating from initial objects (since reindexing functors preserve those), this gives no extra conditions for the morphisms involved; similarly for the equality including $o, o'$.

In order to prove that there is a double equivalence (\cref{ShulDubEquiv}) between $\dcsp{F}$ and $\scsp{L_F}{\inta F}$, we define a %n identity-on-objects and vertical morphism
double functor
\begin{displaymath}
\mathbb{E}\maps\scsp{L_F}{\inta F} \longrightarrow\dcsp{F}
\end{displaymath}
as follows. The object component of $\mathbb{E}=(\mathbb{E}_0,\mathbb{E}_1)$ is $\mathbb{E}_0 = \id_{\A}$ since both double categories have $\A$ as their vertical category; trivially, $\mathbb{E}_0$ is an equivalence of categories. Given a horizontal 1-cell in $\scsp{L_F}{\inta F}$ like \cref{eq:scsphor1cell}, the arrow component $\mathbb{E}_1$ of the double functor simply maps it to the decorated cospan
\begin{displaymath}
 a\xrightarrow{\; i \;}m\xleftarrow{\; o \;}b\;\textrm{ with decoration }x\in F(m)
\end{displaymath}
Notice that this really is a bijective correspondence since the unique maps from the initial objects in the fibres provide no extra information. Finally, given a 2-morphism of $L_F$-structured cospans as described above, $\mathbb{E}_1$ maps it to the underlying cospan 2-morphism
\begin{displaymath}
 \begin{tikzcd}
a\ar[r,"i"]\ar[d,"f"'] & m\ar[d,"k"] & b\ar[l,"o"']\ar[d,"g"] \\
a'\ar[r,"i'"']& m' & b'\ar[l,"o'"]
 \end{tikzcd}
\end{displaymath}
 in $\A$ along with the map $\tau\maps Fk(x)\to x'$ in $F(m')$, namely a decorated cospan 2-morphism as per \cref{eq:FCsp2morph}. Once again, this is a bijective correspondence since, as discussed above, the commutativity \cref{eq:Grothcommutativity} holds automatically by initiality of the domain.
%!!!Christina: commented out big chunks of proof. They are right below if we want to revive parts of them, if we think the reader needs deeper level of detail. The idea is that the above description suffices as fully faithful and essentially surjective evidence, but perhaps we would like to spell that out more. On to strongness next.
\begin{comment}
which is a cospan in $\X$ of the form:
\[
\begin{tikzpicture}[scale=1.5]
\node (A) at (0,0) {$L(c)$};
\node (B) at (1,0) {$x$};
\node (C) at (2,0) {$L(c')$};
\path[->,font=\scriptsize,>=angle 90]
(A) edge node[above]{$i$} (B)
(C) edge node[above]{$o$} (B);
\end{tikzpicture}
\]
the image $\mathbb{E}_1(M)$ is given by the pair:
\[
\begin{tikzpicture}[scale=1.5]
\node (A) at (0,0) {$c$};
\node (B) at (1,0) {$R(x)$};
\node (C) at (2,0) {$c'$};
\node (D) at (3,0) {$x \in F(R(x))$};
\path[->,font=\scriptsize,>=angle 90]
(A) edge node[above]{$R(i) {\eta_c}$} (B)
(C) edge node[above]{$R(o) {\eta_{c'}}$} (B);
\end{tikzpicture}
\]
where $R\maps \X \to \A$ is the right adjoint to the functor $L \maps \A \to \X$ and $\eta \maps 1_{\A} \to RL$ is the unit of the adjunction $L \dashv R$ which is an isomorphism since $L$ is fully faithful. \textbf{Hopefully...actually, it's a `lari'!} Similarly, the image of a 2-morphism $\alpha \maps M \to N$ in $_L \lCsp(\X)$:
\[
\begin{tikzpicture}[scale=1.5]
\node (A) at (0,0) {$L(c_1)$};
\node (B) at (1,0) {$x$};
\node (C) at (2,0) {$L(c_2)$};
\node (A') at (0,-1) {$L(c_1')$};
\node (B') at (1,-1) {$x'$};
\node (C') at (2,-1) {$L(c_2')$};
\path[->,font=\scriptsize,>=angle 90]
(A) edge node[above]{$i$} (B)
(C) edge node[above]{$o$} (B)
(A') edge node[above]{$i'$} (B')
(C') edge node[above]{$o'$} (B')
(A) edge node [left]{$L(f)$} (A')
(B) edge node [left]{$\alpha$} (B')
(C) edge node [right]{$L(g)$} (C');
\end{tikzpicture}
\]
is given by the 2-morphism $\mathbb{E}_1(\alpha) \maps \mathbb{E}_1(M) \to \mathbb{E}_1(N)$ in $\mathbb{F}\textnormal{Cospan}(\A)$ given by:
\[
\begin{tikzpicture}[scale=1.5]
\node (A) at (0,0) {$c_1$};
\node (B) at (1,0) {$R(x)$};
\node (C) at (2,0) {$c_2$};
\node (A') at (0,-1) {$c_1'$};
\node (B') at (1,-1) {$R(x')$};
\node (C') at (2,-1) {$c_2'$};
\node (D) at (3,0) {$x \in F(R(x))$};
\node (D') at (3,-1) {$x' \in F(R(x'))$};
\path[->,font=\scriptsize,>=angle 90]
(A) edge node[above]{$R(i) \eta_{c_1}$} (B)
(C) edge node[above]{$R(o) \eta_{c_2}$} (B)
(A') edge node[above]{$R(i') \eta_{c_1'}$} (B')
(C') edge node[above]{$R(o') \eta_{c_2'}$} (B')
(A) edge node [left]{$f$} (A')
(B) edge node [left]{$R(\alpha)$} (B')
(C) edge node [right]{$g$} (C');
\end{tikzpicture}
\]
together with a morphism $\tau \maps F(R(\alpha))(x) \to x'$ in $F(R(x')) \subseteq \mathrm{X}$ which comes from the Grothendieck construction of the pseudofunctor $F \maps \A \to \Cat$.  That $\mathbb{E}_0$ is a functor is clear. For $\mathbb{E}_1$, given two vertically composable 2-morphisms $M$ and $M'$ in $_L \lCsp(\X)$,
\[
\begin{tikzpicture}[scale=1.5]
\node (A) at (0,0) {$L(c_1)$};
\node (B) at (1,0) {$x$};
\node (C) at (2,0) {$L(c_2)$};
\node (A') at (0,-1) {$L(c_1')$};
\node (B') at (1,-1) {$x'$};
\node (C') at (2,-1) {$L(c_2')$};
\node (A'') at (0,-2) {$L(c_1')$};
\node (B'') at (1,-2) {$x'$};
\node (C'') at (2,-2) {$L(c_2')$};
\node (A''') at (0,-3) {$L(c_1'')$};
\node (B''') at (1,-3) {$x''$};
\node (C''') at (2,-3) {$L(c_2'')$};
\path[->,font=\scriptsize,>=angle 90]
(A) edge node[above]{$i$} (B)
(C) edge node[above]{$o$} (B)
(A') edge node[above]{$i'$} (B')
(C') edge node[above]{$o'$} (B')
(A) edge node [left]{$L(f)$} (A')
(B) edge node [left]{$\alpha$} (B')
(C) edge node [right]{$L(g)$} (C')
(A'') edge node[above]{$i'$} (B'')
(C'') edge node[above]{$o'$} (B'')
(A''') edge node[above]{$i''$} (B''')
(C''') edge node[above]{$o''$} (B''')
(A'') edge node [left]{$L(f')$} (A''')
(B'') edge node [left]{$\alpha'$} (B''')
(C'') edge node [right]{$L(g')$} (C''');
\end{tikzpicture}
\]
their vertical composite $M'M$ is given by:
\[
\begin{tikzpicture}[scale=1.5]
\node (A) at (0,0) {$L(c_1)$};
\node (B) at (1,0) {$x$};
\node (C) at (2,0) {$L(c_2)$};
\node (A') at (0,-1) {$L(c_1'')$};
\node (B') at (1,-1) {$x''$};
\node (C') at (2,-1) {$L(c_2'')$};
\path[->,font=\scriptsize,>=angle 90]
(A) edge node[above]{$i$} (B)
(C) edge node[above]{$o$} (B)
(A') edge node[above]{$i''$} (B')
(C') edge node[above]{$o''$} (B')
(A) edge node [left]{$L(f'f)$} (A')
(B) edge node [left]{$\alpha' \alpha$} (B')
(C) edge node [right]{$L(g' g)$} (C');
\end{tikzpicture}
\]
and the image of this 2-morphism $\mathbb{E}_1(M'M)$ is given by:
\[
\begin{tikzpicture}[scale=1.5]
\node (A) at (0,0) {$c_1$};
\node (B) at (1.5,0) {$R(x)$};
\node (C) at (3,0) {$c_2$};
\node (A') at (0,-1) {$c_1''$};
\node (B') at (1.5,-1) {$R(x'')$};
\node (C') at (3,-1) {$c_2''$};
\node (D) at (4,0) {$x \in F(R(x))$};
\node (D') at (4,-1) {$x'' \in F(R(x''))$};
\path[->,font=\scriptsize,>=angle 90]
(A) edge node[above]{$R(i) \eta_{c_1}$} (B)
(C) edge node[above]{$R(o) \eta_{c_2}$} (B)
(A') edge node[above]{$R(i'') \eta_{c_1''}$} (B')
(C') edge node[above]{$R(o'') \eta_{c_2''}$} (B')
(A) edge node [left]{$f'f$} (A')
(B) edge node [left]{$R(\alpha' \alpha)$} (B')
(C) edge node [right]{$g'g$} (C');
\end{tikzpicture}
\]
together with a morphism $\tau_{M' M} \maps F(R(\alpha' \alpha))(x) \to x''$ in $F(R(x''))$. On the other hand, the individual images $\mathbb{E}_1(M)$ and $\mathbb{E}_1(M')$ are given by:
\[
\begin{tikzpicture}[scale=1.5]
\node (A) at (0,0) {$c_1$};
\node (B) at (1.5,0) {$R(x)$};
\node (C) at (3,0) {$c_2$};
\node (A') at (0,-1) {$c_1'$};
\node (B') at (1.5,-1) {$R(x')$};
\node (C') at (3,-1) {$c_2'$};
\node (D) at (4,0) {$x \in F(R(x))$};
\node (D') at (4,-1) {$x' \in F(R(x'))$};
\node (A'') at (0,-2) {$c_1'$};
\node (B'') at (1.5,-2) {$R(x')$};
\node (C'') at (3,-2) {$c_2'$};
\node (A''') at (0,-3) {$c_1''$};
\node (B''') at (1.5,-3) {$R(x'')$};
\node (C''') at (3,-3) {$c_2''$};
\node (D'') at (4,-2) {$x' \in F(R(x'))$};
\node (D''') at (4,-3) {$x'' \in F(R(x''))$};
\path[->,font=\scriptsize,>=angle 90]
(A) edge node[above]{$R(i) \eta_{c_1}$} (B)
(C) edge node[above]{$R(o) \eta_{c_2}$} (B)
(A') edge node[above]{$R(i') \eta_{c_1'}$} (B')
(C') edge node[above]{$R(o') \eta_{c_2'}$} (B')
(A) edge node [left]{$f$} (A')
(B) edge node [left]{$R(\alpha)$} (B')
(C) edge node [right]{$g$} (C')
(A'') edge node[above]{$R(i') \eta_{c_1'}$} (B'')
(C'') edge node[above]{$R(o') \eta_{c_2'}$} (B'')
(A''') edge node[above]{$R(i'') \eta_{c_1''}$} (B''')
(C''') edge node[above]{$R(o'') \eta_{c_2''}$} (B''')
(A'') edge node [left]{$f'$} (A''')
(B'') edge node [left]{$R(\alpha')$} (B''')
(C'') edge node [right]{$g'$} (C''');
\end{tikzpicture}
\]
together with morphisms $\tau_M \maps F(R(\alpha))(x) \to x'$ in $F(R(x'))$ and $\tau_{M'} \maps F(R(\alpha')(x') \to x''$ in $F(R(x''))$, respectively. The vertical composite $\mathbb{E}_1(M')\mathbb{E}_1(M)$ of the above two 2-morphisms is given by $\mathbb{E}_1(M'M)$ as $R$ is a functor and $\tau_{M'M}=\tau_{M'} \tau_M$. The functors $\mathbb{E}_0$ and $\mathbb{E}_1$ satisfy the equations $\mathbb{E}_0 S = S \mathbb{E}_1$ and $\mathbb{E}_0 T = T \mathbb{E}_1$.

To see that this functor is essentially surjective, given a horizontal 1-cell in $F\lCsp$:
\[
\begin{tikzpicture}[scale=1.5]
\node (A) at (0,0) {$c_1$};
\node (B) at (1,0) {$c$};
\node (C) at (2,0) {$c_2$};
\node (D) at (3,0) {$x \in F(c)$};
\path[->,font=\scriptsize,>=angle 90]
(A) edge node[above]{$i$} (B)
(C) edge node[above]{$o$} (B);
\end{tikzpicture}
\]
we can find a 2-isomorphism in $F\lCsp$ whose codomain is the above horizontal 1-cell and whose domain is the image of the following horizontal 1-cell in $_L \lCsp(\X)$:
\[
\begin{tikzpicture}[scale=1.5]
\node (A) at (0,0) {$L(c_1)$};
\node (B) at (1,0) {$x$};
\node (C) at (2,0) {$L(c_2)$};
\path[->,font=\scriptsize,>=angle 90]
(A) edge node[above]{$i'$} (B)
(C) edge node[above]{$o'$} (B);
\end{tikzpicture}
\]
with the 2-isomorphism in $F\lCsp$ given by:
\[
\begin{tikzpicture}[scale=1.5]
\node (A) at (0,0) {$c_1$};
\node (B) at (1,0) {$R(x)$};
\node (C) at (2,0) {$c_2$};
\node (A') at (0,-1) {$c_1$};
\node (B') at (1,-1) {$c$};
\node (C') at (2,-1) {$c_2$};
\node (D) at (3,0) {$x \in F(R(x))$};
\node (D') at (3,-1) {$x \in F(c)$};
\path[->,font=\scriptsize,>=angle 90]
(A) edge node[above]{$R(i') \eta_{c_1}$} (B)
(C) edge node[above]{$R(o') \eta_{c_2}$} (B)
(A') edge node[above]{$i$} (B')
(C') edge node[above]{$o$} (B')
(A) edge node [left]{$1$} (A')
(B) edge node [left]{${(R(e) \eta_c)}^{-1}$} (B')
(C) edge node [right]{$1$} (C');
\end{tikzpicture}
\]
$$\tau \maps F({(R(e)\eta_c)}^{-1})(x) \to x$$
where $e \maps L(c) \to x$ is given by the map from the trivial decoration on $c$ to $x \in F(c)$. %The object and arrow components $\mathbb{E}_0$ and $\mathbb{E}_1$ satisfy the equations $S \mathbb{E}_1 = \mathbb{E}_0 S$ and $T \mathbb{E}_1 = \mathbb{E}_0 T$.

To show that the double functor $\mathbb{E}$ is fully faithful, we need to show that the map  $$\mathbb{E}_1 \maps _f { _L \lCsp(\X)}_g(M,N) \to {_{\mathbb{E}(f)} {F\lCsp}_{\mathbb{E}(g)}}(\mathbb{E}(M),\mathbb{E}(N))$$ is bijective for arbitrary vertical 1-morphisms $f$ and $g$ and horizontal 1-cells $M$ and $N$ of $_L \lCsp(\X)$.  Consider a 2-morphism in $_L \lCsp(\X)$ with horizontal source and target $M$ and $N$, respectively and vertical source and target $f$ and $g$, respectively:
\[
\begin{tikzpicture}[scale=1.5]
\node (A) at (0,0) {$L(c_1)$};
\node (B) at (1,0) {$x$};
\node (C) at (2,0) {$L(c_2)$};
\node (A') at (0,-1) {$L(c_1')$};
\node (B') at (1,-1) {$x'$};
\node (C') at (2,-1) {$L(c_2')$};
\node (D) at (1,0.5) {$M$};
\node (E) at (-1,-0.5) {$f$};
\node (F) at (1,-1.5) {$N$};
\node (G) at (3,-0.5) {$g$};
\path[->,font=\scriptsize,>=angle 90]
(A) edge node[above]{$i$} (B)
(C) edge node[above]{$o$} (B)
(A') edge node[above]{$i'$} (B')
(C') edge node[above]{$o'$} (B')
(A) edge node [left]{$L(f)$} (A')
(B) edge node [left]{$\alpha$} (B')
(C) edge node [right]{$L(g)$} (C');
\end{tikzpicture}
\]
Thus the set $$_f { _L \lCsp(\X)}_g(M,N)$$ consists of triples $$(f,\alpha,g)$$ rendering the above diagram commutative where $f$ and $g$ are morphisms of $\A$ and $\alpha$ is a morphism of $\X$. The image of the above 2-morphism under the double functor $\mathbb{E}$ is given by:
\[
\begin{tikzpicture}[scale=1.5]
\node (A) at (0,0) {$c_1$};
\node (B) at (1,0) {$R(x)$};
\node (C) at (2,0) {$c_2$};
\node (A') at (0,-1) {$c_1'$};
\node (B') at (1,-1) {$R(x')$};
\node (C') at (2,-1) {$c_2'$};
\node (D) at (1,0.5) {$x \in F(R(x))$};
\node (D') at (1,-1.5) {$x' \in F(R(x'))$};
\node (E) at (1,1) {$\mathbb{E}(M)$};
\node (F) at (-1,-0.5) {$\mathbb{E}(f)$};
\node (G) at (1,-2) {$\mathbb{E}(N)$};
\node (H) at (3,-0.5) {$\mathbb{E}(g)$};
\path[->,font=\scriptsize,>=angle 90]
(A) edge node[above]{$R(i)\eta_{c_1}$} (B)
(C) edge node[above]{$R(o)\eta_{c_2}$} (B)
(A') edge node[above]{$R(i')\eta_{c_1'}$} (B')
(C') edge node[above]{$R(o')\eta_{c_2'}$} (B')
(A) edge node [left]{$f$} (A')
(B) edge node [left]{$R(\alpha)$} (B')
(C) edge node [right]{$g$} (C');
\end{tikzpicture}
\]
together with a morphism $\tau \maps F(R(\alpha))(x) \to x'$ of $F(R(x'))$.
Thus the set $$_{\mathbb{E}(f)} {F\lCsp}_{\mathbb{E}(g)}(\mathbb{E}(M),\mathbb{E}(N))$$ consists of 4-tuples $$(f,R(\alpha),g,\tau)$$ rendering the above diagram commutative and where $f,g$ and $R(\alpha)$ are morphisms of $\A$ and $\tau$ is a morphism in $F(R(x'))$. The morphisms $R(\alpha) \maps R(x) \to R(x')$ and $\tau \maps F(R(\alpha))(x) \to x'$ together determine the morphism $\alpha \maps x \to x'$ in $\mathrm{X}$ and conversely; given two objects $x=(c,x \in F(c))$ and $x'=(c',x' \in F(c'))$ of $\X=\inta{F}$, a morphism from $\alpha \maps x \to x'$ is a pair $$(h \maps c \to c', \tau \maps F(h)(x) \to x')$$ where $h \maps c \to c'$ is given by $R(\alpha) \maps R(x) \to R(x')$. This shows that $\mathbb{E}$ is fully faithful. \textbf{At least in my favorite example...}
\end{comment}
Functoriality of $\mathbb{E}_1$ can be verified, and moreover $(\mathbb{E}_0,\mathbb{E}_1)$ indeed define a strong double functor: there exist natural isomorphisms
\begin{gather*}
 \mathbb{E}(M) \odot \mathbb{E}(N) \xrightarrow{\sim} \mathbb{E}(M \odot N) \\
U_{\mathbb{E}m} \xrightarrow{\sim} \mathbb{E}(U_m)
\end{gather*}
for any pair of composable horizontal 1-cells $M=\cspn{(a,\bot_a)}{(m,x)}{(b,\bot_b)}{}{}$ and $N=\cspn{(b,\bot_b)}{(n,y)}{(c,\bot_c)}{}{}$ and any object $m$ of $\scsp{L_F}{\inta F}$. In more detail, $\mathbb{E}(M)\odot \mathbb{E}(N)$ is given as in \cref{eq:dcospanscomposition} via the pushout cospan and decoration
\begin{displaymath}
 \begin{tikzcd}
  & m+_bn & \\
  a\ar[ur,"j_m\circ i"] && b,\ar[ul,"j_n\circ o"']
 \end{tikzcd}
 \begin{tikzcd}
  1\ar[r,"x\times y"]\ar[drr,dashed] & F(m)\times F(n)\ar[r,"\phi_{m,n}"] & F(m+ n)\ar[d,"F(j)"] \\ 
  && F(m+_b n)
 \end{tikzcd}
\end{displaymath}
where $j_m, j_n$ denote the maps into a pushout.
On the other hand, first composing $M$ and $N$ in the Grothendieck category $\scsp{L_F}{\inta F}$ using fibrewise pushouts constructed using \cref{lem:fibrewiselimits} 
%!!!Chris: I could make this specific construction into a lemma earlier if you want, I have it in my notes
produces 
\begin{displaymath}
 \begin{tikzcd}
 && (m+_b n,F(j_m)x+_{\bot_{m+_bn}}F(j_n)y) && \\
 & (m,x) \ar[ur] && (n,y)\ar[ul] \\
 (a,\bot_a)\ar[ur] && (b,\bot_b)\ar[ur]\ar[ul] && (c,\bot_c)\ar[ul]
 \end{tikzcd}
\end{displaymath}
which is mapped, under $\mathbb{E}$, to precisely the cospan $\cspn{a}{m+_bn}{b}{}{}$ as above, and the decorations are the same 
%!!!Christina: isomorphic would be enough but they look equal to me!
due to the following commutative diagram
\begin{displaymath}
 \begin{tikzcd}
F(m)\times F(n)\ar[rr,"\phi"]\ar[d,"F(j_m)\times F(j_n)"'] && F(m+n)\ar[d,"F(j)"] \\
F(m+_bn)\times F(m+_bn)\ar[r,"\phi"] & F((m+_bn)+(m+_bn))\ar[r,"F\nabla"] & F(m+_bn)
 \end{tikzcd}
\end{displaymath}
since taking a pushout over an initial object is really a coproduct, and the fibrewise coproduct in $F(m+_bn)$ is given as in \cref{eq:explicitstructure1}.

Regarding the identity morphisms, $U_m$ is $\cspn{(m,\bot_m)}{(m,\bot_m)}{(m,\bot_m)}{}{}$ with $1_m$ as the $A$-component of the morphisms and isomorphisms between initial objects in the fibres. Thus $\mathbb{E}(U_m)$ is the identity cospan in $A$ together with the `initial decoration' $\bot_m\in F(m)$, whereas $U_{\mathbb{E}(m)}$ is exactly the same thing.

%!!!Christina: Below, up to my next comment, is Kenny's earlier proof on the same thing.
\begin{comment}
For any object $c$, the horizontal 1-cell $\hat{U}_{\mathbb{E}(c)}$ is given by $\hat{U}_c$ which is given by the pair:
\[
\begin{tikzpicture}[scale=1.5]
\node (A) at (0,0) {$c$};
\node (B) at (1,0) {$c$};
\node (C) at (2,0) {$c$};
\node (D) at (3,0) {$!_c \in F(c)$};
\path[->,font=\scriptsize,>=angle 90]
(A) edge node[above]{$1$} (B)
(C) edge node[above]{$1$} (B);
\end{tikzpicture}
\]
The horizontal 1-cell $U_c$ is given by
\[
\begin{tikzpicture}[scale=1.5]
\node (A) at (0,0) {$L(c)$};
\node (B) at (1,0) {$L(c)$};
\node (C) at (2,0) {$L(c)$};
%\node (D) at (3,0.5) {$!_c \in F(c)$};
\path[->,font=\scriptsize,>=angle 90]
(A) edge node[above]{$1$} (B)
(C) edge node[above]{$1$} (B);
\end{tikzpicture}
\]
and so $\mathbb{E}(U_c)$ is given by the pair:
\[
\begin{tikzpicture}[scale=1.5]
\node (A) at (0,0) {$c$};
\node (B) at (1,0) {$R(L(c))$};
\node (C) at (2,0) {$c$};
\node (D) at (3.25,0) {$!_c \in F(R(L(c)))$};
\path[->,font=\scriptsize,>=angle 90]
(A) edge node[above]{$\eta_c$} (B)
(C) edge node[above]{$\eta_c$} (B);
\end{tikzpicture}
\]
Then we can obtain the natural isomorphism $\mathbb{E}_c \maps \hat{U}_{\mathbb{E}(c)} \xrightarrow{\sim} \mathbb{E}(U_c)$ as the 2-morphism
\[
\begin{tikzpicture}[scale=1.5]
\node (A) at (0,0) {$c$};
\node (B) at (1,0) {$c$};
\node (C) at (2,0) {$c$};
\node (A') at (0,-1) {$c$};
\node (B') at (1,-1) {$R(L(c))$};
\node (C') at (2,-1) {$c$};
\node (D) at (3,0) {$!_c \in F(c)$};
\node (D') at (3.25,-1) {$!_{R(L(c))} \in F(R(L(c)))$};
\path[->,font=\scriptsize,>=angle 90]
(A) edge node[above]{$1$} (B)
(C) edge node[above]{$1$} (B)
(A') edge node[above]{$\eta_c$} (B')
(C') edge node[above]{$\eta_c$} (B')
(A) edge node [left]{$1$} (A')
(B) edge node [left]{$\eta_c$} (B')
(C) edge node [left]{$1$} (C');
\end{tikzpicture}
\]
$$\tau \maps F(\eta_c)(!_c) \xrightarrow{!} !_{R(L(c))}$$
of $F\lCsp$.

Next, given composable horizontal 1-cells $M$ and $N$ in $_L \lCsp(\X)$:
\[
\begin{tikzpicture}[scale=1.5]
\node (A) at (0,0) {$L(c_1)$};
\node (B) at (1,0) {$x$};
\node (C) at (2,0) {$L(c_2)$};
\node (D) at (3,0) {$L(c_2)$};
\node (E) at (4,0) {$x'$};
\node (F) at (5,0) {$L(c_3)$};
%\node (D) at (3,0.5) {$!_c \in F(c)$};
\path[->,font=\scriptsize,>=angle 90]
(A) edge node[above]{$i$} (B)
(C) edge node[above]{$o$} (B)
(D) edge node[above]{$i'$} (E)
(F) edge node[above]{$o'$} (E);
\end{tikzpicture}
\]
their images $\mathbb{E}(M)$ and $\mathbb{E}(N)$ are given by:
\[
\begin{tikzpicture}[scale=1.5]
\node (A) at (0,0) {$c_1$};
\node (B) at (1,0) {$R(x)$};
\node (C) at (2,0) {$c_2$};
\node (D) at (3,0) {$c_2$};
\node (E) at (4,0) {$R(x')$};
\node (F) at (5,0) {$c_3$};
\node (G) at (1,-0.5) {$x \in F(R(x))$};
\node (H) at (4,-0.5) {$x' \in F(R(x'))$};
%\node (D) at (3,0.5) {$!_c \in F(c)$};
\path[->,font=\scriptsize,>=angle 90]
(A) edge node[above]{$R(i) \eta_{c_1}$} (B)
(C) edge node[above]{$R(o) \eta_{c_2}$} (B)
(D) edge node[above]{$R(i') \eta_{c_2}$} (E)
(F) edge node[above]{$R(o') \eta_{c_3}$} (E);
\end{tikzpicture}
\]
and so $\mathbb{E}(M) \odot \mathbb{E}(N)$ is given by:
\[
\begin{tikzpicture}[scale=1.5]
\node (A) at (0,0) {$c_1$};
\node (B) at (1.5,0) {$R(x)+_{c_2}R(x')$};
\node (C) at (3,0) {$c_3$};
\node (G) at (1.5,-0.5) {$d_{\mathbb{E}(M) \odot \mathbb{E}(N)} = \hat{x} \in F(R(x)+_{c_2}R(x'))$};
%\node (D) at (3,0.5) {$!_c \in F(c)$};
\path[->,font=\scriptsize,>=angle 90]
(A) edge node[above]{$j \psi R(i) \eta_{c_1}$} (B)
(C) edge node[above]{$j \psi R(o') \eta_{c_3}$} (B);
\end{tikzpicture}
\]
$$\hat{x} \maps 1 \xrightarrow{\lambda^{-1}} 1 \times 1 \xrightarrow{x \times x'} F(R(x)) \times F(R(x')) \xrightarrow{\phi_{R(x),R(x')}} F(R(x)+R(x')) \xrightarrow{F(j_{R(x),R(x')})} F(R(x)+_{c_2}R(x'))$$where $\psi$ denotes each natural map into the coproduct and $j$ denotes the natural map from the coproduct to the pushout. On the other hand, $M \odot N$ is given by
\[
\begin{tikzpicture}[scale=1.5]
\node (A) at (0,0) {$L(c_1)$};
\node (B) at (1.25,0) {$x+_{L(c_2)}x'$};
\node (C) at (2.5,0) {$L(c_3)$};
%\node (G) at (1,-0.5) {$\hat{d} \in F(R(d)+_{c_2}R(d'))$};
%\node (D) at (3,0.5) {$!_c \in F(c)$};
\path[->,font=\scriptsize,>=angle 90]
(A) edge node[above]{$J \zeta i$} (B)
(C) edge node[above]{$J \zeta o'$} (B);
\end{tikzpicture}
\]
where $\zeta$ is a natural map into a coproduct and $J$ is the natural map from the coproduct to the pushout. Then $E(M \odot N)$ is given by
\[
\begin{tikzpicture}[scale=1.5]
\node (A) at (0,0) {$c_1$};
\node (B) at (1.5,0) {$R(x+_{L(c_2)}x')$};
\node (C) at (3,0) {$c_3$};
\node (G) at (1.5,-0.5) {$d_{\mathbb{E}(M \odot N)} = x+_{L(c_2)} x' \in F(R(x+_{L(c_2)}x'))$};
%\node (D) at (3,0.5) {$!_c \in F(c)$};
\path[->,font=\scriptsize,>=angle 90]
(A) edge node[above]{$R(J \zeta i) \eta_{c_1}$} (B)
(C) edge node[above]{$R(J \zeta o') \eta_{c_3}$} (B);
\end{tikzpicture}
\]
and so $\mathbb{E}_{M,N} \maps \mathbb{E}(M) \odot \mathbb{E}(N) \xrightarrow{\sim} \mathbb{E}(M \odot N)$ is given by the 2-morphism:
\[
\begin{tikzpicture}[scale=1.5]
\node (A) at (0,0) {$c_1$};
\node (B) at (1.5,0) {$R(x)+_{c_2}R(x')$};
\node (C) at (3,0) {$c_3$};
\node (A') at (0,-1) {$c_1$};
\node (B') at (1.5,-1) {$R(x+_{L(c_2)}x')$};
\node (C') at (3,-1) {$c_3$};
\node (D) at (5.5 ,0) {$d_{\mathbb{E}(M) \odot \mathbb{E}(N)} \in F(R(x)+_{c_2}R(x'))$};
\node (D') at (5.5,-1) {$d_{\mathbb{E}(M \odot N)} \in F(R(x+_{L(c_2)}x'))$};
\path[->,font=\scriptsize,>=angle 90]
(A) edge node[above]{$j \psi R(i) \eta_{c_1}$} (B)
(C) edge node[above]{$j \psi R(o') \eta_{c_3}$} (B)
(A') edge node[above]{$R(J \zeta i) \eta_{c_1}$} (B')
(C') edge node[above]{$R(J \zeta o') \eta_{c_3}$} (B')
(A) edge node [left]{$1$} (A')
(B) edge node [left]{$\sigma$} (B')
(C) edge node [left]{$1$} (C');
\end{tikzpicture}
\]
together with a morphism of decorations which is obtained as follows. First, as $R \maps \X \to \A$ preserves finite colimits, we have an isomorphism $$\kappa \maps R(x) +_{R(L(c_2))} R(x') \to R(x+_{L(c_2)}x').$$ Also, since the left adjoint $L \maps \A \to \X$ is fully faithful, the unit of the adjunction $L \dashv R$ at the object $c_2$ gives an isomorphism $\eta_{c_2} \maps c_2 \to R(L(c_2))$ which results in an isomorphism $$j_{\eta_{c_2}} \maps R(x) +_{c_2} R(x') \to R(x) +_{R(L(c_2))} R(x').$$ Composing these two results in an isomorphism $$\sigma \mapseqq \kappa j_{\eta_{c_2}} \maps R(x) +_{c_2} R(x') \to R(x+_{L(c_2)}x').$$
Next, to see that the above diagram commutes, it suffices to show that for the object $c_1 \in \mathrm{A}$, $$R(J)R(\zeta)R(i)\eta_{c_1}(c_1) = R(J \zeta i)\eta_{c_1}(c_1)  \stackrel{!}{=} \sigma j \psi R(i)\eta_{c_1}(c_1) = \kappa j_{\eta_{c_2}} \psi R(i) \eta_{c_1}(c_1).$$ This follows as $R(i) \eta_{c_1} \maps c_1 \to R(x)$ and the following diagram commutes:
\[
\begin{tikzpicture}[scale=1.5]
\node (B) at (0,0) {$R(x)$};
\node (C) at (2,0) {$R(x)+R(x')$};
\node (A') at (4,0) {$R(x)+_{c_2}R(x')$};
\node (B') at (4,-2) {$R(x+_{L(c_2)}x')$};
\node (D) at (0,-2) {$R(x+x')$};
\node (D') at (4,-1) {$R(x)+_{R(L(c_2))} R(x')$};
\path[->,font=\scriptsize,>=angle 90]
(C) edge node[above]{$j$} (A')
(B) edge node[above]{$\psi$} (C)
(D) edge node[above]{$R(J)$} (B')
(B) edge node [left]{$R(\zeta)$} (D)
(A') edge node [right]{$j_{\eta_{c_2}}$} (D')
(A') edge [out=345,in=15] node [right]{$\sigma$} (B')
(D') edge node [right]{$\kappa$} (B');
\end{tikzpicture}
\]
Lastly, this map of cospans comes with an isomorphism $$\tau \maps F(\sigma)(d_{\mathbb{E}(M) \odot \mathbb{E}(N)}) \to d_{\mathbb{E}(M \odot N)}$$ in $F(R(x+_{L(c_2)}x'))$. This shows that $\mathbb{E}$ is strong, and so 
\end{comment}
%!!!Christina: Up to here was Kenny's strong functoriality.

Based on the above description of its mapping, it follows that the strong double functor $\mathbb{E}=(\mathbb{E}_0,\mathbb{E}_1)$ is full, faithful and essentially surjective on objects as per \cref{def:fullfaithful,def:essentiallysurj}. Therefore $\mathbb{E} \maps \scsp{L_F}{\inta F} \to \dcsp{F}$ is a double equivalence, by \cref{ShulDubEquiv}.
%!!!Christina: In fact, Christina and Kenny believe it is an isomorphism of double categories! Bijective on all levels of objects and morphisms...should we say?

%!!!Christina Have not edited below, but everything should be fine up to notation changes. Happy to do if needed in the future.

Next, we show that if both double categories $_L \lCsp(\X)$ and $F\lCsp$ are symmetric monoidal, as they are if both $\A$ and $\X$ have finite colimits, then this equivalence of double categories $\mathbb{E} \maps _L \lCsp(\X) \to F\lCsp$ will be symmetric monoidal. First, note that we have an isomorphism $\epsilon \maps 1_{F\lCsp} \to \mathbb{E}(1_{_L \lCsp(\X)})$ and natural isomorphisms $\mu_{c_1,c_2} \maps \mathbb{E}(c_1) \otimes \mathbb{E}(c_2) \to \mathbb{E}(c_1 \otimes c_2)$ for every pair of objects $c_1,c_2 \in {_L \mathbb{C} \textnormal{sp}(\X)}$ both of which are given by identities since both double categories $_L \lCsp(\X)$ and $F\lCsp$ have $\A$ as their category of objects and $\mathbb{E}_0=\id_{\A}$. The diagrams utilizing these maps that are required to commute do so trivially.

For the arrow component $\mathbb{E}_1$, we have an isomorphism $\delta \maps U_{1_{F\lCsp}} \to \mathbb{E}(U_{1_{_L \lCsp(\X)}})$ where the horizontal 1-cell $U_{1_{F\lCsp}}$ is given by:
\[
\begin{tikzpicture}[scale=1.5]
\node (A) at (0,0) {$1_\A$};
\node (B) at (1,0) {$1_\A$};
\node (C) at (2,0) {$1_\A$};
\node (D) at (3,0) {$!_{1_{\A}} \in F(1_\A)$};
\path[->,font=\scriptsize,>=angle 90]
(A) edge node[above]{$1$} (B)
(C) edge node[above]{$1$} (B);
\end{tikzpicture}
\]
where $!_{1_{\A}} = \phi \maps 1 \to F(1_\A)$ is the trivial decoration which comes from the structure of the symmetric lax monoidal pseudofunctor $F \maps \A \to \Cat$. The horizontal 1-cell $U_{1_{_L \lCsp(\X)}}$ is given by:
\[
\begin{tikzpicture}[scale=1.5]
\node (A) at (0,0) {$L(1_\A)$};
\node (B) at (1,0) {$L(1_\A)$};
\node (C) at (2,0) {$L(1_\A)$};
%\node (D) at (3,0.5) {$I \in F(1_\mathrm{A})$};
\path[->,font=\scriptsize,>=angle 90]
(A) edge node[above]{$1$} (B)
(C) edge node[above]{$1$} (B);
\end{tikzpicture}
\]
where here we make use of the fact that the left adjoint $L \maps (\A,+,1_\A) \to (\X,+,1_\X)$ preserves all colimits and thus $L(1_\A) \cong 1_\X$. The horizontal 1-cell $\mathbb{E}(U_{1_{_L \mathbb{C} \textnormal{sp}(\X)}})$ is then given by the pair:
\[
\begin{tikzpicture}[scale=1.5]
\node (A) at (0,0) {$1_\A$};
\node (B) at (1.25,0) {$R(L(1_\A))$};
\node (C) at (2.5,0) {$1_\A$};
\node (D) at (4.5,0) {$!_{R(L(1_\A))} \in F(R(L(1_\A)))$};
\path[->,font=\scriptsize,>=angle 90]
(A) edge node[above]{${\eta_{1_\A}}$} (B)
(C) edge node[above]{${\eta_{1_\A}}$} (B);
\end{tikzpicture}
\]
The isomorphism $\delta$ is then given by the 2-morphism:
\[
\begin{tikzpicture}[scale=1.5]
\node (A) at (0,0) {$1_\A$};
\node (B) at (1,0) {$1_{\A}$};
\node (C) at (2,0) {$1_\A$};
\node (A') at (0,-1) {$1_\A$};
\node (B') at (1,-1) {$R(L(1_\A))$};
\node (C') at (2,-1) {$1_\A$};
\node (D) at (3,0) {$!_{1_\A} \in F(1_\A)$};
\node (D') at (3.75,-1) {$!_{R(L(1_\A))} \in F(R(L(1_\A)))$};
\path[->,font=\scriptsize,>=angle 90]
(A) edge node[above]{$1$} (B)
(C) edge node[above]{$1$} (B)
(A') edge node[above]{${\eta_{1_\A}}$} (B')
(C') edge node[above]{${\eta_{1_\A}}$} (B')
(A) edge node [left]{$1$} (A')
(B) edge node [left]{${\eta_{1_\A}}$} (B')
(C) edge node [left]{$1$} (C');
\end{tikzpicture}
\]
$$\tau_{{\eta_{1_\A}}} \maps F({\eta_{1_\A}})(!_{1_\A})) \to !_{R(L(1_\A))}$$
of $F\lCsp$. This is just the natural isomorphism $\mathbb{E}_{1_\A}$ from earlier.

Given two horizontal 1-cells $M$ and $N$ of $_L \lCsp(\X)$:
\[
\begin{tikzpicture}[scale=1.5]
\node (A) at (0,0) {$L(c_1)$};
\node (B) at (1,0) {$x$};
\node (C) at (2,0) {$L(c_2)$};
\node (D) at (3,0) {$L(c_1')$};
\node (E) at (4,0) {$x'$};
\node (F) at (5,0) {$L(c_2')$};
%\node (D) at (3,0.5) {$!_c \in F(c)$};
\path[->,font=\scriptsize,>=angle 90]
(A) edge node[above]{$i$} (B)
(C) edge node[above]{$o$} (B)
(D) edge node[above]{$i'$} (E)
(F) edge node[above]{$o'$} (E);
\end{tikzpicture}
\]
their images $\mathbb{E}(M)$ and $\mathbb{E}(N)$ are given by:
\[
\begin{tikzpicture}[scale=1.5]
\node (A) at (0,0) {$c_1$};
\node (B) at (1,0) {$R(x)$};
\node (C) at (2,0) {$c_2$};
\node (D) at (3,0) {$c_1'$};
\node (E) at (4,0) {$R(x')$};
\node (F) at (5,0) {$c_2'$};
\node (G) at (1,-0.5) {$x \in F(R(x))$};
\node (H) at (4,-0.5) {$x' \in F(R(x'))$};
%\node (D) at (3,0.5) {$!_c \in F(c)$};
\path[->,font=\scriptsize,>=angle 90]
(A) edge node[above]{$R(i) \eta_{c_1}$} (B)
(C) edge node[above]{$R(o) \eta_{c_2}$} (B)
(D) edge node[above]{$R(i') \eta_{c_1'}$} (E)
(F) edge node[above]{$R(o') \eta_{c_2'}$} (E);
\end{tikzpicture}
\]
and so $\mathbb{E}(M) \otimes \mathbb{E}(N)$ is given by:
\[
\begin{tikzpicture}[scale=1.5]
\node (A) at (0,0) {$c_1+c_1'$};
\node (B) at (3.25,0) {$R(x)+R(x')$};
\node (C) at (6.5,0) {$c_2+c_2'$};
\node (D) at (3.25,-0.5) {$d_{\mathbb{E}(M) \otimes \mathbb{E}(N)} \in F(R(x)+R(x'))$}; 
%\node (D) at (3,0.5) {$!_c \in F(c)$};
\path[->,font=\scriptsize,>=angle 90]
(A) edge node[above]{$R(i) \eta_{c_1} +R(i') \eta_{c_1'}$} (B)
(C) edge node[above]{$R(o) \eta_{c_2} + R(o) \eta_{c_2'}$} (B);
\end{tikzpicture}
\]
where $$d_{\mathbb{E}(M) \otimes \mathbb{E}(N)} \maps 1 \xrightarrow{\lambda^{-1}} 1 \times 1 \xrightarrow{x \times x'} F(R(x)) \times F(R(x')) \xrightarrow{\phi_{R(x),R(x')}} F(R(x)+R(x')).$$ On the other hand, $M \otimes N$ is given by
\[
\begin{tikzpicture}[scale=1.5]
\node (A) at (0,0) {$L(c_1+c_1')$};
\node (B) at (2,0) {$x+x'$};
\node (C) at (4,0) {$L(c_2+c_2')$};
%\node (D) at (3,0.5) {$!_c \in F(c)$};
\path[->,font=\scriptsize,>=angle 90]
(A) edge node[above]{$(i+i')\phi_{c_1,c_1'}^{-1}$} (B)
(C) edge node[above]{$(o+o')\phi_{c_2,c_2'}^{-1}$} (B);
\end{tikzpicture}
\]
and $\mathbb{E}(M \otimes N)$ is given by:
\[
\begin{tikzpicture}[scale=1.5]
\node (A) at (0,0) {$c_1+c_1'$};
\node (B) at (4,0) {$R(x+x')$};
\node (C) at (8,0) {$c_2+c_2'$};
\node (D) at (4,-0.5) {$d_{\mathbb{E}(M \odot N)} \in F(R(x+x')).$};
%\node (D) at (3,0.5) {$!_c \in F(c)$};
\path[->,font=\scriptsize,>=angle 90]
(A) edge node[above]{$R((i+i')\phi_{c_1,c_1'}^{-1}) \eta_{c_1+c_1'}$} (B)
(C) edge node[above]{$R((o+o')\phi_{c_2,c_2'}^{-1}) \eta_{c_2+c_2'}$} (B);
\end{tikzpicture}
\]
We then have a 2-isomorphism $\mu_{M,N} \maps E(M) \otimes E(N) \xrightarrow{\sim} E(M \otimes N)$ in $F\lCsp$ given by:
\[
\begin{tikzpicture}[scale=1.5]
\node (A) at (-2.5,0) {$c_1+c_1'$};
\node (B) at (1,0) {$R(x)+R(x')$};
\node (C) at (4.5,0) {$c_2+c_2'$};
\node (A') at (-2.5,-1) {$c_1+c_1'$};
\node (B') at (1,-1) {$R(x+x')$};
\node (C') at (4.5,-1) {$c_2+c_2'$};
\node (D) at (6.55,0) {$d_{\mathbb{E}(M) \otimes \mathbb{E}(N)} \in F(R(x)+R(x'))$};
\node (D') at (6.55,-1) {$d_{\mathbb{E}(M \otimes N)} \in F(R(x+x'))$};
\node (E) at (2,-1.5) {$\tau_\mu \maps F(\kappa)(d_{\mathbb{E}(M) \odot \mathbb{E}(N)}) \to d_{\mathbb{E}(M \odot N)}$};
\path[->,font=\scriptsize,>=angle 90]
(A) edge node[above]{$R(i)\eta_{c_1} + R(i')\eta_{c_1'}$} (B)
(C) edge node[above]{$R(o)\eta_{c_2} + R(o')\eta_{c_2'}$} (B)
(A') edge node[above]{$R((i+i')\phi_{c_1,c_1'}^{-1}) \eta_{c_1+c_1'}$} (B')
(C') edge node[above]{$R((o+o')\phi_{c_2,c_2'}^{-1}) \eta_{c_2+c_2'}$} (B')
(A) edge node [left]{$1$} (A')
(B) edge node [left]{$\kappa$} (B')
(C) edge node [right]{$1$} (C');
\end{tikzpicture}
\]
where $\kappa$ is the isomorphism which comes from $R \maps \X \to \A$ preserving finite colimits.

The isomorphisms $\delta$ and $\mu$ satisfy the left and right unitality squares, associativity hexagon and braiding square. 
%Kenny: I think this might be a good place to cut the monoidal stuff.
%Christina: Deal! They are right below.

\begin{comment}
To see this, let $M_1,M_2$ and $M_3$ be horizontal 1-cells in $_L \lCsp(\X)$ given by:
\[
\begin{tikzpicture}[scale=1.5]
\node (A) at (0,0) {$L(c_1)$};
\node (B) at (1,0) {$x_1$};
\node (C) at (2,0) {$L(c_1')$};
\node (D) at (3,0) {$L(c_2)$};
\node (E) at (4,0) {$x_2$};
\node (F) at (5,0) {$L(c_2')$};
\node (G) at (6,0) {$L(c_3)$};
\node (H) at (7,0) {$x_3$};
\node (I) at (8,0) {$L(c_3')$};
%\node (D) at (3,0.5) {$!_c \in F(c)$};
\path[->,font=\scriptsize,>=angle 90]
(A) edge node[above]{$i_1$} (B)
(C) edge node[above]{$o_1$} (B)
(D) edge node[above]{$i_2$} (E)
(F) edge node[above]{$o_2$} (E)
(G) edge node[above]{$i_3$} (H)
(I) edge node[above]{$o_3$} (H);
\end{tikzpicture}
\]
The left unitality square:
\[
\begin{tikzpicture}[scale=1.5]
\node (A) at (0,0) {$1_{F\lCsp} \otimes \mathbb{E}(M_1)$};
\node (B) at (3,0) {$\mathbb{E}(1_{ _L \lCsp(\X)}) \otimes \mathbb{E}(M_1)$};
\node (C) at (0,-1) {$\mathbb{E}(M_1)$};
\node (D) at (3,-1) {$\mathbb{E}(1_{ _L \lCsp(\X)} \otimes M_1)$};
\path[->,font=\scriptsize,>=angle 90]
(A) edge node[above]{$\delta \otimes 1$} (B)
(B) edge node[right]{$\mu_{1,M_1}$} (D)
(A) edge node[left]{$\lambda'$} (C)
(D) edge node[above]{$\mathbb{E}(\lambda)$} (C);
\end{tikzpicture}
\]
has an underlying diagram of maps of cospans given by:
\[
		\begin{tikzpicture}
			\node (d) at (7.5,0) {$\mathbb{E}(1_{ _L \lCsp(\X)}) \otimes \mathbb{E}(M_1)$};
			\node (a) at (-6,0) {$1_\A+c_1$};
			\node (b) at (-0.5,0) {$R(L(1_\A)) + R(x_1)$};
			\node (c) at (5,0) {$1_\A+c_1'$};
			\node (d2) at (7.5,1) {$1_{F\lCsp} \otimes \mathbb{E}(M_1)$};
			\node (a2) at (-6,1) {$1_\A+c_1$};
			\node (b2) at (-0.5,1) {$1_\A+R(x_1)$};
			\node (c2) at (5,1) {$1_\A+c_1'$};
			\node (d3) at (7.5,2) {$\mathbb{E}(M_1)$};
                                \node (a3) at (-6,2) {$c_1$};
			\node (b3) at (-0.5,2) {$R(x_1)$};
			\node (c3) at (5,2) {$c_1'$};
			\node (d4) at (7.5,-1) {$\mathbb{E}(1_{ _L \lCsp(\X)} \otimes M_1)$};
                                \node (a5) at (-6,-1) {$1_\A+c_1'$};
			\node (b5) at (-0.5,-1) {$R(L(1_\A)+x_1)$};
			\node (c5) at (5,-1) {$1_\A+c_1'$};
			\node (d5) at (7.5,-2) {$\mathbb{E}(M_1)$};
                                \node (a6) at (-6,-2) {$c_1$};
			\node (b6) at (-0.5,-2) {$R(x_1)$};
			\node (c6) at (5,-2) {$c_1'$};
			\path[->,font=\scriptsize,>=angle 90]
			(d2) edge node [left]{$\lambda$} (d3)
			(d2) edge node [left] {$\delta \otimes 1$} (d)
			(d) edge node [left] {$\mu_{1,M_1}$} (d4)
			(d4) edge node [left] {$\mathbb{E}(\lambda)$} (d5)
			(a) edge node[above]{$\eta_{1_\A}+R(i_1)\eta_{c_1}$} (b)
			(c) edge node[above]{$\eta_{1_\A}+R(o_1)\eta_{c_1'}$} (b)
                                (a2) edge node[above]{$1+R(i_1)\eta_{c_1}$} (b2)
			(c2) edge node[above]{$1+R(o_1)\eta_{c_1'}$} (b2)
                                (a2) edge node[left]{$1$} (a)
                                (b2) edge node[left]{$\eta_{1_\A}+1$} (b)
(b2) edge node[right]{$\tau_2$} (b)
			(c2) edge node[right]{$1$} (c)
                                (a3) edge node[above]{$R(i_1)\eta_{c_1}$} (b3)
			(c3) edge node[above]{$R(o_1)\eta_{c_1'}$} (b3)
                                (a2) edge node[left]{$\lambda_{\A}$} (a3)
                                (b2) edge node[left]{$\lambda_{\A}$} (b3)
(b2) edge node[right]{$\tau_1$} (b3)
			(c2) edge node[right]{$\lambda_{\A}$} (c3)
                                (a5) edge node[above]{$(\mu_{L(1_\A),d_1})(\eta_{1_\A}+R(i_1)\eta_{c_1})$} (b5)
			(c5) edge node[above]{$(\mu_{L(1_\A),d_1})(\eta_{1_\A}+R(o_1)\eta_{c_1'})$} (b5)
                                (a) edge node[left]{$1$} (a5)
                                (b) edge node[left]{$\mu_{L(1_\A),x_1}$} (b5)
(b) edge node[right]{$\tau_3$} (b5)
			(c) edge node[right]{$1$} (c5)
                                (a6) edge node[above]{$R(i_1)\eta_{c_1}$} (b6)
			(c6) edge node[above]{$R(o_1)\eta_{c_1'}$} (b6)
                                (a5) edge node[left]{$\lambda_{\A}$} (a6)
                                (b5) edge node[left]{$R(\lambda_{\X})$} (b6)
 (b5) edge node[right]{$\tau_4$} (b6)
			(c5) edge node[right]{$\lambda_{\A}$} (c6);
		\end{tikzpicture}
	\]
with the corresponding maps of decorations amounting to the following commutative diagram in $F(R(x_1))$:
\[
\begin{tikzpicture}[scale=1.5]
\node (A) at (-.5,0) {$F(\lambda_{\A})(!_{1_\A}+x_1)$};
\node (C) at (4,0) {$F(R(\lambda_{\X})(\mu_{L(1_\A),x_1}))(!_{R(L(1_\A))} +x_1$)};
\node (D) at (-.5,-1) {$x_1$};
\node (E) at (4,-1) {$F(R(\lambda_\X))(x_{!+1})$};
\path[->,font=\scriptsize,>=angle 90]
(A) edge node[above]{$F(R(\lambda_{\X})(\mu_{L(1_\A),x_1}))(\tau_2)$} (C)
(A) edge node[left]{$\tau_1$} (D)
(E) edge node[above]{$\tau_4$} (D)
(C) edge node[right]{$F(R(\lambda_\X))(\tau_3)$} (E);
\end{tikzpicture}
\]
where $x_{!+1}$ is the decoration $x_1$ on the object $R(L(1_\A)+x_1) \in \A$. The above square commutes because $$F(\lambda_\A)(!_{1_\A}+x_1) = F(R(\lambda_\X)(\mu_{L(1_\A),x_1})(\eta_{1_\A}+1))(!_{1_\A}+x_1)$$ as the corresponding left unitality square for the finite colimit preserving functor $R \maps (\X,1_\X,+) \to (\A,1_\A,+)$ commutes. The right unitality square is similar. The associator hexagon:
\[
\begin{tikzpicture}[scale=1.5]
\node (A) at (0,0) {$(\mathbb{E}(M_1) \otimes \mathbb{E}(M_2)) \otimes \mathbb{E}(M_3)$};
\node (B) at (3.5,0) {$\mathbb{E}(M_1 \otimes M_2) \otimes \mathbb{E}(M_3)$};
\node (C) at (7,0) {$\mathbb{E}((M_1 \otimes M_2) \otimes M_3)$};
\node (A') at (0,-1) {$\mathbb{E}(M_1) \otimes (\mathbb{E}(M_2) \otimes \mathbb{E}(M_3))$};
\node (B') at (3.5,-1) {$\mathbb{E}(M_1) \otimes \mathbb{E}(M_2 \otimes M_3)$};
\node (C') at (7,-1) {$\mathbb{E}(M_1 \otimes (M_2 \otimes M_3))$};
\path[->,font=\scriptsize,>=angle 90]
(A) edge node[above]{$\mu_{M_1,M_2} \otimes 1$} (B)
(B) edge node[above]{$\mu_{M_1 \otimes M_2,M_3}$} (C)
(A') edge node[above]{$1 \otimes \mu_{M_2,M_3}$} (B')
(B') edge node[above]{$\mu_{M_1,M_2 \otimes M_3}$} (C')
(A) edge node [left]{$a'$} (A')
(C) edge node [right]{$\mathbb{E}(a)$} (C');
\end{tikzpicture}
\]
has underlying maps of cospans given by:
\[
		\begin{tikzpicture}
\node(d) at (1.5,7) {$(\mathbb{E}(M_1) \otimes \mathbb{E}(M_2)) \otimes \mathbb{E}(M_3)$};
\node (d2) at (1.5,8) {$\mathbb{E}(M_1 \otimes M_2) \otimes \mathbb{E}(M_3)$};
\node (d3) at (1.5,9) {$\mathbb{E}((M_1 \otimes M_2) \otimes M_3)$};
\node (d4) at (1.5,10) {$\mathbb{E}(M_1 \otimes (M_2 \otimes M_3))$};
\node (d5) at (1.5,6) {$\mathbb{E}(M_1) \otimes (\mathbb{E}(M_2) \otimes \mathbb{E}(M_3))$};
\node (d6) at (1.5,5) {$\mathbb{E}(M_1) \otimes \mathbb{E}(M_2 \otimes M_3)$};
\node (d7) at (1.5,4) {$\mathbb{E}(M_1 \otimes (M_2 \otimes M_3))$};
			\node (a) at (-4.25,0) {$(c_1+c_2)+c_3$};
			\node (b) at (2.5,0) {$(R(x_1)+R(x_2))+R(x_3)$};
			\node (c) at (9.25,0) {$(c_1'+c_2')+c_3'$};
			\node (a2) at (-4.25,1) {$(c_1+c_2)+c_3$};
			\node (b2) at (2.5,1) {$R(x_1+x_2)+R(x_3)$};
			\node (c2) at (9.25,1) {$(c_1'+c_2')+c_3'$};
                                \node (a3) at (-4.25,2) {$(c_1+c_2)+c_3$};
			\node (b3) at (2.5,2) {$R((x_1+x_2)+x_3)$};
			\node (c3) at (9.25,2) {$(c_1'+c_2')+c_3'$};
                                \node (a4) at (-4.25,3) {$c_1+(c_2+c_3)$};
			\node (b4) at (2.5,3) {$R(x_1+(x_2+x_3))$};
			\node (c4) at (9.25,3) {$c_1' + (c_2' + c_3')$};
                                \node (a5) at (-4.25,-1) {$c_1+(c_2+c_3)$};
			\node (b5) at (2.5,-1) {$R(x_1)+(R(x_2)+R(x_3))$};
			\node (c5) at (9.25,-1) {$c_1' + (c_2' + c_3')$};
                                \node (a6) at (-4.25,-2) {$c_1+(c_2+c_3)$};
			\node (b6) at (2.5,-2) {$R(x_1)+R(x_2+x_3)$};
			\node (c6) at (9.25,-2) {$c_1' + (c_2' + c_3')$};
                                \node (a7) at (-4.25,-3) {$c_1+(c_2+c_3)$};
			\node (b7) at (2.5,-3) {$R(x_1+(x_2+x_3))$};
			\node (c7) at (9.25,-3) {$c_1' + (c_2' + c_3')$};
			\path[->,font=\scriptsize,>=angle 90]
(d) edge node[left] {$\mu_{M_1,M_2} \otimes 1$} (d2)
(d2) edge node[left] {$\mu_{M_1 \otimes M_2,M_3}$} (d3)
(d3) edge node[left] {$\mathbb{E}(a)$} (d4)
(d) edge node[left] {$a'$} (d5)
(d5) edge node[left] {$1 \otimes \mu_{M_2,M_3}$} (d6)
(d6) edge node[left] {$\mu_{M_1,M_2 \otimes M_3}$} (d7)
			(a) edge node[above]{$(R(i_1)\eta_{c_1}+R(i_2)\eta_{c_2})+R(i_3)\eta_{c_3}$} (b)
			(c) edge node[above]{$(R(o_1)\eta_{c_1'}+R(o_2)\eta_{c_2'})+R(o_3)\eta_{c_3'}$} (b)
                                (a2) edge node[above]{$R(i_1+i_2)\eta_{c_1+c_2}+R(i_3)\eta_{c_3}$} (b2)
			(c2) edge node[above]{$R(o_1+o_2)\eta_{c_1'+c_2'}+R(o_3)\eta_{c_3'}$} (b2)
                                (a) edge node[left]{$1$} (a2)
                                (b) edge node[left]{$\kappa + 1$} (b2)
(b) edge node[right]{$\tau_1$} (b2)
			(c) edge node[right]{$1$} (c2)
                                (a3) edge node[above]{$R((i_1+i_2)+i_3)\eta_{(c_1+c_2)+c_3}$} (b3)
			(c3) edge node[above]{$R((o_1+o_2)+o_3)\eta_{(c_1'+c_2')+c_3'}$} (b3)
                                (a2) edge node[left]{$1$} (a3)
                                (b2) edge node[left]{$\kappa$} (b3)
(b2) edge node[right]{$\tau_2$} (b3)
			(c2) edge node[right]{$1$} (c3)
                                (a4) edge node[above]{$R(i_1+(i_2+i_3))\eta_{c_1+(c_2+c_3)}$} (b4)
			(c4) edge node[above]{$R(o_1+(o_2+o_3))\eta_{c_1'+(c_2'+c_3')}$} (b4)
                                (a3) edge node[left]{$a_\A$} (a4)
                                (b3) edge node[left]{$R(a_\X)$} (b4)
(b3) edge node[right]{$\tau_3$} (b4)
			(c3) edge node[right]{$a_\A$} (c4)
                                (a5) edge node[above]{$R(i_1)\eta_{c_1}+(R(i_2)\eta_{c_2}+R(i_3)\eta_{c_3})$} (b5)
			(c5) edge node[above]{$R(o_1)\eta_{c_1'}+(R(o_2)\eta_{c_2'}+R(o_3)\eta_{c_3'})$} (b5)
                                (a) edge node[left]{$a_\A$} (a5)
                                (b) edge node[left]{$a_\A$} (b5)
(b) edge node[right]{$\tau_4$} (b5)
			(c) edge node[right]{$a_\A$} (c5)
                                (a6) edge node[above]{$R(i_1)\eta_{c_1}+R(i_2+i_3)\eta_{c_2+c_3}$} (b6)
			(c6) edge node[above]{$R(o_1)\eta_{c_1'}+R(o_2+o_3)\eta_{c_2'+c_3'}$} (b6)
                                (a5) edge node[left]{$1$} (a6)
                                (b5) edge node[left]{$1+\kappa$} (b6)
 (b5) edge node[right]{$\tau_5$} (b6)
			(c5) edge node[right]{$1$} (c6)
                                (a7) edge node[above]{$R(i_1+(i_2+i_3))\eta_{c_1+(c_2+c_3)}$} (b7)
			(c7) edge node[above]{$R(o_1+(o_2+o_3))\eta_{c_1'+(c_2'+c_3')}$} (b7)
                                (a6) edge node[left]{$1$} (a7)
                                (b6) edge node[left]{$\kappa$} (b7)
(b6) edge node[right]{$\tau_6$} (b7)
			(c6) edge node[right]{$1$} (c7);
		\end{tikzpicture}
	\]
Here we have omitted the natural isomorphisms $\phi_{c_i,c_j} \maps L(c_i) + L(c_j) \to L(c_i + c_j)$ on the inward pointing morphisms which make up the legs of each cospan due to limited space. The corresponding maps of decorations amount to the following commutative diagram in $F(R(x_1+(x_2+x_3)))$:
\[
\begin{tikzpicture}[scale=1.5]
\node (A) at (0,0) {$F((\kappa)(1+\kappa)(a_\A))((x_1+x_2)+x_3)$};
\node (B) at (5,0) {$F((R(a_\mathrm{X}))(\kappa))((x_1+x_2)+x_3)$};
\node (C) at (5,-1) {$F(R(a_\mathrm{X}))((x_1+x_2)+x_3)$};
\node (D) at (5,-2) {$x_1+(x_2+x_3)$};
\node (E) at (0,-1) {$F((\kappa)(1+\kappa))((x_1+x_2)+x_3)$};
\node (F) at (0,-2) {$F(\kappa)(x_1+(x_2+x_3))$};
\path[->,font=\scriptsize,>=angle 90]
(A) edge node[above]{$F((R(a_\X))(\kappa))(\tau_1)$} (B)
(B) edge node[right]{$F(R(a_\X))(\tau_2)$} (C)
(C) edge node[right]{$\tau_3$} (D)
(A) edge node[left]{$F((\kappa)(1+\kappa))(\tau_4)$} (E)
(E) edge node[left]{$F(\kappa)(\tau_5)$} (F)
(F) edge node[above]{$\tau_6$} (D);
\end{tikzpicture}
\]
The above square commutes because $$F((\kappa)(1+\kappa)(a_\A))((x_1+x_2)+x_3) = F((R(a_\X))(\kappa)(\kappa+1))((x_1+x_2)+x_3)$$ as the corresponding associator hexagon for the finite colimit preserving functor $R \maps (\X,1_\X,+) \to (\A,1_\A,+)$ commutes. Lastly, the braiding square:
\[
\begin{tikzpicture}[scale=1.5]
\node (A) at (0,0) {$\mathbb{E}(M_1) \otimes \mathbb{E}(M_2)$};
\node (B) at (3,0) {$\mathbb{E}(M_2) \otimes \mathbb{E}(M_1)$};
\node (C) at (0,-1) {$\mathbb{E}(M_1 \otimes M_2)$};
\node (D) at (3,-1) {$\mathbb{E}(M_2 \otimes M_1)$};
\path[->,font=\scriptsize,>=angle 90]
(A) edge node[above]{$\beta'$} (B)
(B) edge node[right]{$\mu_{M_2,M_1}$} (D)
(A) edge node[left]{$\mu_{M_1,M_2}$} (C)
(C) edge node[above]{$\mathbb{E}(\beta)$} (D);
\end{tikzpicture}
\]
has underlying map of cospans given by:
\[
		\begin{tikzpicture}
\node (d) at (7,0) {$\mathbb{E}(M_1) \otimes \mathbb{E}(M_2)$};
\node (d2) at (7,1) {$\mathbb{E}(M_2) \otimes \mathbb{E}(M_1)$};
\node (d3) at (7,2) {$\mathbb{E}(M_2 \otimes M_1)$};
\node (d4) at (7,-1) {$\mathbb{E}(M_1 \otimes M_2)$};
\node (d5) at (7,-2) {$\mathbb{E}(M_2 \otimes M_1)$};
			\node (a) at (-4,0) {$c_1+c_2$};
			\node (b) at (0.5,0) {$R(x_1)+R(x_2)$};
			\node (c) at (5,0) {$c_1'+c_2'$};
			\node (a2) at (-4,1) {$c_2+c_1$};
			\node (b2) at (0.5,1) {$R(x_2)+R(x_1)$};
			\node (c2) at (5,1) {$c_2'+c_1'$};
                                \node (a3) at (-4,2) {$c_2+c_1$};
			\node (b3) at (0.5,2) {$R(x_2+x_1)$};
			\node (c3) at (5,2) {$c_2' + c_1'$};
                                \node (a5) at (-4,-1) {$c_1+c_2$};
			\node (b5) at (0.5,-1) {$R(x_1+x_2)$};
			\node (c5) at (5,-1) {$c_1'+c_2'$};
                                \node (a6) at (-4,-2) {$c_2+c_1$};
			\node (b6) at (0.5,-2) {$R(x_2+x_1)$};
			\node (c6) at (5,-2) {$c_2' + c_1'$};
			\path[->,font=\scriptsize,>=angle 90]
(d) edge node[left]{$\beta'$} (d2)
(d2) edge node[left]{$\mu_{M_2,M_1}$}(d3)
(d) edge node[left] {$\mu_{M_1,M_2}$}(d4)
(d4)edge node[left]{$\mathbb{E}(\beta)$}(d5)
			(a) edge node[above]{$R(i_1) \eta_{c_1} + R(i_2)\eta_{c_2}$} (b)
			(c) edge node[above]{$R(o_1) \eta_{c_1'} + R(o_2) \eta_{c_2'}$} (b)
                                (a2) edge node[above]{$R(i_2) \eta_{c_2} + R(i_1) \eta_{c_1}$} (b2)
			(c2) edge node[above]{$R(o_2) \eta_{c_2'} + R(o_1) \eta_{c_1'}$} (b2)
                                (a) edge node[left]{$\beta_\A$} (a2)
                                (b) edge node[left]{$\beta_\A$} (b2)
(b) edge node[right]{$\tau_1$} (b2)
			(c) edge node[right]{$\beta_\A$} (c2)
                                (a3) edge node[above]{$R(i_2+i_1)\eta_{c_2+c_1}$} (b3)
			(c3) edge node[above]{$R(o_2+o_1)\eta_{c_2'+c_1'}$} (b3)
                                (a2) edge node[left]{$1$} (a3)
                                (b2) edge node[left]{$\kappa$} (b3)
(b2) edge node[right]{$\tau_2$} (b3)
			(c2) edge node[right]{$1$} (c3)
                                (a5) edge node[above]{$R(i_1+i_2)\eta_{c_1+c_2}$} (b5)
			(c5) edge node[above]{$R(o_1+o_2)\eta_{c_1'+c_2'}$} (b5)
                                (a) edge node[left]{$1$} (a5)
                                (b) edge node[left]{$\kappa$} (b5)
(b) edge node[right]{$\tau_3$} (b5)
			(c) edge node[right]{$1$} (c5)
                                (a6) edge node[above]{$R(i_2+i_1)\eta_{c_2+c_1}$} (b6)
			(c6) edge node[above]{$R(o_2+o_1)\eta_{c_2'+c_1'}$} (b6)
                                (a5) edge node[left]{$\beta_\A$} (a6)
                                (b5) edge node[left]{$R(\beta_\X)$} (b6)
 (b5) edge node[right]{$\tau_4$} (b6)
			(c5) edge node[right]{$\beta_\A$} (c6);
		\end{tikzpicture}
	\]
Again, we have omitted the natural isomorphisms $\phi_{c_i,c_j}$ on the inward pointing morphisms on each cospan leg due to space restrictions. The corresponding maps of decorations amounting to the following commutative diagram in $F(R(x_2+x_1))$:
\[
\begin{tikzpicture}[scale=1.5]
\node (A) at (-.5,0) {$F((\kappa)(\beta_\A))(x_1+x_2)$};
\node (C) at (4,0) {$F(\kappa)(x_2+x_1)$};
\node (D) at (-.5,-1) {$F(R(\beta_\X))(x_1+x_2)$};
\node (E) at (4,-1) {$x_2+x_1$};
\path[->,font=\scriptsize,>=angle 90]
(A) edge node[above]{$F(\kappa)(\tau_1)$} (C)
(A) edge node[left]{$F(R(\beta_\X))(\tau_3)$} (D)
(D) edge node[above]{$\tau_4$} (E)
(C) edge node[right]{$\tau_2$} (E);
\end{tikzpicture}
\]
The above square commutes because $$F((\kappa)(\beta_\A))(x_1+x_2) = F((R(\beta_\X))(\kappa))(x_1+x_2)$$ as the corresponding braiding square for the finite colimit preserving functor $R \maps (\X,1_\X,+) \to (\A,1_\A,+)$ commutes. Thus the double functor $\mathbb{E} \maps _L\lCsp(\X) \to F\lCsp$ is symmetric monoidal.
\end{comment}

\end{proof}




%!!!Kenny: Is this biequivalnce symmetric monoidal? Mike says yes, but it follows from a result that he has yet to officially publish. It's a result of a paper he's currently working on with someone.

\begin{comment}
We can also define the part of the double equivalence $\mathbb{G} \maps F\lCsp \to {_L \mathbb{C} \textnormal{sp}(\X)}$ which goes in the other direction: again, the object component of this double functor will be $\mathbb{G}_0 = \id_{\A}$.

Given a horizontal 1-cell $M$ of $F\lCsp$:
\[
\begin{tikzpicture}[scale=1.5]
\node (A) at (0,0) {$c_1$};
\node (B) at (1,0) {$c$};
\node (C) at (2,0) {$c_2$};
\node (D) at (1,-0.5) {$x \in F(c)$};
\path[->,font=\scriptsize,>=angle 90]
(A) edge node[above]{$I$} (B)
(C) edge node[above]{$O$} (B);
\end{tikzpicture}
\]
the image $\mathbb{G}(M)$ is the horizontal 1-cell in $_L \lCsp(\X)$ given by:
\[
\begin{tikzpicture}[scale=1.5]
\node (A) at (0,0) {$L(c_1)$};
\node (B) at (1,0) {$x$};
\node (C) at (2,0) {$L(c_2)$};
\path[->,font=\scriptsize,>=angle 90]
(A) edge node[above]{$!_{c} L(I)$} (B)
(C) edge node[above]{$!_{c} L(O)$} (B);
\end{tikzpicture}
\]
where $!_{c} \maps L(c) \to x$ is the unique morphism from the trival decoration on $c$ given by $$!_{c} \mapseqq 1 \xrightarrow{\phi} F(0) \xrightarrow{F(!)} F(c)$$ to $x \in F(c)$. In other words, the trivial decoraction $!_{c}$ is initial in $F(c)$. Similarly, given a 2-morphism $(f,h,g,\tau) \maps M \to M'$ of $F\lCsp$:
\[
\begin{tikzpicture}[scale=1.5]
\node (D) at (1,0.5) {$x \in F(c)$};
\node (D') at (1,-1.5) {$x' \in F(c')$};
\node (A) at (0,0) {$c_1$};
\node (B) at (1,0) {$c$};
\node (C) at (2,0) {$c_2$};
\node (A') at (0,-1) {$c_1'$};
\node (B') at (1,-1) {$c'$};
\node (C') at (2,-1) {$c_2'$};
\path[->,font=\scriptsize,>=angle 90]
(A) edge node[above]{$I$} (B)
(C) edge node[above]{$O$} (B)
(A') edge node[above]{$I'$} (B')
(C') edge node[above]{$O'$} (B')
(A) edge node [left]{$f$} (A')
(B) edge node [left]{$h$} (B')
(C) edge node [right]{$g$} (C');
\end{tikzpicture}
\]
$$\tau \maps F(h)(x) \to x'$$
the image $\mathbb{G}(f,h,g,\tau)$ in $_L \lCsp(\X)$ is given by the 2-morphism:
\[
\begin{tikzpicture}[scale=1.5]
\node (A) at (0,0) {$L(c_1)$};
\node (B) at (1.5,0) {$x$};
\node (C) at (3,0) {$L(c_2)$};
\node (A') at (0,-1) {$L(c_1')$};
\node (B') at (1.5,-1) {$x'$};
\node (C') at (3,-1) {$L(c_2')$};
\path[->,font=\scriptsize,>=angle 90]
(A) edge node[above]{$!_{c} L(I)$} (B)
(C) edge node[above]{$!_{c} L(O)$} (B)
(A') edge node[above]{$!_{c'} L(I')$} (B')
(C') edge node[above]{$!_{c'} L(O')$} (B')
(A) edge node [left]{$L(f)$} (A')
(B) edge node [left]{$\alpha$} (B')
(C) edge node [right]{$L(g)$} (C');
\end{tikzpicture}
\]
where $\alpha \maps x \to x'$ is the morphism in the Grothendieck construction of $F$ given by $\alpha = (h \maps c \to c', \tau \maps F(h)(x) \to x')$. 

Next, we exhibit natural isomorphisms $\eta \maps \id_{ _L \lCsp(\X)} \cong \mathbb{G} \mathbb{E}$ and $\epsilon \maps \mathbb{E} \mathbb{G} \cong \id_{F\lCsp}$. Specifically, these are double natural isomorphisms given by double transformations \cite{Shul2} $\eta$ and $\epsilon$ whose object and arrow components are natural isomorphisms.

First we compute the composites $\mathbb{G} \mathbb{E}$ and $\mathbb{E} \mathbb{G}$. On the object categories, both composites are $\id_\A$ and we have natural isomorphisms $\eta \maps \id_{\A} \cong \mathbb{G}_0 \mathbb{E}_0$ and $\epsilon \maps \mathbb{E}_0 \mathbb{G}_0 \cong \id_{\A}$.

Given a horizontal 1-cell $M$ in $_L \mathbb{C} \textnormal{sp}(\X)$:
\[
\begin{tikzpicture}[scale=1.5]
\node (A) at (0,0) {$L(c_1)$};
\node (B) at (1,0) {$x$};
\node (C) at (2,0) {$L(c_2)$};
\path[->,font=\scriptsize,>=angle 90]
(A) edge node[above]{$i$} (B)
(C) edge node[above]{$o$} (B);
\end{tikzpicture}
\]
the horizontal 1-cell $\mathbb{E}(M)$ is given by:
\[
\begin{tikzpicture}[scale=1.5]
\node (A) at (0,0) {$c_1$};
\node (B) at (1,0) {$R(x)$};
\node (C) at (2,0) {$c_2$};
\node (D) at (1,-0.5) {$x \in F(R(x))$};
\path[->,font=\scriptsize,>=angle 90]
(A) edge node[above]{$R(i) \eta_{c_1}$} (B)
(C) edge node[above]{$R(o) \eta_{c_2}$} (B);
\end{tikzpicture}
\]
and then the horizontal 1-cell $\mathbb{G} \mathbb{E}(M)$ is given by:
\[
\begin{tikzpicture}[scale=1.5]
\node (A) at (0,0) {$L(c_1)$};
\node (B) at (1.5,0) {$x$};
\node (C) at (3,0) {$L(c_2)$};
\path[->,font=\scriptsize,>=angle 90]
(A) edge node[above]{$!_{R(x)}L(R(i)\eta_{c_1})$} (B)
(C) edge node[above]{$!_{R(x)}L(R(o)\eta_{c_2})$} (B);
\end{tikzpicture}
\]
Then we can find a 2-isomorphism $\eta_M \maps M \xrightarrow{\sim} \mathbb{G}\mathbb{E}(M) $ in $_L \mathbb{C} \textnormal{sp}(\X)$ given by:
\[
\begin{tikzpicture}[scale=1.5]
\node (A) at (0,0) {$L(c_1)$};
\node (B) at (1.5,0) {$x$};
\node (C) at (3,0) {$L(c_2)$};
\node (A') at (0,-1) {$L(c_1)$};
\node (B') at (1.5,-1) {$x$};
\node (C') at (3,-1) {$L(c_2)$};
\path[->,font=\scriptsize,>=angle 90]
(A) edge node[above]{$i$} (B)
(C) edge node[above]{$o$} (B)
(A') edge node[above]{$!_{R(x)}L(R(i)\eta_{c_1})$} (B')
(C') edge node[above]{$!_{R(x)}L(R(o)\eta_{c_2})$} (B')
(A) edge node [left]{$1$} (A')
(B) edge node [left]{$1$} (B')
(C) edge node [right]{$1$} (C');
\end{tikzpicture}
\]
where
\[
\begin{tikzpicture}[scale=1.5]
\node (A) at (-0.5,0) {$L(c_1)$};
\node (B) at (1,0) {$L(R(L(c_1)))$};
\node (C) at (2.5,0) {$L(R(x))$};
\node (D) at (3.5,0) {$x$};
\path[->,font=\scriptsize,>=angle 90]
(A) edge node[above]{$L(\eta_{c_1})$} (B)
(B) edge node[above]{$L(R(i))$} (C)
(C) edge node[above]{$!_{R(x)}$} (D)
(A) edge[bend right] node [above]{$i$} (D);
\end{tikzpicture}
\]
commutes.

On the other hand, given a horizontal 1-cell $N$ in $F\lCsp$:
\[
\begin{tikzpicture}[scale=1.5]
\node (A) at (0,0) {$c_1$};
\node (B) at (1,0) {$c$};
\node (C) at (2,0) {$c_2$};
\node (D) at (1,-0.5) {$x \in F(c)$};
\path[->,font=\scriptsize,>=angle 90]
(A) edge node[above]{$I$} (B)
(C) edge node[above]{$O$} (B);
\end{tikzpicture}
\]
the horizontal 1-cell $\mathbb{G}(N)$ is given by:
\[
\begin{tikzpicture}[scale=1.5]
\node (A) at (0,0) {$L(c_1)$};
\node (B) at (1,0) {$x$};
\node (C) at (2,0) {$L(c_2)$};
\path[->,font=\scriptsize,>=angle 90]
(A) edge node[above]{$!_c L(I)$} (B)
(C) edge node[above]{$!_c L(O)$} (B);
\end{tikzpicture}
\]
and then the horizontal 1-cell $\mathbb{E} \mathbb{G}(N)$ is given by:
\[
\begin{tikzpicture}[scale=1.5]
\node (A) at (0,0) {$c_1$};
\node (B) at (1.5,0) {$R(x)$};
\node (C) at (3,0) {$c_2$};
\node (D) at (1.5,-0.5) {$x \in F(R(x))$};
\path[->,font=\scriptsize,>=angle 90]
(A) edge node[above]{$R(!_c L(I))\eta_{c_1}$} (B)
(C) edge node[above]{$R(!_c L(O))\eta_{c_2}$} (B);
\end{tikzpicture}
\]
Then we have a 2-isomorphism $\epsilon_N \maps \mathbb{E} \mathbb{G} (N) \xrightarrow{\sim} N$ in $F\lCsp$ given by:
\[
\begin{tikzpicture}[scale=1.5]
\node (D) at (1.5,0.5) {$x \in F(R(x))$};
\node (D') at (1.5,-1.5) {$x \in F(c)$};
\node (A) at (0,0) {$c_1$};
\node (B) at (1.5,0) {$R(x)$};
\node (C) at (3,0) {$c_2$};
\node (A') at (0,-1) {$c_1$};
\node (B') at (1.5,-1) {$c$};
\node (C') at (3,-1) {$c_2$};
\path[->,font=\scriptsize,>=angle 90]
(A) edge node[above]{$R(!_c L(I))\eta_{c_1}$} (B)
(C) edge node[above]{$R(!_c L(O))\eta_{c_2}$} (B)
(A') edge node[above]{$I$} (B')
(C') edge node[above]{$O$} (B')
(A) edge node [left]{$1$} (A')
(B) edge node [left]{$\hat{e}$} (B')
(C) edge node [right]{$1$} (C');
\end{tikzpicture}
\]
$$\tau \maps F(\hat{e})(x) \to x$$
where 
\[
\begin{tikzpicture}[scale=1.5]
\node (A) at (0,0) {$c_1$};
\node (B) at (1,0) {$R(L(c_1))$};
\node (C) at (2.5,0) {$R(L(c))$};
\node (D) at (3.5,0) {$R(x)$};
\node (E) at (4.5,0) {$c$};
\path[->,font=\scriptsize,>=angle 90]
(A) edge node[above]{$\eta_{c_1}$} (B)
(B) edge node[above]{$R(L(I))$} (C)
(C) edge node[above]{$R(!_c)$} (D)
(D) edge node[above]{$\hat{e}$} (E)
(A) edge[bend right] node [above]{$I$} (E);
\end{tikzpicture}
\]
commutes.

\end{comment}


\section{Bicategorical and categorical aspects}
\label{spinoffs}

While double categories are a natural context for studying cospans, bicategories are more 
familiar---and of course, mere \emph{categories} are even more so!   Luckily, all our results 
phrased in the language of double categories have analogues for bicategories and categories.  
We explain those here.

As noted for example by Shulman \cite{Shulman2010}, any double category $\lX$ has a 
\define{horizontal bicategory}, denoted $\bX$, in which:
\begin{itemize}
\item objects are objects of $\lX$,
\item morphisms are horizontal 1-cells of $\lX$,
\item 2-morphisms are \define{globular} 2-morphisms of $\lX$, meaning 2-morphisms whose source and target vertical 1-morphisms are identities,
\item composition of morphisms is given by horizontal composition of horizontal 1-cells in $\lX$,
\item vertical and horizontal composition of 2-morphisms are given by vertical and horizontal
composition of 2-morphisms in $\lX$.
\end{itemize}
The bicategory $\bX$ then has a \define{decategorification}, a category $\X$ in which:
\begin{itemize}
\item objects are objects of $\bX$,
\item morphisms are isomorphism classes of morphisms of $\bX$.   
\end{itemize}
Thus, the double category $F\lCsp$ of structured cospans constructed in \cref{thm:decorated_cospans} automatically gives rise to a bicategory, which we call $F\bCsp$, and a category, which we call $F\Csp$.   In \cref{DC} we gave conditions under which the double category $F\lCsp$ becomes symmetric monoidal.   We would like the bicategory $F\bCsp$ and the category $F\Csp$ to become symmetric monoidal under the same conditions, and indeed this is true.   

A double category is `fibrant' if every vertical 1-morphism has a `companion' and a `conjoint'---concepts explained in \cref{def:companion}.   Shulman \cite[Thm.\ 1.2]{Shulman2010} showed how when a double category $\lX$ is fibrant, any symmetrical monoidal structure on $\lX$ gives one on $\bX$.     We can apply this to decorated cospans as follows:

\begin{lem}
The double category $F\lCsp$ is fibrant.
\end{lem}

\begin{proof}
We show that any vertical 1-morphism $f \maps c \to c'$ in $F\lCsp$ has a companion and a conjoint.  This horizontal 1-cell $\hat{f}$:
\[
\begin{tikzpicture}[scale=1.5]
\node (A) at (0,0) {$c$};
\node (B) at (1,0) {$c'$};
\node (C) at (2,0) {$c'$};
\node (D) at (1,-0.5) {$!_{c'} \in F(c')$};
\path[->,font=\scriptsize,>=angle 90]
(A) edge node[above]{$f$} (B)
(C) edge node[above]{$1$} (B);
\end{tikzpicture}
\]
is a companion of $f$ because the following two 2-morphisms, having $\hat{f}$ as
source and target:
\[
\begin{tikzpicture}[scale=1.5]
\node (A) at (0,0.5) {$c$};
\node (A') at (0,-0.5) {$c'$};
\node (B) at (1,0.5) {$c'$};
\node (C) at (2,0.5) {$c'$};
\node (C') at (2,-0.5) {$c'$};
\node (D) at (1,-0.5) {$c'$};
\node (E) at (3,0.5) {$!_{c'} \in F(c')$};
\node (F) at (3,-0.5) {$!_{c'} \in F(c')$};
\node (G) at (5,0.5) {$c$};
\node (H) at (6,0.5) {$c$};
\node (I) at (7,0.5) {$c$};
\node (G') at (5,-0.5) {$c$};
\node (H') at (6,-0.5) {$c'$};
\node (I') at (7,-0.5) {$c'$};
\node (J) at (8,0.5) {$!_{c} \in F(c)$};
\node (K) at (8,-0.5) {$!_{c'} \in F(c')$};
\node (L) at (1,-1) {$\tau_{1_{c'}} = 1_{!_{c'}}$};
\node (M) at (6,-1) {$\tau_f \maps F(f)(!_c) \to !_{c'}$};
\path[->,font=\scriptsize,>=angle 90]
(A) edge node[above]{$f$} (B)
(C) edge node[above]{$1$} (B)
(A) edge node[left]{$f$} (A')
(C) edge node[left]{$1$} (C')
(A') edge node[above] {$1$} (D)
(C') edge node[above] {$1$} (D)
(B) edge node [left] {$1$} (D)
(G) edge node [above] {$1$} (H)
(G) edge node [left] {$1$} (G')
(H) edge node [left] {$f$} (H')
(G') edge node [above] {$f$} (H')
(I) edge node [above] {$1$} (H)
(I) edge node [left] {$f$} (I')
(I') edge node [above] {$1$} (H');
\end{tikzpicture}
\]
satisfy the equations required of a companion:
\[
\begin{tikzpicture}[scale=1.5]
\node (N) at (0,1.5) {$c$};
\node (O) at (1,1.5) {$c$};
\node (P) at (2,1.5) {$c$};
\node (Q) at (-1,1.5) {$!_c \in F(c)$};
\node (A) at (0,0.5) {$c$};
\node (A') at (0,-0.5) {$c'$};
\node (B) at (1,0.5) {$c'$};
\node (C) at (2,0.5) {$c'$};
\node (C') at (2,-0.5) {$c'$};
\node (D) at (1,-0.5) {$c'$};
\node (E) at (-1,0.5) {$!_{c'} \in F(c')$};
\node (F) at (-1,-0.5) {$!_{c'} \in F(c')$};
\node (G) at (4,1) {$c$};
\node (H) at (5,1) {$c$};
\node (I) at (6,1) {$c$};
\node (G') at (4,0) {$c'$};
\node (H') at (5,0) {$c'$};
\node (I') at (6,0) {$c'$};
\node (J) at (7,1) {$!_{c} \in F(c)$};
\node (K) at (7,0) {$!_{c'} \in F(c')$};
\node (Q) at (1,-1) {$\tau_f \maps F(f)(!_c) \to !_{c'}$};
\node (L) at (1,-1.5) {$\tau_{c'} = 1_{!_{c'}}$};
\node (M) at (5,-0.5) {$\tau_f \maps F(f)(!_c) \to !_{c'}$};
\node (R) at (3,0.5) {$=$};
\path[->,font=\scriptsize,>=angle 90]
(N) edge node[above]{$1$} (O)
(P) edge node[above]{$1$} (O)
(N) edge node[left]{$1$} (A)
(O) edge node[left]{$f$} (B)
(P) edge node[left]{$f$} (C)
(A) edge node[above]{$f$} (B)
(C) edge node[above]{$1$} (B)
(A) edge node[left]{$f$} (A')
(C) edge node[left]{$1$} (C')
(A') edge node[above] {$1$} (D)
(C') edge node[above] {$1$} (D)
(B) edge node [left] {$1$} (D)
(G) edge node [above] {$1$} (H)
(G) edge node [left] {$f$} (G')
(H) edge node [left] {$f$} (H')
(G') edge node [above] {$1$} (H')
(I) edge node [above] {$1$} (H)
(I) edge node [left] {$f$} (I')
(I') edge node [above] {$1$} (H');
\end{tikzpicture}
\]
\[
\begin{tikzpicture}[scale=1.5]
\node (G) at (-1,0.5) {$c$};
\node (H) at (-1,-0.5)  {$c'$};
\node (I) at (-2,0.5) {$c$};
\node (J) at (-2,-0.5) {$c$};
\node (A) at (0,0.5) {$c$};
\node (A') at (0,-0.5) {$c'$};
\node (B) at (1,0.5) {$c'$};
\node (C) at (2,0.5) {$c'$};
\node (C') at (2,-0.5) {$c'$};
\node (D) at (1,-0.5) {$c'$};
\node (E) at (1,1) {$!_{c'} \in F(c')$};
\node (F) at (1,-1) {$!_{c'} \in F(c')$};

\node (L) at (1,-1.5) {$\tau_{c'} = 1_{!_{c'}}$};
\node (E') at (-1,1) {$!_{c} \in F(c)$};
\node (F') at (-1,-1) {$!_{c'} \in F(c')$};

\node (L') at (-1,-1.5) {$\tau_{f} \maps F(f)(!_c) \to !_{c'}$};

\node (M) at (2.5,0) {$=$};
\node (N) at (3,0.5) {$c$};
\node (O) at (3,-0.5) {$c$};
\node (P) at (4,0.5) {$c'$};
\node (Q) at (4,-0.5) {$c'$};
\node (R) at (5,0.5) {$c'$};
\node (S) at (5,-0.5) {$c'$};
\node (T) at (4,1) {$!_{c'} \in F(c')$};
\node (U) at (4,-1) {$!_{c'} \in F(c')$};
\node (V) at (4,-1.5) {$\tau_{c'} = 1_{!_{c'}}$};

\path[->,font=\scriptsize,>=angle 90]
(N) edge node[left]{$1$} (O)
(P) edge node[left]{$1$} (Q)
(R) edge node[left]{$1$} (S)
(N) edge node[above]{$f$} (P)
(O) edge node[above]{$f$} (Q)
(R) edge node[above]{$1$} (P)
(S) edge node[above]{$1$} (Q)

(A) edge node[above]{$f$} (B)
(C) edge node[above]{$1$} (B)
(A) edge node[left]{$f$} (A')
(C) edge node[left]{$1$} (C')
(A') edge node[above] {$1$} (D)
(C') edge node[above] {$1$} (D)
(B) edge node [left] {$1$} (D)
(A) edge node[above]{$1$} (G)
(G) edge node[left]{$f$} (H)
(A') edge node[above]{$1$} (H)
(J) edge node[above] {$f$} (H)
(I) edge node[left] {$1$} (J)
(I) edge node [above] {$1$} (G);
\end{tikzpicture}
\]
and right hand sides of the above two equations are given respectively by the 2-morphisms $U_f$ and $1_{\hat{f}}$. The conjoint of $f$ is given by this horizontal 1-cell $\check{f}$, which is just the opposite of the companion above:
\[
\begin{tikzpicture}[scale=1.5]
\node (A) at (0,0) {$c'$};
\node (B) at (1,0) {$c'$};
\node (C) at (2,0) {$c$};
\node (D) at (4,0) {$!_{c'} \in F(c')$};
\path[->,font=\scriptsize,>=angle 90]
(A) edge node[above]{$1$} (B)
(C) edge node[above]{$f$} (B);
\end{tikzpicture}
\]
Just as $\hat{f}$ obeys the equations required of a companion, $\check{f}$ obeys the equations required of a conjoint.
\end{proof}


\begin{thm}
\label{thm:bicat}
Let $\A$ be a category with finite colimits and $F \maps (\A, +) \to (\Cat,\times)$ a symmetric lax monoidal pseudofunctor. Then there exists a symmetric monoidal bicategory $F \mathbf{Csp}$ in which:
\begin{enumerate}
\item objects are those of $\A$,
\item morphisms are $F$-decorated cospans:
\[
\begin{tikzpicture}[scale=1.5]
\node (A) at (0,0) {$a$};
\node (B) at (1,0) {$c$};
\node (C) at (2,0) {$b$};
\node (D) at (4,0) {$d \in F(c)$,};
\path[->,font=\scriptsize,>=angle 90]
(A) edge node[above]{$i$} (B)
(C) edge node[above]{$o$} (B);
\end{tikzpicture}
\]
\item a 2-morphism is a map of cospans in $\A$ 
\[
\begin{tikzpicture}[scale=1.5]
\node (A) at (0,0) {$a$};
\node (B) at (1,0.5) {$c$};
\node (C) at (2,0) {$b$};
\node (E) at (1,-0.5) {$c'$};
\node (D) at (3,0.5) {$d \in F(c)$};
\node (F) at (3,-0.5) {$d' \in F(c')$};
\path[->,font=\scriptsize,>=angle 90]
(A) edge node[above]{$i$} (B)
(C) edge node[above]{$o$} (B)
(A) edge node[below]{$i'$} (E)
(B) edge node[left]{$h$} (E)
(C) edge node[below]{$o'$} (E);
\end{tikzpicture}
\]
together with a morphism $\tau \maps F(h)(d) \to d'$ in $F(c')$.
\end{enumerate}
\end{thm}

\begin{proof}
This follows by applying Shulman's result \cite[Thm.\ 1.2]{Shulman2010} to the fibrant symmetric monoidal double category $F\lCsp$.
\end{proof}

This symmetric monoidal bicategory $F\bCsp$ is a superior version of the one constructed earlier by the second author \cite{Courser}, in that there is greater flexibility in the allowed 2-morphisms.    We can decategorify $F\bCsp$ to obtain a symmetric monoidal category generalizing the kind considered by Fong \cite{Fong}:

\begin{cor}
Let $\A$ be a category with finite colimits and $F \maps (\A, +) \to (\Cat, \times)$ a symmetric lax monoidal pseudofunctor.  Then there exists a symmetric monoidal category $F\Csp$ in which:
\begin{enumerate}
\item objects are those of $\A$
\item morphisms as isomorphism classes of $F$-decorated cospans of $\A$, where two
$F$-decorated cospans
\[
\begin{tikzpicture}[scale=1.5]
\node (A) at (0,0) {$a$};
\node (B) at (1,0) {$c$};
\node (C) at (2,0) {$b$};
\node (D) at (4,0) {$d \in F(c)$};
\path[->,font=\scriptsize,>=angle 90]
(A) edge node[above]{$i$} (B)
(C) edge node[above]{$o$} (B);
\end{tikzpicture}
\]
\[
\begin{tikzpicture}[scale=1.5]
\node (A) at (0,0) {$a$};
\node (B) at (1,0) {$c'$};
\node (C) at (2,0) {$b$};
\node (D) at (4,0) {$d' \in F(c')$};
\path[->,font=\scriptsize,>=angle 90]
(A) edge node[below]{$i'$} (B)
(C) edge node[below]{$o'$} (B);
\end{tikzpicture}
\]
are isomorphic if and only if there exists an isomorphism $f \maps c \to c'$ in $\A$ such that following diagram commutes:
\[
\begin{tikzpicture}[scale=1.5]
\node (A) at (0,0) {$a$};
\node (B') at (1,0.5) {$c$};
\node (B) at (1,-0.5) {$c'$};
\node (C) at (2,0) {$b$};
\path[->,font=\scriptsize,>=angle 90]
(A) edge node[below]{$i'$} (B)
(C) edge node[below]{$o'$} (B)
(A) edge node[above]{$i$} (B')
(C) edge node[above]{$o$} (B')
(B') edge node[left]{$f$} (B);
\end{tikzpicture}
\]
and there exists an isomorphism $\tau \maps F(f)(d) \to d'$ in $F(c')$.
\end{enumerate}
\end{cor}

In \cref{thm:equiv} we gave conditions under which the symmetric monoidal double category of \emph{decorated} cospans $F\lCsp$ is equivalent to the  symmetric monoidal double category of \emph{structured} cospans ${}_L \lCsp(\int F)$.   We now show that under the same conditions we get an equivalence of symmetric monoidal bicategories, and of categories.

\begin{thm} \label{thm:bicat_equiv}
Suppose $\ca{A}$ has finite colimits and $F \maps(\ca{A},+) \to (\bicat{Cat},\times)$ is a symmetric lax monoidal pseudofunctor that factors through $\bicat{Rex}$ as an ordinary pseudofunctor.    Define the symmetric monoidal bicategory $_L\bCsp(\inta F)$ as in \cref{thm:equiv}.   Then there is an equivalence of symmetric monoidal bicategories
\[      F\bCsp \simeq {}_L \bCsp(\inta F)   \]
of symmetric monoidal categories
\[      F\Csp \simeq {}_L \Csp(\inta F)  . \]
\end{thm}

\begin{proof} Hansen and Shulman \cite{HS} showed that the passage from symmetric monoidal double categories to symmetric monoidal bicategories is  functorial in a suitable sense, implying
that an equivalence of symmetric monoidal double categories $\lX \simeq \lY$ gives an equivalence of symmetric monoidal bicategories $\bX \simeq \bY$.    Since the process of decategorifying a bicategory merely discards 2-morphisms and takes isomorphism classes of 1-morphisms, the equivalence of symmetric monoidal bicategories $\bX \simeq \bY$ in turn induces an equivalence of symmetric monoidal categories $\X \simeq \Y$.   Thus, the theorem follows from \cref{thm:equiv}. \end{proof}

CHECK THAT OUR SORT OF EQUIVALENCE OF SYMMETRIC MONOIDAL DOUBLE CATEGORIES IS REALLY THAT USED BY HANSEN AND SHULMAN!!!

%Let $F \maps \A \to \Cat$ be a symmetric lax monoidal pseudofunctor. Then by Theorem \cref{DC} of the previous section, we get a symmetric monoidal double category $F\lCsp$. This symmetric monoidal double category has:
%\begin{enumerate}
%\item{objects as those of $\A$,}
%\item{vertical 1-morphisms as morphisms of $\A$,}
%\item{horizontal 1-cells as pairs:
%\[
%\begin{tikzpicture}[scale=1.5]
%\node (A) at (0,0) {$a$};
%\node (B) at (1,0) {$c$};
%\node (C) at (2,0) {$b$};
%\node (D) at (4,0) {$d \in F(c)$};
%\path[->,font=\scriptsize,>=angle 90]
%(A) edge node[above]{$i$} (B)
%(C) edge node[above]{$o$} (B);
%\end{tikzpicture}
%\]
%and}
%\item{2-morphisms as maps of cospans in $\A$
%\[
%\begin{tikzpicture}[scale=1.5]
%\node (A) at (0,0.5) {$a$};
%\node (A') at (0,-0.5) {$a'$};
%\node (B) at (1,0.5) {$c$};
%\node (C) at (2,0.5) {$b$};
%\node (C') at (2,-0.5) {$b'$};
%\node (D) at (1,-0.5) {$c'$};
%\node (E) at (3,0.5) {$d \in F(c)$};
%\node (F) at (3,-0.5) {$d' \in F(c')$};
%\path[->,font=\scriptsize,>=angle 90]
%(A) edge node[above]{$$} (B)
%(C) edge node[above]{$$} (B)
%(A) edge node[left]{$f$} (A')
%(C) edge node[left]{$g$} (C')
%(A') edge node {$$} (D)
%(C') edge node {$$} (D)
%(B) edge node [left] {$h$} (D);
%\end{tikzpicture}
%\]
%together with a morphism $\tau \maps F(h)(d) \to d'$ in $F(c')$.}
%\end{enumerate}


\section{Applications}\label{Applications}
\begin{comment}
In this section we present several examples each of which may be realized in the context of decorated cospans or in the context of structured cospans. The first example regarding graphs was mentioned in the introduction and used as a reoccurring theme throughout the paper. The next three examples which take on more of an applied flavor, consists of electrical circuits, Markov processes and Petri nets. Each of these has been studied extensively by Baez, Fong, Master and Pollard by way of `black-boxing' \cite{BCR,BF,BFP,BM,BP}. Black-boxing is a way of interpreting the behavior of an open system, that is, a system with prescribed inputs and outputs such as the terminals of an electrical circuit, by observing the activity at the inputs and the outputs. The semantics of the activity at an open system's inputs and outputs is typically described in a category such as $\mathsf{LinRel}$ of finite dimensonal vector spaces and linear relations. Thus, in each case, black-boxing results in functors such as: $$\blacksquare_1 \maps \mathsf{Circ} \to \mathsf{LinRel}$$ $$\blacksquare_2 \maps \mathsf{Mark} \to \mathsf{LinRel}$$ $$\blacksquare_3 \maps \Petri \to \mathsf{LinRel}.$$ Each of these black-boxing functors also possess other convenient properties such as being symmetric monoidal. The first two of these were first done using Fong's theory of decorated cospans and then extended by the first two authors using structured cospans. The last two of these were also extended by being realized as double functors between double categories in recent works \cite{BC,BM}.
\subsection{Graphs}
As a first example that was also mentioned in the introduction, let $L \maps \mathsf{Set} \to \mathsf{Graph}$ be the functor that assigns to a set $N$ the \emph{discrete graph} on $N$ which is the edgeless graph $L(N)$ with no edges and $N$ as its set of vertices. Both $\mathsf{Set}$ and $\mathsf{Graph}$ are cocartesian monoidal and the functor $L \maps \mathsf{Set} \to \mathsf{Graph}$ is left adjoint to the forgetful functor $R \maps \mathsf{Graph} \to \mathsf{Set}$ which assigns to a graph $G$ its underlying set of vertices $U(G)$. Using structured cospans and appealing to Theorem \cref{SC}, we get a symmetric monoidal double category $_L \lCsp(\mathsf{Graph})$ which has:
\begin{enumerate}
\item{sets as objects,}
\item{functions as vertical 1-morphisms,}
\item{cospans of graphs, or, \emph{open} graphs of the form
\[
\begin{tikzpicture}[scale=1.5]
\node (A) at (0,0) {$L(N)$};
\node (B) at (1,0) {$G$};
\node (C) at (2,0) {$L(M)$};
\path[->,font=\scriptsize,>=angle 90]
(A) edge node[above]{$I$} (B)
(C) edge node[above]{$O$} (B);
\end{tikzpicture}
\]
as horizontal 1-cells, where $L(N)$ and $L(M)$ are discrete graphs on the sets $N$ and $M$, respectively, $G$ is a graph and $I$ and $O$ are graph morphisms, and}
\item{maps of cospans of graphs of the form
\[
\begin{tikzpicture}[scale=1.5]
\node (A) at (0,0) {$L(N_1)$};
\node (B) at (1,0) {$G_1$};
\node (C) at (2,0) {$L(M_1)$};
\node (A') at (0,-1) {$L(N_2)$};
\node (B') at (1,-1) {$G_2$};
\node (C') at (2,-1) {$L(M_2)$};
\path[->,font=\scriptsize,>=angle 90]
(A) edge node[above]{$I_1$} (B)
(C) edge node[above]{$O_1$} (B)
(A') edge node[above]{$I_2$} (B')
(C') edge node[above]{$O_2$} (B')
(A) edge node [left]{$L(f)$} (A')
(B) edge node [left]{$\alpha$} (B')
(C) edge node [left]{$L(g)$} (C');
\end{tikzpicture}
\]
as 2-morphisms, where $L(f)$ and $L(g)$ are maps of discrete graphs induced by the underlying functions $f$ and $g$, respectively, and $\alpha \maps G_1 \to G_2$ is a graph morphism.
}
\end{enumerate}

We can obtain a similar symmetric monoidal double category using decorated cospans. Let $F \maps \mathsf{Set} \to \mathsf{Cat}$ be the symmetric lax monoidal pseudofunctor that assigns to a set $N$ the \emph{category} of all graph structures whose underlying set of vertices is $N$. Using Theorem \cref{DC}, we then obtain a symmetric monoidal double category $F\lCsp$ which has:
\begin{enumerate}
\item{sets as objects,}
\item{functions as vertical 1-morphisms,}
\item{horizontal 1-cells as pairs:
\[
\begin{tikzpicture}[scale=1.5]
\node (A) at (0,0) {$N$};
\node (B) at (1,0) {$P$};
\node (C) at (2,0) {$M$};
\node (D) at (3.25,0) {$G \in F(P)$};
\path[->,font=\scriptsize,>=angle 90]
(A) edge node[above]{$i$} (B)
(C) edge node[above]{$o$} (B);
\end{tikzpicture}
\]
which can also be thought of as open graphs, and}
\item{2-morphisms as maps of cospans of sets
\[
\begin{tikzpicture}[scale=1.5]
\node (A) at (0,0) {$N_1$};
\node (A') at (0,-1) {$N_2$};
\node (C') at (2,-1) {$M_2$};
\node (B) at (1,0) {$P_1$};
\node (C) at (2,0) {$M_1$};
\node (D) at (1,-1) {$P_2$};
\node (E) at (3,0) {$G_1 \in F(P_1)$};
\node (F) at (3,-1) {$G_2 \in F(P_2)$};
\path[->,font=\scriptsize,>=angle 90]
(A) edge node[above]{$i_1$} (B)
(C) edge node[above]{$o_1$} (B)
(A) edge node[left]{$f$} (A')
(C) edge node[right]{$g$} (C')
(C') edge node [above] {$o_2$} (D)
(A') edge node [above] {$i_2$} (D)
(B) edge node [left] {$h$} (D);
\end{tikzpicture}
\]
together with a graph morphism $\tau \maps F(h)(G_1) \to G_2$ in $F(P_2)$.}
\end{enumerate}
We thus have two symmetric monoidal double categories: $_L \lCsp(\mathsf{Graph})$ obtained from structured cospans and $F\lCsp$ obtained from decorated cospans. Both of these double categories have $\mathsf{Set}$ as their categories of objects, open graphs as horizontal 1-cells and maps of open graphs as 2-morphisms, and by Theorem \cref{thm:equiv}, we have an equivalence of symmetric monoidal double categories $$_L \lCsp(\mathsf{Graph}) \sim F\lCsp.$$

\subsection{Passive linear circuits}
In a previous work \cite{BP}, Fong and the first author used decorated cospans to construct a symmetric monoidal category of `open passive linear circuits'. Roughly speaking, given a field $k$, a passive linear circuit is given by a `$k$-graph' which is a diagram in $\mathsf{FinSet}$ of the form:
\[
\begin{tikzpicture}[scale=1.5]
\node (B) at (1,0) {$E$};
\node (C) at (2,0) {$V$};
\node (A) at (0,0) {$k^+$};
\path[->,font=\scriptsize,>=angle 90]
(B)edge node[above] {$r$}(A)
(B)edge[bend left] node[above]{$s$}(C)
(B)edge[bend right] node[below]{$t$}(C);
\end{tikzpicture}
\]
Here the finite sets $E$ and $V$ are the sets of edges and vertices, respectively, and if we take the field $k = \mathbb{R}$, the function $r \maps E \to \mathbb{R}^+$ assigns to each edge $e \in E$ a positive real number $r(e) \in \mathbb{R}^+$ which can be interpreted as the resistance at the edge $e$. An open passive linear circuit is then given by a cospan of finite sets 
\[
\begin{tikzpicture}[scale=1.5]
\node (B) at (1,0) {$V$};
\node (C) at (2,0) {$Y$};
\node (A) at (0,0) {$X$};
\path[->,font=\scriptsize,>=angle 90]
(A)edge node[above] {$i$}(B)
(C)edge node[above]{$o$}(B);
\end{tikzpicture}
\]
where the apex $V$ is equipped with the structure of a passive linear circuit. See the original paper for more details \cite{BP}.

Let $\mathsf{Graph}_k$ be the category whose objects are given by $k$-graphs and morphisms by morphisms of $k$-graphs, where a morphism of $k$-graphs is given by a pair of functions $f \maps E \to E'$ and $g \maps V \to V'$ between the edge sets and vertex sets, respectively, of two $k$-graphs that respect the source and target functions of each. In a previous work introducing structured cospans, the first two authors showed that the category $\mathsf{Graph}_k$ has finite colimits \cite{BC2}. We can then obtain a double category of open passive linear circuits by defining a left adjoint $L \maps \mathsf{Set} \to \mathsf{Graph}_k$ that assigns to a set $V$ the discrete passive linear circuit $L(V)$ given by passive linear circuit with $V$ as its set of vertices and no edges. The resulting symmetric monoidal double category $_L \lCsp(\mathsf{Graph}_k)$ has:
\begin{enumerate}
\item{finite sets as objects,}
\item{functions as vertical 1-morphisms,}
\item{open passive linear circuits as horizontal 1-cells
\[
\begin{tikzpicture}[scale=1.5]
\node (B) at (1,0) {$V$};
\node (C) at (2,0) {$Y$};
\node (A) at (0,0) {$X$};
\node (D) at (3,0) {$k^+$};
\node (E) at (4,0) {$E$};
\node (F) at (5,0) {$V$};
\path[->,font=\scriptsize,>=angle 90]
(E) edge node [above] {$r$} (D)
(E) edge [bend left] node [above] {$s$} (F)
(E) edge [bend right] node [below] {$t$} (F)
(A)edge node[above] {$i$}(B)
(C)edge node[above]{$o$}(B);
\end{tikzpicture}
\]
and}
\item{maps of cospans as 2-morphisms together with a map of passive linear circuits between the apices.
\[
\begin{tikzpicture}[scale=1.5]
\node (A) at (0,0) {$X_1$};
\node (A') at (0,-1) {$X_2$};
\node (C') at (2,-1) {$Y_2$};
\node (B) at (1,0) {$V_1$};
\node (C) at (2,0) {$Y_1$};
\node (D) at (1,-1) {$V_2$};
\node (E) at (3,0) {$k^+$};
\node (E') at (3,-1) {$k^+$};
\node (F) at (4,0) {$E_1$};
\node (F') at (4,-1) {$E_2$};
\node (G) at (5,0) {$V_1$};
\node (G') at (5,-1) {$V_2$};
\path[->,font=\scriptsize,>=angle 90]
(F) edge node[above] {$r_1$} (E)
(F) edge [bend left] node [above] {$s_1$} (G)
(F) edge [bend right] node [below] {$t_1$} (G)
(F') edge node[above] {$r_2$} (E')
(F') edge [bend left] node [above] {$s_2$} (G')
(F') edge [bend right] node [below] {$t_2$} (G')
(A) edge node[above]{$i_1$} (B)
(C) edge node[above]{$o_1$} (B)
(A) edge node[left]{$h$} (A')
(C) edge node[left]{$h'$} (C')
(C') edge node [above] {$o_2$} (D)
(A') edge node [above] {$i_2$} (D)
(B) edge node [left] {$g$} (D)
(F) edge node [left] {$f$} (F')
(G) edge node [right] {$g$} (G');
\end{tikzpicture}
\]
}
\end{enumerate}
We can also obtain a similar double category using decorated cospans: define a pseudofunctor $F \maps \mathsf{FinSet} \to \Cat$ that assigns to a finite set $V$ the category of all $k$-graph structures on the finite set $V$ and to a function $f \maps V \to V'$ the corresponding functor $F(f) \maps F(V) \to F(V')$ between decoration categories. Both categories $\mathsf{FinSet}$ and $\Cat$ are symmetric monoidal and the pseudofunctor $F \maps \mathsf{FinSet} \to \Cat$ is symmetric lax monoidal, as given a $k$-graph structure on a finite set $V_1$ denoted by an element $K_1 \in F(V_1)$ and a $k$-graph structure on a finite set $V_2$ denoted by an element $K_2 \in F(V_2)$, we can consider the $k$-graph structures simultaneously as a single graph structure $\phi_{V_1,V_2}(K_1,K_2)$ on the finite set $V_1+V_2$. Thus we get a family of natural transformations $$\phi_{V_1,V_2} \maps F(V_1) \times F(V_2) \to F(V_1+V_2)$$ as well as a morphism $\phi \maps 1_{\mathsf{Graph}_k} \to F(\emptyset)$ which together satisfy the coherence conditions of a monoidal functor. The braiding is also clear as the following diagram commutes:
\[
\begin{tikzpicture}[scale=1.5]
\node (E) at (3,0) {$F(V_1) \times F(V_2)$};
\node (G) at (5,0) {$F(V_2) \times F(V_1)$};
\node (E') at (3,-1) {$F(V_1+V_2)$};
\node (G') at (5,-1) {$F(V_2+V_1)$};
\path[->,font=\scriptsize,>=angle 90]
(E) edge node[left]{$\phi_{V_1,V_2}$} (E')
(G) edge node[right]{$\phi_{V_2,V_1}$} (G')
(E) edge node[above]{$\beta_{V_1,V_2}$} (G)
(E') edge node[above]{$F(\beta_{V_1,V_2})$} (G');
\end{tikzpicture}
\]
Thus the pseudofunctor $F$ is symmetric lax monoidal and so by Theorem \cref{DC} we get a symmetric monoidal double category $F\lCsp$ which has:
\begin{enumerate}
\item{objects as finite sets,}
\item{vertical 1-morphisms as functions,}
\item{horizontal 1-cells as cospans of finite sets together with the structure of a $k$-graph given by an element of the image of the apex under the pseudofunctor $F$:
\[
\begin{tikzpicture}[scale=1.5]
\node (D) at (-3,0) {$U$};
\node (E) at (-2,0) {$V$};
\node (F) at (-1,0) {$W$};
\node (A) at (0,0) {$K \in F(V)$};
\path[->,font=\scriptsize,>=angle 90]
(D) edge node [above] {$i$} (E)
(F) edge node [above] {$o$} (E);
\end{tikzpicture}
\]
and}
\item{2-morphisms as maps of cospans of finite sets 
\[
\begin{tikzpicture}[scale=1.5]
\node (E) at (3,0) {$U_1$};
\node (F) at (5,0) {$W_1$};
\node (G) at (4,0) {$V_1$};
\node (E') at (3,-1) {$U_2$};
\node (F') at (5,-1) {$W_2$};
\node (G') at (4,-1) {$V_2$};
\node (A) at (6,0) {$K_1 \in F(V_1)$};
\node (B) at (6,-1) {$K_2 \in F(V_2)$};
\path[->,font=\scriptsize,>=angle 90]
(F) edge node[above]{$o_1$} (G)
(E) edge node[left]{$f$} (E')
(F) edge node[right]{$g$} (F')
(G) edge node[left]{$h$} (G')
(E) edge node[above]{$i_1$} (G)
(E') edge node[above]{$i_2$} (G')
(F') edge node[above]{$o_2$} (G');
\end{tikzpicture}
\]
together with a morphism of $k$-graphs $\tau \maps F(h)(K_1) \to K_2$ in $F(V_2)$.}
\end{enumerate}
These two double categories $_L \lCsp(\mathsf{Graph}_k)$ and $F\lCsp$ are equivalent by Theorem \cref{thm:equiv}.

\subsection{Markov processes}
In another previous work \cite{BFP}, Fong, Pollard and the first author used decorated cospans to construct a symmetric monoidal category of `open Markov processes'. In this framework, a Markov process on a finite set $N$ is given by a diagram in $\mathsf{Set}$:
\[
\begin{tikzpicture}[scale=1.5]
\node (D) at (3,0) {$(0,\infty)$};
\node (E) at (4,0) {$E$};
\node (F) at (5,0) {$N$};
\path[->,font=\scriptsize,>=angle 90]
(E) edge node [above] {$r$} (D)
(E) edge [bend left] node [above] {$s$} (F)
(E) edge [bend right] node [below] {$t$} (F);
\end{tikzpicture}
\]
where $E$ and $N$ are finite sets of edges and nodes, respectively. This is really just a special case of the previous example of passive linear circuits with $k = \mathbb{R}$. An open Markov process is then of course a cospan of finite sets where the apex is equipped with a Markov process:
\[
\begin{tikzpicture}[scale=1.5]
\node (B) at (1,0) {$N$};
\node (C) at (2,0) {$Y$};
\node (A) at (0,0) {$X$};
\node (D) at (3,0) {$(0,\infty)$};
\node (E) at (4,0) {$E$};
\node (F) at (5,0) {$N$};
\path[->,font=\scriptsize,>=angle 90]
(E) edge node [above] {$r$} (D)
(E) edge [bend left] node [above] {$s$} (F)
(E) edge [bend right] node [below] {$t$} (F)
(A)edge node[above] {$i$}(B)
(C)edge node[above]{$o$}(B);
\end{tikzpicture}
\]
For example:
\[
\begin{tikzpicture}[->,>=stealth',shorten >=1pt,auto,node distance=3.7cm,
thick,main node/.style={circle,fill=white!20,draw,font=\sffamily\Large\bfseries},terminal/.style={circle,fill=blue!20,draw,font=\sffamily\Large\bfseries}]]
\node[main node](1) {$b_1$};
\node[main node](2) [left=0.8cm of 1] {$a_1$};.5
\node[main node](3) [below right=0.7cm and 2.4cm of 1] {$c_2$};
\node[main node](5) [above right=0.7cm and 2.4cm of 1] {$c_1$};
\node[main node](4) [above right=0.7cm and 2.4cm of 3] {$d_1$};
\node(A) [left=1cm of 2,circle,draw,inner sep=1pt,fill=gray,color=purple] {};
\node(C) [right=1cm of 4,circle,draw,inner sep=1pt,fill=gray,color=purple] {};
\node[left=1.25 cm of 2,color=purple] {$S$};
\node[right=1.25 cm of 4,color=purple] {$T$};
\path[every node/.style={font=\sffamily\small}, shorten >=1pt]
(3) edge [bend right =15] node[below] {$e_6$} (4)
(5) edge [bend left =15] node[below] {$e_5$} (4)
(2) edge [bend left=15] node[below] {$e_1$} (1)
(1) edge [bend left =15] node[below] {$e_2$} (5)
(5) edge [bend right =15] node[right] {$e_4$} (3)
(1) edge [bend right =15] node[below] {$e_3$} (3)
(3) edge [bend right =15] node[above] {$6$} (4)
(5) edge [bend left =15] node[above] {$6$} (4)
(2) edge [bend left=15] node[above] {$5$} (1)
(1) edge [bend left =15] node[above] {$8$} (5)
(5) edge [bend right =15] node[left] {$4$} (3)
(1) edge [bend right =15] node[above] {$8$} (3);
\path[color=purple, very thick, shorten >=10pt, ->, >=stealth] (C) edge (4);
\path[color=purple, very thick, shorten >=10pt, ->, >=stealth] (A) edge (2);
\end{tikzpicture}
\]
Here we have an open Markov process on the finite set $N = \{a_1,b_1,c_1,c_2,d_1 \}$ with input and output sets given by the singletons $S$ and $T$, respectively.

Fong, Pollard and the first author then add extra structure to open Markov processes such as populations at each node to obtain a symmetric monoidal category $\mathsf{DetlBalMark}$ of open Markov processes in `detailed balance' and then construct a black box functor $\blacksquare \maps \mathsf{DetBalMark} \to \mathsf{LinRel}$ that describes the steady state behavior of an open Markov process in detailed balance. On the way to doing this, one of the categories they construct using Fong's decorated cospan machinery is a symmetric monoidal category $\mathsf{Mark}$ which has:
\begin{enumerate}
\item{objects as finite sets and}
\item{morphisms as isomorphism classes of open Markov processes, where composition is by pushout.}
\end{enumerate}
This is done using a symmetric lax monoidal functor $F \maps \mathsf{FinSet} \to \mathsf{Set}$ which assigns to each finite set $N$ the (large) set of all Markov processes on $N$ as defined above. Viewing this functor $F$ as now a symmetric lax monoidal pseudofunctor $F \maps \mathsf{FinSet} \to \Cat$ that assigns to a finite set $N$ the \emph{category} $F(N)$ of all Markov processes on $N$, we then get by Theorem \cref{DC} a symmetric monoidal double category $F\lCsp$ which has:
\begin{enumerate}
\item{finite sets as objects,}
\item{functions as vertical 1-morphisms,}
\item{open Markov processes as horizontal 1-cells, and}
\item{maps of open Markov processes as 2-morphisms which are given by maps of cospans of finite sets:
\[
\begin{tikzpicture}[scale=1.5]
\node (A) at (0,0) {$X_1$};
\node (A') at (0,-1) {$X_2$};
\node (C') at (2,-1) {$Y_2$};
\node (B) at (1,0) {$N_1$};
\node (C) at (2,0) {$Y_1$};
\node (D) at (1,-1) {$N_2$};
\node (E) at (3,0) {$M_1 \in F(N_1)$};
\node (E') at (3,-1) {$M_2 \in F(N_2)$};
\path[->,font=\scriptsize,>=angle 90]
(A) edge node[above]{$i_1$} (B)
(C) edge node[above]{$o_1$} (B)
(A) edge node[left]{$f$} (A')
(C) edge node[left]{$f'$} (C')
(C') edge node [above] {$o_2$} (D)
(A') edge node [above] {$i_2$} (D)
(B) edge node [left] {$h$} (D);
\end{tikzpicture}
\]
together with a map of Markov processes $\tau \maps F(h)(M_1) \to M_2$ in $F(N_2)$, where a map between two Markov processes to be given by a pair of functions $(g,h)$ that make the following diagram commute:
\[
\begin{tikzpicture}[scale=1.5]
\node (E) at (3,0) {$(0,\infty)$};
\node (E') at (3,-1) {$(0,\infty)$};
\node (F) at (4,0) {$E_1$};
\node (F') at (4,-1) {$E_2$};
\node (G) at (5,0) {$N_1$};
\node (G') at (5,-1) {$N_2$};
\path[->,font=\scriptsize,>=angle 90]
(F) edge node[above] {$r_1$} (E)
(F) edge [bend left] node [above] {$s_1$} (G)
(F) edge [bend right] node [below] {$t_1$} (G)
(F') edge node[above] {$r_2$} (E')
(F') edge [bend left] node [above] {$s_2$} (G')
(F') edge [bend right] node [below] {$t_2$} (G')
(F) edge node [left] {$g$} (F')
(G) edge node [right] {$h$} (G');
\end{tikzpicture}
\]
}
\end{enumerate}
A symmetric monoidal double category of open Markov processes can also be obtained using structured cospans by defining a functor $L \maps \mathsf{FinSet} \to \mathsf{Mark}$ which assigns to a finite set $N$ the discrete Markov process $L(N)$ with no edges and to a function $f \maps N \to N'$ the induced map of discrete Markov processes. Both categories $\mathsf{FinSet}$ and $\mathsf{Mark}$ have finite colimits and the functor $L$ is left adjoint to the forgetful functor $R \maps \mathsf{Mark} \to \mathsf{FinSet}$ which maps a Markov process to its underlying finite set of states. By Theorem \cref{SC}, we get a symmetric monoidal double category $_L \lCsp(\mathsf{Mark})$ which has:
\begin{enumerate}
\item{objects as finite sets,}
\item{vertical 1-morphisms as functions,}
\item{horizontal 1-cells as cospans in $\mathrm{Mark}$ of the form:
\[
\begin{tikzpicture}[scale=1.5]
\node (D) at (-3,0) {$L(N_1)$};
\node (E) at (-2,0) {$M$};
\node (F) at (-1,0) {$L(N_2)$};
\path[->,font=\scriptsize,>=angle 90]
(D) edge node [above] {$I$} (E)
(F) edge node [above] {$O$} (E);
\end{tikzpicture}
\]
and}
\item{2-morphisms as maps of cospans in $\mathrm{Mark}$.
\[
\begin{tikzpicture}[scale=1.5]
\node (E) at (3,0) {$L(N_1)$};
\node (F) at (5,0) {$L(N_2)$};
\node (G) at (4,0) {$M$};
\node (E') at (3,-1) {$L(N_1')$};
\node (F') at (5,-1) {$L(N_2')$};
\node (G') at (4,-1) {$M'$};
\path[->,font=\scriptsize,>=angle 90]
(F) edge node[above]{$O$} (G)
(E) edge node[left]{$L(f_1)$} (E')
(F) edge node[right]{$L(f_2)$} (F')
(G) edge node[left]{$(g,h)$} (G')
(E) edge node[above]{$I$} (G)
(E') edge node[above]{$I'$} (G')
(F') edge node[above]{$O'$} (G');
\end{tikzpicture}
\]
}
\end{enumerate}
The two double categories $F\lCsp$ and $_L \mathbb{C} \textnormal{sp}(\mathsf{Mark})$ are equivalent by Theorem \cref{thm:equiv}.

In a more recent work \cite{BC}, the first two authors have constructed a symmetric monoidal double category of `open Markov processes' and `coarse-grainings', where roughly speaking, a coarse-graining is a way of approximating a larger open Markov process by a smaller one by partitioning the set of states into `lumps'. This construction uses neither decorated cospans nor structured cospans.

\subsection{Petri nets}
In a previous work, Master and the first author used structured cospans to obtain a symmetric monoidal double category of `open Petri nets'. A Petri net is given by a diagram in $\mathsf{Set}$ of the form:
\[
\begin{tikzpicture}[scale=1.5]
\node (B) at (1,0) {$T$};
\node (C) at (2,0) {$\mathbb{N}[S].$};
\path[->,font=\scriptsize,>=angle 90]
(B)edge[bend left] node[above]{$s$}(C)
(B)edge[bend right] node[below]{$t$}(C);
\end{tikzpicture}
\]
Here, $T$ is the set of \emph{transitions} and $S$ is the set of \emph{species}, and then $\mathbb{N}[S]$ is the free commutative monoid on the set $S$. Each transition then has a formal linear combination of species given by an element of $\mathbb{N}[S]$ as its source and target as prescribed by the functions $s$ and $t$, respectively. An example of a Petri net is given by:
\[
\begin{tikzpicture}
	\begin{pgfonlayer}{nodelayer}
		\node [style=species] (I) at (0,1) {H};
		\node [style=species] (T) at (0,-1) {O};
		\node [style=transition] (W) at (2,0) {$\alpha$};
		\node [style=species] (Water) at (4,0) {$\textnormal{H}_2$O};
%		\node [style=transition] (Something) at (6,0) {\tiny{Something}};
%		\node [style=species] (A) at (8,1) {O$\textnormal{H}^{-}$};
%		\node [style=species] (B) at (8,-1) {$\textnormal{H}_3 \textnormal{O}^{+}$};
	\end{pgfonlayer}
	\begin{pgfonlayer}{edgelayer}
		\draw [style=inarrow, bend right=40, looseness=1.00] (I) to (W);
		\draw [style=inarrow, bend left=40, looseness=1.00] (I) to (W);
		\draw [style=inarrow, bend right=40, looseness=1.00] (T) to (W);
		\draw [style=inarrow] (W) to (Water);
%		\draw [style=inarrow, bend left=40, looseness=1.00] (Water) to (Something);
%		\draw [style=inarrow, bend right=40, looseness=1.00] (Water) to (Something);
%		\draw [style=inarrow, bend left=40, looseness=1.00] (Something) to (A);
%		\draw [style=inarrow, bend right=40, looseness=1.00] (Something) to (B);
	\end{pgfonlayer}
\end{tikzpicture}
\]
This Petri net has a single transition $\alpha$ with $2\textnormal{H}+\textnormal{O}$ as its source and $\textnormal{H}_2 \textnormal{O}$ as its target. See the original paper for more details on Petri nets \cite{BM}.

Each set of species $S$ gives rise to a discrete Petri net $L(S)$ with $S$ as its set of species and no transitions. Master and the first author show the existence of a left adjoint $L \maps \mathsf{Set} \to \Petri$ where $\Petri$ is the category whose objects are Petri nets and whose `morphisms are morphisms of Petri nets'. They also show that $\Petri$ has finite colimits and thus using Theorem \cref{SC} obtain a symmetric monoidal double category $\lOpen(\Petri)$ of open Petri nets which has:
\begin{enumerate}
\item{objects given by sets,}
\item{vertical 1-morphisms given by functions,}
\item{horizontal 1-cells as open Petri nets which are given by cospans in $\Petri$ of the form:
\[
\begin{tikzpicture}[scale=1.5]
\node (D) at (-3,0) {$L(X)$};
\node (E) at (-2,0) {$P$};
\node (F) at (-1,0) {$L(Y)$};
\path[->,font=\scriptsize,>=angle 90]
(D) edge node [above] {$I$} (E)
(F) edge node [above] {$O$} (E);
\end{tikzpicture}
\]
and}
\item{2-morphisms as maps of cospans in $\Petri$ of the form:
\[
\begin{tikzpicture}[scale=1.5]
\node (E) at (3,0) {$L(X_1)$};
\node (F) at (5,0) {$L(Y_1)$};
\node (G) at (4,0) {$P_1$};
\node (E') at (3,-1) {$L(X_2)$};
\node (F') at (5,-1) {$L(Y_2)$};
\node (G') at (4,-1) {$P_2$};
\path[->,font=\scriptsize,>=angle 90]
(F) edge node[above]{$O_1$} (G)
(E) edge node[left]{$L(f)$} (E')
(F) edge node[right]{$L(g)$} (F')
(G) edge node[left]{$\alpha$} (G')
(E) edge node[above]{$I_1$} (G)
(E') edge node[above]{$I_2$} (G')
(F') edge node[above]{$O_2$} (G');
\end{tikzpicture}
\]
}
\end{enumerate}
We can also obtain a similar double category using decorated cospans: define a pseudofunctor $F \maps \mathsf{Set} \to \Cat$ where given a set $s$, $F(s)$ is the category of all Petri net structures with $s$ as its set of species. This pseudofunctor $F$ is symmetric lax monoidal as both $(\mathsf{Set},+,\emptyset)$ and $(\Cat,\times,1)$ are symmetric monoidal and given Petri nets $P \in F(s)$ and $P' \in F(s')$, we can place them side by side and consider them together as a single Petri net $P+P' \in F(s+s')$ with set of species $s+s'$, and thus we have natural transformations $\phi_{s,s'} \maps F(s) \times F(s') \to F(s+s')$ for any two sets $s$ and $s'$. The other structure morphism between monoidal units $\phi \maps 1_{\Petri} \to F(\emptyset)$ is defined by the unique morphism from the empty Petri net with the empty set forits set of species to the only possible Petri net on the empty set, which is also the empty Petri net. All of the diagrams that are required to commute are straightforward. Using Theorem \cref{DC}, we obtain a symmetric monoidal double category $F\lCsp$ which has:
\begin{enumerate}
\item{objects given by sets,}
\item{vertical 1-morphisms given by functions,}
\item{horizontal 1-cells given by pairs:
\[
\begin{tikzpicture}[scale=1.5]
\node (D) at (-3,0) {$X$};
\node (E) at (-2,0) {$Z$};
\node (F) at (-1,0) {$Y$};
\node (A) at (0,0) {$P \in F(Z)$};
\path[->,font=\scriptsize,>=angle 90]
(D) edge node [above] {$i$} (E)
(F) edge node [above] {$o$} (E);
\end{tikzpicture}
\]
and}
\item{2-morphisms as maps of cospans in $\mathsf{Set}$:
\[
\begin{tikzpicture}[scale=1.5]
\node (E) at (3,0) {$X_1$};
\node (F) at (5,0) {$Y_1$};
\node (G) at (4,0) {$Z_1$};
\node (E') at (3,-1) {$X_2$};
\node (F') at (5,-1) {$Y_2$};
\node (G') at (4,-1) {$Z_2$};
\node (A) at (6,0) {$P_1 \in F(Z_1)$};
\node (B) at (6,-1) {$P_2 \in F(Z_2)$};
\path[->,font=\scriptsize,>=angle 90]
(F) edge node[above]{$o_1$} (G)
(E) edge node[left]{$f$} (E')
(F) edge node[right]{$g$} (F')
(G) edge node[left]{$h$} (G')
(E) edge node[above]{$i_1$} (G)
(E') edge node[above]{$i_2$} (G')
(F') edge node[above]{$o_2$} (G');
\end{tikzpicture}
\]
together with a morphism of Petri nets $\tau \maps F(h)(P_1) \to P_2$ in $F(Z_2)$.}
\end{enumerate}
Thus we have a symmetric monoidal double category $\lOpen(\Petri)$ of open Petri nets obtained from structured cospans and a symmetric monoidal double category $F\lCsp$ of open Petri nets obtain from decorated cospans, and of course, we have an equivalence $\lOpen(\Petri) \sim F\lCsp$ of symmetric monoidal double categories by \cref{thm:equiv}.
\end{comment}



\appendix

\section{}
In this appendix, we gather some well-known concepts required to make the material self-contained, as well as references to more detailed expositions.

\subsection{2-categories}\label{sec:2cats}
For standard 2-categorical material, we refer the reader to e.g. \cite{KS} or \cite{DS} for monoidal structures.
Recall that a \define{pseudofunctor} $F\maps\bicat{A}\to\bicat{B}$ between bicategories $\bicat{A}$ and $\bicat{B}$ is functorial up to coherent natural isomorphism, namely for composable arrows we have $F(g\circ f)\cong Fg\circ Ff$ and $F(1_a)\cong1_{Fa}$ satisfying standard axioms. 
Given pseudofunctors $F,G\maps \bicat{A} \to \bicat{B}$, a \define{pseudonatural transformation} $\sigma$ consists of
\begin{itemize}
\item for each object $a \in \bicat{A}$, a morphism $\sigma_a \maps F(a) \to G(a)$ in $\bicat{B}$ 
\item 
%for every pair of objects $a$ and $b$ of $\mathbf{A}$, we have natural isomorphisms
%\[
%\begin{tikzpicture}[scale=1.5]
%\node (A) at (0,0) {$\mathbf{A}(a,b)$};
%\node (B) at (2,0) {$\mathbf{X}(F(a),F(b))$};
%\node (C) at (0,-1) {$\mathbf{X}(G(a),G(b))$};
%\node (D) at (2,-1) {$\mathbf{X}(F(a),G(b))$};
%\node (E) at (1,-0.5) {$\sigma_{a,b} \Nearrow$};
%\path[->,font=\scriptsize,>=angle 90]
%(A) edge node[above]{$F$} (B)
%(B) edge node[right]{$(\sigma_{b})_*$} (D)
%(A) edge node[left]{$G$} (C)
%(C) edge node[above]{$(\sigma_a)^*$} (D);
%\end{tikzpicture}
%\]
%where $(\sigma_a)^*$ and $(\sigma_b)_*$ are the functors induced by precomposition and postcompositon, respectively. Thus 
for each morphism $f \maps a \to b$ in $\mathbf{A}$, an invertible natural 2-morphism $\sigma_f \maps G(f) \sigma_a \xrightarrow{\sim} \sigma_{b} F(f)$ in $\mathbf{B}$
%:
%\[
%\begin{tikzpicture}[scale=1.5]
%\node (A) at (0,0) {$F(a)$};
%\node (B) at (1,0) {$F(b)$};
%\node (C) at (0,-1) {$G(a)$};
%\node (D) at (1,-1) {$G(b)$};
%\node (E) at (0.5,-0.5) {$\sigma_f \Nearrow$};
%\path[->,font=\scriptsize,>=angle 90]
%(A) edge node[above]{$F(f)$} (B)
%(B) edge node[right]{$\sigma_{b}$} (D)
%(A) edge node[left]{$\sigma_a$} (C)
%(C) edge node[above]{$G(f)$} (D);
%\end{tikzpicture}
%\]
compatible with composition and identities.
\end{itemize}
%such that for each composable pair of morphisms $f \maps a \to b$ and $g \maps b \to c$ of $\mathbf{A}$, the following diagrams commute:
%\[
%\begin{tikzpicture}[scale=1.5]
%\node (A) at (0,0.5) {$(G(g)G(f))\sigma_a$};
%\node (A') at (2,0.5) {$G(g)(G(f) \sigma_a)$};
%\node (B) at (0,-0.5) {$G(gf) \sigma_a$};
%\node (C) at (4,0.5) {$G(g) (\sigma_b F(f))$};
%\node (C') at (4,-0.5) {$\sigma_c F(gf)$};
%\node (D) at (6,0.5) {$(G(g) \sigma_b) F(f)$};
%\node (D') at (8,-0.5) {$\sigma_c (F(g)F(f))$};
%\node (F) at (8,0.5) {$(\sigma_c F(g)) F(f)$};
%\path[->,font=\scriptsize,>=angle 90]
%(A) edge node[above]{$a'$} (A')
%(A) edge node[left]{$\psi 1_{\sigma_a}$} (B)
%(A') edge node[above]{$1_{G(g)} \sigma_f$} (C)
%(B) edge node[above]{$\sigma_{gf}$} (C')
%(C) edge node [above] {${a'}^{-1}$} (D)
%(D') edge node [above] {$1_{\sigma_c} \phi$} (C')
%(D) edge node [above] {$\sigma_g 1_{F(f)}$} (F)
%(F) edge node [right] {$a'$} (D');
%\end{tikzpicture}
%\]
%\[
%\begin{tikzpicture}[scale=1.5]
%\node (A) at (0,0) {$1_{G(a)} \sigma_a$};
%\node (B) at (1,1) {$G(1_a) \sigma_a$};
%\node (C) at (2,0) {$\sigma_a F(1_a)$};
%\node (D) at (0.5,-1) {$\sigma_a$};
%\node (E) at (1.5,-1) {$\sigma_a 1_{F(a)}$};
%\path[->,font=\scriptsize,>=angle 90]
%(A) edge node[left]{$\psi 1_{\sigma_a}$} (B)
%(B) edge node[right]{$\sigma_{1_a}$} (C)
%(A) edge node [left] {$\ell'$} (D)
%(D) edge node [above] {${\rho'}^{-1}$} (E)
%(E) edge node [right] {$1_{\sigma_a} \phi$} (C);
%\end{tikzpicture}
%\]
%Christina: I removed the above, they are not needed I believe. Check again in the end.
We denote by $[\A,\bicat{Cat}]_\pse$ the 2-category 
of pseudofunctors, pseudonatural transformations and modifications from an ordinary category $\A$ viewed as a 2-category with trivial 2-cells, into $\bicat{Cat}$. This is also referred to as the 2-category of \emph{opindexed categories}, since an indexed category is a contravariant pseudofunctor into $\bicat{Cat}$.

A \define{monoidal} bicategory $\A$ comes with a pseudofunctor $\otimes\maps\A\times\A\to\A$ and a unit object $I$ that are associative and unital up to coherent equivalence. A \define{braided} monoidal bicategory comes with a pseudonatural equivalence $\beta_{a,b}\maps a\ot b\to b\ot a$ and appropriate invertible modifications; it is \define{sylleptic} if there is an appropriate invertible modification $1_{a\ot b}\Rrightarrow\beta_{b,a}\circ\beta_{a,b}$ and it is \define{symmetric} if one further axiom holds. A \define{lax monoidal} pseudofunctor between monoidal bicategories $F\maps\bicat{A}\to\bicat{B}$ is a pseudofunctor equipped with pseudonatural transformations with components $\mu_{a,b}\maps Fa\otimes Fb\to F(a\otimes b)$ and $\mu_0\maps I\to FI$ along with coherent invertible modifications for associativity and unitality. This is commonly also called `weak monoidal' pseudofunctor. A \define{braided lax monoidal} pseudofunctor between symmetric monoidal bicategories comes with an invertible modification with components $F(\beta_{a,b})\circ \mu_{a,b}\cong\mu_{b,a}\circ \beta_{Fa,Fb}$.


\subsection{Fibrations}\label{sec:fibrations}
Basic material regarding the theory of fibrations can be found, for example, in \cite{Borc,Gray}. Recall that a functor $U \maps \X \to \A$ is an \textbf{opfibration} if for every $x\in\X$ with $U(x)=a$ and $f \maps a \to b$ in $\A$, there exists a \textbf{cocartesian lifting} of $x$ along $f$, namely a morphism $\beta$ in $\X$ with domain $x$ above $f$ with the following universal property: for any $g\maps b\to b'$ in $\A$ and $\gamma\maps x\to y'$ in $\X$ above the composite $g\circ f$, there exists a unique $\delta\maps y\to y'$ such that $U(\delta)=g$ and $\gamma=\delta\circ\beta$ as shown below
\begin{displaymath}
\xymatrix @R=.1in @C=.6in
{&& y'\ar @{.>}@/_/[dd] &&\\
x\ar[r]_-{\beta} \ar @{.>}@/_/[dd]
\ar[urr]^-{\gamma} & 
y \ar @{.>}@/_/[dd] \ar @{-->}[ur]_-{\exists! \delta}
&& \textrm{in }\X\\
&& b' &&\\
a\ar[r]_-{f=U\beta} \ar[urr]^-{g\circ f=U\gamma}
 & b \ar[ur]_-g && \textrm{in }\A}
\end{displaymath}
The category $\X$ is called the \textbf{total} category and $\A$ is called the \textbf{base} category of the opfibration. For any $a\in\A$, the \textbf{fibre category} $\X_a$ consists of all objects that map to $a$ and vertical morphisms between them, i.e. mapping to $1_a$.
Assuming the axiom of choice, we may select a cocartesian arrow over each $f\maps a\to b$ in $\A$ and $x\in\X _a$, denoted by $\mathrm{Cocart}(f,x)\maps x\to f_!(x)$, rendering $U$ a so-called \textbf{cloven} opfibration. This choice induces \textbf{reindexing functors} $f_!\maps\X _a\to\X _b$ between the fibre categories, which by the liftings universal property adhere to natural isomorphisms $(1_a)_!\cong1_{\X _a}$ and $(f\circ g)_!\cong f_!\circ g_!$. If these isomorphisms are equalities, we have the notion of a \textbf{split} opfibration. We chose to primarily present this dual notion to \define{fibration} due to the current work's needs.
%Christina: decide later if needed
%Notice that as a result,
%any arrow in the total category of an opfibration factorizes uniquely into a vertical morphism (above the identity) followed by a (co)cartesian one:
%\[
%\xymatrix @C=.4in @R=.2in
%{x \ar @{.>}[dd]\ar[rr]^\gamma \ar[drr]_-{\mathrm{Cocart}(g,C)} && D &\\
%&& f_!C \ar @{-->}[u]_-{\delta} \ar @{.>}[d] & \textrm{in }\C  \\
%X\ar[rr]_-{g} && Y & \textrm{in }\X.}
%\]

Let $\OpFib(\A)$ denote the 2-subcategory of the slice 2-category $\Cat/ \A$ of opfibrations over $\A$, functors that preserve cocartesian liftings and natural transformations with vertical components. In fact, there is a 2-equivalence between opfibrations and pseudofunctors induced by the so-called \emph{Grothendieck construction}. 
\begin{defn}\label{def:GrothCat}
For any pseudofunctor $F\maps\A\to\bicat{Cat}$ where $\A$ is a category viewed as a 2-category with trivial 2-cells, the \textbf{Grothendieck category}
$\inta F$ has
\begin{itemize}
\item objects pairs $(a, x \in F(a))$ and
\item a morphism from $(a, x \in F(a))$ to $(b, y\in F(b))$ is a pair $(f \maps a \to b,\delta \maps F(f)(x) \to y)\in\A\times F(b)$.
\end{itemize}
This is an opfibred category over $\A$ via the obvious forgetful functor, with fibre categories $(\inta F)_a=F(a)$ and reindexing functors $f_!=F(f)$.
\end{defn}
The above in fact provides the one direction of the following well-known equivalence.

\begin{thm}\label{thm:Grothendieck}\hfill
    \begin{enumerate}
        \item Every opfibration $\U \maps \X \to \A$ gives rise to a pseudofunctor $\A \to \bicat{Cat}$.
        \item Every pseudofunctor $F \maps \A \to \bicat{Cat}$ gives rise to  an opfibration $\maps \inta F\to\A$.
        \item The above correspondences yield an equivalence of 2-categories 
        \begin{displaymath}
            [\A,\bicat{Cat}]_\pse \simeq \OpFib(\A)
        \end{displaymath}
%        \item The above 2-equivalence extends to one between 2-categories of arbitrary-base fibrations and arbitrary-domain indexed categories
%        \begin{equation}\label{eq:Gr_equiv}
%        \ICat\simeq\Fib    
%        \end{equation}
    \end{enumerate}
\end{thm}



%\begin{defn}
%Let $P \maps \mathbf{D} \to \mathrm{A}$ be a functor. A morphism $f \maps d_1 \to d_{2}$ in the category $\mathbf{D}$ is \textbf{cartesian (with respect to the functor P)} if for any object $d'$ in $\mathbf{D}$ and morphism $g \maps d' \to d_{2}$ and every $p \maps P(d') \to P(d_1)$ such that $P(g)=P(f)p$, there exists a unique $h \maps d' \to d_1$ such that $g=fh$ and $p=P(h)$.
%\[
%\begin{tikzpicture}[scale=1.5]
%\node (A) at (0,0) {$d'$};
%\node (B) at (0,-1.5) {$d_1$};
%\node (C) at (1.5,-1.5) {$d_{2}$};
%\node (H) at (2,-0.5) {$P$};
%\node (D) at (2,-.75) {$\mapsto$};
%\node (E) at (3,0) {$P(d')$};
%\node (F) at (3,-1.5) {$P(d_1)$};
%\node (G) at (4.5,-1.5) {$P(d_{2})$};
%\path[->,font=\scriptsize,>=angle 90]
%(A) edge[dashed] node[above,left]{$\exists ! h$} (B)
%(A) edge node[above]{$g$} (C)
%(E) edge node[above,left]{$p=P(h)$} (F)
%(F) edge node[above]{$P(f)$} (G)
%(E) edge node[above,right]{$P(g)$} (G)
%(B)edge node[above]{$f$}(C);
%\end{tikzpicture}
%\]
%\end{defn}




\subsection{Double categories}\label{sec:doublecats}

Before formally defining `pseudo double category', it is helpful to have the following picture in mind. A pseudo double category has 2-morphisms shaped like:

\[
\begin{tikzpicture}[scale=1]
\node (D) at (-4,0.5) {$A$};
\node (E) at (-2,0.5) {$B$};
\node (F) at (-4,-1) {$C$};
\node (A) at (-2,-1) {$D$};
\node (B) at (-3,-0.25) {$\Downarrow a$};
\path[->,font=\scriptsize,>=angle 90]
(D) edge node [above]{$M$}(E)
(E) edge node [right]{$g$}(A)
(D) edge node [left]{$f$}(F)
(F) edge node [above]{$N$} (A);
\end{tikzpicture}
\]

We call $A, B, C$ and $D$ \textbf{objects} or \textbf{0-cells}, $f$ and $g$ \textbf{vertical 1-morphisms}, $M$ and $N$ \textbf{horizontal 1-cells} and $a$ a \textbf{2-morphism}. Note that a vertical 1-morphism is a morphism between 0-cells and a 2-morphism is a morphism between horizontal 1-cells. We will denote both kinds of morphisms and horizontal 1-cells as a single arrow, namely `$\to$'. We follow the notation of Shulman \cite{Shulman2008} with the following definitions.

\begin{defn}\label{defn:double_category}
A \textbf{pseudo double category} $\lD$, or $\textbf{double category}$ for short, consists of a category of objects $\mathbf{D_{0}}$ and a category of arrows $\mathbf{D_1}$ with the following functors
\begin{center}
$U\maps \mathbf{D_{0}} \to \mathbf{D_1}$\\
$S,T \maps \mathbf{D_1} \rightrightarrows \mathbf{D_{0}}$\\
$\odot \maps \mathbf{D_1} \times_{\mathbf{D_{0}}} \mathbf{D_1} \to \mathbf{D_1}$ (where the pullback is taken over $\mathbf{D_1} \xrightarrow[]{T} \mathbf{D_{0}} \xleftarrow[]{S} \mathbf{D_1}$) \\
\end{center}
 such that \\
\begin{center}
$S(U_{A})=A=T(U_{A})$\\
$S(M \odot N)=SN$\\
$T(M \odot N)=TM$\\
\end{center}
equipped with natural isomorphisms
\begin{center}

$\alpha \maps (M \odot N) \odot P \xrightarrow{\sim} M \odot (N \odot P)$\\
$\lambda \maps U_{B} \odot M \xrightarrow{\sim} M$\\
$\rho \maps M \odot U_{A} \xrightarrow{\sim} M$

\end{center}
such that $S(\alpha), S(\lambda), S(\rho), T(\alpha), T(\lambda)$ and $T(\rho)$ are all identities and that the coherence axioms of a monoidal category are satisfied. Following the notation of Shulman, objects of $\mathbf{D_{0}}$ are called $\textbf{0-cells}$ and morphisms of $\mathbf{D_{0}}$ are called $\textbf{vertical 1-morphisms}$. Objects of $\mathbf{D_1}$ are called $\textbf{horizontal 1-cells}$ and morphisms of $\mathbf{D_1}$ are called $\textbf{2-morphisms}$. The morphisms of $\mathbf{D_{0}}$, which are vertical 1-morphisms, will be denoted $f \maps A \to C$ and we denote a 1-cell $M$ with $S(M)=A,T(M)=B$ by $M \maps A \to B$. Then a 2-morphism $a \maps M \to N$ of $\mathbf{D_1}$ with $S(a)=f,T(a)=g$ would look like:
\[
\begin{tikzpicture}[scale=1]
\node (D) at (-4,0.5) {$A$};
\node (E) at (-2,0.5) {$B$};
\node (F) at (-4,-1) {$C$};
\node (A) at (-2,-1) {$D$};
\node (B) at (-3,-0.25) {$\Downarrow a$};
\path[->,font=\scriptsize,>=angle 90]
(D) edge node [above]{$M$}(E)
(E) edge node [right]{$g$}(A)
(D) edge node [left]{$f$}(F)
(F) edge node [above]{$N$} (A);
\end{tikzpicture}
\]
\end{defn}

The key difference between a `strict' double category and a pseudo double category is that in a pseudo double category, horizontal composition is associative and unital only up to natural isomorphism. Equivalently, as a double category can be viewed as a category internal to $\Cat$, we can view a pseudo double category as a category `weakly' internal to $\Cat$. We will sometimes omit the word pseudo and simply say double category.

\begin{defn}
A 2-morphism where $f$ and $g$ are identities is called a \textbf{globular 2-morphism}.
\end{defn}

\begin{defn}
Let $\lD$ be a pseudo double category. Then the $\textbf{horizontal bicategory}$ of $\lD$, which we denote as $H(\lD)$, is the bicategory consisting of objects of $\lD$, morphisms that are horizontal 1-cells of $\lD$ and 2-morphisms that are globular 2-morphisms of $\lD$.
\end{defn}

\begin{defn}
Let $\mathbb{A}$ and $\mathbb{B}$ be pseudo double categories. A \define{lax double functor} is a functor $\mathbb{F} \maps\mathbb{A}\to \mathbb{B}$ that takes items of $\mathbb{A}$ to items of $\mathbb{B}$ of the corresponding type, respecting vertical composition in the strict sense and the horizontal composition up to an assigned comparison $\phi$. This means that we have functors $\mathbb{F}_0 \maps \mathbb{A}_0 \to \mathbb{B}_0$ and $\mathbb{F}_1 \maps \mathbb{A}_1 \to \mathbb{B}_1$ such that the following equations are satisfied: $$S \circ \mathbb{F}_1 = \mathbb{F}_0 \circ S$$ $$T \circ \mathbb{F}_1 = \mathbb{F}_0 \circ T$$ Sometimes for brevity, we will omit the subscripts and simply say $\mathbb{F}$. As to whether we mean $\mathbb{F}_0$ or $\mathbb{F}_1$ will be clear from context.

Also, every object $a$ is equipped with a special globular 2-morphism $\phi_{a} \maps 1_{\mathbb{F}(a)} \to \mathbb{F}(1_{a})$ (the identity comparison), and every horizontal composition $N_1 \odot N_{2}$ is equipped with a special globular 2-morphism $\phi(N_1,N_{2}) \maps \mathbb{F}(N_1) \odot \mathbb{F}(N_{2}) \to \mathbb{F}(N_1 \odot N_{2})$ (the composition comparison), in a coherent way. This means that the following diagrams commute.

\begin{enumerate}

\item For a horizontal composite, $\beta \odot \alpha$,


\begin{equation}\label{eq:square}
  \xymatrix@-.5pc{
    \mathbb{F}(A) \ar[r]|{|}^{\mathbb{F}(N_{2})}  \ar[d] \ar@{}[dr]|{\mathbb{F}(\alpha)}&
    \mathbb{F}(B) \ar[d] \ar[r]|{|}^{\mathbb{F}(N_1)} \ar@{}[dr]|{\mathbb{F}(\beta)}&
    \mathbb{F}(C) \ar[d] &
     &
    \mathbb{F}(A) \ar[r]|{|}^{\mathbb{F}(N_{2})} \ar@{}[drr]|{\phi(N_1,N_{2})} \ar[d]_1 &
    \mathbb{F}(B) \ar[r]|{|}^{\mathbb{F}(N_1)} &
    \mathbb{F}(C) \ar[d]^{1} \\
    \mathbb{F}(A') \ar[r]|{|}_{\mathbb{F}(N_{4})} \ar@{}[drr]|{\phi(N_{3},N_{4})} \ar[d]_1&
    \mathbb{F}(B') \ar[r]|{|}_{\mathbb{F}(N_{3})} &
    \mathbb{F}(C') \ar[d]^{1}&
    = &
    \mathbb{F}(A) \ar[rr]|{|}^{\mathbb{F}(N_1 \odot N_{2})} \ar[d] \ar@{}[drr]|{\mathbb{F}(\beta \odot \alpha)}&
     &
    \mathbb{F}(C) \ar[d] \\
    \mathbb{F}(A') \ar[rr]|{|}_{\mathbb{F}(N_{3} \odot N_{4})} & 
     & 
    \mathbb{F}(C') &
     &
    \mathbb{F}(A') \ar[rr]|{|}_{\mathbb{F}(N_{3} \odot N_{4})} &
     &
    \mathbb{F}(C') \\
  }.
\end{equation}

\item For a horizontal 1-cell $N \maps A \to B$, the following diagrams are commutative (under horizontal composition).

\[
\begin{tikzpicture}[scale=1.5]
\node (A) at (1,1) {$\mathbb{F}(N) \odot 1_{\mathbb{F}(A)}$};
\node (C) at (3,1) {$\mathbb{F}(N)$};
\node (A') at (1,-1) {$\mathbb{F}(N) \odot \mathbb{F}(1_{A})$};
\node (C') at (3,-1) {$\mathbb{F}(N \odot 1_{A})$};
\node (B) at (5,1) {$1_{\mathbb{F}(B)} \odot \mathbb{F}(N)$};
\node (B') at (5,-1) {$\mathbb{F}(1_{B}) \odot \mathbb{F}(N)$};
\node (D) at (7,1) {$\mathbb{F}(N)$};
\node (D') at (7,-1) {$\mathbb{F}(1_{B} \odot N)$};
\path[->,font=\scriptsize,>=angle 90]
(A) edge node[left]{$1 \odot \phi_{A}$} (A')
(C') edge node[right]{$\mathbb{F} \rho$} (C)
(A) edge node[above]{$\rho_{\mathbb{F}(N)}$} (C)
(A') edge node[above]{$\phi(N,1_{A})$} (C')
(B) edge node[left]{$\phi_{B} \odot 1$} (B')
(B') edge node[above]{$\phi(1_{B},N)$} (D')
(B) edge node[above]{$\lambda_{\mathbb{F}(N)}$} (D)
(D') edge node[right]{$F \lambda$} (D);
\end{tikzpicture}
\]

\item For consecutive horizontal 1-cells $N_1,N_{2}$ and $N_{3}$, the following diagram is commutative.

 \[\xymatrix{
    (\mathbb{F}(N_1) \odot \mathbb{F}(N_{2})) \odot \mathbb{F}(N_{3}) \ar[r]^{a'}\ar[d]_{\phi(N_1,N_{2}) \odot 1}
    & \mathbb{F}(N_1) \odot (\mathbb{F}(N_{2}) \odot \mathbb{F}(N_{3})) \ar[d]^{1 \odot \phi(N_{2},N_{3})}\\
    \mathbb{F}(N_1 \odot N_{2}) \odot \mathbb{F}(N_{3}) \ar[d]_{\phi(N_1 \odot N_{2},N_{3})} &
    \mathbb{F}(N_1) \odot \mathbb{F}(N_{2} \odot N_{3}) \ar[d]^{\phi(N_1,N_{2} \odot N_{3})}\\
    \mathbb{F}((N_1 \odot N_{2}) \odot N_{3})\ar[r]^{\mathbb{F}a} &
    \mathbb{F}(N_1 \odot (N_{2} \odot N_{3}))}\]
\end{enumerate}
We say the double functor $\mathbb{F}$ is \define{strict} if the comparison constraints $\phi_a$ and $\phi_{N_1,N_2}$ are identities, \define{pseudo} if the comparison constraints are isomorphisms, and \define{oplax} if the comparison constraints go in the opposite direction.
\end{defn}

%ChristinaL Pretty sure the def below can be incorporated above, however got a bit lost in different notation. Kenny Fix?
\begin{defn}
A double functor $\mathbb{F} \maps \mathbb{A} \to \mathbb{X}$ is \define{strong} if the comparison and unit constraints are globular isomorphisms, meaning that for each composable pair of horizontal 1-cells $M$ and $N$ we have a natural isomorphism $$\mathbb{F}_{M,N} \maps \mathbb{F}(M) \odot \mathbb{F}(N) \xrightarrow{\sim} \mathbb{F}(M \odot N)$$and for each object $a \in \mathbb{A}$ a natural isomorphism $$\mathbb{F}_a \maps \hat{U}_{\mathbb{F}(a)} \xrightarrow{\sim} \mathbb{F}(U_a).$$
\end{defn}


Following the notation of Shulman \cite{Shulman2010}, given a double category $\mathbb{A}$, we write $_f \mathbb{A}_g(M,N)$ for the set of 2-morphisms in $\mathbb{A}$ of the form:
\[
  \xymatrix@-.5pc{
    A \ar[r]|{|}^{M}  \ar[d]_f \ar@{}[dr]|{\Downarrow a}&
    B\ar[d]^g\\
    C \ar[r]|{|}_N & D
  }
\]
We call $M$ and $N$ the \define{horizontal source and target} of the 2-morphism $a$, respectively, and likewise we call $f$ and $g$ the \define{vertical source and target} of the 2-morphism $a$, respectively. Thus $_f \mathbb{A}_g(M,N)$ denotes the set of 2-morphisms in $\mathbb{A}$ with horizontal source and target $M$ and $N$ and vertical source and target $f$ and $g$.
\begin{defn}\label{def:fullfaithful}
A (possibly lax or oplax) double functor $\mathbb{F} \maps \mathbb{A} \to \mathbb{X}$ is \define{full} (respectively, \define{faithful}) if $\mathbb{F}_0 \maps \mathbb{A}_0 \to \mathbb{X}_0$ is full (respectively, faithful) and each map $$\mathbb{F}_1 \maps _f \mathbb{A}_g(M,N) \to _{\mathbb{F}(f)} \mathbb{X}_{\mathbb{F}(g)}(\mathbb{F}(M),\mathbb{F}(N))$$ is surjective (respectively, injective).
\end{defn}
\begin{defn}\label{def:essentiallysurj}
A (possibly lax or oplax) double functor $\mathbb{F} \maps \mathbb{A} \to \mathbb{X}$ is \define{essentially surjective} if we can simultaneously make the following choices:
\begin{enumerate}
\item{For each object $x \in \mathbb{X}$, we can find an object $a \in \mathbb{A}$ together with a vertical 1-isomorphism $\alpha_x \maps \mathbb{F}(a) \to x$, and}
\item{For each horizontal 1-cell $N \maps x_1 \tobar x_2$  of $\mathbb{X}$, we can find a horizontal 1-cell $M \maps a_1 \tobar a_2$ of $\mathbb{A}$ and a 2-isomorphism $a_{N}$ of $\mathbb{X}$ as in the following diagram:
\[
  \xymatrix@-.5pc{
    \mathbb{F}(a_1) \ar[r]|{|}^{\mathbb{F}(M)}  \ar[d]_{\alpha_{x_1}} \ar@{}[dr]|{\Downarrow a_N}&
    \mathbb{F}(a_1) \ar[d]^{\alpha_{x_2}}\\
    x_1 \ar[r]|{|}_N & x_2
  }
\]
}
\end{enumerate}
\end{defn}

Formally, two strong double functors between two double categories form a \emph{double equivalence} if ...
%{\chris their two-way composite is isomorphic to the identity, better expressed, possibly in a definition environment}. 
Similarly to ordinary equivalence of categories, [Shulman,7.8] allows us to use the following equivalent characterization.

\begin{thm}\label{ShulDubEquiv}
A strong double functor $\mathbb{F} \maps \mathbb{A} \to \mathbb{X}$ is part of a double equivalence if and only if it is full, faithful and essentially surjective. {\chris on objects?}
\end{thm}

\begin{prop}
Let $\mathbb{A}$ and $\mathbb{X}$ be symmetric monoidal double categories and let $\mathbb{F} \maps \mathbb{A} \to \mathbb{X}$ be a symmetric monoidal strong double functor. If $\mathbb{F}$ is part of a double equivalence, then $\mathbb{F}$ is in fact part of a symmetric monoidal double equivalence, and $\mathbb{A}$ and $\mathbb{X}$ are equivalent as symmetric monoidal double categories.
\end{prop}


\begin{defn}\label{defn:smdc}
  A \textbf{monoidal double category} is a double category equipped the following
structure.
\begin{enumerate}
\item $\mathbf{D_{0}}$ and $\mathbf{D_1}$ are both monoidal categories.
\item If $I$ is the monoidal unit of $\mathbf{D_{0}}$, then $U_I$ is the
  monoidal unit of $\mathbf{D_1}$.
\item The functors $S$ and $T$ are strict monoidal, i.e.\ $S(M\ten N)
  = SM\ten SN$ and $T(M\ten N)=TM\ten TN$ and $S$ and $T$ also
  preserve the associativity and unit constraints.
\item We have globular isomorphisms
  \[\chi \maps (M_1\ten N_1)\odot (M_2\ten N_2)\too[\sim] (M_1\odot M_2)\ten (N_1\odot N_2)\]
  and
  \[\mu\maps U_{A\ten B} \too[\sim] (U_A \ten U_B)\]
  such that the following diagrams commute:
		\item \label{diag:MonDblCat}
			The following diagrams commute expressing the constraint data for the double functor $\otimes$.
			\[
			\begin{tikzpicture}
				\node (A) at (0,3) {\footnotesize{
							$((M_1\otimes N_1)\odot (M_2\otimes N_2)) \odot (M_3\otimes N_3)$}
				};
				\node (B) at (7,3) {\footnotesize{
						$((M_1\odot M_2)\otimes (N_1\odot N_2)) \odot (M_3\otimes N_3) $}
				};
				\node (A') at (0,1.5) {\footnotesize{
						$(M_1\otimes N_1)\odot ((M_2\otimes N_2) \odot (M_3\otimes N_3)) $}
				};
				\node (B') at (7,1.5) {\footnotesize{
						$((M_1\odot M_2)\odot M_3) \otimes ((N_1\odot N_2)\odot N_3)$}
				};
				\node (A'') at (0,0) {\footnotesize{
						$(M_1\otimes N_1) \odot ((M_2\odot M_3) \otimes (N_2\odot N_3))$}
				};
				\node (B'') at (7,0) {\footnotesize{
						$(M_1\odot (M_2\odot M_3)) \otimes (N_1\odot (N_2\odot N_3))$}
				};
			%
			\path[->,font=\scriptsize]
				(A) edge node[left]{$\alpha$} (A')
				(A') edge node[left]{$1 \odot \chi$} (A'')
				(B) edge node[right]{$\chi$} (B')
				(B') edge node[right]{$\alpha \otimes \alpha$} (B'')
				(A) edge node[above]{$\chi \odot 1$} (B)
				(A'') edge node[above]{$\chi$} (B'');
		\end{tikzpicture}
		\]
		\[
		\begin{tikzpicture}
			\node (UL) at (0,1.5) {\footnotesize{
					$(M\otimes N) \odot U_{C\otimes D}$}
			};
			\node (LL) at (0,0) {\footnotesize{
					$M\otimes N$}
			};
			\node (UR) at (3.5,1.5) {\footnotesize{
					$(M\otimes N)\odot (U_C\otimes U_D)$}
			};
			\node (LR) at (3.5,0) {\footnotesize{
					$(M\odot U_C) \otimes (N\odot U_D)$}
			};
			%
			\path[->,font=\scriptsize]
				(UL) edge node[above]{$1 \odot \mu$} (UR) 
				(UL) edge node[left]{$\rho$} (LL)
				(LR) edge node[above]{$\rho \otimes \rho$} (LL)
				(UR) edge node[right]{$\chi$} (LR);
		\end{tikzpicture}
		%
		\quad
		%
		\begin{tikzpicture}
			\node (UL) at (0,1.5) {\scriptsize{$U_{A\otimes B}\odot (M\otimes N)$}};
			\node (LL) at (0,0) {\scriptsize{$M\otimes N$}};
			\node (UR) at (3.5,1.5) {\scriptsize{$(U_A\otimes U_B)\odot (M\otimes N)$}};
			\node (LR) at (3.5,0) {\scriptsize{$(U_A \odot M) \otimes (U_B\odot N)$}};
			%
			\path[->,font=\scriptsize]
				(UL) edge node[above]{$\chi \odot 1$} (UR) 
				(UL) edge node[left]{$\lambda$} (LL)
				(LR) edge node[above]{$\lambda \otimes \lambda$} (LL)
				(UR) edge node[right]{$\chi$} (LR);
		\end{tikzpicture}
		\]
		%
		\item The following diagrams commute expressing 
		the associativity isomorphism for $\otimes$ is a transformation of double categories.
		\[
		\begin{tikzpicture}
			\node (A) at (0,3) {\footnotesize{
					$((M_1\otimes N_1)\otimes P_1) \odot ((M_2\otimes N_2)\otimes P_2)$}
			};
			\node (B) at (7,3) {\footnotesize{
					$(M_1\otimes (N_1\otimes P_1)) \odot (M_2\otimes (N_2\otimes P_2))$}
			};
			\node (A') at (0,1.5) {\footnotesize{
					$((M_1\otimes N_1) \odot (M_2\otimes N_2)) \otimes (P_1\odot P_2)$}
			};
			\node (B') at (7,1.5) {\footnotesize{
					$(M_1\odot M_2) \otimes ((N_1\otimes P_1)\odot (N_2\otimes P_2))$}
			};
			\node (A'') at (0,0) {\footnotesize{
					$((M_1\odot M_2) \otimes(N_1\odot N_2)) \otimes (P_1\odot P_2)$}
			};
			\node (B'') at (7,0) {\footnotesize{
					$(M_1\odot M_2) \otimes ((N_1\odot N_2)\otimes (P_1\odot P_2))$}
			};
			%
			\path[->,font=\scriptsize]
				(A) edge node[left]{$\chi$} (A')
				(A') edge node[left]{$\chi \otimes 1$} (A'')
				(B) edge node[right]{$\chi$} (B')
				(B') edge node[right]{$1 \otimes \chi$} (B'')
				(A) edge node[above]{$\alpha \odot \alpha$} (B)
				(A'') edge node[above]{$\alpha$} (B'');
		\end{tikzpicture}
		\]
		\[
		\begin{tikzpicture}
			\node (A) at (0,3) {\footnotesize{$U_{(A\otimes B)\otimes C}$}};
			\node (B) at (4,3) {\footnotesize{$U_{A\otimes (B\otimes C)} $}};
			\node (A') at (0,1.5) {\footnotesize{$U_{A\otimes B} \otimes U_C $}};
			\node (B') at (4,1.5) {\footnotesize{$U_A\otimes U_{B\otimes C}$}};
			\node (A'') at (0,0) {\footnotesize{$(U_A\otimes U_B)\otimes U_C$}};
			\node (B'') at (4,0) {\footnotesize{$U_A\otimes (U_B\otimes U_C) $}};
			%
			\path[->,font=\scriptsize]
				(A) edge node[left]{$\mu$} (A')
				(A') edge node[left]{$\mu \otimes 1$} (A'')
				(B) edge node[right]{$\mu$} (B')
				(B') edge node[right]{$1 \otimes \mu$} (B'')
				(A) edge node[above]{$U_{\alpha}$} (B)
				(A'') edge node[above]{$\alpha$} (B'');
		\end{tikzpicture}
		\]
		\item The following diagrams commute expressing that 
		the unit isomorphisms for $\otimes$ are transformations of double categories. 
		\[
		\begin{tikzpicture}
			\node (A) at (0,1.5) {\footnotesize{$(M\otimes U_I)\odot (N\otimes U_I)$}};
			\node (A') at (0,0) {\footnotesize{$M\odot N $}};
			\node (B) at (4,1.5) {\footnotesize{$(M\odot N)\otimes (U_I \odot U_I) $}};
			\node (B') at (4,0) {\footnotesize{$(M\odot N)\otimes U_I $}};
			%
			\path[->,font=\scriptsize]
				(A) edge node[left]{$r \odot r$} (A')
				(A) edge node[above]{$\chi$} (B)
				(B) edge node[right]{$1 \otimes \rho$} (B')
				(B') edge node[above]{$r$} (A');
		\end{tikzpicture}
		%
		\quad
		%
		\begin{tikzpicture}
			\node (A) at (0,0.75) {\footnotesize{$U_{A\otimes I} $}};
			\node (B) at (1.5,1.5) {\footnotesize{$U_A\otimes U_I $}};
			\node (B') at (1.5,0) {\footnotesize{$U_A$}};
			%
			\path[->,font=\scriptsize]
				(A) edge node[above]{$\mu$} (B)
				(A) edge node[below]{$U_{r}$} (B')
				(B) edge node[right]{$r$} (B');
		\end{tikzpicture}
		\]
		%
		%
		%
		%
		\[
		\begin{tikzpicture}
			\node (A) at (0,1.5) {\footnotesize{$(U_I\otimes M)\odot (U_I\otimes N)$}};
			\node (A') at (0,0) {\footnotesize{$M\odot N$}};
			\node (B) at (4,1.5) {\footnotesize{$(U_I \odot U_I) \otimes (M\odot N)$}};
			\node (B') at (4,0) {\footnotesize{$U_I\otimes (M\odot N) $}};
			%
			\path[->,font=\scriptsize]
				(A) edge node[left]{$\ell \odot \ell$} (A')
				(A) edge node[above]{$\chi$} (B)
				(B) edge node[right]{$\lambda \otimes 1$} (B')
				(B') edge node[above]{$\ell$} (A');
		\end{tikzpicture}
		%
		\quad
		\begin{tikzpicture}
			\node (A) at (0,0.75) {\footnotesize{$U_{I\otimes A}$}};
			\node (B) at (1.5,1.5) {\footnotesize{$U_I\otimes U_A$}};
			\node (B') at (1.5,0) {\footnotesize{$U_A$}};
			%
			\path[->,font=\scriptsize]
				(A) edge node[above]{$\mu$} (B)
				(A) edge node[below]{$U_{\ell}$} (B')
				(B) edge node[right]{$\ell$} (B');
		\end{tikzpicture}
		\]
		\newcounter{mondbl}
		\setcounter{mondbl}{\value{enumi}}
	\end{enumerate}
	A \textbf{braided monoidal double category} 
	is a monoidal double category 
	such that:
	\begin{enumerate}
		\setcounter{enumi}{\value{mondbl}}
		\item $\dblcat{D}_{0}$ and $\dblcat{D}_1$ are braided monoidal categories.
		\item The functors $S$ and $T$ are strict braided monoidal functors.
		\item The following diagrams commute expressing that the braiding is a transformation of double categories.
		\[
		\begin{tikzpicture}
			\node (A) at (0,1.5) {\footnotesize{$(M_1 \odot M_2) \otimes (N_1 \odot N_2)$}};
			\node (A') at (0,0) {\footnotesize{$(M_1\otimes N_1) \odot (M_2\otimes N_2)$}};
			\node (B) at (5,1.5) {\footnotesize{$(N_1\odot N_2) \otimes (M_1 \odot M_2)$}};
			\node (B') at (5,0) {\footnotesize{$(N_1 \otimes M_1) \odot (N_2 \otimes M_2)$}};
			%
			\path[->,font=\scriptsize]
				(A) edge node[left]{$\chi$} (A')
				(A) edge node[above]{$\beta$} (B)
				(B) edge node[right]{$\chi$} (B')
				(A') edge node[above]{$\beta \odot \beta$} (B');
		\end{tikzpicture}
		%
		\quad
		%
		\begin{tikzpicture}
			\node (A) at (0,1.5) {\footnotesize{$U_A \otimes U_B$}};
			\node (A') at (0,0) {\footnotesize{$U_B\otimes U_A$}};
			\node (B) at (2,1.5) {\footnotesize{$U_{A\otimes B} $}};
			\node (B') at (2,0) {\footnotesize{$U_{B\otimes A}$}};
			%
			\path[->,font=\scriptsize]
				(A) edge node[left]{$\beta$} (A')
				(B) edge node[above]{$\mu$} (A)
				(B) edge node[right]{$U_\beta$} (B')
				(B') edge node[above]{$\mu$} (A');
		\end{tikzpicture}
		\]
		\setcounter{mondbl}{\value{enumi}}
	\end{enumerate}
	Finally, a \textbf{symmetric monoidal double category} 
	is a braided monoidal double category $\mathbb{D}$ such that:
	\begin{enumerate}
		\setcounter{enumi}{\value{mondbl}}
		\item $\dblcat{D}_{0}$ and $\dblcat{D}_1$ are symmetric monoidal.
	\end{enumerate}
\end{defn}


\begin{defn}\label{def:companion}
  Let $\lD$ be a double category and $f\maps A\to B$ a vertical
  1-morphism.  A \textbf{companion} of $f$ is a horizontal 1-cell
  $\fhat\maps A\to B$ together with 2-morphisms
	\[
	\raisebox{-0.5\height}{
	\begin{tikzpicture}
		\node (A) at (0,1) {$A$};
		\node (B) at (1,1) {$B$};
		\node (A') at (0,0) {$B$};
		\node (B') at (1,0) {$B$};
		%
		\path[->,font=\scriptsize,>=angle 90]
			(A) edge node[above]{$\widehat{f}$} (B)
			(A) edge node[left]{$f$} (A')
			(B) edge node[right]{$1$} (B')
			(A') edge node[below]{$U_B$} (B');
		%
	%	\draw (0.5,.925) -- (0.5,1.075);
	%	\draw (0.5,-.075) -- (0.5,.075);
		\node () at (0.5,0.5) {\scriptsize{$\Downarrow$}};
	\end{tikzpicture}
	}
	%
	\quad \text{ and } \quad
	%
	\raisebox{-0.5\height}{
	\begin{tikzpicture}
		\node (A) at (0,1) {$A$};
		\node (B) at (1,1) {$A$};
		\node (A') at (0,0) {$A$};
		\node (B') at (1,0) {$B$};
		%
		\path[->,font=\scriptsize,>=angle 90]
			(A) edge node[above]{$U_A$} (B)
			(A) edge node[left]{$1$} (A')
			(B) edge node[right]{$f$} (B')
			(A') edge node[below]{ $\widehat{f}$} (B');
		%
	%	\draw (0.5,.925) -- (0.5,1.075);
	%	\draw (0.5,-.075) -- (0.5,.075);
		\node () at (0.5,0.5) {\scriptsize{$\Downarrow$}};
	\end{tikzpicture}
	}
	\]
  such that the following equations hold.
	\begin{equation}
	\label{eq:CompanionEq}
	\raisebox{-0.5\height}{
	\begin{tikzpicture}
		\node (A) at (0,2) {$A$};
		\node (B) at (1.1,2) {$A$};
		\node (A') at (0,1) {$A$};
		\node (B') at (1.1,1) {$B$};
		\node (A'') at (0,0) {$B$};
		\node (B'') at (1.1,0) {$B$};
		%
		\path[->,font=\scriptsize,>=angle 90]
			(A) edge node[left]{$1$} (A')
			(A') edge node[left]{$f$} (A'')
			(B) edge node[right]{$f$} (B')
			(B') edge node[right]{$1$} (B'')
			(A) edge node[above]{$U_A$} (B)
			(A') edge  (B')
			(A'') edge node[below]{$U_B$} (B'');
		%
	%	\draw (0.5,1.925) -- (0.5,2.075);
		\draw[line width=2mm,white] (0.5,.925) -- (0.5,1.075);
	%	\draw (0.5,-.075) -- (0.5,.075);
		\node () at (0.5,0.5) {\scriptsize{$\Downarrow$}};
		\node () at (0.5,1.5) {\scriptsize{$\Downarrow$}};
		\node () at (0.5,1) {\scriptsize $\widehat{f}$};
	\end{tikzpicture}
	}
	%
	\raisebox{-0.5\height}{=}
	%
	\raisebox{-0.5\height}{
	\begin{tikzpicture}
		\node (A) at (0,1) {$A$};
		\node (B) at (1,1) {$A$};
		\node (A') at (0,0) {$B$};
		\node (B') at (1,0) {$B$};
		%
		\path[->,font=\scriptsize,>=angle 90]
		(A) edge node[left]{$f$} (A')
		(B) edge node[right]{$f$} (B')
		(A) edge node[above]{$U_A$} (B)
		(A') edge node[below]{$U_B$} (B');
		%
		%\draw (0.5,.925) -- (0.5,1.075);
		%\draw (0.5,-.075) -- (0.5,.075);
		\node () at (0.5,0.5) {\scriptsize{$\Downarrow U_f$}};
	\end{tikzpicture}
	}
	%
	\raisebox{-0.5\height}{\text{   and   }}
	%
	\raisebox{-0.5\height}{
	\begin{tikzpicture}
		\node (A) at (0,1) {$A$};
		\node (A') at (0,0) {$A$};
		\node (B) at (1,1) {$A$};
		\node (B') at (1,0) {$B$};
		\node (C) at (2,1) {$B$};
		\node (C') at (2,0) {$B$};
		%
		\path[->,font=\scriptsize,>=angle 90]
			(A) edge node[left]{$1$} (A')
			(B) edge node[left]{$f$} (B')
			(C) edge node[right]{$1$} (C')
			(A) edge node[above]{$U_A$} (B)
			(B) edge node[above]{$\widehat{f}$} (C)
			(A') edge node[below]{$\widehat{f}$} (B')
			(B') edge node[below]{$U_B$} (C');
		%
	%	\draw (1.5,0.925) -- (1.5,1.075);
	%	\draw (1.5,0.925) -- (1.5,1.075);
	%	\draw (0.5,.925) -- (0.5,1.075);
	%	\draw (0.5,-.075) -- (0.5,.075);
		\node () at (0.5,0.5) {\scriptsize{$\Downarrow$}};
		\node () at (1.5,0.5) {\scriptsize{$\Downarrow$}};
	\end{tikzpicture}
	}
	%
	\raisebox{-0.5\height}{=}
	%
	\raisebox{-0.5\height}{
	\begin{tikzpicture}
		\node (A) at (0,1) {$A$};
		\node (B) at (1,1) {$B$};
		\node (A') at (0,0) {$A$};
		\node (B') at (1,0) {$B$};
		%
		\path[->,font=\scriptsize,>=angle 90]
			(A) edge node[left]{$1$} (A')
			(B) edge node[right]{$1$} (B')
			(A) edge node[above]{$\widehat{f}$} (B)
			(A') edge node[below]{$\widehat{f}$} (B');
		%
	%	\draw (0.5,.925) -- (0.5,1.075);
	%	\draw (0.5,-.075) -- (0.5,.075);
		\node () at (0.5,0.5) {\scriptsize{$\Downarrow \id_{\widehat{f}}$}};
	\end{tikzpicture}
	}
	\end{equation}
  A \textbf{conjoint} of $f$, denoted $\fchk \maps B\to A$, is a
  companion of $f$ in the double category $\lD^{h\cdot\mathrm{op}}$
  obtained by reversing the horizontal 1-cells, but not the vertical
  1-morphisms, of $\lD$.
\end{defn}
\noindent
In a pseudo double category, the second equation above requires an insertion of unit isomorphisms to make sense due to horizontal composition only holding up to isomorphism.

\begin{defn}
\label{defn:fibrant}
We say that a double category is \textbf{fibrant} if every vertical
1-morphism has both a companion and a conjoint and \define{isofibrant} if every vertical 1-isomorphism has both a companion and a conjoint.
\end{defn}


\begin{defn}
\label{defn:monoidal_double_functor}
A \define{(strong) monoidal lax double functor} $\mathbb{F} \maps \mathbb{C} \to \mathbb{D}$ between monoidal double categories $\mathbb{C}$ and $\mathbb{D}$ is a lax double functor $\mathbb{F} \maps \mathbb{C} \to \mathbb{D}$ such that
	\begin{itemize}
		\item $\mathbb{F}_0$ and $\mathbb{F}_1$ are monoidal functors, meaning that there exists

\begin{enumerate}
\item{an isomorphism $\epsilon \maps 1_{\lD} \to \mathbb{F}(1_{\lC})$}
\item{a natural isomorphism $\mu_{A,B} \maps \mathbb{F}(A) \otimes \mathbb{F}(B) \to \mathbb{F}(A \otimes B)$ for all objects $A$ and $B$ of $\lC$}
\item{an isomorphism $\delta \maps U_{1_{\lD}} \to \mathbb{F}(U_{1_{\lC}})$}
\item{a natural isomorphism $\nu_{M,N} \maps \mathbb{F}(M) \otimes \mathbb{F}(N) \to \mathbb{F}(M \otimes N)$ for all horizontal 1-cells $N$ and $M$ of $\lC$}
\end{enumerate}
such that the following diagrams commute: for objects $A,B$ and $C$ of $\lC$,
 \[\xymatrix{
    (\mathbb{F}(A) \otimes \mathbb{F}(B)) \otimes \mathbb{F}(C) \ar[r]^{\alpha'}\ar[d]_{\mu_{A,B} \otimes 1}
    & \mathbb{F}(A) \otimes (\mathbb{F}(B) \otimes \mathbb{F}(C)) \ar[d]^{1 \otimes \mu_{B,C}}\\
    \mathbb{F}(A \otimes B) \otimes \mathbb{F}(C) \ar[d]_{\mu_{A \otimes B,C}} &
    \mathbb{F}(A) \otimes \mathbb{F}(B \otimes C) \ar[d]^{\mu_{A,B \otimes C}}\\
    \mathbb{F}((A \otimes B) \otimes C)\ar[r]^{\mathbb{F}\alpha} &
    \mathbb{F}(A \otimes (B \otimes C))}\]
\[
\begin{tikzpicture}[scale=1.5]
\node (A) at (1,1) {$\mathbb{F}(A) \otimes 1_{\lD}$};
\node (C) at (3,1) {$\mathbb{F}(A)$};
\node (A') at (1,-1) {$\mathbb{F}(A) \otimes \mathbb{F}(1_{\lC})$};
\node (C') at (3,-1) {$\mathbb{F}(A \otimes 1_{\lC})$};
\node (B) at (5,1) {$1_{\lD} \otimes \mathbb{F}(A)$};
\node (B') at (5,-1) {$\mathbb{F}(1_{\lC}) \otimes \mathbb{F}(A)$};
\node (D) at (7,1) {$\mathbb{F}(A)$};
\node (D') at (7,-1) {$\mathbb{F}(1_{\lC} \otimes A)$};
\path[->,font=\scriptsize,>=angle 90]
(A) edge node[left]{$1 \otimes \epsilon$} (A')
(C') edge node[right]{$\mathbb{F}(r_{A})$} (C)
(A) edge node[above]{$r_{\mathbb{F}(A)}$} (C)
(A') edge node[above]{$\mu_{A,1_{\lC}}$} (C')
(B) edge node[left]{$\epsilon \otimes 1$} (B')
(B') edge node[above]{$\mu_{1_{\lC},A}$} (D')
(B) edge node[above]{$\ell_{\mathbb{F}(A)}$} (D)
(D') edge node[right]{$\mathbb{F}(\ell_{A})$} (D);
\end{tikzpicture}
\]
and for horizontal 1-cells $N_1,N_{2}$ and $N_{3}$ of $\lC$,
 \[\xymatrix{
    (\mathbb{F}(N_1) \otimes \mathbb{F}(N_{2})) \otimes \mathbb{F}(N_{3}) \ar[r]^{\alpha'}\ar[d]_{\nu_{N_1,N_{2}} \otimes 1}
    & \mathbb{F}(N_1) \otimes (\mathbb{F}(N_{2}) \otimes \mathbb{F}(N_{3})) \ar[d]^{1 \otimes \nu_{N_{2},N_{3}}}\\
    \mathbb{F}(N_1 \otimes N_{2}) \otimes \mathbb{F}(N_{3}) \ar[d]_{\nu_{N_1 \otimes N_{2},N_{3}}} &
    \mathbb{F}(N_1) \otimes \mathbb{F}(N_{2} \otimes N_{3}) \ar[d]^{\nu_{N_1,N_{2} \otimes N_{3}}}\\
    \mathbb{F}((N_1 \otimes N_{2}) \otimes N_{3})\ar[r]^{\mathbb{F}\alpha} &
    \mathbb{F}(N_1 \otimes (N_{2} \otimes N_{3}))}\]
\[
\begin{tikzpicture}[scale=1.5]
\node (A) at (0,1) {$\mathbb{F}(N_1) \otimes U_{1_{\lD}}$};
\node (C) at (3,1) {$\mathbb{F}(N_1)$};
\node (A') at (0,-1) {$\mathbb{F}(N_1) \otimes \mathbb{F}(U_{1_{\lC}})$};
\node (C') at (3,-1) {$\mathbb{F}(N_1 \otimes U_{1_{\lC}})$};
\node (B) at (5,1) {$U_{1_{\lD}} \otimes \mathbb{F}(N_1)$};
\node (B') at (5,-1) {$\mathbb{F}(U_{1_{\lC}}) \otimes \mathbb{F}(N_1)$};
\node (D) at (8,1) {$\mathbb{F}(N_1)$};
\node (D') at (8,-1) {$\mathbb{F}(U_{1_{\lC}} \otimes N_1)$};
\path[->,font=\scriptsize,>=angle 90]
(A) edge node[left]{$1 \otimes \delta$} (A')
(C') edge node[right]{$\mathbb{F}(r_{N_1})$} (C)
(A) edge node[above]{$r_{\mathbb{F}(N_1)}$} (C)
(A') edge node[above]{$\nu_{N_1,U_{1_{\lC}}}$} (C')
(B) edge node[left]{$\delta \otimes 1$} (B')
(B') edge node[above]{$\nu_{U_{1_{\lC}},N_1}$} (D')
(B) edge node[above]{$\ell_{\mathbb{F}(N_1)}$} (D)
(D') edge node[right]{$\mathbb{F}(\ell_{N_1})$} (D);
\end{tikzpicture}
\]

		\item $S\mathbb{F}_1= \mathbb{F}_0S$ and $T\mathbb{F}_1 = \mathbb{F}_0T$ are equations between monoidal functors, and
		\item the composition and unit comparisons $\phi(N_1,N_2) \maps \mathbb{F}_1(N_1) \odot \mathbb{F}_1(N_2) \to \mathbb{F}_1(N_1\odot N_2)$ and $\phi_A \maps U_{\mathbb{F}_0 (A)} \to \mathbb{F}_1(U_A)$ are monoidal natural transformations.
	\end{itemize}
The monoidal lax double functor is \define{braided} if $\mathbb{F}_0$ and $\mathbb{F}_1$ are braided monoidal functors and \define{symmetric} if they are symmetric monoidal functors, and lax monoidal or oplax monoidal if the isomorphisms and families of natural isomorphisms of items (1)-(4) above are merely morphisms and natural transformations going in the appropriate directions.
\end{defn}

\begin{thebibliography}{100}

\bibitem{BC} J.\ C.\ Baez and K.\ Courser, Coarse-graining open Markov processes, \textsl{Theory Appl.\ Categ.\ }\textbf{33} (2018), 1223--1268.  Available as \href{https://arxiv.org/abs/1710.11343}{arXiv:1710.11343}.

\bibitem{BC2} J.\ C.\ Baez and K.\ Courser, Structured cospans,  \textsl{Theory Appl.\ Categ.\ }\textbf{35} (2020), 1771–1822.   Available as \href{http://arxiv.org/abs/1911.04630}{arXiv:1911.04630}.

\bibitem{BCR} J.\ C.\ Baez, B.\ Coya and F.\ Rebro, Props in circuit theory, \textsl{Theory Appl.\ Categ.\ }\textbf{33} (2018), 727--783.    Available as \href{https://arxiv.org/abs/1707.08321}{arXiv:1707.08321}. 

\bibitem{BF}  J.\ C.\ Baez and B.\ Fong, A compositional framework for passive linear networks, \textsl{Theory Appl.\ Categ.\ }\textbf{33} (2018), 1158--1222.  Available as \href{http://arxiv.org/abs/1504.05625}{arXiv:1504.05625}.

\bibitem{BFP} J.\ C.\ Baez, B.\ Fong and B.\ S.\ Pollard, A compositional framework for Markov processes, \textsl{Jour. Math. Phys.} \textbf{57} (2016), 033301. Available as \href{http://arxiv.org/abs/1508.06448}{arXiv:1508.06448}.

\bibitem{BM}  J.\ C.\ Baez and J.\ Master, Open Petri nets, \textsl{Math.\ Struct.\ Comput.\ Sci.\ }\textbf{30} (2020), 314--341. Available as 
\href{https://arxiv.org/abs/1808.05415}{arXiv:1808.05415}. 

\bibitem{BP} J.\ C.\ Baez and B.\ S.\ Pollard, A compositional framework for chemical reaction networks, \textsl{Rev.\ Math.\ Phys.\ }\textbf{29} (2017), 1750028.  Available as \href{http://arxiv.org/abs/1704.02051}{arXiv:1704.02051}.

\bibitem{Borc} F. \ Borceux, \textsl{Handbook of Categorical Algebra}, vol.\ 2, 
Cambridge University Press, Cambridge, 1994

\bibitem{Brown1} R.\ Brown and C.\ B.\ Spencer, Double groupoids and crossed modules, 
\textsl{Cah.\ Top.\ G\'eom.\ Diff.} \textbf{17} (1976), 343--362.

\bibitem{Brown2} R.\ Brown, K.\ Hardie, H.\ Kamps and T.\ Porter, The homotopy double groupoid of a Hausdorff space, \textsl{Th.\ Appl.\ Categ.} \textbf{10} (2002), 71--93.

%\bibitem{Be} J.\ B\'enabou, Introduction to bicategories, in {\sl Reports
%of the Midwest Category Seminar}, Lecture Notes in Mathematics, vol.\ \textbf{47}, Springer, Berlin, 1967, pp.\ 1--77.

%\bibitem{Brown1} R.\ Brown and C.\ B.\ Spencer, Double groupoids and crossed modules, 
%\textsl{Cah.\ Top.\ G\'eom.\ Diff.} \textbf{17} (1976), 343--362.

%\bibitem{Brown2} R.\ Brown, K.\ Hardie, H.\ Kamps and T.\ Porter, The homotopy double groupoid of a Hausdorff space, \textsl{Th.\ Appl.\ Categ.} \textbf{10} (2002), 71--93. 

%\bibitem{CC} D.\ Cicala and K.\ Courser, Spans of cospans in a topos. Available as \href{https://arxiv.org/abs/1707.02098}{arXiv:1707.02098}.

\bibitem{CV} D.\ Cicala and C.\ Vasilakopoulou, On Adjoints and fibrations. In preparation.

\bibitem{Courser} K.\ Courser, A bicategory of decorated cospans, \emph{Theory Appl.\ Categ.} \textbf{32} (2017), 995--1027. Available as \href{https://arxiv.org/abs/1605.08100}{arXiv:1605.08100}.

\bibitem{CourserThesis} K.\ Courser, \textsl{Open Systems: a Double Categorical Perspective}, Ph.D.\ thesis, Department of Mathematics, U.\ C.\ Riverside, 2020.  Available as \href{https://arxiv.org/abs/2008.02394}{arxiv:2008.02394}.

\bibitem{DS} B.\ Day and R.\ Street, Monoidal Bicategories and Hopf algebroids, \textsl{Adv.\ Math.} \textbf{129} (1997), 99--157.

%\bibitem{Ehresmann63} C.\ Ehresmann, Cat\'egories structur\'ees III: Quintettes et applications covariantes,  \textsl{Cah.\ Top.\ G\'eom.\ Diff.} \textbf{5} (1963), 1--22.

%\bibitem{Ehresmann65} C.\ Ehresmann, {\sl Cat\'egories et Structures,} Dunod, Paris, 1965.

\bibitem{Ehresmann63} C.\ Ehresmann, Cat\'egories structur\'ees III: Quintettes et applications covariantes,  \textsl{Cah.\ Top.\ G\'eom.\ Diff.} \textbf{5} (1963), 1--22.

\bibitem{Ehresmann65} C.\ Ehresmann, {\sl Cat\'egories et Structures,} Dunod, Paris, 1965.

\bibitem{Fong} B.\ Fong, Decorated cospans, \emph{Theory Appl.\ Categ.} \textbf{30} (2015), 1096--1120.  Available as \href{http://arxiv.org/abs/1502.00872}{arXiv:1502.00872}.

\bibitem{FongThesis} B.\ Fong, \textsl{The Algebra of Open and Interconnected Systems},
Ph.D.\ thesis, Computer Science Department, University of Oxford, 2016.
Available as \href{https://arxiv.org/abs/1609.05382}{arXiv:1609.05382}.

%\bibitem{GP1} M.\ Grandis and R.\ Par\'e, Limits in double categories, \textsl{Cah.\ Top.\ G\'eom.\ Diff.} \textbf{40} (1999), 162--220.

%\bibitem{GP2} M.\ Grandis and R.\ Par\'e, Adjoints for double categories, 
% \textsl{Cah.\ Top.\ G\'eom.\ Diff.} \textbf{45} (2004), 193--240.

\bibitem{Gray} J. \ Gray, Fibred and Cofibred categories, \emph{Proc. Conf. Categorical Algebra} (La Jolla, California, 1965), 21---83, New York, 1966

\bibitem{HS}  L.\ W.\ Hansen and M.\ Shulman, Constructing symmetric monoidal bicategories functorially.  Available as \href{https://arxiv.org/abs/1910.09240}{arXiv:1910.09240}.

%\bibitem{Haug} R.\ Haugseng, Iterated spans and ``classical" topological field theories. Available as \href{https://arxiv.org/abs/1409.0837}{arXiv:1409.0837}.

\bibitem{Herm} C. \ Hermida, Fibrations, logical predicates and indeterminates, PhD Thesis, University of Edinburgh, 1993.

\bibitem{Hermida1999} C. \ Hermida, Some properties of Fib as a fibred 2-category. \emph{J. Pure Appl. Algebra}, 134(1), 83 -- 109, 1999

%\bibitem{Hoff} A.\ Hoffnung, Spans in 2-categories: a monoidal tricategory. Available as \href{http://arxiv.org/abs/1112.0560}{arXiv:1112.0560}.

\bibitem{JM} J.\ C.\ Baez and J.\ Master, A compositional framework for Petri nets. In preparation.

\bibitem{KS} G.\ Kelly and R.\ Street, Review of the elements of 2-categories, \emph{Lecture Notes in Mathematics} vol. 40, pages 75---103, Springer Berlin/Heidelberg, 1974.

\bibitem{LS} E.\ Lerman and D.\ Spivak, An algebra of open continuous time dynamical systems and networks. Available as \href{http://arxiv.org/abs/1602.01017}{arXiv:1602.01017}.

%\bibitem{Lack} S.\ Lack, Limits for lax morphisms, \emph{Applied Categorical Structures} $\mathbf{30}$ (2005), 189--203. Available %at \href{http://maths.mq.edu.au/~slack/papers/talgl.pdf}{http://maths.mq.edu.au/$\sim$slack/papers/talgl.pdf}.

   %\bibitem{Lerm} E.\ Lerman and D.\ Spivak, An algebra of open continuous time dynamical systems and networks. Available as %%\href{http://arxiv.org/abs/1602.01017}{arXiv:1602.01017}.

%   \bibitem{ML} S.\ Mac Lane, {\sl Categories for the Working Mathematician},
%     Springer, Berlin, 1998.

\bibitem{MV} J.\ Moeller and C.\ Vasilakopoulou, Monoidal Grothendieck construction, \emph{Theory Appl.\ Categ.} \textbf{35} (2020), no. 31, 1159--1207. Available as \href{https://arxiv.org/abs/1809.00727}{arXiv:1809.00727}.

\bibitem{Niefield} S.~Niefield, Span, cospan, and other double categories, \textsl{Theory Appl.\ Categ.} \textbf{26} (2012), 729--742. Available as \href{https://arxiv.org/abs/1201.3789}{arXiv:1201.3789}.

%\bibitem{Panan} F.\ Clerc, H.\ Humphrey and P.\ Panangaden, Bicategories of Markov processes,  to appear.

\bibitem{PollardThesis} B.\ S.\ Pollard, \textsl{Open Markov Processes and Reaction Networks}, Ph.D. thesis, U. C. Riverside, 2017.  Available as \href{https://arxiv.org/abs/1709.09743}{arXiv:1709.09743}.

%\bibitem{RSW} R.\ Rosebrugh, N.\ Sabadini and R.\ F.\ C.\ Walters, Generic commutative separable algebras and cospans of graphs, \textsl{Theory Appl. Categ.} \textbf{15} (2005), 164--177. Available at \href{http:/.www.tac.mta.ca/tac/volumes/15/6/15-06.pdf}{http:/.www.tac.mta.ca/tac/volumes/15/6/15-06.pdf}.

%\bibitem{Reb} F.\ Rebro, Constructing the bicategory Span$_{2}(\mathrm{A})$. Available as \href{https://arxiv.org/abs/1501.00792}{arXiv:1501.00792}.

\bibitem{Shulman2008} M.\ Shulman, Framed bicategories and monoidal fibrations, \emph{Theory Appl.Categ.} \textbf{20} (2008), no. 18, 650 -- 738, 2008. Available as \href{https://arxiv.org/abs/0706.1286}{arXiv:0706.1286}.

\bibitem{Shulman2010} M.\ Shulman, Constructing symmetric monoidal bicategories. Available as \href{http://arxiv.org/abs/1004.0993}{arXiv:1004.0993}.

%\bibitem{Stay} M.\ Stay, Compact closed bicategories.  Available as \href{http://arxiv.org/abs/1301.1053}{arXiv:1301.1053}.

\bibitem{Yass} A.\ Yassine, Open Systems in Classical Mechanics. Available as \href{https://arxiv.org/abs/1710.11392}{arXiv:1710.11392}.

\end{thebibliography}
\end{document}
