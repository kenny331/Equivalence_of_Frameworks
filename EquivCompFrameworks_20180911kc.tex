% An equivalence of compositional frameworks
% John Baez, Kenny Courser and Christina Vasilakopoulou
% 2019/09/11 - KC
\documentclass{amsart}
\usepackage{amssymb,amsmath,stmaryrd,txfonts,mathrsfs,amsthm}

\usepackage[all,2cell]{xy}\UseAllTwocells\SilentMatrices
\usepackage[neveradjust]{paralist}
\usepackage{hyperref}
\usepackage{mathtools}
\usepackage{multirow}
\usepackage[outline]{contour}
\contourlength{1.2pt}
\usepackage{tikz}
\usepackage{tikz-cd}
\usepackage{xcolor}
\usepackage{framed,color}
\usepackage[draft]{fixme}
\usetikzlibrary{matrix,arrows,decorations.pathmorphing,positioning}
\usetikzlibrary{intersections,decorations.markings}
\usetikzlibrary{arrows,positioning,fit,matrix,shapes.geometric,external}
\usetikzlibrary{backgrounds,circuits,circuits.ee.IEC,shapes,fit,matrix}
\usepackage{tikz}
\usetikzlibrary{matrix,arrows}
\usepackage{comment}
\usepackage[capitalize]{cleveref}
\definecolor{rewritecolor}{rgb}{0,.9,1}
\tikzset{rewritenode/.style={shape=circle,fill=rewritecolor,scale=0.25,font=\Huge}}
\tikzset{RWopen/.style={shape=circle,draw=black,fill=white,scale=0.5,font=\Huge}}
\tikzset{RWclosed/.style={shape=circle,fill=black,scale=0.5,font=\Huge}}
\tikzset{CDnode/.style={shape=circle,fill=white,scale=.5}}
\makeatletter
\let\ea\expandafter

\pgfdeclarelayer{edgelayer}
\pgfdeclarelayer{nodelayer}
\pgfsetlayers{edgelayer,nodelayer,main}

% Petri nets
\definecolor{lblue}{rgb}{0,250,255}
\tikzstyle{species}=[circle,fill=yellow,draw=black,scale=1.15]
\tikzstyle{transition}=[rectangle,fill=lblue,draw=black,scale=1.15]
\tikzstyle{inarrow}=[->, >=stealth, shorten >=.03cm,line width=1.5]
\tikzstyle{empty}=[circle,fill=none, draw=none]
\tikzstyle{inputdot}=[circle,fill=purple,draw=purple, scale=.25]
\tikzstyle{inputarrow}=[->,draw=purple, shorten >=.05cm]
\tikzstyle{simple}=[-,draw=purple,line width=1.000]
\tikzstyle{none}=[inner sep=0pt]

\definecolor{shadecolor}{rgb}{1,0.8,0.3}
\definecolor{myurlcolor}{rgb}{0.6,0,0}
\definecolor{mycitecolor}{rgb}{0,0,0.8}
\definecolor{myrefcolor}{rgb}{0,0,0.8}
\hypersetup{colorlinks, linkcolor=myrefcolor, citecolor=mycitecolor, urlcolor=myurlcolor}

\tikzset{->-/.style={decoration={
  markings,
  mark=at position .5 with {\arrow{>}}},postaction={decorate}}}

%% Defining commands that are always in math mode.
\def\mdef#1#2{\ea\ea\ea\gdef\ea\ea\noexpand#1\ea{\ea\ensuremath\ea{#2}}}
\def\alwaysmath#1{\ea\ea\ea\global\ea\ea\ea\let\ea\ea\csname your@#1\endcsname\csname #1\endcsname
  \ea\def\csname #1\endcsname{\ensuremath{\csname your@#1\endcsname}}}
\newcommand{\define}[1]{{\bf \boldmath{#1}}}

% blackboard bold letters
\newcommand{\lA}{\ensuremath{\mathbb{A}}}
\newcommand{\lC}{\ensuremath{\mathbb{C}}}
\newcommand{\lD}{\ensuremath{\mathbb{D}}}
\newcommand{\lE}{\ensuremath{\mathbb{E}}}
\newcommand{\lR}{\ensuremath{\mathbb{R}}}
\newcommand{\lX}{\ensuremath{\mathbb{X}}}
\mdef\fahat{\hat{\fa}}

% MISCELLANEOUS SYMBOLS
\newcommand{\inv}{^{-1}}
\newcommand{\op}{^{\mathit{op}}}
\newcommand{\co}{^{\mathit{co}}}
\newcommand{\coop}{^{\mathit{coop}}}
\newcommand{\id}{\mathm{id}}
\let\adj\dashv
\newcommand{\pullbackcorner}[1][dr]{\save*!/#1-1.2pc/#1:(-1,1)@^{|-}\restore}
\let\iso\cong
\let\eqv\simeq
\let\cng\equiv
\mdef\Id{\mathrm{Id}}
\mdef\id{\mathrm{id}}
\alwaysmath{ell}
\alwaysmath{infty}
\alwaysmath{odot}
\def\frc#1/#2.{\frac{#1}{#2}}   % \frc x^2+1 / x^2-1 .
\mdef\ten{\mathrel{\otimes}}

%% OPERATORS
\DeclareMathOperator\colim{colim}
\DeclareMathOperator\eq{eq}
\DeclareMathOperator\Aut{Aut}
\DeclareMathOperator\End{End}
\DeclareMathOperator\Hom{Hom}
\DeclareMathOperator\Map{Map}

%% ARROWS
% \to already exists
\newcommand{\too}[1][]{\ensuremath{\overset{#1}{\longrightarrow}}}
\newcommand{\oot}[1][]{\ensuremath{\overset{#1}{\longleftarrow}}}
\let\toot\rightleftarrows
\let\otto\leftrightarrows
\let\maps\colon

%% EXTENSIBLE ARROWS
\let\xto\xrightarrow
\let\xot\xleftarrow

% THEOREM-TYPE ENVIRONMENTS, hacked to
%% (a) number all with the same numbers, and
%% (b) have the right names for autoref
\def\defthm#1#2{%
  \newtheorem{#1}{#2}[section]%
  \expandafter\def\csname #1autorefname\endcsname{#2}%
  \expandafter\let\csname c@#1\endcsname\c@thm}
\newtheorem{thm}{Theorem}[section]
\newcommand{\thmautorefname}{Theorem}
\defthm{cor}{Corollary}
\defthm{prop}{Proposition}
\defthm{lem}{Lemma}
\defthm{conj}{Conjecture}
\defthm{hyp}{Hypothesis}
\defthm{fact}{Fact}
\theoremstyle{definition}
\defthm{defn}{Definition}
\defthm{notn}{Notation}
\theoremstyle{remark}
\defthm{rmk}{Remark}
\defthm{eg}{Example}

\newcommand{\fhat}{\ensuremath{\hat{f}}}

% Also number formulas with the theorem counter
\let\c@equation\c@thm
\numberwithin{equation}{section}

% Only show numbers for equations that are actually referenced (or
% whose tags are specified manually) - uses mathtools.
%\mathtoolsset{showonlyrefs,showmanualtags}

\def\tobar{\mathrel{\mkern3mu  \vcenter{\hbox{$\scriptscriptstyle+$}}%
                    \mkern-12mu{\to}}}

%\input{decls}
\UseAllTwocells

\newcommand{\dblcat}[1]{\mathbb{#1}}
\mdef\fchk{\check{f}}

\definecolor{purple(x11)}{rgb}{0.5, 0.0, 0.5}
\def\purple{\color{purple(x11)}}
\def\chris{\purple}

%Christina: change below accordingly if needed!
\newcommand{\ca}{\mathsf}
\newcommand{\bicat}{\mathbf}
\newcommand{\U}{U}
\newcommand{\D}{\ca{A}}
\newcommand{\C}{\ca{X}} 
\newcommand{\A}{\ca{A}}
\newcommand{\B}{\ca{B}}
\newcommand{\X}{\ca{X}}
\newcommand{\dcsp}[1]{{#1}\mathbb{C}\textnormal{sp}}
\tikzset{tick/.style={postaction={decorate,decoration={markings,
mark=at position 0.4 with {\draw[-] (0,.4ex) -- (0,-.4ex);}}}}}
\newcommand{\tickar}{\begin{tikzcd}[baseline=-0.5ex,cramped,sep=small,ampersand replacement=\&]{}\ar[r,tick]\&{}\end{tikzcd}}
\newcommand{\cspn}[5]{\begin{tikzcd}[baseline=-0.5ex,cramped,sep=small,ampersand replacement=\&]{#1}\ar[r,"#4"] \& {#2} \& {#3}\ar[l,"#5"']\end{tikzcd}}
\newcommand{\OpICat}{\bicat{OpICat}}%probably too much to use?
\newcommand{\pse}{\mathrm{ps}}
\newcommand{\OpFib}{\bicat{OpFib}}

\usepackage[foot]{amsaddr}

\title{An equivalence of compositional frameworks}

\author{John\ C.\ Baez$^{*,\bullet}$, Kenny Courser$^*$, and Christina Vasilakopoulou$^*$}
\address{$^*$Department of Mathematics, University of California, Riverside CA, USA 92521}
\address{$^\bullet$Centre for Quantum Technologies, National University of Singapore, Singapore 117543}
\email{baez@ucr.edu, kcour001@ucr.edu, vasilak@ucr.edu}


\begin{document}
\begin{abstract}
\noindent
The first two authors have developed a compositional framework well-suited for studying networks that are built out of finite sets equipped with extra stuff. This framework, which goes by the name of `structured cospans', utilizes double categories where the objects are seen as inputs and outputs, horizontal 1-cells are `open networks', and 2-morphisms are maps between open networks. In this setup a functor $L \colon \textsf{A} \to \textsf{X}$, which is typically a left adjoint, is used to replace the objects and vertical 1-morphisms of a given double category $\mathbb{X}$ with the objects and morphisms, respectively, of the category $\textsf{A}$. Horizontal 1-cells are then cospans in $\textsf{X}$ of a particular form with 2-morphisms given by maps of these cospans. Fong has also developed a similar framework utilizing cospans to study open networks which goes by the name of `decorated cospans'. In this setup, a lax monoidal functor $F \colon \textsf{A} \to \textsf{Set}$ is used to `decorate' the apices of cospans in $\textsf{A}$ with elements of $\textsf{Set}$ giving the cospans extra structure. Using a slight variation of Fong's framework, we prove that these two frameworks are equivalent in the situation where a left adjoint can be obtained from a lax monoidal pseudofunctor using a well known construction of Grothendieck.
\end{abstract}

\maketitle

\setcounter{tocdepth}{1} % comment this out if you want to see the subsections in the table of contents
\tableofcontents

\section*{Various issues-in-progress}
 
\begin{itemize}
\item {\chris Discuss notation for Decorated and Structured cospan double categories one last time? :)}
\item {\chris Where should we put some standard things we use? For exampe, lax monoidal pseudofunctor should be recalled early. Together with pseudonatural transformations I believe. For now, I put them in an unnamed section}
\item {\chris Notation between \cref{Equiv} and \cref{thm:decoratedcospans} right now does not match! Need to fix.}
%\item {\chris There might be a way to completely hide the details of version 'fibres have, reindexing preserve' if that's what we want to do.}
\end{itemize}

\section*{Various notions to be recalled, earlier rather than later}

Recall that a pseudofunctor $F\colon\bicat{A}\to\bicat{B}$ between bicategories $\bicat{A}$ and $\bicat{B}$ is functorial up to coherent natural isomorphism, namely for composable arrows we have $F(g\circ f)\cong Fg\circ Ff$ and $F(1_a)\cong1_{Fa}$ satisfying standard axioms. 
Given pseudofunctors $F,G\colon \bicat{A} \to \bicat{B}$, a \define{pseudonatural transformation} $\sigma$ consists of {\chris notation below needs fixing}
\begin{itemize}
\item for each object $a \in \bicat{A}$, a morphism $\sigma_a \colon F(a) \to G(a)$ in $\bicat{B}$ 
\item for every pair of objects $a$ and $b$ of $\mathbf{A}$, we have natural isomorphisms
\[
\begin{tikzpicture}[scale=1.5]
\node (A) at (0,0) {$\mathbf{A}(a,b)$};
\node (B) at (2,0) {$\mathbf{X}(F(a),F(b))$};
\node (C) at (0,-1) {$\mathbf{X}(G(a),G(b))$};
\node (D) at (2,-1) {$\mathbf{X}(F(a),G(b))$};
\node (E) at (1,-0.5) {$\sigma_{a,b} \Nearrow$};
\path[->,font=\scriptsize,>=angle 90]
(A) edge node[above]{$F$} (B)
(B) edge node[right]{$(\sigma_{b})_*$} (D)
(A) edge node[left]{$G$} (C)
(C) edge node[above]{$(\sigma_a)^*$} (D);
\end{tikzpicture}
\]
where $(\sigma_a)^*$ and $(\sigma_b)_*$ are the functors induced by precomposition and postcompositon, respectively. Thus for each morphism $f \colon a \to b$ in $\mathbf{A}$, we have an invertible 2-morphism $\sigma_f \colon G(f) \sigma_a \xrightarrow{\sim} \sigma_{b} F(f)$ in $\mathbf{X}$:
\[
\begin{tikzpicture}[scale=1.5]
\node (A) at (0,0) {$F(a)$};
\node (B) at (1,0) {$F(b)$};
\node (C) at (0,-1) {$G(a)$};
\node (D) at (1,-1) {$G(b)$};
\node (E) at (0.5,-0.5) {$\sigma_f \Nearrow$};
\path[->,font=\scriptsize,>=angle 90]
(A) edge node[above]{$F(f)$} (B)
(B) edge node[right]{$\sigma_{b}$} (D)
(A) edge node[left]{$\sigma_a$} (C)
(C) edge node[above]{$G(f)$} (D);
\end{tikzpicture}
\]
which is compatible with composition and identities.
\end{itemize}
%such that for each composable pair of morphisms $f \colon a \to b$ and $g \colon b \to c$ of $\mathbf{A}$, the following diagrams commute:
%\[
%\begin{tikzpicture}[scale=1.5]
%\node (A) at (0,0.5) {$(G(g)G(f))\sigma_a$};
%\node (A') at (2,0.5) {$G(g)(G(f) \sigma_a)$};
%\node (B) at (0,-0.5) {$G(gf) \sigma_a$};
%\node (C) at (4,0.5) {$G(g) (\sigma_b F(f))$};
%\node (C') at (4,-0.5) {$\sigma_c F(gf)$};
%\node (D) at (6,0.5) {$(G(g) \sigma_b) F(f)$};
%\node (D') at (8,-0.5) {$\sigma_c (F(g)F(f))$};
%\node (F) at (8,0.5) {$(\sigma_c F(g)) F(f)$};
%\path[->,font=\scriptsize,>=angle 90]
%(A) edge node[above]{$a^\prime$} (A')
%(A) edge node[left]{$\psi 1_{\sigma_a}$} (B)
%(A') edge node[above]{$1_{G(g)} \sigma_f$} (C)
%(B) edge node[above]{$\sigma_{gf}$} (C')
%(C) edge node [above] {${a^\prime}^{-1}$} (D)
%(D') edge node [above] {$1_{\sigma_c} \phi$} (C')
%(D) edge node [above] {$\sigma_g 1_{F(f)}$} (F)
%(F) edge node [right] {$a^\prime$} (D');
%\end{tikzpicture}
%\]
%\[
%\begin{tikzpicture}[scale=1.5]
%\node (A) at (0,0) {$1_{G(a)} \sigma_a$};
%\node (B) at (1,1) {$G(1_a) \sigma_a$};
%\node (C) at (2,0) {$\sigma_a F(1_a)$};
%\node (D) at (0.5,-1) {$\sigma_a$};
%\node (E) at (1.5,-1) {$\sigma_a 1_{F(a)}$};
%\path[->,font=\scriptsize,>=angle 90]
%(A) edge node[left]{$\psi 1_{\sigma_a}$} (B)
%(B) edge node[right]{$\sigma_{1_a}$} (C)
%(A) edge node [left] {$\ell^\prime$} (D)
%(D) edge node [above] {${\rho^\prime}^{-1}$} (E)
%(E) edge node [right] {$1_{\sigma_a} \phi$} (C);
%\end{tikzpicture}
%\]
%Christina: I removed the above, they are not needed I believe
We denote by $[\A,\bicat{Cat}]_\pse$ the 2-category 
of pseudofunctors, pseudonatural transformations and modifications from an ordinary category $\A$ viewed as a 2-category with trivial 2-cells, into $\bicat{Cat}$. Classicaly this has been referred to as the 2-category of \emph{opindexed categories}, since an indexed category is a contravariant pseudofunctor into $\bicat{Cat}$.

A \define{lax monoidal} pseudofunctor is....
A \define{symmetric lax monoidal} pseudofunctor...
{\chris Not sure if we need monoidal transformations?}

\begin{comment}
\section{Introduction}
Networks are playing an increasingly prominent role in our understanding of the world and as a result, methods of characterizing and studying networks have become evermore necessary. Applied category theory provides such an avenue and much work has been done towards this effort  \cite{BC,BCR,BF,BFP,BP,Cour,Fong,JM,LS,Yass}. One of the more recent frameworks developed that is suitable for studying networks is Fong's `decorated cospans' \cite{Fong}. A $\textbf{cospan}$ in any category is a diagram of the form:
\[
\begin{tikzpicture}[scale=1.5]
\node (A) at (0,0) {$a_1$};
\node (B) at (1,1) {$b$};
\node (C) at (2,0) {$a_2$};
\path[->,font=\scriptsize,>=angle 90]
(A) edge node[above]{$i$} (B)
(C) edge node[above]{$o$} (B);
\end{tikzpicture}
\]
We call $b$ the $\textbf{apex}$ of the cospan and $a_1$ and $a_2$ the $\textbf{feet}$ of the cospan. The morphisms $i$ and $o$ are called the $\textbf{legs}$. Cospans are ideal for realizing networks and more generally `open' systems where here open means that each system or network comes with prescribed inputs and outputs which are represented by the feet of the cospan $a$ and $b$, respectively. Two open systems viewed as cospans such that the outputs of the first and the inputs of the second coincide can then naturally be composed via pushout to obtain an open system whose inputs are that of the first and outputs are that of the second. For example, we can compose the above cospan with the following:
\[
\begin{tikzpicture}[scale=1.5]
\node (A) at (0,0) {$a_2$};
\node (B) at (1,1) {$b^\prime$};
\node (C) at (2,0) {$a_3$};
\path[->,font=\scriptsize,>=angle 90]
(A) edge node[above]{$i^\prime$} (B)
(C) edge node[above]{$o^\prime$} (B);
\end{tikzpicture}
\]
to obtain a new cospan whose left foot is the left foot of the first, whose right foot is the right foot of the second, and whose apex and legs are given by the pushout.
\[
\begin{tikzpicture}[scale=1.5]
\node (A) at (0,0) {$a_1$};
\node (B) at (1,1) {$b$};
\node (C) at (2,0) {$a_2$};
\node (D) at (3,1) {$b^\prime$};
\node (E) at (4,0) {$a_3$};
\node (F) at (2,2) {$b+_{a_2} b^\prime$};
\node (G) at (2,1) {$b+b^\prime$};
\path[->,font=\scriptsize,>=angle 90]
(A) edge node[above]{$i$} (B)
(C) edge node[above]{$o$} (B)
(C) edge node[above]{$i^\prime$} (D)
(E) edge node[above]{$o^\prime$} (D)
(B) edge node[left]{$\psi j_b$} (F)
(D) edge node[right]{$\psi j_{b^\prime}$} (F)
(B) edge node[above]{$j_b$} (G)
(D) edge node[above]{$j_{b^\prime}$} (G)
(G) edge node [left] {$\psi$} (F);
\end{tikzpicture}
\]

In Fong's theory of decorated cospans, given a category $\textsf{A}$ with finite colmits, a symmetric lax monoidal functor $F \colon (\textsf{A},+,0) \to (\textsf{Set},\times,1)$ is used to `decorate' the apex of a cospan in $\textsf{A}$ with an element of the image of its apex under the functor $F$. In the case of the first cospan above, an element of the image of the apex under the functor $F$ is given by a morphism $d \colon 1 \to F(b)$, and we call $d$ a \textbf{decoration} on $b$. Here, $1$ is any terminal object in $\mathsf{Set}$, $F(b)$ is the collection of all $F$-decorations on the object $b$ and the morphism $d$ is selecting a particular one. Thus a decorated cospan is given by a pair:
\[
\begin{tikzpicture}[scale=1.5]
\node (A) at (0,0) {$a_1$};
\node (B) at (1,0) {$b$};
\node (C) at (2,0) {$a_2$};
\node (D) at (3,0) {$1$};
\node (E) at (4,0) {$F(b)$};
\path[->,font=\scriptsize,>=angle 90]
(A) edge node[above]{$i$} (B)
(C) edge node[above]{$o$} (B)
(D) edge node [above] {$d$} (E);
\end{tikzpicture}
\]
To obtain a decoration on the composition of two composable $F$-decorated cospans:
\[
\begin{tikzpicture}[scale=1.5]
\node (A) at (0,0) {$a_1$};
\node (B) at (1,0) {$b$};
\node (C) at (2,0) {$a_2$};
\node (F) at (3,0) {$a_2$};
\node (G) at (4,0) {$b^\prime$};
\node (H) at (5,0) {$a_3$};
\node (D) at (0.5,-0.5) {$1$};
\node (E) at (1.5,-0.5) {$F(b)$};
\node (D') at (3.5,-0.5) {$1$};
\node (E') at (4.5,-0.5) {$F(b^\prime)$};
\path[->,font=\scriptsize,>=angle 90]
(A) edge node[above]{$i$} (B)
(C) edge node[above]{$o$} (B)
(F) edge node[above]{$i^\prime$} (G)
(H) edge node[above]{$o^\prime$} (G)
(D) edge node [above] {$d_1$} (E)
(D') edge node [above] {$d_2$} (E');
\end{tikzpicture}
\]
we use the natural map from a coproduct to a pushout as well as the structure maps, or `laxators', that come as part of the structure of a lax monoidal functor:
 $$1 \xrightarrow{\lambda^{-1}} 1 \times 1 \xrightarrow{d_1 \times d_2} F(b) \times F(b^\prime) \xrightarrow{\phi_{b,b^\prime}} F(b+ b^\prime) \xrightarrow{F(j_{b,b^\prime})} F(b+_{a_2} b^\prime).$$
From this symmetric lax monoidal functor $F \colon \textsf{A} \to \mathsf{Set}$, Fong then constructs a symmetric monoidal category $F\textnormal{Cospan}(\mathsf{A})$ which has:
\begin{enumerate}
\item{objects as those of $\mathsf{A}$ and}
\item{morphisms as isomorphism classes of $F$-decorated cospans, where an $F$-decorated cospan is given as above, and two $F$-decorated cospans are in the same isomorphism class if there exists an isomorphism $f \colon b \to b^\prime$ between the apices such that the following diagrams commute:
\[
\begin{tikzpicture}[scale=1.5]
\node (A) at (0,0) {$a_1$};
\node (B) at (1,1) {$b$};
\node (B') at (1,-1) {$b^\prime$};
\node (C) at (2,0) {$a_2$};
\node (D) at (3,0) {$1$};
\node (E) at (4,0.5) {$F(b)$};
\node (E') at (4,-0.5) {$F(b^\prime)$};
\path[->,font=\scriptsize,>=angle 90]
(A) edge node[above]{$i$} (B)
(C) edge node[above]{$o$} (B)
(A) edge node[below]{$i^\prime$} (B')
(C) edge node[below]{$o^\prime$} (B')
(B) edge node [left] {$f$} (B')
(B) edge node [right] {$\sim$} (B')
(D) edge node [above] {$d$} (E)
(D) edge node [below] {$d^\prime$} (E')
(E) edge node [right] {$F(f)$} (E');
\end{tikzpicture}
\]
}
\end{enumerate}


As an example, let $F \colon \textsf{FinSet} \to \textsf{Set}$ be the symmetric lax monoidal functor that assigns to a finite set $b$ the (large) set of all possible graph structures on the finite set $b$, where a graph structure on $b$ is given by a diagram in $\textsf{Set}$ of the form:
\[
\begin{tikzpicture}[scale=1.5]
\node (A) at (0,0) {$E$};
\node (B) at (1,0) {$b.$};
\path[->,font=\scriptsize,>=angle 90]
(A) edge[bend left] node[above]{$s$} (B)
(A) edge[bend right] node[below]{$t$} (B);
\end{tikzpicture}
\]
Let $b=\{ v_1,v_2 \}$ be a two element set. Then one element of the (large) set $F(b)$, which is the collection of all graph structures on the finite set $b$, is given by a single edge $e$ whose source and target are $v_1$ and $v_2$, respectively.
\[
\begin{tikzpicture}
  [scale=.8,auto=left]
  \node [style={circle,fill=teal}] (n1) at (1,10) {$v_1$};
  \node[style={circle,fill=teal}] (n2) at (4,10)  {$v_2$};
\path[->,font=\scriptsize,>=angle 90]
(n1) edge node[above]{$e$} (n2);
\end{tikzpicture}
\]
%\[
%\begin{tikzpicture}[scale=1.5]
%\node (A) at (0,0) {$v_1$};
%\node (B) at (1,0) {$v_2$};
%\path[->,font=\scriptsize,>=angle 90]
%(A) edge node[above]{$e$} (B);
%\end{tikzpicture}
%\]
Denote this element of $F(b)$ as $d \colon 1 \to F(b)$. Let $a_1=\{ 1 \}$ and $a_2=\{2,3\}$ and define functions $i \colon a_1 \to b$ and $o \colon a_2 \to b$ by $i(1)=v_1$ and $o(2)=o(3)=v_2$. Then we have an $F$-decorated cospan: 
\[
\begin{tikzpicture}[scale=1.5]
\node (A) at (0,0) {$a_1$};
\node (B) at (1,0) {$b$};
\node (C) at (2,0) {$a_2$};
\node (D) at (3,0) {$1$};
\node (E) at (4,0) {$F(b)$};
\path[->,font=\scriptsize,>=angle 90]
(A) edge node[above]{$i$} (B)
(C) edge node[above]{$o$} (B)
(D) edge node [above] {$d$} (E);
\end{tikzpicture}
\]
which is given by the open graph:
\[
\begin{tikzpicture}
  [scale=.8,auto=left]
\node (a) at (-2,10) {$1$};
\node (b) at (7,10.5) {$2$};
\node (c) at (7,9.5) {$3$};
\node (i) at (-0.5,11) {$i$};
\node (o) at (5.5,11) {$o$};
  \node [style={circle,fill=teal}] (n1) at (1,10) {$v_1$};
  \node[style={circle,fill=teal}] (n2) at (4,10)  {$v_2$};
\path[->,font=\scriptsize,>=angle 90]
(a) edge[dashed] node[above] {$$} (n1)
(b) edge[dashed] node[below] {$$} (n2)
(c) edge[dashed] node[above] {$$} (n2)
(n1) edge node[above]{$e$} (n2);
\end{tikzpicture}
\]
We can compose this open graph with another whose set inputs coincide with the set of outputs of the above open graph, namely the set $a_2=\{2,3\}$. For example:
\[
\begin{tikzpicture}
  [scale=.8,auto=left]
%\node (a) at (-2,10) {$1$};
\node (b) at (7,10.5) {$2$};
\node (c) at (7,9.5) {$3$};
%\node (i) at (-0.5,11) {$i$};
\node (o) at (8.5,11.5) {$i^\prime$};
\node (d) at (16,10) {$4$};
\node (e) at (14.5,11.5) {$o^\prime$};
  \node [style={circle,fill=teal}] (n1) at (10,11.5) {$v_4$};
  \node[style={circle,fill=teal}] (n2) at (10,8.5)  {$v_3$};
  \node[style={circle,fill=teal}] (n3) at (13,10)  {$v_5$};
\path[->,font=\scriptsize,>=angle 90]
%(a) edge[dashed] node[above] {$$} (n1)
(b) edge[dashed] node[below] {$$} (n1)
(c) edge[dashed] node[above] {$$} (n2)
(n1) edge[bend left] node[above]{$e_2$} (n3)
(n3) edge[bend left] node[below]{$e_3$} (n2)
(n2) edge[bend left] node[left]{$e_1$} (n1)
(d) edge[dashed] node[above]{$$}(n3);
\end{tikzpicture}
\]
This second open graph can be represented by the $F$-decorated cospan:
\[
\begin{tikzpicture}[scale=1.5]
\node (A) at (0,0) {$a_2$};
\node (B) at (1,0) {$b^\prime$};
\node (C) at (2,0) {$a_3$};
\node (D) at (3,0) {$1$};
\node (E) at (4,0) {$F(b^\prime)$};
\path[->,font=\scriptsize,>=angle 90]
(A) edge node[above]{$i^\prime$} (B)
(C) edge node[above]{$o^\prime$} (B)
(D) edge node [above] {$d^\prime$} (E);
\end{tikzpicture}
\]
The composite of these two open graphs is then given by:
\[
\begin{tikzpicture}
  [scale=.8,auto=left]
\node (a) at (-2,10) {$1$};
%\node (b) at (7,10.5) {$2$};
\node (c) at (10,10) {$4$};
\node (i) at (-0.5,11) {$i$};
\node (o) at (8.5,11) {$o^\prime$};
  \node [style={circle,fill=teal}] (n1) at (1,10) {$v_1$};
  \node[style={circle,fill=teal}] (n2) at (4,10)  {$v_2$};
  \node[style={circle,fill=teal}] (n3) at (7,10)  {$v_5$};
\path[->,font=\scriptsize,>=angle 90]
(a) edge[dashed] node[above] {$$} (n1)
%(b) edge[dashed] node[below] {$$} (n2)
(c) edge[dashed] node[above] {$$} (n3)
(n1) edge node[above]{$e$} (n2)
(n2) edge[bend left] node[above]{$e_2$} (n3)
(n2) edge[loop below]node{$e_1$} (n2)
(n3) edge[bend left] node[below]{$e_3$} (n2);
\end{tikzpicture}
\]
where we identify $v_2=v_3=v_4$ in taking the pushout of the two cospans of finite sets. The composite open graph is represented by the $F$-decorated cospan:
\[
\begin{tikzpicture}[scale=1.5]
\node (A) at (0,0) {$a_1$};
\node (B) at (1.25,0) {$b+_{a_2}b^\prime$};
\node (C) at (2.5,0) {$a_3$};
\node (D) at (3,0) {$1$};
\node (E) at (4,0) {$F(b+_{a_2}b^\prime)$};
\path[->,font=\scriptsize,>=angle 90]
(A) edge node[above]{$\psi j_b i$} (B)
(C) edge node[above]{$\psi j_{b^\prime} o^\prime$} (B)
(D) edge node [above] {$d''$} (E);
\end{tikzpicture}
\]
where $j$ is the natural map into a coproduct and $\psi$ the natural map from a coproduct into a pushout. 

We can also consider the above two open graphs in parallel:
\[
\begin{tikzpicture}
  [scale=.8,auto=left]
\node (a') at (7,12.5) {$1$};
\node (b') at (16,12.5) {$2$};
\node (c') at (16,11.5) {$3$};
\node (i') at (8.5,13.5) {$i+i^\prime$};
\node (o') at (14.5,13.5) {$o+o^\prime$};
  \node [style={circle,fill=teal}] (n1') at (10,13) {$v_1$};
  \node[style={circle,fill=teal}] (n2') at (13,13)  {$v_2$};
\node (b) at (7,11.5) {$2$};
\node (c) at (7,10.5) {$3$};
\node (d) at (16,10.5) {$4$};
  \node [style={circle,fill=teal}] (n1) at (10,11.5) {$v_4$};
  \node[style={circle,fill=teal}] (n2) at (10,8.5)  {$v_3$};
  \node[style={circle,fill=teal}] (n3) at (13,10)  {$v_5$};
\path[->,font=\scriptsize,>=angle 90]
(a') edge[dashed] node[above] {$$} (n1')
(b') edge[dashed] node[below] {$$} (n2')
(c') edge[dashed] node[above] {$$} (n2')
(n1') edge node[above]{$e$} (n2')
(b) edge[dashed] node[below] {$$} (n1)
(c) edge[dashed] node[above] {$$} (n2)
(n1) edge[bend left] node[above]{$e_2$} (n3)
(n3) edge[bend left] node[below]{$e_3$} (n2)
(n2) edge[bend left] node[left]{$e_1$} (n1)
(d) edge[dashed] node[above]{$$}(n3);
\end{tikzpicture}
\]
which is represented by the tensor product of the above two $F$-decorated cospans:
\[
\begin{tikzpicture}[scale=1.5]
\node (A) at (0,0) {$a_1+a_2$};
\node (B) at (1.25,0) {$b+b^\prime$};
\node (C) at (2.5,0) {$a_2+a_3$};
\node (D) at (3.5,0) {$1$};
\node (E) at (4.5,0) {$F(b+b^\prime)$};
\path[->,font=\scriptsize,>=angle 90]
(A) edge node[above]{$i+i^\prime$} (B)
(C) edge node[above]{$o+ o^\prime$} (B)
(D) edge node [above] {$d+d^\prime$} (E);
\end{tikzpicture}
\]

There are some subtleties to this framework; consider two decorated cospans with the same inputs and outputs.
\[
\begin{tikzpicture}[scale=1.5]
\node (A) at (0,0) {$a_1$};
\node (B) at (1,0) {$b$};
\node (C) at (2,0) {$a_2$};
\node (F) at (3,0) {$a_1$};
\node (G) at (4,0) {$b^\prime$};
\node (H) at (5,0) {$a_2$};
\node (D) at (0.5,-0.5) {$1$};
\node (E) at (1.5,-0.5) {$F(b)$};
\node (D') at (3.5,-0.5) {$1$};
\node (E') at (4.5,-0.5) {$F(b^\prime)$};
\path[->,font=\scriptsize,>=angle 90]
(A) edge node[above]{$i$} (B)
(C) edge node[above]{$o$} (B)
(F) edge node[above]{$i^\prime$} (G)
(H) edge node[above]{$o^\prime$} (G)
(D) edge node [above] {$d$} (E)
(D') edge node [above] {$d^\prime$} (E');
\end{tikzpicture}
\]
For these two $F$-decorated cospans to be in the same isomorphism class, the following triangle is to commute:
\[
\begin{tikzpicture}[scale=1.5]
\node (D) at (3,0) {$1$};
\node (E) at (4,0.5) {$F(b)$};
\node (E') at (4,-0.5) {$F(b^\prime)$};
\path[->,font=\scriptsize,>=angle 90]
(D) edge node [above] {$d$} (E)
(D) edge node [below] {$d^\prime$} (E')
(E) edge node [right] {$F(f)$} (E');
\end{tikzpicture}
\]
This commutative triangle in $\mathsf{Set}$ in the context of the symmetric lax monoidal functor $F \colon \mathsf{FinSet} \to \mathsf{Set}$ says the following: given a decoration $d \in F(b)$, which is a graph structure with underlying set of vertices $b$, the function $F(f)$ pushes forward the graph structure $d$ to the graph structure $d^\prime \in F(b^\prime)$ with underlying set of vertices $b^\prime$, and \emph{precisely} this graph structure. The graph structure is given by set of edges of $d$. For example, if we take $b = \{v_1 ,v_2\}$ as before and let $d \in F(b)$ be given by:
\[
\begin{tikzpicture}
  [scale=.8,auto=left]
\node (a) at (-2,10) {$1$};
\node (b) at (7,10) {$2$};
\node (i) at (-0.5,11) {$i$};
\node (o) at (5.5,11) {$o$};
  \node [style={circle,fill=teal}] (n1) at (1,10) {$v_1$};
  \node[style={circle,fill=teal}] (n2) at (4,10)  {$v_2$};
\path[->,font=\scriptsize,>=angle 90]
(a) edge[dashed] node[above] {$$} (n1)
(b) edge[dashed] node[below] {$$} (n2)
(n1) edge node[above]{$e$} (n2);
\end{tikzpicture}
\]
Let $b^\prime = \{w_1,w_2\}$ and a define bijection $f \colon b \to b^\prime$ by $f(v_i)=w_i$ for $i=1,2$. Then the requirement $F(f)(d)=d^\prime$ says that $d^\prime \in F(b^\prime)$ must be given by:
\[
\begin{tikzpicture}
  [scale=.8,auto=left]
\node (a) at (-2,10) {$1$};
\node (b) at (7,10) {$2$};
\node (i) at (-0.5,11) {$i^\prime$};
\node (o) at (5.5,11) {$o^\prime$};
  \node [style={circle,fill=teal}] (n1) at (1,10) {$w_1$};
  \node[style={circle,fill=teal}] (n2) at (4,10)  {$w_2$};
\path[->,font=\scriptsize,>=angle 90]
(a) edge[dashed] node[above] {$$} (n1)
(b) edge[dashed] node[below] {$$} (n2)
(n1) edge node[above]{$e$} (n2);
\end{tikzpicture}
\]
The point to be made here is that the single edge of $d^\prime$ must also be $e$. If we were to label it say, $e^\prime$, there is no bijection $f \colon b \to b^\prime$ such that the triangle on the right commutes, and hence no isomorphism between these two $F$-decorated cospans.
\[
\begin{tikzpicture}\label{distinctisoclasses}
  [scale=.8,auto=left]
\node (D) at (9,11.5) {$1$};
\node (E) at (11,12.5) {$F(b)$};
\node (E') at (11,10.5) {$F(b^\prime)$};
\node (a) at (-2,11.5) {$1$};
\node (b) at (7,11.5) {$2$};
\node (i) at (-0.5,12.5) {$i$};
\node (o) at (5.5,12.5) {$o$};
\node (i') at (-0.5,10.5) {$i^\prime$};
\node (o') at (5.5,10.5) {$o^\prime$};
\node (A) at (2.5,11.5) {$\nexists F(f) \Downarrow$};
  \node [style={circle,fill=teal}] (n1) at (1,13) {$v_1$};
  \node[style={circle,fill=teal}] (n2) at (4,13)  {$v_2$};
  \node [style={circle,fill=red}] (n1') at (1,10) {$w_1$};
  \node[style={circle,fill=red}] (n2') at (4,10)  {$w_2$};
\path[->,font=\scriptsize,>=angle 90]
(D) edge node [above] {$d$} (E)
(D) edge node [below] {$d^\prime$} (E')
(E) edge[dashed] node [right] {$\nexists F(f)$} (E')
(a) edge[dashed] node[above] {$$} (n1)
(b) edge[dashed] node[above] {$$} (n2)
(a) edge[dashed] node[below] {$$} (n1')
(b) edge[dashed] node[below] {$$} (n2')
(n1) edge node[above]{$e$} (n2)
(n1') edge node[above]{$e^\prime$} (n2');
\end{tikzpicture}
\]
Thus these two $F$-decorated cospans constitute distinct isomorphism classes. This nuisance is amplified when viewed from a higher categorical perspective.

In a previous work, the second author extended Fong's theory of decorated cospans to a symmetric monoidal bicategory of decorated cospans \cite{Cour}. Namely, under the same hypotheses used by Fong amounting to a category $\mathsf{A}$ with finite colimits and a symmetric lax monoidal functor $F \colon (\mathsf{A},+,0) \to (\mathsf{Set}, \times, 1)$, there exists a symmetric monoidal bicategory $F\mathbf{Cospan}(\mathsf{A})$ which has:
\begin{enumerate}
\item objects as those of $\mathsf{A}$,
\item morphisms as $F$-decorated cospans in $\mathsf{A}$, which again are pairs
\[
\begin{tikzpicture}[scale=1.5]
\node (A) at (0,0) {$a_1$};
\node (B) at (1,0) {$b$};
\node (C) at (2,0) {$a_2$};
\node (D) at (3,0) {$1$};
\node (E) at (4,0) {$F(b)$};
\path[->,font=\scriptsize,>=angle 90]
(A) edge node[above]{$i$} (B)
(C) edge node[above]{$o$} (B)
(D) edge node [above] {$d$} (E);
\end{tikzpicture}
\]
and
\item 2-morphisms as pairs of commuting diagrams where the left diagram is a map of cospans in $\mathsf{A}$ and the right diagram is a commutative triangle in $\mathsf{Set}$.
\[
\begin{tikzpicture}[scale=1.5]
\node (A) at (0,0) {$a_1$};
\node (B) at (1,1) {$b$};
\node (B') at (1,-1) {$b^\prime$};
\node (C) at (2,0) {$a_2$};
\node (D) at (3,0) {$1$};
\node (E) at (4,0.5) {$F(b)$};
\node (E') at (4,-0.5) {$F(b^\prime)$};
\path[->,font=\scriptsize,>=angle 90]
(A) edge node[above]{$i$} (B)
(C) edge node[above]{$o$} (B)
(A) edge node[below]{$i^\prime$} (B')
(C) edge node[below]{$o^\prime$} (B')
(B) edge node [left] {$f$} (B')
%(B) edge node [right] {$\sim$} (B')
(D) edge node [above] {$d$} (E)
(D) edge node [below] {$d^\prime$} (E')
(E) edge node [right] {$F(f)$} (E');
\end{tikzpicture}
\]
\end{enumerate}
Returning to the previous example where we take the functor $F$ to be the functor $F \colon \mathsf{FinSet} \to \mathsf{Set}$, the situation becomes more dire. Previously, in the ordinary symmetric monoidal category $F$Cospan$(\mathsf{FinSet})$ the following two $F$-decorated cospans resided in distinct isomorphism classes:
\[
\begin{tikzpicture}\label{no2morphism}
  [scale=.8,auto=left]
%\node (D) at (9,11.5) {$1$};
\node (E) at (9,13) {$d \in F(b)$};
\node (E') at (9,10) {$d^\prime \in F(b^\prime)$};
\node (a) at (-2,13) {$1$};
\node (a') at (-2,10) {$1$};
\node (b) at (7,13) {$2$};
\node (b') at (7,10) {$2$};
\node (i) at (-0.5,13.5) {$i$};
\node (o) at (5.5,13.5) {$o$};
\node (i') at (-0.5,10.5) {$i^\prime$};
\node (o') at (5.5,10.5) {$o^\prime$};
%\node (A) at (2.5,11.5) {$\nexists F(f) \Downarrow$};
  \node [style={circle,fill=teal}] (n1) at (1,13) {$v_1$};
  \node[style={circle,fill=teal}] (n2) at (4,13)  {$v_2$};
  \node [style={circle,fill=red}] (n1') at (1,10) {$w_1$};
  \node[style={circle,fill=red}] (n2') at (4,10)  {$w_2$};
\path[->,font=\scriptsize,>=angle 90]
%(D) edge node [above] {$d$} (E)
%(D) edge node [below] {$d^\prime$} (E')
%(E) edge[dashed] node [right] {$\nexists F(f)$} (E')
(a) edge[dashed] node[above] {$$} (n1)
(b) edge[dashed] node[above] {$$} (n2)
(a') edge[dashed] node[below] {$$} (n1')
(b') edge[dashed] node[below] {$$} (n2')
(n1) edge node[above]{$e$} (n2)
(n1') edge node[above]{$e^\prime$} (n2');
\end{tikzpicture}
\]
In the symmetric monoidal bicategory $F\mathbf{Cospan}(\mathsf{FinSet})$, the above two $F$-decorated cospans are two morphisms - with \emph{no 2-morphism between them!} The culprit for this is the same as before - the requirement that the triangle of decorations in $\mathsf{Set}$ commute `on the nose'. In this paper we construct a similar symmetric monoidal bicategory $F$Csp which will remedy this situation, and whose decategorification will remedy the analogous problem for decorated cospan categories above. 

It is worth noting that in both examples involving sets and graphs above, we are decorating each finite set with extra \emph{stuff}. This phenomenon does not occur when the decorations only involve extra structure, and this has been utilized in other works \cite{BFP,BP,Yass}.

Our approach is to instead view the functor $F$ as a pseudofunctor $F \colon \textrm{A} \to \mathbf{Cat}$ and take advantage of the 2-categorical structure of $\mathbf{Cat}$. Given an object $a \in \mathsf{A}$, this then allows us to view $F(a)$ not as a \emph{set} of decorations on the object $a$ but as a \emph{category} of decorations on the object $a$. From this pseudofunctor, we construct a symmetric monoidal double category $F \mathbb{C}$sp. Double categories were first introduced by Ehresmann \cite{Ehresmann63, Ehresmann65} and symmetric monoidal double categories by Shulman \cite{Shul}. They have long been used in topology and other branches of pure mathematics \cite{Brown1,Brown2}.  More recently they have been used to study open dynamical systems \cite{LS} and open Markov processes \cite{BC}. While a mere category has only objects and morphisms, a double category has a few more types of entities:
\[
\begin{tikzpicture}[scale=1]
\node (D) at (-4,0.5) {$A$};
\node (E) at (-2,0.5) {$B$};
\node (F) at (-4,-1) {$C$};
\node (A) at (-2,-1) {$D$};
\node (B) at (-3,-0.25) {$\Downarrow a$};
\path[->,font=\scriptsize,>=angle 90]
(D) edge node [above]{$M$}(E)
(E) edge node [right]{$g$}(A)
(D) edge node [left]{$f$}(F)
(F) edge node [above]{$N$} (A);
\end{tikzpicture}
\]
We call $A, B, C$ and $D$ `objects', $f$ and $g$ `vertical 1-morphisms', $M$ and $N$ `horizontal 1-cells', and $a$ a `2-morphism'.   We can compose vertical 1-morphisms to get new vertical 1-morphisms and compose horizontal 1-cells to get new horizontal 1-cells.  We can compose the 2-morphisms in two ways: horizontally by setting squares side by side, and vertically by setting one on top of the other.   In a `strict' double category all these forms of composition are associative.  In a `pseudo' double category, horizontal 1-cells compose in a weakly associative manner: that is, the associative law holds only up to an invertible 2-morphism, called the `associator', which obeys a coherence law.

The symmetric monoidal double category $F \mathbb{C}$sp we construct has:
\begin{enumerate}
\item{objects as those of $\mathrm{A}$,}
\item{vertical 1-morphisms as morphisms of $\mathrm{A}$,}
\item{horizontal 1-cells as $F$-decorated cospans in $\mathrm{A}$
\[
\begin{tikzpicture}[scale=1.5]
\node (A) at (0,0) {$a_1$};
\node (B) at (1,0) {$b$};
\node (C) at (2,0) {$a_2$};
\node (D) at (3,0) {$d \in F(b)$};
\path[->,font=\scriptsize,>=angle 90]
(A) edge node[above]{$i$} (B)
(C) edge node[above]{$o$} (B);
\end{tikzpicture}
\]
and}
\item{2-morphisms as maps of cospans together with a natural transformation.
\[
\begin{tikzpicture}[scale=1.5]
\node (A) at (0,0.5) {$a_1$};
\node (A') at (0,-0.5) {$a_1'$};
\node (B) at (1,0.5) {$b$};
\node (B') at (1,-0.5) {$b^\prime$};
\node (C) at (2,0.5) {$a_2$};
\node (C') at (2,-0.5) {$a_2;$};
\node (D) at (3,0) {$1$};
\node (E) at (4,0.5) {$F(b)$};
\node (E') at (4,-0.5) {$F(b^\prime)$};
\node (F) at (3.65,0) {$\Swarrow \iota$};
\path[->,font=\scriptsize,>=angle 90]
(A) edge node[above]{$i$} (B)
(C) edge node[above]{$o$} (B)
(A) edge node {$$} (A')
(C) edge node {$$} (C')
(A') edge node[above]{$i^\prime$} (B')
(C') edge node[above]{$o^\prime$} (B')
(B) edge node [left] {$f$} (B')
(D) edge node [above] {$d$} (E)
(D) edge node [below] {$d^\prime$} (E')
(E) edge node [right] {$F(f)$} (E');
\end{tikzpicture}
\]
}
\end{enumerate}
Note that the natural transformation is a family of morphisms in the category $F(b^\prime)$ indexed by $1$, and so is the same as just a morphism $\iota \colon F(f)(d) \to d^\prime$ in $F(b^\prime)$. We will typically write the transformation $\iota$ in later sections as a morphism in $F(b^\prime)$ to conserve space. In the context of the example above involving graphs, this morphism $\iota$ will be a map between two possibly distinct sets of edges and thus already solves the problem.

This double category is in fact fibrant and so a result of Shulman \cite{Shul} allows us to extract from this symmetric monoidal double category $F \mathbb{C}$sp a symmetric monoidal bicategory $F \mathbf{Csp} \coloneqq H(F \mathbb{C}\textnormal{sp})$ as the `horizontal edge bicategory' of the double category $F \mathbb{C}$sp. This bicategory $F \mathbf{Csp}$ has:
\begin{enumerate}
\item{objects as those of $\mathrm{A}$,}
\item{morphisms as the horizontal 1-cells of $F\mathbb{C}$sp, which are $F$-decorated cospans of $\mathsf{A}$
\[
\begin{tikzpicture}[scale=1.5]
\node (A) at (0,0) {$a_1$};
\node (B) at (1,0) {$b$};
\node (C) at (2,0) {$a_2$};
\node (D) at (3,0) {$d \in F(b)$};
\path[->,font=\scriptsize,>=angle 90]
(A) edge node[above]{$i$} (B)
(C) edge node[above]{$o$} (B);
\end{tikzpicture}
\]
and}
\item{2-morphisms as maps of $F$-decorated cospans in $\mathsf{A}$ with the same feet together with a natrual transformation.
\[
\begin{tikzpicture}[scale=1.5]
\node (A) at (0,0) {$a_1$};
\node (B) at (1,1) {$b$};
\node (B') at (1,-1) {$b^\prime$};
\node (C) at (2,0) {$a_2$};
\node (D) at (3,0) {$1$};
\node (E) at (4,0.5) {$F(b)$};
\node (E') at (4,-0.5) {$F(b^\prime)$};
\node (F) at (3.65,0) {$\Swarrow \iota$};
\path[->,font=\scriptsize,>=angle 90]
(A) edge node[above]{$i$} (B)
(C) edge node[above]{$o$} (B)
(A) edge node[below]{$i^\prime$} (B')
(C) edge node[below]{$o^\prime$} (B')
(B) edge node [left] {$f$} (B')
(D) edge node [above] {$d$} (E)
(D) edge node [below] {$d^\prime$} (E')
(E) edge node [right] {$F(f)$} (E');
\end{tikzpicture}
\]
}
\end{enumerate}
This bicategory $F\mathbf{Csp}$ is superior to the original bicategory $F$Cospan$(A)$ constructed by the second author \cite{Cour} in that it has the 2-morphisms that one would expect such a bicategory to have, such as (\ref{no2morphism}). 

Finally, we can decategorify the symmetric monoidal bicategory $F \mathbf{Csp}$ to obtain a symmetric monoidal category $F \textnormal{Csp} \coloneqq D(F\mathbf{Csp})$. This symmetric monoidal category $F$Csp has:

\begin{enumerate}
\item{objects as those of $\mathsf{A}$ and}
\item{morphisms as isomorphism classes of $F$-decorated cospans in $\mathsf{A}$ with the same feet together with a natural isomorphism, where two $F$-decorated cospans are in the same isomorphism class if $f$ and $\iota$ are isomorphisms.
\[
\begin{tikzpicture}[scale=1.5]
\node (A) at (0,0) {$a_1$};
\node (B) at (1,1) {$b$};
\node (B') at (1,-1) {$b^\prime$};
\node (C) at (2,0) {$a_2$};
\node (D) at (3,0) {$1$};
\node (E) at (4,0.5) {$F(b)$};
\node (E') at (4,-0.5) {$F(b^\prime)$};
\node (F) at (3.5,0) {$\Swarrow \iota$};
\path[->,font=\scriptsize,>=angle 90]
(A) edge node[above]{$i$} (B)
(C) edge node[above]{$o$} (B)
(A) edge node[below]{$i^\prime$} (B')
(C) edge node[below]{$o^\prime$} (B')
(B) edge node [left] {$f$} (B')
(B) edge node [right] {$\sim$} (B')
(D) edge node [above] {$d$} (E)
(D) edge node [below] {$d^\prime$} (E')
(E) edge node [right] {$F(f)$} (E')
(E) edge node [left] {$\sim$} (E');
\end{tikzpicture}
\]
}
\end{enumerate}
This symmetric monoidal category $F$Csp has isomorphism classes as one would expect, and in the context of the example with finite sets and graphs above solves the problem in (\ref{distinctisoclasses}). In this particular example, $\iota \colon F(f)(d) \xrightarrow{\sim} d^\prime$ is a bijection between possibly different sets of edges which respects the source and target of each edge.

\subsection*{Outline}

{\chris Amend at the end}
The outline for the paper is as follows: In Section \ref{DecCospansDoubleCat}, we construct the symmetric monoidal double category $F\mathbb{C}\textnormal{sp}$, obtain the underlying symmetric monoidal bicategory $F \mathbf{Csp}$ via a result of Shulman, and decategorify this symmetric monoidal bicategory to obtain the symmetric monoidal category $F$Csp. In Section \ref{MapsDecCospansDoubleCat} we define maps between decorated cospan double categories which are double functors of an appropriate type. In Section \ref{some other section} we investigate when a symmetric lax monoidal pseudofunctor gives rise to a left adjoint via the Grothendieck construction, which naturally leads to another compositional framework known as `Structured cospans'. In Section \ref{EquivDoubleCats} we briefly review the framework Structured cospans and prove that the double categorical versions of decorated cospans and structured cospans are equivalent when both are appropriate. Finally, in Section \ref{Applications}, we present a few examples that can be realized with either framework and discuss the realized equivalences.

\subsection*{Acknowledgements}
Daniel Cicala, person that Christina emailed about lax slice category? {\chris did you use it after all?} No, I did not.
\end{comment}




\section{A symmetric monoidal double category of structured cospans}\label{sec:structuredcospans}

{\chris Moved it into a new section here, amend to make nice with surroundings}
In this section, we recall the double category of \emph{structured cospans}, a formalism introduced to the first two authors. One of the main goals of this paper is to provide a double equivalence between this double category and one of \emph{decorated cospans}, described in detail in the subsequent section.

Once again following the notation of Shulman \cite{Shul2}, given a double category $\mathbb{A}$, we write $_f \mathbb{A}_g(M,N)$ for the set of 2-morphisms in $\mathbb{A}$ of the form:
\[
  \xymatrix@-.5pc{
    A \ar[r]|{|}^{M}  \ar[d]_f \ar@{}[dr]|{\Downarrow a}&
    B\ar[d]^g\\
    C \ar[r]|{|}_N & D
  }
\]
We call $M$ and $N$ the \define{horizontal source and target} of the 2-morphism $a$, respectively, and likewise we call $f$ and $g$ the \define{vertical source and target} of the 2-morphism $a$, respectively. Thus $_f \mathbb{A}_g(M,N)$ denotes the set of 2-morphisms in $\mathbb{A}$ with horizontal source and target $M$ and $N$ and vertical source and target $f$ and $g$.
\begin{defn}
A (possibly lax or oplax) double functor $\mathbb{F} \colon \mathbb{A} \to \mathbb{X}$ is \define{full} (respectively, \define{faithful}) if $\mathbb{F}_0 \colon \mathbb{A}_0 \to \mathbb{X}_0$ is full (respectively, faithful) and each map $$\mathbb{F}_1 \colon _f \mathbb{A}_g(M,N) \to _{\mathbb{F}(f)} \mathbb{X}_{\mathbb{F}(g)}(\mathbb{F}(M),\mathbb{F}(N))$$ is surjective (respectively, injective).
\end{defn}
\begin{defn}
A (possibly lax or oplax) double functor $\mathbb{F} \colon \mathbb{A} \to \mathbb{X}$ is \define{essentially surjective} if we can simultaneously make the following choices:
\begin{enumerate}
\item{For each object $x \in \mathbb{X}$, we can find an object $a \in \mathbb{A}$ together with a vertical 1-isomorphism $\alpha_x \colon \mathbb{F}(a) \to x$, and}
\item{For each horizontal 1-cell $N \colon x_1 \tobar x_2$  of $\mathbb{X}$, we can find a horizontal 1-cell $M \colon a_1 \tobar a_2$ of $\mathbb{A}$ and a 2-isomorphism $a_{N}$ of $\mathbb{X}$ as in the following diagram:
\[
  \xymatrix@-.5pc{
    \mathbb{F}(a_1) \ar[r]|{|}^{\mathbb{F}(M)}  \ar[d]_{\alpha_{x_1}} \ar@{}[dr]|{\Downarrow a_N}&
    \mathbb{F}(a_1) \ar[d]^{\alpha_{x_2}}\\
    x_1 \ar[r]|{|}_N & x_2
  }
\]
}
\end{enumerate}
\end{defn}
\begin{defn}
A double functor $\mathbb{F} \colon \mathbb{A} \to \mathbb{X}$ is \define{strong} if the comparison and unit constraints are globular isomorphisms, meaning that for each composable pair of horizontal 1-cells $M$ and $N$ we have a natural isomorphism $$\mathbb{F}_{M,N} \colon \mathbb{F}(M) \odot \mathbb{F}(N) \xrightarrow{\sim} \mathbb{F}(M \odot N)$$and for each object $a \in \mathbb{A}$ a natural isomorphism $$\mathbb{F}_a \colon \hat{U}_{\mathbb{F}(a)} \xrightarrow{\sim} \mathbb{F}(U_a).$$
\end{defn}

Formally, two strong double functors between two double categories form a \emph{double equivalence} if {\chris their two-way composite is isomorphic to the identity, better expressed, possibly in a definition environment}. Similarly to ordinary equivalence of categories, {\chris cite shulman's theorem here [Shulman,7.8]} allows us to use the following equivalent characterization.

\begin{thm}\label{ShulDubEquiv}
A strong double functor $\mathbb{F} \colon \mathbb{A} \to \mathbb{X}$ is part of a double equivalence if and only if it is full, faithful and essentially surjective. {\chris on objects?}
\end{thm}

\begin{prop}
Let $\mathbb{A}$ and $\mathbb{X}$ be symmetric monoidal double categories and let $\mathbb{F} \colon \mathbb{A} \to \mathbb{X}$ be a symmetric monoidal strong double functor. If $\mathbb{F}$ is part of a double equivalence, then $\mathbb{F}$ is in fact part of a symmetric monoidal double equivalence, and $\mathbb{A}$ and $\mathbb{X}$ are equivalent as symmetric monoidal double categories.
\end{prop}

Our next goal is to show that the symmetric monoidal double category $F\mathbb{C}\textnormal{sp}$ of Section \ref{DecCospansDoubleCat} is equivalent as a symmetric monoidal double category to the symmetric monoidal double category $_L \mathbb{C}\textnormal{sp}(\mathsf{X})$ obtained using structured cospans. {\chris Wait, why do you need to consider them as monoidal in the assumptions of the theorem?}

\begin{thm}\label{SC}
Given a category $\mathsf{X}$ with finite colimits and a category $\mathsf{A}$ with finite coproducts and a finite coproduct preserving functor $L \colon \mathsf{A} \to \mathsf{X}$ with $\mathsf{A}$ and $\mathsf{X}$ regarded as cocartesian monoidal categories, there exists a symmetric monoidal double category $_L \mathbb{C}\textnormal{sp}(\mathsf{X})$ which has:
\begin{enumerate}
\item{objects given by objects of $\mathsf{A}$,}
\item{vertical 1-morphisms given by morphisms of $\mathsf{A}$,}
\item{horizontal 1-cells given by cospans of $\mathsf{X}$ of the form:
\begin{equation}\label{eq:structuredcospan}
\begin{tikzpicture}[scale=1.5]
\node (A) at (0,0) {$L(c)$};
\node (B) at (1,0) {$x$};
\node (C) at (2,0) {$L(c^\prime)$};
\path[->,font=\scriptsize,>=angle 90]
(A) edge node[above]{$$} (B)
(C) edge node[above]{$$} (B);
\end{tikzpicture}
\end{equation}
and}
\item{2-morphisms given by maps of cospans of $\mathsf{X}$ of the form:
\begin{equation}\label{eq:2cellsStrCsp}
\begin{tikzpicture}[scale=1.5]
\node (A) at (0,0) {$L(c_1)$};
\node (B) at (1,0) {$x$};
\node (C) at (2,0) {$L(c_2)$};
\node (A') at (0,-1) {$L(c_1^\prime)$};
\node (B') at (1,-1) {$x^\prime$};
\node (C') at (2,-1) {$L(c_2^\prime)$};
\path[->,font=\scriptsize,>=angle 90]
(A) edge node[above]{$$} (B)
(C) edge node[above]{$$} (B)
(A') edge node[above]{$$} (B')
(C') edge node[above]{$$} (B')
(A) edge node [left]{$L(f)$} (A')
(B) edge node [left]{$\alpha$} (B')
(C) edge node [left]{$L(g)$} (C');
\end{tikzpicture}
\end{equation}
Composition of horizontal 1-cells and 2-morphisms is given by pushouts in $\mathsf{X}$ and tensoring of objects is under binary coproducts in $\mathsf{A}$.
}
\end{enumerate}
\end{thm}
\begin{proof}
See the first two authors' work on structured cospans \cite{BC2}.
\end{proof}
{\chris Below, I commented out a nice short recalled version of decorated cospans, since now their definition is later. Use if we want to recall at some point later}
%In the previous section, we started with a symmetric lax monoidal pseudofunctor $F \colon \mathsf{A} \to \mathbf{Cat}$ and created a symmetric monoidal double category $F\mathbb{C}\textnormal{sp}$ which has:
%\begin{enumerate}
%\item{objects of $\mathsf{A}$ as objects,}
%\item{morphisms of $\mathsf{A}$ as vertical 1-morphisms,}
%\item{horizontal 1-cells are given by $F$-decorated cospans, which are pairs:
%\[
%\begin{tikzpicture}[scale=1.5]
%\node (A) at (0,0) {$c_1$};
%\node (B) at (1,0) {$c$};
%\node (C) at (2,0) {$c_2$};
%\node (D) at (3,0) {$x \in F(c)$};
%\path[->,font=\scriptsize,>=angle 90]
%(A) edge node[above]{$i$} (B)
%(C) edge node[above]{$o$} (B);
%\end{tikzpicture}
%\]
%and}
%\item{2-morphisms are given by maps of cospans in $\mathsf{A}$:
%\[
%\begin{tikzpicture}[scale=1.5]
%\node (A) at (0,0) {$c_1$};
%\node (B) at (1,0) {$c$};
%\node (C) at (2,0) {$c_2$};
%\node (A') at (0,-1) {$c_1^\prime$};
%\node (B') at (1,-1) {$c^\prime$};
%\node (C') at (2,-1) {$c_2^\prime$};
%\node (D) at (3,0) {$x \in F(c)$};
%\node (D') at (3,-1) {$x^\prime \in F(c^\prime)$};
%\path[->,font=\scriptsize,>=angle 90]
%(A) edge node[above]{$i$} (B)
%(C) edge node[above]{$o$} (B)
%(A') edge node[above]{$i^\prime$} (B')
%(C') edge node[above]{$o^\prime$} (B')
%(A) edge node [left]{$f$} (A')
%(B) edge node [left]{$h$} (B')
%(C) edge node [left]{$g$} (C');
%\end{tikzpicture}
%\]
%together with a morphism $\iota \colon F(h)(d) \to d^\prime$ in $F(c^\prime)$.}
%\end{enumerate}






\section{A symmetric monoidal double category of decorated cospans}\label{DecCospansDoubleCat}
In this section we build the symmetric monoidal double category $F \mathbb{C}$sp mentioned in the introduction. Then we obtain an underlying symmetric monoidal bicategory $F \mathbf{Csp}$ using a result of Shulman \cite{Shul} and then finally decategorify the obtained symmetric monoidal bicategory to obtain a symmetric monoidal category $F$Csp which is a generalization of Fong's original decorated cospan category \cite{Fong}.
\begin{thm}\label{thm:decoratedcospans}
Let $\mathsf{A}$ be a category with finite colimits and $F \colon \mathsf{A} \to \mathbf{Cat}$ a lax monoidal pseudofunctor. Then there exists a (pseudo) double category $F\mathbb{C}$sp which has:
\begin{enumerate}
\item{objects as those of $\mathsf{A}$,}
\item{vertical 1-morphisms as morphisms of $\mathsf{A}$,}
\item{horizontal 1-cells as $F$-decorated cospans in $\mathsf{A}$ which are pairs:
\[
\begin{tikzpicture}[scale=1.5]
\node (A) at (0,0) {$a_1$};
\node (B) at (1,0) {$b$};
\node (C) at (2,0) {$a_2$};
\node (D) at (3,0) {$d \in F(b)$};
\path[->,font=\scriptsize,>=angle 90]
(A) edge node[above]{$i$} (B)
(C) edge node[above]{$o$} (B);
\end{tikzpicture}
\]
and}
\item{2-morphisms as maps of $F$-decorated cospans in $\mathsf{A}$
\begin{equation}\label{eq:FCsp2morph}
\begin{tikzpicture}[scale=1.5]
\node (A) at (0,0.5) {$a_1$};
\node (A') at (0,-0.5) {$a_1^\prime$};
\node (B) at (1,0.5) {$b$};
\node (C) at (2,0.5) {$a_2$};
\node (C') at (2,-0.5) {$a_2^\prime$};
\node (D) at (1,-0.5) {$b^\prime$};
\node (E) at (3,0.5) {$d \in F(b)$};
\node (F) at (3,-0.5) {$d^\prime \in F(b^\prime)$};
\path[->,font=\scriptsize,>=angle 90]
(A) edge node[above]{$i$} (B)
(C) edge node[above]{$o$} (B)
(A) edge node[left]{$f$} (A')
(C) edge node[left]{$g$} (C')
(A') edge node[above] {$i'$} (D)
(C') edge node[above] {$o'$} (D)
(B) edge node [left] {$h$} (D);
\end{tikzpicture}
\end{equation}
together with a morphism $\iota \colon F(h)(d) \to d^\prime$ in $F(b^\prime)$.}
\end{enumerate}
\end{thm}
\begin{proof}
The unit structure functor $U \colon F\mathbb{C}\textnormal{sp}_0 \to F\mathbb{C}\textnormal{sp}_1$ is defined on objects as: 
\[
\begin{tikzpicture}[scale=1.5]
\node (E) at (-1,0) {$a$};
\node (F) at (-.5,0) {$\mapsto$};
\node (A) at (0,0) {$a$};
\node (B) at (1,0) {$a$};
\node (C) at (2,0) {$a$};
\node (D) at (3,0) {$!_a \in F(a)$};
\path[->,font=\scriptsize,>=angle 90]
(A) edge node[above]{$1$} (B)
(C) edge node[above]{$1$} (B);
\end{tikzpicture}
\]
where $!_a \in F(a)$ is the trivial decoration on $a$ given by the composition of the unique map $F(!) \colon F(0) \to F(a)$ and the morphism $\phi \colon 1 \to F(0)$  which comes from the structure of the lax monoidal pseudofunctor $F \colon \mathsf{A} \to \mathbf{Cat}$. For morphisms, the structure functor $U$ is defined as:
\[
\begin{tikzpicture}[scale=1.5]
\node (G) at (-1,0.5) {$a$};
\node (G') at (-1,-0.5) {$a^\prime$};
\node (A) at (0,0.5) {$a$};
\node (A') at (0,-0.5) {$a^\prime$};
\node (B) at (1,0.5) {$a$};
\node (C) at (2,0.5) {$a$};
\node (C') at (2,-0.5) {$a^\prime$};
\node (D) at (1,-0.5) {$a^\prime$};
\node (E) at (3,0.5) {$!_a \in F(a)$};
\node (F) at (3,-0.5) {$!_{a^\prime} \in F(a^\prime)$};
\node (H) at (-0.5,0) {$\mapsto$};
\path[->,font=\scriptsize,>=angle 90]
(A) edge node[above]{$1$} (B)
(C) edge node[above]{$1$} (B)
(A) edge node[left]{$f$} (A')
(C) edge node[left]{$f$} (C')
(A') edge node[above] {$1$} (D)
(C') edge node[above] {$1$} (D)
(B) edge node [left] {$f$} (D)
(G) edge node [left] {$f$} (G');
\end{tikzpicture}
\]
together with the morphism $\iota_{!_f} = F(f) F(!) \phi \colon 1 \to F(a^\prime)$. We also have source and target structure functors $S, T \colon F\mathbb{C}\textnormal{sp}_1 \to F\mathbb{C}\textnormal{sp}_0$ where the source of the horizontal 1-cell
\[
\begin{tikzpicture}[scale=1.5]
\node (A) at (0,0) {$a_1$};
\node (B) at (1,0) {$b$};
\node (C) at (2,0) {$a_2$};
\node (D) at (3,0) {$d \in F(b)$};
\path[->,font=\scriptsize,>=angle 90]
(A) edge node[above]{$i$} (B)
(C) edge node[above]{$o$} (B);
\end{tikzpicture}
\]
is the object $a_1$ in $\mathsf{A}$ and the source of the 2-morphism
\[
\begin{tikzpicture}[scale=1.5]
\node (A) at (0,0.5) {$a$};
\node (A') at (0,-0.5) {$a^\prime$};
\node (B) at (1,0.5) {$b$};
\node (C) at (2,0.5) {$a_2$};
\node (C') at (2,-0.5) {$a_2^\prime$};
\node (D) at (1,-0.5) {$b^\prime$};
\node (E) at (3,0.5) {$d \in F(b)$};
\node (F) at (3,-0.5) {$d^\prime \in F(b^\prime)$};
\path[->,font=\scriptsize,>=angle 90]
(A) edge node[above]{$i$} (B)
(C) edge node[above]{$o$} (B)
(A) edge node[left]{$f$} (A')
(C) edge node[left]{$g$} (C')
(A') edge node[above] {$i'$} (D)
(C') edge node[above] {$o'$} (D)
(B) edge node [left] {$h$} (D);
\end{tikzpicture}
\]
$$\iota \colon F(h)(d) \to d^\prime$$
is the source of the underlying map of cospans in $\mathsf{A}$, namely the morphism $f$ in $\mathsf{A}$; the target structure functor is defined similarly. These structure functors satisfy the equations $$SU(a)=1(a)=TU(a)$$for all objects $a$ of $\mathsf{A}$.

Given two composable horizontal 1-cells $M$ and $N$:
\[
\begin{tikzpicture}[scale=1.5]
\node (A) at (0,0) {$a_1$};
\node (B) at (1,0) {$b$};
\node (C) at (2,0) {$a_2$};
\node (D) at (1,-0.5) {$d \in F(b)$};
\node (E) at (3,0) {$a_2$};
\node (F) at (4,0) {$b^\prime$};
\node (G) at (5,0) {$a_3$};
\node (H) at (4,-0.5) {$d^\prime \in F(b^\prime)$};
\path[->,font=\scriptsize,>=angle 90]
(A) edge node[above]{$i$} (B)
(C) edge node[above]{$o$} (B)
(E) edge node[above]{$i^\prime$} (F)
(G) edge node[above]{$o^\prime$} (F);
\end{tikzpicture}
\]
the composite $N \odot M$ is given by:
\[
\begin{tikzpicture}[scale=1.5]
\node (A) at (0,0) {$a_1$};
\node (B) at (1,1) {$b$};
\node (C) at (2,0) {$a_2$};
\node (D) at (3,1) {$b^\prime$};
\node (E) at (4,0) {$a_3$};
\node (F) at (2,2) {$b+b^\prime$};
\node (G) at (2,3) {$b+_{a_2} b^\prime$};
\path[->,font=\scriptsize,>=angle 90]
(A) edge node[above]{$i$} (B)
(C) edge node[above]{$o$} (B)
(C) edge node [above] {$i^\prime$} (D)
(E) edge node [above] {$o^\prime$} (D)
(B) edge node [above] {$j$} (F)
(D) edge node [above] {$j^\prime$} (F)
(F) edge node [left] {$\psi$} (G)
(A) edge[bend left] node [left] {$\psi j i$} (G)
(E) edge[bend right] node [right] {$\psi j^\prime o^\prime$} (G);
\end{tikzpicture}
\]
with the corresponding decoration of the apex $\hat{d} \in F(b+_{a_2} b^\prime)$ given by:
$$1 \xrightarrow{\lambda^{-1}} 1 \times 1 \xrightarrow{d \times d^\prime} F(b) \times F(b^\prime) \xrightarrow{\phi_{b,b^\prime}} F(b+b^\prime) \xrightarrow{F(\psi)} F(b+_{a_2}b^\prime)$$
where $\psi \colon b + b^\prime \to b+_{a_2} b^\prime$ is the natural map from the coproduct to the pushout and $\phi_{b,b^\prime} \colon F(b) \times F(b^\prime) \to F(b+b^\prime)$ is the natural transformation coming from the structure of the lax monoidal pseudofunctor $F \colon \mathsf{A} \to \mathbf{Cat}$. Denoting the first and second of these horizontal 1-cells as $M$ and $N$, respectively, the source and target structure functors satisfy the equations $S(N \odot M)=S(M)$ and $T(N \odot M)=T(N)$.

Given three composable horizontal 1-cells $M_1, M_2$ and $M_3$:
\[
\begin{tikzpicture}[scale=1.5]
\node (A) at (0,0) {$a_1$};
\node (B) at (1,0) {$b$};
\node (C) at (2,0) {$a_2$};
\node (D) at (1,-0.5) {$d_{M_1} \in F(b)$};
\node (E) at (3,0) {$a_2$};
\node (F) at (4,0) {$b^\prime$};
\node (G) at (5,0) {$a_3$};
\node (H) at (4,-0.5) {$d_{M_2} \in F(b^\prime)$};
\node (I) at (6,0) {$a_3$};
\node (J) at (7,0) {$b''$};
\node (K) at (8,0) {$a_4$};
\node (L) at (7,-0.5) {$d_{M_3} \in F(b'')$};
\path[->,font=\scriptsize,>=angle 90]
(A) edge node[above]{$i$} (B)
(C) edge node[above]{$o$} (B)
(E) edge node[above]{$i^\prime$} (F)
(G) edge node[above]{$o^\prime$} (F)
(I) edge node[above]{$i''$} (J)
(K) edge node[above]{$o''$} (J);
\end{tikzpicture}
\]
we get a natural isomorphism $a_{M_1,M_2,M_3} \colon (M_1 \odot M_2) \odot M_3 \to M_1 \odot (M_2 \odot M_3)$ which is a globular 2-morphism given by a map of cospans $(\id_{a_1},\sigma,\id_{a_4})$:
\[
\begin{tikzpicture}[scale=1.5]
\node (A) at (0,0.5) {$a_1$};
\node (A') at (0,-0.5) {$a_1$};
\node (B) at (1.5,0.5) {$(b+_{a_2} b^\prime)+_{a_3} b''$};
\node (C) at (3,0.5) {$a_4$};
\node (C') at (3,-0.5) {$a_4$};
\node (D) at (1.5,-0.5) {$b+_{a_2}(b^\prime +_{a_3} b'')$};
\node (E) at (5.5,0.5) {$d_{(M_1 \odot M_2) \odot M_3} \in F((b+_{a_2} b^\prime)+_{a_3} b'')$};
\node (F) at (5.5,-0.5) {$d_{M_1 \odot (M_2 \odot M_3)} \in F(b+_{a_2} (b^\prime +_{a_3} b''))$};
\path[->,font=\scriptsize,>=angle 90]
(A) edge node[above]{$$} (B)
(C) edge node[above]{$$} (B)
(A) edge node[left]{$1$} (A')
(C) edge node[left]{$1$} (C')
(A') edge node {$$} (D)
(C') edge node {$$} (D)
(B) edge node [left] {$a$} (D);
\end{tikzpicture}
\]
with the decorations on the cospan's apices given by:
$$ d_{(M_1 \odot M_2) \odot M_3} \coloneqq 1 \xrightarrow{\zeta_1} F(b+_{a_2} b^\prime) \times F(b'') \xrightarrow{\phi_{b+_{a_2} b^\prime, b''}} F((b+_{a_2}b^\prime) +b'') \xrightarrow{F(j_{b+_{a_2} b^\prime,b''})} F((b+_{a_2} b^\prime)+_{a_3} b'')$$ $$\zeta_1 = d_{M_3} \lambda^{-1} F(j_{b,b^\prime}) \phi_{b,b^\prime} (d_{M_1} \times d_{M_2}) \lambda^{-1}$$
and
$$ d_{M_1 \odot (M_2 \odot M_3)} \coloneqq 1 \xrightarrow{\zeta_2} F(b) \times F(b^\prime +_{a_3} b'') \xrightarrow{\phi_{b, b^\prime +_{a_3} b''}} F(b+(b^\prime +_{a_3} b'')) \xrightarrow{F(j_{b,b^\prime +_{a_3} b''})} F(b+_{a_2} (b^\prime+_{a_3} b''))$$ $$\zeta_2 = d_{M_1} \lambda^{-1} F(j_{b^\prime,b''}) \phi_{b^\prime,b''} (d_{M_2} \times d_{M_3}) \lambda^{-1}$$
together with the isomorphism $\iota_a \colon F(a)(d_{(M_1 \odot M_2) \odot M_3}) \to d_{M_1 \odot (M_2 \odot M_3)}$. Note that the map $a \colon (b+_{a_2} b')+_{a_3}b'' \to b+_{a_2}(b'+_{a_3}b'')$ is the universal map between two colimits of the same diagram. We also have left and right unitors where given a horizontal 1-cell $M$:
\[
\begin{tikzpicture}[scale=1.5]
\node (A) at (0,0) {$a_1$};
\node (B) at (1,0) {$b$};
\node (C) at (2,0) {$a_2$};
\node (D) at (3,0) {$d \in F(b)$};
\path[->,font=\scriptsize,>=angle 90]
(A) edge node[above]{$i$} (B)
(C) edge node[above]{$o$} (B);
\end{tikzpicture}
\]
if we, say, compose with the identity horizontal 1-cell of $a_2$ on the right:
\[
\begin{tikzpicture}[scale=1.5]
\node (A) at (0,0) {$a_1$};
\node (B) at (1,0) {$b$};
\node (C) at (2,0) {$a_2$};
\node (D) at (1,-0.5) {$d \in F(b)$};
\node (E) at (3,0) {$a_2$};
\node (F) at (4,0) {$a_2$};
\node (G) at (5,0) {$a_2$};
\node (H) at (4,-0.5) {$!_{a_2} \in F(a_2)$};
\path[->,font=\scriptsize,>=angle 90]
(A) edge node[above]{$i$} (B)
(C) edge node[above]{$o$} (B)
(E) edge node[above]{$1$} (F)
(G) edge node[above]{$1$} (F);
\end{tikzpicture}
\]
where $!_{a_2} = F(!)  \phi \colon 1 \to F(a_2)$ is the trivial decoration on $a_2$, composing these then gives:
\[
\begin{tikzpicture}[scale=1.5]
\node (A) at (0,0) {$a_1$};
\node (B) at (1,0) {$b+_{a_2} a_2$};
\node (C) at (2,0) {$a_2$};
\node (D) at (3.5,0) {$d_{M+!_{a_2}} \in F(b +_{a_2} a_2)$};
\path[->,font=\scriptsize,>=angle 90]
(A) edge node[above]{$j \psi_b i$} (B)
(C) edge node[above]{$j \psi_{a_2}$} (B);
\end{tikzpicture}
\]
where $\psi_b \colon b \to b+a_2$ is the natural map into the coproduct and likewise for $\psi_{a_2}$ and $j \colon b+a_2 \to b+_{a_2} a_2$ is the natural map from the coproduct to the pushout. The decoration $d_{M+!_{a_2}} \colon 1 \to F(b+_{a_2} a_2)$ is given by: $$1 \xrightarrow{\lambda^{-1}} 1 \times 1 \xrightarrow{d_{M} \times !_{a_2}} F(b) \times F(a_2) \xrightarrow{\phi_{b,a_2}} F(b+a_2) \xrightarrow{F(j_{b,a_2})} F(b+_{a_2} a_2).$$ We then have that the right unitor $R \colon M \odot 1_{a_2} \xrightarrow{\sim} M$ is given by the globular 2-morphism $(\id_{a_1},r,\id_{a_2})$ from the above composite to $M$:
\[
\begin{tikzpicture}[scale=1.5]
\node (A) at (0,0.5) {$a_1$};
\node (A') at (0,-0.5) {$a_1$};
\node (B) at (1.5,0.5) {$b+_{a_2} a_2$};
\node (C) at (3,0.5) {$a_2$};
\node (C') at (3,-0.5) {$a_2$};
\node (D) at (1.5,-0.5) {$b$};
\node (E) at (4.5,0.5) {$d_{M+!_{a_2}} \in F(b+_{a_2} a_2)$};
\node (F) at (4.5,-0.5) {$d_{M} \in F(b)$};
\path[->,font=\scriptsize,>=angle 90]
(A) edge node[above]{$j \psi_b i$} (B)
(C) edge node[above]{$j \psi_{a_2}$} (B)
(A) edge node[left]{$\id_{a_1}$} (A')
(C) edge node[left]{$\id_{a_2}$} (C')
(A') edge node [above]{$i$} (D)
(C') edge node [above]{$o$} (D)
(B) edge node [left] {$r$} (D);
\end{tikzpicture}
\]
where $r \colon b+_{a_2} a_2 \xrightarrow{\sim} b$ is a universal map together with the isomorphism $\iota_r \colon F(r)(d_{M+!_{a_2}}) \to d_M$. The left unitor is similar. The source and target functor applied to the left and right unitors and associators yield identities. The left and right unitors together with the associator satisfy the standard pentagon and triangle identities of a monoidal category or bicategory. Finally, for the interchange law, given four 2-morphisms $\alpha, \beta, \alpha^\prime$ and $\beta^\prime$:
\[
\begin{tikzpicture}[scale=1.5]
\node (A) at (0,0.5) {$a_1$};
\node (A') at (0,-0.5) {$a_1^\prime$};
\node (B) at (1,0.5) {$b$};
\node (C) at (2,0.5) {$a_2$};
\node (C') at (2,-0.5) {$a_2^\prime$};
\node (D) at (1,-0.5) {$b^\prime$};
\node (E) at (3,0.5) {$d_{M_1} \in F(b)$};
\node (F) at (3,-0.5) {$d_{M_2} \in F(b^\prime)$};
\node (G) at (4,0.5) {$a_2$};
\node (H) at (5,0.5) {$c$};
\node (I) at (6,0.5) {$a_3$};
\node (G') at (4,-0.5) {$a_2^\prime$};
\node (H') at (5,-0.5) {$c^\prime$};
\node (I') at (6,-0.5) {$a_3^\prime$};
\node (J) at (7,0.5) {$d_{N_1} \in F(c)$};
\node (K) at (7,-0.5) {$d_{N_2} \in F(c^\prime)$};
\node (L) at (1,-1) {$\iota_\alpha \colon F(h_1)(d_{M_1}) \to d_{M_2}$};
\node (M) at (5,-1) {$\iota_\beta \colon F(h_2)(d_{N_1}) \to d_{N_2}$};
\node (A'') at (0,-1.5) {$a_1^\prime$};
\node (A''') at (0,-2.5) {$a_1''$};
\node (B'') at (1,-1.5) {$b^\prime$};
\node (C'') at (2,-1.5) {$a_2^\prime$};
\node (C''') at (2,-2.5) {$a_2''$};
\node (D'') at (1,-2.5) {$b''$};
\node (E'') at (3,-1.5) {$d_{M_2} \in F(b^\prime)$};
\node (F'') at (3,-2.5) {$d_{M_3} \in F(b'')$};
\node (G'') at (4,-1.5) {$a_2^\prime$};
\node (H'') at (5,-1.5) {$c^\prime$};
\node (I'') at (6,-1.5) {$a_3^\prime$};
\node (G''') at (4,-2.5) {$a_2''$};
\node (H''') at (5,-2.5) {$c''$};
\node (I''') at (6,-2.5) {$a_3''$};
\node (J'') at (7,-1.5) {$d_{N_2} \in F(b^\prime)$};
\node (K'') at (7,-2.5) {$d_{N_3} \in F(b'')$};
\node (L'') at (1,-3) {$\iota_{\alpha^\prime} \colon F(h_1^\prime)(d_{M_2}) \to d_{M_3}$};
\node (M'') at (5,-3) {$\iota_{\beta^\prime} \colon F(h_2^\prime)(d_{N_2}) \to d_{N_3}$};
\path[->,font=\scriptsize,>=angle 90]
(A) edge node[above]{$i_1$} (B)
(C) edge node[above]{$o_1$} (B)
(A) edge node[left]{$f$} (A')
(C) edge node[left]{$g$} (C')
(A') edge node[above] {$i_1^\prime$} (D)
(C') edge node[above] {$o_1^\prime$} (D)
(B) edge node [left] {$h_1$} (D)
(G) edge node [above] {$i_2$} (H)
(G) edge node [left] {$g$} (G')
(H) edge node [left] {$h_2$} (H')
(G') edge node [above] {$i_2^\prime$} (H')
(I) edge node [above] {$o_2$} (H)
(I) edge node [left] {$k$} (I')
(I') edge node [above] {$o_2^\prime$} (H')
(A'') edge node[above]{$i_1^\prime$} (B'')
(C'') edge node[above]{$o_1^\prime$} (B'')
(A'') edge node[left]{$f^\prime$} (A''')
(C'') edge node[left]{$g^\prime$} (C''')
(A''') edge node[above] {$i_1''$} (D'')
(C''') edge node[above] {$o_1''$} (D'')
(B'') edge node [left] {$h_1^\prime$} (D'')
(G'') edge node [above] {$i_2^\prime$} (H'')
(G'') edge node [left] {$g^\prime$} (G''')
(H'') edge node [left] {$h_2^\prime$} (H''')
(G''') edge node [above] {$i_2''$} (H''')
(I'') edge node [above] {$o_2^\prime$} (H'')
(I'') edge node [left] {$k^\prime$} (I''')
(I''') edge node [above] {$o_2''$} (H''');
\end{tikzpicture}
\]
if we first compose horizontally we obtain:
\[
\begin{tikzpicture}[scale=1.5]
\node (A) at (0,0.5) {$a_1$};
\node (A') at (0,-0.5) {$a_1^\prime$};
\node (B) at (1.5,0.5) {$b+_{a_2} c$};
\node (C) at (3,0.5) {$a_3$};
\node (C') at (3,-0.5) {$a_3^\prime$};
\node (D) at (1.5,-0.5) {$b^\prime +_{a_2^\prime} c^\prime$};
\node (E) at (4.5,0.5) {$d_{M_1 \odot N_1} \in F(b+_{a_2} c)$};
\node (F) at (4.5,-0.5) {$d_{M_2 \odot N_2} \in F(b^\prime +_{a_2^\prime} c^\prime)$};
\node (G) at (1.5,-1) {$\iota_{\alpha \odot \beta} \colon F(h_1 +_g h_2)(d_{M_1 \odot N_1}) \to d_{M_1^\prime \odot N_1^\prime}$};
\node (A'') at (0,-1.5) {$a_1^\prime$};
\node (A''') at (0,-2.5) {$a_1''$};
\node (B'') at (1.5,-1.5) {$b^\prime +_{a_2^\prime} c^\prime$};
\node (C'') at (3,-1.5) {$a_3^\prime$};
\node (C''') at (3,-2.5) {$a_3''$};
\node (D'') at (1.5,-2.5) {$b'' +_{a_2''} c''$};
\node (E'') at (4.5,-1.5) {$d_{M_2 \odot N_2} \in F(b^\prime+_{a_2^\prime} c^\prime)$};
\node (F'') at (4.5,-2.5) {$d_{M_3 \odot N_3} \in F(b'' +_{a_2''} c'')$};
\node (G'') at (1.5,-3) {$\iota_{\alpha^\prime \odot \beta^\prime} \colon F(h_1^\prime +_{g^\prime} h_2^\prime)(d_{M_2 \odot N_2}) \to d_{M_3 \odot N_3}.$};
\path[->,font=\scriptsize,>=angle 90]
(A) edge node[above]{$j \psi_{a_1} i_1$} (B)
(C) edge node[above]{$j \psi_{a_3} o_2$} (B)
(A) edge node[left]{$f$} (A')
(C) edge node[left]{$k$} (C')
(A') edge node [above]{$j \psi_{a_1^\prime} i_1^\prime$} (D)
(C') edge node [above]{$j \psi_{a_3^\prime} o_2^\prime$} (D)
(B) edge node [left] {$h_1 +_g h_2$} (D)
(A'') edge node[above]{$j \psi_{a_1^\prime} i_1^\prime$} (B'')
(C'') edge node[above]{$j \psi_{a_3^\prime} o_2^\prime$} (B'')
(A'') edge node[left]{$f^\prime$} (A''')
(C'') edge node[left]{$k^\prime$} (C''')
(A''') edge node [above]{$j \psi_{a_1''} i_1''$} (D'')
(C''') edge node [above]{$j \psi_{a_3''} o_2''$} (D'')
(B'') edge node [left] {$h_1^\prime +_{g^\prime} h_2^\prime$} (D'');
\end{tikzpicture}
\]
To obtain the morphism of decorations for a horizontal composite, we have as initial data:
\[
\begin{tikzpicture}[scale=1.5]
\node (A) at (4,0) {$\iota_\alpha \Swarrow$};
\node (D) at (3,0) {$1$};
\node (E) at (4.5,0.5) {$F(b)$};
\node (E') at (4.5,-0.5) {$F(b^\prime)$};
\node (A') at (6.5,0) {$\iota_\beta \Swarrow$};
\node (D') at (5.5,0) {$1$};
\node (E'') at (7,0.5) {$F(c)$};
\node (E''') at (7,-0.5) {$F(c^\prime)$};
\path[->,font=\scriptsize,>=angle 90]
(D) edge node [above] {$d_{M_1}$} (E)
(D) edge node [below] {$d_{M_2}$} (E')
(E) edge node [right] {$F(h_1)$} (E')
(D') edge node [above] {$d_{N_1}$} (E'')
(D') edge node [below] {$d_{N_2}$} (E''')
(E'') edge node [right] {$F(h_2)$} (E''');
\end{tikzpicture}
\]
These two 2-morphisms $\iota_\alpha$ and $\iota_\beta$ are two 2-morphisms in the monoidal 2-category $(\mathbf{Cat},\times,1)$ and so we can tensor them which results in:
\[
\begin{tikzpicture}[scale=1.5]
\node (A) at (4.75,0) {$\iota_\alpha \times \iota_\beta \Swarrow$};
\node (D) at (3,0) {$1 \xrightarrow{\lambda^{-1}} 1 \times 1$};
\node (E) at (5.5,0.5) {$F(b) \times F(c)$};
\node (E') at (5.5,-0.5) {$F(b^\prime) \times F(c^\prime)$};
\node (B) at (7.5,0.5) {$F(b+c)$};
\node (B') at (7.5,-0.5) {$F(b^\prime + c^\prime)$};
\node (C) at (9.25,0.5) {$F(b+_{a_2} c)$};
\node (C') at (9.25,-0.5) {$F(b^\prime +_{a_2^\prime} c^\prime)$};
\path[->,font=\scriptsize,>=angle 90]
(E) edge node [above] {$\phi_{b,c}$} (B)
(E') edge node [above] {$\phi_{b^\prime,c^\prime}$} (B')
(B) edge node [above] {$F(j_{b,c})$} (C)
(B') edge node [above] {$F(j_{b^\prime,c^\prime})$} (C')
(C) edge node [right] {$F(h_1 +_g h_2)$} (C')
(B) edge node [right] {$F(h_1 + h_2)$} (B')
(D) edge node [above] {$d_{M_1} \times d_{N_1}$} (E)
(D) edge node [below] {$d_{M_2} \times d_{N_2}$} (E')
(E) edge node [right] {$F(h_1) \times F(h_2)$} (E');
\end{tikzpicture}
\]
where the middle square commutes since $F$ is a lax monoidal pseudofunctor and the right square commutes as the underlying diagram commutes. The decorations $d_{M_1 \odot N_1}$ and $d_{M_2 \odot N_2}$ are given respectively by top and bottom composite of arrows and the morphism of decorations $\iota_{\alpha \odot \beta}$ is given by composing $\iota_\alpha \times \iota_\beta$ with the two commuting squares, which can equivalently be viewed as a morphism in $F(b' +_{a_2'} c')$. 

Returning to the interchange law, composing the two horizontal compositions above vertically then results in:
\[
\begin{tikzpicture}[scale=1.5]
\node (A) at (0,0.5) {$a_1$};
\node (A') at (0,-0.5) {$a_1''$};
\node (B) at (2,0.5) {$b+_{a_2} c$};
\node (C) at (4,0.5) {$a_3$};
\node (C') at (4,-0.5) {$a_3''$};
\node (D) at (2,-0.5) {$b'' +_{a_2''} c''$};
\node (E) at (5.5,0.5) {$d_{M_1 \odot N_1} \in F(b+_{a_2} c)$};
\node (F) at (5.5,-0.5) {$d_{M_3 \odot N_3} \in F(b'' +_{a_2''} c'')$};
\node (G) at (2,-1) {$\iota_{(a^\prime \odot \beta^\prime)(\alpha \odot \beta)} \colon F((h_1^\prime +_{g^\prime} h_2^\prime)(h_1 +_g h_2))(d_{M_1 \odot N_1}) \to d_{M_3 \odot N_3}.$};
\path[->,font=\scriptsize,>=angle 90]
(A) edge node[above]{$j \psi_{a_1} i_1$} (B)
(C) edge node[above]{$j \psi_{a_3} o_2$} (B)
(A) edge node[left]{$f^\prime f$} (A')
(C) edge node[left]{$k^\prime k$} (C')
(A') edge node [above]{$j \psi_{a_1''} i_1''$} (D)
(C') edge node [above]{$j \psi_{a_3''} o_2''$} (D)
(B) edge node [left] {$(h_1^\prime +_{g^\prime} h_2^\prime)(h_1 +_g h_2)$} (D);
\end{tikzpicture}
\]
The vertical composite of two morphisms of decorations is straightforward. On the other hand, if we first compose vertically we obtain:
\[
\begin{tikzpicture}[scale=1.5]
\node (A) at (0,0.5) {$a_1$};
\node (A') at (0,-0.5) {$a_1''$};
\node (B) at (1,0.5) {$b$};
\node (C) at (2,0.5) {$a_2$};
\node (C') at (2,-0.5) {$a_2''$};
\node (D) at (1,-0.5) {$b''$};
\node (E) at (3,0.5) {$d_{M_1} \in F(b)$};
\node (F) at (3,-0.5) {$d_{M_3} \in F(b'')$};
\node (G) at (4,0.5) {$a_2$};
\node (H) at (5,0.5) {$c$};
\node (I) at (6,0.5) {$a_3$};
\node (G') at (4,-0.5) {$a_2''$};
\node (H') at (5,-0.5) {$c''$};
\node (I') at (6,-0.5) {$a_3''$};
\node (J) at (7,0.5) {$d_{N_1} \in F(c)$};
\node (K) at (7,-0.5) {$d_{N_3} \in F(c'')$};
\node (L) at (1,-1) {$\iota_{\alpha^\prime \alpha} \colon F(h_1^\prime h_1)(d_{M_1}) \to d_{M_3}$};
\node (M) at (5,-1) {$\iota_{\beta^\prime \beta} \colon F(h_2^\prime h_2)(d_{N_1}) \to d_{N_3}$};
\path[->,font=\scriptsize,>=angle 90]
(A) edge node[above]{$i_1$} (B)
(C) edge node[above]{$o_1$} (B)
(A) edge node[left]{$f^\prime f$} (A')
(C) edge node[left]{$g^\prime g$} (C')
(A') edge node[above] {$i_1''$} (D)
(C') edge node[above] {$o_1''$} (D)
(B) edge node [left] {$h_1^\prime h_1$} (D)
(G) edge node [above] {$i_2$} (H)
(G) edge node [left] {$g^\prime g$} (G')
(H) edge node [left] {$h_2^\prime h_2$} (H')
(G') edge node [above] {$i_2''$} (H')
(I) edge node [above] {$o_2$} (H)
(I) edge node [left] {$k^\prime k$} (I')
(I') edge node [above] {$o_2''$} (H');
\end{tikzpicture}
\]
and then composing horizontally results in:
\[
\begin{tikzpicture}[scale=1.5]
\node (A) at (0,0.5) {$a_1$};
\node (A') at (0,-0.5) {$a_1''$};
\node (B) at (2,0.5) {$b+_{a_2} c$};
\node (C) at (4,0.5) {$a_3$};
\node (C') at (4,-0.5) {$a_3''$};
\node (D) at (2,-0.5) {$b'' +_{a_2''} c''$};
\node (E) at (5.5,0.5) {$d_{M_1 \odot N_1} \in F(b+_{a_2} c)$};
\node (F) at (5.5,-0.5) {$d_{M_3 \odot N_3} \in F(b'' +_{a_3''} c'')$};
\node (G) at (2,-1) {$\iota_{(\alpha^\prime \alpha) \odot (\beta^\prime \beta)} \colon F((h_1^\prime h_1)+_{g^\prime g} (h_2^\prime h_2))(d_{M_1 \odot N_1}) \to d_{M_3 \odot N_3}.$};
\path[->,font=\scriptsize,>=angle 90]
(A) edge node[above]{$j \psi_{a_1} i_1$} (B)
(C) edge node[above]{$j \psi_{a_3} o_2$} (B)
(A) edge node[left]{$f^\prime f$} (A')
(C) edge node[left]{$k^\prime k$} (C')
(A') edge node [above]{$j \psi_{a_1''} i_1''$} (D)
(C') edge node [above]{$j \psi_{a_3''} o_2''$} (D)
(B) edge node [left] {$(h_1^\prime h_1) +_{g^\prime g} (h_2^\prime h_2)$} (D);
\end{tikzpicture}
\]
%The decorations of 
%\[
%\begin{tikzpicture}[scale=1.5]
%\node (A) at (0,0) {$1$};
%\node (B) at (2,0) {$F(c) \times F(e)$};
%\node (C) at (4,0) {$F(c+e)$};
%\node (D) at (4,-1) {$F(c+_b e)$};
%\node (E) at (4,-2) {$F(c'' +_{b''} e'')$};
%\node (F) at (0,-1) {$F(c) \times F(e)$};
%\node (G) at (0,-2) {$F(c'') \times F(e'')$};
%\node (H) at (2,-2) {$F(c'' + e'')$};
%\path[->,font=\scriptsize,>=angle 90]
%(A) edge node [above]{$d_1 \times d_2$} (B)
%(B) edge node [above] {$\phi_{c,e}$} (C)
%(C) edge node [right]{$F(j_{c,e})$} (D)
%(D) edge node [right] {$F((h_1^\prime +_{g^\prime} h_2^\prime)(h_1 +_g h_2))$} (E)
%(A) edge node [left] {$d_1 \times d_2$} (F)
%(F) edge node [left] {$F((h_1^\prime h_1) \times (h_2^\prime h_2))$} (G)
%(G) edge node [above] {$\phi_{c'',e''}$} (H)
%(H) edge node [above] {$F(j_{c'',e''})$} (E);
%\end{tikzpicture}
%\]
As is usual concerning the interchange law of double categories of this nature, only the `interior' of the two compositions appears different, but the two morphisms $(h_1^\prime +_{g^\prime} h_2^\prime)(h_1 +_g h_2) \colon b+_{a_2} c \to b'' +_{a_2''} c''$ and $(h_1^\prime h_1) +_{g^\prime g} (h_2^\prime h_2) \colon b+_{a_2} c \to b'' +_{a_2''}c''$ are the same universal map realized in two different ways. The two morphisms of decorations $\iota_{(\alpha^\prime \odot \beta^\prime)(\alpha \odot \beta)}$ and $\iota_{(\alpha^\prime \alpha) \odot (\beta^\prime \beta)}$ are obtained as two different compositions of four 2-morphisms in $\mathbf{Cat}$, namely horizontally then vertically and vertically then horizontally. As $\mathbf{Cat}$ is a 2-category, the interchange law for these 2-morphisms already holds, and as a result, the morphisms $$\iota_{(\alpha' \odot \beta')(\alpha \odot \beta)} \colon F((h_1^\prime +_{g^\prime} h_2^\prime)(h_1 +_g h_2))(d_{M_1 \odot N_1}) \to d_{M_3 \odot N_3}$$ and $$\iota_{(\alpha^\prime \alpha)\odot(\beta^\prime \beta)} \colon F((h_1^\prime h_1)+_{g^\prime g} (h_2^\prime h_2))(d_{M_1 \odot N_1}) \to d_{M_3 \odot N_3}$$ are also the same. Thus the interchange law for 2-morphisms holds and $F\mathbb{C}$sp is a double category.
\end{proof}

\begin{thm}\label{DC}
Let $\mathsf{A}$ be a category with finite colimits and $F \colon \mathsf{A} \to \mathbf{Cat}$ a symmetric lax monoidal pseudofunctor. Then the double category $F\mathbb{C}$sp is symmetric monoidal.
\end{thm}
\begin{proof}
First we note that the category of objects $F\mathbb{C}\textnormal{sp}_0=\mathsf{A}$ is symmetric monoidal under binary coproducts and the left and right unitors, associators and braidings are given as natural maps. The category of arrows $F\mathbb{C}\textnormal{sp}_1$ has:
\begin{enumerate}
\item{objects as $F$-decorated cospans which are pairs:
\[
\begin{tikzpicture}[scale=1.5]
\node (A) at (0,0) {$a_1$};
\node (B) at (1,0) {$b$};
\node (C) at (2,0) {$a_2$};
\node (D) at (3,0) {$d \in F(b)$};
\path[->,font=\scriptsize,>=angle 90]
(A) edge node[above]{$i$} (B)
(C) edge node[above]{$o$} (B);
\end{tikzpicture}
\]
and}
\item{morphisms as maps of cospans in $\mathsf{A}$
\[
\begin{tikzpicture}[scale=1.5]
\node (A) at (0,0.5) {$a_1$};
\node (A') at (0,-0.5) {$a_1^\prime$};
\node (B) at (1,0.5) {$b$};
\node (C) at (2,0.5) {$a_2$};
\node (C') at (2,-0.5) {$a_2^\prime$};
\node (D) at (1,-0.5) {$b^\prime$};
\node (E) at (3,0.5) {$d \in F(b)$};
\node (F) at (3,-0.5) {$d^\prime \in F(b^\prime)$};
\path[->,font=\scriptsize,>=angle 90]
(A) edge node[above]{$i$} (B)
(C) edge node[above]{$o$} (B)
(A) edge node[left]{$f$} (A')
(C) edge node[left]{$g$} (C')
(A') edge node [above]{$i^\prime$} (D)
(C') edge node [above]{$o^\prime$} (D)
(B) edge node [left] {$h$} (D);
\end{tikzpicture}
\]
together with a morphism $\iota \colon F(h)(d) \to d^\prime$.
}
\end{enumerate}
Given two objects $M_1$ and $M_2$ of $F\mathbb{C}\textnormal{sp}_1$:
\[
\begin{tikzpicture}[scale=1.5]
\node (A) at (0,0) {$a_1$};
\node (B) at (1,0) {$b$};
\node (C) at (2,0) {$a_2$};
\node (D) at (1,-0.5) {$d_{M_1} \in F(b)$};
\node (E) at (3,0) {$a_1^\prime$};
\node (F) at (4,0) {$b^\prime$};
\node (G) at (5,0) {$a_2^\prime$};
\node (H) at (4,-0.5) {$d_{M_2} \in F(b^\prime)$};
\path[->,font=\scriptsize,>=angle 90]
(A) edge node[above]{$i$} (B)
(C) edge node[above]{$o$} (B)
(E) edge node[above]{$i^\prime$} (F)
(G) edge node[above]{$o^\prime$} (F);
\end{tikzpicture}
\]
their tensor product $M_1 \otimes M_2$ is given by taking the coproducts of the cospans of $\mathsf{A}$
\[
\begin{tikzpicture}[scale=1.5]
\node (A) at (0,0) {$a_1+a_1^\prime$};
\node (B) at (1.25,0) {$b+b^\prime$};
\node (C) at (2.5,0) {$a_2+a_2^\prime$};
\node (D) at (4.25,0) {$d_{M_1 \otimes M_2} \in F(b+b')$};
\path[->,font=\scriptsize,>=angle 90]
(A) edge node[above]{$i+i'$} (B)
(C) edge node[above]{$o+o'$} (B);
\end{tikzpicture}
\]
and where the decoration on the apex is obtained using the natural transformation of the symmetric lax monoidal pseudofunctor $F:$ $$d_{M_1 \otimes M_2} \coloneqq 1 \xrightarrow{\lambda^{-1}} 1 \times 1 \xrightarrow{d_{M_1} \times d_{M_2}} F(b) \times F(b') \xrightarrow{\phi_{b,b'}} F(b+b').$$The monoidal unit 0 is given by:
\[
\begin{tikzpicture}[scale=1.5]
\node (A) at (0,0) {$0$};
\node (B) at (1,0) {$0$};
\node (C) at (2,0) {$0$};
\node (D) at (3,0) {$!_0 \in F(0)$};
\path[->,font=\scriptsize,>=angle 90]
(A) edge node[above]{$!$} (B)
(C) edge node[above]{$!$} (B);
\end{tikzpicture}
\]
where $0$ is the monoidal unit of $\mathsf{A}$ and $!_0 \colon 1 \to F(0)$ is the morphism which is part of the structure of the symmetric lax monoidal pseudofunctor $F \colon \mathsf{A} \to \mathbf{Cat}$. Tensoring an object with the monoidal unit, say, on the left:
\[
\begin{tikzpicture}[scale=1.5]
\node (A) at (0,0) {$0$};
\node (B) at (1,0) {$0$};
\node (C) at (2,0) {$0$};
\node (D) at (1,-0.5) {$!_0 \in F(0)$};
\node (E) at (3,0) {$a_1$};
\node (F) at (4,0) {$b$};
\node (G) at (5,0) {$a_2$};
\node (H) at (4,-0.5) {$d_M \in F(b)$};
\node (I) at (2.5,0) {$\otimes$};
\path[->,font=\scriptsize,>=angle 90]
(A) edge node[above]{$!$} (B)
(C) edge node[above]{$!$} (B)
(E) edge node[above]{$i$} (F)
(G) edge node[above]{$o$} (F);
\end{tikzpicture}
\]
results in:
\[
\begin{tikzpicture}[scale=1.5]
\node (A) at (0,0) {$0+a_1$};
\node (B) at (1,0) {$0+b$};
\node (C) at (2,0) {$0+a_2$};
\node (D) at (3.5,0) {$d_{0 \otimes M} \in F(0+b)$};
\path[->,font=\scriptsize,>=angle 90]
(A) edge node[above]{$!+i$} (B)
(C) edge node[above]{$!+o$} (B);
\end{tikzpicture}
\]
where $d_{0\otimes M} \in F(0+b)$ is given by $$1 \xrightarrow{\lambda^{-1}} 1 \times 1 \xrightarrow{!_0 \times d_M} F(0) \times F(b) \xrightarrow{\phi_{0,b}} F(0+b).$$The left unitor is then an isomorphism in $F\mathbb{C}\textnormal{sp}_1$ given by:
\[
\begin{tikzpicture}[scale=1.5]
\node (A) at (0,0.5) {$0+a_1$};
\node (A') at (0,-0.5) {$a_1$};
\node (B) at (1,0.5) {$0+b$};
\node (C) at (2,0.5) {$0+a_2$};
\node (C') at (2,-0.5) {$a_2$};
\node (D) at (1,-0.5) {$b$};
\node (E) at (3.5,0.5) {$d_{0+M} \in F(0+b)$};
\node (F) at (3.5,-0.5) {$d_M \in F(b)$};
\path[->,font=\scriptsize,>=angle 90]
(A) edge node[above]{$!+i$} (B)
(C) edge node[above]{$!+o$} (B)
(A) edge node[left]{$\ell$} (A')
(C) edge node[left]{$\ell$} (C')
(A') edge node [above]{$i$} (D)
(C') edge node [above]{$o$} (D)
(B) edge node [left] {$\ell$} (D);
\end{tikzpicture}
\]
where $\ell$ is the left unitor of $(\mathsf{A},+,0)$, together with the isomorphism $\iota_{\lambda} \colon F(\ell)(d_{0 \otimes M}) \to d_M$. The right unitor is similar.

Given three objects $M_1, M_2$ and $M_3$ in $F\mathbb{C}\textnormal{sp}_1$:
\[
\begin{tikzpicture}[scale=1.5]
\node (A) at (0,0) {$a_1$};
\node (B) at (1,0) {$c_1$};
\node (C) at (2,0) {$b_1$};
\node (D) at (1,-0.5) {$d_{M_1} \in F(c_1)$};
\node (E) at (3,0) {$a_2$};
\node (F) at (4,0) {$c_2$};
\node (G) at (5,0) {$b_2$};
\node (H) at (4,-0.5) {$d_{M_2} \in F(c_2)$};
\node (I) at (6,0) {$a_3$};
\node (J) at (7,0) {$c_3$};
\node (K) at (8,0) {$b_3$};
\node (L) at (7,-0.5) {$d_{M_3} \in F(c_3)$};
\path[->,font=\scriptsize,>=angle 90]
(A) edge node[above]{$i_1$} (B)
(C) edge node[above]{$o_1$} (B)
(E) edge node[above]{$i_2$} (F)
(G) edge node[above]{$o_2$} (F)
(I) edge node[above]{$i_3$} (J)
(K) edge node[above]{$o_3$} (J);
\end{tikzpicture}
\]
tensoring the first two and then the third results in $(M_1 \otimes M_2) \otimes M_3$:
\[
\begin{tikzpicture}[scale=1.5]
\node (A) at (0,0) {$(a_1+a_2)+a_3$};
\node (B) at (2.25,0){$(c_1+c_2)+c_3$};
\node (C) at (4.5,0) {$(b_1+b_2)+b_3$};
\node (D) at (2.25,-0.5) {$d_{(M_1 \otimes M_2) \otimes M_3} \in F((c_1+c_2)+c_3)$};
\path[->,font=\scriptsize,>=angle 90]
(A) edge node[above]{$(i_1+i_2)+i_3$} (B)
(C) edge node[above]{$(o_1+o_2)+o_3$} (B);
\end{tikzpicture}
\]
where $d_{(M_1 \otimes M_2) \otimes M_3} \colon 1 \to F((c_1+c_2)+c_3)$ is given by: $$1 \xrightarrow{(d_{M_1} \times d_{M_2}) \times d_{M_3})} (F(c_1) \times F(c_2)) \times F(c_3) \xrightarrow{\phi_{c_1,c_2} \times 1} F(c_1+c_2) \times F(c_3) \xrightarrow{\phi_{c_1+c_2,c_3}} F((c_1+c_2)+c_3)$$whereas tensoring the last two and then the first results in $M_1 \otimes (M_2 \otimes M_3)$:
\[
\begin{tikzpicture}[scale=1.5]
\node (A) at (0,0) {$a_1+(a_2+a_3)$};
\node (B) at (2.25,0) {$c_1+(c_2+c_3)$};
\node (C) at (4.5,0) {$b_1+(b_2+b_3)$};
\node (D) at (2.25,-0.5) {$d_{M_1 \otimes (M_2 \otimes M_3)} \in F(c_1+(c_2+c_3))$};
\path[->,font=\scriptsize,>=angle 90]
(A) edge node[above]{$i_1+(i_2+i_3)$} (B)
(C) edge node[above]{$o_1+(o_2+o_3)$} (B);
\end{tikzpicture}
\]
where $d_{M_1 \otimes (M_2 \otimes M_3)} \colon 1 \to F(c_1+(c_2+c_3))$ is given by: $$1 \xrightarrow{d_{M_1} \times (d_{M_2} \times d_{M_3})} F(c_1) \times (F(c_2) \times F(c_3)) \xrightarrow{1 \times \phi_{c_2,c_3}} F(c_1) \times F(c_2+c_3) \xrightarrow{\phi_{c_1,c_2+c_3}} F(c_1+(c_2+c_3)).$$If we let $a$ denote the associator of $(\mathsf{A},+,0)$, the associator of $F\mathbb{C}\textnormal{sp}_1$ is then a map of cospans in $\mathsf{A}$ from $(M_1 \otimes M_2) \otimes M_3$ to $M_1 \otimes (M_2 \otimes M_3)$ given by:
\[
\begin{tikzpicture}[scale=1.5]
\node (A) at (0,0.5) {$(a_1+a_2)+a_3$};
\node (A') at (0,-0.5) {$a_1+(a_2+a_3)$};
\node (B) at (2.25,0.5) {$(c_1+c_2)+c_3$};
\node (C) at (4.5,0.5) {$(b_1+b_2)+b_3$};
\node (C') at (4.5,-0.5) {$b_1+(b_2+b_3)$};
\node (D) at (2.25,-0.5) {$c_1+(c_2+c_3)$};
\node (E) at (7,0.5) {$d_{(M_1 \otimes M_2) \otimes M_3} \in F((c_1+c_2)+c_3)$};
\node (F) at (7,-0.5) {$d_{M_1 \otimes (M_2 \otimes M_3)} \in F(c_1+(c_2+c_3))$};
\path[->,font=\scriptsize,>=angle 90]
(A) edge node[above]{$(i_1+i_2)+i_3$} (B)
(C) edge node[above]{$(o_1+o_2)+o_3$} (B)
(A) edge node[left]{$a$} (A')
(C) edge node[left]{$a$} (C')
(A') edge node [above]{$i_1+(i_2+i_3)$} (D)
(C') edge node [above]{$o_1+(o_2+o_3)$} (D)
(B) edge node [left] {$a$} (D);
\end{tikzpicture}
\]
together with the isomorphism $\iota_a \colon F(a)(d_{(M_1 \otimes M_2) \otimes M_3}) \to d_{M_1 \otimes (M_2 \otimes M_3)}$. If we denote the above associator simply as $a$ and the left and right unitors as $\lambda$ and $\rho$, respectively, then given four objects in $F\mathbb{C}\textnormal{sp}_1$, say $M_1, M_2, M_3$ and $M_4$:
\[
\begin{tikzpicture}[scale=1.5]
\node (A) at (0,0) {$a_1$};
\node (B) at (1,0) {$c_1$};
\node (C) at (2,0) {$b_1$};
\node (D) at (1,-0.5) {$d_{M_1} \in F(c_1)$};
\node (E) at (3,0) {$a_2$};
\node (F) at (4,0) {$c_2$};
\node (G) at (5,0) {$b_2$};
\node (H) at (4,-0.5) {$d_{M_2} \in F(c_2)$};
\node (I) at (0,-1.5) {$a_3$};
\node (J) at (1,-1.5) {$c_3$};
\node (K) at (2,-1.5) {$b_3$};
\node (L) at (1,-2) {$d_{M_3} \in F(c_3)$};
\node (M) at (3,-1.5) {$a_4$};
\node (N) at (4,-1.5) {$c_4$};
\node (O) at (5,-1.5) {$b_4$};
\node (P) at (4,-2) {$d_{M_4} \in F(c_4)$};
\path[->,font=\scriptsize,>=angle 90]
(A) edge node[above]{$i_1$} (B)
(C) edge node[above]{$o_1$} (B)
(E) edge node[above]{$i_2$} (F)
(G) edge node[above]{$o_2$} (F)
(I) edge node[above]{$i_3$} (J)
(K) edge node[above]{$o_3$} (J)
(M) edge node[above]{$i_4$} (N)
(O) edge node[above]{$o_4$} (N);
\end{tikzpicture}
\]
then as $\mathbb{C}\textnormal{sp}(\mathsf{A})$ is a symmetric monoidal double category, the following pentagon of underlying cospans commutes:
\[
\begin{tikzpicture}[scale=1.5]
\node (A) at (0,0) {$((M_1 \otimes M_2) \otimes M_3) \otimes M_4$};
\node (B) at (2.25,1) {$(M_1 \otimes M_2) \otimes (M_3 \otimes M_4)$};
\node (C) at (4.5,0) {$M_1 \otimes (M_2 \otimes (M_3 \otimes M_4))$};
\node (D) at (.75,-1.5) {$(M_1 \otimes (M_2 \otimes M_3)) \otimes M_4$};
\node (E) at (3.75,-1.5) {$M_1 \otimes ((M_2 \otimes M_3) \otimes M_4)$};
\path[->,font=\scriptsize,>=angle 90]
(A) edge node[above]{$a$} (B)
(B) edge node[above]{$a$} (C)
(A) edge node[left]{$a \otimes 1$} (D)
(D) edge node[above]{$a$} (E)
(E) edge node[right]{$1 \otimes a$} (C);
\end{tikzpicture}
\]
as well as the following pentagon of corresponding decorations in the category $F(c_1 +(c_2 +(c_3+c_4)))$:
\[
\begin{tikzpicture}[scale=1.5]
\node (A) at (-.5,0) {$F(aa)(d_{((M_1 \otimes M_2) \otimes M_3) \otimes M_4}) $};
\node (B) at (2.25,1) {$F(a)(d_{(M_1 \otimes M_2) \otimes (M_3 \otimes M_4)})$};
\node (C) at (5,0) {$d_{M_1 \otimes (M_2 \otimes (M_3 \otimes M_4))}$};
\node (D) at (-.5,-1) {$F((1 \otimes a)a)(d_{(M_1 \otimes (M_2 \otimes M_3)) \otimes M_4}) $};
\node (E) at (5,-1) {$F(1 \otimes a)(d_{M_1 \otimes ((M_2 \otimes M_3) \otimes M_4)}) $};
\path[->,font=\scriptsize,>=angle 90]
(A) edge node[left]{$F(a)(\iota_a)$} (B)
(B) edge node[above]{$\iota_a$} (C)
(A) edge node[left]{$F((1 \otimes a)a)(\iota_{a \otimes 1})$} (D)
(D) edge node[above]{$F(1 \otimes a)(\iota_a)$} (E)
(E) edge node[right]{$\iota_{1 \otimes a}$} (C);
\end{tikzpicture}
\]
since $$F(aa)(d_{((M_1 \otimes M_2) \otimes M_3) \otimes M_4})=F((1 \otimes a)a(a \otimes 1))(d_{((M_1 \otimes M_2) \otimes M_3) \otimes M_4})$$ as the corresponding pentagon in the symmetric monoidal category $(\mathsf{A},+,0)$ commutes.

Similarly, if we denote the left and right unitors as $\lambda$ and $\rho$, respectively, then the following triangle of underlying maps of cospans commutes:
\[
\begin{tikzpicture}[scale=1.5]
\node (A) at (0,0) {$(M_1 \otimes 0) \otimes M_2$};
\node (B) at (2.25,1) {$M_1 \otimes M_2$};
\node (C) at (4.5,0) {$M_1 \otimes (0 \otimes M_2)$};
\path[->,font=\scriptsize,>=angle 90]
(A) edge node[above]{$\rho \otimes 1$} (B)
(C) edge node[above]{$1 \otimes \lambda$} (B)
(A) edge node[above]{$a$} (C);
\end{tikzpicture}
\]
as well as the following triangle of corresponding decorations in the category $F(c_1+c_2)$:
\[
\begin{tikzpicture}[scale=1.5]
\node (A) at (0,0) {$F(\rho \otimes 1)(d_{(M_1 \otimes 0) \otimes M_2})$};
\node (B) at (2.25,1) {$d_{M_1 \otimes M_2}$};
\node (C) at (4.5,0) {$F(1 \otimes \lambda)(d_{M_1 \otimes (0 \otimes M_2)}) $};
\path[->,font=\scriptsize,>=angle 90]
(A) edge node[left]{$\iota_{\rho \otimes 1}$} (B)
(C) edge node[right]{$\iota_{1 \otimes \lambda}$} (B)
(A) edge node[above]{$F(1 \otimes \lambda)(\iota_a)$} (C);
\end{tikzpicture}
\]
since $$F(\rho \otimes 1)(d_{(M_1 \otimes 0) \otimes M_2})=F((1 \otimes \lambda)a)(d_{(M_1 \otimes 0) \otimes M_2})$$ as the corresponding triangle in the symmetric monoidal category $(\mathsf{A},+,0)$ commutes.

For a tensor product of objects $M_1 \otimes M_2$ in $F\mathbb{C}\textnormal{sp}_1$, the source and target structure functors $S,T \colon F\mathbb{C}\textnormal{sp}_1 \to F\mathbb{C}\textnormal{sp}_0$ satisfy the following equations: $$S(M_1 \otimes M_2)=S(M_1) \otimes S(M_2)$$ $$T(M_1 \otimes M_2)=T(M_1) \otimes T(M_2).$$
For two objects $M_1$ and $M_2$ in $F\mathbb{C}\textnormal{sp}_1$, we have a braiding $\beta_{M_1,M_2} \colon M_1 \otimes M_2 \to M_2 \otimes M_1$ given by:
\[
\begin{tikzpicture}[scale=1.5]
\node (A) at (0,0.5) {$a_1+a_2$};
\node (A') at (0,-0.5) {$a_2+a_1$};
\node (B) at (2,0.5) {$c_1+c_2$};
\node (C) at (4,0.5) {$b_1+b_2$};
\node (C') at (4,-0.5) {$b_2+b_1$};
\node (D) at (2,-0.5) {$c_2+c_1$};
\node (E) at (5.5,0.5) {$d_{M_1 \otimes M_2} \in F(c_1+c_2)$};
\node (F) at (5.5,-0.5) {$d_{M_2 \otimes M_1} \in F(c_2+c_1)$};
\path[->,font=\scriptsize,>=angle 90]
(A) edge node[above]{$i_1+i_2$} (B)
(C) edge node[above]{$o_1+o_2$} (B)
(A) edge node[left]{$\beta_{a_1,a_2}$} (A')
(C) edge node[left]{$\beta_{b_1,b_2}$} (C')
(A') edge node [above]{$i_2+i_1$} (D)
(C') edge node [above]{$o_2+o_1$} (D)
(B) edge node [left] {$\beta_{c_1,c_2}$} (D);
\end{tikzpicture}
\]
$$\iota_{\beta_{M_1,M_2}} \colon F(\beta_{c_1,c_2})(d_{M_1 \otimes M_2}) \xrightarrow{\sim} d_{M_2 \otimes M_1}$$ where the vertical 1-morphisms are given by braidings in $(\mathsf{A},+,0)$. This braiding makes the following triangle of underlying cospans commute:
\[
\begin{tikzpicture}[scale=1.5]
\node (A) at (0,0) {$M_1 \otimes M_2$};
\node (B) at (2.25,1) {$M_1 \otimes M_2$};
\node (C) at (4.5,0) {$M_2 \otimes M_1$};
\path[->,font=\scriptsize,>=angle 90]
(A) edge node[above]{$1$} (B)
(C) edge node[above]{$\beta_{M_2,M_1}$} (B)
(A) edge node[above]{$\beta_{M_1,M_2}$} (C);
\end{tikzpicture}
\]
as well as the following diagram of corresponding decorations in the category $F(c_1+c_2)$:
\[
\begin{tikzpicture}[scale=1.5]
\node (A) at (0,0) {$d_{M_1 \otimes M_2}$};
\node (B) at (2.25,1) {$d_{M_1 \otimes M_2} $};
\node (C) at (4.5,0) {$F(\beta_{c_2,c_1})(d_{M_2 \otimes M_1}) $};
\path[->,font=\scriptsize,>=angle 90]
(A) edge node[above]{$1$} (B)
(C) edge node[right]{$\iota_{\beta_{M_2,M_1}}$} (B)
(A) edge node[above]{$F(\beta_{c_2,c_1})(\iota_{\beta_{M_1,M_2}})$} (C);
\end{tikzpicture}
\]
since $F(\beta_{c_2,c_1} \beta_{c_1,c_2}) (d_{M_1 \otimes M_2}) = d_{M_1 \otimes M_2}$. Thus $F\mathbb{C}\textnormal{sp}_1$ is also symmetric monoidal.

Now, given four horizontal 1-cells $M_1, M_2, N_1$ and $N_2$ respectively by:
\[
\begin{tikzpicture}[scale=1.5]
\node (A) at (0,0) {$a_1$};
\node (B) at (1,0) {$b$};
\node (C) at (2,0) {$a_2$};
\node (D) at (1,-0.5) {$d_{M_1} \in F(b)$};
\node (E) at (3,0) {$a_2$};
\node (F) at (4,0) {$b'$};
\node (G) at (5,0) {$a_3$};
\node (H) at (4,-0.5) {$d_{M_2} \in F(b')$};
\node (I) at (0,-1.5) {$a_1'$};
\node (J) at (1,-1.5) {$c$};
\node (K) at (2,-1.5) {$a_2'$};
\node (L) at (1,-2) {$d_{N_1} \in F(c)$};
\node (M) at (3,-1.5) {$a_2'$};
\node (N) at (4,-1.5) {$c'$};
\node (O) at (5,-1.5) {$a_3'$};
\node (P) at (4,-2) {$d_{N_2} \in F(c')$};
\path[->,font=\scriptsize,>=angle 90]
(A) edge node[above]{$i_1$} (B)
(C) edge node[above]{$o_1$} (B)
(E) edge node[above]{$i_2$} (F)
(G) edge node[above]{$o_2$} (F)
(I) edge node[above]{$i_1^\prime$} (J)
(K) edge node[above]{$o_1^\prime$} (J)
(M) edge node[above]{$i_2^\prime$} (N)
(O) edge node[above]{$o_2^\prime$} (N);
\end{tikzpicture}
\]
we have that $(M_1 \otimes N_1) \odot (M_2 \otimes N_2)$ is given by:
\[
\begin{tikzpicture}[scale=1.5]
\node (A) at (0,0) {$a_1+a_1'$};
\node (B) at (2.5,0) {$(b+c)+_{a_2+a_2'}(b'+c')$};
\node (C) at (5,0) {$a_3+a_3'$};
\node (D) at (2.5,-0.5) {$d_{(M_1 \otimes N_1) \odot (M_2 \otimes N_2)} \in F((b+c)+_{a_2+a_2'}(b'+c'))$};
\path[->,font=\scriptsize,>=angle 90]
(A) edge node[above]{$j \psi(i_1+i_1')$} (B)
(C) edge node[above]{$j \psi(o_2 + o_2^\prime)$} (B);
\end{tikzpicture}
\]
where the decoration $d_{(M_1 \otimes N_1) \odot (M_2 \otimes N_2)} \in F((b+c)+_{a_2+a_2'}(b'+c'))$ is given by:
\[
\begin{tikzpicture}[scale=1.5]
\node (A) at (0,0) {$1$};
\node (B) at (0,-1) {$1 \times 1$};
\node (C) at (0,-2) {$(1 \times 1) \times (1 \times 1)$};
\node (D) at (0,-3) {$(F(b) \times F(c)) \times (F(b') \times F(c'))$};
\node (E) at (0,-4) {$F(b+c) \times F(b'+c')$};
\node (F) at (0,-5) {$F((b+c)+(b'+c'))$};
\node (G) at (0,-6) {$F((b+c)+_{a_2+a_2'}(b'+c'))$};
\path[->,font=\scriptsize,>=angle 90]
(A) edge node[left]{$\lambda^{-1}$} (B)
(B) edge node[left]{$\lambda^{-1} \times \lambda^{-1}$} (C)
(C) edge node[left]{$(d_{M_1} \times d_{N_1}) \times (d_{M_2} \times d_{N_2})$} (D)
(D) edge node[left]{$\phi_{b,c} \times \phi_{b',c'}$} (E)
(E) edge node[left]{$\phi_{b+c,b'+c'}$} (F)
(F) edge node[left]{$F(j_{b+c,b'+c'})$} (G);
\end{tikzpicture}
\]
and $(M_1 \odot M_2) \otimes (N_1 \odot N_2)$ is given by:
\[
\begin{tikzpicture}[scale=1.5]
\node (A) at (0,0) {$a_1+a_1'$};
\node (B) at (2.5,0) {$(b+_{a_2} c) + (b' +_{a_2'} c')$};
\node (C) at (5,0) {$a_3+a_3'$};
\node (D) at (2.5,-0.5) {$d_{(M_1 \odot M_2) \otimes (N_1 \odot N_2)} \in F((b+_{a_2}c)+(b'+_{a_2'}c'))$};
\path[->,font=\scriptsize,>=angle 90]
(A) edge node[above]{$(j \psi i_1)+(j \psi i_1')$} (B)
(C) edge node[above]{$(j \psi o_2)+(j \psi o_2^\prime)$} (B);
\end{tikzpicture}
\]
where the decoration $d_{((M_1 \odot M_2) \otimes (N_1 \odot N_2))} \in F((b+_{a_2}c)+(b'+_{a_2'}c'))$ is given by:
\[
\begin{tikzpicture}[scale=1.5]
\node (A) at (0,0) {$1$};
\node (B) at (0,-1) {$1 \times 1$};
\node (C) at (0,-2) {$(1 \times 1) \times (1 \times 1)$};
\node (D) at (0,-3) {$(F(b) \times F(c)) \times (F(b') \times F(c'))$};
\node (E) at (0,-4) {$F(b+c) \times F(b'+c')$};
\node (F) at (0,-5) {$F(b+_{a_2}c) \times F(b'+_{a_2'}c')$};
\node (G) at (0,-6) {$F((b+_{a_2}c)+(b'+_{a_2'}c'))$};
\path[->,font=\scriptsize,>=angle 90]
(A) edge node[left]{$\lambda^{-1}$} (B)
(B) edge node[left]{$\lambda^{-1} \times \lambda^{-1}$} (C)
(C) edge node[left]{$(d_{M_1} \times d_{N_1}) \times (d_{M_2} \times d_{N_2})$} (D)
(D) edge node[left]{$\phi_{b,c} \times \phi_{b',c'}$} (E)
(E) edge node[left]{$F(j_{b,c}) \times F(j_{b',c'})$} (F)
(F) edge node[left]{$\phi_{b+_{a_2}c,b'+_{a_2'}c'}$} (G);
\end{tikzpicture}
\]
and where $\psi$ and $j$ are the natural maps into a coproduct and from a coproduct into a pushout, respectively. We then get a globular 2-morphism $$\chi \colon (M_1 \otimes N_1) \odot (M_2 \otimes N_2) \to (M_1 \odot M_2) \otimes (N_1 \odot N_2)$$ given by:
\[
\begin{tikzpicture}[scale=1.5]
\node (A) at (0,0.5) {$a_1+a_1'$};
\node (A') at (0,-0.5) {$a_1+a_1'$};
\node (B) at (2.5,0.5) {$(b+c)+_{a_2+a_2'}(b'+c')$};
\node (C) at (5,0.5) {$a_3+a_3'$};
\node (C') at (5,-0.5) {$a_3+a_3'$};
\node (D) at (2.5,-0.5) {$(b+_{a_2}c)+(b'+_{a_2'}c')$};
\node (E) at (2.5,1) {$d_{(M_1 \otimes N_1) \odot (M_2 \otimes N_2)} \in F((b+c)+_{a_2+a_2'}(b'+c'))$};
\node (F) at (2.5,-1) {$d_{(M_1 \odot M_2) \otimes (N_1 \odot N_2)} \in F((b+_{a_2}c)+(b'+_{a_2'}c'))$};
\path[->,font=\scriptsize,>=angle 90]
(A) edge node[above]{$j \psi (i_1+i_1')$} (B)
(C) edge node[above]{$j \psi (o_2 + o_2^\prime)$} (B)
(A) edge node[left]{$1$} (A')
(C) edge node[left]{$1$} (C')
(A') edge node [above]{$(j \psi i_1')+(j \psi i_1)$} (D)
(C') edge node [above]{$(j \psi o_2)+(j \psi o_2^\prime)$} (D)
(B) edge node [left] {$\hat{\chi}$} (D);
\end{tikzpicture}
\]
$$\iota_{\hat{\chi}} \colon F(\hat{\chi})(d_{(M_1 \otimes N_1) \odot (M_2 \otimes N_2)}) \to d_{(M_1 \odot M_2) \otimes (N_1 \odot N_2)}$$
where $\hat{\chi}$ is the universal map between two colimits of the same diagram.
For two objects $a,b \in \mathrm{A}$, $U_{a+b}$ is given by:
\[
\begin{tikzpicture}[scale=1.5]
\node (A) at (0,0) {$a+b$};
\node (B) at (1,0) {$a+b$};
\node (C) at (2,0) {$a+b$};
\node (D) at (1,-0.5) {$!_{a+b} \in F(a+b)$};
\path[->,font=\scriptsize,>=angle 90]
(A) edge node[above]{$1_{a+b}$} (B)
(C) edge node[above]{$1_{a+b}$} (B);
\end{tikzpicture}
\]
where $$!_{a+b} \colon 1 \xrightarrow{\phi} F(0) \xrightarrow{F(!_{a+b})}  F(a+b).$$
Similarly, we have $U_a$ and $U_b$ given respectively by:
\[
\begin{tikzpicture}[scale=1.5]
\node (A) at (0,0) {$a$};
\node (B) at (1,0) {$a$};
\node (C) at (2,0) {$a$};
\node (D) at (1,-0.5) {$!_a \in F(a)$};
\node (E) at (3,0) {$b$};
\node (F) at (4,0) {$b$};
\node (G) at (5,0) {$b$};
\node (H) at (4,-0.5) {$!_b \in F(b)$};
\path[->,font=\scriptsize,>=angle 90]
(A) edge node[above]{$1_a$} (B)
(C) edge node[above]{$1_a$} (B)
(E) edge node[above]{$1_b$} (F)
(G) edge node[above]{$1_b$} (F);
\end{tikzpicture}
\]
and then $U_a + U_b$ is given by:
\[
\begin{tikzpicture}[scale=1.5]
\node (A) at (0,0) {$a+b$};
\node (B) at (1.25,0) {$a+b$};
\node (C) at (2.5,0) {$a+b$};
\node (D) at (1.25,-0.5) {$!_a + !_b \in F(a+b)$};
\path[->,font=\scriptsize,>=angle 90]
(A) edge node[above]{$1_a + 1_b$} (B)
(C) edge node[above]{$1_a + 1_b$} (B);
\end{tikzpicture}
\]
where
$$!_a + !_b \colon 1 \xrightarrow{\lambda^{-1}} 1 \times 1 \xrightarrow{\phi \times \phi} F(0) \times F(0) \xrightarrow{F(!_a) \times F(!_b)} F(a) \times F(b) \xrightarrow{\phi_{a,b}} F(a+b).$$
We then have another globular isomorphism $$\mu_{a,b} \colon U_{a+b} \to U_a + U_b$$ given by the identity 2-morphism:
\[
\begin{tikzpicture}[scale=1.5]
\node (A) at (0,0.5) {$a+b$};
\node (A') at (0,-0.5) {$a+b$};
\node (B) at (2,0.5) {$a+b$};
\node (C) at (4,0.5) {$a+b$};
\node (C') at (4,-0.5) {$a+b$};
\node (D) at (2,-0.5) {$a+b$};
\node (E) at (5.5,0.5) {$!_{a+b} \in F(a+b)$};
\node (F) at (5.5,-0.5) {$!_a + !_b \in F(a+b)$};
\path[->,font=\scriptsize,>=angle 90]
(A) edge node[above]{$1_{a+b}$} (B)
(C) edge node[above]{$1_{a+b}$} (B)
(A) edge node[left]{$1$} (A')
(C) edge node[left]{$1$} (C')
(A') edge node [above]{$1_a + 1_b$} (D)
(C') edge node [above]{$1_a + 1_b$} (D)
(B) edge node [left] {$1$} (D);
\end{tikzpicture}
\]
$$\iota_{a,b} \colon !_{a+b} \xrightarrow{\sim} !_a + !_b$$
where $!_{a+b}$ and $!_a + !_b$ are both initial objects in $F(a+b)$, hence isomorphic.

There are a fair amount of coherence diagrams to verify, many of which are similar in flavor and make use of the two above globular ismorphisms. We check a few to give a sense of what these are like. For example,  given horizontal 1-cells $M_i,N_i,P_i$, the following commutative diagram expresses the associativity isomorphism as a transformation of double categories.
\[
\begin{tikzpicture}[scale=1.5]
\node (A) at (0,0.5) {$((M_1 \otimes N_1) \otimes P_1) \odot ((M_2 \otimes N_2) \otimes P_2)$};
\node (A') at (4.5,0.5) {$(M_1 \otimes (N_1 \otimes P_1)) \odot (M_2 \otimes (N_2 \otimes P_2))$};
\node (B) at (0,-0.25) {$((M_1 \otimes N_1) \odot (M_2 \otimes N_2)) \otimes (P_1 \odot P_2)$};
\node (C) at (4.5,-0.25) {$(M_1 \odot M_2) \otimes ((N_1 \otimes P_1) \odot (N_2 \otimes P_2))$};
\node (C') at (0,-1) {$((M_1 \odot M_2) \otimes (N_1 \odot N_2)) \otimes (P_1 \odot P_2)$};
\node (D) at (4.5,-1) {$(M_1 \odot M_2) \otimes ((N_1 \odot N_2) \otimes (P_1 \odot P_2))$};
\path[->,font=\scriptsize,>=angle 90]
(A) edge node[above]{$a \odot a$} (A')
(A) edge node[left]{$\chi$} (B)
(A') edge node[right]{$\chi$} (C)
(B) edge node[left]{$\chi \otimes 1$} (C')
(C) edge node [right] {$1 \otimes \chi$} (D)
(C') edge node [above] {$a$} (D);
\end{tikzpicture}
\]
Here, $a$ is the associator of $F\mathbb{C}\textnormal{sp}_1$ and $\chi$ is the first globular isomorphism above. To see that this diagram does indeed commute, we first consider this diagram with respect to only the underlying cospans of each horizontal 1-cell. For notation:
	\[
		\begin{tikzpicture}
			\node (k) at (0,0) {$k$};
			\node (l) at (1,0) {$l$};
			\node (m) at (2,0) {$m$};
			\node (M1) at (-1,0) {$M_1 =$};
			\node (N1) at (3,0) {$N_1 =$};
			\node (q) at (4,0) {$q$};
			\node (r) at (5,0) {$r$};
			\node (s) at (6,0) {$s$};
			\node (P1) at (7,0) {$P_1 =$};
			\node (v) at (8,0) {$v$};
			\node (w) at (9,0) {$w$};
			\node (x) at (10,0) {$x$};
			\node (m2) at (0,-1.5) {$m$};
			\node (n) at (1,-1.5) {$n$};
			\node (p) at (2,-1.5) {$p$};
			\node (M2) at (-1,-1.5) {$M_2 =$};
			\node (N2) at (3,-1.5) {$N_2 =$};
			\node (s2) at (4,-1.5) {$s$};
			\node (t) at (5,-1.5) {$t$};
			\node (u) at (6,-1.5) {$u$};
			\node (P2) at (7,-1.5) {$P_2 =$};
			\node (x2) at (8,-1.5) {$x$};
			\node (y) at (9,-1.5) {$y$};
			\node (z) at (10,-1.5) {$z$};
\node (dM1) at (1,-0.5) {$d_{M_1} \in F(l)$};
\node (dM2) at (1,-2) {$d_{M_2} \in F(n)$};
\node (dN1) at (5,-0.5) {$d_{N_1} \in F(r)$};
\node (dN2) at (5,-2) {$d_{N_2} \in F(t)$};
\node (dP1) at (9,-0.5) {$d_{P_1} \in F(w)$};
\node (dP2) at (9,-2) {$d_{P_2} \in F(y)$};
			\path[->,font=\scriptsize,>=angle 90]
			(k) edge node[above]{$$} (l)
			(m) edge node[above]{$$} (l)
			(q) edge node[above]{$$} (r)
			(s) edge node[above]{$$} (r)
			(v) edge node[above]{$$} (w)
			(x) edge node[above]{$$} (w)
			(m2) edge node[above]{$$} (n)
			(p) edge node[above]{$$} (n)
			(s2) edge node[above]{$$} (t)
			(u) edge node[above]{$$} (t)
			(x2) edge node[above]{$$} (y)
			(z) edge node[above]{$$} (y);
		\end{tikzpicture}
	\]
The above diagram then becomes:
\[
		\begin{tikzpicture}
			\node (a) at (-4,0) {$k+m$};
			\node (b) at (1,0) {$((l+r)+w) +_{((m+s)+x)}((n+t)+y)$};
			\node (c) at (6,0) {$v+x$};
			\node (a2) at (-4,1) {$k+m$};
			\node (b2) at (1,1) {$(l+(r+w)) +_{(m+(s+x))}(n+(t+y))$};
			\node (c2) at (6,1) {$v+x$};
                                \node (a3) at (-4,2) {$k+m$};
			\node (b3) at (1,2) {$(l+_m n)+((r+w)+_{(s+x)}(t+y))$};
			\node (c3) at (6,2) {$v+x$};
                                \node (a4) at (-4,3) {$k+m$};
			\node (b4) at (1,3) {$(l+_m n)+((r+_s t)+(w+_x y))$};
			\node (c4) at (6,3) {$v+x$};
                                \node (a5) at (-4,-1) {$k+m$};
			\node (b5) at (1,-1) {$((l+r)+_{(m+s)}(n+t))+(w+_x y)$};
			\node (c5) at (6,-1) {$v+x$};
                                \node (a6) at (-4,-2) {$k+m$};
			\node (b6) at (1,-2) {$((l+_m n)+(r+_s t))+(w+_x y)$};
			\node (c6) at (6,-2) {$v+x$};
                                \node (a7) at (-4,-3) {$k+m$};
			\node (b7) at (1,-3) {$(l+_m n)+((r+_s t)+(w+_x y))$};
			\node (c7) at (6,-3) {$v+x$};
			\path[->,font=\scriptsize,>=angle 90]
			(a) edge node[above]{$$} (b)
			(c) edge node[above]{$$} (b)
                                (a2) edge node[above]{$$} (b2)
			(c2) edge node[above]{$$} (b2)
                                (a) edge node[above]{$$} (a2)
                                (b) edge node[left]{$a \odot a$} (b2)
(b) edge node[right]{$\iota_1$} (b2)
			(c) edge node[above]{$$} (c2)
                                (a3) edge node[above]{$$} (b3)
			(c3) edge node[above]{$$} (b3)
                                (a2) edge node[above]{$$} (a3)
                                (b2) edge node[left]{$\chi$} (b3)
(b2) edge node[right]{$\iota_2$} (b3)
			(c2) edge node[above]{$$} (c3)
                                (a4) edge node[above]{$$} (b4)
			(c4) edge node[above]{$$} (b4)
                                (a3) edge node[above]{$$} (a4)
                                (b3) edge node[left]{$1 \otimes \chi$} (b4)
(b3) edge node[right]{$\iota_3$} (b4)
			(c3) edge node[above]{$$} (c4)
                                (a5) edge node[above]{$$} (b5)
			(c5) edge node[above]{$$} (b5)
                                (a) edge node[above]{$$} (a5)
                                (b) edge node[left]{$\chi$} (b5)
(b) edge node[right]{$\iota_4$} (b5)
			(c) edge node[above]{$$} (c5)
                                (a6) edge node[above]{$$} (b6)
			(c6) edge node[above]{$$} (b6)
                                (a5) edge node[above]{$$} (a6)
                                (b5) edge node[left]{$\chi \otimes 1$} (b6)
 (b5) edge node[right]{$\iota_5$} (b6)
			(c5) edge node[above]{$$} (c6)
                                (a7) edge node[above]{$$} (b7)
			(c7) edge node[above]{$$} (b7)
                                (a6) edge node[above]{$$} (a7)
                                (b6) edge node[left]{$a$} (b7)
(b6) edge node[right]{$\iota_6$} (b7)
			(c6) edge node[above]{$$} (c7);
		\end{tikzpicture}
	\]
Here all of the vertical 1-morphisms on the left and right are identities, the middle vertical 1-morphisms are the 2-morphisms from the previous commutative diagram, and the horizontal vertical 1-morphisms pointing towards the middle are natural maps into each colimit, all of which are naturally isomorphic to each other as all the middle objects are colimits of the same diagram, namely the previous collection of cospans, taken in various ways. The above diagram of maps of cospans can then be visualized as a hexagonal prism in which all the faces commute by identifying the top and the bottom as the same. As for the morphisms of decorations, each isomorphism $\iota_n$ goes from the domain under the image of the functor $F$ applied to natural isomorphism adjacent to it to the codomain as written, meaning that, for example: $$\iota_1 \colon F(a \odot a)(d_{((M_1 \otimes N_1) \otimes P_1) \odot ((M_2 \otimes N_2) \otimes P_2)}) \to d_{(M_1 \otimes (N_1 \otimes P_1)) \odot (M_2 \otimes (N_2 \otimes P_2))}.$$  The following diagram commutes in the category $F((l+_m n) + ((r+_s t) + (w+_x y)))$:
\[
\begin{tikzpicture}[scale=1.5]
\node (A) at (0,0.5) {$F( a (\chi \otimes 1)\chi)(d_{((M_1 \otimes N_1) \otimes P_1) \odot ((M_2 \otimes N_2) \otimes P_2)})$};
\node (A') at (4.5,0.5) {$F((1 \otimes \chi) \chi)(d_{(M_1 \otimes (N_1 \otimes P_1)) \odot (M_2 \otimes (N_2 \otimes P_2))})$};
\node (B) at (0,-0.5) {$F(a(\chi  \otimes 1))(d_{((M_1 \otimes N_1) \odot (M_2 \otimes N_2)) \otimes (P_1 \odot P_2)})$};
\node (C) at (4.5,-0.5) {$F(1 \otimes \chi)(d_{(M_1 \odot M_2) \otimes ((N_1 \otimes P_1) \odot (N_2 \otimes P_2))})$};
\node (C') at (0,-1.5) {$F(a)(d_{((M_1 \odot M_2) \otimes (N_1 \odot N_2)) \otimes (P_1 \odot P_2)})$};
\node (D) at (4.5,-1.5) {$d_{(M_1 \odot M_2) \otimes ((N_1 \odot N_2) \otimes (P_1 \odot P_2))}$};
\path[->,font=\scriptsize,>=angle 90]
(A) edge node[above]{$F((1 \otimes \chi) \chi)(\iota_1)$} (A')
(A) edge node[left]{$F(a(\chi \otimes 1))(\iota_4)$} (B)
(A') edge node[right]{$F(1 \otimes \chi)(\iota_2)$} (C)
(B) edge node[left]{$F(a)(\iota_5)$} (C')
(C) edge node [right] {$\iota_3$} (D)
(C') edge node [above] {$\iota_6$} (D);
\end{tikzpicture}
\]
since $$F(a(\chi \otimes 1)\chi)(d_{((M_1 \otimes N_1) \otimes P_1) \odot ((M_2 \otimes N_2) \otimes P_2)}) = F((1 \otimes \chi) \chi (a \odot a))(d_{((M_1 \otimes N_1) \otimes P_1) \odot ((M_2 \otimes N_2) \otimes P_2)})$$
as the above underlying diagram of maps of cospans commutes.

Another requirement for a double category to be symmetric monoidal is that the braiding $$\beta_{ ( \_, \_ ) } \colon F\mathbb{C}\textnormal{sp}_1 \times F\mathbb{C}\textnormal{sp}_1 \to F\mathbb{C}\textnormal{sp}_1 \times F\mathbb{C}\textnormal{sp}_1$$ be a transformation of double categories, and one of the diagrams that is required to commute is the following:
\[
\begin{tikzpicture}[scale=1.5]
\node (A) at (0,0) {$(M_1 \odot M_2) \otimes (N_1 \odot N_2)$};
\node (B) at (3,0) {$(N_1 \odot N_2) \otimes (M_1 \odot M_2)$};
\node (C) at (0,-.75) {$(M_1 \otimes N_1) \odot (M_2 \otimes N_2)$};
\node (D) at (3,-.75) {$(N_1 \otimes M_1) \odot (N_2 \otimes M_2)$};
\path[->,font=\scriptsize,>=angle 90]
(A) edge node[above]{$\beta$} (B)
(B) edge node[right]{$\chi$} (D)
(A) edge node[left]{$\chi$} (C)
(C) edge node[above]{$\beta \odot \beta$} (D);
\end{tikzpicture}
\]
Using the same notation as the previous coherence diagram, the diagram for the underlying maps of cospans becomes:
\[
		\begin{tikzpicture}
			\node (a) at (-4,0) {$k+m$};
			\node (b) at (1,0) {$(l+_m n) + (r+_s t)$};
			\node (c) at (6,0) {$s+u$};
			\node (a2) at (-4,1) {$k+m$};
			\node (b2) at (1,1) {$(r +_s t) + (l +_m n)$};
			\node (c2) at (6,1) {$s+u$};
                                \node (a3) at (-4,2) {$k+m$};
			\node (b3) at (1,2) {$(r+l) +_{(s+m)} (t+n)$};
			\node (c3) at (6,2) {$s+u$};
                                \node (a5) at (-4,-1) {$k+m$};
			\node (b5) at (1,-1) {$(l+r) +_{(m+s)} (n+t)$};
			\node (c5) at (6,-1) {$s+u$};
                                \node (a6) at (-4,-2) {$k+m$};
			\node (b6) at (1,-2) {$(r+l) +_{(s+m)} (t+n)$};
			\node (c6) at (6,-2) {$s+u$};
			\path[->,font=\scriptsize,>=angle 90]
			(a) edge node[above]{$$} (b)
			(c) edge node[above]{$$} (b)
                                (a2) edge node[above]{$$} (b2)
			(c2) edge node[above]{$$} (b2)
                                (a) edge node[above]{$$} (a2)
                                (b) edge node[left]{$\beta$} (b2)
(b) edge node[right]{$\iota_1$} (b2)
			(c) edge node[above]{$$} (c2)
                                (a3) edge node[above]{$$} (b3)
			(c3) edge node[above]{$$} (b3)
                                (a2) edge node[above]{$$} (a3)
                                (b2) edge node[left]{$\chi$} (b3)
(b2) edge node[right]{$\iota_2$} (b3)
			(c2) edge node[above]{$$} (c3)
                                (a5) edge node[above]{$$} (b5)
			(c5) edge node[above]{$$} (b5)
                                (a) edge node[above]{$$} (a5)
                                (b) edge node[left]{$\chi$} (b5)
(b) edge node[right]{$\iota_4$} (b5)
			(c) edge node[above]{$$} (c5)
                                (a6) edge node[above]{$$} (b6)
			(c6) edge node[above]{$$} (b6)
                                (a5) edge node[above]{$$} (a6)
                                (b5) edge node[left]{$\beta \odot \beta$} (b6)
 (b5) edge node[right]{$\iota_5$} (b6)
			(c5) edge node[above]{$$} (c6);
		\end{tikzpicture}
	\]
All the comments about the previous underlying coherence diagram of maps of cospans apply to this one. As for the decorations, the following diagram commutes in the category $F((r+l)+_{(s+m)}(t+n))$:
\[
\begin{tikzpicture}[scale=1.5]
\node (A) at (0,0) {$F(\chi \beta)(d_{(M_1 \odot M_2) \otimes (N_1 \odot N_2)})$};
\node (B) at (3,0) {$F(\chi)(d_{(N_1 \odot N_2) \otimes (M_1 \odot M_2)})$};
\node (C) at (0,-1) {$F(\beta \odot \beta)(d_{(M_1 \otimes N_1) \odot (M_2 \otimes N_2)})$};
\node (D) at (3,-1) {$d_{(N_1 \otimes M_1) \odot (N_2 \otimes M_2)}$};
\path[->,font=\scriptsize,>=angle 90]
(A) edge node[above]{$F(\chi)(\iota_1)$} (B)
(B) edge node[right]{$\iota_2$} (D)
(A) edge node[left]{$F(\beta \odot \beta)(\iota_3)$} (C)
(C) edge node[above]{$\iota_4$} (D);
\end{tikzpicture}
\]
since $$F(\chi \beta)(d_{(M_1 \odot M_2) \otimes (N_1 \odot N_2)}) = F((\beta \odot \beta)\chi)(d_{(M_1 \odot M_2) \otimes (N_1 \odot N_2)})$$
as the above underlying diagram of maps of cospans commutes. The other diagrams are shown to commute similarly.
\end{proof}

\begin{lem}
The double category $F\mathbb{C}\textnormal{sp}$ is fibrant.
\end{lem}

\begin{proof}
Let $f \colon c \to c^\prime$ be a vertical 1-morphism in $F\mathbb{C}\textnormal{sp}$. We can lift $f$ to the companion horizontal 1-cell $\hat{f}$:
\[
\begin{tikzpicture}[scale=1.5]
\node (A) at (0,0) {$c$};
\node (B) at (1,0) {$c^\prime$};
\node (C) at (2,0) {$c^\prime$};
\node (D) at (1,-0.5) {$!_{c^\prime} \in F(c^\prime)$};
\path[->,font=\scriptsize,>=angle 90]
(A) edge node[above]{$f$} (B)
(C) edge node[above]{$1$} (B);
\end{tikzpicture}
\]
and then obtain the following two 2-morphisms:
\[
\begin{tikzpicture}[scale=1.5]
\node (A) at (0,0.5) {$c$};
\node (A') at (0,-0.5) {$c^\prime$};
\node (B) at (1,0.5) {$c^\prime$};
\node (C) at (2,0.5) {$c^\prime$};
\node (C') at (2,-0.5) {$c^\prime$};
\node (D) at (1,-0.5) {$c^\prime$};
\node (E) at (3,0.5) {$!_{c^\prime} \in F(c^\prime)$};
\node (F) at (3,-0.5) {$!_{c^\prime} \in F(c^\prime)$};
\node (G) at (4,0.5) {$c$};
\node (H) at (5,0.5) {$c$};
\node (I) at (6,0.5) {$c$};
\node (G') at (4,-0.5) {$c$};
\node (H') at (5,-0.5) {$c^\prime$};
\node (I') at (6,-0.5) {$c^\prime$};
\node (J) at (7,0.5) {$!_{c} \in F(c)$};
\node (K) at (7,-0.5) {$!_{c^\prime} \in F(c^\prime)$};
\node (L) at (1,-1) {$\iota_{1_{c^\prime}} = 1_{!_{c^\prime}}$};
\node (M) at (5,-1) {$\iota_f \colon F(f)(!_c) \to !_{c^\prime}$};
\path[->,font=\scriptsize,>=angle 90]
(A) edge node[above]{$f$} (B)
(C) edge node[above]{$1$} (B)
(A) edge node[left]{$f$} (A')
(C) edge node[left]{$1$} (C')
(A') edge node[above] {$1$} (D)
(C') edge node[above] {$1$} (D)
(B) edge node [left] {$1$} (D)
(G) edge node [above] {$1$} (H)
(G) edge node [left] {$1$} (G')
(H) edge node [left] {$f$} (H')
(G') edge node [above] {$f$} (H')
(I) edge node [above] {$1$} (H)
(I) edge node [left] {$f$} (I')
(I') edge node [above] {$1$} (H');
\end{tikzpicture}
\]
which satisfy the equations:
\[
\begin{tikzpicture}[scale=1.5]
\node (N) at (0,1.5) {$c$};
\node (O) at (1,1.5) {$c$};
\node (P) at (2,1.5) {$c$};
\node (Q) at (-1,1.5) {$!_c \in F(c)$};
\node (A) at (0,0.5) {$c$};
\node (A') at (0,-0.5) {$c^\prime$};
\node (B) at (1,0.5) {$c^\prime$};
\node (C) at (2,0.5) {$c^\prime$};
\node (C') at (2,-0.5) {$c^\prime$};
\node (D) at (1,-0.5) {$c^\prime$};
\node (E) at (-1,0.5) {$!_{c^\prime} \in F(c^\prime)$};
\node (F) at (-1,-0.5) {$!_{c^\prime} \in F(c^\prime)$};
\node (G) at (4,1) {$c$};
\node (H) at (5,1) {$c$};
\node (I) at (6,1) {$c$};
\node (G') at (4,0) {$c^\prime$};
\node (H') at (5,0) {$c^\prime$};
\node (I') at (6,0) {$c^\prime$};
\node (J) at (7,1) {$!_{c} \in F(c)$};
\node (K) at (7,0) {$!_{c^\prime} \in F(c^\prime)$};
\node (Q) at (1,-1) {$\iota_f \colon F(f)(!_c) \to !_{c^\prime}$};
\node (L) at (1,-1.5) {$\iota_{c^\prime} = 1_{!_{c^\prime}}$};
\node (M) at (5,-0.5) {$\iota_f \colon F(f)(!_c) \to !_{c^\prime}$};
\node (R) at (3,0.5) {$=$};
\path[->,font=\scriptsize,>=angle 90]
(N) edge node[above]{$1$} (O)
(P) edge node[above]{$1$} (O)
(N) edge node[left]{$1$} (A)
(O) edge node[left]{$f$} (B)
(P) edge node[left]{$f$} (C)
(A) edge node[above]{$f$} (B)
(C) edge node[above]{$1$} (B)
(A) edge node[left]{$f$} (A')
(C) edge node[left]{$1$} (C')
(A') edge node[above] {$1$} (D)
(C') edge node[above] {$1$} (D)
(B) edge node [left] {$1$} (D)
(G) edge node [above] {$1$} (H)
(G) edge node [left] {$f$} (G')
(H) edge node [left] {$f$} (H')
(G') edge node [above] {$1$} (H')
(I) edge node [above] {$1$} (H)
(I) edge node [left] {$f$} (I')
(I') edge node [above] {$1$} (H');
\end{tikzpicture}
\]
\[
\begin{tikzpicture}[scale=1.5]
\node (G) at (-1,0.5) {$c$};
\node (H) at (-1,-0.5)  {$c^\prime$};
\node (I) at (-2,0.5) {$c$};
\node (J) at (-2,-0.5) {$c$};
\node (A) at (0,0.5) {$c$};
\node (A') at (0,-0.5) {$c^\prime$};
\node (B) at (1,0.5) {$c^\prime$};
\node (C) at (2,0.5) {$c^\prime$};
\node (C') at (2,-0.5) {$c^\prime$};
\node (D) at (1,-0.5) {$c^\prime$};
\node (E) at (1,1) {$!_{c^\prime} \in F(c^\prime)$};
\node (F) at (1,-1) {$!_{c^\prime} \in F(c^\prime)$};

\node (L) at (1,-1.5) {$\iota_{c^\prime} = 1_{!_{c^\prime}}$};
\node (E') at (-1,1) {$!_{c} \in F(c)$};
\node (F') at (-1,-1) {$!_{c^\prime} \in F(c^\prime)$};

\node (L') at (-1,-1.5) {$\iota_{f} \colon F(f)(!_c) \to !_{c^\prime}$};

\node (M) at (2.5,0) {$=$};
\node (N) at (3,0.5) {$c$};
\node (O) at (3,-0.5) {$c$};
\node (P) at (4,0.5) {$c^\prime$};
\node (Q) at (4,-0.5) {$c^\prime$};
\node (R) at (5,0.5) {$c^\prime$};
\node (S) at (5,-0.5) {$c^\prime$};
\node (T) at (4,1) {$!_{c^\prime} \in F(c^\prime)$};
\node (U) at (4,-1) {$!_{c^\prime} \in F(c^\prime)$};
\node (V) at (4,-1.5) {$\iota_{c^\prime} = 1_{!_{c^\prime}}$};

\path[->,font=\scriptsize,>=angle 90]
(N) edge node[left]{$1$} (O)
(P) edge node[left]{$1$} (Q)
(R) edge node[left]{$1$} (S)
(N) edge node[above]{$f$} (P)
(O) edge node[above]{$f$} (Q)
(R) edge node[above]{$1$} (P)
(S) edge node[above]{$1$} (Q)

(A) edge node[above]{$f$} (B)
(C) edge node[above]{$1$} (B)
(A) edge node[left]{$f$} (A')
(C) edge node[left]{$1$} (C')
(A') edge node[above] {$1$} (D)
(C') edge node[above] {$1$} (D)
(B) edge node [left] {$1$} (D)
(A) edge node[above]{$1$} (G)
(G) edge node[left]{$f$} (H)
(A') edge node[above]{$1$} (H)
(J) edge node[above] {$f$} (H)
(I) edge node[left] {$1$} (J)
(I) edge node [above] {$1$} (G);
\end{tikzpicture}
\]
The right hand sides of the above two equations are given respectively by the 2-morphisms $U_f$ and $1_{\hat{f}}$. The conjoint of $f$ is given by the $F$-decorated cospan $\check{f}$ which is just the opposite of the companion above:
\[
\begin{tikzpicture}[scale=1.5]
\node (A) at (0,0) {$c^\prime$};
\node (B) at (1,0) {$c^\prime$};
\node (C) at (2,0) {$c$};
\node (D) at (4,0) {$!_{c^\prime} \in F(c^\prime)$};
\path[->,font=\scriptsize,>=angle 90]
(A) edge node[above]{$1$} (B)
(C) edge node[above]{$f$} (B);
\end{tikzpicture}
\]
\end{proof}

The property of being fibrant is what allows us to lift the monoidal structure from the object category of a double category to its arrow category and obtain a symmetric monoidal bicategory. The following result, which only requires fibrancy on vertical 1-isomorphisms (isofibrancy) is due to Shulman \cite{Shul}:

\begin{thm}[Shulman]\label{Shul}
Let $\mathbb{X}$ be an isofibrant symmetric monoidal pseudo double category. Then the horizontal bicategory $H(\mathbb{X})$ of $\mathbb{X}$ is a symmetric monoidal bicategory which has:
\begin{enumerate}
\item{objects as those of $\mathbb{X}$,}
\item{morphisms as horizontal 1-cells of $\mathbb{X}$, and}
\item{2-morphisms as globular 2-morphisms of $\mathbb{X}$.}
\end{enumerate}
\end{thm}

\begin{cor}
Let $(\mathsf{C},+,0)$ be a category with finite colimts and $F \colon \mathsf{C} \to \mathbf{Cat}$ a symmetric lax monoidal pseudofunctor. Then there exists a symmetric monoidal bicategory $F \mathbf{Csp} \coloneqq H(F\mathbb{C}\textnormal{sp})$ which has:
\begin{enumerate}
\item{objects as those of $\mathsf{A}$,}
\item{morphisms as $F$-decorated cospans:
\[
\begin{tikzpicture}[scale=1.5]
\node (A) at (0,0) {$a$};
\node (B) at (1,0) {$c$};
\node (C) at (2,0) {$b$};
\node (D) at (4,0) {$d \in F(c)$};
\path[->,font=\scriptsize,>=angle 90]
(A) edge node[above]{$i$} (B)
(C) edge node[above]{$o$} (B);
\end{tikzpicture}
\]
and}
\item{2-morphisms as maps of cospans in $\mathsf{A}$ of the form:
\[
\begin{tikzpicture}[scale=1.5]
\node (A) at (0,0) {$a$};
\node (B) at (1,0.5) {$c$};
\node (C) at (2,0) {$b$};
\node (E) at (1,-0.5) {$c^\prime$};
\node (D) at (3,0.5) {$d \in F(c)$};
\node (F) at (3,-0.5) {$d^\prime \in F(c^\prime)$};
\path[->,font=\scriptsize,>=angle 90]
(A) edge node[above]{$i$} (B)
(C) edge node[above]{$o$} (B)
(A) edge node[below]{$i^\prime$} (E)
(B) edge node[left]{$h$} (E)
(C) edge node[below]{$o^\prime$} (E);
\end{tikzpicture}
\]
together with a morphism $\iota \colon F(h)(d) \to d^\prime$ in $F(c^\prime)$.}
\end{enumerate}
\end{cor}

\begin{proof}
This follows immediately by Shulman's Theorem \ref{Shul} above applied to the fibrant symmetric monoidal double category $F\mathbb{C}\textnormal{sp}$.
\end{proof}

This symmetric monoidal bicategory $F\mathbf{Csp}$ is a superior version of the symmetric monoidal bicategory $F$Cospan$(\mathsf{A})$ constructed earlier by the second author \cite{Cour} in that there is greater flexibility in what 2-morphisms are allowed. 

%The previous bicategory constructed by the second author suffered even more so from the issue discussed in the introduction. Given a symmetric lax monoidal functor $F \colon \mathrm{A} \to \mathrm{Set}$, a result of Fong \cite{Fong} yields a symmetric monoidal category $F \textnormal{Cospan}(\mathrm{A})$ which has:
%\begin{enumerate}
%\item{objects as those of $\mathrm{A}$ and}
%\item{morphisms as isomorphism classes of decorated cospans in $\mathrm{A}$.}
%\end{enumerate}
%The second author then using a result of Shulman \cite{Shul} extended this to a bicategory also called $F \textnormal{Cospan}(\mathrm{A})$ which has:
%\begin{enumerate}
%\item{objects as those of $\mathrm{A}$,}
%\item{morphisms as now just decorated cospans in $\mathrm{A}$, and}
%\item{2-morphisms as pairs of commuting diagrams:
%\[
%\begin{tikzpicture}[scale=1.5]
%\node (A) at (0,0) {$a$};
%\node (B) at (1,0.5) {$c$};
%\node (B') at (1,-0.5) {$c^\prime$};
%\node (C) at (2,0) {$b$};
%\node (D) at (3,0) {$1$};
%\node (E) at (4,0.5) {$F(c)$};
%\node (E') at (4,-0.5) {$F(c^\prime)$};
%\path[->,font=\scriptsize,>=angle 90]
%(A) edge node[above]{$i$} (B)
%(C) edge node[above]{$o$} (B)
%(A) edge node[below]{$i^\prime$} (B')
%(C) edge node[below]{$o^\prime$} (B')
%(B) edge node [left] {$f$} (B')
%(D) edge node [above] {$d$} (E)
%(D) edge node [below] {$d^\prime$} (E')
%(E) edge node [right] {$F(f)$} (E');
%\end{tikzpicture}
%\]
%}
%\end{enumerate}
%This was discussed in the introduction and it was mentioned how in the symmetric monoidal category version of $F \textnormal{Cospan}(\mathrm{A})$, that the two single-edged graphs:
%\[
%\begin{tikzpicture}[scale=1.5]
%\node (A) at (0,0) {$v_1$};
%\node (B) at (1,0) {$v_2$};
%\path[->,font=\scriptsize,>=angle 90]
%(A) edge node[above]{$e$} (B);
%\end{tikzpicture}
%\]
%and
%\[
%\begin{tikzpicture}[scale=1.5]
%\node (A) at (0,0) {$v_1$};
%\node (B) at (1,0) {$v_2.$};
%\path[->,font=\scriptsize,>=angle 90]
%(A) edge node[above]{$e^\prime$} (B);
%\end{tikzpicture}
%\]
%resided in distinct isomorphism classes. In the bicategorical version of $F \textnormal{Cospan}(\mathrm{A})$ where we are no longer considering decorated cospans up to isomorphism class, there is no 2-morphism between these two graphs due to the strict commutativity of the triangle to the right above. This problem does not occur in the symmetric monoidal bicategory $H(\mathbb{F}\textnormal{Cospan}(\mathrm{A}))$ and there is in fact a 2-(iso)morphism between these two graphs given by the map $\iota \colon F(f)(d) \to d^\prime$ which maps the edge $e$ to the edge $e^\prime$, where $f$ is the underlying map of vertices.

%Again, let $F \colon \mathbf{FinSet} \to \mathbf{Set}$ be the symmetric lax monoidal functor that maps a finite set $N$ to the large set of all possible graph structures whose underlying set of vertices is $N$. Then a result of Fong \cite{Fong} yields a symmetric monoidal category $F\textnormal{Cospan}(\mathbf{FinSet})$ which has:
%\begin{enumerate}
%\item{finite sets as objects and}
%\item{isomorphism classes of `open' graphs as morphisms, where an open graph is a cospan of finite sets whose apex is equipped with the structure of a graph.}
%\end{enumerate}

%An example of an open graph is:
%\begin{center}
%\begin{tikzpicture}[scale=2.15]
%\node (N) at (0.5,-1.25) {$N=\{v_1,v_2,v_3\}$};
%    \node[circle,draw,inner sep=1pt,fill=gray,color=gray]         (x) at (-1.4,-.43) {};
%    \node at (-1.4,-.9) {$X$};
%    \node       (A) at (0,0) {$v_1$};
%    \node     (B) at (1,0) {$v_2$};
%    \node        (C) at (0.5,-.86) {$v_3$};
%    \node[circle,draw,inner sep=1pt,fill=gray,color=gray]         (y1) at (2.4,-.25) {};
%    \node[circle,draw,inner sep=1pt,fill=gray,color=gray]         (y2) at (2.4,-.61) {};
%    \node at (2.4,-.9) {$Y$};
%    \path (B) edge  [bend right,->-] node[above] {$e_1$} (A);
%    \path (A) edge  [bend right,->-] node[below] {$e_2$} (B);
%    \path (A) edge  [->-] node[left] {$e_3$} (C);
%    \path (C) edge  [->-] node[right] {$e_4$} (B);
%    \path[color=gray, very thick, shorten >=10pt, shorten <=5pt, ->, >=stealth] (x) edge (A);
%    \path[color=gray, very thick, shorten >=10pt, shorten <=5pt, ->, >=stealth] (y1) edge (B);
%    \path[color=gray, very thick, shorten >=10pt, shorten <=5pt, ->, >=stealth] (y2) edge (B);
%\end{tikzpicture}
%\end{center}
%Here, the sets $X$ and $Y$ serve as inputs and outputs. The second author then extended Fong's category using a result of Shulman \cite{Shul} to obtain a symmetric monoidal bicategory which has:
 
%the following two single-edged graphs constituted distinct isomorphism classes:
%\[
%\begin{tikzpicture}[scale=1.5]
%\node (A) at (0,0) {$v_1$};
%\node (B) at (2,0) {$v_2$};
%\node (C) at (0,-2) {$v_1$};
%\node (D) at (2,-2) {$v_2$};
%\path[->,font=\scriptsize,>=angle 90]
%(A) edge node[above]{$e$} (B)
%(C) edge node[above]{$e^\prime$} (D);
%\end{tikzpicture}
%\]

We can then decategorify this symmetric monoidal bicategory to obtain a symmetric monoidal category similar to the one obtained using Fong's result, but with larger isomorphism classes:

\begin{cor}
Given a symmetric lax monoidal pseudofunctor $F \colon \mathsf{A} \to \mathbf{Cat}$ where $\mathsf{A}$ is a category with finite colimits and whose monoidal structure is given by binary coproducts, there exists a symmetric monoidal category $F\textnormal{Csp} \coloneqq D(F\mathbf{Csp})$ which has:
\begin{enumerate}
\item{objects as those of $\mathsf{A}$ and}
\item{morphisms as isomorphism classes of $F$-decorated cospans of $\mathsf{A}$, where an $F$-decorated cospan is given by a pair:
\[
\begin{tikzpicture}[scale=1.5]
\node (A) at (0,0) {$a$};
\node (B) at (1,0) {$c$};
\node (C) at (2,0) {$b$};
\node (D) at (4,0) {$d \in F(c)$};
\path[->,font=\scriptsize,>=angle 90]
(A) edge node[above]{$i$} (B)
(C) edge node[above]{$o$} (B);
\end{tikzpicture}
\]
Given another $F$-decorated cospan:
\[
\begin{tikzpicture}[scale=1.5]
\node (A) at (0,0) {$a$};
\node (B) at (1,0) {$c^\prime$};
\node (C) at (2,0) {$b$};
\node (D) at (4,0) {$d^\prime \in F(c^\prime)$};
\path[->,font=\scriptsize,>=angle 90]
(A) edge node[below]{$i^\prime$} (B)
(C) edge node[below]{$o^\prime$} (B);
\end{tikzpicture}
\]
these two $F$-decorated cospans are in the same isomorphism class if there exists an isomorphism $f \colon c \to c^\prime$ such that following diagram commutes:
\[
\begin{tikzpicture}[scale=1.5]
\node (A) at (0,0) {$a$};
\node (B') at (1,0.5) {$c$};
\node (B) at (1,-0.5) {$c^\prime$};
\node (C) at (2,0) {$b$};
\path[->,font=\scriptsize,>=angle 90]
(A) edge node[below]{$i^\prime$} (B)
(C) edge node[below]{$o^\prime$} (B)
(A) edge node[above]{$i$} (B')
(C) edge node[above]{$o$} (B')
(B') edge node[left]{$f$} (B);
\end{tikzpicture}
\]
and there exists an isomorphism $\iota \colon F(f)(d) \to d^\prime$ in $F(c^\prime)$.}
\end{enumerate}
\end{cor}
In this symmetric monoidal category, isomorphism classes are as they should morally be, and the instance of two graphs having different edge sets does not prevent them from being in the same isomorphism class due to the isomorphism $\iota$.

\begin{comment}
\section{Maps of decorated cospan double categories}\label{MapsDecCospansDoubleCat}
Let $F \colon \mathsf{A} \to \mathbf{Cat}$ be a symmetric lax monoidal pseudofunctor. Then by Theorem \ref{DC} of the previous section, we get a symmetric monoidal double category $F\mathbb{C}\textnormal{sp}$. This symmetric monoidal double category has:
\begin{enumerate}
\item{objects as those of $\mathsf{A}$,}
\item{vertical 1-morphisms as morphisms of $\mathsf{A}$,}
\item{horizontal 1-cells as pairs:
\[
\begin{tikzpicture}[scale=1.5]
\node (A) at (0,0) {$a$};
\node (B) at (1,0) {$c$};
\node (C) at (2,0) {$b$};
\node (D) at (4,0) {$d \in F(c)$};
\path[->,font=\scriptsize,>=angle 90]
(A) edge node[above]{$i$} (B)
(C) edge node[above]{$o$} (B);
\end{tikzpicture}
\]
and}
\item{2-morphisms as maps of cospans in $\mathsf{A}$
\[
\begin{tikzpicture}[scale=1.5]
\node (A) at (0,0.5) {$a$};
\node (A') at (0,-0.5) {$a^\prime$};
\node (B) at (1,0.5) {$c$};
\node (C) at (2,0.5) {$b$};
\node (C') at (2,-0.5) {$b^\prime$};
\node (D) at (1,-0.5) {$c^\prime$};
\node (E) at (3,0.5) {$d \in F(c)$};
\node (F) at (3,-0.5) {$d^\prime \in F(c^\prime)$};
\path[->,font=\scriptsize,>=angle 90]
(A) edge node[above]{$$} (B)
(C) edge node[above]{$$} (B)
(A) edge node[left]{$f$} (A')
(C) edge node[left]{$g$} (C')
(A') edge node {$$} (D)
(C') edge node {$$} (D)
(B) edge node [left] {$h$} (D);
\end{tikzpicture}
\]
together with a morphism $\iota \colon F(h)(d) \to d^\prime$ in $F(c^\prime)$.}
\end{enumerate}
Given another symmetric lax monoidal pseudofunctor $F^\prime \colon \mathsf{A^\prime} \to \mathbf{Cat}$, we can obtain another symmetric monoidal double category $F^\prime \mathbb{C}\textnormal{sp}$. Then a map from $F\mathbb{C}\textnormal{sp}$ to $F^\prime \mathbb{C}\textnormal{sp}$ will be a double functor $\mathbb{H} \colon F\mathbb{C}\textnormal{sp} \to F' \mathbb{C}\textnormal{sp}$ whose object component is given by a finite colimit preserving functor $\mathbb{H}_0 = H \colon \mathsf{A} \to \mathsf{A^\prime}$ and whose arrow component is given by a functor $\mathbb{H}_1$ defined on horizontal 1-cells by:
\[
\begin{tikzpicture}[scale=1.5]
\node (A) at (0,0) {$a$};
\node (B) at (1,0) {$c$};
\node (C) at (2,0) {$b$};
\node (D) at (1,-0.5) {$d \in F(c)$};
\node (E) at (2.75,0) {$\mapsto$};
\node (A') at (3.5,0) {$H(a)$};
\node (B') at (4.5,0) {$H(c)$};
\node (C') at (5.5,0) {$H(b)$};
\node (D') at (4.5,-0.5) {$E(d) \in F^\prime(H(c))$};
\path[->,font=\scriptsize,>=angle 90]
(A) edge node[above]{$i$} (B)
(C) edge node[above]{$o$} (B)
(A') edge node[above]{$H(i)$} (B')
(C') edge node[above]{$H(o)$} (B');
\end{tikzpicture}
\]
and on 2-morphisms by:
\[
\begin{tikzpicture}[scale=1.5]
\node (A) at (0,0.5) {$a$};
\node (A') at (0,-0.5) {$a^\prime$};
\node (B) at (1,0.5) {$c$};
\node (C) at (2,0.5) {$b$};
\node (C') at (2,-0.5) {$b^\prime$};
\node (D) at (1,-0.5) {$c^\prime$};
\node (E) at (3,0.5) {$d \in F(c)$};
\node (F) at (3,-0.5) {$d^\prime \in F(c^\prime)$};
\node (G) at (1,-1) {$\iota \colon F(h)(d) \to d^\prime$};
\node (A'') at (4.25,0.5) {$H(a)$};
\node (A''') at (4.25,-0.5) {$H(a^\prime)$};
\node (B'') at (5.25,0.5) {$H(c)$};
\node (C'') at (6.25,0.5) {$H(b)$};
\node (C''') at (6.25,-0.5) {$H(b^\prime)$};
\node (D'') at (5.25,-0.5) {$H(c^\prime)$};
\node (E'') at (7.5,0.5) {$E(d) \in F^\prime(H(c))$};
\node (F'') at (7.5,-0.5) {$E(d^\prime) \in F^\prime(H(c^\prime))$};
\node (G'') at (5.25,-1) {$E(\iota) \colon F^\prime(H(h))(E(d)) \to E(d^\prime)$};
\node (H) at (3.5,0) {$\mapsto$};
\path[->,font=\scriptsize,>=angle 90]
(A) edge node[above]{$$} (B)
(C) edge node[above]{$$} (B)
(A) edge node[left]{$f$} (A')
(C) edge node[left]{$g$} (C')
(A') edge node {$$} (D)
(C') edge node {$$} (D)
(B) edge node [left] {$h$} (D)
(A'') edge node[above]{$$} (B'')
(C'') edge node[above]{$$} (B'')
(A'') edge node[left]{$H(f)$} (A''')
(C'') edge node[left]{$H(g)$} (C''')
(A''') edge node {$$} (D'')
(C''') edge node {$$} (D'')
(B'') edge node [left] {$H(h)$} (D'');
\end{tikzpicture}
\]
where $E \colon \mathbf{Cat} \to \mathbf{Cat}$ is a symmetric lax monoidal pseudofuctor such that the following diagram commutes:
\[
\begin{tikzpicture}[scale=1.5]
\node (A) at (0,0) {$\mathsf{A}$};
\node (B) at (1,0) {$\mathbf{Cat}$};
\node (C) at (0,-1) {$\mathsf{A^\prime}$};
\node (D) at (1,-1) {$\mathbf{Cat}$};
\path[->,font=\scriptsize,>=angle 90]
(A) edge node[above]{$F$} (B)
(A) edge node[left]{$H$} (C)
(B) edge node[right]{$E$} (D)
(C) edge node[above]{$F^\prime$} (D);
\end{tikzpicture}
\]
Recall that we can think of the element $d \in F(c)$ as a morphism $d \colon 1 \to F(c)$ and the morphism $\iota \colon F(h)(d) \to d^\prime$ of $F(c^\prime)$ as a 2-morphism  in $\mathbf{Cat}$:
\[
\begin{tikzpicture}[scale=1.5]
\node (A) at (0,-0.5) {$1$};
\node (B) at (1,0) {$F(c)$};
\node (D) at (1,-1) {$F(c^\prime)$};
\node (C) at (0.75,-0.5) {$\Swarrow$};
\path[->,font=\scriptsize,>=angle 90]
(A) edge node[above]{$d$} (B)
(A) edge node[below]{$d^\prime$} (D)
(B) edge node[right]{$F(h)$} (D);
\end{tikzpicture}
\]
Applying the pseudofunctor $E \colon \mathbf{Cat} \to \mathbf{Cat}$ to this diagram yields:
\[
\begin{tikzpicture}[scale=1.5]
\node (A) at (0,-0.5) {$1 \xrightarrow{\sim} E(1)$};
\node (B) at (1.5,0) {$E(F(c))$};
\node (D) at (1.5,-1) {$E(F(c^\prime))$};
\node (C) at (1,-0.5) {$\Swarrow$};
\path[->,font=\scriptsize,>=angle 90]
(A) edge node[above]{$E(d)$} (B)
(A) edge node[below]{$E(d^\prime)$} (D)
(B) edge node[right]{$E(F(h))$} (D);
\end{tikzpicture}
\]
Then because the above square commutes, this is the same as 
\[
\begin{tikzpicture}[scale=1.5]
\node (A) at (0,-0.5) {$1 \xrightarrow{\sim} E(1)$};
\node (B) at (1.5,0) {$F^\prime(H(c))$};
\node (D) at (1.5,-1) {$F^\prime(H(c^\prime))$};
\node (C) at (1,-0.5) {$\Swarrow$};
\path[->,font=\scriptsize,>=angle 90]
(A) edge node[above]{$E(d)$} (B)
(A) edge node[below]{$E(d^\prime)$} (D)
(B) edge node[right]{$F^\prime(H((h))$} (D);
\end{tikzpicture}
\]
which is the same as a morphism $E(\iota) \colon F^\prime(H(h))(E(d)) \to E(d^\prime)$ in $F^\prime(H(c^\prime))$. To check that the above recipe is functorial, given two vertically composable 2-morphisms in $F\mathbb{C}\textnormal{sp}$:
\[
\begin{tikzpicture}[scale=1.5]
\node (A) at (0,0.5) {$a$};
\node (A') at (0,-0.5) {$a^\prime$};
\node (B) at (1.5,0.5) {$c$};
\node (C) at (3,0.5) {$b$};
\node (C') at (3,-0.5) {$b^\prime$};
\node (D) at (1.5,-0.5) {$c^\prime$};
\node (E) at (4.5,0.5) {$d \in F(c)$};
\node (F) at (4.5,-0.5) {$d^\prime \in F(c^\prime)$};
\node (G) at (1.5,-1) {$\iota \colon F(h)(d) \to d^\prime$};
\node (A'') at (0,-1.5) {$a^\prime$};
\node (A''') at (0,-2.5) {$a''$};
\node (B'') at (1.5,-1.5) {$c^\prime$};
\node (C'') at (3,-1.5) {$b^\prime$};
\node (C''') at (3,-2.5) {$b''$};
\node (D'') at (1.5,-2.5) {$c''$};
\node (E'') at (4.5,-1.5) {$d^\prime \in F(c^\prime)$};
\node (F'') at (4.5,-2.5) {$d'' \in F(c'')$};
\node (G'') at (1.5,-3) {$\iota^\prime \colon F(h^\prime)(d^\prime) \to d''$};
\path[->,font=\scriptsize,>=angle 90]
(A) edge node[above]{$$} (B)
(C) edge node[above]{$$} (B)
(A) edge node[left]{$f$} (A')
(C) edge node[left]{$g$} (C')
(A') edge node [above]{$$} (D)
(C') edge node [above]{$$} (D)
(B) edge node [left] {$h$} (D)
(A'') edge node[above]{$$} (B'')
(C'') edge node[above]{$$} (B'')
(A'') edge node[left]{$f^\prime$} (A''')
(C'') edge node[left]{$g^\prime$} (C''')
(A''') edge node [above]{$$} (D'')
(C''') edge node [above]{$$} (D'')
(B'') edge node [left] {$h^\prime$} (D'');
\end{tikzpicture}
\]
if we first compose these, the result is:
\[
\begin{tikzpicture}[scale=1.5]
\node (A) at (0,0.5) {$a$};
\node (A') at (0,-0.5) {$a''$};
\node (B) at (1.5,0.5) {$c$};
\node (C) at (3,0.5) {$b$};
\node (C') at (3,-0.5) {$b''$};
\node (D) at (1.5,-0.5) {$c''$};
\node (E) at (4.5,0.5) {$d \in F(c)$};
\node (F) at (4.5,-0.5) {$d'' \in F(c'')$};
\node (G) at (1.5,-1) {$\iota ^\prime \iota \colon F(h^\prime h)(d) \to d''$};
\path[->,font=\scriptsize,>=angle 90]
(A) edge node[above]{$$} (B)
(C) edge node[above]{$$} (B)
(A) edge node[left]{$f^\prime f$} (A')
(C) edge node[left]{$g^\prime g$} (C')
(A') edge node [above]{$$} (D)
(C') edge node [above]{$$} (D)
(B) edge node [left] {$h^\prime h$} (D);
\end{tikzpicture}
\]
and then the image of this 2-morphism under the double functor $\mathbb{H}$ is given by:
\[
\begin{tikzpicture}[scale=1.5]
\node (A) at (0,0.5) {$H(a)$};
\node (A') at (0,-0.5) {$H(a'')$};
\node (B) at (1.5,0.5) {$H(c)$};
\node (C) at (3,0.5) {$H(b)$};
\node (C') at (3,-0.5) {$H(b'')$};
\node (D) at (1.5,-0.5) {$H(c'')$};
\node (E) at (4.5,0.5) {$E(d) \in F^\prime(H(c))$};
\node (F) at (4.5,-0.5) {$E(d'') \in F^\prime(H(c''))$};
\node (G) at (1.5,-1) {$E(\iota ^\prime \iota) \colon F^\prime(H(h^\prime h))(E(d)) \to E(d'').$};
\path[->,font=\scriptsize,>=angle 90]
(A) edge node[above]{$$} (B)
(C) edge node[above]{$$} (B)
(A) edge node[left]{$H(f^\prime f)$} (A')
(C) edge node[left]{$H(g^\prime g)$} (C')
(A') edge node [above]{$$} (D)
(C') edge node [above]{$$} (D)
(B) edge node [left] {$H(h^\prime h)$} (D);
\end{tikzpicture}
\]
On the other hand, applying the double functor $\mathbb{H}$ first gives:
\[
\begin{tikzpicture}[scale=1.5]
\node (A) at (0,0.5) {$H(a)$};
\node (A') at (0,-0.5) {$H(a^\prime)$};
\node (B) at (1.5,0.5) {$H(c)$};
\node (C) at (3,0.5) {$H(b)$};
\node (C') at (3,-0.5) {$H(b^\prime)$};
\node (D) at (1.5,-0.5) {$H(c^\prime)$};
\node (E) at (4.5,0.5) {$E(d) \in F^\prime(H(c))$};
\node (F) at (4.5,-0.5) {$E(d^\prime) \in F^\prime(H(c^\prime))$};
\node (G) at (1.5,-1) {$E(\iota) \colon F^\prime(H(h))(E(d)) \to E(d^\prime)$};
\node (A'') at (0,-1.5) {$H(a^\prime)$};
\node (A''') at (0,-2.5) {$H(a'')$};
\node (B'') at (1.5,-1.5) {$H(c^\prime)$};
\node (C'') at (3,-1.5) {$H(b^\prime)$};
\node (C''') at (3,-2.5) {$H(b'')$};
\node (D'') at (1.5,-2.5) {$H(c'')$};
\node (E'') at (4.5,-1.5) {$E(d^\prime) \in F^\prime(H(c^\prime))$};
\node (F'') at (4.5,-2.5) {$E(d'') \in F^\prime(H(c''))$};
\node (G'') at (1.5,-3) {$E(\iota^\prime) \colon F^\prime(H(h^\prime))(E(d^\prime)) \to E(d'')$};
\path[->,font=\scriptsize,>=angle 90]
(A) edge node[above]{$$} (B)
(C) edge node[above]{$$} (B)
(A) edge node[left]{$H(f)$} (A')
(C) edge node[left]{$H(g)$} (C')
(A') edge node [above]{$$} (D)
(C') edge node [above]{$$} (D)
(B) edge node [left] {$H(h)$} (D)
(A'') edge node[above]{$$} (B'')
(C'') edge node[above]{$$} (B'')
(A'') edge node[left]{$H(f^\prime)$} (A''')
(C'') edge node[left]{$H(g^\prime)$} (C''')
(A''') edge node [above]{$$} (D'')
(C''') edge node [above]{$$} (D'')
(B'') edge node [left] {$H(h^\prime)$} (D'');
\end{tikzpicture}
\]
and then composing these gives:
\[
\begin{tikzpicture}[scale=1.5]
\node (A) at (0,0.5) {$H(a)$};
\node (A') at (0,-0.5) {$H(a'')$};
\node (B) at (1.5,0.5) {$H(c)$};
\node (C) at (3,0.5) {$H(b)$};
\node (C') at (3,-0.5) {$H(b'')$};
\node (D) at (1.5,-0.5) {$H(c'')$};
\node (E) at (4.5,0.5) {$E(d) \in F^\prime(H(c))$};
\node (F) at (4.5,-0.5) {$E(d'') \in F^\prime(H(c''))$};
\node (G) at (1.5,-1) {$E(\iota ^\prime \iota) \colon F^\prime(H(h^\prime h))(E(d)) \to E(d'').$};
\path[->,font=\scriptsize,>=angle 90]
(A) edge node[above]{$$} (B)
(C) edge node[above]{$$} (B)
(A) edge node[left]{$H(f^\prime f)$} (A')
(C) edge node[left]{$H(g^\prime g)$} (C')
(A') edge node [above]{$$} (D)
(C') edge node [above]{$$} (D)
(B) edge node [left] {$H(h^\prime h)$} (D);
\end{tikzpicture}
\]
This double functor $\mathbb{H}$ satisfies the equations $S \mathbb{H}_1 = HS$ and $T \mathbb{H}_1=HT$.

Given two composable horizontal 1-cells $M$ and $N$ in $F\mathbb{C}\textnormal{sp}$:
\[
\begin{tikzpicture}[scale=1.5]
\node (A) at (0,0) {$a_1$};
\node (B) at (1,0) {$c_1$};
\node (C) at (2,0) {$b$};
\node (D) at (1,-0.5) {$d_1 \in F(c_1)$};
\node (E) at (3,0) {$b$};
\node (F) at (4,0) {$c_2$};
\node (G) at (5,0) {$a_2$};
\node (H) at (4,-0.5) {$d_2 \in F(c_2)$};
\path[->,font=\scriptsize,>=angle 90]
(A) edge node[above]{$i_1$} (B)
(C) edge node[above]{$o_1$} (B)
(E) edge node[above]{$i_2$} (F)
(G) edge node[above]{$o_2$} (F);
\end{tikzpicture}
\]
composing first gives $M \odot N$:
\[
\begin{tikzpicture}[scale=1.5]
\node (A) at (0,0) {$a_1$};
\node (B) at (1.5,0) {$c_1+_b c_2$};
\node (C) at (3,0) {$a_2$};
\node (D) at (1.5,-0.5) {$d \in F(c_1 +_b c_2)$};
\path[->,font=\scriptsize,>=angle 90]
(A) edge node[above]{$\psi j_{c_1} i_1$} (B)
(C) edge node[above]{$\psi j_{c_2} o_2$} (B);
\end{tikzpicture}
\]
where $$d \colon 1 \xrightarrow{\lambda^{-1}} 1 \times 1 \xrightarrow{d_1 \times d_2} F(c_1) \times F(c_2) \xrightarrow{\phi_{c_1,c_2}} F(c_1+c_2) \xrightarrow{F(\psi)}F(c_1 +_b c_2).$$ The image of this horizontal 1-cell is then given by $\mathbb{H}(M \odot N)$:
\[
\begin{tikzpicture}[scale=1.5]
\node (A) at (0,0) {$H(a_1)$};
\node (B) at (2,0) {$H(c_1+_b c_2)$};
\node (C) at (4,0) {$H(a_2)$};
\node (D) at (2,-0.5) {$E(d) \in F^\prime(H(c_1 +_b c_2))$};
\path[->,font=\scriptsize,>=angle 90]
(A) edge node[above]{$H(\psi j_{c_1} i_1)$} (B)
(C) edge node[above]{$H(\psi j_{c_2} o_2)$} (B);
\end{tikzpicture}
\]
where $$E(d) \colon 1 \xrightarrow{E(d)} E(F(c_1 +_b c_2)) = F^\prime(H(c_1 +_b c_2)).$$ On the other hand, the image of each horizontal 1-cell under the double functor $\mathbb{H}$ is given respectively by $\mathbb{H}(M)$ and $\mathbb{H}(N)$:
\[
\begin{tikzpicture}[scale=1.5]
\node (A) at (0,0) {$H(a_1)$};
\node (B) at (1,0) {$H(c_1)$};
\node (C) at (2,0) {$H(b)$};
\node (D) at (1,-0.5) {$E(d_1) \in F^\prime(H(c_1))$};
\node (E) at (3,0) {$H(b)$};
\node (F) at (4,0) {$H(c_2)$};
\node (G) at (5,0) {$H(a_2)$};
\node (H) at (4,-0.5) {$E(d_2) \in F^\prime(H(c_2))$};
\path[->,font=\scriptsize,>=angle 90]
(A) edge node[above]{$H(i_1)$} (B)
(C) edge node[above]{$H(o_1)$} (B)
(E) edge node[above]{$H(i_2)$} (F)
(G) edge node[above]{$H(o_2)$} (F);
\end{tikzpicture}
\]
Composing these then gives $\mathbb{H}(M) \odot \mathbb{H}(N)$:
\[
\begin{tikzpicture}[scale=1.5]
\node (A) at (0,0) {$H(a_1)$};
\node (B) at (2.25,0) {$H(c_1)+_{H(b)} H(c_2)$};
\node (C) at (4.5,0) {$H(a_2)$};
\node (D) at (2.25,-0.5) {$d^\prime \in F^\prime(H(c_1) +_{H(b)} H(c_2))$};
\path[->,font=\scriptsize,>=angle 90]
(A) edge node[above]{$\Psi j_{H(c_1)} H(i_1)$} (B)
(C) edge node[above]{$\Psi j_{H(c_2)} H(o_2)$} (B);
\end{tikzpicture}
\]
where $$d^\prime \colon 1 \xrightarrow{E(d_1) \times E(d_2)} F^\prime(H(c_1)) \times F^\prime(H(c_2)) \xrightarrow{\Phi_{H(c_1),H(c_2)}} F^\prime(H(c_1)+ H(c_2)) \xrightarrow{F^\prime (H(\Psi))} F^\prime(H(c_1) +_{H(b)} H(c_2)).$$
We then have a comparison constraint: $$\mathbb{H}_{M,N} \colon \mathbb{H}(M) \odot \mathbb{H}(N) \xrightarrow{\sim} \mathbb{H}(M \odot N)$$given by the globular 2-isomorphism:
\[
\begin{tikzpicture}[scale=1.5]
\node (A) at (0,0.5) {$H(a_1)$};
\node (A') at (0,-0.5) {$H(a_1)$};
\node (B) at (2.5,0.5) {$H(c_1)+_{H(b)} H(c_2)$};
\node (C) at (5,0.5) {$H(a_2)$};
\node (C') at (5,-0.5) {$H(a_2)$};
\node (D) at (2.5,-0.5) {$H(c_1 +_b c_2)$};
\node (E) at (7,0.5) {$d^\prime \in F^\prime(H(c_1)+_{H(b_1)}H(c_2))$};
\node (F) at (7,-0.5) {$E(d) \in F^\prime(H(c_1 +_b c_2))$};
\node (G) at (2.5,-1) {$\iota_{\kappa^{-1}} \colon F^\prime(\kappa^{-1})(d^\prime) \to E(d).$};
\path[->,font=\scriptsize,>=angle 90]
(A) edge node[above]{$\Psi j_{H(c_1)} H(i_1)$} (B)
(C) edge node[above]{$\Psi j_{H(c_2)} H(o_2)$} (B)
(A) edge node[left]{$1$} (A')
(C) edge node[left]{$1$} (C')
(A') edge node [above]{$H(\psi j_{c_1} i_1)$} (D)
(C') edge node [above]{$H(\psi j_{c_2} o_2)$} (D)
(B) edge node [left] {$\kappa^{-1}$} (D);
\end{tikzpicture}
\]
where $\kappa$ is the isomorphism $$\kappa \colon H(c_1 +_b c_2) \xrightarrow{\sim} H(c_1) +_{H(b)} H(c_2)$$ comes from the finite colimit preserving functor $H \colon \mathsf{A} \to \mathsf{A^\prime}$. The above diagram commutes by a similar argument to the one used in Theorem \ref{Equiv}. Similarly we have a unit comparison constraint $$\mathbb{H}_U \colon U_{\mathbb{H}(c)} \to \mathbb{H}(U_c)$$ given by the globular 2-isomorphism:
\[
\begin{tikzpicture}[scale=1.5]
\node (A) at (0,0.5) {$H(c)$};
\node (A') at (0,-0.5) {$H(c)$};
\node (B) at (1.5,0.5) {$H(c)$};
\node (C) at (3,0.5) {$H(c)$};
\node (C') at (3,-0.5) {$H(c)$};
\node (D) at (1.5,-0.5) {$H(c)$};
\node (E) at (4.5,0.5) {$!_{H(c)} \in F^\prime(H(c))$};
\node (F) at (4.5,-0.5) {$E(!_c) \in F^\prime(H(c))$};
\path[->,font=\scriptsize,>=angle 90]
(A) edge node[above]{$1$} (B)
(C) edge node[above]{$1$} (B)
(A) edge node[left]{$1$} (A')
(C) edge node[left]{$1$} (C')
(A') edge node [above]{$1$} (D)
(C') edge node [above]{$1$} (D)
(B) edge node [left] {$1$} (D);
\end{tikzpicture}
\]
where the morphism of decorations is the morphism $\iota \colon !_{H(c)} \to E(!_c)$ in $F'(H(c))$. These comparison constrains satisfy the coherence axioms of a monoidal category, namely:
\[
\begin{tikzpicture}[scale=1.5]
\node (A) at (0,0.5) {$(\mathbb{H}(M) \odot \mathbb{H}(N)) \odot \mathbb{H}(P)$};
\node (B) at (0,-0.5) {$\mathbb{H}(M \odot N) \odot \mathbb{H}(P)$};
\node (C) at (0,-1.5) {$\mathbb{H}((M \odot N) \odot P)$};
\node (A') at (3,0.5) {$\mathbb{H}(M) \odot (\mathbb{H}(N) \odot \mathbb{H}(P))$};
\node (B') at (3,-0.5) {$\mathbb{H}(M) \odot \mathbb{H}(N \odot P)$};
\node (C') at (3,-1.5) {$\mathbb{H}(M \odot (N \odot P))$};
\path[->,font=\scriptsize,>=angle 90]
(A) edge node[left]{$\mathbb{H}_{M,N} \odot 1$} (B)
(B) edge node[left]{$\mathbb{H}_{M \odot N,P}$} (C)
(A) edge node[above]{$a$} (A')
(C) edge node [above] {$\mathbb{H}(a^\prime)$} (C')
(B') edge node [right] {$\mathbb{H}_{M,N \odot P}$} (C')
(A') edge node [right]{$1 \odot \mathbb{H}_{N,P}$} (B');
\end{tikzpicture}
\]
\[
\begin{tikzpicture}[scale=1.5]
\node (A) at (0,0) {$U_{\mathbb{H}(a)} \odot \mathbb{H}(M)$};
\node (B) at (2.5,0) {$\mathbb{H}(U_a) \odot \mathbb{H}(M)$};
\node (C) at (0,-1) {$\mathbb{H}(M)$};
\node (D) at (2.5,-1) {$\mathbb{H}(U_a \odot M)$};
\node (A') at (5,0) {$\mathbb{H}(M) \odot U_{\mathbb{H}(b)}$};
\node (B') at (7.5,0) {$\mathbb{H}(M) \odot \mathbb{H}(U_b)$};
\node (C') at (5,-1) {$\mathbb{H}(M)$};
\node (D') at (7.5,-1) {$\mathbb{H}(M \odot U_b)$};
\path[->,font=\scriptsize,>=angle 90]
(A) edge node[above]{$\mathbb{H}_U \odot 1$} (B)
(A) edge node[left]{$\lambda$} (C)
(B) edge node[right]{$\mathbb{H}_{U_a,M}$} (D)
(D) edge node[above]{$\mathbb{H}(\lambda^\prime)$} (C)
(A') edge node[above]{$1 \odot \mathbb{H}_U$} (B')
(A') edge node[left]{$\rho$} (C')
(B') edge node[right]{$\mathbb{H}_{M,U_b}$} (D')
(D') edge node[above]{$\mathbb{H}(\rho^\prime)$} (C');
\end{tikzpicture}
\]
The diagrams involving the morphisms of decorations are similar to those in Theorem \ref{DC}. This shows that $\mathbb{H}=(H,E)$ is a double functor. Next we show that this double functor is symmetric monoidal. First, that the object component $\mathbb{H}_0=H$ is symmetric monoidal is clear as $H \colon \mathsf{A} \to \mathsf{A}^\prime$ preserves finite colimits. As for the arrow component $\mathbb{H}_1$, given two horizontal 1-cells $M_1$ and $M_2$ in $F\mathbb{C}\textnormal{sp}$:
\[
\begin{tikzpicture}[scale=1.5]
\node (A) at (0,0) {$a_1$};
\node (B) at (1,0) {$c_1$};
\node (C) at (2,0) {$b_1$};
\node (D) at (1,-0.5) {$d_1 \in F(c_1)$};
\node (E) at (3,0) {$a_2$};
\node (F) at (4,0) {$c_2$};
\node (G) at (5,0) {$b_2$};
\node (H) at (4,-0.5) {$d_2 \in F(c_2)$};
\path[->,font=\scriptsize,>=angle 90]
(A) edge node[above]{$i_1$} (B)
(C) edge node[above]{$o_1$} (B)
(E) edge node[above]{$i_2$} (F)
(G) edge node[above]{$o_2$} (F);
\end{tikzpicture}
\]
their tensor product $M_1 \otimes M_2$ in $F\mathbb{C}\textnormal{sp}$ is given by:
\[
\begin{tikzpicture}[scale=1.5]
\node (A) at (0,0) {$a_1+a_2$};
\node (B) at (1.5,0) {$c_1+c_2$};
\node (C) at (3,0) {$b_1+b_2$};
\node (D) at (1.5,-0.5) {$d_1+d_2 \in F(c_1+c_2)$};
\path[->,font=\scriptsize,>=angle 90]
(A) edge node[above]{$i_1+i_2$} (B)
(C) edge node[above]{$o_1+o_2$} (B);
\end{tikzpicture}
\]
$$d_1+d_2 \colon 1 \xrightarrow{d_1 \times d_2} F(c_1) \times F(c_2) \xrightarrow{\phi_{c_1,c_2}} F(c_1+c_2)$$
and the image of this horizontal 1-cell under the double functor $\mathbb{H}$ is $\mathbb{H}(M_1 \otimes M_2)$ given by:
\[
\begin{tikzpicture}[scale=1.5]
\node (A) at (0,0) {$H(a_1+a_2)$};
\node (B) at (2,0) {$H(c_1+c_2)$};
\node (C) at (4,0) {$H(b_1+b_2)$};
\node (D) at (2,-0.5) {$E(d_1+d_2) \in E(F(c_1+c_2)) = F^\prime(H(c_1+c_2))$};
\path[->,font=\scriptsize,>=angle 90]
(A) edge node[above]{$H(i_1+i_2)$} (B)
(C) edge node[above]{$H(o_1+o_2)$} (B);
\end{tikzpicture}
\]
On the other hand, the image of $M_1$ and $M_2$ is given by $\mathbb{H}(M_1)$ and $\mathbb{H}(M_2)$:
\[
\begin{tikzpicture}[scale=1.5]
\node (A) at (0,0) {$H(a_1)$};
\node (B) at (1,0) {$H(c_1)$};
\node (C) at (2,0) {$H(b_1)$};
\node (D) at (1,-0.5) {$E(d_1) \in F^\prime(H(c_1))$};
\node (E) at (3,0) {$H(a_2)$};
\node (F) at (4,0) {$H(c_2)$};
\node (G) at (5,0) {$H(b_2)$};
\node (H) at (4,-0.5) {$E(d_2) \in F^\prime(H(c_2))$};
\path[->,font=\scriptsize,>=angle 90]
(A) edge node[above]{$H(i_1)$} (B)
(C) edge node[above]{$H(o_1)$} (B)
(E) edge node[above]{$H(i_2)$} (F)
(G) edge node[above]{$H(o_2)$} (F);
\end{tikzpicture}
\]
and their tensor product $\mathbb{H}(M_1) \otimes \mathbb{H}(M_2)$ is given by:
\[
\begin{tikzpicture}[scale=1.5]
\node (A) at (0,0) {$H(a_1)+H(a_2)$};
\node (B) at (2.5,0) {$H(c_1)+H(c_2)$};
\node (C) at (5,0) {$H(b_1)+H(b_2)$};
\node (D) at (2.5,-0.5) {$E(d_1)+E(d_2) \in F^\prime(H(c_1)+H(c_2))$};
\path[->,font=\scriptsize,>=angle 90]
(A) edge node[above]{$H(i_1)+H(i_2)$} (B)
(C) edge node[above]{$H(o_1)+H(o_2)$} (B);
\end{tikzpicture}
\]
$$E(d_1)+E(d_2) \colon 1 \xrightarrow{E(d_1) \times E(d_2)} F^\prime(H(c_1)) \times F^\prime(H(c_2)) \xrightarrow{\Phi_{H(c_1),H(c_2)}} F^\prime (H(c_1)+H(c_2)).$$  We then have a natural 2-isomorphism $\mu_{M_1,M_2} \colon \mathbb{H}(M_1) \otimes \mathbb{H}(M_2) \to \mathbb{H}(M_1 \otimes M_2)$ in $F'\mathbb{C}\textnormal{sp}$ given by:
\[
\begin{tikzpicture}[scale=1.5]
\node (A) at (0,0.5) {$H(a_1)+H(a_2)$};
\node (A') at (0,-0.5) {$H(a_1+a_2)$};
\node (B) at (2.5,0.5) {$H(c_1)+H(c_2)$};
\node (C) at (5,0.5) {$H(b_1)+H(b_2)$};
\node (C') at (5,-0.5) {$H(b_1+b_2)$};
\node (D) at (2.5,-0.5) {$H(c_1+c_2)$};
\node (E) at (2.5,1) {$E(d_1)+E(d_2) \in F^\prime(H(c_1)+H(c_2))$};
\node (F) at (2.5,-1) {$E(d_1+d_2) \in F^\prime(H(c_1+c_2))$};
\path[->,font=\scriptsize,>=angle 90]
(A) edge node[above]{$H(i_1)+H(i_2)$} (B)
(C) edge node[above]{$H(o_1)+H(o_2)$} (B)
(A) edge node[left]{$\kappa$} (A')
(C) edge node[left]{$\kappa$} (C')
(A') edge node [above]{$H(i_1+i_2)$} (D)
(C') edge node [above]{$H(o_1+o_2)$} (D)
(B) edge node [left] {$\kappa$} (D);
\end{tikzpicture}
\]
$$\iota_\kappa \colon F^\prime(\kappa)(E(d_1)+E(d_2)) \to E(d_1+d_2)$$
where $\kappa$ denotes the isomorphism arising from $H$ preserving finite colimits. This natural 2-isomorphism together with the associators of $F\mathbb{C}\textnormal{sp}$ and $F'\mathbb{C}\textnormal{sp}$, respectively $\alpha$ and $\alpha^\prime$, make the following diagram commute:
\[
\begin{tikzpicture}[scale=1.5]
\node (A) at (0,0.5) {$(\mathbb{H}(M_1) \otimes \mathbb{H}(M_2)) \otimes \mathbb{H}(M_3)$};
\node (B) at (0,-0.5) {$\mathbb{H}(M_1 \otimes M_2) \otimes \mathbb{H}(M_3)$};
\node (C) at (0,-1.5) {$\mathbb{H}((M_1 \otimes M_2) \otimes M_3)$};
\node (A') at (3,0.5) {$\mathbb{H}(M_1) \otimes (\mathbb{H}(M_2) \otimes \mathbb{H}(M_3))$};
\node (B') at (3,-0.5) {$\mathbb{H}(M_1) \otimes \mathbb{H}(M_2 \otimes M_3)$};
\node (C') at (3,-1.5) {$\mathbb{H}(M_1 \otimes (M_2 \otimes M_3))$};
\path[->,font=\scriptsize,>=angle 90]
(A) edge node[left]{$\mu_{M_1,M_2} \otimes 1$} (B)
(B) edge node[left]{$\mu_{M_1 \otimes M_2,M_3}$} (C)
(A) edge node[above]{$\alpha^\prime$} (A')
(C) edge node [above] {$\mathbb{H}(\alpha)$} (C')
(B') edge node [right] {$\mu_{M_1,M_2 \otimes M_3}$} (C')
(A') edge node [right]{$1 \otimes \mu_{M_2 \otimes M_3}$} (B');
\end{tikzpicture}
\]
with the corresponding diagram of decorations:
\[
\begin{tikzpicture}[scale=1.5]
\node (A) at (0,0.5) {$F^\prime(\alpha \kappa \kappa)((E(d_1)+E(d_2))+E(d_3))$};
\node (B) at (0,-0.5) {$F^\prime(\alpha \kappa)(E(d_1+d_2)+E(d_3))$};
\node (C) at (0,-1.5) {$F^\prime(\alpha)(E((d_1+d_2)+d_3))$};
\node (A') at (5,0.5) {$F^\prime(\kappa \kappa)(E(d_1)+(E(d_2)+E(d_3)))$};
\node (B') at (5,-0.5) {$F^\prime(\kappa)(E(d_1)+E(d_2+d_3))$};
\node (C') at (5,-1.5) {$E(d_1+(d_2+d_3))$};
\path[->,font=\scriptsize,>=angle 90]
(A) edge node[left]{$F^\prime(\alpha \kappa)(\iota_\kappa + 1)$} (B)
(B) edge node[left]{$F^\prime(\alpha)(\iota_\kappa)$} (C)
(A) edge node[above]{$F^\prime(\kappa \kappa)(\iota_{\alpha^\prime})$} (A')
(C) edge node [above] {$\iota_\alpha$} (C')
(B') edge node [right] {$\iota_\kappa$} (C')
(A') edge node [right]{$F^\prime(\kappa)(1+\iota_\kappa)$} (B');
\end{tikzpicture}
\]
where $$F^\prime(\alpha \kappa \kappa)((E(d_1)+E(d_2))+E(d_3)) = F^\prime(\kappa \kappa \alpha^\prime)((E(d_1)+E(d_2))+E(d_3))$$
as the corresponding hexagon for the symmetric lax monoidal pseudofunctor $E \colon \mathbf{Cat} \to \mathbf{Cat}$ commutes. 

We also have that the monoidal unit of $F\mathbb{C}\textnormal{sp}_1$ is given by:
\[
\begin{tikzpicture}[scale=1.5]
\node (A) at (0,0) {$1_\mathsf{A}$};
\node (B) at (1,0) {$1_\mathsf{A}$};
\node (C) at (2,0) {$1_\mathsf{A}$};
\node (D) at (1,-0.5) {$!_{1_\mathsf{A}} \in F(1_\mathsf{A})$};
\path[->,font=\scriptsize,>=angle 90]
(A) edge node[above]{$1$} (B)
(C) edge node[above]{$1$} (B);
\end{tikzpicture}
\]
where $1_\mathsf{A}$ is the monoidal unit of the finitely cocartesian category $\mathsf{A}$. The image of this horizontal 1-cell under $\mathbb{H}$ is given by:
\[
\begin{tikzpicture}[scale=1.5]
\node (A) at (0,0) {$H(1_\mathsf{A})$};
\node (B) at (1,0) {$H(1_\mathsf{A})$};
\node (C) at (2,0) {$H(1_\mathsf{A})$};
%\node (E) at (3,0) {$=$};
%\node (F) at (4,0) {$1_\mathsf{A^\prime}$};
%\node (G) at (5,0) {$1_\mathsf{A^\prime}$};
%\node (H) at (6,0) {$1_\mathsf{A^\prime}$};
%\node (I) at (5,-0.5) {$!_{1_{\mathsf{A^\prime}}} \in F^\prime(1_{\mathsf{A^\prime}})$};
\node (D) at (1,-0.5) {$E(!_{1_\mathsf{A}}) \in F^\prime(H(1_\mathsf{A}))$};
\path[->,font=\scriptsize,>=angle 90]
(A) edge node[above]{$1$} (B)
(C) edge node[above]{$1$} (B);
%(F) edge node[above]{$1$} (G)
%(H) edge node[above]{$1$} (G);
\end{tikzpicture}
\]
as $H$ preserves finite colimits. We then have a 2-isomorphism in $F'\mathbb{C}\textnormal{sp}$ given by: $$\mu \colon 1_{F'\mathbb{C}\textnormal{sp}_1} \to \mathbb{H}(1_{F\mathbb{C}\textnormal{sp}_1})$$ 
\[
\begin{tikzpicture}[scale=1.5]
\node (A) at (0,0.5) {$1_{\mathsf{A^\prime}}$};
\node (A') at (0,-0.5) {$H(1_\mathsf{A})$};
\node (B) at (1.5,0.5) {$1_{\mathsf{A^\prime}}$};
\node (C) at (3,0.5) {$1_{\mathsf{A^\prime}}$};
\node (C') at (3,-0.5) {$H(1_\mathsf{A})$};
\node (D) at (1.5,-0.5) {$H(1_\mathsf{A})$};
\node (E) at (4.5,0.5) {$!_{1_\mathsf{A^\prime}} \in F^\prime(1_\mathsf{A^\prime})$};
\node (F) at (4.5,-0.5) {$E(!_{1_\mathsf{A}}) \in F^\prime(H(1_\mathsf{A}))$};
\path[->,font=\scriptsize,>=angle 90]
(A) edge node[above]{$1$} (B)
(C) edge node[above]{$1$} (B)
(A) edge node[left]{$\kappa$} (A')
(C) edge node[left]{$\kappa$} (C')
(A') edge node [above]{$1$} (D)
(C') edge node [above]{$1$} (D)
(B) edge node [left] {$\kappa$} (D);
\end{tikzpicture}
\]
together with the morphism $\iota_\mu \colon F^\prime(\kappa)(!_{1_\mathsf{A^\prime}}) \to E(!_{1_\mathsf{A}})$ in $F^\prime(H(1_\mathsf{A}))$. The following square then commutes: 
\[
\begin{tikzpicture}[scale=1.5]
\node (A) at (0,0) {$1_\mathsf{A^\prime} \otimes \mathbb{H}(M)$};
\node (B) at (2.5,0) {$\mathbb{H}(1_\mathsf{A}) \otimes \mathbb{H}(M)$};
\node (C) at (0,-1) {$\mathbb{H}(M)$};
\node (D) at (2.5,-1) {$\mathbb{H}(1_\mathsf{A} + M)$};
\path[->,font=\scriptsize,>=angle 90]
(A) edge node[above]{$\mu \otimes 1$} (B)
(A) edge node[left]{$\ell$} (C)
(B) edge node[right]{$\mu_{1_\mathrm{A},M}$} (D)
(D) edge node[above]{$\mathbb{H}(\ell^\prime)$} (C);
\end{tikzpicture}
\]
where we have abbreviated the monoidal units of $F\mathbb{C}\textnormal{sp}_1$ and $F'\mathbb{C}\textnormal{sp}_1$ as $1_\mathsf{A}$ and $1_\mathsf{A^\prime}$, respectively. The diagram of corresponding decorations is given by:
\[
\begin{tikzpicture}[scale=1.5]
\node (A) at (0,0.5) {$F^\prime(\ell)(!_{1_\mathsf{A^\prime}} + E(d))$};
\node (B) at (0,-0.5) {$E(d)$};
\node (A') at (4,0.5) {$F^\prime(H(\ell^\prime) \kappa)(E(!_{1_\mathsf{A}})+E(d))$};
\node (B') at (4,-0.5) {$F^\prime(H(\ell^\prime))(E(!_{1_\mathsf{A}} + d))$};
\path[->,font=\scriptsize,>=angle 90]
(A) edge node[left]{$\iota_\ell$} (B)
(A) edge node[above]{$F^\prime(H(\ell^\prime)\kappa)(\iota_{\mu+1})$} (A')
(A') edge node [right]{$F^\prime(H(\ell^\prime))(\iota_\kappa)$} (B')
(B') edge node [above] {$\iota_{H(\ell^\prime)}$} (B);
\end{tikzpicture}
\]
where $$F^\prime(\ell)(!_{1_\mathrm{A^\prime}} + E(d))=F^\prime(H(\ell^\prime)\kappa(\mu+1))(!_{1_\mathrm{A^\prime}} + E(d))$$ since the corresponding square involving left unitors for the symmetric lax monoidal pseudofunctor $E \colon \mathbf{Cat} \to \mathbf{Cat}$ commutes. The other square involving the right unitors $r$ and $r^\prime$ is similar. Note that because $\mu$ and $\mu_{(\_ , \_)}$ are both isomorphisms, the symmetric monoidal double functor $\mathbb{H}$ is strong.

\begin{thm}
Given two finitely cocomplete categories $\mathsf{A}$ and $\mathsf{A}'$ and two symmetric lax monoidal pseudofunctors $F \colon \mathsf{A} \to \mathbf{Cat}$ and $F' \colon \mathsf{A}' \to \mathbf{Cat}$, a map $\mathbb{H} \colon F\mathbb{C}\textnormal{sp} \to F'\mathbb{C}\textnormal{sp}$ is given by a pair $(H,E)$ where $H \colon \mathsf{A} \to \mathsf{A}'$ is a finite colimit preserving functor and a symmetric lax monoidal pseudofunctor $E \colon \mathbf{Cat} \to \mathbf{Cat}$ that make the following square commute.
\[
\begin{tikzpicture}[scale=1.5]
\node (A) at (0,0) {$\mathsf{A}$};
\node (B) at (1,0) {$\mathbf{Cat}$};
\node (C) at (0,-1) {$\mathsf{A^\prime}$};
\node (D) at (1,-1) {$\mathbf{Cat}$};
\path[->,font=\scriptsize,>=angle 90]
(A) edge node[above]{$F$} (B)
(A) edge node[left]{$H$} (C)
(B) edge node[right]{$E$} (D)
(C) edge node[above]{$F^\prime$} (D);
\end{tikzpicture}
\]
\end{thm}

%\newpage
%Scratch work for Petri net problem
%\newline
%Let $P = (s_1,t_1 \colon T_1 \to \mathbb{N}(S_1))$ and $Q = (s_2,t_2 \colon T_2 \to \mathbb{N}(S_2))$. We want to show that the functor $F \colon \textrm{Petri} \to \textrm{CMC}$ preserves pushouts of cospans of the form $$P \xleftarrow{} L(Y) \xrightarrow{} Q$$ meaning that $F(P+_{L(Y)} Q) \cong F(P) +_{F(L(Y))} F(Q)$. Here, $L \colon \textrm{Set} \to \textrm{Petri}$ is the left adjoint which sends a set $Y$ to the discrete Petri net with $Y$ as its set of species and no transitions. Thus no transitions are identified in the pushout $P+_{L(Y)} Q$, and so the set of transitions of $P+_{L(Y)} Q$ is given by $$\textrm{Trans}(P)+\textrm{Trans}(Q).$$ Thus the morphisms \emph{contributed from transitions} in both $F(P+_{L(Y)}Q)$ and $F(P)+_{F(L(Y)} F(Q)$ are the same. If we can show that $$\textrm{Ob}(F(P+_{L(Y)}Q) \cong \textrm{Ob}(F(P)+_{F(L(Y))} F(Q)$$ then the remaining morphisms in each commutative monoidal category will be canonically isomorphic as sets.

%Now, what can we say about $\textrm{Ob}(F(P+_{L(Y)}Q))$ vs. $\textrm{Ob}(F(P)+_{F(L(Y))} F(Q))$? Well, $\textrm{Ob}(F(P+_{L(Y)}Q)) = \mathbb{N}(S_1+_Y S_2)$ and $\textrm{Ob}(F(P)+_{F(L(Y))} F(Q))= \mathbb{N}(S_1) +_{\mathbb{N}(Y)} \mathbb{N}(S_2)$.









%\newpage



















%is given by $$1 \to F(c_1) \times F(c_2) \xrightarrow{} F(c_1 + c_2) \xrightarrow{} F(c_1 + _b c_2)$$ and then applying the functor $E$ results in $1 \to E(F(c_1+_b c_2))$, which by the comutative square is the same as $1 \to F^\prime(H(c_1+_b c_2))$. On the other hand, applying the functor $E$ to each decoration results in $$1 \to E(F(c_1)) \times E(F(c_2)) \xrightarrow{} E(F(c_1) \times F(c_2)) \xrightarrow{} E(F(c_1+c_2)) \xrightarrow E(F(c_1+_b c_2)) \xrightarrow{} F^\prime(H(c_1 +_b c_2))$$
\end{comment}

\section{An equivalence of symmetric monoidal double categories} \label{EquivDoubleCats}
In this section we prove that the two different double categorical network frameworks, namely structured cospans and decorated cospans described respectively in \cref{DecCospansDoubleCat,sec:structuredcospans}, are equivalent under certain assumptions. This is established by \cref{Equiv}, which relies on conditions relevant to the existence of left adjoints for opfibrations. We begin by sketching this underlying framework, which is worked out in detail in \cite{MV} and \cite{CV}.
%In this section we investigate the necessary conditions to obtain a category $\mathrm{X}$ and a left adjoint $L \colon \mathrm{A} \to \mathrm{X}$ from a category $\mathrm{A}$ with finite colimits and a symmetric lax monoidal pseudofunctor $F \colon \mathrm{A} \to \mathbf{Cat}$. The main tool that allows us to do this will be the Grothendieck construction which we recall:
First of all, we recall some basics regarding opfibrations and their relation to pseudofunctors into $\bicat{Cat}$ via the Grothendieck construction. Standard material on the theory of fibrations can be found for example in \cite{Herm}, \cite{Borc}.

A functor $U \maps \X \to \A$ is a \textbf{Grothendieck opfibration} if for every $x\in\X$ with $U(x)=a$ and $f \colon a \to b$ in $\A$, there exists a \textbf{cocartesian lifting} of $x$ along $f$, namely a morphism $\beta$ in $\X$ with domain $x$ above $f$ with the following universal property: for any $g\colon b\to b'$ in $\A$ and $\gamma\colon x\to y'$ in $\X$ above the composite $g\circ f$, there exists a unique $\delta\colon y\to y'$ such that $U(\delta)=g$ and $\gamma=\delta\circ\beta$ as shown below
\begin{displaymath}
\xymatrix @R=.1in @C=.6in
{&& y'\ar @{.>}@/_/[dd] &&\\
x\ar[r]_-{\beta} \ar @{.>}@/_/[dd]
\ar[urr]^-{\gamma} & 
y \ar @{.>}@/_/[dd] \ar @{-->}[ur]_-{\exists! \delta}
&& \textrm{in }\X\\
&& b' &&\\
a\ar[r]_-{f=U\beta} \ar[urr]^-{g\circ f=U\gamma}
 & b \ar[ur]_-g && \textrm{in }\A}
\end{displaymath}


%\begin{defn}
%Let $P \colon \ca{X} \to \ca{A}$ be a functor. A morphism $f \colon d_{1} \to d_{2}$ in the category $\mathrm{D}$ is \textbf{cocartesian (with respect to the functor P)} if for any object $d^\prime$ in $\mathrm{D}$ and morphism $g \colon d_{1} \to d^\prime$ and every $p \colon P(d_{2}) \to P(d^\prime)$ such that $P(g)=p P(f)$, there exists a unique $h \colon d_{2} \to d^\prime$ such that $g=hf$ and $p=P(h)$.
%\[
%\begin{tikzpicture}[scale=1.5]
%\node (A) at (0,0) {$d^\prime$};
%\node (B) at (0,-1) {$d_{2}$};
%\node (C) at (1.5,-1) {$d_{1}$};
%\node (H) at (2,-0.25) {$P$};
%\node (D) at (2,-.5) {$\mapsto$};
%\node (E) at (3,0) {$P(d^\prime)$};
%\node (F) at (3,-1) {$P(d_{2})$};
%\node (G) at (4.5,-1) {$P(d_{1})$};
%\path[->,font=\scriptsize,>=angle 90]
%(B) edge[dashed] node[above,left]{$\exists ! h$} (A)
%(C) edge node[above,right]{$g$} (A)
%(F) edge node[above,left]{$p=P(h)$} (E)
%(G) edge node[above]{$P(f)$} (F)
%(G) edge node[above,right]{$P(g)$} (E)
%(C)edge node[above]{$f$}(B);
%\end{tikzpicture}
%\]
%\end{defn}
%\begin{defn}
%A functor $P \colon \mathrm{D} \to \mathrm{A}$ is a \textbf{(Grothendieck) opfibration} if for any object $d$ in $\mathbf{D}$ and morphism $f \colon c %\to P(d)$ there exists a cocartesian morphism $\phi \colon d^\prime \to d$ such that $P(\phi)=f$.
%\end{defn}
%\begin{defn}
%Let $P \colon \mathrm{D} \to \mathrm{A}$ and $P^\prime \colon \mathrm{D^\prime} \to \mathrm{A}$ be opfibrations over a category $\mathrm{A}$. A \define{morphism of opfibrations} is a functor $F \colon \mathrm{D} \to \mathrm{D}^\prime$ such that $F(p)$ is cocartesian with respect to $P^\prime$ if $p$ is cocartesian with respect to $P$ and $P=P^\prime F$.
%\end{defn}
The category $\X$ is called the \textbf{total} category and $\A$ is called the \textbf{base} category of the opfibration. For any $a\in\A$, the \textbf{fibre category} $\X_a$ consists of all objects that map to $a$ and vertical morphisms between them, i.e. mapping to $1_a$.
Assuming the axiom of choice, we may select a cocartesian arrow over each $f\maps a\to b$ in $\A$ and $x\in\X _a$, denoted by $\mathrm{Cocart}(f,x)\maps x\to f_!(x)$, rendering $U$ a so-called \textbf{cloven} opfibration. This choice induces \textbf{reindexing functors} $f_!\colon\X _a\to\X _b$ between the fibre categories, which by the liftings universal property adhere to natural $(1_a)_!\cong1_{\X _a}$ and $(f\circ g)_!\cong f_!\circ g_!$. If these isomorphisms are equalities, we have the notion of a \textbf{split} opfibration.
%Christina: decide later if needed
%Notice that as a result,
%any arrow in the total category of an opfibration factorizes uniquely into a vertical morphism (above the identity) followed by a (co)cartesian one:
%\[
%\xymatrix @C=.4in @R=.2in
%{x \ar @{.>}[dd]\ar[rr]^\gamma \ar[drr]_-{\Cocart(g,C)} && D &\\
%&& f_!C \ar @{-->}[u]_-{\delta} \ar @{.>}[d] & \textrm{in }\C  \\
%X\ar[rr]_-{g} && Y & \textrm{in }\X.}
%\]

Let $\OpFib(\A)$ denote the 2-subcategory of the slice 2-category $\mathbf{Cat}/ \A$ of opfibrations over $\A$, functors that preserve cocartesian liftings and natural transformations with vertical components. In fact, there is a 2-equivalence between opfibrations and pseudofunctors, see Section ?, induced by the so-called \emph{Grothendieck construction}.
%{\chris I don't mind this presentation of the Grothendieck construction which I admit not having seeing before. If we are to keep, we should put a reference. Otherwise we could stay more low level about this and just briefly describe what's going on.}
%\begin{defn}
%Let $\mathbf{Cat}_{\textnormal{lax},\star}$ denote the \define{2-category of lax-pointed categories} which has:
%\begin{enumerate}
%\item{objects as pairs $(\mathrm{A},c)$ where $\mathrm{A}$ is a category and $c$ is an object of $\mathrm{A}$, and}
%\item{a morphism from $(\mathrm{A},c)$ to $(\mathrm{X},y)$ is a pair $(F,f)$ where $F \colon \mathrm{A} \to \mathrm{X}$ is a functor and $f \colon F(c) \to y$ is a morphism in $\mathrm{X}$.}
%\end{enumerate}
%\end{defn}
%\begin{defn}
%Given a pseudofunctor $F \colon \mathrm{A} \to \mathbf{Cat}$, the \define{Grothendieck construction of} $F$ is given by the `strict 2-pullback' of the following cospan:
% \[
%\begin{tikzpicture}[scale=1.5]
%\node (D) at (1,1) {$\inta F$};
%\node (A) at (0,0) {$\mathrm{A}$};
%\node (B) at (1,-1) {$\mathbf{Cat}$};
%\node (C) at (2,0) {$\mathbf{Cat}_{\textnormal{lax},\star}$};
%\path[->,font=\scriptsize,>=angle 90]
%(D) edge[dashed] node[above]{$p$} (A)
%(D) edge[dashed] node[above]{$q$} (C)
%(A) edge node[below]{$F$} (B)
%(C) edge node[below]{$P$} (B);
%\end{tikzpicture}
%\]
%where $P \colon \mathbf{Cat}_{\textnormal{lax},\star} \to \mathbf{Cat}$ is the forgetful functor. 
\begin{defn}\label{def:GrothCat}
For any pseudofunctor $F\colon\A\to\bicat{Cat}$ where $\A$ is a category viewed as a 2-category with trivial 2-cells, the \textbf{Grothendieck category}
$\inta F$ has
\begin{itemize}
\item objects pairs $(a, x \in F(a))$ and
\item a morphism from $(a, x \in F(a))$ to $(b, y\in F(b))$ is a pair $(f \colon a \to b,\delta \colon F(f)(x) \to y)\in\A\times F(b)$.
\end{itemize}
{\chris Might need to write the composition rule? Depending on the final proof of Big Theorem.}
This is an opfibred category over $\A$ via the obvious forgetful functor, with fibre categories $(\inta F)_a=F(a)$ and reindexing functors $f_!=F(f)$.
\end{defn}
%This can also be thought of as a morphism and a 2-morphism:
%\[
%\begin{tikzpicture}[scale=1.5]
%\node (A) at (0,0) {$\star$};
%\node (B) at (1,0.5) {$F(c)$};
%\node (C) at (1,-0.5) {$F(c^\prime)$};
%\node (D) at (2,0.5) {$c$};
%\node (E) at (2,-0.5) {$c^\prime$};
%\node (F) at (0.65,0) {$\Swarrow \alpha$};
%\path[->,font=\scriptsize,>=angle 90]
%(A) edge node[above]{$d$} (B)
%(B) edge node[right]{$F(f)$} (C)
%(A) edge node[below]{$d^\prime$} (C)
%(D) edge node[left] {$f$} (E);
%\end{tikzpicture}
%\]
The above in fact provides the one direction of the following well-known equivalence.
%gives rise to a 2-functor $[\A,\bicat{Cat}]_\pse\to \mathbf{Cat}/ \A$ which factors through the embedding $\OpFib(\A) \hookrightarrow \bicat{Cat} / \A$ as per the following theorem.
%: $$\inta F \colon [ \mathrm{A}, \mathbf{Cat} ] \to \textnormal{Opfib}(\mathrm{A}) \hookrightarrow \mathbf{Cat} / \mathrm{A}.$$

\begin{thm}\label{thm:Grothendieck}\hfill
    \begin{enumerate}
        \item Every opfibration $\U \maps \X \to \A$ gives rise to a pseudofunctor $F_U \maps \A \to \bicat{Cat}$.
        \item Every pseudofunctor $F \maps \A \to \bicat{Cat}$ gives rise to  an opfibration $\U_F \colon \inta F\to\A$.
        \item The above correspondences yield an equivalence of 2-categories 
        \begin{displaymath}
            [\A,\bicat{Cat}]_\pse \simeq \OpFib(\A)
        \end{displaymath}
        so that $F_{\U_F} \cong F$ and $\U_{F_\U} \cong \U$.
%        \item The above 2-equivalence extends to one between 2-categories of arbitrary-base fibrations and arbitrary-domain indexed categories
%        \begin{equation}\label{eq:Gr_equiv}
%        \ICat\simeq\Fib    
%        \end{equation}
    \end{enumerate}
\end{thm}



%\begin{defn}
%Let $P \colon \mathbf{D} \to \mathrm{A}$ be a functor. A morphism $f \colon d_{1} \to d_{2}$ in the category $\mathbf{D}$ is \textbf{cartesian (with respect to the functor P)} if for any object $d^\prime$ in $\mathbf{D}$ and morphism $g \colon d^\prime \to d_{2}$ and every $p \colon P(d^\prime) \to P(d_{1})$ such that $P(g)=P(f)p$, there exists a unique $h \colon d^\prime \to d_{1}$ such that $g=fh$ and $p=P(h)$.
%\[
%\begin{tikzpicture}[scale=1.5]
%\node (A) at (0,0) {$d^\prime$};
%\node (B) at (0,-1.5) {$d_{1}$};
%\node (C) at (1.5,-1.5) {$d_{2}$};
%\node (H) at (2,-0.5) {$P$};
%\node (D) at (2,-.75) {$\mapsto$};
%\node (E) at (3,0) {$P(d^\prime)$};
%\node (F) at (3,-1.5) {$P(d_{1})$};
%\node (G) at (4.5,-1.5) {$P(d_{2})$};
%\path[->,font=\scriptsize,>=angle 90]
%(A) edge[dashed] node[above,left]{$\exists ! h$} (B)
%(A) edge node[above]{$g$} (C)
%(E) edge node[above,left]{$p=P(h)$} (F)
%(F) edge node[above]{$P(f)$} (G)
%(E) edge node[above,right]{$P(g)$} (G)
%(B)edge node[above]{$f$}(C);
%\end{tikzpicture}
%\]
%\end{defn}

In \cite{MV}, the Grothendieck construction is generalized to the monoidal setting, namely lax monoidal structures on a pseudofunctor $F\colon\A\to\bicat{Cat}$ bijectively correspond to monoidal structures on the total category $\inta F$ such that the induced opfibration is a strict monoidal functor and $\otimes_{\inta F}$ preserves liftings, see \cite[Thm. 3.10]{MV}. If the monoidal category $\A$ is in fact cocartesian, there is a further correspondence of the two structures with an ordinary pseudofunctor $F\colon\A\to\bicat{MonCat}$ into the 2-category of monoidal categories, strong monoidal functors and monoidal natural transformations:
\begin{gather}
\textrm{lax monoidal pseudofunctors }F\colon(\A,+,0)\to(\bicat{Cat},\times,1) \notag\\
\Updownarrow \notag\\
\textrm{monoidal opfibrations }U\colon(\X,\otimes,I)\to(\A,+,0) \label{monGroth}\\
\Updownarrow \notag\\
\textrm{pseudofunctors }F\colon\A\to\bicat{MonCat} \notag
\end{gather}
The second equivalence was in fact earlier observed by Shulman in \cite{Shul2}. In more detail, if $(\phi,\phi_0)$ is the lax monoidal structure of $F$, each fibre category $\X_a=F(a)$ obtains a monoidal structure via
\begin{gather}\label{eq:explicitstructure1}
\otimes_a\colon F(a)\times F(a)\xrightarrow{\phi_{a,a}}F(a+a)\xrightarrow{F(\nabla)}F(a)\\
I_x\colon\mathbf{1}\xrightarrow{\phi_0}F(0)\xrightarrow{F(!)}F(a)\nonumber
\end{gather}
Moreover, these correspondences further restrict to the case when the Grothendieck category is specifically cocartesian monoidal itself; for more details, see \cite[Cor. 4.8]{MV}. In that case, the monoidal opfibration clauses for $U\colon(\X,+,0)\to(\A,+,0)$ translate to the functor (strictly) preserving coproducts and the initial object, and bijectively correspond to pseudofunctors $F\colon\A\to\bicat{cocartCat}$ into the 2-category of cocartesian categories, functors that preserve coproducts and initial objects and natural transformations. 

Finally, using the following result we can further restrict to our current case of interest, namely opfibrations that also preserve pushouts, i.e. all finite colimits. Of course, the following statement is more general since it relates the existence of any class of colimits in the total category of an opfibration to their existence in the fibres; for a more detailed exposition and proof, see \cite{Herm}.

\begin{lem}\label{lem:fibrewiselimits}
Suppose $\ca{J}$ is a small category and $\U\colon\X\to\A$ is an opfibration. If the base $\A$ has $\ca{J}$-colimits,
the following are equivalent:
\begin{enumerate}
 \item all fibres have $\ca{J}$-colimits, and the reindexing functors preserve them;
 \item the total $\X$ has $\ca{J}$-colimits, and $\U$ preserves them. %removed strict for now!
\end{enumerate}
\end{lem}

The first part formulates the existence of colimits \emph{locally} in each fibre, which can equivalently be expressed by the corresponding pseudofunctor $F\colon\A\to\bicat{Cat}$ landing on the sub-2-category $\bicat{fcocCat}$ of finitely cocomplete categories and functors that preserve colimits. The second part formulates the existence of colimits \emph{globally} in the total category $\inta F$. Combining these, we can deduce the following.

\begin{cor}\label{cor:fcocMonGroth}
 Suppose $\A$ is a finitely cocomplete category and $F\colon(\A,+,0)\to(\bicat{Cat},\times,1)$ is a lax monoidal pseudofunctor. If its corresponding pseudofunctor $\A\to\bicat{MonCat}$ via \cref{monGroth} 
 factors through $\bicat{fcocCat}$, then the Grothendieck category $\inta F$ has all finite colimits and the induced opfibration $\U_F\colon\inta F\to\A$ preserves them. 
\end{cor}

On a different but related subject, for our purposes we are interested in the existence of a left adjoint $L_F$ to the induced monoidal opfibration $ \inta{F} \to \A$ by the Grothendieck construction as discussed above. The following result provides sufficient conditions for that, which go hand-in-hand with \cref{cor:fcocMonGroth} and \cref{lem:fibrewiselimits}.

\begin{prop}\cite[Prop.\ 4.4]{Gray} \label{prop:opfibtolari}
  Let $\U \colon\X \to \A$ be an opfibration. Then
  $\U$ is a right adjoint `left inverse', a.k.a the unit is the identity, if (and only if) its fibers have
  initial objects which are preserved by the
  reindexing functors.
\end{prop}

\begin{proof}\emph{(Sketch)}
 The left adjoint $L\colon\A\to\X$ maps an object $a$ to the initial object in its fibre $\X_a$, denoted by $\bot_a$. By construction, $U(L(a))=U(\bot_a)=a$. On morphisms $f\colon a\to a'$, $L(f)$ is defined by
 \begin{equation}\label{eq:Lonarrows}
  \bot_a\xrightarrow{\Cocart(f,\bot_a)}f_!(\bot_a)\xrightarrow{\;\stackrel{\chi_a}{\cong}\;}\bot_{a'}
 \end{equation}
where $\chi_a$ is the unique isomorphism between the initial objects in the fibre above $a'$ since $f_!$ preserves them.
\end{proof}

\begin{rmk}\label{rmk:important}
Notice that under \cref{lem:fibrewiselimits}, if $\A$ has an initial object $0_\A$, the above conditions are equivalent to $\X$ having an initial object $0_\X$ above $0_\A$. %Christina: hiding strict preservence under the rug
In this case, $\bot_a$ is precisely the cocartesian lifting of $0_\X$ along the unique map $u_a\colon0_\A\to a$ in the base category:
\begin{displaymath}
\xymatrix @C=.4in @R=.2in
{0_\X \ar @{.>}[d]\ar[rr]^-{\mathrm{Cocart}(0_\X,u_a)} && (u_a)_!(0_X)=:\bot_a\ar @{.>}[d] & \textrm{in }\X  \\
0_\A\ar[rr]^-{\exists!u_a} && a & \textrm{in }\A}
\end{displaymath}
Moreover, if $U=U_F$ for a pseudofunctor $F\colon\A\to\bicat{Cat}$ under \cref{thm:Grothendieck}, the reindexing functors $(u_a)_!$ of the opfibration are precisely $F(u_a)$, therefore $\bot_a=(a,F(u_a)(0_\X))$ in the Grothendieck category, for $u_a$ the unique map from $0_\A$ as above. 

Finally, if that pseudofunctor is lax monoidal $(F,\phi,\phi_0)\colon(\A,+,0)\to(\bicat{Cat},\times,\mathbf{1})$ to begin with, the monoidal Grothendieck construction in the cocartesian case discussed earlier (\cref{eq:explicitstructure1}) expresses $\bot_a$ as the image of the composite
\begin{equation}\label{eq:initialinfibre}
\mathbf{1}\xrightarrow{\phi_0}F(0_\A)\xrightarrow{F(u_a)}F(a) 
\end{equation}
\end{rmk}

%will be a right adjoint to some adjoint , and moreover the unit of the adjunction is the identity; this is called \emph{lari} in [Gray].

{\chris not sure if the below is needed after all}
In the opposite direction, we also have the following result. For a discussion on the unusual strict cocontinuity condition, we refer the reader to \cite{CV}.

\begin{prop}\label{prop:CV}\cite[Prop. \ 3.3]{CV}
Suppose $\U\colon\X\to\A$ is a right adjoint `left inverse'. If $\X$ and $\A$ have chosen pushouts and initial objects and $\U$ strictly preserves them, then $\U$ is an opfibration.
\end{prop}
%Christina: do we need this direction after all? Discuss.
%\begin{thm}\label{thm:mainthmDX}\cite{CV}
%Suppose that $\X$ and $\A$ have chosen pushouts and initial objects, and a functor $\U\colon\X\to\A$ strictly preserves them. Then $\U$ is a right %adjoint left inverse if and only if $\U$ is an opfibration.
%\end{thm}

We have now laid all the necessary background to formally establish an equivalence between the double category of decorated cospans and the double category of structured cospans, constructed from a symmetric lax monoidal pseudofunctor into $\bicat{Cat}$.

\begin{thm}\label{Equiv}
Suppose $\ca{A}$ is a finitely cocomplete category and $(F,\phi,\phi_0)\colon(\ca{A},+,0) \to (\bicat{Cat},\times,\mathbf{1})$ is a symmetric lax monoidal pseudofunctor. %such that as an ordinary pseudofunctor, it factors through $\bicat{fcocCat}$. 
Then the symmetric monoidal double category $F\mathbb{C}\textnormal{sp}$ of decorated cospans (\cref{thm:decoratedcospans}) is equivalent to the symmetric monoidal double category $\scsp{L_F}{\inta F}$ of structured cospans (\cref{SC}), where $L_F \colon \mathsf{A} \to \inta{F}$ is the left adjoint of the induced Grothendieck opfibration $\U_F\colon \inta{F} \to \mathsf{A}$ %by \cref{thm:Grothendieck}, 
under the assumptions of \cref{cor:fcocMonGroth}.
\end{thm}

\begin{proof}
Recall that the double category of decorated cospans $\dcsp{F}$ has objects and vertical 1-morphisms those of $\A$, horizontal 1-cells $a\tickar b$ are cospans $\cspn{a}{c}{b}{}{}$ in $\A$ along with $d\in F(c)$, and 2-morphisms are cospan maps $k\colon c\to c'$ together with a morphism $\iota\colon(Fk)(d)\to d'$, see \cref{eq:FCsp2morph}.

By \cref{cor:fcocMonGroth}, when $F$ as an ordinary pseudofunctor factors through $\bicat{fcocCat}$, the (monoidal) Gro\-the\-ndieck construction gives rise to a finitely cocomplete $\inta F$ such that the corresponding opfibration $U_F\colon(\inta{F},+,0)\to(\A,+,0)$ preserves all finite colimits. %Christina: maybe the cocartesian structures are redundant to write here, since U_F really preserves all finite colimits anyways.
Since in particular it preserves the initial object, \cref{prop:opfibtolari} through \cref{lem:fibrewiselimits} applies 
%which equivalently means that all fibres have initial objects and the reindexing functors preserve them by , %Christina: there is a little bit of circularity here. We knew fibres have and reindexing functors preserve from the factorization already! Change Gray accordingly perhaps?
to construct a left adjoint $L_F\colon\A\to\inta F$ which is also a right inverse, namely $U_FL_F=\mathrm{id}_\A$. Explicitly, this left adjoint is given by $L(a)=(a,\bot_a)$, picking the initial object in the finitely cocomplete fibre also expressed as $F(u_a)\circ\phi_0(*)$ in \cref{eq:initialinfibre}. Diagrammatically,
\begin{displaymath}
 F\colon\X\to\bicat{Cat}\quad\mapsto\quad\begin{tikzcd}[baseline=.3]\inta F\ar[d,"U_F"'] \\ \X \end{tikzcd}\quad\mapsto\quad\begin{tikzcd}\X\ar[r,bend left,pos=.55,"L_F"]\ar[r,phantom,"\bot"description] & \inta F\ar[l,bend left,pos=.45,"U_F"]\end{tikzcd}
\end{displaymath}
roughly describes the processes between the original $F$ and the resulting $L_F$.

Using this functor $L_F$ between finitely cocomplete categories that preserves all colimits that exist (as a left adjoint), we can construct the double category of structured cospans $\scsp{L_F}{\inta F}$. Its objects and vertical morphisms are again those of the category $\A$, whereas horizontal 1-cells $a\tickar b$ as in \cref{eq:structuredcospan} are now cospans of the form $\cspn{L_F(a)}{v}{L_F(b)}{}{}$ in the Grothendieck category $\inta F$. Explicitly, they consist of two pairs of morphisms 
\begin{equation}\label{eq:scsphor1cell}
 (a,\bot_a)\xrightarrow{\scalebox{0.7}{$\begin{cases}i\colon a\to c &\textrm{in }\A \\!\colon F(i)(\bot_a)\to x &\textrm{in }F(c)\end{cases}$}}(c,x)\xleftarrow{\scalebox{0.7}{$\begin{cases}o\colon b\to c &\textrm{in }\A \\!\colon F(o)(\bot_b)\to x &\textrm{in }F(c)\end{cases}$}}(b,\bot_b)
\end{equation}
where $x\in F(c)$, according to \cref{def:GrothCat}. Finally, the 2-morphisms in this double category are as described in \cref{eq:2cellsStrCsp}, in this context fully unravelled below:
\begin{displaymath}
 \begin{tikzcd}[sep=1.5in,ampersand replacement=\&]
 (a,\bot_a)\ar[r,"{\begin{cases}i\colon a\to c &\textrm{in }\A \\!\colon F(i)(\bot_a)\to x &\textrm{in }F(c)\end{cases}}"]\ar[d,"{\begin{cases}f\colon a\to a' &\textrm{in }\A \\\chi_a\colon F(f)(\bot_a)\cong \bot_{a'} &\textrm{in }F(a')\end{cases}}"description] \& (c,x) \ar[d,"{\begin{cases}k\colon c\to c' &\textrm{in }\A \\h\colon F(k)(x)\to x' &\textrm{in }F(c')\end{cases}}"description] \& (b,\bot_b)\ar[l,"{\begin{cases}o\colon b\to c &\textrm{in }\A \\!\colon F(o)(\bot_b)\to x &\textrm{in }F(c)\end{cases}}"']\ar[d,"{\begin{cases}g\colon b\to b' &\textrm{in }\A \\\chi_b\colon F(g)(\bot_b)\cong \bot_{b'} &\textrm{in }F(b')\end{cases}}"description] \\
 (a',\bot_{a'})\ar[r,"{\begin{cases}i'\colon a'\to c' &\textrm{in }\A \\!\colon F(i')(\bot_{a'})\to x' &\textrm{in }F(c')\end{cases}}"'] \& (c',x') \& (b',\bot_{b'})\ar[l,"{\begin{cases}o'\colon b'\to c' &\textrm{in }\A \\!\colon F(o')(\bot_{b'})\to x' &\textrm{in }F(c')\end{cases}}"]
 \end{tikzcd}
\end{displaymath}
where the outside vertical legs come from the definition of $L_F$ on arrows, \cref{eq:Lonarrows}. 
{\chris The above diagram might be a bit overwhelming, but all info is there. Choose which parts are required for a smooth proof!}
The square commutativities translate to $k\circ i=i'\circ f$ and $k\circ o=o'\circ g$ in $\A$, and also by composition in the Grothendieck category, to
\begin{gather*}
 F(k\circ i)(\bot_a)\cong Fk(Fi(\bot_a))\xrightarrow{Fk(!)}Fk(x)\xrightarrow{h}x'= \\
 F(i'\circ f)(\bot_a)\cong Fi'(Ff(\bot_a))\xrightarrow{Fi'(\chi_a)}Fi'(\bot_{a'})\xrightarrow{!}x'
\end{gather*}
in the category $F(c')$. However, since all maps are unique in the above equality {\chris discuss with John and Kenny}, this gives no extra conditions for the morphisms involved, and similarly for the equality including $o, o'$.

In order to prove that there is a double equivalence (\cref{ShulDubEquiv}) between $\dcsp{F}$ and $\scsp{L_F}{\inta F}$, we define a %n identity-on-objects and vertical morphism
double functor
\begin{displaymath}
\mathbb{E}\colon\scsp{L_F}{\inta F} \longrightarrow\dcsp{F}
\end{displaymath}
as follows. The object component of $\mathbb{E}=(\mathbb{E}_0,\mathbb{E}_1)$ is $\mathbb{E}_0 = \id_{\mathsf{A}}$ since both double categories have $\mathsf{A}$ as their vertical category; trivially, $\mathbb{E}_0$ is an equivalence of categories. Given a horizontal 1-cell in $\scsp{L_F}{\inta F}$ like \cref{eq:scsphor1cell}, the arrow component $\mathbb{E}_1$ of the double functor simply maps it to the decorated cospan
\begin{displaymath}
 a\xrightarrow{\; i \;}c\xleftarrow{\; o \;}b\;\textrm{ with decoration }x\in F(c)
\end{displaymath}
Notice that this really is a bijective correspondence since the unique maps from the initial objects in the fibres provide no extra information. Finally, given a 2-morphism of $L_F$-structured cospans as described above, $\mathbb{E}_1$ maps it to the underlying cospan 2-morphism
\begin{displaymath}
 \begin{tikzcd}
a\ar[r,"i"]\ar[d,"f"'] & c\ar[d,"k"] & b\ar[l,"o"']\ar[d,"g"] \\
a'\ar[r,"i'"']& c' & b'\ar[l,"o'"]
 \end{tikzcd}
\end{displaymath}
 in $\A$ along with the map $h\colon Fk(x)\to x'$ in $F(c')$, namely a decorated cospan 2-morphism as per \cref{eq:FCsp2morph}.



{\chris Now the checking of the equivalence should follow! perhaps in less detail than currently?}
which is a cospan in $\mathsf{X}$ of the form:
\[
\begin{tikzpicture}[scale=1.5]
\node (A) at (0,0) {$L(c)$};
\node (B) at (1,0) {$x$};
\node (C) at (2,0) {$L(c^\prime)$};
\path[->,font=\scriptsize,>=angle 90]
(A) edge node[above]{$i$} (B)
(C) edge node[above]{$o$} (B);
\end{tikzpicture}
\]
the image $\mathbb{E}_1(M)$ is given by the pair:
\[
\begin{tikzpicture}[scale=1.5]
\node (A) at (0,0) {$c$};
\node (B) at (1,0) {$R(x)$};
\node (C) at (2,0) {$c^\prime$};
\node (D) at (3,0) {$x \in F(R(x))$};
\path[->,font=\scriptsize,>=angle 90]
(A) edge node[above]{$R(i) {\eta_c}$} (B)
(C) edge node[above]{$R(o) {\eta_{c^\prime}}$} (B);
\end{tikzpicture}
\]
where $R\colon \mathsf{X} \to \mathsf{A}$ is the right adjoint to the functor $L \colon \mathsf{A} \to \mathsf{X}$ and $\eta \colon 1_{\mathsf{A}} \to RL$ is the unit of the adjunction $L \dashv R$ which is an isomorphism since $L$ is fully faithful. \textbf{Hopefully...actually, it's a `lari'!} Similarly, the image of a 2-morphism $\alpha \colon M \to N$ in $_L \mathbb{C}\textnormal{sp}(\mathsf{X})$:
\[
\begin{tikzpicture}[scale=1.5]
\node (A) at (0,0) {$L(c_1)$};
\node (B) at (1,0) {$x$};
\node (C) at (2,0) {$L(c_2)$};
\node (A') at (0,-1) {$L(c_1^\prime)$};
\node (B') at (1,-1) {$x^\prime$};
\node (C') at (2,-1) {$L(c_2^\prime)$};
\path[->,font=\scriptsize,>=angle 90]
(A) edge node[above]{$i$} (B)
(C) edge node[above]{$o$} (B)
(A') edge node[above]{$i^\prime$} (B')
(C') edge node[above]{$o^\prime$} (B')
(A) edge node [left]{$L(f)$} (A')
(B) edge node [left]{$\alpha$} (B')
(C) edge node [left]{$L(g)$} (C');
\end{tikzpicture}
\]
is given by the 2-morphism $\mathbb{E}_1(\alpha) \colon \mathbb{E}_1(M) \to \mathbb{E}_1(N)$ in $\mathbb{F}\textnormal{Cospan}(\mathsf{A})$ given by:
\[
\begin{tikzpicture}[scale=1.5]
\node (A) at (0,0) {$c_1$};
\node (B) at (1,0) {$R(x)$};
\node (C) at (2,0) {$c_2$};
\node (A') at (0,-1) {$c_1^\prime$};
\node (B') at (1,-1) {$R(x^\prime)$};
\node (C') at (2,-1) {$c_2^\prime$};
\node (D) at (3,0) {$x \in F(R(x))$};
\node (D') at (3,-1) {$x^\prime \in F(R(x^\prime))$};
\path[->,font=\scriptsize,>=angle 90]
(A) edge node[above]{$R(i) \eta_{c_1}$} (B)
(C) edge node[above]{$R(o) \eta_{c_2}$} (B)
(A') edge node[above]{$R(i^\prime) \eta_{c_1^\prime}$} (B')
(C') edge node[above]{$R(o^\prime) \eta_{c_2^\prime}$} (B')
(A) edge node [left]{$f$} (A')
(B) edge node [left]{$R(\alpha)$} (B')
(C) edge node [left]{$g$} (C');
\end{tikzpicture}
\]
together with a morphism $\iota \colon F(R(\alpha))(x) \to x^\prime$ in $F(R(x^\prime)) \subseteq \mathrm{X}$ which comes from the Grothendieck construction of the pseudofunctor $F \colon \mathsf{A} \to \mathbf{Cat}$. First, to see that this functor is essentially surjective, given a horizontal 1-cell in $F\mathbb{C}\textnormal{sp}$:
\[
\begin{tikzpicture}[scale=1.5]
\node (A) at (0,0) {$c_1$};
\node (B) at (1,0) {$c$};
\node (C) at (2,0) {$c_2$};
\node (D) at (3,0) {$x \in F(c)$};
\path[->,font=\scriptsize,>=angle 90]
(A) edge node[above]{$i$} (B)
(C) edge node[above]{$o$} (B);
\end{tikzpicture}
\]
we can find a 2-isomorphism in $F\mathbb{C}\textnormal{sp}$ whose codomain is the above horizontal 1-cell and whose domain is the image of the following horizontal 1-cell in $_L \mathbb{C}\textnormal{sp}(\mathsf{X})$:
\[
\begin{tikzpicture}[scale=1.5]
\node (A) at (0,0) {$L(c_1)$};
\node (B) at (1,0) {$x$};
\node (C) at (2,0) {$L(c_2)$};
\path[->,font=\scriptsize,>=angle 90]
(A) edge node[above]{$i^\prime$} (B)
(C) edge node[above]{$o^\prime$} (B);
\end{tikzpicture}
\]
with the 2-isomorphism in $F\mathbb{C}\textnormal{sp}$ given by:
\[
\begin{tikzpicture}[scale=1.5]
\node (A) at (0,0) {$c_1$};
\node (B) at (1,0) {$R(x)$};
\node (C) at (2,0) {$c_2$};
\node (A') at (0,-1) {$c_1$};
\node (B') at (1,-1) {$c$};
\node (C') at (2,-1) {$c_2$};
\node (D) at (3,0) {$x \in F(R(x))$};
\node (D') at (3,-1) {$x \in F(c)$};
\path[->,font=\scriptsize,>=angle 90]
(A) edge node[above]{$R(i^\prime) \eta_{c_1}$} (B)
(C) edge node[above]{$R(o^\prime) \eta_{c_2}$} (B)
(A') edge node[above]{$i$} (B')
(C') edge node[above]{$o$} (B')
(A) edge node [left]{$1$} (A')
(B) edge node [left]{${(R(e) \eta_c)}^{-1}$} (B')
(C) edge node [left]{$1$} (C');
\end{tikzpicture}
\]
$$\iota \colon F({(R(e)\eta_c)}^{-1})(x) \to x$$
where $e \colon L(c) \to x$ is given by the map from the trivial decoration on $c$ to $x \in F(c)$. The object and arrow components $\mathbb{E}_0$ and $\mathbb{E}_1$ satisfy the equations $S \mathbb{E}_1 = \mathbb{E}_0 S$ and $T \mathbb{E}_1 = \mathbb{E}_0 T$.

To show that the double functor $\mathbb{E}$ is fully faithful, we need to show that the map  $$\mathbb{E}_1 \colon _f { _L \mathbb{C}\textnormal{sp}(\mathsf{X})}_g(M,N) \to _{\mathbb{E}(f)} {F\mathbb{C}\textnormal{sp}}_{\mathbb{E}(g)}(\mathbb{E}(M),\mathbb{E}(N))$$ is bijective for arbitrary vertical 1-morphisms $f$ and $g$ and horizontal 1-cells $M$ and $N$ of $_L \mathbb{C}\textnormal{sp}(\mathsf{X})$. Consider a 2-morphism in $_L \mathbb{C}\textnormal{sp}(\mathsf{X})$:
\[
\begin{tikzpicture}[scale=1.5]
\node (A) at (0,0) {$L(c_1)$};
\node (B) at (1,0) {$x$};
\node (C) at (2,0) {$L(c_2)$};
\node (A') at (0,-1) {$L(c_1^\prime)$};
\node (B') at (1,-1) {$x^\prime$};
\node (C') at (2,-1) {$L(c_2^\prime)$};
\node (D) at (1,0.5) {$M$};
\node (E) at (-1,-0.5) {$f$};
\node (F) at (1,-1.5) {$N$};
\node (G) at (3,-0.5) {$g$};
\path[->,font=\scriptsize,>=angle 90]
(A) edge node[above]{$i$} (B)
(C) edge node[above]{$o$} (B)
(A') edge node[above]{$i^\prime$} (B')
(C') edge node[above]{$o^\prime$} (B')
(A) edge node [left]{$L(f)$} (A')
(B) edge node [left]{$\alpha$} (B')
(C) edge node [left]{$L(g)$} (C');
\end{tikzpicture}
\]
Thus the set $$_f { _L \mathbb{C}\textnormal{sp}(\mathsf{X})}_g(M,N)$$ consists of triples $$(f,\alpha,g)$$ rendering the above diagram commutative where $f$ and $g$ are morphisms of $\mathsf{A}$ and $\alpha$ is a morphism of $\mathsf{X}$. The image of the above 2-morphism under the double functor $\mathbb{E}$ is given by:
\[
\begin{tikzpicture}[scale=1.5]
\node (A) at (0,0) {$c_1$};
\node (B) at (1,0) {$R(x)$};
\node (C) at (2,0) {$c_2$};
\node (A') at (0,-1) {$c_1^\prime$};
\node (B') at (1,-1) {$R(x^\prime)$};
\node (C') at (2,-1) {$c_2^\prime$};
\node (D) at (1,0.5) {$x \in F(R(x))$};
\node (D') at (1,-1.5) {$x^\prime \in F(R(x^\prime))$};
\node (E) at (1,1) {$\mathbb{E}(M)$};
\node (F) at (-1,-0.5) {$\mathbb{E}(f)$};
\node (G) at (1,-2) {$\mathbb{E}(N)$};
\node (H) at (3,-0.5) {$\mathbb{E}(g)$};
\path[->,font=\scriptsize,>=angle 90]
(A) edge node[above]{$R(i)\eta_{c_1}$} (B)
(C) edge node[above]{$R(o)\eta_{c_2}$} (B)
(A') edge node[above]{$R(i^\prime)\eta_{c_1^\prime}$} (B')
(C') edge node[above]{$R(o^\prime)\eta_{c_2^\prime}$} (B')
(A) edge node [left]{$f$} (A')
(B) edge node [left]{$R(\alpha)$} (B')
(C) edge node [left]{$g$} (C');
\end{tikzpicture}
\]
together with a morphism $\iota \colon F(R(\alpha))(x) \to x^\prime$ of $F(R(x^\prime))$.
Thus the set $$_{\mathbb{E}(f)} {F\mathbb{C}\textnormal{sp}}_{\mathbb{E}(g)}(\mathbb{E}(M),\mathbb{E}(N))$$ consists of 4-tuples $$(f,R(\alpha),g,\iota)$$ rendering the above diagram commutative and where $f,g$ and $R(\alpha)$ are morphisms of $\mathsf{A}$ and $\iota$ is a morphism in $F(R(x'))$. The morphisms $R(\alpha) \colon R(x) \to R(x^\prime)$ and $\iota \colon F(R(\alpha))(x) \to x^\prime$ together determine the morphism $\alpha \colon x \to x^\prime$ in $\mathrm{X}$ and conversely; given two objects $x=(c,x \in F(c))$ and $x^\prime=(c^\prime,x^\prime \in F(c^\prime))$ of $\mathsf{X}=\inta{F}$, a morphism from $\alpha \colon x \to x^\prime$ is a pair $$(h \colon c \to c^\prime, \iota \colon F(h)(x) \to x^\prime)$$ where $h \colon c \to c^\prime$ is given by $R(\alpha) \colon R(x) \to R(x^\prime)$. This shows that $\mathbb{E}$ is fully faithful. \textbf{At least in my favorite example...}

Next we show that the double functor $\mathbb{E}$ is strong by exhibiting a natural isomorphism $$\mathbb{E}_{M,N} \colon \mathbb{E}(M) \odot \mathbb{E}(N) \xrightarrow{\sim} \mathbb{E}(M \odot N)$$ for every pair of composable horizontal 1-cells $M$ and $N$ of $_L \mathbb{C}\textnormal{sp}(\mathsf{X})$ and for each object $c \in { _L \mathbb{C}\textnormal{sp}(\mathsf{X})}$ a natural isomorphism $$\mathbb{E}_c \colon \hat{U}_{\mathbb{E}(c)} \xrightarrow{\sim} \mathbb{E}(U_c)$$ where $U$ and $\hat{U}$ are the unit structure functors of $_L \mathbb{C} \textnormal{sp}(\mathsf{X})$ and $F\mathbb{C}\textnormal{sp}$, respectively. For any object $c$, the horizontal 1-cell $\hat{U}_{\mathbb{E}(c)}$ is given by $\hat{U}_c$ which is given by the pair:
\[
\begin{tikzpicture}[scale=1.5]
\node (A) at (0,0) {$c$};
\node (B) at (1,0) {$c$};
\node (C) at (2,0) {$c$};
\node (D) at (3,0) {$!_c \in F(c)$};
\path[->,font=\scriptsize,>=angle 90]
(A) edge node[above]{$1$} (B)
(C) edge node[above]{$1$} (B);
\end{tikzpicture}
\]
The horizontal 1-cell $U_c$ is given by
\[
\begin{tikzpicture}[scale=1.5]
\node (A) at (0,0) {$L(c)$};
\node (B) at (1,0) {$L(c)$};
\node (C) at (2,0) {$L(c)$};
%\node (D) at (3,0.5) {$!_c \in F(c)$};
\path[->,font=\scriptsize,>=angle 90]
(A) edge node[above]{$1$} (B)
(C) edge node[above]{$1$} (B);
\end{tikzpicture}
\]
and so $\mathbb{E}(U_c)$ is given by the pair:
\[
\begin{tikzpicture}[scale=1.5]
\node (A) at (0,0) {$c$};
\node (B) at (1,0) {$R(L(c))$};
\node (C) at (2,0) {$c$};
\node (D) at (3.25,0) {$!_c \in F(R(L(c)))$};
\path[->,font=\scriptsize,>=angle 90]
(A) edge node[above]{$\eta_c$} (B)
(C) edge node[above]{$\eta_c$} (B);
\end{tikzpicture}
\]
Then we can obtain the natural isomorphism $\mathbb{E}_c \colon \hat{U}_{\mathbb{E}(c)} \xrightarrow{\sim} \mathbb{E}(U_c)$ as the 2-morphism
\[
\begin{tikzpicture}[scale=1.5]
\node (A) at (0,0) {$c$};
\node (B) at (1,0) {$c$};
\node (C) at (2,0) {$c$};
\node (A') at (0,-1) {$c$};
\node (B') at (1,-1) {$R(L(c))$};
\node (C') at (2,-1) {$c$};
\node (D) at (3,0) {$!_c \in F(c)$};
\node (D') at (3.25,-1) {$!_{R(L(c))} \in F(R(L(c)))$};
\path[->,font=\scriptsize,>=angle 90]
(A) edge node[above]{$1$} (B)
(C) edge node[above]{$1$} (B)
(A') edge node[above]{$\eta_c$} (B')
(C') edge node[above]{$\eta_c$} (B')
(A) edge node [left]{$1$} (A')
(B) edge node [left]{$\eta_c$} (B')
(C) edge node [left]{$1$} (C');
\end{tikzpicture}
\]
$$\iota \colon F(\eta_c)(!_c) \xrightarrow{!} !_{R(L(c))}$$
of $F\mathbb{C}\textnormal{sp}$.

Next, given composable horizontal 1-cells $M$ and $N$ in $_L \mathbb{C}\textnormal{sp}(\mathsf{X})$:
\[
\begin{tikzpicture}[scale=1.5]
\node (A) at (0,0) {$L(c_1)$};
\node (B) at (1,0) {$x$};
\node (C) at (2,0) {$L(c_2)$};
\node (D) at (3,0) {$L(c_2)$};
\node (E) at (4,0) {$x^\prime$};
\node (F) at (5,0) {$L(c_3)$};
%\node (D) at (3,0.5) {$!_c \in F(c)$};
\path[->,font=\scriptsize,>=angle 90]
(A) edge node[above]{$i$} (B)
(C) edge node[above]{$o$} (B)
(D) edge node[above]{$i^\prime$} (E)
(F) edge node[above]{$o^\prime$} (E);
\end{tikzpicture}
\]
their images $\mathbb{E}(M)$ and $\mathbb{E}(N)$ are given by:
\[
\begin{tikzpicture}[scale=1.5]
\node (A) at (0,0) {$c_1$};
\node (B) at (1,0) {$R(x)$};
\node (C) at (2,0) {$c_2$};
\node (D) at (3,0) {$c_2$};
\node (E) at (4,0) {$R(x^\prime)$};
\node (F) at (5,0) {$c_3$};
\node (G) at (1,-0.5) {$x \in F(R(x))$};
\node (H) at (4,-0.5) {$x^\prime \in F(R(x^\prime))$};
%\node (D) at (3,0.5) {$!_c \in F(c)$};
\path[->,font=\scriptsize,>=angle 90]
(A) edge node[above]{$R(i) \eta_{c_1}$} (B)
(C) edge node[above]{$R(o) \eta_{c_2}$} (B)
(D) edge node[above]{$R(i^\prime) \eta_{c_2}$} (E)
(F) edge node[above]{$R(o^\prime) \eta_{c_3}$} (E);
\end{tikzpicture}
\]
and so $\mathbb{E}(M) \odot \mathbb{E}(N)$ is given by:
\[
\begin{tikzpicture}[scale=1.5]
\node (A) at (0,0) {$c_1$};
\node (B) at (1.5,0) {$R(x)+_{c_2}R(x^\prime)$};
\node (C) at (3,0) {$c_3$};
\node (G) at (1.5,-0.5) {$\hat{x} \in F(R(x)+_{c_2}R(x^\prime))$};
%\node (D) at (3,0.5) {$!_c \in F(c)$};
\path[->,font=\scriptsize,>=angle 90]
(A) edge node[above]{$j \psi R(i) \eta_{c_1}$} (B)
(C) edge node[above]{$j \psi R(o^\prime) \eta_{c_3}$} (B);
\end{tikzpicture}
\]
$$\hat{x} \colon 1 \xrightarrow{\lambda^{-1}} 1 \times 1 \xrightarrow{x \times x^\prime} F(R(x)) \times F(R(x^\prime)) \xrightarrow{\phi_{R(x),R(x^\prime)}} F(R(x)+R(x^\prime)) \xrightarrow{F(j_{R(x),R(x^\prime)})} F(R(x)+_{c_2}R(x^\prime))$$where $\psi$ denotes each natural map into the coproduct and $j$ denotes the natural map from the coproduct to the pushout. On the other hand, $M \odot N$ is given by
\[
\begin{tikzpicture}[scale=1.5]
\node (A) at (0,0) {$L(c_1)$};
\node (B) at (1.25,0) {$x+_{L(c_2)}x^\prime$};
\node (C) at (2.5,0) {$L(c_3)$};
%\node (G) at (1,-0.5) {$\hat{d} \in F(R(d)+_{c_2}R(d^\prime))$};
%\node (D) at (3,0.5) {$!_c \in F(c)$};
\path[->,font=\scriptsize,>=angle 90]
(A) edge node[above]{$J \zeta i$} (B)
(C) edge node[above]{$J \zeta o^\prime$} (B);
\end{tikzpicture}
\]
where $\zeta$ is a natural map into a coproduct and $J$ is the natural map from the coproduct to the pushout. Then $E(M \odot N)$ is given by
\[
\begin{tikzpicture}[scale=1.5]
\node (A) at (0,0) {$c_1$};
\node (B) at (1.5,0) {$R(x+_{L(c_2)}x^\prime)$};
\node (C) at (3,0) {$c_3$};
\node (G) at (1.5,-0.5) {$x+_{L(c_2)} x^\prime \in F(R(x+_{L(c_2)}x^\prime))$};
%\node (D) at (3,0.5) {$!_c \in F(c)$};
\path[->,font=\scriptsize,>=angle 90]
(A) edge node[above]{$R(J \zeta i) \eta_{c_1}$} (B)
(C) edge node[above]{$R(J \zeta o^\prime) \eta_{c_3}$} (B);
\end{tikzpicture}
\]
and so $\mathbb{E}_{M,N} \colon \mathbb{E}(M) \odot \mathbb{E}(N) \xrightarrow{\sim} \mathbb{E}(M \odot N)$ is given by the 2-morphism:
\[
\begin{tikzpicture}[scale=1.5]
\node (A) at (0,0) {$c_1$};
\node (B) at (1.5,0) {$R(x)+_{c_2}R(x^\prime)$};
\node (C) at (3,0) {$c_3$};
\node (A') at (0,-1) {$c_1$};
\node (B') at (1.5,-1) {$R(x+_{L(c_2)}x^\prime)$};
\node (C') at (3,-1) {$c_3$};
\node (D) at (5.5 ,0) {$\hat{x} \in F(R(x)+_{c_2}R(x^\prime))$};
\node (D') at (5.5,-1) {$x+_{L(c_2)}x^\prime \in F(R(x+_{L(c_2)}x^\prime))$};
\path[->,font=\scriptsize,>=angle 90]
(A) edge node[above]{$j \psi R(i) \eta_{c_1}$} (B)
(C) edge node[above]{$j \psi R(o^\prime) \eta_{c_3}$} (B)
(A') edge node[above]{$R(J \zeta i) \eta_{c_1}$} (B')
(C') edge node[above]{$R(J \zeta o^\prime) \eta_{c_3}$} (B')
(A) edge node [left]{$1$} (A')
(B) edge node [left]{$\sigma$} (B')
(C) edge node [left]{$1$} (C');
\end{tikzpicture}
\]
First, if the right adjoint $R \colon \mathsf{X} \to \mathsf{A}$ is also a left adjoint, then $R$ also preserves all colimits and we have an isomorphism $$\kappa \colon R(x) +_{R(L(c_2))} R(x^\prime) \to R(x+_{L(c_2)}x^\prime).$$ Also, since the left adjoint $L \colon \mathsf{A} \to \mathsf{X}$ is fully faithful, the unit of the adjunction $L \dashv R$ at the object $c_2$ gives an isomorphism $\eta_{c_2} \colon c_2 \to R(L(c_2))$ which results in an isomorphism $$j_{\eta_{c_2}} \colon R(x) +_{c_2} R(x^\prime) \to R(x) +_{R(L(c_2))} R(x^\prime).$$ Composing these two results in an isomorphism $$\sigma \coloneqq \kappa j_{\eta_{c_2}} \colon R(x) +_{c_2} R(x^\prime) \to R(x+_{L(c_2)}x^\prime).$$
Next, to see that the above diagram commutes, it suffices to show that for the object $c_1 \in \mathrm{A}$, $$R(J)R(\zeta)R(i)\eta_{c_1}(c_1) = R(J \zeta i)\eta_{c_1}(c_1)  \stackrel{!}{=} \sigma j \psi R(i)\eta_{c_1}(c_1) = \kappa j_{\eta_{c_2}} \psi R(i) \eta_{c_1}(c_1).$$ This follows as $R(i) \eta_{c_1} \colon c_1 \to R(x)$ and the following diagram commutes:
\[
\begin{tikzpicture}[scale=1.5]
\node (B) at (0,0) {$R(x)$};
\node (C) at (2,0) {$R(x)+R(x^\prime)$};
\node (A') at (4,0) {$R(x)+_{c_2}R(x^\prime)$};
\node (B') at (4,-2) {$R(x+_{L(c_2)}x^\prime)$};
\node (D) at (0,-2) {$R(x+x^\prime)$};
\node (D') at (4,-1) {$R(x)+_{R(L(c_2))} R(x^\prime)$};
\path[->,font=\scriptsize,>=angle 90]
(C) edge node[above]{$j$} (A')
(B) edge node[above]{$\psi$} (C)
(D) edge node[above]{$R(J)$} (B')
(B) edge node [left]{$R(\zeta)$} (D)
(A') edge node [right]{$j_{\eta_{c_2}}$} (D')
(A') edge [out=345,in=15] node [right]{$\sigma$} (B')
(D') edge node [right]{$\kappa$} (B');
\end{tikzpicture}
\]
Lastly, this map of cospans comes with an isomorphism $\iota \colon F(\sigma)(\hat{x}) \to (x+_{L(c_2)}x^\prime)$ in $F(R(x+_{L(c_2)}x^\prime))$. This shows that $\mathbb{E}$ is strong, and so $\mathbb{E} \colon _L \mathbb{C}\textnormal{sp}(\mathsf{X}) \xrightarrow{\sim} F\mathbb{C}\textnormal{sp}$ is part of a double equivalence by Theorem \ref{ShulDubEquiv}.

Next, if both double categories $_L \mathbb{C}\textnormal{sp}(\mathsf{X})$ and $F\mathbb{C}\textnormal{sp}$ are symmetric monoidal, as they are if both $\mathsf{A}$ and $\mathsf{X}$ have finite colimits, then this equivalence of double categories $\mathbb{E} \colon _L \mathbb{C}\textnormal{sp}(\mathsf{X}) \to F\mathbb{C}\textnormal{sp}$ will be symmetric monoidal. First, note that we have an isomorphism $\epsilon \colon 1_{F\mathbb{C}\textnormal{sp}} \to \mathbb{E}(1_{_L \mathbb{C}\textnormal{sp}(\mathsf{X})})$ and natural isomorphisms $\mu_{c_1,c_2} \colon \mathbb{E}(c_1) \otimes \mathbb{E}(c_2) \to \mathbb{E}(c_1 \otimes c_2)$ for every pair of objects $c_1,c_2 \in {_L \mathbb{C} \textnormal{sp}(\mathsf{X})}$ both of which are given by identities since both double categories $_L \mathbb{C}\textnormal{sp}(\mathsf{X})$ and $F\mathbb{C}\textnormal{sp}$ have $\mathsf{A}$ as their category of objects and $\mathbb{E}_0=\id_{\mathsf{A}}$. The diagrams utilizing these maps that are required to commute do so trivially.

For the arrow component $\mathbb{E}_1$, we have an isomorphism $\delta \colon U_{1_{F\mathbb{C}\textnormal{sp}}} \to \mathbb{E}(U_{1_{_L \mathbb{C}\textnormal{sp}(\mathsf{X})}})$ where the horizontal 1-cell $U_{1_{F\mathbb{C}\textnormal{sp}}}$ is given by:
\[
\begin{tikzpicture}[scale=1.5]
\node (A) at (0,0) {$1_\mathsf{A}$};
\node (B) at (1,0) {$1_\mathsf{A}$};
\node (C) at (2,0) {$1_\mathsf{A}$};
\node (D) at (3,0) {$!_{1_{\mathsf{A}}} \in F(1_\mathsf{A})$};
\path[->,font=\scriptsize,>=angle 90]
(A) edge node[above]{$1$} (B)
(C) edge node[above]{$1$} (B);
\end{tikzpicture}
\]
where $!_{1_{\mathsf{A}}} = \phi \colon 1 \to F(1_\mathsf{A})$ is the trivial decoration which comes from the structure of the symmetric lax monoidal pseudofunctor $F \colon \mathsf{A} \to \mathbf{Cat}$. The horizontal 1-cell $U_{1_{_L \mathbb{C}\textnormal{sp}(\mathsf{X})}}$ is given by:
\[
\begin{tikzpicture}[scale=1.5]
\node (A) at (0,0) {$L(1_\mathsf{A})$};
\node (B) at (1,0) {$L(1_\mathsf{A})$};
\node (C) at (2,0) {$L(1_\mathsf{A})$};
%\node (D) at (3,0.5) {$I \in F(1_\mathrm{A})$};
\path[->,font=\scriptsize,>=angle 90]
(A) edge node[above]{$1$} (B)
(C) edge node[above]{$1$} (B);
\end{tikzpicture}
\]
where here we make use of the fact that the left adjoint $L \colon (\mathsf{A},+,1_\mathsf{A}) \to (\mathsf{X},+,1_\mathsf{X})$ preserves all colimits and thus $L(1_\mathsf{A}) \cong 1_\mathsf{X}$. The horizontal 1-cell $\mathbb{E}(U_{1_{_L \mathbb{C} \textnormal{sp}(\mathsf{X})}})$ is then given by the pair:
\[
\begin{tikzpicture}[scale=1.5]
\node (A) at (0,0) {$1_\mathsf{A}$};
\node (B) at (1.25,0) {$R(L(1_\mathsf{A}))$};
\node (C) at (2.5,0) {$1_\mathsf{A}$};
\node (D) at (4.5,0) {$!_{R(L(1_\mathsf{A}))} \in F(R(L(1_\mathsf{A})))$};
\path[->,font=\scriptsize,>=angle 90]
(A) edge node[above]{${\eta_{1_\mathsf{A}}}$} (B)
(C) edge node[above]{${\eta_{1_\mathsf{A}}}$} (B);
\end{tikzpicture}
\]
The isomorphism $\delta$ is then given by the 2-morphism:
\[
\begin{tikzpicture}[scale=1.5]
\node (A) at (0,0) {$1_\mathsf{A}$};
\node (B) at (1,0) {$1_{\mathsf{A}}$};
\node (C) at (2,0) {$1_\mathsf{A}$};
\node (A') at (0,-1) {$1_\mathsf{A}$};
\node (B') at (1,-1) {$R(L(1_\mathsf{A}))$};
\node (C') at (2,-1) {$1_\mathsf{A}$};
\node (D) at (3,0) {$!_{1_\mathsf{A}} \in F(1_\mathsf{A})$};
\node (D') at (3.75,-1) {$!_{R(L(1_\mathsf{A}))} \in F(R(L(1_\mathsf{A})))$};
\path[->,font=\scriptsize,>=angle 90]
(A) edge node[above]{$1$} (B)
(C) edge node[above]{$1$} (B)
(A') edge node[above]{${\eta_{1_\mathsf{A}}}$} (B')
(C') edge node[above]{${\eta_{1_\mathsf{A}}}$} (B')
(A) edge node [left]{$1$} (A')
(B) edge node [left]{${\eta_{1_\mathsf{A}}}$} (B')
(C) edge node [left]{$1$} (C');
\end{tikzpicture}
\]
$$\iota_{{\eta_{1_\mathsf{A}}}} \colon F({\eta_{1_\mathsf{A}}})(!_{1_\mathsf{A}})) \to !_{R(L(1_\mathsf{A}))}$$
of $F\mathbb{C}\textnormal{sp}$. This is just the natural isomorphism $\mathbb{E}_{1_\mathsf{A}}$ from earlier.

Given two horizontal 1-cells $M$ and $N$ of $_L \mathbb{C}\textnormal{sp}(\mathsf{X})$:
\[
\begin{tikzpicture}[scale=1.5]
\node (A) at (0,0) {$L(c_1)$};
\node (B) at (1,0) {$x$};
\node (C) at (2,0) {$L(c_2)$};
\node (D) at (3,0) {$L(c_1^\prime)$};
\node (E) at (4,0) {$x^\prime$};
\node (F) at (5,0) {$L(c_2^\prime)$};
%\node (D) at (3,0.5) {$!_c \in F(c)$};
\path[->,font=\scriptsize,>=angle 90]
(A) edge node[above]{$i$} (B)
(C) edge node[above]{$o$} (B)
(D) edge node[above]{$i^\prime$} (E)
(F) edge node[above]{$o^\prime$} (E);
\end{tikzpicture}
\]
their images $\mathbb{E}(M)$ and $\mathbb{E}(N)$ are given by:
\[
\begin{tikzpicture}[scale=1.5]
\node (A) at (0,0) {$c_1$};
\node (B) at (1,0) {$R(x)$};
\node (C) at (2,0) {$c_2$};
\node (D) at (3,0) {$c_1^\prime$};
\node (E) at (4,0) {$R(x^\prime)$};
\node (F) at (5,0) {$c_2^\prime$};
\node (G) at (1,-0.5) {$x \in F(R(x))$};
\node (H) at (4,-0.5) {$x^\prime \in F(R(x^\prime))$};
%\node (D) at (3,0.5) {$!_c \in F(c)$};
\path[->,font=\scriptsize,>=angle 90]
(A) edge node[above]{$R(i) \eta_{c_1}$} (B)
(C) edge node[above]{$R(o) \eta_{c_2}$} (B)
(D) edge node[above]{$R(i^\prime) \eta_{c_1^\prime}$} (E)
(F) edge node[above]{$R(o^\prime) \eta_{c_2^\prime}$} (E);
\end{tikzpicture}
\]
and so $\mathbb{E}(M) \otimes \mathbb{E}(N)$ is given by:
\[
\begin{tikzpicture}[scale=1.5]
\node (A) at (0,0) {$c_1+c_1^\prime$};
\node (B) at (2.25,0) {$R(x)+R(x^\prime)$};
\node (C) at (4.5,0) {$c_2+c_2^\prime$};
\node (D) at (2.25,-0.5) {$\hat{x} \in F(R(x)+R(x^\prime))$}; 
%\node (D) at (3,0.5) {$!_c \in F(c)$};
\path[->,font=\scriptsize,>=angle 90]
(A) edge node[above]{$R(i) \eta_{c_1} +R(i^\prime) \eta_{c_1^\prime}$} (B)
(C) edge node[above]{$R(o) \eta_{c_2} + R(o) \eta_{c_2^\prime}$} (B);
\end{tikzpicture}
\]
where $$\hat{x} \colon 1 \xrightarrow{\lambda^{-1}} 1 \times 1 \xrightarrow{x \times x^\prime} F(R(x)) \times F(R(x^\prime)) \xrightarrow{\phi_{R(x),R(x^\prime)}} F(R(x)+R(x^\prime)).$$ On the other hand, $M \otimes N$ is given by
\[
\begin{tikzpicture}[scale=1.5]
\node (A) at (0,0) {$L(c_1+c_1^\prime)$};
\node (B) at (1.5,0) {$x+x^\prime$};
\node (C) at (3,0) {$L(c_2+c_2^\prime)$};
%\node (D) at (3,0.5) {$!_c \in F(c)$};
\path[->,font=\scriptsize,>=angle 90]
(A) edge node[above]{$i+i^\prime$} (B)
(C) edge node[above]{$o+o^\prime$} (B);
\end{tikzpicture}
\]
and $\mathbb{E}(M \otimes N)$ is given by:
\[
\begin{tikzpicture}[scale=1.5]
\node (A) at (0,0) {$c_1+c_1^\prime$};
\node (B) at (2,0) {$R(x+x^\prime)$};
\node (C) at (4,0) {$c_2+c_2^\prime$};
\node (D) at (2,-0.5) {$x+x^\prime \in F(R(x+x^\prime)).$};
%\node (D) at (3,0.5) {$!_c \in F(c)$};
\path[->,font=\scriptsize,>=angle 90]
(A) edge node[above]{$R(i+i^\prime) \eta_{c_1+c_1^\prime}$} (B)
(C) edge node[above]{$R(o+o^\prime) \eta_{c_2+c_2^\prime}$} (B);
\end{tikzpicture}
\]
We then have a 2-isomorphism $\mu_{M,N} \colon E(M) \otimes E(N) \xrightarrow{\sim} E(M \otimes N)$ in $F\mathbb{C}\textnormal{sp}$ given by:
\[
\begin{tikzpicture}[scale=1.5]
\node (A) at (-0.5,0) {$c_1+c_1^\prime$};
\node (B) at (2,0) {$R(x)+R(x^\prime)$};
\node (C) at (4.5,0) {$c_2+c_2^\prime$};
\node (A') at (-0.5,-1) {$c_1+c_1^\prime$};
\node (B') at (2,-1) {$R(x+x^\prime)$};
\node (C') at (4.5,-1) {$c_2+c_2^\prime$};
\node (D) at (6.25,0) {$\hat{x} \in F(R(x)+R(x^\prime))$};
\node (D') at (6.25,-1) {$x+x^\prime \in F(R(x+x^\prime))$};
\node (E) at (2,-1.5) {$\iota_\mu \colon F(\kappa)(\hat{x}) \to x+x^\prime$};
\path[->,font=\scriptsize,>=angle 90]
(A) edge node[above]{$R(i)\eta_{c_1} + R(i^\prime)\eta_{c_1^\prime}$} (B)
(C) edge node[above]{$R(o)\eta_{c_2} + R(o^\prime)\eta_{c_2^\prime}$} (B)
(A') edge node[above]{$R(i+i^\prime)\eta_{c_1+c_1^\prime}$} (B')
(C') edge node[above]{$R(o+o^\prime)\eta_{c_2+c_2^\prime}$} (B')
(A) edge node [left]{$1$} (A')
(B) edge node [left]{$\kappa$} (B')
(C) edge node [left]{$1$} (C');
\end{tikzpicture}
\]
where $\kappa$ is an isomorphism as $R \colon \mathsf{X} \to \mathsf{A}$ preserves finite colimits.

The isomorphisms $\delta$ and $\mu$ satisfy the left and right unitality squares, associativity hexagon and braiding square. Let $M_1,M_2$ and $M_3$ be horizontal 1-cells in $_L \mathbb{C}\textnormal{sp}(\mathsf{X})$ given by:
\[
\begin{tikzpicture}[scale=1.5]
\node (A) at (0,0) {$L(c_1)$};
\node (B) at (1,0) {$x_1$};
\node (C) at (2,0) {$L(c_1^\prime)$};
\node (D) at (3,0) {$L(c_2)$};
\node (E) at (4,0) {$x_2$};
\node (F) at (5,0) {$L(c_2^\prime)$};
\node (G) at (6,0) {$L(c_3)$};
\node (H) at (7,0) {$x_3$};
\node (I) at (8,0) {$L(c_3^\prime)$};
%\node (D) at (3,0.5) {$!_c \in F(c)$};
\path[->,font=\scriptsize,>=angle 90]
(A) edge node[above]{$i_1$} (B)
(C) edge node[above]{$o_1$} (B)
(D) edge node[above]{$i_2$} (E)
(F) edge node[above]{$o_2$} (E)
(G) edge node[above]{$i_3$} (H)
(I) edge node[above]{$o_3$} (H);
\end{tikzpicture}
\]
The left unitality square:
\[
\begin{tikzpicture}[scale=1.5]
\node (A) at (0,0) {$1_{F\mathbb{C}\textnormal{sp}} \otimes \mathbb{E}(M_1)$};
\node (B) at (3,0) {$\mathbb{E}(1_{ _L \mathbb{C}\textnormal{sp}(\mathsf{X})}) \otimes \mathbb{E}(M_1)$};
\node (C) at (0,-1) {$\mathbb{E}(M_1)$};
\node (D) at (3,-1) {$\mathbb{E}(1_{ _L \mathbb{C}\textnormal{sp}(\mathsf{X})} \otimes M_1)$};
\path[->,font=\scriptsize,>=angle 90]
(A) edge node[above]{$\delta \otimes 1$} (B)
(B) edge node[right]{$\mu_{1,M_1}$} (D)
(A) edge node[left]{$\lambda$} (C)
(D) edge node[above]{$\mathbb{E}(\lambda)$} (C);
\end{tikzpicture}
\]
has underlying maps of cospans given by:
\[
		\begin{tikzpicture}
			\node (d) at (7.5,0) {$\mathbb{E}(1_{ _L \mathbb{C}\textnormal{sp}(\mathsf{X})}) \otimes \mathbb{E}(M_1)$};
			\node (a) at (-4,0) {$1_\mathsf{A}+c_1$};
			\node (b) at (0.5,0) {$R(L(1_\mathsf{A})) + R(x_1)$};
			\node (c) at (5,0) {$1_\mathsf{A}+c_1^\prime$};
			\node (d2) at (7.5,1) {$1_{F\mathbb{C}\textnormal{sp}} \otimes \mathbb{E}(M_1)$};
			\node (a2) at (-4,1) {$1_\mathsf{A}+c_1$};
			\node (b2) at (0.5,1) {$1_\mathsf{A}+R(x_1)$};
			\node (c2) at (5,1) {$1_\mathsf{A}+c_1^\prime$};
			\node (d3) at (7.5,2) {$\mathbb{E}(M_1)$};
                                \node (a3) at (-4,2) {$c_1$};
			\node (b3) at (0.5,2) {$R(x_1)$};
			\node (c3) at (5,2) {$c_1^\prime$};
			\node (d4) at (7.5,-1) {$\mathbb{E}(1_{ _L \mathbb{C}\textnormal{sp}(\mathsf{X})} \otimes M_1)$};
                                \node (a5) at (-4,-1) {$1_\mathsf{A}+c_1^\prime$};
			\node (b5) at (0.5,-1) {$R(L(1_\mathsf{A})+x_1)$};
			\node (c5) at (5,-1) {$1_\mathsf{A}+c_1^\prime$};
			\node (d5) at (7.5,-2) {$\mathbb{E}(M_1)$};
                                \node (a6) at (-4,-2) {$c_1$};
			\node (b6) at (0.5,-2) {$R(x_1)$};
			\node (c6) at (5,-2) {$c_1^\prime$};
			\path[->,font=\scriptsize,>=angle 90]
			(d2) edge node [left]{$\lambda$} (d3)
			(d2) edge node [left] {$\delta \otimes 1$} (d)
			(d) edge node [left] {$\mu_{1,M_1}$} (d4)
			(d4) edge node [left] {$\mathbb{E}(\lambda)$} (d5)
			(a) edge node[above]{$\eta_{1_\mathsf{A}}+R(i_1)\eta_{c_1}$} (b)
			(c) edge node[above]{$\eta_{1_\mathsf{A}}+R(o_1)\eta_{c_1^\prime}$} (b)
                                (a2) edge node[above]{$1+R(i_1)\eta_{c_1}$} (b2)
			(c2) edge node[above]{$1+R(o_1)\eta_{c_1^\prime}$} (b2)
                                (a2) edge node[left]{$1$} (a)
                                (b2) edge node[left]{$\eta_{1_\mathsf{A}}+1$} (b)
(b2) edge node[right]{$\iota_2$} (b)
			(c2) edge node[left]{$1$} (c)
                                (a3) edge node[above]{$R(i_1)\eta_{c_1}$} (b3)
			(c3) edge node[above]{$R(o_1)\eta_{c_1^\prime}$} (b3)
                                (a2) edge node[left]{$\lambda_{\mathsf{A}}$} (a3)
                                (b2) edge node[left]{$\lambda_{\mathsf{A}}$} (b3)
(b2) edge node[right]{$\iota_1$} (b3)
			(c2) edge node[left]{$\lambda_{\mathsf{A}}$} (c3)
                                (a5) edge node[above]{$(\mu_{L(1_\mathsf{A}),d_1})(\eta_{1_\mathsf{A}}+R(i_1)\eta_{c_1})$} (b5)
			(c5) edge node[above]{$(\mu_{L(1_\mathsf{A}),d_1})(\eta_{1_\mathsf{A}}+R(o_1)\eta_{c_1^\prime})$} (b5)
                                (a) edge node[left]{$1$} (a5)
                                (b) edge node[left]{$\mu_{L(1_\mathsf{A}),x_1}$} (b5)
(b) edge node[right]{$\iota_3$} (b5)
			(c) edge node[left]{$1$} (c5)
                                (a6) edge node[above]{$R(i_1)\eta_{c_1}$} (b6)
			(c6) edge node[above]{$R(o_1)\eta_{c_1^\prime}$} (b6)
                                (a5) edge node[left]{$\lambda_{\mathsf{A}}$} (a6)
                                (b5) edge node[left]{$R(\lambda_{\mathsf{X}})$} (b6)
 (b5) edge node[right]{$\iota_4$} (b6)
			(c5) edge node[left]{$\lambda_{\mathsf{A}}$} (c6);
		\end{tikzpicture}
	\]
with the corresponding maps of decorations amounting to the following commutative diagram in $F(R(x_1))$:
\[
\begin{tikzpicture}[scale=1.5]
\node (A) at (-.5,0) {$F(\lambda_{\mathsf{A}})(!_{1_\mathsf{A}}+x_1)$};
\node (C) at (4,0) {$F(R(\lambda_{\mathsf{X}})(\mu_{L(1_\mathsf{A}),x_1}))(!_{R(L(1_\mathsf{A}))} +x_1$)};
\node (D) at (-.5,-1) {$x_1$};
\node (E) at (4,-1) {$F(R(\lambda_\mathsf{X}))(x_{!+1})$};
\path[->,font=\scriptsize,>=angle 90]
(A) edge node[above]{$F(R(\lambda_{\mathsf{X}})(\mu_{L(1_\mathsf{A}),x_1}))(\iota_2)$} (C)
(A) edge node[left]{$\iota_1$} (D)
(E) edge node[above]{$\iota_4$} (D)
(C) edge node[right]{$F(R(\lambda_\mathsf{X}))(\iota_3)$} (E);
\end{tikzpicture}
\]
where $x_{!+1}$ is the decoration $x_1$ on the object $R(L(1_\mathsf{A})+x_1) \in \mathsf{A}$. The above square commutes because $$F(\lambda_\mathsf{A})(!_{1_\mathsf{A}}+x_1) = F(R(\lambda_\mathsf{X})(\mu_{L(1_\mathsf{A}),x_1})(\eta_{1_\mathsf{A}}+1))(!_{1_\mathsf{A}}+x_1)$$ as the corresponding left unitality square for the symmetric monoidal functor $R \colon (\mathsf{X},1_\mathsf{X},+) \to (\mathsf{A},1_\mathsf{A},+)$ commutes. The right unitality square is similar. The associator hexagon:
\[
\begin{tikzpicture}[scale=1.5]
\node (A) at (0,0) {$(\mathbb{E}(M_1) \otimes \mathbb{E}(M_2)) \otimes \mathbb{E}(M_3)$};
\node (B) at (3.5,0) {$\mathbb{E}(M_1 \otimes M_2) \otimes \mathbb{E}(M_3)$};
\node (C) at (7,0) {$\mathbb{E}((M_1 \otimes M_2) \otimes M_3)$};
\node (A') at (0,-1) {$\mathbb{E}(M_1) \otimes (\mathbb{E}(M_2) \otimes \mathbb{E}(M_3))$};
\node (B') at (3.5,-1) {$\mathbb{E}(M_1) \otimes \mathbb{E}(M_2 \otimes M_3)$};
\node (C') at (7,-1) {$\mathbb{E}(M_1 \otimes (M_2 \otimes M_3))$};
\path[->,font=\scriptsize,>=angle 90]
(A) edge node[above]{$\mu_{M_1,M_2} \otimes 1$} (B)
(B) edge node[above]{$\mu_{M_1 \otimes M_2,M_3}$} (C)
(A') edge node[above]{$1 \otimes \mu_{M_2,M_3}$} (B')
(B') edge node[above]{$\mu_{M_1,M_2 \otimes M_3}$} (C')
(A) edge node [left]{$a^\prime$} (A')
(C) edge node [left]{$\mathbb{E}(a)$} (C');
\end{tikzpicture}
\]
has underlying maps of cospans given by:
\[
		\begin{tikzpicture}
\node(d) at (1.5,7) {$(\mathbb{E}(M_1) \otimes \mathbb{E}(M_2)) \otimes \mathbb{E}(M_3)$};
\node (d2) at (1.5,8) {$\mathbb{E}(M_1 \otimes M_2) \otimes \mathbb{E}(M_3)$};
\node (d3) at (1.5,9) {$\mathbb{E}((M_1 \otimes M_2) \otimes M_3)$};
\node (d4) at (1.5,10) {$\mathbb{E}(M_1 \otimes (M_2 \otimes M_3))$};
\node (d5) at (1.5,6) {$\mathbb{E}(M_1) \otimes (\mathbb{E}(M_2) \otimes \mathbb{E}(M_3))$};
\node (d6) at (1.5,5) {$\mathbb{E}(M_1) \otimes \mathbb{E}(M_2 \otimes M_3)$};
\node (d7) at (1.5,4) {$\mathbb{E}(M_1 \otimes (M_2 \otimes M_3))$};
			\node (a) at (-4.25,0) {$(c_1+c_2)+c_3$};
			\node (b) at (1.5,0) {$(R(x_1)+R(x_2))+R(x_3)$};
			\node (c) at (7.25,0) {$(c_1^\prime+c_2^\prime)+c_3^\prime$};
			\node (a2) at (-4.25,1) {$(c_1+c_2)+c_3$};
			\node (b2) at (1.5,1) {$R(x_1+x_2)+R(x_3)$};
			\node (c2) at (7.25,1) {$(c_1^\prime+c_2^\prime)+c_3^\prime$};
                                \node (a3) at (-4.25,2) {$(c_1+c_2)+c_3$};
			\node (b3) at (1.5,2) {$R((x_1+x_2)+x_3)$};
			\node (c3) at (7.25,2) {$(c_1^\prime+c_2^\prime)+c_3^\prime$};
                                \node (a4) at (-4.25,3) {$c_1+(c_2+c_3)$};
			\node (b4) at (1.5,3) {$R(x_1+(x_2+x_3))$};
			\node (c4) at (7.25,3) {$c_1^\prime + (c_2^\prime + c_3^\prime)$};
                                \node (a5) at (-4.25,-1) {$c_1+(c_2+c_3)$};
			\node (b5) at (1.5,-1) {$R(x_1)+(R(x_2)+R(x_3))$};
			\node (c5) at (7.25,-1) {$c_1^\prime + (c_2^\prime + c_3^\prime)$};
                                \node (a6) at (-4.25,-2) {$c_1+(c_2+c_3)$};
			\node (b6) at (1.5,-2) {$R(x_1)+R(x_2+x_3)$};
			\node (c6) at (7.25,-2) {$c_1^\prime + (c_2^\prime + c_3^\prime)$};
                                \node (a7) at (-4.25,-3) {$c_1+(c_2+c_3)$};
			\node (b7) at (1.5,-3) {$R(x_1+(x_2+x_3))$};
			\node (c7) at (7.25,-3) {$c_1^\prime + (c_2^\prime + c_3^\prime)$};
			\path[->,font=\scriptsize,>=angle 90]
(d) edge node[left] {$\mu_{M_1,M_2} \otimes 1$} (d2)
(d2) edge node[left] {$\mu_{M_1 \otimes M_2,M_3}$} (d3)
(d3) edge node[left] {$\mathbb{E}(a)$} (d4)
(d) edge node[left] {$a^\prime$} (d5)
(d5) edge node[left] {$1 \otimes \mu_{M_2,M_3}$} (d6)
(d6) edge node[left] {$\mu_{M_1,M_2 \otimes M_3}$} (d7)
			(a) edge node[above]{$(R(i_1)\eta_{c_1}+R(i_2)\eta_{c_2})+R(i_3)\eta_{c_3}$} (b)
			(c) edge node[above]{$(R(o_1)\eta_{c_1^\prime}+R(o_2)\eta_{c_2^\prime})+R(o_3)\eta_{c_3^\prime}$} (b)
                                (a2) edge node[above]{$R(i_1+i_2)\eta_{c_1+c_2}+R(i_3)\eta_{c_3}$} (b2)
			(c2) edge node[above]{$R(o_1+o_2)\eta_{c_1^\prime+c_2^\prime}+R(o_3)\eta_{c_3^\prime}$} (b2)
                                (a) edge node[left]{$1$} (a2)
                                (b) edge node[left]{$\kappa + 1$} (b2)
(b) edge node[right]{$\iota_1$} (b2)
			(c) edge node[left]{$1$} (c2)
                                (a3) edge node[above]{$R((i_1+i_2)+i_3)\eta_{(c_1+c_2)+c_3}$} (b3)
			(c3) edge node[above]{$R((o_1+o_2)+o_3)\eta_{(c_1^\prime+c_2^\prime)+c_3^\prime}$} (b3)
                                (a2) edge node[left]{$1$} (a3)
                                (b2) edge node[left]{$\kappa$} (b3)
(b2) edge node[right]{$\iota_2$} (b3)
			(c2) edge node[left]{$1$} (c3)
                                (a4) edge node[above]{$R(i_1+(i_2+i_3))\eta_{c_1+(c_2+c_3)}$} (b4)
			(c4) edge node[above]{$R(o_1+(o_2+o_3))\eta_{c_1^\prime+(c_2^\prime+c_3^\prime)}$} (b4)
                                (a3) edge node[left]{$a_\mathsf{A}$} (a4)
                                (b3) edge node[left]{$R(a_\mathsf{X})$} (b4)
(b3) edge node[right]{$\iota_3$} (b4)
			(c3) edge node[left]{$a_\mathsf{A}$} (c4)
                                (a5) edge node[above]{$R(i_1)\eta_{c_1}+(R(i_2)\eta_{c_2}+R(i_3)\eta_{c_3})$} (b5)
			(c5) edge node[above]{$R(o_1)\eta_{c_1^\prime}+(R(o_2)\eta_{c_2^\prime}+R(o_3)\eta_{c_3^\prime})$} (b5)
                                (a) edge node[left]{$a_\mathsf{A}$} (a5)
                                (b) edge node[left]{$a_\mathsf{A}$} (b5)
(b) edge node[right]{$\iota_4$} (b5)
			(c) edge node[left]{$a_\mathsf{A}$} (c5)
                                (a6) edge node[above]{$R(i_1)\eta_{c_1}+R(i_2+i_3)\eta_{c_2+c_3}$} (b6)
			(c6) edge node[above]{$R(o_1)\eta_{c_1^\prime}+R(o_2+o_3)\eta_{c_2^\prime+c_3^\prime}$} (b6)
                                (a5) edge node[left]{$1$} (a6)
                                (b5) edge node[left]{$1+\kappa$} (b6)
 (b5) edge node[right]{$\iota_5$} (b6)
			(c5) edge node[left]{$1$} (c6)
                                (a7) edge node[above]{$R(i_1+(i_2+i_3))\eta_{c_1+(c_2+c_3)}$} (b7)
			(c7) edge node[above]{$R(o_1+(o_2+o_3))\eta_{c_1^\prime+(c_2^\prime+c_3^\prime)}$} (b7)
                                (a6) edge node[left]{$1$} (a7)
                                (b6) edge node[left]{$\kappa$} (b7)
(b6) edge node[right]{$\iota_6$} (b7)
			(c6) edge node[left]{$1$} (c7);
		\end{tikzpicture}
	\]
with the corresponding maps of decorations amounting to the following commutative diagram in $F(R(x_1+(x_2+x_3)))$:
\[
\begin{tikzpicture}[scale=1.5]
\node (A) at (0,0) {$F((\kappa)(1+\kappa)(a_\mathsf{A}))((x_1+x_2)+x_3)$};
\node (B) at (5,0) {$F((R(a_\mathrm{X}))(\kappa))((x_1+x_2)+x_3)$};
\node (C) at (5,-1) {$F(R(a_\mathrm{X}))((x_1+x_2)+x_3)$};
\node (D) at (5,-2) {$x_1+(x_2+x_3)$};
\node (E) at (0,-1) {$F((\kappa)(1+\kappa))((x_1+x_2)+x_3)$};
\node (F) at (0,-2) {$F(\kappa)(x_1+(x_2+x_3))$};
\path[->,font=\scriptsize,>=angle 90]
(A) edge node[above]{$F((R(a_\mathsf{X}))(\kappa))(\iota_1)$} (B)
(B) edge node[right]{$F(R(a_\mathsf{X}))(\iota_2)$} (C)
(C) edge node[right]{$\iota_3$} (D)
(A) edge node[left]{$F((\kappa)(1+\kappa))(\iota_4)$} (E)
(E) edge node[left]{$F(\kappa)(\iota_5)$} (F)
(F) edge node[above]{$\iota_6$} (D);
\end{tikzpicture}
\]
The above square commutes because $$F((\kappa)(1+\kappa)(a_\mathsf{A}))((x_1+x_2)+x_3) = F((R(a_\mathsf{X}))(\kappa)(\kappa+1))((x_1+x_2)+x_3)$$ as the corresponding associator hexagon for the symmetric monoidal functor $R \colon (\mathsf{X},1_\mathsf{X},+) \to (\mathsf{A},1_\mathsf{A},+)$ commutes. Lastly, the braiding square:
\[
\begin{tikzpicture}[scale=1.5]
\node (A) at (0,0) {$\mathbb{E}(M_1) \otimes \mathbb{E}(M_2)$};
\node (B) at (3,0) {$\mathbb{E}(M_2) \otimes \mathbb{E}(M_1)$};
\node (C) at (0,-1) {$\mathbb{E}(M_1 \otimes M_2)$};
\node (D) at (3,-1) {$\mathbb{E}(M_2 \otimes M_1)$};
\path[->,font=\scriptsize,>=angle 90]
(A) edge node[above]{$\beta^\prime$} (B)
(B) edge node[right]{$\mu_{M_2,M_1}$} (D)
(A) edge node[left]{$\mu_{M_1,M_2}$} (C)
(C) edge node[above]{$\mathbb{E}(\beta)$} (D);
\end{tikzpicture}
\]
has underlying map of cospans given by:
\[
		\begin{tikzpicture}
\node (d) at (7,0) {$\mathbb{E}(M_1) \otimes \mathbb{E}(M_2)$};
\node (d2) at (7,1) {$\mathbb{E}(M_2) \otimes \mathbb{E}(M_1)$};
\node (d3) at (7,2) {$\mathbb{E}(M_2 \otimes M_1)$};
\node (d4) at (7,-1) {$\mathbb{E}(M_1 \otimes M_2)$};
\node (d5) at (7,-2) {$\mathbb{E}(M_2 \otimes M_1)$};
			\node (a) at (-4,0) {$c_1+c_2$};
			\node (b) at (0.5,0) {$R(x_1)+R(x_2)$};
			\node (c) at (5,0) {$c_1^\prime+c_2^\prime$};
			\node (a2) at (-4,1) {$c_2+c_1$};
			\node (b2) at (0.5,1) {$R(x_2)+R(x_1)$};
			\node (c2) at (5,1) {$c_2^\prime+c_1^\prime$};
                                \node (a3) at (-4,2) {$c_2+c_1$};
			\node (b3) at (0.5,2) {$R(x_2+x_1)$};
			\node (c3) at (5,2) {$c_2^\prime + c_1^\prime$};
                                \node (a5) at (-4,-1) {$c_1+c_2$};
			\node (b5) at (0.5,-1) {$R(x_1+x_2)$};
			\node (c5) at (5,-1) {$c_1^\prime+c_2^\prime$};
                                \node (a6) at (-4,-2) {$c_2+c_1$};
			\node (b6) at (0.5,-2) {$R(x_2+x_1)$};
			\node (c6) at (5,-2) {$c_2^\prime + c_1^\prime$};
			\path[->,font=\scriptsize,>=angle 90]
(d) edge node[left]{$\beta^\prime$} (d2)
(d2) edge node[left]{$\mu_{M_2,M_1}$}(d3)
(d) edge node[left] {$\mu_{M_1,M_2}$}(d4)
(d4)edge node[left]{$\mathbb{E}(\beta)$}(d5)
			(a) edge node[above]{$R(i_1) \eta_{c_1} + R(i_2)\eta_{c_2}$} (b)
			(c) edge node[above]{$R(o_1) \eta_{c_1^\prime} + R(o_2) \eta_{c_2^\prime}$} (b)
                                (a2) edge node[above]{$R(i_2) \eta_{c_2} + R(i_1) \eta_{c_1}$} (b2)
			(c2) edge node[above]{$R(o_2) \eta_{c_2^\prime} + R(o_1) \eta_{c_1^\prime}$} (b2)
                                (a) edge node[left]{$\beta_\mathsf{A}$} (a2)
                                (b) edge node[left]{$\beta_\mathsf{A}$} (b2)
(b) edge node[right]{$\iota_1$} (b2)
			(c) edge node[left]{$\beta_\mathsf{A}$} (c2)
                                (a3) edge node[above]{$R(i_2+i_1)\eta_{c_2+c_1}$} (b3)
			(c3) edge node[above]{$R(o_2+o_1)\eta_{c_2^\prime+c_1^\prime}$} (b3)
                                (a2) edge node[left]{$1$} (a3)
                                (b2) edge node[left]{$\kappa$} (b3)
(b2) edge node[right]{$\iota_2$} (b3)
			(c2) edge node[left]{$1$} (c3)
                                (a5) edge node[above]{$R(i_1+i_2)\eta_{c_1+c_2}$} (b5)
			(c5) edge node[above]{$R(o_1+o_2)\eta_{c_1^\prime+c_2^\prime}$} (b5)
                                (a) edge node[left]{$1$} (a5)
                                (b) edge node[left]{$\kappa$} (b5)
(b) edge node[right]{$\iota_3$} (b5)
			(c) edge node[left]{$1$} (c5)
                                (a6) edge node[above]{$R(i_2+i_1)\eta_{c_2+c_1}$} (b6)
			(c6) edge node[above]{$R(o_2+o_1)\eta_{c_2^\prime+c_1^\prime}$} (b6)
                                (a5) edge node[left]{$\beta_\mathsf{A}$} (a6)
                                (b5) edge node[left]{$R(\beta_\mathsf{X})$} (b6)
 (b5) edge node[right]{$\iota_4$} (b6)
			(c5) edge node[left]{$\beta_\mathsf{A}$} (c6);
		\end{tikzpicture}
	\]
with the corresponding maps of decorations amounting to the following commutative diagram in $F(R(x_2+x_1))$:
\[
\begin{tikzpicture}[scale=1.5]
\node (A) at (-.5,0) {$F((\kappa)(\beta_\mathsf{A}))(x_1+x_2)$};
\node (C) at (4,0) {$F(\kappa)(x_2+x_1)$};
\node (D) at (-.5,-1) {$F(R(\beta_\mathsf{X}))(x_1+x_2)$};
\node (E) at (4,-1) {$x_2+x_1$};
\path[->,font=\scriptsize,>=angle 90]
(A) edge node[above]{$F(\kappa)(\iota_1)$} (C)
(A) edge node[left]{$F(R(\beta_\mathsf{X}))(\iota_3)$} (D)
(D) edge node[above]{$\iota_4$} (E)
(C) edge node[right]{$\iota_2$} (E);
\end{tikzpicture}
\]
The above square commutes because $$F((\kappa)(\beta_\mathsf{A}))(x_1+x_2) = F((R(\beta_\mathsf{X}))(\kappa))(x_1+x_2)$$ as the corresponding braiding square for the symmetric monoidal functor $R \colon (\mathsf{X},1_\mathsf{X},+) \to (\mathsf{A},1_\mathsf{A},+)$ commutes. Thus the double functor $\mathbb{E} \colon _L\mathbb{C}\textnormal{sp}(\mathsf{X}) \to F\mathbb{C}\textnormal{sp}$ is symmetric monoidal.
\end{proof}
Using a result of Shulman \cite{Shul}, each of the isofibrant symmetric monoidal double categories $F\mathbb{C}\textnormal{sp}$ and $_L \mathbb{C}\textnormal{sp}(\mathrm{X})$ give rise to underlying symmetric monoidal bicategories, namely $F\mathbb{C}\textnormal{sp}$ induces a symmetric monoidal bicategory $F\mathbf{Csp} \coloneqq H(F\mathbb{C}\textnormal{sp})$ which has:
\begin{enumerate}
\item{objects as those of $\mathsf{A}$,}
\item{morphisms as horizontal 1-cells of $F\mathbb{C}\textnormal{sp}$, and}
\item{2-morphisms as globular 2-morphisms of $F\mathbb{C}\textnormal{sp}$.}
\end{enumerate} 
Likewise, $_L \mathbb{C}\textnormal{sp}(\mathsf{X})$ induces a symmetric monoidal bicategory $H( {_L \mathbb{C}\textnormal{sp}(\mathsf{X})})$ which has:
\begin{enumerate}
\item{objects as those of $\mathsf{A}$,}
\item{morphisms as horizontal 1-cells of $_L \mathbb{C}\textnormal{sp}(\mathsf{X})$, and}
\item{2-morphisms as globular 2-morphisms of $_L \mathbb{C}\textnormal{sp}(\mathsf{X})$.}
\end{enumerate}
Another result of Shulman \cite{Shul2} is the following:
\begin{prop}[Shulman, Prop. B.3]
An equivalence of fibrant double categories induces a biequivalence of horizontal bicategories.
\end{prop}
\begin{cor}
The bicategories $F\mathbf{Csp}$ and $H( {_L \mathbb{C}\textnormal{sp}(\mathsf{X})})$ are biequivalent.
\end{cor}

\textbf{Is this biequivalnce symmetric monoidal? Probably}
\newline


We can also define the part of the double equivalence $\mathbb{G} \colon F\mathbb{C}\textnormal{sp} \to {_L \mathbb{C} \textnormal{sp}(\mathsf{X})}$ which goes in the other direction: again, the object component of this double functor will be $\mathbb{G}_0 = \id_{\mathsf{A}}$.

Given a horizontal 1-cell $M$ of $F\mathbb{C}\textnormal{sp}$:
\[
\begin{tikzpicture}[scale=1.5]
\node (A) at (0,0) {$c_1$};
\node (B) at (1,0) {$c$};
\node (C) at (2,0) {$c_2$};
\node (D) at (1,-0.5) {$x \in F(c)$};
\path[->,font=\scriptsize,>=angle 90]
(A) edge node[above]{$I$} (B)
(C) edge node[above]{$O$} (B);
\end{tikzpicture}
\]
the image $\mathbb{G}(M)$ is the horizontal 1-cell in $_L \mathbb{C}\textnormal{sp}(\mathsf{X})$ given by:
\[
\begin{tikzpicture}[scale=1.5]
\node (A) at (0,0) {$L(c_1)$};
\node (B) at (1,0) {$x$};
\node (C) at (2,0) {$L(c_2)$};
\path[->,font=\scriptsize,>=angle 90]
(A) edge node[above]{$!_{c} L(I)$} (B)
(C) edge node[above]{$!_{c} L(O)$} (B);
\end{tikzpicture}
\]
where $!_{c} \colon L(c) \to x$ is the unique morphism from the trival decoration on $c$ given by $$!_{c} \coloneqq 1 \xrightarrow{\phi} F(0) \xrightarrow{F(!)} F(c)$$ to $x \in F(c)$. In other words, the trivial decoraction $!_{c}$ is initial in $F(c)$. Similarly, given a 2-morphism $(f,h,g,\iota) \colon M \to M^\prime$ of $F\mathbb{C}\textnormal{sp}$:
\[
\begin{tikzpicture}[scale=1.5]
\node (D) at (1,0.5) {$x \in F(c)$};
\node (D') at (1,-1.5) {$x^\prime \in F(c^\prime)$};
\node (A) at (0,0) {$c_1$};
\node (B) at (1,0) {$c$};
\node (C) at (2,0) {$c_2$};
\node (A') at (0,-1) {$c_1^\prime$};
\node (B') at (1,-1) {$c^\prime$};
\node (C') at (2,-1) {$c_2^\prime$};
\path[->,font=\scriptsize,>=angle 90]
(A) edge node[above]{$I$} (B)
(C) edge node[above]{$O$} (B)
(A') edge node[above]{$I^\prime$} (B')
(C') edge node[above]{$O^\prime$} (B')
(A) edge node [left]{$f$} (A')
(B) edge node [left]{$h$} (B')
(C) edge node [left]{$g$} (C');
\end{tikzpicture}
\]
$$\iota \colon F(h)(x) \to x^\prime$$
the image in $\mathbb{G}(f,h,g,\iota)$ in $_L \mathbb{C}\textnormal{sp}(\mathsf{X})$ is given by the 2-morphism:
\[
\begin{tikzpicture}[scale=1.5]
\node (A) at (0,0) {$L(c_1)$};
\node (B) at (1.5,0) {$x$};
\node (C) at (3,0) {$L(c_2)$};
\node (A') at (0,-1) {$L(c_1^\prime)$};
\node (B') at (1.5,-1) {$x^\prime$};
\node (C') at (3,-1) {$L(c_2^\prime)$};
\path[->,font=\scriptsize,>=angle 90]
(A) edge node[above]{$!_{c} L(I)$} (B)
(C) edge node[above]{$!_{c} L(O)$} (B)
(A') edge node[above]{$!_{c^\prime} L(I^\prime)$} (B')
(C') edge node[above]{$!_{c^\prime} L(O^\prime)$} (B')
(A) edge node [left]{$L(f)$} (A')
(B) edge node [left]{$\alpha$} (B')
(C) edge node [left]{$L(g)$} (C');
\end{tikzpicture}
\]
where $\alpha \colon x \to x^\prime$ is the morphism in the Grothendieck construction of $F$ given by $\alpha = (h \colon c \to c^\prime, \iota \colon F(h)(x) \to x^\prime)$. 

Next, we exhibit natural isomorphisms $\eta \colon \id_{ _L \mathbb{C}\textnormal{sp}(\mathsf{X})} \cong \mathbb{G} \mathbb{E}$ and $\epsilon \colon \mathbb{E} \mathbb{G} \cong \id_{F\mathbb{C}\textnormal{sp}}$. Specifically, these are double natural isomorphisms given by double transformations \cite{Shul2} $\eta$ and $\epsilon$ whose object and arrow components are isomorphisms.

First we compute the composites $\mathbb{G} \mathbb{E}$ and $\mathbb{E} \mathbb{G}$. On the object categories, both composites are $\id_\mathsf{A}$ and we have natural isomorphisms $\eta \colon \id_{\mathsf{A}} \cong \mathbb{G}_0 \mathbb{E}_0$ and $\epsilon \colon \mathbb{E}_0 \mathbb{G}_0 \cong \id_{\mathsf{A}}$.

Given a horizontal 1-cell $M$ in $_L \mathbb{C} \textnormal{sp}(\mathsf{X})$:
\[
\begin{tikzpicture}[scale=1.5]
\node (A) at (0,0) {$L(c_1)$};
\node (B) at (1,0) {$x$};
\node (C) at (2,0) {$L(c_2)$};
\path[->,font=\scriptsize,>=angle 90]
(A) edge node[above]{$i$} (B)
(C) edge node[above]{$o$} (B);
\end{tikzpicture}
\]
the horizontal 1-cell $\mathbb{E}(M)$ is given by:
\[
\begin{tikzpicture}[scale=1.5]
\node (A) at (0,0) {$c_1$};
\node (B) at (1,0) {$R(x)$};
\node (C) at (2,0) {$c_2$};
\node (D) at (1,-0.5) {$x \in F(R(x))$};
\path[->,font=\scriptsize,>=angle 90]
(A) edge node[above]{$R(i) \eta_{c_1}$} (B)
(C) edge node[above]{$R(o) \eta_{c_2}$} (B);
\end{tikzpicture}
\]
and then the horizontal 1-cell $\mathbb{G} \mathbb{E}(M)$ is given by:
\[
\begin{tikzpicture}[scale=1.5]
\node (A) at (0,0) {$L(c_1)$};
\node (B) at (1.5,0) {$x$};
\node (C) at (3,0) {$L(c_2)$};
\path[->,font=\scriptsize,>=angle 90]
(A) edge node[above]{$!_{R(x)}L(R(i)\eta_{c_1})$} (B)
(C) edge node[above]{$!_{R(x)}L(R(o)\eta_{c_2})$} (B);
\end{tikzpicture}
\]
Then we can find a 2-isomorphism $\eta_M \colon M \xrightarrow{\sim} \mathbb{G}\mathbb{E}(M) $ in $_L \mathbb{C} \textnormal{sp}(\mathsf{X})$ given by:
\[
\begin{tikzpicture}[scale=1.5]
\node (A) at (0,0) {$L(c_1)$};
\node (B) at (1.5,0) {$x$};
\node (C) at (3,0) {$L(c_2)$};
\node (A') at (0,-1) {$L(c_1)$};
\node (B') at (1.5,-1) {$x$};
\node (C') at (3,-1) {$L(c_2)$};
\path[->,font=\scriptsize,>=angle 90]
(A) edge node[above]{$i$} (B)
(C) edge node[above]{$o$} (B)
(A') edge node[above]{$!_{R(x)}L(R(i)\eta_{c_1})$} (B')
(C') edge node[above]{$!_{R(x)}L(R(o)\eta_{c_2})$} (B')
(A) edge node [left]{$1$} (A')
(B) edge node [left]{$1$} (B')
(C) edge node [left]{$1$} (C');
\end{tikzpicture}
\]
where
\[
\begin{tikzpicture}[scale=1.5]
\node (A) at (-0.5,0) {$L(c_1)$};
\node (B) at (1,0) {$L(R(L(c_1)))$};
\node (C) at (2.5,0) {$L(R(x))$};
\node (D) at (3.5,0) {$x$};
\path[->,font=\scriptsize,>=angle 90]
(A) edge node[above]{$L(\eta_{c_1})$} (B)
(B) edge node[above]{$L(R(i))$} (C)
(C) edge node[above]{$!_{R(x)}$} (D)
(A) edge[bend right] node [above]{$i$} (D);
\end{tikzpicture}
\]
commutes.

On the other hand, given a horizontal 1-cell $N$ in $F\mathbb{C}\textnormal{sp}$:
\[
\begin{tikzpicture}[scale=1.5]
\node (A) at (0,0) {$c_1$};
\node (B) at (1,0) {$c$};
\node (C) at (2,0) {$c_2$};
\node (D) at (1,-0.5) {$x \in F(c)$};
\path[->,font=\scriptsize,>=angle 90]
(A) edge node[above]{$I$} (B)
(C) edge node[above]{$O$} (B);
\end{tikzpicture}
\]
the horizontal 1-cell $\mathbb{G}(N)$ is given by:
\[
\begin{tikzpicture}[scale=1.5]
\node (A) at (0,0) {$L(c_1)$};
\node (B) at (1,0) {$x$};
\node (C) at (2,0) {$L(c_2)$};
\path[->,font=\scriptsize,>=angle 90]
(A) edge node[above]{$!_c L(I)$} (B)
(C) edge node[above]{$!_c L(O)$} (B);
\end{tikzpicture}
\]
and then the horizontal 1-cell $\mathbb{E} \mathbb{G}(N)$ is given by:
\[
\begin{tikzpicture}[scale=1.5]
\node (A) at (0,0) {$c_1$};
\node (B) at (1.5,0) {$R(x)$};
\node (C) at (3,0) {$c_2$};
\node (D) at (1.5,-0.5) {$x \in F(R(x))$};
\path[->,font=\scriptsize,>=angle 90]
(A) edge node[above]{$R(!_c L(I))\eta_{c_1}$} (B)
(C) edge node[above]{$R(!_c L(O))\eta_{c_2}$} (B);
\end{tikzpicture}
\]
Then we have a 2-isomorphism $\epsilon_N \colon \mathbb{E} \mathbb{G} (N) \xrightarrow{\sim} N$ in $F\mathbb{C}\textnormal{sp}$ given by:
\[
\begin{tikzpicture}[scale=1.5]
\node (D) at (1.5,0.5) {$x \in F(R(x))$};
\node (D') at (1.5,-1.5) {$x \in F(c)$};
\node (A) at (0,0) {$c_1$};
\node (B) at (1.5,0) {$R(x)$};
\node (C) at (3,0) {$c_2$};
\node (A') at (0,-1) {$c_1$};
\node (B') at (1.5,-1) {$c$};
\node (C') at (3,-1) {$c_2$};
\path[->,font=\scriptsize,>=angle 90]
(A) edge node[above]{$R(!_c L(I))\eta_{c_1}$} (B)
(C) edge node[above]{$R(!_c L(O))\eta_{c_2}$} (B)
(A') edge node[above]{$I$} (B')
(C') edge node[above]{$O$} (B')
(A) edge node [left]{$1$} (A')
(B) edge node [left]{$e$} (B')
(C) edge node [left]{$1$} (C');
\end{tikzpicture}
\]
$$\iota \colon F(e)(x) \to x$$
where 
\[
\begin{tikzpicture}[scale=1.5]
\node (A) at (0,0) {$c_1$};
\node (B) at (1,0) {$R(L(c_1))$};
\node (C) at (2.5,0) {$R(L(c))$};
\node (D) at (3.5,0) {$R(x)$};
\node (E) at (4.5,0) {$c$};
\path[->,font=\scriptsize,>=angle 90]
(A) edge node[above]{$\eta_{c_1}$} (B)
(B) edge node[above]{$R(L(I))$} (C)
(C) edge node[above]{$R(!_c)$} (D)
(D) edge node[above]{$e$} (E)
(A) edge[bend right] node [above]{$I$} (E);
\end{tikzpicture}
\]
commutes.


\begin{comment}
\section{Applications}\label{Applications}
In this section we present several examples each of which may be realized in the context of decorated cospans or in the context of structured cospans. The first example regarding graphs was mentioned in the introduction and used as a reoccurring theme throughout the paper. The next three examples which take on more of an applied flavor, consists of electrical circuits, Markov processes and Petri nets. Each of these has been studied extensively by Baez, Fong, Master and Pollard by way of `black-boxing' \cite{BCR,BF,BFP,BM,BP}. Black-boxing is a way of interpreting the behavior of an open system, that is, a system with prescribed inputs and outputs such as the terminals of an electrical circuit, by observing the activity at the inputs and the outputs. The semantics of the activity at an open system's inputs and outputs is typically described in a category such as $\mathsf{LinRel}$ of finite dimensonal vector spaces and linear relations. Thus, in each case, black-boxing results in functors such as: $$\blacksquare_1 \colon \mathsf{Circ} \to \mathsf{LinRel}$$ $$\blacksquare_2 \colon \mathsf{Mark} \to \mathsf{LinRel}$$ $$\blacksquare_3 \colon \mathsf{Petri} \to \mathsf{LinRel}.$$ Each of these black-boxing functors also possess other convenient properties such as being symmetric monoidal. The first two of these were first done using Fong's theory of decorated cospans and then extended by the first two authors using structured cospans. The last two of these were also extended by being realized as double functors between double categories in recent works \cite{BC,BM}.
\subsection{Graphs}
As a first example that was also mentioned in the introduction, let $L \colon \mathsf{Set} \to \mathsf{Graph}$ be the functor that assigns to a set $N$ the \emph{discrete graph} on $N$ which is the edgeless graph $L(N)$ with no edges and $N$ as its set of vertices. Both $\mathsf{Set}$ and $\mathsf{Graph}$ are cocartesian monoidal and the functor $L \colon \mathsf{Set} \to \mathsf{Graph}$ is left adjoint to the forgetful functor $R \colon \mathsf{Graph} \to \mathsf{Set}$ which assigns to a graph $G$ its underlying set of vertices $U(G)$. Using structured cospans and appealing to Theorem \ref{SC}, we get a symmetric monoidal double category $_L \mathbb{C}\textnormal{sp}(\mathsf{Graph})$ which has:
\begin{enumerate}
\item{sets as objects,}
\item{functions as vertical 1-morphisms,}
\item{cospans of graphs, or, \emph{open} graphs of the form
\[
\begin{tikzpicture}[scale=1.5]
\node (A) at (0,0) {$L(N)$};
\node (B) at (1,0) {$G$};
\node (C) at (2,0) {$L(M)$};
\path[->,font=\scriptsize,>=angle 90]
(A) edge node[above]{$I$} (B)
(C) edge node[above]{$O$} (B);
\end{tikzpicture}
\]
as horizontal 1-cells, where $L(N)$ and $L(M)$ are discrete graphs on the sets $N$ and $M$, respectively, $G$ is a graph and $I$ and $O$ are graph morphisms, and}
\item{maps of cospans of graphs of the form
\[
\begin{tikzpicture}[scale=1.5]
\node (A) at (0,0) {$L(N_1)$};
\node (B) at (1,0) {$G_1$};
\node (C) at (2,0) {$L(M_1)$};
\node (A') at (0,-1) {$L(N_2)$};
\node (B') at (1,-1) {$G_2$};
\node (C') at (2,-1) {$L(M_2)$};
\path[->,font=\scriptsize,>=angle 90]
(A) edge node[above]{$I_1$} (B)
(C) edge node[above]{$O_1$} (B)
(A') edge node[above]{$I_2$} (B')
(C') edge node[above]{$O_2$} (B')
(A) edge node [left]{$L(f)$} (A')
(B) edge node [left]{$\alpha$} (B')
(C) edge node [left]{$L(g)$} (C');
\end{tikzpicture}
\]
as 2-morphisms, where $L(f)$ and $L(g)$ are maps of discrete graphs induced by the underlying functions $f$ and $g$, respectively, and $\alpha \colon G_1 \to G_2$ is a graph morphism.
}
\end{enumerate}

We can obtain a similar symmetric monoidal double category using decorated cospans. Let $F \colon \mathsf{Set} \to \mathsf{Cat}$ be the symmetric lax monoidal pseudofunctor that assigns to a set $N$ the \emph{category} of all graph structures whose underlying set of vertices is $N$. Using Theorem \ref{DC}, we then obtain a symmetric monoidal double category $F\mathbb{C}\textnormal{sp}$ which has:
\begin{enumerate}
\item{sets as objects,}
\item{functions as vertical 1-morphisms,}
\item{horizontal 1-cells as pairs:
\[
\begin{tikzpicture}[scale=1.5]
\node (A) at (0,0) {$N$};
\node (B) at (1,0) {$P$};
\node (C) at (2,0) {$M$};
\node (D) at (3.25,0) {$G \in F(P)$};
\path[->,font=\scriptsize,>=angle 90]
(A) edge node[above]{$i$} (B)
(C) edge node[above]{$o$} (B);
\end{tikzpicture}
\]
which can also be thought of as open graphs, and}
\item{2-morphisms as maps of cospans of sets
\[
\begin{tikzpicture}[scale=1.5]
\node (A) at (0,0) {$N_1$};
\node (A') at (0,-1) {$N_2$};
\node (C') at (2,-1) {$M_2$};
\node (B) at (1,0) {$P_1$};
\node (C) at (2,0) {$M_1$};
\node (D) at (1,-1) {$P_2$};
\node (E) at (3,0) {$G_1 \in F(P_1)$};
\node (F) at (3,-1) {$G_2 \in F(P_2)$};
\path[->,font=\scriptsize,>=angle 90]
(A) edge node[above]{$i_1$} (B)
(C) edge node[above]{$o_1$} (B)
(A) edge node[left]{$f$} (A')
(C) edge node[right]{$g$} (C')
(C') edge node [above] {$o_2$} (D)
(A') edge node [above] {$i_2$} (D)
(B) edge node [left] {$h$} (D);
\end{tikzpicture}
\]
together with a graph morphism $\iota \colon F(h)(G_1) \to G_2$ in $F(P_2)$.}
\end{enumerate}
We thus have two symmetric monoidal double categories: $_L \mathbb{C}\textnormal{sp}(\mathsf{Graph})$ obtained from structured cospans and $F\mathbb{C}\textnormal{sp}$ obtained from decorated cospans. Both of these double categories have $\mathsf{Set}$ as their categories of objects, open graphs as horizontal 1-cells and maps of open graphs as 2-morphisms, and by Theorem \ref{Equiv}, we have an equivalence of symmetric monoidal double categories $$_L \mathbb{C}\textnormal{sp}(\mathsf{Graph}) \sim F\mathbb{C}\textnormal{sp}.$$

\subsection{Passive linear circuits}
In a previous work \cite{BP}, Fong and the first author used decorated cospans to construct a symmetric monoidal category of `open passive linear circuits'. Roughly speaking, given a field $k$, a passive linear circuit is given by a `$k$-graph' which is a diagram in $\mathsf{FinSet}$ of the form:
\[
\begin{tikzpicture}[scale=1.5]
\node (B) at (1,0) {$E$};
\node (C) at (2,0) {$V$};
\node (A) at (0,0) {$k^+$};
\path[->,font=\scriptsize,>=angle 90]
(B)edge node[above] {$r$}(A)
(B)edge[bend left] node[above]{$s$}(C)
(B)edge[bend right] node[below]{$t$}(C);
\end{tikzpicture}
\]
Here the finite sets $E$ and $V$ are the sets of edges and vertices, respectively, and if we take the field $k = \mathbb{R}$, the function $r \colon E \to \mathbb{R}^+$ assigns to each edge $e \in E$ a positive real number $r(e) \in \mathbb{R}^+$ which can be interpreted as the resistance at the edge $e$. An open passive linear circuit is then given by a cospan of finite sets 
\[
\begin{tikzpicture}[scale=1.5]
\node (B) at (1,0) {$V$};
\node (C) at (2,0) {$Y$};
\node (A) at (0,0) {$X$};
\path[->,font=\scriptsize,>=angle 90]
(A)edge node[above] {$i$}(B)
(C)edge node[above]{$o$}(B);
\end{tikzpicture}
\]
where the apex $V$ is equipped with the structure of a passive linear circuit. See the original paper for more details \cite{BP}.

Let $\mathsf{Graph}_k$ be the category whose objects are given by $k$-graphs and morphisms by morphisms of $k$-graphs, where a morphism of $k$-graphs is given by a pair of functions $f \colon E \to E^\prime$ and $g \colon V \to V^\prime$ between the edge sets and vertex sets, respectively, of two $k$-graphs that respect the source and target functions of each. In a previous work introducing structured cospans, the first two authors showed that the category $\mathsf{Graph}_k$ has finite colimits \cite{BC2}. We can then obtain a double category of open passive linear circuits by defining a left adjoint $L \colon \mathsf{Set} \to \mathsf{Graph}_k$ that assigns to a set $V$ the discrete passive linear circuit $L(V)$ given by passive linear circuit with $V$ as its set of vertices and no edges. The resulting symmetric monoidal double category $_L \mathbb{C}\textnormal{sp}(\mathsf{Graph}_k)$ has:
\begin{enumerate}
\item{finite sets as objects,}
\item{functions as vertical 1-morphisms,}
\item{open passive linear circuits as horizontal 1-cells
\[
\begin{tikzpicture}[scale=1.5]
\node (B) at (1,0) {$V$};
\node (C) at (2,0) {$Y$};
\node (A) at (0,0) {$X$};
\node (D) at (3,0) {$k^+$};
\node (E) at (4,0) {$E$};
\node (F) at (5,0) {$V$};
\path[->,font=\scriptsize,>=angle 90]
(E) edge node [above] {$r$} (D)
(E) edge [bend left] node [above] {$s$} (F)
(E) edge [bend right] node [below] {$t$} (F)
(A)edge node[above] {$i$}(B)
(C)edge node[above]{$o$}(B);
\end{tikzpicture}
\]
and}
\item{maps of cospans as 2-morphisms together with a map of passive linear circuits between the apices.
\[
\begin{tikzpicture}[scale=1.5]
\node (A) at (0,0) {$X_1$};
\node (A') at (0,-1) {$X_2$};
\node (C') at (2,-1) {$Y_2$};
\node (B) at (1,0) {$V_1$};
\node (C) at (2,0) {$Y_1$};
\node (D) at (1,-1) {$V_2$};
\node (E) at (3,0) {$k^+$};
\node (E') at (3,-1) {$k^+$};
\node (F) at (4,0) {$E_1$};
\node (F') at (4,-1) {$E_2$};
\node (G) at (5,0) {$V_1$};
\node (G') at (5,-1) {$V_2$};
\path[->,font=\scriptsize,>=angle 90]
(F) edge node[above] {$r_1$} (E)
(F) edge [bend left] node [above] {$s_1$} (G)
(F) edge [bend right] node [below] {$t_1$} (G)
(F') edge node[above] {$r_2$} (E')
(F') edge [bend left] node [above] {$s_2$} (G')
(F') edge [bend right] node [below] {$t_2$} (G')
(A) edge node[above]{$i_1$} (B)
(C) edge node[above]{$o_1$} (B)
(A) edge node[left]{$h$} (A')
(C) edge node[left]{$h^\prime$} (C')
(C') edge node [above] {$o_2$} (D)
(A') edge node [above] {$i_2$} (D)
(B) edge node [left] {$g$} (D)
(F) edge node [left] {$f$} (F')
(G) edge node [right] {$g$} (G');
\end{tikzpicture}
\]
}
\end{enumerate}
We can also obtain a similar double category using decorated cospans: define a pseudofunctor $F \colon \mathsf{FinSet} \to \mathbf{Cat}$ that assigns to a finite set $V$ the category of all $k$-graph structures on the finite set $V$ and to a function $f \colon V \to V^\prime$ the corresponding functor $F(f) \colon F(V) \to F(V^\prime)$ between decoration categories. Both categories $\mathsf{FinSet}$ and $\mathbf{Cat}$ are symmetric monoidal and the pseudofunctor $F \colon \mathsf{FinSet} \to \mathbf{Cat}$ is symmetric lax monoidal, as given a $k$-graph structure on a finite set $V_1$ denoted by an element $K_1 \in F(V_1)$ and a $k$-graph structure on a finite set $V_2$ denoted by an element $K_2 \in F(V_2)$, we can consider the $k$-graph structures simultaneously as a single graph structure $\phi_{V_1,V_2}(K_1,K_2)$ on the finite set $V_1+V_2$. Thus we get a family of natural transformations $$\phi_{V_1,V_2} \colon F(V_1) \times F(V_2) \to F(V_1+V_2)$$ as well as a morphism $\phi \colon 1_{\mathsf{Graph}_k} \to F(\emptyset)$ which together satisfy the coherence conditions of a monoidal functor. The braiding is also clear as the following diagram commutes:
\[
\begin{tikzpicture}[scale=1.5]
\node (E) at (3,0) {$F(V_1) \times F(V_2)$};
\node (G) at (5,0) {$F(V_2) \times F(V_1)$};
\node (E') at (3,-1) {$F(V_1+V_2)$};
\node (G') at (5,-1) {$F(V_2+V_1)$};
\path[->,font=\scriptsize,>=angle 90]
(E) edge node[left]{$\phi_{V_1,V_2}$} (E')
(G) edge node[right]{$\phi_{V_2,V_1}$} (G')
(E) edge node[above]{$\beta_{V_1,V_2}$} (G)
(E') edge node[above]{$F(\beta_{V_1,V_2})$} (G');
\end{tikzpicture}
\]
Thus the pseudofunctor $F$ is symmetric lax monoidal and so by Theorem \ref{DC} we get a symmetric monoidal double category $F\mathbb{C}\textnormal{sp}$ which has:
\begin{enumerate}
\item{objects as finite sets,}
\item{vertical 1-morphisms as functions,}
\item{horizontal 1-cells as cospans of finite sets together with the structure of a $k$-graph given by an element of the image of the apex under the pseudofunctor $F$:
\[
\begin{tikzpicture}[scale=1.5]
\node (D) at (-3,0) {$U$};
\node (E) at (-2,0) {$V$};
\node (F) at (-1,0) {$W$};
\node (A) at (0,0) {$K \in F(V)$};
\path[->,font=\scriptsize,>=angle 90]
(D) edge node [above] {$i$} (E)
(F) edge node [above] {$o$} (E);
\end{tikzpicture}
\]
and}
\item{2-morphisms as maps of cospans of finite sets 
\[
\begin{tikzpicture}[scale=1.5]
\node (E) at (3,0) {$U_1$};
\node (F) at (5,0) {$W_1$};
\node (G) at (4,0) {$V_1$};
\node (E') at (3,-1) {$U_2$};
\node (F') at (5,-1) {$W_2$};
\node (G') at (4,-1) {$V_2$};
\node (A) at (6,0) {$K_1 \in F(V_1)$};
\node (B) at (6,-1) {$K_2 \in F(V_2)$};
\path[->,font=\scriptsize,>=angle 90]
(F) edge node[above]{$o_1$} (G)
(E) edge node[left]{$f$} (E')
(F) edge node[right]{$g$} (F')
(G) edge node[left]{$h$} (G')
(E) edge node[above]{$i_1$} (G)
(E') edge node[above]{$i_2$} (G')
(F') edge node[above]{$o_2$} (G');
\end{tikzpicture}
\]
together with a morphism of $k$-graphs $\iota \colon F(h)(K_1) \to K_2$ in $F(V_2)$.}
\end{enumerate}
These two double categories $_L \mathbb{C}\textnormal{sp}(\mathsf{Graph}_k)$ and $F\mathbb{C}\textnormal{sp}$ are equivalent by Theorem \ref{Equiv}.

\subsection{Markov processes}
In another previous work \cite{BFP}, Fong, Pollard and the first author used decorated cospans to construct a symmetric monoidal category of `open Markov processes'. In this framework, a Markov process on a finite set $N$ is given by a diagram in $\mathsf{Set}$:
\[
\begin{tikzpicture}[scale=1.5]
\node (D) at (3,0) {$(0,\infty)$};
\node (E) at (4,0) {$E$};
\node (F) at (5,0) {$N$};
\path[->,font=\scriptsize,>=angle 90]
(E) edge node [above] {$r$} (D)
(E) edge [bend left] node [above] {$s$} (F)
(E) edge [bend right] node [below] {$t$} (F);
\end{tikzpicture}
\]
where $E$ and $N$ are finite sets of edges and nodes, respectively. This is really just a special case of the previous example of passive linear circuits with $k = \mathbb{R}$. An open Markov process is then of course a cospan of finite sets where the apex is equipped with a Markov process:
\[
\begin{tikzpicture}[scale=1.5]
\node (B) at (1,0) {$N$};
\node (C) at (2,0) {$Y$};
\node (A) at (0,0) {$X$};
\node (D) at (3,0) {$(0,\infty)$};
\node (E) at (4,0) {$E$};
\node (F) at (5,0) {$N$};
\path[->,font=\scriptsize,>=angle 90]
(E) edge node [above] {$r$} (D)
(E) edge [bend left] node [above] {$s$} (F)
(E) edge [bend right] node [below] {$t$} (F)
(A)edge node[above] {$i$}(B)
(C)edge node[above]{$o$}(B);
\end{tikzpicture}
\]
For example:
\[
\begin{tikzpicture}[->,>=stealth',shorten >=1pt,auto,node distance=3.7cm,
thick,main node/.style={circle,fill=white!20,draw,font=\sffamily\Large\bfseries},terminal/.style={circle,fill=blue!20,draw,font=\sffamily\Large\bfseries}]]
\node[main node](1) {$b_1$};
\node[main node](2) [left=0.8cm of 1] {$a_1$};.5
\node[main node](3) [below right=0.7cm and 2.4cm of 1] {$c_2$};
\node[main node](5) [above right=0.7cm and 2.4cm of 1] {$c_1$};
\node[main node](4) [above right=0.7cm and 2.4cm of 3] {$d_1$};
\node(A) [left=1cm of 2,circle,draw,inner sep=1pt,fill=gray,color=purple] {};
\node(C) [right=1cm of 4,circle,draw,inner sep=1pt,fill=gray,color=purple] {};
\node[left=1.25 cm of 2,color=purple] {$S$};
\node[right=1.25 cm of 4,color=purple] {$T$};
\path[every node/.style={font=\sffamily\small}, shorten >=1pt]
(3) edge [bend right =15] node[below] {$e_6$} (4)
(5) edge [bend left =15] node[below] {$e_5$} (4)
(2) edge [bend left=15] node[below] {$e_1$} (1)
(1) edge [bend left =15] node[below] {$e_2$} (5)
(5) edge [bend right =15] node[right] {$e_4$} (3)
(1) edge [bend right =15] node[below] {$e_3$} (3)
(3) edge [bend right =15] node[above] {$6$} (4)
(5) edge [bend left =15] node[above] {$6$} (4)
(2) edge [bend left=15] node[above] {$5$} (1)
(1) edge [bend left =15] node[above] {$8$} (5)
(5) edge [bend right =15] node[left] {$4$} (3)
(1) edge [bend right =15] node[above] {$8$} (3);
\path[color=purple, very thick, shorten >=10pt, ->, >=stealth] (C) edge (4);
\path[color=purple, very thick, shorten >=10pt, ->, >=stealth] (A) edge (2);
\end{tikzpicture}
\]
Here we have an open Markov process on the finite set $N = \{a_1,b_1,c_1,c_2,d_1 \}$ with input and output sets given by the singletons $S$ and $T$, respectively.

Fong, Pollard and the first author then add extra structure to open Markov processes such as populations at each node to obtain a symmetric monoidal category $\mathsf{DetlBalMark}$ of open Markov processes in `detailed balance' and then construct a black box functor $\blacksquare \colon \mathsf{DetBalMark} \to \mathsf{LinRel}$ that describes the steady state behavior of an open Markov process in detailed balance. On the way to doing this, one of the categories they construct using Fong's decorated cospan machinery is a symmetric monoidal category $\mathsf{Mark}$ which has:
\begin{enumerate}
\item{objects as finite sets and}
\item{morphisms as isomorphism classes of open Markov processes, where composition is by pushout.}
\end{enumerate}
This is done using a symmetric lax monoidal functor $F \colon \mathsf{FinSet} \to \mathsf{Set}$ which assigns to each finite set $N$ the (large) set of all Markov processes on $N$ as defined above. Viewing this functor $F$ as now a symmetric lax monoidal pseudofunctor $F \colon \mathsf{FinSet} \to \mathbf{Cat}$ that assigns to a finite set $N$ the \emph{category} $F(N)$ of all Markov processes on $N$, we then get by Theorem \ref{DC} a symmetric monoidal double category $F\mathbb{C}\textnormal{sp}$ which has:
\begin{enumerate}
\item{finite sets as objects,}
\item{functions as vertical 1-morphisms,}
\item{open Markov processes as horizontal 1-cells, and}
\item{maps of open Markov processes as 2-morphisms which are given by maps of cospans of finite sets:
\[
\begin{tikzpicture}[scale=1.5]
\node (A) at (0,0) {$X_1$};
\node (A') at (0,-1) {$X_2$};
\node (C') at (2,-1) {$Y_2$};
\node (B) at (1,0) {$N_1$};
\node (C) at (2,0) {$Y_1$};
\node (D) at (1,-1) {$N_2$};
\node (E) at (3,0) {$M_1 \in F(N_1)$};
\node (E') at (3,-1) {$M_2 \in F(N_2)$};
\path[->,font=\scriptsize,>=angle 90]
(A) edge node[above]{$i_1$} (B)
(C) edge node[above]{$o_1$} (B)
(A) edge node[left]{$f$} (A')
(C) edge node[left]{$f^\prime$} (C')
(C') edge node [above] {$o_2$} (D)
(A') edge node [above] {$i_2$} (D)
(B) edge node [left] {$h$} (D);
\end{tikzpicture}
\]
together with a map of Markov processes $\iota \colon F(h)(M_1) \to M_2$ in $F(N_2)$, where a map between two Markov processes to be given by a pair of functions $(g,h)$ that make the following diagram commute:
\[
\begin{tikzpicture}[scale=1.5]
\node (E) at (3,0) {$(0,\infty)$};
\node (E') at (3,-1) {$(0,\infty)$};
\node (F) at (4,0) {$E_1$};
\node (F') at (4,-1) {$E_2$};
\node (G) at (5,0) {$N_1$};
\node (G') at (5,-1) {$N_2$};
\path[->,font=\scriptsize,>=angle 90]
(F) edge node[above] {$r_1$} (E)
(F) edge [bend left] node [above] {$s_1$} (G)
(F) edge [bend right] node [below] {$t_1$} (G)
(F') edge node[above] {$r_2$} (E')
(F') edge [bend left] node [above] {$s_2$} (G')
(F') edge [bend right] node [below] {$t_2$} (G')
(F) edge node [left] {$g$} (F')
(G) edge node [right] {$h$} (G');
\end{tikzpicture}
\]
}
\end{enumerate}
A symmetric monoidal double category of open Markov processes can also be obtained using structured cospans by defining a functor $L \colon \mathsf{FinSet} \to \mathsf{Mark}$ which assigns to a finite set $N$ the discrete Markov process $L(N)$ with no edges and to a function $f \colon N \to N^\prime$ the induced map of discrete Markov processes. Both categories $\mathsf{FinSet}$ and $\mathsf{Mark}$ have finite colimits and the functor $L$ is left adjoint to the forgetful functor $R \colon \mathsf{Mark} \to \mathsf{FinSet}$ which maps a Markov process to its underlying finite set of states. By Theorem \ref{SC}, we get a symmetric monoidal double category $_L \mathbb{C}\textnormal{sp}(\mathsf{Mark})$ which has:
\begin{enumerate}
\item{objects as finite sets,}
\item{vertical 1-morphisms as functions,}
\item{horizontal 1-cells as cospans in $\mathrm{Mark}$ of the form:
\[
\begin{tikzpicture}[scale=1.5]
\node (D) at (-3,0) {$L(N_1)$};
\node (E) at (-2,0) {$M$};
\node (F) at (-1,0) {$L(N_2)$};
\path[->,font=\scriptsize,>=angle 90]
(D) edge node [above] {$I$} (E)
(F) edge node [above] {$O$} (E);
\end{tikzpicture}
\]
and}
\item{2-morphisms as maps of cospans in $\mathrm{Mark}$.
\[
\begin{tikzpicture}[scale=1.5]
\node (E) at (3,0) {$L(N_1)$};
\node (F) at (5,0) {$L(N_2)$};
\node (G) at (4,0) {$M$};
\node (E') at (3,-1) {$L(N_1^\prime)$};
\node (F') at (5,-1) {$L(N_2^\prime)$};
\node (G') at (4,-1) {$M^\prime$};
\path[->,font=\scriptsize,>=angle 90]
(F) edge node[above]{$O$} (G)
(E) edge node[left]{$L(f_1)$} (E')
(F) edge node[right]{$L(f_2)$} (F')
(G) edge node[left]{$(g,h)$} (G')
(E) edge node[above]{$I$} (G)
(E') edge node[above]{$I^\prime$} (G')
(F') edge node[above]{$O^\prime$} (G');
\end{tikzpicture}
\]
}
\end{enumerate}
The two double categories $F\mathbb{C}\textnormal{sp}$ and $_L \mathbb{C} \textnormal{sp}(\mathsf{Mark})$ are equivalent by Theorem \ref{Equiv}.

In a more recent work \cite{BC}, the first two authors have constructed a symmetric monoidal double category of `open Markov processes' and `coarse-grainings', where roughly speaking, a coarse-graining is a way of approximating a larger open Markov process by a smaller one by partitioning the set of states into `lumps'. This construction uses neither decorated cospans nor structured cospans.

\subsection{Petri nets}
In a previous work, Master and the first author used structured cospans to obtain a symmetric monoidal double category of `open Petri nets'. A Petri net is given by a diagram in $\mathsf{Set}$ of the form:
\[
\begin{tikzpicture}[scale=1.5]
\node (B) at (1,0) {$T$};
\node (C) at (2,0) {$\mathbb{N}[S].$};
\path[->,font=\scriptsize,>=angle 90]
(B)edge[bend left] node[above]{$s$}(C)
(B)edge[bend right] node[below]{$t$}(C);
\end{tikzpicture}
\]
Here, $T$ is the set of \emph{transitions} and $S$ is the set of \emph{species}, and then $\mathbb{N}[S]$ is the free commutative monoid on the set $S$. Each transition then has a formal linear combination of species given by an element of $\mathbb{N}[S]$ as its source and target as prescribed by the functions $s$ and $t$, respectively. An example of a Petri net is given by:
\[
\begin{tikzpicture}
	\begin{pgfonlayer}{nodelayer}
		\node [style=species] (I) at (0,1) {H};
		\node [style=species] (T) at (0,-1) {O};
		\node [style=transition] (W) at (2,0) {$\alpha$};
		\node [style=species] (Water) at (4,0) {$\textnormal{H}_2$O};
%		\node [style=transition] (Something) at (6,0) {\tiny{Something}};
%		\node [style=species] (A) at (8,1) {O$\textnormal{H}^{-}$};
%		\node [style=species] (B) at (8,-1) {$\textnormal{H}_3 \textnormal{O}^{+}$};
	\end{pgfonlayer}
	\begin{pgfonlayer}{edgelayer}
		\draw [style=inarrow, bend right=40, looseness=1.00] (I) to (W);
		\draw [style=inarrow, bend left=40, looseness=1.00] (I) to (W);
		\draw [style=inarrow, bend right=40, looseness=1.00] (T) to (W);
		\draw [style=inarrow] (W) to (Water);
%		\draw [style=inarrow, bend left=40, looseness=1.00] (Water) to (Something);
%		\draw [style=inarrow, bend right=40, looseness=1.00] (Water) to (Something);
%		\draw [style=inarrow, bend left=40, looseness=1.00] (Something) to (A);
%		\draw [style=inarrow, bend right=40, looseness=1.00] (Something) to (B);
	\end{pgfonlayer}
\end{tikzpicture}
\]
This Petri net has a single transition $\alpha$ with $2\textnormal{H}+\textnormal{O}$ as its source and $\textnormal{H}_2 \textnormal{O}$ as its target. See the original paper for more details on Petri nets \cite{BM}.

Each set of species $S$ gives rise to a discrete Petri net $L(S)$ with $S$ as its set of species and no transitions. Master and the first author show the existence of a left adjoint $L \colon \mathsf{Set} \to \mathsf{Petri}$ where $\mathsf{Petri}$ is the category whose objects are Petri nets and whose `morphisms are morphisms of Petri nets'. They also show that $\mathsf{Petri}$ has finite colimits and thus using Theorem \ref{SC} obtain a symmetric monoidal double category $\mathbb{O}\mathbf{pen}(\mathrm{Petri})$ of open Petri nets which has:
\begin{enumerate}
\item{objects given by sets,}
\item{vertical 1-morphisms given by functions,}
\item{horizontal 1-cells as open Petri nets which are given by cospans in $\mathsf{Petri}$ of the form:
\[
\begin{tikzpicture}[scale=1.5]
\node (D) at (-3,0) {$L(X)$};
\node (E) at (-2,0) {$P$};
\node (F) at (-1,0) {$L(Y)$};
\path[->,font=\scriptsize,>=angle 90]
(D) edge node [above] {$I$} (E)
(F) edge node [above] {$O$} (E);
\end{tikzpicture}
\]
and}
\item{2-morphisms as maps of cospans in $\mathsf{Petri}$ of the form:
\[
\begin{tikzpicture}[scale=1.5]
\node (E) at (3,0) {$L(X_1)$};
\node (F) at (5,0) {$L(Y_1)$};
\node (G) at (4,0) {$P_1$};
\node (E') at (3,-1) {$L(X_2)$};
\node (F') at (5,-1) {$L(Y_2)$};
\node (G') at (4,-1) {$P_2$};
\path[->,font=\scriptsize,>=angle 90]
(F) edge node[above]{$O_1$} (G)
(E) edge node[left]{$L(f)$} (E')
(F) edge node[right]{$L(g)$} (F')
(G) edge node[left]{$\alpha$} (G')
(E) edge node[above]{$I_1$} (G)
(E') edge node[above]{$I_2$} (G')
(F') edge node[above]{$O_2$} (G');
\end{tikzpicture}
\]
}
\end{enumerate}
We can also obtain a similar double category using decorated cospans: define a pseudofunctor $F \colon \mathsf{Set} \to \mathbf{Cat}$ where given a set $s$, $F(s)$ is the category of all Petri net structures with $s$ as its set of species. This pseudofunctor $F$ is symmetric lax monoidal as both $(\mathsf{Set},+,\emptyset)$ and $(\mathbf{Cat},\times,1)$ are symmetric monoidal and given Petri nets $P \in F(s)$ and $P^\prime \in F(s^\prime)$, we can place them side by side and consider them together as a single Petri net $P+P^\prime \in F(s+s^\prime)$ with set of species $s+s^\prime$, and thus we have natural transformations $\phi_{s,s^\prime} \colon F(s) \times F(s^\prime) \to F(s+s^\prime)$ for any two sets $s$ and $s^\prime$. The other structure morphism between monoidal units $\phi \colon 1_{\mathsf{Petri}} \to F(\emptyset)$ is defined by the unique morphism from the empty Petri net with the empty set forits set of species to the only possible Petri net on the empty set, which is also the empty Petri net. All of the diagrams that are required to commute are straightforward. Using Theorem \ref{DC}, we obtain a symmetric monoidal double category $F\mathbb{C}\textnormal{sp}$ which has:
\begin{enumerate}
\item{objects given by sets,}
\item{vertical 1-morphisms given by functions,}
\item{horizontal 1-cells given by pairs:
\[
\begin{tikzpicture}[scale=1.5]
\node (D) at (-3,0) {$X$};
\node (E) at (-2,0) {$Z$};
\node (F) at (-1,0) {$Y$};
\node (A) at (0,0) {$P \in F(Z)$};
\path[->,font=\scriptsize,>=angle 90]
(D) edge node [above] {$i$} (E)
(F) edge node [above] {$o$} (E);
\end{tikzpicture}
\]
and}
\item{2-morphisms as maps of cospans in $\mathsf{Set}$:
\[
\begin{tikzpicture}[scale=1.5]
\node (E) at (3,0) {$X_1$};
\node (F) at (5,0) {$Y_1$};
\node (G) at (4,0) {$Z_1$};
\node (E') at (3,-1) {$X_2$};
\node (F') at (5,-1) {$Y_2$};
\node (G') at (4,-1) {$Z_2$};
\node (A) at (6,0) {$P_1 \in F(Z_1)$};
\node (B) at (6,-1) {$P_2 \in F(Z_2)$};
\path[->,font=\scriptsize,>=angle 90]
(F) edge node[above]{$o_1$} (G)
(E) edge node[left]{$f$} (E')
(F) edge node[right]{$g$} (F')
(G) edge node[left]{$h$} (G')
(E) edge node[above]{$i_1$} (G)
(E') edge node[above]{$i_2$} (G')
(F') edge node[above]{$o_2$} (G');
\end{tikzpicture}
\]
together with a morphism of Petri nets $\iota \colon F(h)(P_1) \to P_2$ in $F(Z_2)$.}
\end{enumerate}
Thus we have a symmetric monoidal double category $\mathbb{O}\mathbf{pen}(\mathrm{Petri})$ of open Petri nets obtained from structured cospans and a symmetric monoidal double category $F\mathbb{C}\textnormal{sp}$ of open Petri nets obtain from decorated cospans, and of course, we have an equivalence $\mathbb{O}\mathbf{pen}(\mathrm{Petri}) \sim F\mathbb{C}\textnormal{sp}$ of symmetric monoidal double categories by Theorem \ref{Equiv}.
\end{comment}


\begin{comment}
\section{Appendix}
Before formally defining `pseudo double category', it is helpful to have the following picture in mind. A pseudo double category has 2-morphisms shaped like:

\[
\begin{tikzpicture}[scale=1]
\node (D) at (-4,0.5) {$A$};
\node (E) at (-2,0.5) {$B$};
\node (F) at (-4,-1) {$C$};
\node (A) at (-2,-1) {$D$};
\node (B) at (-3,-0.25) {$\Downarrow a$};
\path[->,font=\scriptsize,>=angle 90]
(D) edge node [above]{$M$}(E)
(E) edge node [right]{$g$}(A)
(D) edge node [left]{$f$}(F)
(F) edge node [above]{$N$} (A);
\end{tikzpicture}
\]

We call $A, B, C$ and $D$ \textbf{objects} or \textbf{0-cells}, $f$ and $g$ \textbf{vertical 1-morphisms}, $M$ and $N$ \textbf{horizontal 1-cells} and $a$ a \textbf{2-morphism}. Note that a vertical 1-morphism is a morphism between 0-cells and a 2-morphism is a morphism between horizontal 1-cells. We will denote both kinds of morphisms and horizontal 1-cells as a single arrow, namely `$\to$'. We follow the notation of Shulman \cite{Shul} with the following definitions.

\begin{defn}
A \textbf{pseudo double category} $\lD$, or $\textbf{double category}$ for short, consists of a category of objects $\mathbf{D_{0}}$ and a category of arrows $\mathbf{D_{1}}$ with the following functors
\begin{center}
$U\colon \mathbf{D_{0}} \to \mathbf{D_{1}}$\\
$S,T \colon \mathbf{D_{1}} \rightrightarrows \mathbf{D_{0}}$\\
$\odot \colon \mathbf{D_{1}} \times_{\mathbf{D_{0}}} \mathbf{D_{1}} \to \mathbf{D_{1}}$ (where the pullback is taken over $\mathbf{D_{1}} \xrightarrow[]{T} \mathbf{D_{0}} \xleftarrow[]{S} \mathbf{D_{1}}$) \\
\end{center}
 such that \\
\begin{center}
$S(U_{A})=A=T(U_{A})$\\
$S(M \odot N)=SN$\\
$T(M \odot N)=TM$\\
\end{center}
equipped with natural isomorphisms
\begin{center}

$\alpha \colon (M \odot N) \odot P \xrightarrow{\sim} M \odot (N \odot P)$\\
$\lambda \colon U_{B} \odot M \xrightarrow{\sim} M$\\
$\rho \colon M \odot U_{A} \xrightarrow{\sim} M$

\end{center}
such that $S(\alpha), S(\lambda), S(\rho), T(\alpha), T(\lambda)$ and $T(\rho)$ are all identities and that the coherence axioms of a monoidal category are satisfied. Following the notation of Shulman, objects of $\mathbf{D_{0}}$ are called $\textbf{0-cells}$ and morphisms of $\mathbf{D_{0}}$ are called $\textbf{vertical 1-morphisms}$. Objects of $\mathbf{D_{1}}$ are called $\textbf{horizontal 1-cells}$ and morphisms of $\mathbf{D_{1}}$ are called $\textbf{2-morphisms}$. The morphisms of $\mathbf{D_{0}}$, which are vertical 1-morphisms, will be denoted $f \colon A \to C$ and we denote a 1-cell $M$ with $S(M)=A,T(M)=B$ by $M \colon A \to B$. Then a 2-morphism $a \colon M \to N$ of $\mathbf{D_{1}}$ with $S(a)=f,T(a)=g$ would look like:
\[
\begin{tikzpicture}[scale=1]
\node (D) at (-4,0.5) {$A$};
\node (E) at (-2,0.5) {$B$};
\node (F) at (-4,-1) {$C$};
\node (A) at (-2,-1) {$D$};
\node (B) at (-3,-0.25) {$\Downarrow a$};
\path[->,font=\scriptsize,>=angle 90]
(D) edge node [above]{$M$}(E)
(E) edge node [right]{$g$}(A)
(D) edge node [left]{$f$}(F)
(F) edge node [above]{$N$} (A);
\end{tikzpicture}
\]
\end{defn}

The key difference between a `strict' double category and a pseudo double category is that in a pseudo double category, horizontal composition is associative and unital only up to natural isomorphism. Equivalently, as a double category can be viewed as a category internal to $\mathbf{Cat}$, we can view a pseudo double category as a category `weakly' internal to $\mathbf{Cat}$. We will sometimes omit the word pseudo and simply say double category.

\begin{defn}
A 2-morphism where $f$ and $g$ are identities is called a \textbf{globular 2-morphism}.
\end{defn}

\begin{defn}
Let $\lD$ be a pseudo double category. Then the $\textbf{horizontal bicategory}$ of $\lD$, which we denote as $H(\lD)$, is the bicategory consisting of objects of $\lD$, morphisms that are horizontal 1-cells of $\lD$ and 2-morphisms that are globular 2-morphisms of $\lD$.
\end{defn}

\begin{defn}
  A \textbf{monoidal double category} is a double category equipped the following
structure.
\begin{enumerate}
\item $\mathbf{D_{0}}$ and $\mathbf{D_{1}}$ are both monoidal categories.
\item If $I$ is the monoidal unit of $\mathbf{D_{0}}$, then $U_I$ is the
  monoidal unit of $\mathbf{D_{1}}$.
\item The functors $S$ and $T$ are strict monoidal, i.e.\ $S(M\ten N)
  = SM\ten SN$ and $T(M\ten N)=TM\ten TN$ and $S$ and $T$ also
  preserve the associativity and unit constraints.
\item We have globular isomorphisms
  \[\chi \maps (M_1\ten N_1)\odot (M_2\ten N_2)\too[\sim] (M_1\odot M_2)\ten (N_1\odot N_2)\]
  and
  \[\mu\maps U_{A\ten B} \too[\sim] (U_A \ten U_B)\]
  such that the following diagrams commute:
		\item \label{diag:MonDblCat}
			The following diagrams commute expressing the constraint data for the double functor $\otimes$.
			\[
			\begin{tikzpicture}
				\node (A) at (0,3) {\footnotesize{
							$((M_1\otimes N_1)\odot (M_2\otimes N_2)) \odot (M_3\otimes N_3)$}
				};
				\node (B) at (7,3) {\footnotesize{
						$((M_1\odot M_2)\otimes (N_1\odot N_2)) \odot (M_3\otimes N_3) $}
				};
				\node (A') at (0,1.5) {\footnotesize{
						$(M_1\otimes N_1)\odot ((M_2\otimes N_2) \odot (M_3\otimes N_3)) $}
				};
				\node (B') at (7,1.5) {\footnotesize{
						$((M_1\odot M_2)\odot M_3) \otimes ((N_1\odot N_2)\odot N_3)$}
				};
				\node (A'') at (0,0) {\footnotesize{
						$(M_1\otimes N_1) \odot ((M_2\odot M_3) \otimes (N_2\odot N_3))$}
				};
				\node (B'') at (7,0) {\footnotesize{
						$(M_1\odot (M_2\odot M_3)) \otimes (N_1\odot (N_2\odot N_3))$}
				};
			%
			\path[->,font=\scriptsize]
				(A) edge node[left]{$\alpha$} (A')
				(A') edge node[left]{$1 \odot \chi$} (A'')
				(B) edge node[right]{$\chi$} (B')
				(B') edge node[right]{$\alpha \otimes \alpha$} (B'')
				(A) edge node[above]{$\chi \odot 1$} (B)
				(A'') edge node[above]{$\chi$} (B'');
		\end{tikzpicture}
		\]
		\[
		\begin{tikzpicture}
			\node (UL) at (0,1.5) {\footnotesize{
					$(M\otimes N) \odot U_{C\otimes D}$}
			};
			\node (LL) at (0,0) {\footnotesize{
					$M\otimes N$}
			};
			\node (UR) at (3.5,1.5) {\footnotesize{
					$(M\otimes N)\odot (U_C\otimes U_D)$}
			};
			\node (LR) at (3.5,0) {\footnotesize{
					$(M\odot U_C) \otimes (N\odot U_D)$}
			};
			%
			\path[->,font=\scriptsize]
				(UL) edge node[above]{$1 \odot \mu$} (UR) 
				(UL) edge node[left]{$\rho$} (LL)
				(LR) edge node[above]{$\rho \otimes \rho$} (LL)
				(UR) edge node[right]{$\chi$} (LR);
		\end{tikzpicture}
		%
		\quad
		%
		\begin{tikzpicture}
			\node (UL) at (0,1.5) {\scriptsize{$U_{A\otimes B}\odot (M\otimes N)$}};
			\node (LL) at (0,0) {\scriptsize{$M\otimes N$}};
			\node (UR) at (3.5,1.5) {\scriptsize{$(U_A\otimes U_B)\odot (M\otimes N)$}};
			\node (LR) at (3.5,0) {\scriptsize{$(U_A \odot M) \otimes (U_B\odot N)$}};
			%
			\path[->,font=\scriptsize]
				(UL) edge node[above]{$\chi \odot 1$} (UR) 
				(UL) edge node[left]{$\lambda$} (LL)
				(LR) edge node[above]{$\lambda \otimes \lambda$} (LL)
				(UR) edge node[right]{$\chi$} (LR);
		\end{tikzpicture}
		\]
		%
		\item The following diagrams commute expressing 
		the associativity isomorphism for $\otimes$ is a transformation of double categories.
		\[
		\begin{tikzpicture}
			\node (A) at (0,3) {\footnotesize{
					$((M_1\otimes N_1)\otimes P_1) \odot ((M_2\otimes N_2)\otimes P_2)$}
			};
			\node (B) at (7,3) {\footnotesize{
					$(M_1\otimes (N_1\otimes P_1)) \odot (M_2\otimes (N_2\otimes P_2))$}
			};
			\node (A') at (0,1.5) {\footnotesize{
					$((M_1\otimes N_1) \odot (M_2\otimes N_2)) \otimes (P_1\odot P_2)$}
			};
			\node (B') at (7,1.5) {\footnotesize{
					$(M_1\odot M_2) \otimes ((N_1\otimes P_1)\odot (N_2\otimes P_2))$}
			};
			\node (A'') at (0,0) {\footnotesize{
					$((M_1\odot M_2) \otimes(N_1\odot N_2)) \otimes (P_1\odot P_2)$}
			};
			\node (B'') at (7,0) {\footnotesize{
					$(M_1\odot M_2) \otimes ((N_1\odot N_2)\otimes (P_1\odot P_2))$}
			};
			%
			\path[->,font=\scriptsize]
				(A) edge node[left]{$\chi$} (A')
				(A') edge node[left]{$\chi \otimes 1$} (A'')
				(B) edge node[right]{$\chi$} (B')
				(B') edge node[right]{$1 \otimes \chi$} (B'')
				(A) edge node[above]{$\alpha \odot \alpha$} (B)
				(A'') edge node[above]{$\alpha$} (B'');
		\end{tikzpicture}
		\]
		\[
		\begin{tikzpicture}
			\node (A) at (0,3) {\footnotesize{$U_{(A\otimes B)\otimes C}$}};
			\node (B) at (4,3) {\footnotesize{$U_{A\otimes (B\otimes C)} $}};
			\node (A') at (0,1.5) {\footnotesize{$U_{A\otimes B} \otimes U_C $}};
			\node (B') at (4,1.5) {\footnotesize{$U_A\otimes U_{B\otimes C}$}};
			\node (A'') at (0,0) {\footnotesize{$(U_A\otimes U_B)\otimes U_C$}};
			\node (B'') at (4,0) {\footnotesize{$U_A\otimes (U_B\otimes U_C) $}};
			%
			\path[->,font=\scriptsize]
				(A) edge node[left]{$\mu$} (A')
				(A') edge node[left]{$\mu \otimes 1$} (A'')
				(B) edge node[right]{$\mu$} (B')
				(B') edge node[right]{$1 \otimes \mu$} (B'')
				(A) edge node[above]{$U_{\alpha}$} (B)
				(A'') edge node[above]{$\alpha$} (B'');
		\end{tikzpicture}
		\]
		\item The following diagrams commute expressing that 
		the unit isomorphisms for $\otimes$ are transformations of double categories. 
		\[
		\begin{tikzpicture}
			\node (A) at (0,1.5) {\footnotesize{$(M\otimes U_I)\odot (N\otimes U_I)$}};
			\node (A') at (0,0) {\footnotesize{$M\odot N $}};
			\node (B) at (4,1.5) {\footnotesize{$(M\odot N)\otimes (U_I \odot U_I) $}};
			\node (B') at (4,0) {\footnotesize{$(M\odot N)\otimes U_I $}};
			%
			\path[->,font=\scriptsize]
				(A) edge node[left]{$r \odot r$} (A')
				(A) edge node[above]{$\chi$} (B)
				(B) edge node[right]{$1 \otimes \rho$} (B')
				(B') edge node[above]{$r$} (A');
		\end{tikzpicture}
		%
		\quad
		%
		\begin{tikzpicture}
			\node (A) at (0,0.75) {\footnotesize{$U_{A\otimes I} $}};
			\node (B) at (1.5,1.5) {\footnotesize{$U_A\otimes U_I $}};
			\node (B') at (1.5,0) {\footnotesize{$U_A$}};
			%
			\path[->,font=\scriptsize]
				(A) edge node[above]{$\mu$} (B)
				(A) edge node[below]{$U_{r}$} (B')
				(B) edge node[right]{$r$} (B');
		\end{tikzpicture}
		\]
		%
		%
		%
		%
		\[
		\begin{tikzpicture}
			\node (A) at (0,1.5) {\footnotesize{$(U_I\otimes M)\odot (U_I\otimes N)$}};
			\node (A') at (0,0) {\footnotesize{$M\odot N$}};
			\node (B) at (4,1.5) {\footnotesize{$(U_I \odot U_I) \otimes (M\odot N)$}};
			\node (B') at (4,0) {\footnotesize{$U_I\otimes (M\odot N) $}};
			%
			\path[->,font=\scriptsize]
				(A) edge node[left]{$\ell \odot \ell$} (A')
				(A) edge node[above]{$\chi$} (B)
				(B) edge node[right]{$\lambda \otimes 1$} (B')
				(B') edge node[above]{$\ell$} (A');
		\end{tikzpicture}
		%
		\quad
		\begin{tikzpicture}
			\node (A) at (0,0.75) {\footnotesize{$U_{I\otimes A}$}};
			\node (B) at (1.5,1.5) {\footnotesize{$U_I\otimes U_A$}};
			\node (B') at (1.5,0) {\footnotesize{$U_A$}};
			%
			\path[->,font=\scriptsize]
				(A) edge node[above]{$\mu$} (B)
				(A) edge node[below]{$U_{\ell}$} (B')
				(B) edge node[right]{$\ell$} (B');
		\end{tikzpicture}
		\]
		\newcounter{mondbl}
		\setcounter{mondbl}{\value{enumi}}
	\end{enumerate}
	A \textbf{braided monoidal double category} 
	is a monoidal double category 
	such that:
	\begin{enumerate}
		\setcounter{enumi}{\value{mondbl}}
		\item $\dblcat{D}_{0}$ and $\dblcat{D}_{1}$ are braided monoidal categories.
		\item The functors $S$ and $T$ are strict braided monoidal functors.
		\item The following diagrams commute expressing that the braiding is a transformation of double categories.
		\[
		\begin{tikzpicture}
			\node (A) at (0,1.5) {\footnotesize{$(M_1 \odot M_2) \otimes (N_1 \odot N_2)$}};
			\node (A') at (0,0) {\footnotesize{$(M_1\otimes N_1) \odot (M_2\otimes N_2)$}};
			\node (B) at (5,1.5) {\footnotesize{$(N_1\odot N_2) \otimes (M_1 \odot M_2)$}};
			\node (B') at (5,0) {\footnotesize{$(N_1 \otimes M_1) \odot (N_2 \otimes M_2)$}};
			%
			\path[->,font=\scriptsize]
				(A) edge node[left]{$\chi$} (A')
				(A) edge node[above]{$\beta$} (B)
				(B) edge node[right]{$\chi$} (B')
				(A') edge node[above]{$\beta \odot \beta$} (B');
		\end{tikzpicture}
		%
		\quad
		%
		\begin{tikzpicture}
			\node (A) at (0,1.5) {\footnotesize{$U_A \otimes U_B$}};
			\node (A') at (0,0) {\footnotesize{$U_B\otimes U_A$}};
			\node (B) at (2,1.5) {\footnotesize{$U_{A\otimes B} $}};
			\node (B') at (2,0) {\footnotesize{$U_{B\otimes A}$}};
			%
			\path[->,font=\scriptsize]
				(A) edge node[left]{$\beta$} (A')
				(B) edge node[above]{$\mu$} (A)
				(B) edge node[right]{$U_\beta$} (B')
				(B') edge node[above]{$\mu$} (A');
		\end{tikzpicture}
		\]
		\setcounter{mondbl}{\value{enumi}}
	\end{enumerate}
	Finally, a \textbf{symmetric monoidal double category} 
	is a braided monoidal double category $\mathbb{D}$ such that:
	\begin{enumerate}
		\setcounter{enumi}{\value{mondbl}}
		\item $\dblcat{D}_{0}$ and $\dblcat{D}_{1}$ are symmetric monoidal.
	\end{enumerate}
\end{defn}


\begin{defn}\label{def:companion}
  Let \lD\ be a double category and $f\maps A\to B$ a vertical
  1-morphism.  A \textbf{companion} of $f$ is a horizontal 1-cell
  $\fhat\maps A\to B$ together with 2-morphisms
	\[
	\raisebox{-0.5\height}{
	\begin{tikzpicture}
		\node (A) at (0,1) {$A$};
		\node (B) at (1,1) {$B$};
		\node (A') at (0,0) {$B$};
		\node (B') at (1,0) {$B$};
		%
		\path[->,font=\scriptsize,>=angle 90]
			(A) edge node[above]{$\widehat{f}$} (B)
			(A) edge node[left]{$f$} (A')
			(B) edge node[right]{$1$} (B')
			(A') edge node[below]{$U_B$} (B');
		%
	%	\draw (0.5,.925) -- (0.5,1.075);
	%	\draw (0.5,-.075) -- (0.5,.075);
		\node () at (0.5,0.5) {\scriptsize{$\Downarrow$}};
	\end{tikzpicture}
	}
	%
	\quad \text{ and } \quad
	%
	\raisebox{-0.5\height}{
	\begin{tikzpicture}
		\node (A) at (0,1) {$A$};
		\node (B) at (1,1) {$A$};
		\node (A') at (0,0) {$A$};
		\node (B') at (1,0) {$B$};
		%
		\path[->,font=\scriptsize,>=angle 90]
			(A) edge node[above]{$U_A$} (B)
			(A) edge node[left]{$1$} (A')
			(B) edge node[right]{$f$} (B')
			(A') edge node[below]{ $\widehat{f}$} (B');
		%
	%	\draw (0.5,.925) -- (0.5,1.075);
	%	\draw (0.5,-.075) -- (0.5,.075);
		\node () at (0.5,0.5) {\scriptsize{$\Downarrow$}};
	\end{tikzpicture}
	}
	\]
  such that the following equations hold.
	\begin{equation}
	\label{eq:CompanionEq}
	\raisebox{-0.5\height}{
	\begin{tikzpicture}
		\node (A) at (0,2) {$A$};
		\node (B) at (1.1,2) {$A$};
		\node (A') at (0,1) {$A$};
		\node (B') at (1.1,1) {$B$};
		\node (A'') at (0,0) {$B$};
		\node (B'') at (1.1,0) {$B$};
		%
		\path[->,font=\scriptsize,>=angle 90]
			(A) edge node[left]{$1$} (A')
			(A') edge node[left]{$f$} (A'')
			(B) edge node[right]{$f$} (B')
			(B') edge node[right]{$1$} (B'')
			(A) edge node[above]{$U_A$} (B)
			(A') edge  (B')
			(A'') edge node[below]{$U_B$} (B'');
		%
	%	\draw (0.5,1.925) -- (0.5,2.075);
		\draw[line width=2mm,white] (0.5,.925) -- (0.5,1.075);
	%	\draw (0.5,-.075) -- (0.5,.075);
		\node () at (0.5,0.5) {\scriptsize{$\Downarrow$}};
		\node () at (0.5,1.5) {\scriptsize{$\Downarrow$}};
		\node () at (0.5,1) {\scriptsize $\widehat{f}$};
	\end{tikzpicture}
	}
	%
	\raisebox{-0.5\height}{=}
	%
	\raisebox{-0.5\height}{
	\begin{tikzpicture}
		\node (A) at (0,1) {$A$};
		\node (B) at (1,1) {$A$};
		\node (A') at (0,0) {$B$};
		\node (B') at (1,0) {$B$};
		%
		\path[->,font=\scriptsize,>=angle 90]
		(A) edge node[left]{$f$} (A')
		(B) edge node[right]{$f$} (B')
		(A) edge node[above]{$U_A$} (B)
		(A') edge node[below]{$U_B$} (B');
		%
		%\draw (0.5,.925) -- (0.5,1.075);
		%\draw (0.5,-.075) -- (0.5,.075);
		\node () at (0.5,0.5) {\scriptsize{$\Downarrow U_f$}};
	\end{tikzpicture}
	}
	%
	\raisebox{-0.5\height}{\text{   and   }}
	%
	\raisebox{-0.5\height}{
	\begin{tikzpicture}
		\node (A) at (0,1) {$A$};
		\node (A') at (0,0) {$A$};
		\node (B) at (1,1) {$A$};
		\node (B') at (1,0) {$B$};
		\node (C) at (2,1) {$B$};
		\node (C') at (2,0) {$B$};
		%
		\path[->,font=\scriptsize,>=angle 90]
			(A) edge node[left]{$1$} (A')
			(B) edge node[left]{$f$} (B')
			(C) edge node[right]{$1$} (C')
			(A) edge node[above]{$U_A$} (B)
			(B) edge node[above]{$\widehat{f}$} (C)
			(A') edge node[below]{$\widehat{f}$} (B')
			(B') edge node[below]{$U_B$} (C');
		%
	%	\draw (1.5,0.925) -- (1.5,1.075);
	%	\draw (1.5,0.925) -- (1.5,1.075);
	%	\draw (0.5,.925) -- (0.5,1.075);
	%	\draw (0.5,-.075) -- (0.5,.075);
		\node () at (0.5,0.5) {\scriptsize{$\Downarrow$}};
		\node () at (1.5,0.5) {\scriptsize{$\Downarrow$}};
	\end{tikzpicture}
	}
	%
	\raisebox{-0.5\height}{=}
	%
	\raisebox{-0.5\height}{
	\begin{tikzpicture}
		\node (A) at (0,1) {$A$};
		\node (B) at (1,1) {$B$};
		\node (A') at (0,0) {$A$};
		\node (B') at (1,0) {$B$};
		%
		\path[->,font=\scriptsize,>=angle 90]
			(A) edge node[left]{$1$} (A')
			(B) edge node[right]{$1$} (B')
			(A) edge node[above]{$\widehat{f}$} (B)
			(A') edge node[below]{$\widehat{f}$} (B');
		%
	%	\draw (0.5,.925) -- (0.5,1.075);
	%	\draw (0.5,-.075) -- (0.5,.075);
		\node () at (0.5,0.5) {\scriptsize{$\Downarrow \id_{\widehat{f}}$}};
	\end{tikzpicture}
	}
	\end{equation}
  A \textbf{conjoint} of $f$, denoted $\fchk \maps B\to A$, is a
  companion of $f$ in the double category $\lD^{h\cdot\mathrm{op}}$
  obtained by reversing the horizontal 1-cells, but not the vertical
  1-morphisms, of \lD.
\end{defn}
\noindent
In a pseudo double category, the second equation above requires an insertion of unit isomorphisms to make sense due to horizontal composition only holding up to isomorphism.
\begin{defn}
  We say that a double category is \textbf{fibrant} if every vertical
  1-morphism has both a companion and a conjoint and \define{isofibrant} if every vertical 1-isomorphism has both a companion and a conjoint.
\end{defn}
\end{comment}

\begin{thebibliography}{100}

\bibitem{BC} J.\ C.\ Baez and K.\ Courser, Coarse-graining open Markov processes. Available as \href{https://arxiv.org/abs/1710.11343}{arXiv:1710.11343}.

\bibitem{BC2} J.\ C.\ Baez and K.\ Courser, Structured cospans. In preparation.

\bibitem{BCR} J.\ C.\ Baez, B.\ Coya and F.\ Rebro, Props in circuit theory. In preparation. THIS IS FINISHED!!!

\bibitem{BF} J.\ C.\ Baez and B.\ Fong, A compositional framework for passive linear networks. Available as \href{http://arxiv.org/abs/1504.05625}{arXiv:1504.05625}.

\bibitem{BFP} J.\ C.\ Baez, B.\ Fong and B.\ Pollard, A compositional framework for Markov processes, \textsl{Jour. Math. Phys.} \textbf{57} (2016), 033301. Available as \href{http://arxiv.org/abs/1508.06448}{arXiv:1508.06448}.

\bibitem{BM} J.\ C. \ Baez and J.\ Master, Open Petri nets. Available as \href{https://arxiv.org/abs/1808.05415}{arXiv:1808.05415}.

\bibitem{BP} J.\ C.\ Baez and B.\ Pollard, A compositional framework for chemical reaction networks, in preparation.

\bibitem{Borc} F. \ Borceux, Handbook of categorical algebra. 2, volume 51 of Encyclopedia of Mathematics and its Applications,
Cambridge University Press, 1994

\bibitem{Brown1} R.\ Brown and C.\ B.\ Spencer, Double groupoids and crossed modules, 
\textsl{Cah.\ Top.\ G\'eom.\ Diff.} \textbf{17} (1976), 343--362.

\bibitem{Brown2} R.\ Brown, K.\ Hardie, H.\ Kamps and T.\ Porter, The homotopy double groupoid of a Hausdorff space, \textsl{Th.\ Appl.\ Categ.} \textbf{10} (2002), 71--93.

%\bibitem{Be} J.\ B\'enabou, Introduction to bicategories, in {\sl Reports
%of the Midwest Category Seminar}, Lecture Notes in Mathematics, vol.\ \textbf{47}, Springer, Berlin, 1967, pp.\ 1--77.

%\bibitem{Brown1} R.\ Brown and C.\ B.\ Spencer, Double groupoids and crossed modules, 
%\textsl{Cah.\ Top.\ G\'eom.\ Diff.} \textbf{17} (1976), 343--362.

%\bibitem{Brown2} R.\ Brown, K.\ Hardie, H.\ Kamps and T.\ Porter, The homotopy double groupoid of a Hausdorff space, \textsl{Th.\ Appl.\ Categ.} \textbf{10} (2002), 71--93. 

%\bibitem{CC} D.\ Cicala and K.\ Courser, Spans of cospans in a topos. Available as \href{https://arxiv.org/abs/1707.02098}{arXiv:1707.02098}.

\bibitem{CV} D.\ Cicala and C.\ Vasilakopoulou, On Adjoints and Fibrations. In preparation.

\bibitem{Cour} K.\ Courser, A bicategory of decorated cospans, \emph{Theor.\ Appl.\ Cat.} \textbf{32} (2017), 995--1027. Also available as \href{https://arxiv.org/abs/1605.08100}{arXiv:1605.08100}.

%\bibitem{Ehresmann63} C.\ Ehresmann, Cat\'egories structur\'ees III: Quintettes et applications covariantes,  \textsl{Cah.\ Top.\ G\'eom.\ Diff.} \textbf{5} (1963), 1--22.

%\bibitem{Ehresmann65} C.\ Ehresmann, {\sl Cat\'egories et Structures,} Dunod, Paris, 1965.

\bibitem{Ehresmann63} C.\ Ehresmann, Cat\'egories structur\'ees III: Quintettes et applications covariantes,  \textsl{Cah.\ Top.\ G\'eom.\ Diff.} \textbf{5} (1963), 1--22.

\bibitem{Ehresmann65} C.\ Ehresmann, {\sl Cat\'egories et Structures,} Dunod, Paris, 1965.

\bibitem{Fong} B.\ Fong, Decorated cospans, \emph{Theor.\ Appl.\ Cat.} \textbf{30} (2015), 1096--1120. Also available as \href{http://arxiv.org/abs/1502.00872}{arXiv:1502.00872}.

%\bibitem{GP1} M.\ Grandis and R.\ Par\'e, Limits in double categories, \textsl{Cah.\ Top.\ G\'eom.\ Diff.} \textbf{40} (1999), 162--220.

%\bibitem{GP2} M.\ Grandis and R.\ Par\'e, Adjoints for double categories, 
% \textsl{Cah.\ Top.\ G\'eom.\ Diff.} \textbf{45} (2004), 193--240.

\bibitem{Gray} J. \ Gray, Fibred and Cofibred categories, Proc. Conf. Categorical Algebra (La Jolla, California, 1965), 21---83, New York, 1966

%\bibitem{Haug} R.\ Haugseng, Iterated spans and ``classical" topological field theories. Available as \href{https://arxiv.org/abs/1409.0837}{arXiv:1409.0837}.

\bibitem{Herm} C. \ Hermida, Fibrations, logical predicates and indeterminates, PhD Thesis, University of Edinburgh, 1993.

%\bibitem{Hoff} A.\ Hoffnung, Spans in 2-categories: a monoidal tricategory. Available as \href{http://arxiv.org/abs/1112.0560}{arXiv:1112.0560}.

\bibitem{JM} J.\ C.\ Baez and J.\ Master, A compositional framework for Petri nets. In preparation.

\bibitem{LS} E.\ Lerman and D.\ Spivak, An algebra of open continuous time dynamical systems and networks. Available as \href{http://arxiv.org/abs/1602.01017}{arXiv:1602.01017}.

%\bibitem{Lack} S.\ Lack, Limits for lax morphisms, \emph{Applied Categorical Structures} $\mathbf{30}$ (2005), 189--203. Available %at \href{http://maths.mq.edu.au/~slack/papers/talgl.pdf}{http://maths.mq.edu.au/$\sim$slack/papers/talgl.pdf}.

   %\bibitem{Lerm} E.\ Lerman and D.\ Spivak, An algebra of open continuous time dynamical systems and networks. Available as %%\href{http://arxiv.org/abs/1602.01017}{arXiv:1602.01017}.

%   \bibitem{ML} S.\ Mac Lane, {\sl Categories for the Working Mathematician},
%     Springer, Berlin, 1998.

\bibitem{MV} J.\ Moeller and C.\ Vasilakopoulou, Monoidal Grothendieck construction. Available as \href{https://arxiv.org/abs/1809.00727}{arXiv:1809.00727}.

%\bibitem{Pol} B.\ Pollard, Open Markov processes: A compositional perspective on non-equilibrium steady states in biology, %%%\emph{Entropy} $\mathbf{18}$ (2016), 140. Available as \href{http://arxiv.org/abs/1601.00711}{arXiv:1601.00711}.

%\bibitem{Nie}
%S.~Niefield,
%Span, cospan, and other double categories.
%\textsl{Theory Appl.\ Categ.}
%\textbf{26} (2012), 729--742.
%Available as \href{https://arxiv.org/abs/1201.3789}{arXiv:1201.3789}.

%\bibitem{Panan} F.\ Clerc, H.\ Humphrey and P.\ Panangaden, Bicategories of Markov processes,  to appear.

%\bibitem{RSW} R.\ Rosebrugh, N.\ Sabadini and R.\ F.\ C.\ Walters, Generic commutative separable algebras and cospans of graphs, \textsl{Theory Appl. Categ.} \textbf{15} (2005), 164--177. Available at \href{http:/.www.tac.mta.ca/tac/volumes/15/6/15-06.pdf}{http:/.www.tac.mta.ca/tac/volumes/15/6/15-06.pdf}.

%\bibitem{Reb} F.\ Rebro, Constructing the bicategory Span$_{2}(\mathrm{A})$. Available as \href{https://arxiv.org/abs/1501.00792}{arXiv:1501.00792}.

\bibitem{Shul} M.\ Shulman, Constructing symmetric monoidal bicategories. Available as \href{http://arxiv.org/abs/1004.0993}{arXiv:1004.0993}.

\bibitem{Shul2} M.\ Shulman, Framed bicategories and monoidal fibrations. Available as \href{https://arxiv.org/abs/0706.1286}{arXiv:0706.1286}.

%\bibitem{Stay} M.\ Stay, Compact closed bicategories.  Available as \href{http://arxiv.org/abs/1301.1053}{arXiv:1301.1053}.

\bibitem{Yass} A.\ Yassine, Open Systems in Classical Mechanics. Available as \href{https://arxiv.org/abs/1710.11392}{arXiv:1710.11392}.

\end{thebibliography}
\end{document}
