% An equivalence of compositional frameworks
% John Baez, Kenny Courser and Christina Vasilakopoulou
% 2019/09/11 - KC
\documentclass{amsart}
\usepackage{amssymb,amsmath,stmaryrd,txfonts,mathrsfs,amsthm}

\usepackage[all,2cell]{xy}\UseAllTwocells\SilentMatrices
\usepackage[neveradjust]{paralist}
\usepackage{hyperref}
\usepackage{mathtools}
\usepackage{multirow}
\usepackage[outline]{contour}
\contourlength{1.2pt}
\usepackage{tikz}
\usepackage{tikz-cd}
\usepackage{xcolor}
\usepackage{framed,color}
\usepackage[draft]{fixme}
\usetikzlibrary{matrix,arrows,decorations.pathmorphing,positioning}
\usetikzlibrary{intersections,decorations.markings}
\usetikzlibrary{arrows,positioning,fit,matrix,shapes.geometric,external}
\usetikzlibrary{backgrounds,circuits,circuits.ee.IEC,shapes,fit,matrix}
\usepackage{tikz}
\usetikzlibrary{matrix,arrows}
\usepackage{comment}
\usepackage[capitalize]{cleveref}
\definecolor{rewritecolor}{rgb}{0,.9,1}
\tikzset{rewritenode/.style={shape=circle,fill=rewritecolor,scale=0.25,font=\Huge}}
\tikzset{RWopen/.style={shape=circle,draw=black,fill=white,scale=0.5,font=\Huge}}
\tikzset{RWclosed/.style={shape=circle,fill=black,scale=0.5,font=\Huge}}
\tikzset{CDnode/.style={shape=circle,fill=white,scale=.5}}
\makeatletter
\let\ea\expandafter

\pgfdeclarelayer{edgelayer}
\pgfdeclarelayer{nodelayer}
\pgfsetlayers{edgelayer,nodelayer,main}

% Petri nets
\definecolor{lblue}{rgb}{0,250,255}
\tikzstyle{species}=[circle,fill=yellow,draw=black,scale=1.15]
\tikzstyle{transition}=[rectangle,fill=lblue,draw=black,scale=1.15]
\tikzstyle{inarrow}=[->, >=stealth, shorten >=.03cm,line width=1.5]
\tikzstyle{empty}=[circle,fill=none, draw=none]
\tikzstyle{inputdot}=[circle,fill=purple,draw=purple, scale=.25]
\tikzstyle{inputarrow}=[->,draw=purple, shorten >=.05cm]
\tikzstyle{simple}=[-,draw=purple,line width=1.000]
\tikzstyle{none}=[inner sep=0pt]

\definecolor{shadecolor}{rgb}{1,0.8,0.3}
\definecolor{myurlcolor}{rgb}{0.6,0,0}
\definecolor{mycitecolor}{rgb}{0,0,0.8}
\definecolor{myrefcolor}{rgb}{0,0,0.8}
\hypersetup{colorlinks, linkcolor=myrefcolor, citecolor=mycitecolor, urlcolor=myurlcolor}

\tikzset{->-/.style={decoration={
  markings,
  mark=at position .5 with {\arrow{>}}},postaction={decorate}}}

%% Defining commands that are always in math mode.
\def\mdef#1#2{\ea\ea\ea\gdef\ea\ea\noexpand#1\ea{\ea\ensuremath\ea{#2}}}
\def\alwaysmath#1{\ea\ea\ea\global\ea\ea\ea\let\ea\ea\csname your@#1\endcsname\csname #1\endcsname
  \ea\def\csname #1\endcsname{\ensuremath{\csname your@#1\endcsname}}}
\newcommand{\define}[1]{{\bf \boldmath{#1}}}

% blackboard bold letters
\newcommand{\lA}{\ensuremath{\mathbb{A}}}
\newcommand{\lC}{\ensuremath{\mathbb{C}}}
\newcommand{\lD}{\ensuremath{\mathbb{D}}}
\newcommand{\lE}{\ensuremath{\mathbb{E}}}
\newcommand{\lR}{\ensuremath{\mathbb{R}}}
\newcommand{\lX}{\ensuremath{\mathbb{X}}}
\mdef\fahat{\hat{\fa}}

% MISCELLANEOUS SYMBOLS
\newcommand{\inv}{^{-1}}
\newcommand{\op}{^{\mathit{op}}}
\newcommand{\co}{^{\mathit{co}}}
\newcommand{\coop}{^{\mathit{coop}}}
\newcommand{\id}{\mathm{id}}
\let\adj\dashv
\newcommand{\pullbackcorner}[1][dr]{\save*!/#1-1.2pc/#1:(-1,1)@^{|-}\restore}
\let\iso\cong
\let\eqv\simeq
\let\cng\equiv
\mdef\Id{\mathrm{Id}}
\mdef\id{\mathrm{id}}
\alwaysmath{ell}
\alwaysmath{infty}
\alwaysmath{odot}
\def\frc#1/#2.{\frac{#1}{#2}}   % \frc x^2+1 / x^2-1 .
\mdef\ten{\mathrel{\otimes}}

%% OPERATORS
\DeclareMathOperator\colim{colim}
\DeclareMathOperator\eq{eq}
\DeclareMathOperator\Aut{Aut}
\DeclareMathOperator\End{End}
\DeclareMathOperator\Hom{Hom}
\DeclareMathOperator\Map{Map}

%% ARROWS
% \to already exists
\newcommand{\too}[1][]{\ensuremath{\overset{#1}{\longrightarrow}}}
\newcommand{\oot}[1][]{\ensuremath{\overset{#1}{\longleftarrow}}}
\let\toot\rightleftarrows
\let\otto\leftrightarrows
\let\maps\colon

%% EXTENSIBLE ARROWS
\let\xto\xrightarrow
\let\xot\xleftarrow

% THEOREM-TYPE ENVIRONMENTS, hacked to
%% (a) number all with the same numbers, and
%% (b) have the right names for autoref
\def\defthm#1#2{%
  \newtheorem{#1}{#2}[section]%
  \expandafter\def\csname #1autorefname\endcsname{#2}%
  \expandafter\let\csname c@#1\endcsname\c@thm}
\newtheorem{thm}{Theorem}[section]
\newcommand{\thmautorefname}{Theorem}
\defthm{cor}{Corollary}
\defthm{prop}{Proposition}
\defthm{lem}{Lemma}
\defthm{conj}{Conjecture}
\defthm{hyp}{Hypothesis}
\defthm{fact}{Fact}
\theoremstyle{definition}
\defthm{defn}{Definition}
\defthm{notn}{Notation}
\theoremstyle{remark}
\defthm{rmk}{Remark}
\defthm{eg}{Example}

\newcommand{\fhat}{\ensuremath{\hat{f}}}

% Also number formulas with the theorem counter
\let\c@equation\c@thm
\numberwithin{equation}{section}

% Only show numbers for equations that are actually referenced (or
% whose tags are specified manually) - uses mathtools.
%\mathtoolsset{showonlyrefs,showmanualtags}

\def\tobar{\mathrel{\mkern3mu  \vcenter{\hbox{$\scriptscriptstyle+$}}%
                    \mkern-12mu{\to}}}

%\input{decls}
\UseAllTwocells

\newcommand{\dblcat}[1]{\mathbb{#1}}
\mdef\fchk{\check{f}}

\definecolor{purple(x11)}{rgb}{0.5, 0.0, 0.5}
\def\purple{\color{purple(x11)}}
\def\chris{\purple}

%Christina: change below accordingly if needed!
\newcommand{\ca}{\mathsf}
\newcommand{\bicat}{\mathbf}
\newcommand{\U}{U}
\newcommand{\D}{\ca{A}}
\newcommand{\C}{\ca{X}} 
\newcommand{\A}{\ca{A}}
\newcommand{\B}{\ca{B}}
\newcommand{\X}{\ca{X}}
\newcommand{\dcsp}[1]{{#1}\mathbb{C}\textnormal{sp}}
\tikzset{tick/.style={postaction={decorate,decoration={markings,
mark=at position 0.4 with {\draw[-] (0,.4ex) -- (0,-.4ex);}}}}}
\newcommand{\tickar}{\begin{tikzcd}[baseline=-0.5ex,cramped,sep=small,ampersand replacement=\&]{}\ar[r,tick]\&{}\end{tikzcd}}
\newcommand{\cspn}[5]{\begin{tikzcd}[baseline=-0.5ex,cramped,sep=small,ampersand replacement=\&]{#1}\ar[r,"#4"] \& {#2} \& {#3}\ar[l,"#5"']\end{tikzcd}}
\newcommand{\OpICat}{\bicat{OpICat}}%probably too much to use?
\newcommand{\pse}{\mathrm{ps}}
\newcommand{\OpFib}{\bicat{OpFib}}


\title{An equivalence of compositional frameworks}

\begin{document}
\maketitle

\begin{center}   
 % {\bf Structured cospans \\}   
  \vspace{0.3cm}
  {\em John\ C.\ Baez \\}
  \vspace{0.3cm}
  {\small
 Department of Mathematics \\
    University of California \\
  Riverside CA, USA 92521 \\ and \\
 Centre for Quantum Technologies  \\
  National University of Singapore \\
    Singapore 117543  \\    } 
 \vspace{0.4cm}
{\em Kenny Courser \\}
\vspace{0.3cm}
   {\small
  Department of Mathematics \\
  University of California \\
  Riverside CA, USA 92521 \\}
  \vspace{0.4cm}   
{\em Christina Vasilakopoulou \\}
\vspace{0.3cm}
   {\small
  Department of Mathematics \\
  University of California \\
  Riverside CA, USA 92521 \\}
  \vspace{0.3cm}   
  {\small email: baez@math.ucr.edu, courser@math.ucr.edu, vasilak@ucr.edu\\} 
  \vspace{0.3cm}   
  {\small \today}
  \vspace{0.3cm}   
\end{center}   

\begin{abstract}
\noindent
The first two authors have developed a compositional framework well-suited for studying networks that are built out of finite sets equipped with extra stuff. This framework, which goes by the name of `structured cospans', utilizes double categories where the objects are seen as inputs and outputs, morphisms are `open networks', and 2-morphisms are maps between open networks. In this setup a functor $L \colon \textrm{A} \to \textrm{X}$, which is typically a left adjoint, is used to replace the objects and vertical 1-morphisms of a given double category $\mathbb{X}$ with the objects and morphisms, respectively, of the category $\textrm{A}$. Horizontal 1-cells are then cospans in $\textrm{X}$ of a particular form with 2-morphisms given by maps of these cospans. Fong has also developed a similar framework which also uses cospans and goes by the name of `decorated cospans'. In this setup, a lax monoidal functor $F \colon \textrm{A} \to \textrm{Set}$ is used to `decorate' the apices of cospans in $\textrm{A}$ with elements of $\textrm{Set}$ giving the cospans extra structure. Using a slight variation of Fong's framework, we prove that these two frameworks are equivalent in the situation where a left adjoint can be obtained from a lax monoidal pseudofunctor using a well known construction of Grothendieck.
\end{abstract}

\section{Introduction}
Networks are playing an increasingly prominent role in our understanding of the world and as a result, methods of characterizing and studying networks have become evermore necessary. Applied category theory provides such an avenue and much work has been done towards this effort  \cite{BC,BCR,BF,BFP,BP,Cour,Fong,JM,LS,Yass}. One of the more recent frameworks developed suitable for studying networks is Fong's `decorated cospans' \cite{Fong}. A $\textbf{cospan}$ in any category is a diagram of the form:
\[
\begin{tikzpicture}[scale=1.5]
\node (A) at (0,0) {$a$};
\node (B) at (1,1) {$c$};
\node (C) at (2,0) {$b$};
\path[->,font=\scriptsize,>=angle 90]
(A) edge node[above]{$i$} (B)
(C) edge node[above]{$o$} (B);
\end{tikzpicture}
\]
We call $c$ the $\textbf{apex}$ of the cospan and $a$ and $b$ the $\textbf{feet}$ of the cospan. The morphisms $i$ and $o$ are called the $\textbf{legs}$. Cospans are ideal for realizing networks and more generally `open' systems where here open means each system or network comes with prescribed inputs and outputs which are represented by the feet of the cospan $a$ and $b$, respectively. Two open systems viewed as cospans such that the outputs of the first and the inputs of the second coincide can then naturally be composed via pushout. For example, we can compose the above cospan with the following:
\[
\begin{tikzpicture}[scale=1.5]
\node (A) at (0,0) {$b$};
\node (B) at (1,1) {$f$};
\node (C) at (2,0) {$e$};
\path[->,font=\scriptsize,>=angle 90]
(A) edge node[above]{$i^\prime$} (B)
(C) edge node[above]{$o^\prime$} (B);
\end{tikzpicture}
\]
to obtain a new cospan whose left foot is the left foot of the first, whose right foot is the right foot of the second, and whose apex and legs are given by the pushout.
\[
\begin{tikzpicture}[scale=1.5]
\node (A) at (0,0) {$a$};
\node (B) at (1,1) {$c$};
\node (C) at (2,0) {$b$};
\node (D) at (3,1) {$f$};
\node (E) at (4,0) {$e$};
\node (F) at (2,2) {$c+_b f$};
\node (G) at (2,1.25) {$c+f$};
\path[->,font=\scriptsize,>=angle 90]
(A) edge node[above]{$i$} (B)
(C) edge node[above]{$o$} (B)
(C) edge node[above]{$i^\prime$} (D)
(E) edge node[above]{$o^\prime$} (D)
(B) edge node[left]{$\psi j_c$} (F)
(D) edge node[right]{$\psi j_f$} (F)
(B) edge node[above]{$j_c$} (G)
(D) edge node[above]{$j_f$} (G)
(G) edge node [left] {$\psi$} (F);
\end{tikzpicture}
\]

In Fong's theory of decorated cospans, given a category $\textrm{A}$ with finite colmits, a symmetric lax monoidal functor $F \colon (\textrm{A},+,0) \to (\textrm{Set},\times,1)$ is used to `decorate' the apex of a cospan in $\textrm{A}$ with an element of the image of its apex under the functor $F$, in the case of the first cospan above, a morphism $d \colon 1 \to F(c)$. Here $F(c)$ is the collection of all $F$-decorations on the object $c$ and the morphism $d$ is selecting a particular one. Thus a decorated cospan is given by a pair:
\[
\begin{tikzpicture}[scale=1.5]
\node (A) at (0,0) {$a$};
\node (B) at (1,0) {$c$};
\node (C) at (2,0) {$b$};
\node (D) at (3,0) {$I$};
\node (E) at (4,0) {$F(c)$};
\path[->,font=\scriptsize,>=angle 90]
(A) edge node[above]{$i$} (B)
(C) edge node[above]{$o$} (B)
(D) edge node [above] {$d$} (E);
\end{tikzpicture}
\]
To obtain a decoration on the composition of two composable $F$-decorated cospans:
\[
\begin{tikzpicture}[scale=1.5]
\node (A) at (0,0) {$a$};
\node (B) at (1,0) {$c$};
\node (C) at (2,0) {$b$};
\node (F) at (3,0) {$b$};
\node (G) at (4,0) {$f$};
\node (H) at (5,0) {$e$};
\node (D) at (0.5,-0.5) {$I$};
\node (E) at (1.5,-0.5) {$F(c)$};
\node (D') at (3.5,-0.5) {$I$};
\node (E') at (4.5,-0.5) {$F(f)$};
\path[->,font=\scriptsize,>=angle 90]
(A) edge node[above]{$i$} (B)
(C) edge node[above]{$o$} (B)
(F) edge node[above]{$i^\prime$} (G)
(H) edge node[above]{$o^\prime$} (G)
(D) edge node [above] {$d_1$} (E)
(D') edge node [above] {$d_2$} (E');
\end{tikzpicture}
\]
we use the natural map from a coproduct to a pushout as well as the natural map that comes as part of the structure of a lax monoidal functor:
 $$1 \xrightarrow{\lambda^{-1}} 1 \times 1 \xrightarrow{d_1 \times d_2} F(c) \times F(f) \xrightarrow{\phi_{c,f}} F(c+ f) \xrightarrow{F(j_{c,f})} F(c+_b f).$$

From this symmetric lax monoidal functor $F \colon \textrm{A} \to \mathrm{Set}$, Fong then constructs a symmetric monoidal category $F\textnormal{Cospan}(\mathrm{A})$ which has:
\begin{enumerate}
\item{objects as those of $\mathrm{A}$ and}
\item{morphisms as isomorphism classes of $F$-decorated cospans, where an $F$-decorated cospan is given as above, and two $F$-decorated cospans are in the same isomorphism class if there exists an isomorphism between the apices such that the following diagrams commute:
\[
\begin{tikzpicture}[scale=1.5]
\node (A) at (0,0) {$a$};
\node (B) at (1,1) {$c$};
\node (B') at (1,-1) {$c^\prime$};
\node (C) at (2,0) {$b$};
\node (D) at (3,0) {$1$};
\node (E) at (4,0.5) {$F(c)$};
\node (E') at (4,-0.5) {$F(c^\prime)$};
\path[->,font=\scriptsize,>=angle 90]
(A) edge node[above]{$i$} (B)
(C) edge node[above]{$o$} (B)
(A) edge node[below]{$i^\prime$} (B')
(C) edge node[below]{$o^\prime$} (B')
(B) edge node [left] {$f$} (B')
(D) edge node [above] {$d$} (E)
(D) edge node [below] {$d^\prime$} (E')
(E) edge node [right] {$F(f)$} (E');
\end{tikzpicture}
\]
}
\end{enumerate}
There are some subtleties to this approach. To illustrate, consider the example where $F \colon \textrm{FinSet} \to \textrm{Set}$ is the symmetric lax monoidal functor that assigns to a finite set $N$ the (large) set of all possible graph structures on the finite set $N$, where a graph structure on $N$ is given by a diagram in $\textrm{Set}$ of the form:
\[
\begin{tikzpicture}[scale=1.5]
\node (A) at (0,0) {$E$};
\node (B) at (1,0) {$N.$};
\path[->,font=\scriptsize,>=angle 90]
(A) edge[bend left] node[above]{$s$} (B)
(A) edge[bend right] node[below]{$t$} (B);
\end{tikzpicture}
\]
Let $N=\{ v_1,v_2 \}$ be a two element set. Then one element of the set $F(N)$, in other words, one possible decoration, which is a graph structure, on the set $N$ is given by a single edge $e$ whose source and target are $v_1$ and $v_2$, respectively.
\[
\begin{tikzpicture}[scale=1.5]
\node (A) at (0,0) {$v_1$};
\node (B) at (1,0) {$v_2$};
\path[->,font=\scriptsize,>=angle 90]
(A) edge node[above]{$e$} (B);
\end{tikzpicture}
\]
Denote this element of $F(N)$ as $d$. We could label the edge $e$ something else, for instance, $e^\prime$, and this single edge would another element of $F(N)$.
\[
\begin{tikzpicture}[scale=1.5]
\node (A) at (0,0) {$v_1$};
\node (B) at (1,0) {$v_2$};
\path[->,font=\scriptsize,>=angle 90]
(A) edge node[above]{$e^\prime$} (B);
\end{tikzpicture}
\]
Denote this second element of $F(N)$ as $d^\prime$. These two graphs constitute distinct isomorphism classes in the symmetric monoidal category $F$Cospan$(\textrm{FinSet})$, as $f=\id_N$ and thus $F(f)=\id_{F(N)}$, but clearly $d \neq d^\prime$. The issue here is that the functor $F$ is decorating each finite set $N$ not only with extra structure, but with extra \emph{stuff}. Moreover, the finite set $N$ is being decorated with \emph{different} stuff in each instance and this difference in stuff is not entirely detected by the function $F(f)$. This phenomenon doesn't occur when the decorations only involve extra structure, and this has been utilized in other works \cite{BFP,BP,Yass}.

One approach to remedying this nuisance is to instead view $F$ as a functor $F \colon \textrm{A} \to \bold{Cat}$ and then consider $F(c)$ not as a \emph{set} of decorations on the element $c$ but as a \emph{category} of decorations on the element $c$. Then one can construct a similar symmetric monoidal category similar to Fong's, which will be called $D(H(\mathbb{F}\textnormal{Cospan}(\mathrm{A})))$ for reasons that will be explained later. This symmetric monoidal category will have:
\begin{enumerate}
\item{objects as those of $\mathrm{A}$ and}
\item{morphisms as isomorphism classes of cospans of $\mathrm{A}$ together with an element of image of the apex under the functor $F$.
\[
\begin{tikzpicture}[scale=1.5]
\node (A) at (0,0) {$a$};
\node (B) at (1,0) {$c$};
\node (C) at (2,0) {$b$};
\node (D) at (3,0) {$d \in F(c)$};
\path[->,font=\scriptsize,>=angle 90]
(A) edge node[above]{$i$} (B)
(C) edge node[above]{$o$} (B);
\end{tikzpicture}
\]
Except now, given another morphism:
\[
\begin{tikzpicture}[scale=1.5]
\node (A) at (0,0) {$a$};
\node (B) at (1,0) {$c^\prime$};
\node (C) at (2,0) {$b$};
\node (D) at (3,0) {$d^\prime \in F(c^\prime)$};
\path[->,font=\scriptsize,>=angle 90]
(A) edge node[above]{$i^\prime$} (B)
(C) edge node[above]{$o^\prime$} (B);
\end{tikzpicture}
\]
these two morphisms are in the same isomorphism class if the following diagrams commute
\[
\begin{tikzpicture}[scale=1.5]
\node (A) at (0,0) {$a$};
\node (B) at (1,1) {$c$};
\node (B') at (1,-1) {$c^\prime$};
\node (C) at (2,0) {$b$};
\node (D) at (3,0) {$1$};
\node (E) at (4,0.5) {$F(c)$};
\node (E') at (4,-0.5) {$F(c^\prime)$};
\path[->,font=\scriptsize,>=angle 90]
(A) edge node[above]{$i$} (B)
(C) edge node[above]{$o$} (B)
(A) edge node[below]{$i^\prime$} (B')
(C) edge node[below]{$o^\prime$} (B')
(B) edge node [left] {$f$} (B')
(D) edge node [above] {$d$} (E)
(D) edge node [below] {$d^\prime$} (E')
(E) edge node [right] {$F(f)$} (E');
\end{tikzpicture}
\]
and there exists an isomorphism $\iota \colon F(f)(d) \to d^\prime$ in the category $F(c^\prime)$.
}
\end{enumerate}
Note that the existence of an isomorphism $\iota \colon F(f)(d) \to d^\prime$ is equivalent to the triangle above on the right commuting up to a 2-isomorphism.
\[
\begin{tikzpicture}[scale=1.5]
\node (A) at (4,0) {$\iota \Swarrow$};
\node (D) at (3,0) {$1$};
\node (E) at (4.5,1) {$F(c)$};
\node (E') at (4.5,-1) {$F(c^\prime)$};
\path[->,font=\scriptsize,>=angle 90]
(D) edge node [above] {$d$} (E)
(D) edge node [below] {$d^\prime$} (E')
(E) edge node [right] {$F(f)$} (E');
\end{tikzpicture}
\]
In other words, the isomorphism $\iota$ is a \emph{natural} isomorphism. We will typically just write this as $\iota \colon F(f)(d) \to d^\prime$ to conserve space.

In this new category, the added structure of an isomorphism in the category of decorations shapes the isomorphism classes to look as one would expect. Fong's symmetric monoidal category has been extended to a bicategory by the first author \cite{Cour} and this was one of the original illuminations of the above obstacle. In this bicategory, objects and morphisms are given as in Fong's category modulo taking cospans up to isomorphism class and 2-morphisms are given by a pair of commuting diagrams:
\[
\begin{tikzpicture}[scale=1.5]
\node (A) at (0,0) {$a$};
\node (B) at (1,1) {$c$};
\node (B') at (1,-1) {$c^\prime$};
\node (C) at (2,0) {$b$};
\node (D) at (3,0) {$1$};
\node (E) at (4,0.5) {$F(c)$};
\node (E') at (4,-0.5) {$F(c^\prime)$};
\path[->,font=\scriptsize,>=angle 90]
(A) edge node[above]{$i$} (B)
(C) edge node[above]{$o$} (B)
(A) edge node[below]{$i^\prime$} (B')
(C) edge node[below]{$o^\prime$} (B')
(B) edge node [left] {$f$} (B')
(D) edge node [above] {$d$} (E)
(D) edge node [below] {$d^\prime$} (E')
(E) edge node [right] {$F(f)$} (E');
\end{tikzpicture}
\]
In this case, there would be no 2-morphism from:
\[
\begin{tikzpicture}[scale=1.5]
\node (A) at (0,0) {$v_1$};
\node (B) at (1,0) {$v_2$};
\path[->,font=\scriptsize,>=angle 90]
(A) edge node[above]{$e$} (B);
\end{tikzpicture}
\]
to:
\[
\begin{tikzpicture}[scale=1.5]
\node (A) at (0,0) {$v_1$};
\node (B) at (1,0) {$v_2.$};
\path[->,font=\scriptsize,>=angle 90]
(A) edge node[above]{$e^\prime$} (B);
\end{tikzpicture}
\]
Morally speaking, these two simple single-edged graphs should belong to the same isomorphism class, as there is a natural isomorphism of graphs between them.

In the present work, we construct a symmetric monoidal `double category' $\mathbb{F}\textnormal{Cospan}(\textrm{A})$. Double categories were first introduced by Ehresmann \cite{Ehresmann63, Ehresmann65} and symmetric monoidal double categories by Shulman \cite{Shul}. They have long been used in topology and other branches of pure mathematics \cite{Brown1,Brown2}.  More recently they have been used to study open dynamical systems \cite{LS} and open Markov processes \cite{BC}. While a mere category has only objects and morphisms, a double category has a few more types of entities:
\[
\begin{tikzpicture}[scale=1]
\node (D) at (-4,0.5) {$A$};
\node (E) at (-2,0.5) {$B$};
\node (F) at (-4,-1) {$C$};
\node (A) at (-2,-1) {$D$};
\node (B) at (-3,-0.25) {$\Downarrow a$};
\path[->,font=\scriptsize,>=angle 90]
(D) edge node [above]{$M$}(E)
(E) edge node [right]{$g$}(A)
(D) edge node [left]{$f$}(F)
(F) edge node [above]{$N$} (A);
\end{tikzpicture}
\]
We call $A, B, C$ and $D$ `objects', $f$ and $g$ `vertical 1-morphisms', $M$ and $N$ `horizontal 1-cells', and $a$ a `2-morphism'.   We can compose vertical 1-morphisms to get new vertical 1-morphisms and compose horizontal 1-cells to get new horizontal 1-cells.  We can compose the 2-morphisms in two ways: horizontally by setting squares side by side, and vertically by setting one on top of the other.   In a `strict' double category all these forms of composition are associative.  In a `pseudo' double category, horizontal 1-cells compose in a weakly associative manner: that is, the associative law holds only up to an invertible 2-morphism, called the `associator', which obeys a coherence law. The double category $\mathbb{F}\textnormal{Cospan}(\textrm{A})$ that we construct in this paper has:
\begin{enumerate}
\item{objects as those of $\textrm{A}$,}
\item{vertical 1-morphisms as morphisms of $\textrm{A}$,}
\item{horizontal 1-cells as $F$-decorated cospans of $\textrm{A}$, which are pairs:
\[
\begin{tikzpicture}[scale=1.5]
\node (A) at (0,0) {$a$};
\node (B) at (1,0) {$c$};
\node (C) at (2,0) {$b$};
\node (D) at (3,0) {$d \in F(c)$};
\path[->,font=\scriptsize,>=angle 90]
(A) edge node[above]{$i$} (B)
(C) edge node[above]{$o$} (B);
\end{tikzpicture}
\]
and}
\item{2-morphisms as pairs of commuting diagrams:
\[
\begin{tikzpicture}[scale=1.5]
\node (A) at (0,0) {$a$};
\node (B) at (1,0) {$c$};
\node (C) at (2,0) {$b$};
\node (A') at (0,-1) {$a^\prime$};
\node (B') at (1,-1) {$c^\prime$};
\node (C') at (2,-1) {$b^\prime$};
\node (D) at (3,-0.5) {$1$};
\node (E) at (4,0) {$F(c)$};
\node (E') at (4,-1) {$F(c^\prime)$};
\path[->,font=\scriptsize,>=angle 90]
(A) edge node[above]{$i$} (B)
(C) edge node[above]{$i$} (B)
(A') edge node[above]{$i^\prime$} (B')
(C') edge node[above]{$o^\prime$} (B')
(A) edge node [left]{$f$} (A')
(B) edge node [left]{$h$} (B')
(C) edge node [left]{$g$} (C')
(D) edge node [above] {$d$} (E)
(D) edge node [below] {$d^\prime$} (E')
(E) edge node [left] {$F(h)$} (E');
\end{tikzpicture}
\]
together with a morphism $\iota \colon F(h)(d) \to d^\prime$ in $F(c^\prime)$.}
\end{enumerate}
We can then obtain a symmetric monoidal bicategory $H(\mathbb{F}\textnormal{Cospan}(\textrm{A}))$ as the `horizontal bicategory' of the symmetric monoidal double category $\mathbb{F}\textnormal{Cospan}(\textrm{A})$ using a result of Shulman \cite{Shul}. This bicategory will have:
\begin{enumerate}
\item{objects as those of $\textrm{A}$,}
\item{morphisms as pairs:
\[
\begin{tikzpicture}[scale=1.5]
\node (A) at (0,0) {$a$};
\node (B) at (1,0) {$c$};
\node (C) at (2,0) {$b$};
\node (D) at (3,0) {$d \in F(c)$};
\path[->,font=\scriptsize,>=angle 90]
(A) edge node[above]{$i$} (B)
(C) edge node[above]{$o$} (B);
\end{tikzpicture}
\]
and}
\item{2-morphisms as pairs of commuting diagrams:
\[
\begin{tikzpicture}[scale=1.5]
\node (A) at (0,0) {$a$};
\node (B) at (1,1) {$c$};
\node (B') at (1,-1) {$c^\prime$};
\node (C) at (2,0) {$b$};
\node (D) at (3,0) {$1$};
\node (E) at (4,0.5) {$F(c)$};
\node (E') at (4,-0.5) {$F(c^\prime)$};
\path[->,font=\scriptsize,>=angle 90]
(A) edge node[above]{$i$} (B)
(C) edge node[above]{$o$} (B)
(A) edge node[below]{$i^\prime$} (B')
(C) edge node[below]{$o^\prime$} (B')
(B) edge node [left] {$f$} (B')
(D) edge node [above] {$d$} (E)
(D) edge node [below] {$d^\prime$} (E')
(E) edge node [right] {$F(f)$} (E');
\end{tikzpicture}
\]
together with a morphism $\iota \colon F(f)(d) \to d^\prime$ of $F(c^\prime)$.}

\end{enumerate}
Then we can obtain the above improved symmetric monoidal category as a decategorification of the symmetric monoidal bicategory $H(\mathbb{F}\textnormal{Cospan}(\textrm{A}))$.

The outline for the paper is as follows: In the second section, we construct the symmetric monoidal double category $F\textnormal{Cospan}(\textrm{A})$ mentioned above as well as a result of Shulman that allows us to obtain a symmetric monoidal bicategory whose decategorification is an improved version of Fong's symmetric monoidal category. In the third section we define maps between decorated cospan double categories which are double functors of an appropriate type. In the fourth section we investigate when a symmetric lax monoidal pseudofunctor gives rise to a left adjoint via the Grothendieck construction. In the fifth section we briefly review another compositional framework which uses cospans equipped with extra structure due to the first two authors, namely `structured cospans' and prove that the double categorical versions of decorated cospans and structured cospans are equivalent, and in the final section, we present a few examples that can be realized with either framework and discuss the realized equivalences.

\section{A symmetric monoidal double category of decorated cospans}\label{DecCospansDoublecat}
In this section we build the symmetric monoidal double category mentioned in the introduction. 
\textbf{Add more here.}
\begin{thm}
Let $\mathrm{A}$ be a category with finite colimits and $F \colon \mathrm{A} \to \bold{Cat}$ a lax monoidal pseudofunctor. Then there exists a double category $\mathbb{F} \textnormal{Cospan}(\mathrm{A})$ which has:
\begin{enumerate}
\item{objects as those of $\mathrm{A}$,}
\item{vertical 1-morphisms as morphisms of $\mathrm{A}$,}
\item{horizontal 1-cells as pairs:
\[
\begin{tikzpicture}[scale=1.5]
\node (A) at (0,0) {$a$};
\node (B) at (1,0) {$c$};
\node (C) at (2,0) {$b$};
\node (D) at (3,0) {$d \in F(c)$};
\path[->,font=\scriptsize,>=angle 90]
(A) edge node[above]{$i$} (B)
(C) edge node[above]{$o$} (B);
\end{tikzpicture}
\]
and}
\item{2-morphisms as maps of cospans in $\mathrm{A}$
\[
\begin{tikzpicture}[scale=1.5]
\node (A) at (0,0.5) {$a$};
\node (A') at (0,-0.5) {$a^\prime$};
\node (B) at (1,0.5) {$c$};
\node (C) at (2,0.5) {$b$};
\node (C') at (2,-0.5) {$b^\prime$};
\node (D) at (1,-0.5) {$c^\prime$};
\node (E) at (3,0.5) {$d \in F(c)$};
\node (F) at (3,-0.5) {$d^\prime \in F(c^\prime)$};
\path[->,font=\scriptsize,>=angle 90]
(A) edge node[above]{$$} (B)
(C) edge node[above]{$$} (B)
(A) edge node[left]{$f$} (A')
(C) edge node[left]{$g$} (C')
(A') edge node {$$} (D)
(C') edge node {$$} (D)
(B) edge node [left] {$h$} (D);
\end{tikzpicture}
\]
together with a morphism $\iota \colon F(h)(d) \to d^\prime$ in $F(c^\prime)$.}
\end{enumerate}
\end{thm}
\begin{proof}
The unit structure functor $U \colon \mathbb{F}\textnormal{Cospan}(\mathrm{A})_0 \to \mathbb{F}\textnormal{Cospan}(\mathrm{A})_1$ is defined on objects as: 
\[
\begin{tikzpicture}[scale=1.5]
\node (E) at (-1,0) {$c \mapsto$};
\node (A) at (0,0) {$c$};
\node (B) at (1,0) {$c$};
\node (C) at (2,0) {$c$};
\node (D) at (3,0) {$!_c \in F(c)$};
\path[->,font=\scriptsize,>=angle 90]
(A) edge node[above]{$1$} (B)
(C) edge node[above]{$1$} (B);
\end{tikzpicture}
\]
where $!_c \in F(c)$ is the trivial decoration on $c$ given by the composition of the unique map $F(!) \colon F(0) \to F(c)$ and the morphism $\phi \colon 1 \to F(0)$  which comes from the symmetric lax monoidal pseudofunctor $F \colon \mathrm{A} \to \bold{Cat}$. For morphisms, the structure functor $U$ is defined as:
\[
\begin{tikzpicture}[scale=1.5]
\node (G) at (-1,0.5) {$c$};
\node (G') at (-1,-0.5) {$c^\prime$};
\node (A) at (0,0.5) {$c$};
\node (A') at (0,-0.5) {$c^\prime$};
\node (B) at (1,0.5) {$c$};
\node (C) at (2,0.5) {$c$};
\node (C') at (2,-0.5) {$c^\prime$};
\node (D) at (1,-0.5) {$c^\prime$};
\node (E) at (3,0.5) {$!_c \in F(c)$};
\node (F) at (3,-0.5) {$!_{c^\prime} \in F(c^\prime)$};
\node (H) at (-0.5,0) {$\mapsto$};
\path[->,font=\scriptsize,>=angle 90]
(A) edge node[above]{$$} (B)
(C) edge node[above]{$$} (B)
(A) edge node[left]{$f$} (A')
(C) edge node[left]{$f$} (C')
(A') edge node {$$} (D)
(C') edge node {$$} (D)
(B) edge node [left] {$f$} (D)
(G) edge node [left] {$f$} (G');
\end{tikzpicture}
\]
together with the morphism $\iota_{!_f} = F(f) F(!) \phi \colon 1 \to F(c^\prime)$. We also have source and target structure functors $S, T \colon \mathbb{F}\textnormal{Cospan}(\mathrm{A})_1 \to \mathbb{F}\textnormal{Cospan}(\mathrm{A})_0$ where the source of a horizontal 1-cell
\[
\begin{tikzpicture}[scale=1.5]
\node (A) at (0,0) {$a$};
\node (B) at (1,0) {$c$};
\node (C) at (2,0) {$b$};
\node (D) at (3,0) {$d \in F(c)$};
\path[->,font=\scriptsize,>=angle 90]
(A) edge node[above]{$i$} (B)
(C) edge node[above]{$o$} (B);
\end{tikzpicture}
\]
is the object $a$ in $\mathrm{A}$ and the source of a 2-morphism
\[
\begin{tikzpicture}[scale=1.5]
\node (A) at (0,0.5) {$a$};
\node (A') at (0,-0.5) {$a^\prime$};
\node (B) at (1,0.5) {$c$};
\node (C) at (2,0.5) {$b$};
\node (C') at (2,-0.5) {$b^\prime$};
\node (D) at (1,-0.5) {$c^\prime$};
\node (E) at (3,0.5) {$d \in F(c)$};
\node (F) at (3,-0.5) {$d^\prime \in F(c^\prime)$};
\path[->,font=\scriptsize,>=angle 90]
(A) edge node[above]{$$} (B)
(C) edge node[above]{$$} (B)
(A) edge node[left]{$f$} (A')
(C) edge node[left]{$g$} (C')
(A') edge node {$$} (D)
(C') edge node {$$} (D)
(B) edge node [left] {$h$} (D);
\end{tikzpicture}
\]
$$\iota \colon F(h)(d) \to d^\prime$$
is the source of the underlying map of cospans in $\mathrm{A}$, namely the morphism $f$ in $\mathrm{A}$; the target structure functor is defined similarly. The functors satisfy the equations $$SU(c)=1(c)=TU(c)$$for all objects $c$ of $\mathrm{A}$.

Given two composable horizontal 1-cells:
\[
\begin{tikzpicture}[scale=1.5]
\node (A) at (0,0) {$a$};
\node (B) at (1,0) {$c$};
\node (C) at (2,0) {$b$};
\node (D) at (1,-0.5) {$d \in F(c)$};
\node (E) at (3,0) {$b$};
\node (F) at (4,0) {$c^\prime$};
\node (G) at (5,0) {$e$};
\node (H) at (4,-0.5) {$d^\prime \in F(c^\prime)$};
\path[->,font=\scriptsize,>=angle 90]
(A) edge node[above]{$i$} (B)
(C) edge node[above]{$o$} (B)
(E) edge node[above]{$i^\prime$} (F)
(G) edge node[above]{$o^\prime$} (F);
\end{tikzpicture}
\]
the composite is given by:
\[
\begin{tikzpicture}[scale=1.5]
\node (A) at (0,0) {$a$};
\node (B) at (1,1) {$c$};
\node (C) at (2,0) {$b$};
\node (D) at (3,1) {$c^\prime$};
\node (E) at (4,0) {$e$};
\node (F) at (2,2) {$c+c^\prime$};
\node (G) at (2,3) {$c+_{b} c^\prime$};
\path[->,font=\scriptsize,>=angle 90]
(A) edge node[above]{$i$} (B)
(C) edge node[above]{$o$} (B)
(C) edge node [above] {$i^\prime$} (D)
(E) edge node [above] {$o^\prime$} (D)
(B) edge node [above] {$j$} (F)
(D) edge node [above] {$j^\prime$} (F)
(F) edge node [left] {$\psi$} (G)
(A) edge node [left] {$\psi j i$} (G)
(E) edge node [right] {$\psi j^\prime o^\prime$} (G);
\end{tikzpicture}
\]
with the corresponding decoration of the apex $\hat{d} \in F(c+_b c^\prime)$ given by:
$$1 \xrightarrow{\lambda^{-1}} 1 \times 1 \xrightarrow{d \times d^\prime} F(c) \times F(c^\prime) \xrightarrow{\phi_{c,c^\prime}} F(c+c^\prime) \xrightarrow{F(\psi)} F(c+_{b}c^\prime)$$
where $\psi \colon c + c^\prime \to c+_b c^\prime$ is the natural map from the coproduct to the pushout and $\phi_{c,c^\prime} \colon F(c) \times F(c^\prime) \to F(c+c^\prime)$ is the natural transformation of the symmetric lax monoidal pseudofunctor $F \colon \mathrm{A} \to \bold{Cat}$. Denoting the first and second of these horizontal 1-cells as $M$ and $N$, respectively, the source and target structure functors satisfy the equations $S(N \odot M)=S(M)$ and $T(N \odot M)=T(N)$.

Given three composable horizontal 1-cells $M_1, M_2$ and $M_3$:
\[
\begin{tikzpicture}[scale=1.5]
\node (A) at (0,0) {$a$};
\node (B) at (1,0) {$c$};
\node (C) at (2,0) {$b$};
\node (D) at (1,-0.5) {$d_1 \in F(c)$};
\node (E) at (3,0) {$b$};
\node (F) at (4,0) {$c^\prime$};
\node (G) at (5,0) {$e$};
\node (H) at (4,-0.5) {$d_2 \in F(c^\prime)$};
\node (I) at (6,0) {$e$};
\node (J) at (7,0) {$c''$};
\node (K) at (8,0) {$f$};
\node (L) at (7,-0.5) {$d_3 \in F(c'')$};
\path[->,font=\scriptsize,>=angle 90]
(A) edge node[above]{$i$} (B)
(C) edge node[above]{$o$} (B)
(E) edge node[above]{$i^\prime$} (F)
(G) edge node[above]{$o^\prime$} (F)
(I) edge node[above]{$i''$} (J)
(K) edge node[above]{$o''$} (J);
\end{tikzpicture}
\]
we get a natural isomorphism $a \colon (M_1 \odot M_2) \odot M_3 \to M_1 \odot (M_2 \odot M_3)$ which is a globular 2-morphism given by a map of cospans $(\id_a,\sigma,\id_f)$:
\[
\begin{tikzpicture}[scale=1.5]
\node (A) at (0,0.5) {$a$};
\node (A') at (0,-0.5) {$a$};
\node (B) at (1.5,0.5) {$(c+_b c^\prime)+_e c''$};
\node (C) at (3,0.5) {$f$};
\node (C') at (3,-0.5) {$f$};
\node (D) at (1.5,-0.5) {$c+_b(c^\prime +_e c'')$};
\node (E) at (4.5,0.5) {$d \in F((c+_b c^\prime)+_e c'')$};
\node (F) at (4.5,-0.5) {$d^\prime \in F(c+_b (c^\prime +_e c''))$};
\path[->,font=\scriptsize,>=angle 90]
(A) edge node[above]{$$} (B)
(C) edge node[above]{$$} (B)
(A) edge node[left]{$1$} (A')
(C) edge node[left]{$1$} (C')
(A') edge node {$$} (D)
(C') edge node {$$} (D)
(B) edge node [left] {$\sigma$} (D);
\end{tikzpicture}
\]
with elements of the image of the cospan's apices under the pseudofunctor $F$ given by:
$$ d \coloneqq 1 \xrightarrow{\zeta_1} F(c+_b c^\prime) \times F(c'') \xrightarrow{F(\phi_{c+_b c^\prime, c''})} F((c+_{b}c^\prime) +c'') \xrightarrow{F(j_{c+_b c^\prime,c''})} F((c+_b c^\prime)+_e c'')$$ $$\zeta_1 = d_3 \lambda^{-1} F(j_{c,c^\prime}) \phi_{c,c^\prime} (d_1 \times d_2) \lambda^{-1}$$
and
$$ d^\prime \coloneqq 1 \xrightarrow{\zeta_2} F(c) \times F(c^\prime +_e c'') \xrightarrow{F(\phi_{c, c^\prime +_e c''})} F(c+(c^\prime +_e c'')) \xrightarrow{F(j_{c,c^\prime +_e c''})} F(c+_b (c^\prime+_e c''))$$ $$\zeta_2 = d_1 \lambda^{-1} F(j_{c^\prime,c''}) \phi_{c^\prime,c''} (d_2 \times d_3) \lambda^{-1}$$
together with an isomorphism $\iota_a \colon F(\sigma)(d) \to d^\prime$. We also have left and right unitors where given a horizontal 1-cell $M$:
\[
\begin{tikzpicture}[scale=1.5]
\node (A) at (0,0) {$a$};
\node (B) at (1,0) {$c$};
\node (C) at (2,0) {$b$};
\node (D) at (3,0) {$d \in F(c)$};
\path[->,font=\scriptsize,>=angle 90]
(A) edge node[above]{$i$} (B)
(C) edge node[above]{$o$} (B);
\end{tikzpicture}
\]
if we, say, compose with the identity horizontal 1-cell of $b$ on the right:
\[
\begin{tikzpicture}[scale=1.5]
\node (A) at (0,0) {$a$};
\node (B) at (1,0) {$c$};
\node (C) at (2,0) {$b$};
\node (D) at (1,-0.5) {$d \in F(c)$};
\node (E) at (3,0) {$b$};
\node (F) at (4,0) {$b$};
\node (G) at (5,0) {$b$};
\node (H) at (4,-0.5) {$!_b \in F(b)$};
\path[->,font=\scriptsize,>=angle 90]
(A) edge node[above]{$i$} (B)
(C) edge node[above]{$o$} (B)
(E) edge node[above]{$1$} (F)
(G) edge node[above]{$1$} (F);
\end{tikzpicture}
\]
where $!_b = F(!)  \phi \colon 1 \to F(b)$ is the trivial decoration on $b$. Composing these then gives:
\[
\begin{tikzpicture}[scale=1.5]
\node (A) at (0,0) {$a$};
\node (B) at (1,0) {$c+_b b$};
\node (C) at (2,0) {$b$};
\node (D) at (3,0) {$\hat{d} \in F(c +_b b)$};
\path[->,font=\scriptsize,>=angle 90]
(A) edge node[above]{$j \psi_c i$} (B)
(C) edge node[above]{$j \psi_b$} (B);
\end{tikzpicture}
\]
where $\psi_b \colon b \to c+b$ is the natural map into the coproduct and likewise for $\psi_c$ and $j \colon c+b \to c+_b b$ is the natural map from the coproduct to the pushout. The decoration $\hat{d} \colon 1 \to F(c+_b b)$ is given by: $$1 \xrightarrow{\lambda^{-1}} 1 \times 1 \xrightarrow{d \times !_b} F(c) \times F(b) \xrightarrow{\phi_{c,b}} F(c+b) \xrightarrow{F(j_{c,b})} F(c+_b b).$$ We then have that the right unitor of the original horizontal 1-cell $M$ is given by the globular 2-morphism $(\id_a,\gamma,\id_b)$ from the above composite to $M$:
\[
\begin{tikzpicture}[scale=1.5]
\node (A) at (0,0.5) {$a$};
\node (A') at (0,-0.5) {$a$};
\node (B) at (1.5,0.5) {$c+_b b$};
\node (C) at (3,0.5) {$b$};
\node (C') at (3,-0.5) {$b$};
\node (D) at (1.5,-0.5) {$c$};
\node (E) at (4.5,0.5) {$\hat{d} \in F((c+_b b)$};
\node (F) at (4.5,-0.5) {$d \in F(c)$};
\path[->,font=\scriptsize,>=angle 90]
(A) edge node[above]{$j \psi_c i$} (B)
(C) edge node[above]{$j \psi_b$} (B)
(A) edge node[left]{$\id_a$} (A')
(C) edge node[left]{$\id_b$} (C')
(A') edge node [above]{$i$} (D)
(C') edge node [above]{$o$} (D)
(B) edge node [left] {$\gamma$} (D);
\end{tikzpicture}
\]
where $\gamma \colon c+_b b \xrightarrow{\sim} c$ together with an isomorphism $\iota_\rho \colon F(\gamma)(\hat{d}) \to d$. The left unitor is similar. The source and target functor applied to the left and right unitors and associators yield identities. The left and right unitors together with the associator satisfy the standard pentagon and triangle identities of a monoidal category or bicategory. Finally, for the interchange law, given four 2-morphisms:
\[
\begin{tikzpicture}[scale=1.5]
\node (A) at (0,0.5) {$a$};
\node (A') at (0,-0.5) {$a^\prime$};
\node (B) at (1,0.5) {$c$};
\node (C) at (2,0.5) {$b$};
\node (C') at (2,-0.5) {$b^\prime$};
\node (D) at (1,-0.5) {$c^\prime$};
\node (E) at (3,0.5) {$d_1 \in F(c)$};
\node (F) at (3,-0.5) {$d_1^\prime \in F(c^\prime)$};
\node (G) at (4,0.5) {$b$};
\node (H) at (5,0.5) {$e$};
\node (I) at (6,0.5) {$x$};
\node (G') at (4,-0.5) {$b^\prime$};
\node (H') at (5,-0.5) {$e^\prime$};
\node (I') at (6,-0.5) {$x^\prime$};
\node (J) at (7,0.5) {$d_2 \in F(e)$};
\node (K) at (7,-0.5) {$d_2^\prime \in F(e^\prime)$};
\node (L) at (1,-1) {$\iota_1 \colon F(h_1)(d_1) \to d_1^\prime$};
\node (M) at (5,-1) {$\iota_2 \colon F(h_2)(d_2) \to d_2^\prime$};
\node (A'') at (0,-1.5) {$a^\prime$};
\node (A''') at (0,-2.5) {$a''$};
\node (B'') at (1,-1.5) {$c^\prime$};
\node (C'') at (2,-1.5) {$b^\prime$};
\node (C''') at (2,-2.5) {$b''$};
\node (D'') at (1,-2.5) {$c''$};
\node (E'') at (3,-1.5) {$d_1^\prime \in F(c^\prime)$};
\node (F'') at (3,-2.5) {$d_1'' \in F(c'')$};
\node (G'') at (4,-1.5) {$b^\prime$};
\node (H'') at (5,-1.5) {$e^\prime$};
\node (I'') at (6,-1.5) {$x^\prime$};
\node (G''') at (4,-2.5) {$b''$};
\node (H''') at (5,-2.5) {$e''$};
\node (I''') at (6,-2.5) {$x''$};
\node (J'') at (7,-1.5) {$d_2^\prime \in F(e^\prime)$};
\node (K'') at (7,-2.5) {$d_2'' \in F(e'')$};
\node (L'') at (1,-3) {$\iota_1^\prime \colon F(h_1^\prime)(d_1^\prime) \to d_1''$};
\node (M'') at (5,-3) {$\iota_2^\prime \colon F(h_2^\prime)(d_2^\prime) \to d_2''$};
\path[->,font=\scriptsize,>=angle 90]
(A) edge node[above]{$i_1$} (B)
(C) edge node[above]{$o_1$} (B)
(A) edge node[left]{$f$} (A')
(C) edge node[left]{$g$} (C')
(A') edge node[above] {$i_1^\prime$} (D)
(C') edge node[above] {$o_1^\prime$} (D)
(B) edge node [left] {$h_1$} (D)
(G) edge node [above] {$i_2$} (H)
(G) edge node [left] {$g$} (G')
(H) edge node [left] {$h_2$} (H')
(G') edge node [above] {$i_2^\prime$} (H')
(I) edge node [above] {$o_2$} (H)
(I) edge node [left] {$k$} (I')
(I') edge node [above] {$o_2^\prime$} (H')
(A'') edge node[above]{$i_1^\prime$} (B'')
(C'') edge node[above]{$o_1^\prime$} (B'')
(A'') edge node[left]{$f^\prime$} (A''')
(C'') edge node[left]{$g^\prime$} (C''')
(A''') edge node[above] {$i_1''$} (D'')
(C''') edge node[above] {$o_1''$} (D'')
(B'') edge node [left] {$h_1^\prime$} (D'')
(G'') edge node [above] {$i_2^\prime$} (H'')
(G'') edge node [left] {$g^\prime$} (G''')
(H'') edge node [left] {$h_2^\prime$} (H''')
(G''') edge node [above] {$i_2''$} (H''')
(I'') edge node [above] {$o_2^\prime$} (H'')
(I'') edge node [left] {$k^\prime$} (I''')
(I''') edge node [above] {$o_2''$} (H''');
\end{tikzpicture}
\]
if we first compose horizontally we obtain:
\[
\begin{tikzpicture}[scale=1.5]
\node (A) at (0,0.5) {$a$};
\node (A') at (0,-0.5) {$a^\prime$};
\node (B) at (1.5,0.5) {$c+_b e$};
\node (C) at (3,0.5) {$x$};
\node (C') at (3,-0.5) {$x^\prime$};
\node (D) at (1.5,-0.5) {$c^\prime +_{b^\prime} e^\prime$};
\node (E) at (4.5,0.5) {$d_{1,2} \in F(c+_b e)$};
\node (F) at (4.5,-0.5) {$d_{1,2}^\prime \in F(c^\prime +_{b^\prime} e^\prime)$};
\node (G) at (1.5,-1) {$\iota_{1,2} \colon F(h_1 +_g h_2)(d_{1,2}) \to d_{1,2}^\prime$};
\node (A'') at (0,-1.5) {$a^\prime$};
\node (A''') at (0,-2.5) {$a''$};
\node (B'') at (1.5,-1.5) {$c^\prime +_{b^\prime} e^\prime$};
\node (C'') at (3,-1.5) {$x^\prime$};
\node (C''') at (3,-2.5) {$x''$};
\node (D'') at (1.5,-2.5) {$c'' +_{b''} e''$};
\node (E'') at (4.5,-1.5) {$d_{1,2}^\prime \in F(c^\prime+_{b^\prime} e^\prime)$};
\node (F'') at (4.5,-2.5) {$d_{1,2}'' \in F(c'' +_{b''} e'')$};
\node (G'') at (1.5,-3) {$\iota_{1,2}^\prime \colon F(h_1^\prime +_{g^\prime} h_2^\prime)(d_{1,2}^\prime) \to d_{1,2}''.$};
\path[->,font=\scriptsize,>=angle 90]
(A) edge node[above]{$j \psi_a i_1$} (B)
(C) edge node[above]{$j \psi_x o_2$} (B)
(A) edge node[left]{$f$} (A')
(C) edge node[left]{$k$} (C')
(A') edge node [above]{$j \psi_{a^\prime} i_1^\prime$} (D)
(C') edge node [above]{$j \psi_{x^\prime} o_2^\prime$} (D)
(B) edge node [left] {$h_1 +_g h_2$} (D)
(A'') edge node[above]{$j \psi_{a^\prime} i_1^\prime$} (B'')
(C'') edge node[above]{$j \psi_{x^\prime} o_2^\prime$} (B'')
(A'') edge node[left]{$f^\prime$} (A''')
(C'') edge node[left]{$k^\prime$} (C''')
(A''') edge node [above]{$j \psi_{a''} i_1''$} (D'')
(C''') edge node [above]{$j \psi_{x''} o_2''$} (D'')
(B'') edge node [left] {$h_1^\prime +_{g^\prime} h_2^\prime$} (D'');
\end{tikzpicture}
\]
To obtain the morphism of decorations for a horizontal composite, we have as initial data:
\[
\begin{tikzpicture}[scale=1.5]
\node (A) at (4,0) {$\iota_1 \Swarrow$};
\node (D) at (3,0) {$1$};
\node (E) at (4.5,0.5) {$F(c)$};
\node (E') at (4.5,-0.5) {$F(c^\prime)$};
\node (A') at (6.5,0) {$\iota_2 \Swarrow$};
\node (D') at (5.5,0) {$1$};
\node (E'') at (7,0.5) {$F(e)$};
\node (E''') at (7,-0.5) {$F(e^\prime)$};
\path[->,font=\scriptsize,>=angle 90]
(D) edge node [above] {$d_1$} (E)
(D) edge node [below] {$d_1^\prime$} (E')
(E) edge node [right] {$F(h_1)$} (E')
(D') edge node [above] {$d_2$} (E'')
(D') edge node [below] {$d_2^\prime$} (E''')
(E'') edge node [right] {$F(h_2)$} (E''');
\end{tikzpicture}
\]
and then multiply these viewed as two 2-morphisms in $\bold{Cat}$ which results in:
\[
\begin{tikzpicture}[scale=1.5]
\node (A) at (4.75,0) {$\iota_1 \times \iota_2 \Swarrow$};
\node (D) at (3,0) {$1 \xrightarrow{\lambda^{-1}} 1 \times 1$};
\node (E) at (5.5,0.5) {$F(c) \times F(e)$};
\node (E') at (5.5,-0.5) {$F(c^\prime) \times F(e^\prime)$};
\node (B) at (7.5,0.5) {$F(c+e)$};
\node (B') at (7.5,-0.5) {$F(c^\prime + e^\prime)$};
\node (C) at (9,0.5) {$F(c+_b e)$};
\node (C') at (9,-0.5) {$F(c^\prime +_{b^\prime} e^\prime)$};
\path[->,font=\scriptsize,>=angle 90]
(E) edge node [above] {$\phi_{c,e}$} (B)
(E') edge node [above] {$\phi_{c^\prime,e^\prime}$} (B')
(B) edge node [above] {$F(j_{c,e})$} (C)
(B') edge node [above] {$F(j_{c^\prime,e^\prime})$} (C')
(C) edge node [right] {$F(h_1 +_g h_2)$} (C')
(B) edge node [right] {$F(h_1 + h_2)$} (B')
(D) edge node [above] {$d_1 \times d_2$} (E)
(D) edge node [below] {$d_1^\prime \times d_2^\prime$} (E')
(E) edge node [right] {$F(h_1) \times F(h_2)$} (E');
\end{tikzpicture}
\]
where the middle square commutes since $F$ is a lax monoidal pseudofunctor and the right square commutes as the underlying diagram commutes. The objects $d_{1,2}$ and $d^\prime_{1,2}$ are given by top and bottom composite of arrows and the morphism of decorations $\iota_{1,2}$ is given by composing $\iota_1 \times \iota_2$ with the two commuting squares. Returning to the interchange law, composing the two horizontal compositions above vertically then results in:
\[
\begin{tikzpicture}[scale=1.5]
\node (A) at (0,0.5) {$a$};
\node (A') at (0,-0.5) {$a''$};
\node (B) at (2,0.5) {$c+_b e$};
\node (C) at (4,0.5) {$x$};
\node (C') at (4,-0.5) {$x''$};
\node (D) at (2,-0.5) {$c'' +_{b''} e''$};
\node (E) at (5.5,0.5) {$d_{1,2} \in F(c+_b e)$};
\node (F) at (5.5,-0.5) {$d_{1,2}'' \in F(c'' +_{b''} e'')$};
\node (G) at (2,-1) {$(\iota_{1,2}^\prime \iota_{1,2}) \colon F((h_1^\prime +_{g^\prime} h_2^\prime)(h_1 +_g h_2))(d_{1,2}) \to d_{1,2}''.$};
\path[->,font=\scriptsize,>=angle 90]
(A) edge node[above]{$j \psi_a i_1$} (B)
(C) edge node[above]{$j \psi_x o_2$} (B)
(A) edge node[left]{$f^\prime f$} (A')
(C) edge node[left]{$k^\prime k$} (C')
(A') edge node [above]{$j \psi_{a''} i_1''$} (D)
(C') edge node [above]{$j \psi_{x''} o_2''$} (D)
(B) edge node [left] {$(h_1^\prime +_{g^\prime} h_2^\prime)(h_1 +_g h_2)$} (D);
\end{tikzpicture}
\]
\textbf{F being pseudo might do something here...}
The vertical composite of two morphisms of decorations is straightforward. On the other hand, if we first compose vertically we obtain:
\[
\begin{tikzpicture}[scale=1.5]
\node (A) at (0,0.5) {$a$};
\node (A') at (0,-0.5) {$a''$};
\node (B) at (1,0.5) {$c$};
\node (C) at (2,0.5) {$b$};
\node (C') at (2,-0.5) {$b''$};
\node (D) at (1,-0.5) {$c''$};
\node (E) at (3,0.5) {$d_1 \in F(c)$};
\node (F) at (3,-0.5) {$d_1'' \in F(c'')$};
\node (G) at (4,0.5) {$b$};
\node (H) at (5,0.5) {$e$};
\node (I) at (6,0.5) {$x$};
\node (G') at (4,-0.5) {$b''$};
\node (H') at (5,-0.5) {$e''$};
\node (I') at (6,-0.5) {$x''$};
\node (J) at (7,0.5) {$d_2 \in F(e)$};
\node (K) at (7,-0.5) {$d_2'' \in F(e'')$};
\node (L) at (1,-1) {$\iota_1^\prime \iota_1 \colon F(h_1^\prime h_1)(d_1) \to d_1''$};
\node (M) at (5,-1) {$\iota_2^\prime \iota_2 \colon F(h_2^\prime h_2)(d_2) \to d_2''$};
\path[->,font=\scriptsize,>=angle 90]
(A) edge node[above]{$i_1$} (B)
(C) edge node[above]{$o_1$} (B)
(A) edge node[left]{$f^\prime f$} (A')
(C) edge node[left]{$g^\prime g$} (C')
(A') edge node[above] {$i_1''$} (D)
(C') edge node[above] {$o_1''$} (D)
(B) edge node [left] {$h_1^\prime h_1$} (D)
(G) edge node [above] {$i_2$} (H)
(G) edge node [left] {$g^\prime g$} (G')
(H) edge node [left] {$h_2^\prime h_2$} (H')
(G') edge node [above] {$i_2''$} (H')
(I) edge node [above] {$o_2$} (H)
(I) edge node [left] {$k^\prime k$} (I')
(I') edge node [above] {$o_2''$} (H');
\end{tikzpicture}
\]
and then composing horizontally results in:
\[
\begin{tikzpicture}[scale=1.5]
\node (A) at (0,0.5) {$a$};
\node (A') at (0,-0.5) {$a''$};
\node (B) at (2,0.5) {$c+_b e$};
\node (C) at (4,0.5) {$x$};
\node (C') at (4,-0.5) {$x''$};
\node (D) at (2,-0.5) {$c'' +_{b''} e''$};
\node (E) at (5.5,0.5) {$d_{1,2} \in F(c+_b e)$};
\node (F) at (5.5,-0.5) {$d_{1,2}'' \in F(c'' +_{b''} e'')$};
\node (G) at (2,-1) {$(\iota_1^\prime \iota_1,\iota_2^\prime \iota_2) \colon F((h_1^\prime h_1)+_{g^\prime g} (h_2^\prime h_2))(d_{1,2}) \to d_{1,2}''.$};
\path[->,font=\scriptsize,>=angle 90]
(A) edge node[above]{$j \psi_a i_1$} (B)
(C) edge node[above]{$j \psi_x o_2$} (B)
(A) edge node[left]{$f^\prime f$} (A')
(C) edge node[left]{$k^\prime k$} (C')
(A') edge node [above]{$j \psi_{a''} i_1''$} (D)
(C') edge node [above]{$j \psi_{x''} o_2''$} (D)
(B) edge node [left] {$(h_1^\prime h_1) +_{g^\prime g} (h_2^\prime h_2)$} (D);
\end{tikzpicture}
\]
%The decorations of 
%\[
%\begin{tikzpicture}[scale=1.5]
%\node (A) at (0,0) {$1$};
%\node (B) at (2,0) {$F(c) \times F(e)$};
%\node (C) at (4,0) {$F(c+e)$};
%\node (D) at (4,-1) {$F(c+_b e)$};
%\node (E) at (4,-2) {$F(c'' +_{b''} e'')$};
%\node (F) at (0,-1) {$F(c) \times F(e)$};
%\node (G) at (0,-2) {$F(c'') \times F(e'')$};
%\node (H) at (2,-2) {$F(c'' + e'')$};
%\path[->,font=\scriptsize,>=angle 90]
%(A) edge node [above]{$d_1 \times d_2$} (B)
%(B) edge node [above] {$\phi_{c,e}$} (C)
%(C) edge node [right]{$F(j_{c,e})$} (D)
%(D) edge node [right] {$F((h_1^\prime +_{g^\prime} h_2^\prime)(h_1 +_g h_2))$} (E)
%(A) edge node [left] {$d_1 \times d_2$} (F)
%(F) edge node [left] {$F((h_1^\prime h_1) \times (h_2^\prime h_2))$} (G)
%(G) edge node [above] {$\phi_{c'',e''}$} (H)
%(H) edge node [above] {$F(j_{c'',e''})$} (E);
%\end{tikzpicture}
%\]
As is usual concerning the interchange law of double categories of this nature, only the `interior' of the two compositions appears different, but the two morphisms $(h_1^\prime +_{g^\prime} h_2^\prime)(h_1 +_g h_2) \colon c+_b e \to c'' +_{b''} e''$ and $(h_1^\prime h_1) +_{g^\prime g} (h_2^\prime h_2) \colon c+_b e \to c'' +_{b''}e''$ are the same universal map realized in two different ways. The two morphisms of decorations are two compositions of 2-morphisms in $\bold{Cat}$ for which the interchange law already holds, and as a result, the morphisms $$(\iota_{1,2}^\prime \iota_{1,2}) \colon F((h_1^\prime +_{g^\prime} h_2^\prime)(h_1 +_g h_2))(d_{1,2}) \to d_{1,2}''$$ and $$(\iota_1^\prime \iota_1,\iota_2^\prime \iota_2) \colon F((h_1^\prime h_1)+_{g^\prime g} (h_2^\prime h_2))(d_{1,2}) \to d_{1,2}''$$ are also the same.
\end{proof}

\begin{thm}\label{DC}
Let $\mathrm{A}$ be a category with finite colimits and $F \colon \mathrm{A} \to \bold{Cat}$ a symmetric lax monoidal pseudofunctor. Then the double category $\mathbb{F}\textnormal{Cospan}(\mathrm{A})$ is symmetric monoidal.
\end{thm}
\begin{proof}
First we note that the category of objects $\mathbb{F}\textnormal{Cospan}(\mathrm{A})_0=\mathrm{A}$ is symmetric monoidal under binary coproducts and the left and right unitors, associators and braidings are given as natural maps. The category of arrows $\mathbb{F}\textnormal{Cospan}(\mathrm{A})_1$ has:
\begin{enumerate}
\item{objects as pairs:
\[
\begin{tikzpicture}[scale=1.5]
\node (A) at (0,0) {$a$};
\node (B) at (1,0) {$c$};
\node (C) at (2,0) {$b$};
\node (D) at (3,0) {$d \in F(c)$};
\path[->,font=\scriptsize,>=angle 90]
(A) edge node[above]{$i$} (B)
(C) edge node[above]{$o$} (B);
\end{tikzpicture}
\]
and}
\item{morphisms as maps of cospans in $\mathrm{A}$
\[
\begin{tikzpicture}[scale=1.5]
\node (A) at (0,0.5) {$a$};
\node (A') at (0,-0.5) {$a^\prime$};
\node (B) at (1,0.5) {$c$};
\node (C) at (2,0.5) {$b$};
\node (C') at (2,-0.5) {$b^\prime$};
\node (D) at (1,-0.5) {$c^\prime$};
\node (E) at (3,0.5) {$d \in F(c)$};
\node (F) at (3,-0.5) {$d^\prime \in F(c^\prime)$};
\path[->,font=\scriptsize,>=angle 90]
(A) edge node[above]{$i$} (B)
(C) edge node[above]{$o$} (B)
(A) edge node[left]{$f$} (A')
(C) edge node[left]{$g$} (C')
(A') edge node [above]{$i^\prime$} (D)
(C') edge node [above]{$o^\prime$} (D)
(B) edge node [left] {$h$} (D);
\end{tikzpicture}
\]
together with elements of the images of the apices under the pseudofunctor $F$ and a morphism $\iota \colon F(h)(d) \to d^\prime$.
}
\end{enumerate}
Given two objects $M_1$ and $M_2$ of $\mathbb{F}\textnormal{Cospan}(\mathrm{A})_1$:
\[
\begin{tikzpicture}[scale=1.5]
\node (A) at (0,0) {$a_1$};
\node (B) at (1,0) {$c_1$};
\node (C) at (2,0) {$b_1$};
\node (D) at (1,-0.5) {$d_1 \in F(c_1)$};
\node (E) at (3,0) {$a_2$};
\node (F) at (4,0) {$c_2$};
\node (G) at (5,0) {$b_2$};
\node (H) at (4,-0.5) {$d_2 \in F(c_2)$};
\path[->,font=\scriptsize,>=angle 90]
(A) edge node[above]{$i_1$} (B)
(C) edge node[above]{$o_1$} (B)
(E) edge node[above]{$i_2$} (F)
(G) edge node[above]{$o_2$} (F);
\end{tikzpicture}
\]
their tensor product $M_1 \otimes M_2$ is given by taking the coproducts of the cospans of $\mathrm{A}$
\[
\begin{tikzpicture}[scale=1.5]
\node (A) at (0,0) {$a_1+a_2$};
\node (B) at (1.25,0) {$c_1+c_2$};
\node (C) at (2.5,0) {$b_1+b_2$};
\node (D) at (4.25,0) {$d_{M_1 \otimes M_2} \in F(c_1+c_2)$};
\path[->,font=\scriptsize,>=angle 90]
(A) edge node[above]{$i_1+i_2$} (B)
(C) edge node[above]{$o_1+o_2$} (B);
\end{tikzpicture}
\]
and where the element of the image of the apex under $F$  is obtained using the natural transformation of the symmetric lax monoidal pseudofunctor $F:$ $$1 \xrightarrow{\lambda^{-1}} 1 \times 1 \xrightarrow{d_1 \times d_2} F(c_1) \times F(c_2) \xrightarrow{\phi_{c_1,c_2}} F(c_1 + c_2).$$The monoidal unit is given by:
\[
\begin{tikzpicture}[scale=1.5]
\node (A) at (0,0) {$0$};
\node (B) at (1,0) {$0$};
\node (C) at (2,0) {$0$};
\node (D) at (3,0) {$! \in F(0)$};
\path[->,font=\scriptsize,>=angle 90]
(A) edge node[above]{$!$} (B)
(C) edge node[above]{$!$} (B);
\end{tikzpicture}
\]
where $0$ is the monoidal unit of $\mathrm{A}$ and $! \colon 1 \to F(0)$ is the morphism which is part of the structure of the symmetric lax monoidal pseudofunctor $F \colon \mathrm{A} \to \bold{Cat}$. Tensoring an object with the monoidal unit, say, on the left:
\[
\begin{tikzpicture}[scale=1.5]
\node (A) at (0,0) {$0$};
\node (B) at (1,0) {$0$};
\node (C) at (2,0) {$0$};
\node (D) at (1,-0.5) {$! \in F(0)$};
\node (E) at (3,0) {$a$};
\node (F) at (4,0) {$c$};
\node (G) at (5,0) {$b$};
\node (H) at (4,-0.5) {$d \in F(c)$};
\path[->,font=\scriptsize,>=angle 90]
(A) edge node[above]{$!$} (B)
(C) edge node[above]{$!$} (B)
(E) edge node[above]{$i$} (F)
(G) edge node[above]{$o$} (F);
\end{tikzpicture}
\]
results in:
\[
\begin{tikzpicture}[scale=1.5]
\node (A) at (0,0) {$0+a$};
\node (B) at (1,0) {$0+c$};
\node (C) at (2,0) {$0+b$};
\node (D) at (3.5,0) {$d^\prime \in F(0+c)$};
\path[->,font=\scriptsize,>=angle 90]
(A) edge node[above]{$!+i$} (B)
(C) edge node[above]{$!+o$} (B);
\end{tikzpicture}
\]
where $d^\prime \in F(0+c)$ is given by $$1 \xrightarrow{\lambda^{-1}} 1 \times 1 \xrightarrow{\phi \times d} F(0) \times F(c) \xrightarrow{\phi_{0,d}} F(0+c).$$The left unitor is then a morphism in $\mathbb{F} \textnormal{Cospan}(\mathrm{A})_1$ given by:
\[
\begin{tikzpicture}[scale=1.5]
\node (A) at (0,0.5) {$0+a$};
\node (A') at (0,-0.5) {$a$};
\node (B) at (1,0.5) {$0+c$};
\node (C) at (2,0.5) {$0+b$};
\node (C') at (2,-0.5) {$b$};
\node (D) at (1,-0.5) {$c$};
\node (E) at (3.5,0.5) {$d^\prime \in F(0+c)$};
\node (F) at (3.5,-0.5) {$d \in F(c)$};
\path[->,font=\scriptsize,>=angle 90]
(A) edge node[above]{$!+i$} (B)
(C) edge node[above]{$!+o$} (B)
(A) edge node[left]{$\ell$} (A')
(C) edge node[left]{$\ell$} (C')
(A') edge node [above]{$i$} (D)
(C') edge node [above]{$o$} (D)
(B) edge node [left] {$\ell$} (D);
\end{tikzpicture}
\]
where $\ell$ is the left unitor of $(\mathrm{A},+,0)$, together with the morphism $\iota_{\lambda} \colon F(\ell)(d^\prime) \to d$. The right unitor is similar.

Given three objects $M_1, M_2$ and $M_3$ in $\mathbb{F}\textnormal{Cospan}(\mathrm{A})_1$:
\[
\begin{tikzpicture}[scale=1.5]
\node (A) at (0,0) {$a_1$};
\node (B) at (1,0) {$c_1$};
\node (C) at (2,0) {$b_1$};
\node (D) at (1,-0.5) {$d_1 \in F(c_1)$};
\node (E) at (3,0) {$a_2$};
\node (F) at (4,0) {$c_2$};
\node (G) at (5,0) {$b_2$};
\node (H) at (4,-0.5) {$d_2 \in F(c_2)$};
\node (I) at (6,0) {$a_3$};
\node (J) at (7,0) {$c_3$};
\node (K) at (8,0) {$b_3$};
\node (L) at (7,-0.5) {$d_3 \in F(c_3)$};
\path[->,font=\scriptsize,>=angle 90]
(A) edge node[above]{$i_1$} (B)
(C) edge node[above]{$o_1$} (B)
(E) edge node[above]{$i_2$} (F)
(G) edge node[above]{$o_2$} (F)
(I) edge node[above]{$i_3$} (J)
(K) edge node[above]{$o_3$} (J);
\end{tikzpicture}
\]
tensoring the first two and then the third results in $(M_1 \otimes M_2) \otimes M_3$:
\[
\begin{tikzpicture}[scale=1.5]
\node (A) at (0,0) {$(a_1+a_2)+a_3$};
\node (B) at (2.25,0){$(c_1+c_2)+c_3$};
\node (C) at (4.5,0) {$(b_1+b_2)+b_3$};
\node (D) at (2.25,-0.5) {$d_{(M_1 \otimes M_2) \otimes M_3} \in F((c_1+c_2)+c_3)$};
\path[->,font=\scriptsize,>=angle 90]
(A) edge node[above]{$(i_1+i_2)+i_3$} (B)
(C) edge node[above]{$(o_1+o_2)+o_3$} (B);
\end{tikzpicture}
\]
where $d_{(M_1 \otimes M_2) \otimes M_3} \colon 1 \to F((c_1+c_2)+c_3)$ is given by: $$1 \xrightarrow{((d_1 \times d_2) \times d_3)} (F(c_1) \times F(c_2)) \times F(c_3) \xrightarrow{\phi_{c_1,c_2} \times 1} F(c_1+c_2) \times F(c_3) \xrightarrow{\phi_{c_1+c_2,c_3}} F((c_1+c_2)+c_3)$$whereas tensoring the last two and then the first results in $M_1 \otimes (M_2 \otimes M_3)$:
\[
\begin{tikzpicture}[scale=1.5]
\node (A) at (0,0) {$a_1+(a_2+a_3)$};
\node (B) at (2.25,0) {$c_1+(c_2+c_3)$};
\node (C) at (4.5,0) {$b_1+(b_2+b_3)$};
\node (D) at (2.25,-0.5) {$d_{M_1 \otimes (M_2 \otimes M_3)} \in F(c_1+(c_2+c_3))$};
\path[->,font=\scriptsize,>=angle 90]
(A) edge node[above]{$i_1+(i_2+i_3)$} (B)
(C) edge node[above]{$o_1+(o_2+o_3)$} (B);
\end{tikzpicture}
\]
where $d_{M_1 \otimes (M_2 \otimes M_3)} \colon 1 \to F(c_1+(c_2+c_3))$ is given by: $$1 \xrightarrow{d_1 \times (d_2 \times d_3)} F(c_1) \times (F(c_2) \times F(c_3)) \xrightarrow{1 \times \phi_{c_2,c_3}} F(c_1) \times F(c_2+c_3) \xrightarrow{\phi_{c_1,c_2+c_3}} F(c_1+(c_2+c_3)).$$If we let $a$ denote the associator of $(\mathrm{A},+,0)$, the associator of $\mathbb{F} \textnormal{Cospan}(\mathrm{A})_1$ is then a map of cospans in $\mathrm{A}$ from $(M_1 \otimes M_2) \otimes M_3)$ to $M_1 \otimes (M_2 \otimes M_3)$ given by:
\[
\begin{tikzpicture}[scale=1.5]
\node (A) at (0,0.5) {$(a_1+a_2)+a_3$};
\node (A') at (0,-0.5) {$a_1+(a_2+a_3)$};
\node (B) at (2.25,0.5) {$(c_1+c_2)+c_3$};
\node (C) at (4.5,0.5) {$(b_1+b_2)+b_3$};
\node (C') at (4.5,-0.5) {$b_1+(b_2+b_3)$};
\node (D) at (2.25,-0.5) {$c_1+(c_2+c_3)$};
\node (E) at (7,0.5) {$d_{(M_1 \otimes M_2) \otimes M_3} \in F((c_1+c_2)+c_3)$};
\node (F) at (7,-0.5) {$d_{M_1 \otimes (M_2 \otimes M_3)} \in F(c_1+(c_2+c_3))$};
\path[->,font=\scriptsize,>=angle 90]
(A) edge node[above]{$(i_1+i_2)+i_3$} (B)
(C) edge node[above]{$(o_1+o_2)+o_3$} (B)
(A) edge node[left]{$a$} (A')
(C) edge node[left]{$a$} (C')
(A') edge node [above]{$i_1+(i_2+i_3)$} (D)
(C') edge node [above]{$o_1+(o_2+o_3)$} (D)
(B) edge node [left] {$a$} (D);
\end{tikzpicture}
\]
together with the isomorphism $\iota_a \colon F(a)(d_{(M_1 \otimes M_2) \otimes M_3}) \to d_{M_1 \otimes (M_2 \otimes M_3)}$. If we denote the above associator simply as $a$ and the left and right unitors as $\lambda$ and $\rho$, respectively, then given four objects in $\mathbb{F}\textnormal{Cospan}(\mathrm{A})_1$, say $M_1, M_2, M_3$ and $M_4$:
\[
\begin{tikzpicture}[scale=1.5]
\node (A) at (0,0) {$a_1$};
\node (B) at (1,0) {$c_1$};
\node (C) at (2,0) {$b_1$};
\node (D) at (1,-0.5) {$d_1 \in F(c_1)$};
\node (E) at (3,0) {$a_2$};
\node (F) at (4,0) {$c_2$};
\node (G) at (5,0) {$b_2$};
\node (H) at (4,-0.5) {$d_2 \in F(c_2)$};
\node (I) at (0,-1.5) {$a_3$};
\node (J) at (1,-1.5) {$c_3$};
\node (K) at (2,-1.5) {$b_3$};
\node (L) at (1,-2) {$d_3 \in F(c_3)$};
\node (M) at (3,-1.5) {$a_4$};
\node (N) at (4,-1.5) {$c_4$};
\node (O) at (5,-1.5) {$b_4$};
\node (P) at (4,-2) {$d_4 \in F(c_4)$};
\path[->,font=\scriptsize,>=angle 90]
(A) edge node[above]{$i_1$} (B)
(C) edge node[above]{$o_1$} (B)
(E) edge node[above]{$i_2$} (F)
(G) edge node[above]{$o_2$} (F)
(I) edge node[above]{$i_3$} (J)
(K) edge node[above]{$o_3$} (J)
(M) edge node[above]{$i_4$} (N)
(O) edge node[above]{$o_4$} (N);
\end{tikzpicture}
\]
then the following pentagon of underlying cospans commutes:
\[
\begin{tikzpicture}[scale=1.5]
\node (A) at (0,0) {$((M_1 \otimes M_2) \otimes M_3) \otimes M_4$};
\node (B) at (2.25,1) {$(M_1 \otimes M_2) \otimes (M_3 \otimes M_4)$};
\node (C) at (4.5,0) {$M_1 \otimes (M_2 \otimes (M_3 \otimes M_4))$};
\node (D) at (.75,-1.5) {$(M_1 \otimes (M_2 \otimes M_3)) \otimes M_4$};
\node (E) at (3.75,-1.5) {$M_1 \otimes ((M_2 \otimes M_3) \otimes M_4)$};
\path[->,font=\scriptsize,>=angle 90]
(A) edge node[above]{$a$} (B)
(B) edge node[above]{$a$} (C)
(A) edge node[left]{$a \otimes 1$} (D)
(D) edge node[above]{$a$} (E)
(E) edge node[right]{$1 \otimes a$} (C);
\end{tikzpicture}
\]
as well as the following pentagon of corresponding decorations in the category $F(c_1 +(c_2 +(c_3+c_4)))$:
\[
\begin{tikzpicture}[scale=1.5]
\node (A) at (-.5,0) {$F(aa)(d_{((M_1 \otimes M_2) \otimes M_3) \otimes M_4}) $};
\node (B) at (2.25,1) {$F(a)(d_{(M_1 \otimes M_2) \otimes (M_3 \otimes M_4)})$};
\node (C) at (5,0) {$d_{M_1 \otimes (M_2 \otimes (M_3 \otimes M_4))}$};
\node (D) at (-.5,-1) {$F((1 \otimes a)a)(d_{(M_1 \otimes (M_2 \otimes M_3)) \otimes M_4}) $};
\node (E) at (5,-1) {$F(1 \otimes a)(d_{M_1 \otimes ((M_2 \otimes M_3) \otimes M_4)}) $};
\path[->,font=\scriptsize,>=angle 90]
(A) edge node[left]{$F(a)(\iota_a)$} (B)
(B) edge node[above]{$\iota_a$} (C)
(A) edge node[left]{$F((1 \otimes a)a)(\iota_{a \otimes 1})$} (D)
(D) edge node[above]{$F(1 \otimes a)(\iota_a)$} (E)
(E) edge node[right]{$\iota_{1 \otimes a}$} (C);
\end{tikzpicture}
\]
since $$F(aa)(d_{((M_1 \otimes M_2) \otimes M_3) \otimes M_4})=F((1 \otimes a)a(a \otimes 1))(d_{((M_1 \otimes M_2) \otimes M_3) \otimes M_4})$$ as the above underlying diagram of maps of cospans commutes.

Similarly, if we denote the left and right unitors as $\lambda$ and $\rho$, respectively, then the following triangle of underlying maps of cospans commutes:
\[
\begin{tikzpicture}[scale=1.5]
\node (A) at (0,0) {$(M_1 \otimes U_{0}) \otimes M_2$};
\node (B) at (2.25,1) {$M_1 \otimes M_2$};
\node (C) at (4.5,0) {$M_1 \otimes (U_0 \otimes M_2)$};
\path[->,font=\scriptsize,>=angle 90]
(A) edge node[left]{$\rho \otimes 1$} (B)
(C) edge node[right]{$1 \otimes \lambda$} (B)
(A) edge node[above]{$a$} (C);
\end{tikzpicture}
\]
as well as the following triangle of corresponding decorations in the category $F(c_1+c_2)$:
\[
\begin{tikzpicture}[scale=1.5]
\node (A) at (0,0) {$F(\rho \otimes 1)(d_{(M_1 \otimes U_0) \otimes M_2})$};
\node (B) at (2.25,1) {$d_{M_1 \otimes M_2}$};
\node (C) at (4.5,0) {$F(1 \otimes \lambda)(d_{M_1 \otimes (U_0 \otimes M_2)}) $};
\path[->,font=\scriptsize,>=angle 90]
(A) edge node[left]{$\iota_{\rho \otimes 1}$} (B)
(C) edge node[right]{$\iota_{1 \otimes \lambda}$} (B)
(A) edge node[above]{$F(\iota_a)$} (C);
\end{tikzpicture}
\]
since $$F(\rho \otimes 1)(d_{(M_1 \otimes U_0) \otimes M_2})=F(a \otimes (1 \otimes \lambda))(d_{(M_1 \otimes U_0) \otimes M_2})$$ as the above underlying diagram of maps of cospans commutes.

For a monoidal product of objects $M_1 \otimes M_2$ in $\mathbb{F} \textnormal{Cospan}(\mathrm{A})_1$, the source and target structure functors $S,T \colon \mathbb{F} \textnormal{Cospan}(\mathrm{A})_1 \to \mathbb{F} \textnormal{Cospan}(\mathrm{A})_0$ satisfy the following equations: $$S(M_1 \otimes M_2)=S(M_1) \otimes S(M_2)$$ $$T(M_1 \otimes M_2)=T(M_1) \otimes T(M_2).$$
For two objects $M_1$ and $M_2$ in $\mathbb{F} \textnormal{Cospan}(\mathrm{A})_1$, we have a braiding $\beta_{M_1,M_2} \colon M_1 \otimes M_2 \to M_2 \otimes M_1$ given by:
\[
\begin{tikzpicture}[scale=1.5]
\node (A) at (0,0.5) {$a_1+a_2$};
\node (A') at (0,-0.5) {$a_2+a_1$};
\node (B) at (2,0.5) {$c_1+c_2$};
\node (C) at (4,0.5) {$b_1+b_2$};
\node (C') at (4,-0.5) {$b_2+b_1$};
\node (D) at (2,-0.5) {$c_2+c_1$};
\node (E) at (5.5,0.5) {$d_{M_1 \otimes M_2} \in F(c_1+c_2)$};
\node (F) at (5.5,-0.5) {$d_{M_2 \otimes M_1} \in F(c_2+c_1)$};
\path[->,font=\scriptsize,>=angle 90]
(A) edge node[above]{$i_1+i_2$} (B)
(C) edge node[above]{$o_1+o_2$} (B)
(A) edge node[left]{$\beta_{a_1,a_2}$} (A')
(C) edge node[left]{$\beta_{b_1,b_2}$} (C')
(A') edge node [above]{$i_2+i_1$} (D)
(C') edge node [above]{$o_2+o_1$} (D)
(B) edge node [left] {$\beta_{c_1,c_2}$} (D);
\end{tikzpicture}
\]
$$\iota_{\beta_{M_1,M_2}} \colon F(\beta_{c_1,c_2})(d_{M_1 \otimes M_2}) \xrightarrow{\sim} d_{M_2 \otimes M_1}.$$
This braiding makes the following triangle of underlying cospans commute:
\[
\begin{tikzpicture}[scale=1.5]
\node (A) at (0,0) {$M_1 \otimes M_2$};
\node (B) at (2.25,1) {$M_1 \otimes M_2$};
\node (C) at (4.5,0) {$M_2 \otimes M_1$};
\path[->,font=\scriptsize,>=angle 90]
(A) edge node[above]{$1$} (B)
(C) edge node[above]{$\beta_{M_2,M_1}$} (B)
(A) edge node[above]{$\beta_{M_1,M_2}$} (C);
\end{tikzpicture}
\]
as well as the following diagram of corresponding decorations in the category $F(c_2+c_1)$:
\[
\begin{tikzpicture}[scale=1.5]
\node (A) at (0,0) {$F(\beta_{c_1,c_2})(d_{M_1 \otimes M_2})$};
\node (B) at (2.25,1) {$F(\beta_{c_1,c_2})(d_{M_1 \otimes M_2}) $};
\node (C) at (4.5,0) {$d_{M_2 \otimes M_1} $};
\path[->,font=\scriptsize,>=angle 90]
(A) edge node[above]{$1$} (B)
(C) edge node[right]{$F(\beta_{c_1,c_2})(\iota_{\beta_{M_2,M_1}})$} (B)
(A) edge node[above]{$\iota_{\beta_{M_1,M_2}}$} (C);
\end{tikzpicture}
\]
since $F(\beta_{c_1,c_2} \beta_{c_2,c_1}) (d_{M_2 \otimes M_1}) = d_{M_2 \otimes M_1}$. Thus $\mathbb{F} \textnormal{Cospan}(\mathrm{A})_1$ is also symmetric monoidal.

Now, given four horizontal 1-cells $M_1, M_2, N_1$ and $N_2$ respectively by:
\[
\begin{tikzpicture}[scale=1.5]
\node (A) at (0,0) {$a_1$};
\node (B) at (1,0) {$c_1$};
\node (C) at (2,0) {$b_1$};
\node (D) at (1,-0.5) {$d_{M_1} \in F(c_1)$};
\node (E) at (3,0) {$b_1$};
\node (F) at (4,0) {$e_1$};
\node (G) at (5,0) {$x_1$};
\node (H) at (4,-0.5) {$d_{M_2} \in F(e_1)$};
\node (I) at (0,-1.5) {$a_2$};
\node (J) at (1,-1.5) {$c_2$};
\node (K) at (2,-1.5) {$b_2$};
\node (L) at (1,-2) {$d_{N_1} \in F(c_2)$};
\node (M) at (3,-1.5) {$b_2$};
\node (N) at (4,-1.5) {$e_2$};
\node (O) at (5,-1.5) {$x_2$};
\node (P) at (4,-2) {$d_{N_2} \in F(e_2)$};
\path[->,font=\scriptsize,>=angle 90]
(A) edge node[above]{$i_1$} (B)
(C) edge node[above]{$o_1$} (B)
(E) edge node[above]{$i_1^\prime$} (F)
(G) edge node[above]{$o_1^\prime$} (F)
(I) edge node[above]{$i_2$} (J)
(K) edge node[above]{$o_2$} (J)
(M) edge node[above]{$i_2^\prime$} (N)
(O) edge node[above]{$o_2^\prime$} (N);
\end{tikzpicture}
\]
we have that $(M_1 \otimes N_1) \odot (M_2 \otimes N_2)$ is given by:
\[
\begin{tikzpicture}[scale=1.5]
\node (A) at (0,0) {$a_1+a_2$};
\node (B) at (2.5,0) {$(c_1+c_2)+_{b_1+b_2}(e_1+e_2)$};
\node (C) at (5,0) {$x_1+x_2$};
\node (D) at (2.5,-0.5) {$d_{(M_1 \otimes N_1) \odot (M_2 \otimes N_2)} \in F((c_1+c_2)+_{b_1+b_2}(e_1+e_2))$};
\path[->,font=\scriptsize,>=angle 90]
(A) edge node[above]{$j \psi(i_1+i_2)$} (B)
(C) edge node[above]{$j \psi(o_1^\prime + o_2^\prime)$} (B);
\end{tikzpicture}
\]
and $(M_1 \odot M_2) \otimes (N_1 \odot N_2)$ is given by:
\[
\begin{tikzpicture}[scale=1.5]
\node (A) at (0,0) {$a_1+a_2$};
\node (B) at (2.5,0) {$(c_1+_{b_1} e_1) + (c_2 +_{b_2} e_2)$};
\node (C) at (5,0) {$x_1+x_2$};
\node (D) at (2.5,-0.5) {$d_{(M_1 \odot M_2) \otimes (N_1 \odot N_2)} \in F((c_1+_{b_1}e_1)+(c_2+_{b_2}e_2))$};
\path[->,font=\scriptsize,>=angle 90]
(A) edge node[above]{$(j \psi i_2)+(j \psi i_1)$} (B)
(C) edge node[above]{$(j \psi o_2^\prime)+(j \psi o_1^\prime)$} (B);
\end{tikzpicture}
\]
where $\psi$ and $j$ are the natural maps into a coproduct and from a coproduct into a pushout, respectively. We then get a globular 2-morphism $$\chi \colon (M_1 \otimes N_1) \odot (M_2 \otimes N_2) \to (M_1 \odot M_2) \otimes (N_1 \odot N_2)$$ given by:
\[
\begin{tikzpicture}[scale=1.5]
\node (A) at (0,0.5) {$a_1+a_2$};
\node (A') at (0,-0.5) {$a_1+a_2$};
\node (B) at (2.5,0.5) {$(c_1+c_2)+_{b_1+b_2}(e_1+e_2)$};
\node (C) at (5,0.5) {$b_1+b_2$};
\node (C') at (5,-0.5) {$b_1+b_2$};
\node (D) at (2.5,-0.5) {$(c_1+_{b_1}e_1)+(c_2+_{b_2}e_2)$};
\node (E) at (2.5,1) {$\hat{d}_1 \in F((c_1+c_2)+_{b_1+b_2}(e_1+e_2))$};
\node (F) at (2.5,-1) {$\hat{d}_2 \in F((c_1+_{b_1}e_1)+(c_2+_{b_2}e_2))$};
\path[->,font=\scriptsize,>=angle 90]
(A) edge node[above]{$j \psi (i_1+i_2)$} (B)
(C) edge node[above]{$j \psi (o_1^\prime + o_2^\prime)$} (B)
(A) edge node[left]{$1$} (A')
(C) edge node[left]{$1$} (C')
(A') edge node [above]{$(j \psi i_2)+(j \psi i_1)$} (D)
(C') edge node [above]{$(j \psi o_2^\prime)+(j \psi o_1^\prime)$} (D)
(B) edge node [left] {$\hat{\chi}$} (D);
\end{tikzpicture}
\]
$$\iota_{\hat{\chi}} \colon F(\hat{\chi})(d_{(M_1 \otimes N_1) \odot (M_2 \otimes N_2)}) \to d_{(M_1 \odot M_2) \otimes (N_1 \odot N_2)}.$$
For two objects $a,b \in \mathrm{A}$, $U_{a+b}$ is given by:
\[
\begin{tikzpicture}[scale=1.5]
\node (A) at (0,0) {$a+b$};
\node (B) at (1,0) {$a+b$};
\node (C) at (2,0) {$a+b$};
\node (D) at (1,-0.5) {$!_{a+b} \in F(a+b)$};
\path[->,font=\scriptsize,>=angle 90]
(A) edge node[above]{$1_{a+b}$} (B)
(C) edge node[above]{$1_{a+b}$} (B);
\end{tikzpicture}
\]
where $$!_{a+b} \colon 1 \xrightarrow{\lambda^{-1}} 1 \times 1 \xrightarrow{\phi \times \phi} F(0) \times F(0) \xrightarrow{F(!_a) \times F(!_b)} F(a) \times F(b) \xrightarrow{\phi_{a,b}} F(a+b).$$
Similarly, we have $U_a$ and $U_b$ given respectively by:
\[
\begin{tikzpicture}[scale=1.5]
\node (A) at (0,0) {$a$};
\node (B) at (1,0) {$a$};
\node (C) at (2,0) {$a$};
\node (D) at (1,-0.5) {$!_a \in F(a)$};
\node (E) at (3,0) {$b$};
\node (F) at (4,0) {$b$};
\node (G) at (5,0) {$b$};
\node (H) at (4,-0.5) {$!_b \in F(b)$};
\path[->,font=\scriptsize,>=angle 90]
(A) edge node[above]{$1_a$} (B)
(C) edge node[above]{$1_a$} (B)
(E) edge node[above]{$1_b$} (F)
(G) edge node[above]{$1_b$} (F);
\end{tikzpicture}
\]
and then $U_a + U_b$ is given by:
\[
\begin{tikzpicture}[scale=1.5]
\node (A) at (0,0) {$a+b$};
\node (B) at (1.25,0) {$a+b$};
\node (C) at (2.5,0) {$a+b$};
\node (D) at (1.25,-0.5) {$!_a + !_b \in F(a+b)$};
\path[->,font=\scriptsize,>=angle 90]
(A) edge node[above]{$1_a + 1_b$} (B)
(C) edge node[above]{$1_a + 1_b$} (B);
\end{tikzpicture}
\]
where
$$!_a + !_b \colon 1 \xrightarrow{\lambda^{-1}} 1 \times 1 \xrightarrow{!_a \times !_b} F(a) \times F(b) \xrightarrow{\phi_{a,b}} F(a+b).$$
We then have another globular isomorphism $$\mu_{a,b} \colon U_{a+b} \to U_a + U_b$$ given by the identity 2-morphism:
\[
\begin{tikzpicture}[scale=1.5]
\node (A) at (0,0.5) {$a+b$};
\node (A') at (0,-0.5) {$a+b$};
\node (B) at (2,0.5) {$a+b$};
\node (C) at (4,0.5) {$a+b$};
\node (C') at (4,-0.5) {$a+b$};
\node (D) at (2,-0.5) {$a+b$};
\node (E) at (5.5,0.5) {$!_{a+b} \in F(a+b)$};
\node (F) at (5.5,-0.5) {$!_a + !_b \in F(a+b)$};
\path[->,font=\scriptsize,>=angle 90]
(A) edge node[above]{$1_{a+b}$} (B)
(C) edge node[above]{$1_{a+b}$} (B)
(A) edge node[left]{$1$} (A')
(C) edge node[left]{$1$} (C')
(A') edge node [above]{$1_a + 1_b$} (D)
(C') edge node [above]{$1_a + 1_b$} (D)
(B) edge node [left] {$1$} (D);
\end{tikzpicture}
\]
$$\iota_{a,b} \colon !_{a+b} \xrightarrow{\sim} !_a + !_b.$$

There are a fair amount of coherence diagrams to verify, many of which are similar in flavor and make use of the two above globular ismorphisms. We check a few to give a sense of what these are like. For example,  given horizontal 1-cells $M_i,N_i,P_i$, the following commutative diagram expresses the associativity isomorphism as a transformation of double categories.
\[
\begin{tikzpicture}[scale=1.5]
\node (A) at (0,0.5) {$((M_1 \otimes N_1) \otimes P_1) \odot ((M_2 \otimes N_2) \otimes P_2)$};
\node (A') at (4.5,0.5) {$(M_1 \otimes (N_1 \otimes P_1)) \odot (M_2 \otimes (N_2 \otimes P_2))$};
\node (B) at (0,-0.25) {$((M_1 \otimes N_1) \odot (M_2 \otimes N_2)) \otimes (P_1 \odot P_2)$};
\node (C) at (4.5,-0.25) {$(M_1 \odot M_2) \otimes ((N_1 \otimes P_1) \odot (N_2 \otimes P_2))$};
\node (C') at (0,-1) {$((M_1 \odot M_2) \otimes (N_1 \odot N_2)) \otimes (P_1 \odot P_2)$};
\node (D) at (4.5,-1) {$(M_1 \odot M_2) \otimes ((N_1 \odot N_2) \otimes (P_1 \odot P_2))$};
\path[->,font=\scriptsize,>=angle 90]
(A) edge node[above]{$a \odot a$} (A')
(A) edge node[left]{$\mu$} (B)
(A') edge node[right]{$\mu$} (C)
(B) edge node[left]{$\mu \otimes 1$} (C')
(C) edge node [right] {$1 \otimes \mu$} (D)
(C') edge node [above] {$a$} (D);
\end{tikzpicture}
\]
Here, $a$ is the associator of $\mathbb{F}\textnormal{Cospan}(\mathrm{A})_1$ and $\mu$ is the first globular isomorphism above. To see that this diagram does indeed commute, we first consider this diagram with respect to only the underlying cospans of each horizontal 1-cell. For notation:
	\[
		\begin{tikzpicture}
			\node (k) at (0,0) {$k$};
			\node (l) at (1,0) {$l$};
			\node (m) at (2,0) {$m$};
			\node (M1) at (-1,0) {$M_1 =$};
			\node (N1) at (3,0) {$N_1 =$};
			\node (q) at (4,0) {$q$};
			\node (r) at (5,0) {$r$};
			\node (s) at (6,0) {$s$};
			\node (P1) at (7,0) {$P_1 =$};
			\node (v) at (8,0) {$v$};
			\node (w) at (9,0) {$w$};
			\node (x) at (10,0) {$x$};
			\node (m2) at (0,-1.5) {$m$};
			\node (n) at (1,-1.5) {$n$};
			\node (p) at (2,-1.5) {$p$};
			\node (M2) at (-1,-1.5) {$M_2 =$};
			\node (N2) at (3,-1.5) {$N_2 =$};
			\node (s2) at (4,-1.5) {$s$};
			\node (t) at (5,-1.5) {$t$};
			\node (u) at (6,-1.5) {$u$};
			\node (P2) at (7,-1.5) {$P_2 =$};
			\node (x2) at (8,-1.5) {$x$};
			\node (y) at (9,-1.5) {$y$};
			\node (z) at (10,-1.5) {$z$};
\node (dM1) at (1,-0.5) {$d_{M_1} \in F(l)$};
\node (dM2) at (1,-2) {$d_{M_2} \in F(n)$};
\node (dN1) at (5,-0.5) {$d_{N_1} \in F(r)$};
\node (dN2) at (5,-2) {$d_{N_2} \in F(t)$};
\node (dP1) at (9,-0.5) {$d_{P_1} \in F(w)$};
\node (dP2) at (9,-2) {$d_{P_2} \in F(y)$};
			\path[->,font=\scriptsize,>=angle 90]
			(k) edge node[above]{$$} (l)
			(m) edge node[above]{$$} (l)
			(q) edge node[above]{$$} (r)
			(s) edge node[above]{$$} (r)
			(v) edge node[above]{$$} (w)
			(x) edge node[above]{$$} (w)
			(m2) edge node[above]{$$} (n)
			(p) edge node[above]{$$} (n)
			(s2) edge node[above]{$$} (t)
			(u) edge node[above]{$$} (t)
			(x2) edge node[above]{$$} (y)
			(z) edge node[above]{$$} (y);
		\end{tikzpicture}
	\]
The above diagram then becomes:
\[
		\begin{tikzpicture}
			\node (a) at (-4,0) {$k+m$};
			\node (b) at (1,0) {$((l+r)+w) +_{((m+s)+x)}((n+t)+y)$};
			\node (c) at (6,0) {$v+x$};
			\node (a2) at (-4,1) {$k+m$};
			\node (b2) at (1,1) {$(l+(r+w)) +_{(m+(s+x))}(n+(t+y))$};
			\node (c2) at (6,1) {$v+x$};
                                \node (a3) at (-4,2) {$k+m$};
			\node (b3) at (1,2) {$(l+_m n)+((r+w)+_{(s+x)}(t+y))$};
			\node (c3) at (6,2) {$v+x$};
                                \node (a4) at (-4,3) {$k+m$};
			\node (b4) at (1,3) {$(l+_m n)+((r+_s t)+(w+_x y))$};
			\node (c4) at (6,3) {$v+x$};
                                \node (a5) at (-4,-1) {$k+m$};
			\node (b5) at (1,-1) {$((l+r)+_{(m+s)}(n+t))+(w+_x y)$};
			\node (c5) at (6,-1) {$v+x$};
                                \node (a6) at (-4,-2) {$k+m$};
			\node (b6) at (1,-2) {$((l+_m n)+(r+_s t))+(w+_x y)$};
			\node (c6) at (6,-2) {$v+x$};
                                \node (a7) at (-4,-3) {$k+m$};
			\node (b7) at (1,-3) {$(l+_m n)+((r+_s t)+(w+_x y))$};
			\node (c7) at (6,-3) {$v+x$};
			\path[->,font=\scriptsize,>=angle 90]
			(a) edge node[above]{$$} (b)
			(c) edge node[above]{$$} (b)
                                (a2) edge node[above]{$$} (b2)
			(c2) edge node[above]{$$} (b2)
                                (a) edge node[above]{$$} (a2)
                                (b) edge node[left]{$a \odot a$} (b2)
(b) edge node[right]{$\iota_1$} (b2)
			(c) edge node[above]{$$} (c2)
                                (a3) edge node[above]{$$} (b3)
			(c3) edge node[above]{$$} (b3)
                                (a2) edge node[above]{$$} (a3)
                                (b2) edge node[left]{$\mu$} (b3)
(b2) edge node[right]{$\iota_2$} (b3)
			(c2) edge node[above]{$$} (c3)
                                (a4) edge node[above]{$$} (b4)
			(c4) edge node[above]{$$} (b4)
                                (a3) edge node[above]{$$} (a4)
                                (b3) edge node[left]{$1 \otimes \mu$} (b4)
(b3) edge node[right]{$\iota_3$} (b4)
			(c3) edge node[above]{$$} (c4)
                                (a5) edge node[above]{$$} (b5)
			(c5) edge node[above]{$$} (b5)
                                (a) edge node[above]{$$} (a5)
                                (b) edge node[left]{$\mu$} (b5)
(b) edge node[right]{$\iota_4$} (b5)
			(c) edge node[above]{$$} (c5)
                                (a6) edge node[above]{$$} (b6)
			(c6) edge node[above]{$$} (b6)
                                (a5) edge node[above]{$$} (a6)
                                (b5) edge node[left]{$\mu \otimes 1$} (b6)
 (b5) edge node[right]{$\iota_5$} (b6)
			(c5) edge node[above]{$$} (c6)
                                (a7) edge node[above]{$$} (b7)
			(c7) edge node[above]{$$} (b7)
                                (a6) edge node[above]{$$} (a7)
                                (b6) edge node[left]{$a$} (b7)
(b6) edge node[right]{$\iota_6$} (b7)
			(c6) edge node[above]{$$} (c7);
		\end{tikzpicture}
	\]
Here all of the vertical 1-morphisms on the left and right are identities, the middle vertical 1-morphisms are the 2-morphisms from the previous commutative diagram, and the horizontal vertical 1-morphisms pointing towards the middle are natural maps into each colimit, all of which are naturally isomorphic to each other as all the middle objects are colimits of the same diagram, namely the previous collection of cospans, taken in various ways. The above diagram of maps of cospans can then be visualized as a hexagonal prism in which all the faces commute by identifying the top and the bottom as the same. As for the decorations, each isomorphism $\iota_n$ goes from the domain under the image of the functor $F$ applied to natural isomorphism adjacent to it to the codomain as written, meaning that, for example: $$\iota_1 \colon F(a \odot a)(d_{((M_1 \otimes N_1) \otimes P_1) \odot ((M_2 \otimes N_2) \otimes P_2)}) \to d_{(M_1 \otimes (N_1 \otimes P_1)) \odot (M_2 \otimes (N_2 \otimes P_2))}.$$  The following diagram commutes in the category $F((l+_m n) + ((r+_s t) + (w+_x y)))$:
\[
\begin{tikzpicture}[scale=1.5]
\node (A) at (0,0.5) {$F( a (\mu \otimes 1)\mu)(d_{((M_1 \otimes N_1) \otimes P_1) \odot ((M_2 \otimes N_2) \otimes P_2)})$};
\node (A') at (4.5,0.5) {$F((1 \otimes \mu) \mu)(d_{(M_1 \otimes (N_1 \otimes P_1)) \odot (M_2 \otimes (N_2 \otimes P_2))})$};
\node (B) at (0,-0.5) {$F(a(\mu  \otimes 1))(d_{((M_1 \otimes N_1) \odot (M_2 \otimes N_2)) \otimes (P_1 \odot P_2)})$};
\node (C) at (4.5,-0.5) {$F(1 \otimes \mu)(d_{(M_1 \odot M_2) \otimes ((N_1 \otimes P_1) \odot (N_2 \otimes P_2))})$};
\node (C') at (0,-1.5) {$F(a)(d_{((M_1 \odot M_2) \otimes (N_1 \odot N_2)) \otimes (P_1 \odot P_2)})$};
\node (D) at (4.5,-1.5) {$d_{(M_1 \odot M_2) \otimes ((N_1 \odot N_2) \otimes (P_1 \odot P_2))}$};
\path[->,font=\scriptsize,>=angle 90]
(A) edge node[above]{$F((1 \otimes \mu) \mu)(\iota_1)$} (A')
(A) edge node[left]{$F(a(\mu \otimes 1))(\iota_4)$} (B)
(A') edge node[right]{$F(1 \otimes \mu)(\iota_2)$} (C)
(B) edge node[left]{$F(a)(\iota_5)$} (C')
(C) edge node [right] {$\iota_3$} (D)
(C') edge node [above] {$\iota_6$} (D);
\end{tikzpicture}
\]
since $$F(a(\mu \otimes 1)\mu)(d_{((M_1 \otimes N_1) \otimes P_1) \odot ((M_2 \otimes N_2) \otimes P_2)}) = F((1 \otimes \mu) \mu (a \odot a))(d_{((M_1 \otimes N_1) \otimes P_1) \odot ((M_2 \otimes N_2) \otimes P_2)})$$
as the above underlying diagram of maps of cospans commutes.

Another requirement for a double category to be symmetric monoidal is that the braiding $$\beta_{ ( \_, \_ ) } \colon \mathbb{F}\textnormal{Cospan}(\mathrm{A})_1 \times \mathbb{F}\textnormal{Cospan}(\mathrm{A})_1 \to \mathbb{F}\textnormal{Cospan}(\mathrm{A})_1 \times \mathbb{F}\textnormal{Cospan}(\mathrm{A})_1$$ be a transformation of double categories, and one of the diagrams that is required to commute is the following:
\[
\begin{tikzpicture}[scale=1.5]
\node (A) at (0,0) {$(M_1 \odot M_2) \otimes (N_1 \odot N_2)$};
\node (B) at (3,0) {$(N_1 \odot N_2) \otimes (M_1 \odot M_2)$};
\node (C) at (0,-.75) {$(M_1 \otimes N_1) \odot (M_2 \otimes N_2)$};
\node (D) at (3,-.75) {$(N_1 \otimes M_1) \odot (N_2 \otimes M_2)$};
\path[->,font=\scriptsize,>=angle 90]
(A) edge node[above]{$\beta$} (B)
(B) edge node[right]{$\mu$} (D)
(A) edge node[left]{$\mu$} (C)
(C) edge node[above]{$\beta \odot \beta$} (D);
\end{tikzpicture}
\]
Using the same notation as the previous coherence diagram, the diagram for the underlying maps of cospans becomes:
\[
		\begin{tikzpicture}
			\node (a) at (-4,0) {$k+m$};
			\node (b) at (1,0) {$(l+_m n) + (r+_s t)$};
			\node (c) at (6,0) {$s+u$};
			\node (a2) at (-4,1) {$k+m$};
			\node (b2) at (1,1) {$(r +_s t) + (l +_m n)$};
			\node (c2) at (6,1) {$s+u$};
                                \node (a3) at (-4,2) {$k+m$};
			\node (b3) at (1,2) {$(r+l) +_{(s+m)} (t+n)$};
			\node (c3) at (6,2) {$s+u$};
                                \node (a5) at (-4,-1) {$k+m$};
			\node (b5) at (1,-1) {$(l+r) +_{(m+s)} (n+t)$};
			\node (c5) at (6,-1) {$s+u$};
                                \node (a6) at (-4,-2) {$k+m$};
			\node (b6) at (1,-2) {$(r+l) +_{(s+m)} (t+n)$};
			\node (c6) at (6,-2) {$s+u$};
			\path[->,font=\scriptsize,>=angle 90]
			(a) edge node[above]{$$} (b)
			(c) edge node[above]{$$} (b)
                                (a2) edge node[above]{$$} (b2)
			(c2) edge node[above]{$$} (b2)
                                (a) edge node[above]{$$} (a2)
                                (b) edge node[left]{$\beta$} (b2)
(b) edge node[right]{$\iota_1$} (b2)
			(c) edge node[above]{$$} (c2)
                                (a3) edge node[above]{$$} (b3)
			(c3) edge node[above]{$$} (b3)
                                (a2) edge node[above]{$$} (a3)
                                (b2) edge node[left]{$\mu$} (b3)
(b2) edge node[right]{$\iota_2$} (b3)
			(c2) edge node[above]{$$} (c3)
                                (a5) edge node[above]{$$} (b5)
			(c5) edge node[above]{$$} (b5)
                                (a) edge node[above]{$$} (a5)
                                (b) edge node[left]{$\mu$} (b5)
(b) edge node[right]{$\iota_4$} (b5)
			(c) edge node[above]{$$} (c5)
                                (a6) edge node[above]{$$} (b6)
			(c6) edge node[above]{$$} (b6)
                                (a5) edge node[above]{$$} (a6)
                                (b5) edge node[left]{$\beta \odot \beta$} (b6)
 (b5) edge node[right]{$\iota_5$} (b6)
			(c5) edge node[above]{$$} (c6);
		\end{tikzpicture}
	\]
All the comments about the previous underlying coherence diagram of maps of cospans apply to this one. As for the decorations, the following diagram commutes in the category $F((r+l)+_{(s+m)}(t+n))$:
\[
\begin{tikzpicture}[scale=1.5]
\node (A) at (0,0) {$F(\mu \beta)(d_{(M_1 \odot M_2) \otimes (N_1 \odot N_2)})$};
\node (B) at (3,0) {$F(\mu)(d_{(N_1 \odot N_2) \otimes (M_1 \odot M_2)})$};
\node (C) at (0,-1) {$F(\beta \odot \beta)(d_{(M_1 \otimes N_1) \odot (M_2 \otimes N_2)})$};
\node (D) at (3,-1) {$d_{(N_1 \otimes M_1) \odot (N_2 \otimes M_2)}$};
\path[->,font=\scriptsize,>=angle 90]
(A) edge node[above]{$F(\mu)(\iota_1)$} (B)
(B) edge node[right]{$\iota_2$} (D)
(A) edge node[left]{$F(\beta \odot \beta)(\iota_3)$} (C)
(C) edge node[above]{$\iota_4$} (D);
\end{tikzpicture}
\]
since $$F(\mu \beta)(d_{(M_1 \odot M_2) \otimes (N_1 \odot N_2)}) = F((\beta \odot \beta)\mu)(d_{(M_1 \odot M_2) \otimes (N_1 \odot N_2)})$$
as the above underlying diagram of maps of cospans commutes. The other diagrams are shown to commute similarly.
\end{proof}

\begin{lem}
The double category $\mathbb{F}\textnormal{Cospan}(\mathrm{A})$ is fibrant.
\end{lem}

\begin{proof}
Let $f \colon c \to c^\prime$ be a vertical 1-morphism in $\mathbb{F} \textnormal{Cospan}(\mathrm{A})$. We can lift $f$ to the companion horizontal 1-cell $\hat{f}$:
\[
\begin{tikzpicture}[scale=1.5]
\node (A) at (0,0) {$c$};
\node (B) at (1,0) {$c^\prime$};
\node (C) at (2,0) {$c^\prime$};
\node (D) at (1,-0.5) {$!_{c^\prime} \in F(c^\prime)$};
\path[->,font=\scriptsize,>=angle 90]
(A) edge node[above]{$f$} (B)
(C) edge node[above]{$1$} (B);
\end{tikzpicture}
\]
and then obtain the following two 2-morphisms:
\[
\begin{tikzpicture}[scale=1.5]
\node (A) at (0,0.5) {$c$};
\node (A') at (0,-0.5) {$c^\prime$};
\node (B) at (1,0.5) {$c^\prime$};
\node (C) at (2,0.5) {$c^\prime$};
\node (C') at (2,-0.5) {$c^\prime$};
\node (D) at (1,-0.5) {$c^\prime$};
\node (E) at (3,0.5) {$!_{c^\prime} \in F(c^\prime)$};
\node (F) at (3,-0.5) {$!_{c^\prime} \in F(c^\prime)$};
\node (G) at (4,0.5) {$c$};
\node (H) at (5,0.5) {$c$};
\node (I) at (6,0.5) {$c$};
\node (G') at (4,-0.5) {$c$};
\node (H') at (5,-0.5) {$c^\prime$};
\node (I') at (6,-0.5) {$c^\prime$};
\node (J) at (7,0.5) {$!_{c} \in F(c)$};
\node (K) at (7,-0.5) {$!_{c^\prime} \in F(c^\prime)$};
\node (L) at (1,-1) {$\iota_{c^\prime} = 1_{!_{c^\prime}}$};
\node (M) at (5,-1) {$\iota_f \colon F(f)(!_c) \to !_{c^\prime}$};
\path[->,font=\scriptsize,>=angle 90]
(A) edge node[above]{$f$} (B)
(C) edge node[above]{$1$} (B)
(A) edge node[left]{$f$} (A')
(C) edge node[left]{$1$} (C')
(A') edge node[above] {$1$} (D)
(C') edge node[above] {$1$} (D)
(B) edge node [left] {$1$} (D)
(G) edge node [above] {$1$} (H)
(G) edge node [left] {$1$} (G')
(H) edge node [left] {$f$} (H')
(G') edge node [above] {$f$} (H')
(I) edge node [above] {$1$} (H)
(I) edge node [left] {$f$} (I')
(I') edge node [above] {$1$} (H');
\end{tikzpicture}
\]
which satisfy the equations:
\[
\begin{tikzpicture}[scale=1.5]
\node (N) at (0,1.5) {$c$};
\node (O) at (1,1.5) {$c$};
\node (P) at (2,1.5) {$c$};
\node (Q) at (-1,1.5) {$!_c \in F(c)$};
\node (A) at (0,0.5) {$c$};
\node (A') at (0,-0.5) {$c^\prime$};
\node (B) at (1,0.5) {$c^\prime$};
\node (C) at (2,0.5) {$c^\prime$};
\node (C') at (2,-0.5) {$c^\prime$};
\node (D) at (1,-0.5) {$c^\prime$};
\node (E) at (-1,0.5) {$!_{c^\prime} \in F(c^\prime)$};
\node (F) at (-1,-0.5) {$!_{c^\prime} \in F(c^\prime)$};
\node (G) at (4,1) {$c$};
\node (H) at (5,1) {$c$};
\node (I) at (6,1) {$c$};
\node (G') at (4,0) {$c^\prime$};
\node (H') at (5,0) {$c^\prime$};
\node (I') at (6,0) {$c^\prime$};
\node (J) at (7,1) {$!_{c} \in F(c)$};
\node (K) at (7,0) {$!_{c^\prime} \in F(c^\prime)$};
\node (Q) at (1,-1) {$\iota_f \colon F(f)(!_c) \to !_{c^\prime}$};
\node (L) at (1,-1.5) {$\iota_{c^\prime} = 1_{!_{c^\prime}}$};
\node (M) at (5,-0.5) {$\iota_f \colon F(f)(!_c) \to !_{c^\prime}$};
\node (R) at (3,0.5) {$=$};
\path[->,font=\scriptsize,>=angle 90]
(N) edge node[above]{$1$} (O)
(P) edge node[above]{$1$} (O)
(N) edge node[left]{$1$} (A)
(O) edge node[left]{$f$} (B)
(P) edge node[left]{$f$} (C)
(A) edge node[above]{$f$} (B)
(C) edge node[above]{$1$} (B)
(A) edge node[left]{$f$} (A')
(C) edge node[left]{$1$} (C')
(A') edge node[above] {$1$} (D)
(C') edge node[above] {$1$} (D)
(B) edge node [left] {$1$} (D)
(G) edge node [above] {$1$} (H)
(G) edge node [left] {$f$} (G')
(H) edge node [left] {$f$} (H')
(G') edge node [above] {$1$} (H')
(I) edge node [above] {$1$} (H)
(I) edge node [left] {$f$} (I')
(I') edge node [above] {$1$} (H');
\end{tikzpicture}
\]
\[
\begin{tikzpicture}[scale=1.5]
\node (G) at (-1,0.5) {$c$};
\node (H) at (-1,-0.5)  {$c^\prime$};
\node (I) at (-2,0.5) {$c$};
\node (J) at (-2,-0.5) {$c$};
\node (A) at (0,0.5) {$c$};
\node (A') at (0,-0.5) {$c^\prime$};
\node (B) at (1,0.5) {$c^\prime$};
\node (C) at (2,0.5) {$c^\prime$};
\node (C') at (2,-0.5) {$c^\prime$};
\node (D) at (1,-0.5) {$c^\prime$};
\node (E) at (1,1) {$!_{c^\prime} \in F(c^\prime)$};
\node (F) at (1,-1) {$!_{c^\prime} \in F(c^\prime)$};

\node (L) at (1,-1.5) {$\iota_{c^\prime} = 1_{!_{c^\prime}}$};
\node (E') at (-1,1) {$!_{c} \in F(c)$};
\node (F') at (-1,-1) {$!_{c^\prime} \in F(c^\prime)$};

\node (L') at (-1,-1.5) {$\iota_{f} \colon F(f)(!_c) \to !_{c^\prime}$};

\node (M) at (2.5,0) {$=$};
\node (N) at (3,0.5) {$c$};
\node (O) at (3,-0.5) {$c$};
\node (P) at (4,0.5) {$c^\prime$};
\node (Q) at (4,-0.5) {$c^\prime$};
\node (R) at (5,0.5) {$c^\prime$};
\node (S) at (5,-0.5) {$c^\prime$};
\node (T) at (4,1) {$!_{c^\prime} \in F(c^\prime)$};
\node (U) at (4,-1) {$!_{c^\prime} \in F(c^\prime)$};
\node (V) at (4,-1.5) {$\iota_{c^\prime} = 1_{!_{c^\prime}}$};

\path[->,font=\scriptsize,>=angle 90]
(N) edge node[left]{$1$} (O)
(P) edge node[left]{$1$} (Q)
(R) edge node[left]{$1$} (S)
(N) edge node[above]{$f$} (P)
(O) edge node[above]{$f$} (Q)
(R) edge node[above]{$1$} (P)
(S) edge node[above]{$1$} (Q)

(A) edge node[above]{$f$} (B)
(C) edge node[above]{$1$} (B)
(A) edge node[left]{$f$} (A')
(C) edge node[left]{$1$} (C')
(A') edge node[above] {$1$} (D)
(C') edge node[above] {$1$} (D)
(B) edge node [left] {$1$} (D)
(A) edge node[above]{$1$} (G)
(G) edge node[left]{$f$} (H)
(A') edge node[above]{$1$} (H)
(J) edge node[above] {$f$} (H)
(I) edge node[left] {$1$} (J)
(I) edge node [above] {$1$} (G);
\end{tikzpicture}
\]
The right hand sides of the above two equations are given respectively by the 2-morphisms $U_f$ and $1_{\hat{f}}$. The conjoint of $f$ is given by the $F$-decorated cospan $\check{f}$ which is just the opposite of the companion above:
\[
\begin{tikzpicture}[scale=1.5]
\node (A) at (0,0) {$c^\prime$};
\node (B) at (1,0) {$c^\prime$};
\node (C) at (2,0) {$c$};
\node (D) at (4,0) {$!_{c^\prime} \in F(c^\prime)$};
\path[->,font=\scriptsize,>=angle 90]
(A) edge node[above]{$1$} (B)
(C) edge node[above]{$f$} (B);
\end{tikzpicture}
\]
\end{proof}

The property of being fibrant is what allows us to lift the monoidal structure from the object category of a double category to its arrow category and obtain a symmetric monoidal bicategory. The following result, which only requires fibrancy on vertical 1-isomorphisms (isofibrancy) is due to Shulman \cite{Shul}:

\begin{thm}[Shulman]\label{Shul}
Let $\mathbb{X}$ be an isofibrant symmetric monoidal pseudo double category. Then the horizontal bicategory $H(\mathbb{X})$ of $\mathbb{X}$ is a symmetric monoidal bicategory which has:
\begin{enumerate}
\item{objects as those of $\mathbb{X}$,}
\item{morphisms as horizontal 1-cells of $\mathbb{X}$, and}
\item{2-morphisms as globular 2-morphisms of $\mathbb{X}$.}
\end{enumerate}
\end{thm}

\begin{thm}
There exists a symmetric monoidal bicategory $H(\mathbb{F}\textnormal{Cospan}(\mathrm{A}))$ which has:
\begin{enumerate}
\item{objects as those of $\mathrm{A}$,}
\item{morphisms as pairs:
\[
\begin{tikzpicture}[scale=1.5]
\node (A) at (0,0) {$a$};
\node (B) at (1,0) {$c$};
\node (C) at (2,0) {$b$};
\node (D) at (4,0) {$d \in F(c)$};
\path[->,font=\scriptsize,>=angle 90]
(A) edge node[above]{$i$} (B)
(C) edge node[above]{$o$} (B);
\end{tikzpicture}
\]
and}
\item{2-morphisms as maps of cospans in $\mathrm{A}$ of the form:
\[
\begin{tikzpicture}[scale=1.5]
\node (A) at (0,0) {$a$};
\node (B) at (1,0.5) {$c$};
\node (C) at (2,0) {$b$};
\node (E) at (1,-0.5) {$c^\prime$};
\node (D) at (3,0.5) {$d \in F(c)$};
\node (F) at (3,-0.5) {$d^\prime \in F(c^\prime)$};
\path[->,font=\scriptsize,>=angle 90]
(A) edge node[above]{$i$} (B)
(C) edge node[above]{$o$} (B)
(A) edge node[below]{$i^\prime$} (E)
(B) edge node[left]{$h$} (E)
(C) edge node[below]{$o^\prime$} (E);
\end{tikzpicture}
\]
together with a morphism $\iota \colon F(h)(d) \to d^\prime$ in $F(c^\prime)$.}
\end{enumerate}
\end{thm}

\begin{proof}
This follows immediately by Shulman's Theorem \ref{Shul} above.
\end{proof}

This symmetric monoidal bicategory is a superior version of the symmetric monoidal bicategory constructed earlier by the second author \cite{Cour}. The previous bicategory constructed by the second author suffered even more so from the issue discussed in the introduction. Given a symmetric lax monoidal functor $F \colon \mathrm{A} \to \mathrm{Set}$, a result of Fong \cite{Fong} yields a symmetric monoidal category $F \textnormal{Cospan}(\mathrm{A})$ which has:
\begin{enumerate}
\item{objects as those of $\mathrm{A}$ and}
\item{morphisms as isomorphism classes of decorated cospans in $\mathrm{A}$.}
\end{enumerate}
The second author then using a result of Shulman \cite{Shul} extended this to a bicategory also called $F \textnormal{Cospan}(\mathrm{A})$ which has:
\begin{enumerate}
\item{objects as those of $\mathrm{A}$,}
\item{morphisms as now just decorated cospans in $\mathrm{A}$, and}
\item{2-morphisms as pairs of commuting diagrams:
\[
\begin{tikzpicture}[scale=1.5]
\node (A) at (0,0) {$a$};
\node (B) at (1,0.5) {$c$};
\node (B') at (1,-0.5) {$c^\prime$};
\node (C) at (2,0) {$b$};
\node (D) at (3,0) {$1$};
\node (E) at (4,0.5) {$F(c)$};
\node (E') at (4,-0.5) {$F(c^\prime)$};
\path[->,font=\scriptsize,>=angle 90]
(A) edge node[above]{$i$} (B)
(C) edge node[above]{$o$} (B)
(A) edge node[below]{$i^\prime$} (B')
(C) edge node[below]{$o^\prime$} (B')
(B) edge node [left] {$f$} (B')
(D) edge node [above] {$d$} (E)
(D) edge node [below] {$d^\prime$} (E')
(E) edge node [right] {$F(f)$} (E');
\end{tikzpicture}
\]
}
\end{enumerate}
This was discussed in the introduction and it was mentioned how in the symmetric monoidal category version of $F \textnormal{Cospan}(\mathrm{A})$, that the two single-edged graphs:
\[
\begin{tikzpicture}[scale=1.5]
\node (A) at (0,0) {$v_1$};
\node (B) at (1,0) {$v_2$};
\path[->,font=\scriptsize,>=angle 90]
(A) edge node[above]{$e$} (B);
\end{tikzpicture}
\]
and
\[
\begin{tikzpicture}[scale=1.5]
\node (A) at (0,0) {$v_1$};
\node (B) at (1,0) {$v_2.$};
\path[->,font=\scriptsize,>=angle 90]
(A) edge node[above]{$e^\prime$} (B);
\end{tikzpicture}
\]
resided in distinct isomorphism classes. In the bicategorical version of $F \textnormal{Cospan}(\mathrm{A})$ where we are no longer considering decorated cospans up to isomorphism class, there is no 2-morphism between these two graphs due to the strict commutativity of the triangle to the right above. This problem does not occur in the symmetric monoidal bicategory $H(\mathbb{F}\textnormal{Cospan}(\mathrm{A}))$ and there is in fact a 2-(iso)morphism between these two graphs given by the map $\iota \colon F(f)(d) \to d^\prime$ which maps the edge $e$ to the edge $e^\prime$, where $f$ is the underlying map of vertices.

%Again, let $F \colon \bold{FinSet} \to \bold{Set}$ be the symmetric lax monoidal functor that maps a finite set $N$ to the large set of all possible graph structures whose underlying set of vertices is $N$. Then a result of Fong \cite{Fong} yields a symmetric monoidal category $F\textnormal{Cospan}(\bold{FinSet})$ which has:
%\begin{enumerate}
%\item{finite sets as objects and}
%\item{isomorphism classes of `open' graphs as morphisms, where an open graph is a cospan of finite sets whose apex is equipped with the structure of a graph.}
%\end{enumerate}

%An example of an open graph is:
%\begin{center}
%\begin{tikzpicture}[scale=2.15]
%\node (N) at (0.5,-1.25) {$N=\{v_1,v_2,v_3\}$};
%    \node[circle,draw,inner sep=1pt,fill=gray,color=gray]         (x) at (-1.4,-.43) {};
%    \node at (-1.4,-.9) {$X$};
%    \node       (A) at (0,0) {$v_1$};
%    \node     (B) at (1,0) {$v_2$};
%    \node        (C) at (0.5,-.86) {$v_3$};
%    \node[circle,draw,inner sep=1pt,fill=gray,color=gray]         (y1) at (2.4,-.25) {};
%    \node[circle,draw,inner sep=1pt,fill=gray,color=gray]         (y2) at (2.4,-.61) {};
%    \node at (2.4,-.9) {$Y$};
%    \path (B) edge  [bend right,->-] node[above] {$e_1$} (A);
%    \path (A) edge  [bend right,->-] node[below] {$e_2$} (B);
%    \path (A) edge  [->-] node[left] {$e_3$} (C);
%    \path (C) edge  [->-] node[right] {$e_4$} (B);
%    \path[color=gray, very thick, shorten >=10pt, shorten <=5pt, ->, >=stealth] (x) edge (A);
%    \path[color=gray, very thick, shorten >=10pt, shorten <=5pt, ->, >=stealth] (y1) edge (B);
%    \path[color=gray, very thick, shorten >=10pt, shorten <=5pt, ->, >=stealth] (y2) edge (B);
%\end{tikzpicture}
%\end{center}
%Here, the sets $X$ and $Y$ serve as inputs and outputs. The second author then extended Fong's category using a result of Shulman \cite{Shul} to obtain a symmetric monoidal bicategory which has:
 
%the following two single-edged graphs constituted distinct isomorphism classes:
%\[
%\begin{tikzpicture}[scale=1.5]
%\node (A) at (0,0) {$v_1$};
%\node (B) at (2,0) {$v_2$};
%\node (C) at (0,-2) {$v_1$};
%\node (D) at (2,-2) {$v_2$};
%\path[->,font=\scriptsize,>=angle 90]
%(A) edge node[above]{$e$} (B)
%(C) edge node[above]{$e^\prime$} (D);
%\end{tikzpicture}
%\]

We can the decategorify this symmetric monoidal bicategory to obtain a symmetric monoidal category similar to the one obtained using Fong's result:

\begin{cor}
Given a symmetric lax monoidal pseudofunctor $F \colon \mathrm{A} \to \bold{Cat}$ where $\mathrm{A}$ is a category with finite colimits and whose monoidal structure is given by binary coproducts, there exists a symmetric monoidal category $D(H(\mathbb{F}\textnormal{Cospan}(\mathrm{A})))$ which has:
\begin{enumerate}
\item{objects as those of $\mathrm{A}$ and}
\item{morphisms as isomorphism classes of $F$-decorated cospans of $\mathrm{A}$, where an $F$-decorated cospan is given by a pair:
\[
\begin{tikzpicture}[scale=1.5]
\node (A) at (0,0) {$a$};
\node (B) at (1,0) {$c$};
\node (C) at (2,0) {$b$};
\node (D) at (4,0) {$d \in F(c)$};
\path[->,font=\scriptsize,>=angle 90]
(A) edge node[above]{$i$} (B)
(C) edge node[above]{$o$} (B);
\end{tikzpicture}
\]
Given another $F$-decorated cospan:
\[
\begin{tikzpicture}[scale=1.5]
\node (A) at (0,0) {$a$};
\node (B) at (1,0) {$c^\prime$};
\node (C) at (2,0) {$b$};
\node (D) at (4,0) {$d^\prime \in F(c^\prime)$};
\path[->,font=\scriptsize,>=angle 90]
(A) edge node[below]{$i^\prime$} (B)
(C) edge node[below]{$o^\prime$} (B);
\end{tikzpicture}
\]
these two $F$-decorated cospans are in the same isomorphism class if there exists an isomorphism $f \colon c \to c^\prime$ such that following diagram commutes:
\[
\begin{tikzpicture}[scale=1.5]
\node (A) at (0,0) {$a$};
\node (B') at (1,0.5) {$c$};
\node (B) at (1,-0.5) {$c^\prime$};
\node (C) at (2,0) {$b$};
\path[->,font=\scriptsize,>=angle 90]
(A) edge node[below]{$i^\prime$} (B)
(C) edge node[below]{$o^\prime$} (B)
(A) edge node[above]{$i$} (B')
(C) edge node[above]{$o$} (B')
(B') edge node[left]{$f$} (B);
\end{tikzpicture}
\]
and there exists an isomorphism $\iota \colon F(f)(d) \to d^\prime$ in $F(c^\prime)$.}
\end{enumerate}
\end{cor}
In this symmetric monoidal category, isomorphism classes are as they should morally be, and the instance of two graphs having different edge sets does not prevent them from being in the same isomorphism class due to the isomorphism $\iota$.

\section{Maps of decorated cospan double categories}
Let $F \colon \mathrm{A} \to \bold{Cat}$ be a symmetric lax monoidal pseudofunctor. Then by Theorem \ref{main1} of the previous section, we get a symmetric monoidal double category $\mathbb{F}\textnormal{Cospan}(\mathrm{A})$. This symmetric monoidal double category has:
\begin{enumerate}
\item{objects as those of $\mathrm{A}$,}
\item{vertical 1-morphisms as morphisms of $\mathrm{A}$,}
\item{horizontal 1-cells as pairs:
\[
\begin{tikzpicture}[scale=1.5]
\node (A) at (0,0) {$a$};
\node (B) at (1,0) {$c$};
\node (C) at (2,0) {$b$};
\node (D) at (4,0) {$d \in F(c)$};
\path[->,font=\scriptsize,>=angle 90]
(A) edge node[above]{$i$} (B)
(C) edge node[above]{$o$} (B);
\end{tikzpicture}
\]
and}
\item{2-morphisms as maps of cospans in $\mathrm{A}$
\[
\begin{tikzpicture}[scale=1.5]
\node (A) at (0,0.5) {$a$};
\node (A') at (0,-0.5) {$a^\prime$};
\node (B) at (1,0.5) {$c$};
\node (C) at (2,0.5) {$b$};
\node (C') at (2,-0.5) {$b^\prime$};
\node (D) at (1,-0.5) {$c^\prime$};
\node (E) at (3,0.5) {$d \in F(c)$};
\node (F) at (3,-0.5) {$d^\prime \in F(c^\prime)$};
\path[->,font=\scriptsize,>=angle 90]
(A) edge node[above]{$$} (B)
(C) edge node[above]{$$} (B)
(A) edge node[left]{$f$} (A')
(C) edge node[left]{$g$} (C')
(A') edge node {$$} (D)
(C') edge node {$$} (D)
(B) edge node [left] {$h$} (D);
\end{tikzpicture}
\]
together with a morphism $\iota \colon F(h)(d) \to d^\prime$ in $F(c^\prime)$.}
\end{enumerate}
Given another symmetric lax monoidal pseudofunctor $F^\prime \colon \mathrm{A^\prime} \to \bold{Cat}$, we can obtain another symmetric monoidal double category $\mathbb{F^\prime}\textnormal{Cospan}(\mathrm{A^\prime})$. Then a map from $\mathbb{F}\textnormal{Cospan}(\mathrm{A})$ to $\mathbb{F^\prime}\textnormal{Cospan}(\mathrm{A^\prime})$ will be a double functor $\mathbb{H} \colon \mathbb{F}\textnormal{Cospan}(\mathrm{A}) \to \mathbb{F^\prime}\textnormal{Cospan}(\mathrm{A^\prime})$ whose object component is given by a finite colimit preserving functor $\mathbb{H}_0 = H \colon \mathrm{A} \to \mathrm{A^\prime}$ and whose arrow component is given by a functor $\mathbb{H}_1$ defined on horizontal 1-cells by:
\[
\begin{tikzpicture}[scale=1.5]
\node (A) at (0,0) {$a$};
\node (B) at (1,0) {$c$};
\node (C) at (2,0) {$b$};
\node (D) at (1,-0.5) {$d \in F(c)$};
\node (E) at (2.75,0) {$\mapsto$};
\node (A') at (3.5,0) {$H(a)$};
\node (B') at (4.5,0) {$H(c)$};
\node (C') at (5.5,0) {$H(b)$};
\node (D') at (4.5,-0.5) {$E(d) \in F^\prime(H(c))$};
\path[->,font=\scriptsize,>=angle 90]
(A) edge node[above]{$i$} (B)
(C) edge node[above]{$o$} (B)
(A') edge node[above]{$H(i)$} (B')
(C') edge node[above]{$H(o)$} (B');
\end{tikzpicture}
\]
and on 2-morphisms by:
\[
\begin{tikzpicture}[scale=1.5]
\node (A) at (0,0.5) {$a$};
\node (A') at (0,-0.5) {$a^\prime$};
\node (B) at (1,0.5) {$c$};
\node (C) at (2,0.5) {$b$};
\node (C') at (2,-0.5) {$b^\prime$};
\node (D) at (1,-0.5) {$c^\prime$};
\node (E) at (3,0.5) {$d \in F(c)$};
\node (F) at (3,-0.5) {$d^\prime \in F(c^\prime)$};
\node (G) at (1,-1) {$\iota \colon F(h)(d) \to d^\prime$};
\node (A'') at (4.25,0.5) {$H(a)$};
\node (A''') at (4.25,-0.5) {$H(a^\prime)$};
\node (B'') at (5.25,0.5) {$H(c)$};
\node (C'') at (6.25,0.5) {$H(b)$};
\node (C''') at (6.25,-0.5) {$H(b^\prime)$};
\node (D'') at (5.25,-0.5) {$H(c^\prime)$};
\node (E'') at (7.5,0.5) {$E(d) \in F^\prime(H(c))$};
\node (F'') at (7.5,-0.5) {$E(d^\prime) \in F^\prime(H(c^\prime))$};
\node (G'') at (5.25,-1) {$E(\iota) \colon F^\prime(H(h))(E(d)) \to E(d^\prime)$};
\node (H) at (3.5,0) {$\mapsto$};
\path[->,font=\scriptsize,>=angle 90]
(A) edge node[above]{$$} (B)
(C) edge node[above]{$$} (B)
(A) edge node[left]{$f$} (A')
(C) edge node[left]{$g$} (C')
(A') edge node {$$} (D)
(C') edge node {$$} (D)
(B) edge node [left] {$h$} (D)
(A'') edge node[above]{$$} (B'')
(C'') edge node[above]{$$} (B'')
(A'') edge node[left]{$H(f)$} (A''')
(C'') edge node[left]{$H(g)$} (C''')
(A''') edge node {$$} (D'')
(C''') edge node {$$} (D'')
(B'') edge node [left] {$H(h)$} (D'');
\end{tikzpicture}
\]
where $E \colon \bold{Cat} \to \bold{Cat}$ is a terminal category preserving 2-functor such that the following diagram commutes:
\[
\begin{tikzpicture}[scale=1.5]
\node (A) at (0,0) {$\mathrm{A}$};
\node (B) at (1,0) {$\bold{Cat}$};
\node (C) at (0,-1) {$\mathrm{A^\prime}$};
\node (D) at (1,-1) {$\bold{Cat}$};
\path[->,font=\scriptsize,>=angle 90]
(A) edge node[above]{$F$} (B)
(A) edge node[left]{$H$} (C)
(B) edge node[right]{$E$} (D)
(C) edge node[above]{$F^\prime$} (D);
\end{tikzpicture}
\]
Recall that we can think of the element $d \in F(c)$ as a morphism $d \colon 1 \to F(c)$ and the morphism $\iota \colon F(h)(d) \to d^\prime$ of $F(c^\prime)$ as a 2-morphism  in $\bold{Cat}$:
\[
\begin{tikzpicture}[scale=1.5]
\node (A) at (0,-0.5) {$1$};
\node (B) at (1,0) {$F(c)$};
\node (D) at (1,-1) {$F(c^\prime)$};
\node (C) at (0.75,-0.5) {$\Swarrow$};
\path[->,font=\scriptsize,>=angle 90]
(A) edge node[above]{$d$} (B)
(A) edge node[below]{$d^\prime$} (D)
(B) edge node[right]{$F(h)$} (D);
\end{tikzpicture}
\]
Applying the terminal category preserving 2-functor $E \colon \bold{Cat} \to \bold{Cat}$ to this diagram yields:
\[
\begin{tikzpicture}[scale=1.5]
\node (A) at (0,-0.5) {$E(1) \cong 1$};
\node (B) at (1.5,0) {$E(F(c))$};
\node (D) at (1.5,-1) {$E(F(c^\prime))$};
\node (C) at (1,-0.5) {$\Swarrow$};
\path[->,font=\scriptsize,>=angle 90]
(A) edge node[above]{$E(d)$} (B)
(A) edge node[below]{$E(d^\prime)$} (D)
(B) edge node[right]{$E(F(h))$} (D);
\end{tikzpicture}
\]
Then because the above square commutes, this is the same as 
\[
\begin{tikzpicture}[scale=1.5]
\node (A) at (0,-0.5) {$E(1) \cong 1$};
\node (B) at (1.5,0) {$F^\prime(H(c))$};
\node (D) at (1.5,-1) {$F^\prime(H(c^\prime))$};
\node (C) at (1,-0.5) {$\Swarrow$};
\path[->,font=\scriptsize,>=angle 90]
(A) edge node[above]{$E(d)$} (B)
(A) edge node[below]{$E(d^\prime)$} (D)
(B) edge node[right]{$F^\prime(H((h))$} (D);
\end{tikzpicture}
\]
which is the same as a morphism $E(\iota) \colon F^\prime(H(h))(E(d)) \to E(d^\prime)$ in $F^\prime(H(c^\prime))$. To check that the above recipe is functorial, given two vertically composable 2-morphisms in $\mathbb{F}\textnormal{Cospan}(\mathrm{A})$:
\[
\begin{tikzpicture}[scale=1.5]
\node (A) at (0,0.5) {$a$};
\node (A') at (0,-0.5) {$a^\prime$};
\node (B) at (1.5,0.5) {$c$};
\node (C) at (3,0.5) {$b$};
\node (C') at (3,-0.5) {$b^\prime$};
\node (D) at (1.5,-0.5) {$c^\prime$};
\node (E) at (4.5,0.5) {$d \in F(c)$};
\node (F) at (4.5,-0.5) {$d^\prime \in F(c^\prime)$};
\node (G) at (1.5,-1) {$\iota \colon F(h)(d) \to d^\prime$};
\node (A'') at (0,-1.5) {$a^\prime$};
\node (A''') at (0,-2.5) {$a''$};
\node (B'') at (1.5,-1.5) {$c^\prime$};
\node (C'') at (3,-1.5) {$b^\prime$};
\node (C''') at (3,-2.5) {$b''$};
\node (D'') at (1.5,-2.5) {$c''$};
\node (E'') at (4.5,-1.5) {$d^\prime \in F(c^\prime)$};
\node (F'') at (4.5,-2.5) {$d'' \in F(c'')$};
\node (G'') at (1.5,-3) {$\iota^\prime \colon F(h^\prime)(d^\prime) \to d''$};
\path[->,font=\scriptsize,>=angle 90]
(A) edge node[above]{$$} (B)
(C) edge node[above]{$$} (B)
(A) edge node[left]{$f$} (A')
(C) edge node[left]{$g$} (C')
(A') edge node [above]{$$} (D)
(C') edge node [above]{$$} (D)
(B) edge node [left] {$h$} (D)
(A'') edge node[above]{$$} (B'')
(C'') edge node[above]{$$} (B'')
(A'') edge node[left]{$f^\prime$} (A''')
(C'') edge node[left]{$g^\prime$} (C''')
(A''') edge node [above]{$$} (D'')
(C''') edge node [above]{$$} (D'')
(B'') edge node [left] {$h^\prime$} (D'');
\end{tikzpicture}
\]
if we first compose these, the result is:
\[
\begin{tikzpicture}[scale=1.5]
\node (A) at (0,0.5) {$a$};
\node (A') at (0,-0.5) {$a''$};
\node (B) at (1.5,0.5) {$c$};
\node (C) at (3,0.5) {$b$};
\node (C') at (3,-0.5) {$b''$};
\node (D) at (1.5,-0.5) {$c''$};
\node (E) at (4.5,0.5) {$d \in F(c)$};
\node (F) at (4.5,-0.5) {$d'' \in F(c'')$};
\node (G) at (1.5,-1) {$\iota ^\prime \iota \colon F(h^\prime h)(d) \to d''$};
\path[->,font=\scriptsize,>=angle 90]
(A) edge node[above]{$$} (B)
(C) edge node[above]{$$} (B)
(A) edge node[left]{$f^\prime f$} (A')
(C) edge node[left]{$g^\prime g$} (C')
(A') edge node [above]{$$} (D)
(C') edge node [above]{$$} (D)
(B) edge node [left] {$h^\prime h$} (D);
\end{tikzpicture}
\]
and then the image of this 2-morphism under the double functor $\mathbb{H}$ is given by:
\[
\begin{tikzpicture}[scale=1.5]
\node (A) at (0,0.5) {$H(a)$};
\node (A') at (0,-0.5) {$H(a'')$};
\node (B) at (1.5,0.5) {$H(c)$};
\node (C) at (3,0.5) {$H(b)$};
\node (C') at (3,-0.5) {$H(b'')$};
\node (D) at (1.5,-0.5) {$H(c'')$};
\node (E) at (4.5,0.5) {$E(d) \in F^\prime(H(c))$};
\node (F) at (4.5,-0.5) {$E(d'') \in F^\prime(H(c''))$};
\node (G) at (1.5,-1) {$E(\iota ^\prime \iota) \colon F^\prime(H(h^\prime h))(E(d)) \to E(d'').$};
\path[->,font=\scriptsize,>=angle 90]
(A) edge node[above]{$$} (B)
(C) edge node[above]{$$} (B)
(A) edge node[left]{$H(f^\prime f)$} (A')
(C) edge node[left]{$H(g^\prime g)$} (C')
(A') edge node [above]{$$} (D)
(C') edge node [above]{$$} (D)
(B) edge node [left] {$H(h^\prime h)$} (D);
\end{tikzpicture}
\]
On the other hand, applying the double functor $\mathbb{H}$ first gives:
\[
\begin{tikzpicture}[scale=1.5]
\node (A) at (0,0.5) {$H(a)$};
\node (A') at (0,-0.5) {$H(a^\prime)$};
\node (B) at (1.5,0.5) {$H(c)$};
\node (C) at (3,0.5) {$H(b)$};
\node (C') at (3,-0.5) {$H(b^\prime)$};
\node (D) at (1.5,-0.5) {$H(c^\prime)$};
\node (E) at (4.5,0.5) {$E(d) \in F^\prime(H(c))$};
\node (F) at (4.5,-0.5) {$E(d^\prime) \in F^\prime(H(c^\prime))$};
\node (G) at (1.5,-1) {$E(\iota) \colon F^\prime(H(h))(E(d)) \to E(d^\prime)$};
\node (A'') at (0,-1.5) {$H(a^\prime)$};
\node (A''') at (0,-2.5) {$H(a'')$};
\node (B'') at (1.5,-1.5) {$H(c^\prime)$};
\node (C'') at (3,-1.5) {$H(b^\prime)$};
\node (C''') at (3,-2.5) {$H(b'')$};
\node (D'') at (1.5,-2.5) {$H(c'')$};
\node (E'') at (4.5,-1.5) {$E(d^\prime) \in F^\prime(H(c^\prime))$};
\node (F'') at (4.5,-2.5) {$E(d'') \in F^\prime(H(c''))$};
\node (G'') at (1.5,-3) {$E(\iota^\prime) \colon F^\prime(H(h^\prime))(E(d^\prime)) \to E(d'')$};
\path[->,font=\scriptsize,>=angle 90]
(A) edge node[above]{$$} (B)
(C) edge node[above]{$$} (B)
(A) edge node[left]{$H(f)$} (A')
(C) edge node[left]{$H(g)$} (C')
(A') edge node [above]{$$} (D)
(C') edge node [above]{$$} (D)
(B) edge node [left] {$H(h)$} (D)
(A'') edge node[above]{$$} (B'')
(C'') edge node[above]{$$} (B'')
(A'') edge node[left]{$H(f^\prime)$} (A''')
(C'') edge node[left]{$H(g^\prime)$} (C''')
(A''') edge node [above]{$$} (D'')
(C''') edge node [above]{$$} (D'')
(B'') edge node [left] {$H(h^\prime)$} (D'');
\end{tikzpicture}
\]
and then composing these gives:
\[
\begin{tikzpicture}[scale=1.5]
\node (A) at (0,0.5) {$H(a)$};
\node (A') at (0,-0.5) {$H(a'')$};
\node (B) at (1.5,0.5) {$H(c)$};
\node (C) at (3,0.5) {$H(b)$};
\node (C') at (3,-0.5) {$H(b'')$};
\node (D) at (1.5,-0.5) {$H(c'')$};
\node (E) at (4.5,0.5) {$E(d) \in F^\prime(H(c))$};
\node (F) at (4.5,-0.5) {$E(d'') \in F^\prime(H(c''))$};
\node (G) at (1.5,-1) {$E(\iota ^\prime \iota) \colon F^\prime(H(h^\prime h))(E(d)) \to E(d'').$};
\path[->,font=\scriptsize,>=angle 90]
(A) edge node[above]{$$} (B)
(C) edge node[above]{$$} (B)
(A) edge node[left]{$H(f^\prime f)$} (A')
(C) edge node[left]{$H(g^\prime g)$} (C')
(A') edge node [above]{$$} (D)
(C') edge node [above]{$$} (D)
(B) edge node [left] {$H(h^\prime h)$} (D);
\end{tikzpicture}
\]
This double functor $\mathbb{H}$ satisfies the equations $S \mathbb{H}_1 = HS$ and $T \mathbb{H}_1=HT$.

Given two composable horizontal 1-cells $M$ and $N$ in $\mathbb{F}\textnormal{Cospan}(\mathrm{A})$:
\[
\begin{tikzpicture}[scale=1.5]
\node (A) at (0,0) {$a_1$};
\node (B) at (1,0) {$c_1$};
\node (C) at (2,0) {$b$};
\node (D) at (1,-0.5) {$d_1 \in F(c_1)$};
\node (E) at (3,0) {$b$};
\node (F) at (4,0) {$c_2$};
\node (G) at (5,0) {$a_2$};
\node (H) at (4,-0.5) {$d_2 \in F(c_2)$};
\path[->,font=\scriptsize,>=angle 90]
(A) edge node[above]{$i_1$} (B)
(C) edge node[above]{$o_1$} (B)
(E) edge node[above]{$i_2$} (F)
(G) edge node[above]{$o_2$} (F);
\end{tikzpicture}
\]
composing first gives $M \odot N$:
\[
\begin{tikzpicture}[scale=1.5]
\node (A) at (0,0) {$a_1$};
\node (B) at (1.5,0) {$c_1+_b c_2$};
\node (C) at (3,0) {$a_2$};
\node (D) at (1.5,-0.5) {$d \in F(c_1 +_b c_2)$};
\path[->,font=\scriptsize,>=angle 90]
(A) edge node[above]{$\psi j_{c_1} i_1$} (B)
(C) edge node[above]{$\psi j_{c_2} o_2$} (B);
\end{tikzpicture}
\]
where $$d \colon 1 \xrightarrow{\lambda^{-1}} 1 \times 1 \xrightarrow{d_1 \times d_2} F(c_1) \times F(c_2) \xrightarrow{\phi_{c_1,c_2}} F(c_1+c_2) \xrightarrow{F(\psi)}F(c_1 +_b c_2).$$ The image of this horizontal 1-cell is then given by $\mathbb{H}(M \odot N)$:
\[
\begin{tikzpicture}[scale=1.5]
\node (A) at (0,0) {$H(a_1)$};
\node (B) at (1.5,0) {$H(c_1+_b c_2)$};
\node (C) at (3,0) {$H(a_2)$};
\node (D) at (1.5,-0.5) {$E(d) \in F^\prime(H(c_1 +_b c_2))$};
\path[->,font=\scriptsize,>=angle 90]
(A) edge node[above]{$H(\psi j_{c_1} i_1)$} (B)
(C) edge node[above]{$H(\psi j_{c_2} o_2)$} (B);
\end{tikzpicture}
\]
where $$E(d) \colon 1 \xrightarrow{E(d)} E(F(c_1 +_b c_2)) = F^\prime(H(c_1 +_b c_2)).$$ On the other hand, the image of each horizontal 1-cell under the double functor $\mathbb{H}$ is given respectively by $\mathbb{H}(M)$ and $\mathbb{H}(N)$:
\[
\begin{tikzpicture}[scale=1.5]
\node (A) at (0,0) {$H(a_1)$};
\node (B) at (1,0) {$H(c_1)$};
\node (C) at (2,0) {$H(b)$};
\node (D) at (1,-0.5) {$E(d_1) \in F^\prime(H(c_1))$};
\node (E) at (3,0) {$H(b)$};
\node (F) at (4,0) {$H(c_2)$};
\node (G) at (5,0) {$H(a_2)$};
\node (H) at (4,-0.5) {$E(d_2) \in F^\prime(H(c_2))$};
\path[->,font=\scriptsize,>=angle 90]
(A) edge node[above]{$H(i_1)$} (B)
(C) edge node[above]{$H(o_1)$} (B)
(E) edge node[above]{$H(i_2)$} (F)
(G) edge node[above]{$H(o_2)$} (F);
\end{tikzpicture}
\]
Composing these then gives $\mathbb{H}(M) \odot \mathbb{H}(N)$:
\[
\begin{tikzpicture}[scale=1.5]
\node (A) at (0,0) {$H(a_1)$};
\node (B) at (2,0) {$H(c_1)+_{H(b)} H(c_2)$};
\node (C) at (4,0) {$H(a_2)$};
\node (D) at (2,-0.5) {$d^\prime \in F^\prime(H(c_1) +_{H(b)} H(c_2))$};
\path[->,font=\scriptsize,>=angle 90]
(A) edge node[above]{$\Psi j_{H(c_1)} H(i_1)$} (B)
(C) edge node[above]{$\Psi j_{H(c_2)} H(o_2)$} (B);
\end{tikzpicture}
\]
where $$d^\prime \colon 1 \xrightarrow{E(d_1) \times E(d_2)} F^\prime(H(c_1)) \times F^\prime(H(c_2)) \xrightarrow{\Phi_{H(c_1),H(c_2)}} F^\prime(H(c_1)+ H(c_2)) \xrightarrow{F^\prime (H(\Psi))} F^\prime(H(c_1) +_{H(b)} H(c_2)).$$
We then have a comparison constraint: $$\mathbb{H}_{M,N} \colon \mathbb{H}(M) \odot \mathbb{H}(N) \xrightarrow{\sim} \mathbb{H}(M \odot N)$$given by the globular 2-isomorphism:
\[
\begin{tikzpicture}[scale=1.5]
\node (A) at (0,0.5) {$H(a_1)$};
\node (A') at (0,-0.5) {$H(a_1)$};
\node (B) at (2.5,0.5) {$H(c_1)+_{H(b)} H(c_2)$};
\node (C) at (5,0.5) {$H(a_2)$};
\node (C') at (5,-0.5) {$H(a_2)$};
\node (D) at (2.5,-0.5) {$H(c_1 +_b c_2)$};
\node (E) at (7,0.5) {$d^\prime \in F^\prime(H(c_1)+_{H(b_1)}H(c_2))$};
\node (F) at (7,-0.5) {$d \in F^\prime(H(c_1 +_b c_2))$};
\node (G) at (2.5,-1) {$\iota_{\kappa^{-1}} \colon F^\prime(\kappa^{-1})(d^\prime) \to d.$};
\path[->,font=\scriptsize,>=angle 90]
(A) edge node[above]{$\Psi j_{H(c_1)} H(i_1)$} (B)
(C) edge node[above]{$\Psi j_{H(c_2)} H(o_2)$} (B)
(A) edge node[left]{$1$} (A')
(C) edge node[left]{$1$} (C')
(A') edge node [above]{$H(\psi j_{c_1} i_1)$} (D)
(C') edge node [above]{$H(\psi j_{c_2} o_2)$} (D)
(B) edge node [left] {$\kappa^{-1}$} (D);
\end{tikzpicture}
\]
where $\kappa$ is the isomorphism $$\kappa \colon H(c_1 +_b c_2) \xrightarrow{\sim} H(c_1) +_{H(b)} H(c_2)$$ which comes from the functor $H \colon \mathrm{A} \to \mathrm{A^\prime}$ preserving finite colimits. The above diagram commutes by a similar argument as to the one used in Theorem \ref{main2}. Similarly we have a unit comparison constraint $$\mathbb{H}_U \colon U_{\mathbb{H}(c)} \to \mathbb{H}(U_c)$$ given by the globular 2-isomorphism:
\[
\begin{tikzpicture}[scale=1.5]
\node (A) at (0,0.5) {$H(c)$};
\node (A') at (0,-0.5) {$H(c)$};
\node (B) at (1.5,0.5) {$H(c)$};
\node (C) at (3,0.5) {$H(c)$};
\node (C') at (3,-0.5) {$H(c)$};
\node (D) at (1.5,-0.5) {$H(c)$};
\node (E) at (4.5,0.5) {$!_{H(c)} \in F^\prime(H(c))$};
\node (F) at (4.5,-0.5) {$E(!_c) \in F^\prime(H(c))$};
\path[->,font=\scriptsize,>=angle 90]
(A) edge node[above]{$1$} (B)
(C) edge node[above]{$1$} (B)
(A) edge node[left]{$1$} (A')
(C) edge node[left]{$1$} (C')
(A') edge node [above]{$1$} (D)
(C') edge node [above]{$1$} (D)
(B) edge node [left] {$1$} (D);
\end{tikzpicture}
\]
where the morphism of decorations is the identity $!_{H(c)} = E(!_c)$ as $EF=F^\prime H$. These comparison constrains satisfy the coherence axioms of a monoidal category, namely:
\[
\begin{tikzpicture}[scale=1.5]
\node (A) at (0,0.5) {$(\mathbb{H}(M) \odot \mathbb{H}(N)) \odot \mathbb{H}(P)$};
\node (B) at (0,-0.5) {$\mathbb{H}(M \odot N) \odot \mathbb{H}(P)$};
\node (C) at (0,-1.5) {$\mathbb{H}((M \odot N) \odot P)$};
\node (A') at (3,0.5) {$\mathbb{H}(M) \odot (\mathbb{H}(N) \odot \mathbb{H}(P))$};
\node (B') at (3,-0.5) {$\mathbb{H}(M) \odot \mathbb{H}(N \odot P)$};
\node (C') at (3,-1.5) {$\mathbb{H}(M \odot (N \odot P))$};
\path[->,font=\scriptsize,>=angle 90]
(A) edge node[left]{$\mathbb{H}_{M,N} \odot 1$} (B)
(B) edge node[left]{$\mathbb{H}_{M \odot N,P}$} (C)
(A) edge node[above]{$a$} (A')
(C) edge node [above] {$\mathbb{H}(a^\prime)$} (C')
(B') edge node [right] {$\mathbb{H}_{M,N \odot P}$} (C')
(A') edge node [right]{$1 \odot \mathbb{H}_{N \odot P}$} (B');
\end{tikzpicture}
\]
\[
\begin{tikzpicture}[scale=1.5]
\node (A) at (0,0) {$U_{\mathbb{H}(a)} \odot \mathbb{H}(M)$};
\node (B) at (2.5,0) {$\mathbb{H}(U_a) \odot \mathbb{H}(M)$};
\node (C) at (0,-1) {$\mathbb{H}(M)$};
\node (D) at (2.5,-1) {$\mathbb{H}(U_a \odot M)$};
\node (A') at (5,0) {$\mathbb{H}(M) \odot U_{\mathbb{H}(b)}$};
\node (B') at (7.5,0) {$\mathbb{H}(M) \odot \mathbb{H}(U_b)$};
\node (C') at (5,-1) {$\mathbb{H}(M)$};
\node (D') at (7.5,-1) {$\mathbb{H}(M \odot U_b)$};
\path[->,font=\scriptsize,>=angle 90]
(A) edge node[above]{$\mathbb{H}_U \odot 1$} (B)
(A) edge node[left]{$\lambda$} (C)
(B) edge node[right]{$\mathbb{H}_{U_a,M}$} (D)
(D) edge node[above]{$\mathbb{H}(\lambda^\prime)$} (C)
(A') edge node[above]{$1 \odot \mathbb{H}_U$} (B')
(A') edge node[left]{$\rho$} (C')
(B') edge node[right]{$\mathbb{H}_{M,U_b}$} (D')
(D') edge node[above]{$\mathbb{H}(\rho^\prime)$} (C');
\end{tikzpicture}
\]
This shows that $\mathbb{H}=(H,E)$ is a double functor. Next we show that this double functor is in fact symmetric monoidal. First, that the object component $\mathbb{H}_0=H$ is symmetric monoidal is clear as $H$ preserves finite colimits. As for the arrow component $\mathbb{H}_1$, given two horizontal 1-cells $M_1$ and $M_2$ in $\mathbb{F}\textnormal{Cospan}(\mathrm{A})$:
\[
\begin{tikzpicture}[scale=1.5]
\node (A) at (0,0) {$a_1$};
\node (B) at (1,0) {$c_1$};
\node (C) at (2,0) {$b_1$};
\node (D) at (1,-0.5) {$d_1 \in F(c_1)$};
\node (E) at (3,0) {$a_2$};
\node (F) at (4,0) {$c_2$};
\node (G) at (5,0) {$b_2$};
\node (H) at (4,-0.5) {$d_2 \in F(c_2)$};
\path[->,font=\scriptsize,>=angle 90]
(A) edge node[above]{$i_1$} (B)
(C) edge node[above]{$o_1$} (B)
(E) edge node[above]{$i_2$} (F)
(G) edge node[above]{$o_2$} (F);
\end{tikzpicture}
\]
their tensor product $M_1 \otimes M_2$ in $\mathbb{F}\textnormal{Cospan}(\mathrm{A})$ is given by:
\[
\begin{tikzpicture}[scale=1.5]
\node (A) at (0,0) {$a_1+a_2$};
\node (B) at (1.5,0) {$c_1+c_2$};
\node (C) at (3,0) {$b_1+b_2$};
\node (D) at (1.5,-0.5) {$d_1+d_2 \in F(c_1)$};
\path[->,font=\scriptsize,>=angle 90]
(A) edge node[above]{$i_1+i_2$} (B)
(C) edge node[above]{$o_1+o_2$} (B);
\end{tikzpicture}
\]
$$d_1+d_2 \colon 1 \xrightarrow{d_1 \times d_2} F(c_1) \times F(c_2) \xrightarrow{\phi_{c_1,c_2}} F(c_1+c_2)$$
and the image of this horizontal 1-cell under the double functor $\mathbb{H}$ is $\mathbb{H}(M_1 \otimes M_2)$ given by:
\[
\begin{tikzpicture}[scale=1.5]
\node (A) at (0,0) {$H(a_1+a_2)$};
\node (B) at (2,0) {$H(c_1+c_2)$};
\node (C) at (4,0) {$H(b_1+b_2)$};
\node (D) at (2,-0.5) {$E(d_1+d_2) \in E(F(c_1+c_2)) = F^\prime(H(c_1+c_2))$};
\path[->,font=\scriptsize,>=angle 90]
(A) edge node[above]{$H(i_1+i_2)$} (B)
(C) edge node[above]{$H(o_1+o_2)$} (B);
\end{tikzpicture}
\]
On the other hand, the image of $M_1$ and $M_2$ is given by $\mathbb{H}(M_1)$ and $\mathbb{H}(M_2)$:
\[
\begin{tikzpicture}[scale=1.5]
\node (A) at (0,0) {$H(a_1)$};
\node (B) at (1,0) {$H(c_1)$};
\node (C) at (2,0) {$H(b_1)$};
\node (D) at (1,-0.5) {$E(d_1) \in F^\prime(H(c_1))$};
\node (E) at (3,0) {$H(a_2)$};
\node (F) at (4,0) {$H(c_2)$};
\node (G) at (5,0) {$H(b_2)$};
\node (H) at (4,-0.5) {$E(d_2) \in F^\prime(H(c_2))$};
\path[->,font=\scriptsize,>=angle 90]
(A) edge node[above]{$H(i_1)$} (B)
(C) edge node[above]{$H(o_1)$} (B)
(E) edge node[above]{$H(i_2)$} (F)
(G) edge node[above]{$H(o_2)$} (F);
\end{tikzpicture}
\]
and their tensor product $\mathbb{H}(M_1) \otimes \mathbb{H}(M_2)$ is given by:
\[
\begin{tikzpicture}[scale=1.5]
\node (A) at (0,0) {$H(a_1)+H(a_2)$};
\node (B) at (2.5,0) {$H(c_1)+H(c_2)$};
\node (C) at (5,0) {$H(b_1)+H(b_2)$};
\node (D) at (2.5,-0.5) {$E(d_1)+E(d_2) \in F^\prime(H(c_1)+H(c_2))$};
\path[->,font=\scriptsize,>=angle 90]
(A) edge node[above]{$H(i_1)+H(i_2)$} (B)
(C) edge node[above]{$H(o_1)+H(o_2)$} (B);
\end{tikzpicture}
\]
$$E(d_1)+E(d_2) \colon 1 \xrightarrow{E(d_1) \times E(d_2)} F^\prime(H(c_1)) \times F^\prime(H(c_2)) \xrightarrow{\Phi_{H(c_1),H(c_2)}} F^\prime (H(c_1)+H(c_2)).$$  We then have a natural 2-isomorphism $\mu_{M_1,M_2} \colon \mathbb{H}(M_1) \otimes \mathbb{H}(M_2) \to \mathbb{H}(M_1 \otimes M_2)$ in $\mathbb{F^\prime}\textnormal{Cospan}(\mathrm{A^\prime})$ given by:
\[
\begin{tikzpicture}[scale=1.5]
\node (A) at (0,0.5) {$H(a_1)+H(a_2)$};
\node (A') at (0,-0.5) {$H(a_1+a_2)$};
\node (B) at (2.5,0.5) {$H(c_1)+H(c_2)$};
\node (C) at (5,0.5) {$H(b_1)+H(b_2)$};
\node (C') at (5,-0.5) {$H(b_1+b_2)$};
\node (D) at (2.5,-0.5) {$H(c_1+c_2)$};
\node (E) at (2.5,1) {$E(d_1)+E(d_2) \in F^\prime(H(c_1)+H(c_2))$};
\node (F) at (2.5,-1) {$E(d_1+d_2) \in F^\prime(H(c_1+c_2))$};
\path[->,font=\scriptsize,>=angle 90]
(A) edge node[above]{$H(i_1)+H(i_2)$} (B)
(C) edge node[above]{$H(o_1)+H(o_2)$} (B)
(A) edge node[left]{$\kappa$} (A')
(C) edge node[left]{$\kappa$} (C')
(A') edge node [above]{$H(i_1+i_2)$} (D)
(C') edge node [above]{$H(o_1+o_2)$} (D)
(B) edge node [left] {$\kappa$} (D);
\end{tikzpicture}
\]
$$\iota_\kappa \colon F^\prime(\kappa)(E(d_1)+E(d_2)) \to E(d_1+d_2)$$
where $\kappa$ denotes the isomorphism arising from $H$ preserving finite colimits. This natural 2-isomorphism together with the associators of $\mathbb{F}\textnormal{Cospan}(\mathrm{A})$ and $\mathbb{F^\prime} \textnormal{Cospan}(\mathrm{A^\prime})$, respectively $\alpha$ and $\alpha^\prime$, make the following diagram commute:
\[
\begin{tikzpicture}[scale=1.5]
\node (A) at (0,0.5) {$(\mathbb{H}(M_1) \otimes \mathbb{H}(M_2)) \otimes \mathbb{H}(M_3)$};
\node (B) at (0,-0.5) {$\mathbb{H}(M_1 \otimes M_2) \otimes \mathbb{H}(M_3)$};
\node (C) at (0,-1.5) {$\mathbb{H}((M_1 \otimes M_2) \otimes M_3)$};
\node (A') at (3,0.5) {$\mathbb{H}(M_1) \otimes (\mathbb{H}(M_2) \otimes \mathbb{H}(M_3))$};
\node (B') at (3,-0.5) {$\mathbb{H}(M_1) \otimes \mathbb{H}(M_2 \otimes M_3)$};
\node (C') at (3,-1.5) {$\mathbb{H}(M_1 \otimes (M_2 \otimes M_3))$};
\path[->,font=\scriptsize,>=angle 90]
(A) edge node[left]{$\mu_{M_1,M_2} \otimes 1$} (B)
(B) edge node[left]{$\mu_{M_1 \otimes M_2,M_3}$} (C)
(A) edge node[above]{$\alpha^\prime$} (A')
(C) edge node [above] {$\mathbb{H}(\alpha)$} (C')
(B') edge node [right] {$\mathbb{H}_{M_1,M_2 \otimes M_3}$} (C')
(A') edge node [right]{$1 \otimes \mu_{M_2 \otimes M_3}$} (B');
\end{tikzpicture}
\]
with the corresponding diagram of decorations:
\[
\begin{tikzpicture}[scale=1.5]
\node (A) at (0,0.5) {$F^\prime(\alpha \kappa \kappa)((E(d_1)+E(d_2))+E(d_3))$};
\node (B) at (0,-0.5) {$F^\prime(\alpha \kappa)(E(d_1+d_2)+E(d_3))$};
\node (C) at (0,-1.5) {$F^\prime(\alpha)(E((d_1+d_2)+d_3))$};
\node (A') at (5,0.5) {$F^\prime(\kappa \kappa)(E(d_1)+(E(d_2)+E(d_3)))$};
\node (B') at (5,-0.5) {$F^\prime(\kappa)(E(d_1)+E(d_2+d_3))$};
\node (C') at (5,-1.5) {$E(d_1+(d_2+d_3))$};
\path[->,font=\scriptsize,>=angle 90]
(A) edge node[left]{$F^\prime(\alpha \kappa)(\iota_\kappa + 1)$} (B)
(B) edge node[left]{$F^\prime(\alpha)(\iota_\kappa)$} (C)
(A) edge node[above]{$F^\prime(\kappa \kappa)(\iota_{\alpha^\prime})$} (A')
(C) edge node [above] {$\iota_\alpha$} (C')
(B') edge node [right] {$\iota_\kappa$} (C')
(A') edge node [right]{$F^\prime(\kappa)(1+\iota_\kappa)$} (B');
\end{tikzpicture}
\]
where $$F^\prime(\alpha \kappa \kappa)((E(d_1)+E(d_2))+E(d_3)) = F^\prime(\kappa \kappa \alpha^\prime)((E(d_1)+E(d_2))+E(d_3)).$$
We also have that the monoidal unit of $\mathbb{F}\textnormal{Cospan}(\mathrm{A})_1$ is given by:
\[
\begin{tikzpicture}[scale=1.5]
\node (A) at (0,0) {$1_\mathrm{A}$};
\node (B) at (1,0) {$1_\mathrm{A}$};
\node (C) at (2,0) {$1_\mathrm{A}$};
\node (D) at (1,-0.5) {$!_{1_\mathrm{A}} \in F(1_\mathrm{A})$};
\path[->,font=\scriptsize,>=angle 90]
(A) edge node[above]{$1$} (B)
(C) edge node[above]{$1$} (B);
\end{tikzpicture}
\]
where $1_\mathrm{A}$ is the monoidal unit of the finitely cocartesian category $\mathrm{A}$. The image of this horizontal 1-cell under $\mathbb{H}$ is given by:
\[
\begin{tikzpicture}[scale=1.5]
\node (A) at (0,0) {$H(1_\mathrm{A})$};
\node (B) at (1,0) {$H(1_\mathrm{A})$};
\node (C) at (2,0) {$H(1_\mathrm{A})$};
\node (E) at (3,0) {$=$};
\node (F) at (4,0) {$1_\mathrm{A^\prime}$};
\node (G) at (5,0) {$1_\mathrm{A^\prime}$};
\node (H) at (6,0) {$1_\mathrm{A^\prime}$};
\node (I) at (5,-0.5) {$!_{1_{\mathrm{A^\prime}}} \in F^\prime(1_{\mathrm{A^\prime}})$};
\node (D) at (1,-0.5) {$E(!_{1_\mathrm{A}}) \in F^\prime(H(1_\mathrm{A}))$};
\path[->,font=\scriptsize,>=angle 90]
(A) edge node[above]{$1$} (B)
(C) edge node[above]{$1$} (B)
(F) edge node[above]{$1$} (G)
(H) edge node[above]{$1$} (G);
\end{tikzpicture}
\]
since $H$ preserves finite colimits. We then have a 2-isomorphism in $\mathbb{F^\prime}\textnormal{Cospan}(\mathrm{A^\prime})$ given by: $$\mu \colon 1_{\mathbb{F^\prime}\textnormal{Cospan}(\mathrm{A^\prime})_1} \to \mathbb{H}(1_{\mathbb{F}\textnormal{Cospan}(\mathrm{A})_1})$$ 
\[
\begin{tikzpicture}[scale=1.5]
\node (A) at (0,0.5) {$1_{\mathrm{A^\prime}}$};
\node (A') at (0,-0.5) {$H(1_\mathrm{A})$};
\node (B) at (1.5,0.5) {$1_{\mathrm{A^\prime}}$};
\node (C) at (3,0.5) {$1_{\mathrm{A^\prime}}$};
\node (C') at (3,-0.5) {$H(1_\mathrm{A})$};
\node (D) at (1.5,-0.5) {$H(1_\mathrm{A})$};
\node (E) at (4.5,0.5) {$!_{1_\mathrm{A^\prime}} \in F^\prime(1_\mathrm{A^\prime})$};
\node (F) at (4.5,-0.5) {$E(!_{1_\mathrm{A}}) \in F^\prime(H(1_\mathrm{A}))$};
\path[->,font=\scriptsize,>=angle 90]
(A) edge node[above]{$1$} (B)
(C) edge node[above]{$1$} (B)
(A) edge node[left]{$\kappa$} (A')
(C) edge node[left]{$\kappa$} (C')
(A') edge node [above]{$1$} (D)
(C') edge node [above]{$1$} (D)
(B) edge node [left] {$\kappa$} (D);
\end{tikzpicture}
\]
together with the morphism $\iota_\mu \colon F^\prime(\kappa)(!_{1_\mathrm{A^\prime}}) \to E(!_{1_\mathrm{A}})$ in $F^\prime(H(1_\mathrm{A}))$. The following square then commutes: 
\[
\begin{tikzpicture}[scale=1.5]
\node (A) at (0,0) {$1_\mathrm{A^\prime} \otimes \mathbb{H}(M)$};
\node (B) at (2.5,0) {$\mathbb{H}(1_\mathrm{A}) \otimes \mathbb{H}(M)$};
\node (C) at (0,-1) {$\mathbb{H}(M)$};
\node (D) at (2.5,-1) {$\mathbb{H}(1_\mathrm{A} + M)$};
\path[->,font=\scriptsize,>=angle 90]
(A) edge node[above]{$\mu \otimes 1$} (B)
(A) edge node[left]{$\ell$} (C)
(B) edge node[right]{$\mu_{1_\mathrm{A},M}$} (D)
(D) edge node[above]{$\mathbb{H}(\ell^\prime)$} (C);
\end{tikzpicture}
\]
where we have abbreviated the monoidal units of $\mathbb{F}\textnormal{Cospan}(\mathrm{A})_1$ and $\mathbb{F^\prime}\textnormal{Cospan}(\mathrm{A^\prime})_1$ as $1_\mathrm{A}$ and $1_\mathrm{A^\prime}$, respectively. The diagram of corresponding decorations is given by:
\[
\begin{tikzpicture}[scale=1.5]
\node (A) at (0,0.5) {$F^\prime(\ell)(!_{1_\mathrm{A^\prime}} + E(d))$};
\node (B) at (0,-0.5) {$E(d)$};
\node (A') at (4,0.5) {$F^\prime(H(\ell^\prime) \kappa)(E(!_{1_\mathrm{A}})+E(d))$};
\node (B') at (4,-0.5) {$F^\prime(H(\ell^\prime))(E(!_{1_\mathrm{A}} + d))$};
\path[->,font=\scriptsize,>=angle 90]
(A) edge node[left]{$\iota_\ell$} (B)
(A) edge node[above]{$F^\prime(H(\ell^\prime)\kappa)(\iota_{\mu+1})$} (A')
(A') edge node [right]{$F^\prime(H(\ell^\prime))(\iota_\kappa)$} (B')
(B') edge node [above] {$\iota_{H(\ell^\prime)}$} (B);
\end{tikzpicture}
\]
where $$F^\prime(\ell)(!_{1_\mathrm{A^\prime}} + E(d))=F^\prime(H(\ell^\prime)\kappa(\mu+1))(!_{1_\mathrm{A^\prime}} + E(d)).$$ The other square involving the right unitors $r$ and $r^\prime$ is similar. Note that because $\mu$ and $\mu_{(\_ , \_)}$ are both isomorphisms, the symmetric monoidal double functor $\mathbb{H}$ is strong.


%\newpage
%Scratch work for Petri net problem
%\newline
%Let $P = (s_1,t_1 \colon T_1 \to \mathbb{N}(S_1))$ and $Q = (s_2,t_2 \colon T_2 \to \mathbb{N}(S_2))$. We want to show that the functor $F \colon \textrm{Petri} \to \textrm{CMC}$ preserves pushouts of cospans of the form $$P \xleftarrow{} L(Y) \xrightarrow{} Q$$ meaning that $F(P+_{L(Y)} Q) \cong F(P) +_{F(L(Y))} F(Q)$. Here, $L \colon \textrm{Set} \to \textrm{Petri}$ is the left adjoint which sends a set $Y$ to the discrete Petri net with $Y$ as its set of species and no transitions. Thus no transitions are identified in the pushout $P+_{L(Y)} Q$, and so the set of transitions of $P+_{L(Y)} Q$ is given by $$\textrm{Trans}(P)+\textrm{Trans}(Q).$$ Thus the morphisms \emph{contributed from transitions} in both $F(P+_{L(Y)}Q)$ and $F(P)+_{F(L(Y)} F(Q)$ are the same. If we can show that $$\textrm{Ob}(F(P+_{L(Y)}Q) \cong \textrm{Ob}(F(P)+_{F(L(Y))} F(Q)$$ then the remaining morphisms in each commutative monoidal category will be canonically isomorphic as sets.

%Now, what can we say about $\textrm{Ob}(F(P+_{L(Y)}Q))$ vs. $\textrm{Ob}(F(P)+_{F(L(Y))} F(Q))$? Well, $\textrm{Ob}(F(P+_{L(Y)}Q)) = \mathbb{N}(S_1+_Y S_2)$ and $\textrm{Ob}(F(P)+_{F(L(Y))} F(Q))= \mathbb{N}(S_1) +_{\mathbb{N}(Y)} \mathbb{N}(S_2)$.









%\newpage










\[
\begin{tikzpicture}[scale=1.5]
\node (A) at (0,0.5) {$1$};
\node (A') at (0,-0.5) {$F^\prime(1_\mathrm{A^\prime})$};
\node (B) at (1.5,0.5) {$F(1_\mathrm{A})$};
\node (C) at (3.5,0.5) {$F(c)$};
\node (C') at (3.5,-0.5) {$F^\prime(H(c))=E(F(c))$};
\path[->,font=\scriptsize,>=angle 90]
(A) edge node[above]{$\phi$} (B)
(B) edge node[above]{$F(!_c)$} (C)
(A) edge node[left]{$\Phi$} (A')
(C) edge node [left] {$$} (C')
(A') edge node [above]{$F^\prime(!_{H(c)})$} (C');
\end{tikzpicture}
\]
\newline
\textbf{show symmetric monoidal, make sure using all assumptions on the functors H and E}









%is given by $$1 \to F(c_1) \times F(c_2) \xrightarrow{} F(c_1 + c_2) \xrightarrow{} F(c_1 + _b c_2)$$ and then applying the functor $E$ results in $1 \to E(F(c_1+_b c_2))$, which by the comutative square is the same as $1 \to F^\prime(H(c_1+_b c_2))$. On the other hand, applying the functor $E$ to each decoration results in $$1 \to E(F(c_1)) \times E(F(c_2)) \xrightarrow{} E(F(c_1) \times F(c_2)) \xrightarrow{} E(F(c_1+c_2)) \xrightarrow E(F(c_1+_b c_2)) \xrightarrow{} F^\prime(H(c_1 +_b c_2))$$













\section{From pseudofunctors to left adjoints}
In this section we investigate the necessary conditions to obtain a category $\mathrm{X}$ and a left adjoint $L \colon \mathrm{A} \to \mathrm{X}$ from a category $\mathrm{A}$ with finite colimits and a symmetric lax monoidal pseudofunctor $F \colon \mathrm{A} \to \bold{Cat}$. The main tool that allows us to do this will be the Grothendieck construction which we recall:

\begin{defn}
Let $\bold{Cat}_{\textnormal{lax},\star}$ denote the \define{2-category of lax-pointed categories} which has:
\begin{enumerate}
\item{objects as pairs $(\mathrm{A},c)$ where $\mathrm{A}$ is a category and $c$ is an object of $\mathrm{A}$, and}
\item{a morphism from $(\mathrm{A},c)$ to $(\mathrm{X},y)$ is a pair $(F,f)$ where $F \colon \mathrm{A} \to \mathrm{X}$ is a functor and $f \colon F(c) \to y$ is a morphism in $\mathrm{X}$.}
\end{enumerate}
\end{defn}

\begin{defn}
Given a pseudofunctor $F \colon \mathrm{A} \to \bold{Cat}$, the \define{Grothendieck construction of} $F$ is given by the `strict 2-pullback' of the following cospan:
 \[
\begin{tikzpicture}[scale=1.5]
\node (D) at (1,1) {$\int F$};
\node (A) at (0,0) {$\mathrm{A}$};
\node (B) at (1,-1) {$\bold{Cat}$};
\node (C) at (2,0) {$\bold{Cat}_{\textnormal{lax},\star}$};
\path[->,font=\scriptsize,>=angle 90]
(D) edge[dashed] node[above]{$p$} (A)
(D) edge[dashed] node[above]{$q$} (C)
(A) edge node[below]{$F$} (B)
(C) edge node[below]{$P$} (B);
\end{tikzpicture}
\]
where $P \colon \bold{Cat}_{\textnormal{lax},\star} \to \bold{Cat}$ is the forgetful functor. This means that $\int F$ is a category which has:
\begin{enumerate}
\item{objects as pairs $(c, d \in F(c))$ and}
\item{a morphism from $(c, d \in F(c))$ to $(c^\prime, d^\prime \in F(c^\prime))$ is a pair $(f \colon c \to c^\prime,\alpha \colon F(f)(d) \to d^\prime)$. 

This can also be thought of as a morphism and a 2-morphism:
\[
\begin{tikzpicture}[scale=1.5]
\node (A) at (0,0) {$\star$};
\node (B) at (1,0.5) {$F(c)$};
\node (C) at (1,-0.5) {$F(c^\prime)$};
\node (D) at (2,0.5) {$c$};
\node (E) at (2,-0.5) {$c^\prime$};
\node (F) at (0.65,0) {$\Swarrow \alpha$};
\path[->,font=\scriptsize,>=angle 90]
(A) edge node[above]{$d$} (B)
(B) edge node[right]{$F(f)$} (C)
(A) edge node[below]{$d^\prime$} (C)
(D) edge node[left] {$f$} (E);
\end{tikzpicture}
\]
}
\end{enumerate}
\end{defn}

\begin{defn}
Given bicategories $\bold{A}$ and $\bold{X}$ and 2-functors (possibly lax or oplax) $(F,\phi),(G,\psi) \colon \bold{A} \to \bold{X}$, a \define{pseudonatural transformation} $\sigma \colon (F,\phi) \to (G,\psi)$ consists of:
\begin{enumerate}
\item{for each object $a \in \bold{A}$, a morphism $\sigma_a \colon F(a) \to G(a)$ in $\bold{X}$ and}
\item{for every pair of objects $a$ and $b$ of $\bold{A}$, we have natural isomorphisms:
\[
\begin{tikzpicture}[scale=1.5]
\node (A) at (0,0) {$\bold{A}(a,b)$};
\node (B) at (2,0) {$\bold{X}(F(a),F(b))$};
\node (C) at (0,-1) {$\bold{X}(G(a),G(b))$};
\node (D) at (2,-1) {$\bold{X}(F(a),G(b))$};
\node (E) at (1,-0.5) {$\sigma_{a,b} \Nearrow$};
\path[->,font=\scriptsize,>=angle 90]
(A) edge node[above]{$F$} (B)
(B) edge node[right]{$(\sigma_{b})_*$} (D)
(A) edge node[left]{$G$} (C)
(C) edge node[above]{$(\sigma_a)^*$} (D);
\end{tikzpicture}
\]
where $(\sigma_a)^*$ and $(\sigma_b)_*$ are the functors induced by precomposition and postcompositon, respectively. Thus for each morphism $f \colon a \to b$ in $\bold{A}$, we have an invertible 2-morphism $\sigma_f \colon G(f) \sigma_a \xrightarrow{\sim} \sigma_{b} F(f)$ in $\bold{X}$:
\[
\begin{tikzpicture}[scale=1.5]
\node (A) at (0,0) {$F(a)$};
\node (B) at (1,0) {$F(b)$};
\node (C) at (0,-1) {$G(a)$};
\node (D) at (1,-1) {$G(b)$};
\node (E) at (0.5,-0.5) {$\sigma_f \Nearrow$};
\path[->,font=\scriptsize,>=angle 90]
(A) edge node[above]{$F(f)$} (B)
(B) edge node[right]{$\sigma_{b}$} (D)
(A) edge node[left]{$\sigma_a$} (C)
(C) edge node[above]{$G(f)$} (D);
\end{tikzpicture}
\]}
\end{enumerate}
such that for each composable pair of morphisms $f \colon a \to b$ and $g \colon b \to c$ of $\bold{A}$, the following diagrams commute:
\[
\begin{tikzpicture}[scale=1.5]
\node (A) at (0,0.5) {$(G(g)G(f))\sigma_a$};
\node (A') at (2,0.5) {$G(g)(G(f) \sigma_a)$};
\node (B) at (0,-0.5) {$G(gf) \sigma_a$};
\node (C) at (4,0.5) {$G(g) (\sigma_b F(f))$};
\node (C') at (4,-0.5) {$\sigma_c F(gf)$};
\node (D) at (6,0.5) {$(G(g) \sigma_b) F(f)$};
\node (D') at (8,-0.5) {$\sigma_c (F(g)F(f))$};
\node (F) at (8,0.5) {$(\sigma_c F(g)) F(f)$};
\path[->,font=\scriptsize,>=angle 90]
(A) edge node[above]{$a^\prime$} (A')
(A) edge node[left]{$\psi 1_{\sigma_a}$} (B)
(A') edge node[above]{$1_{G(g)} \sigma_f$} (C)
(B) edge node[above]{$\sigma_{gf}$} (C')
(C) edge node [above] {${a^\prime}^{-1}$} (D)
(D') edge node [above] {$1_{\sigma_c} \phi$} (C')
(D) edge node [above] {$\sigma_g 1_{F(f)}$} (F)
(F) edge node [right] {$a^\prime$} (D');
\end{tikzpicture}
\]
\[
\begin{tikzpicture}[scale=1.5]
\node (A) at (0,0) {$1_{G(a)} \sigma_a$};
\node (B) at (1,1) {$G(1_a) \sigma_a$};
\node (C) at (2,0) {$\sigma_a F(1_a)$};
\node (D) at (0.5,-1) {$\sigma_a$};
\node (E) at (1.5,-1) {$\sigma_a 1_{F(a)}$};
\path[->,font=\scriptsize,>=angle 90]
(A) edge node[left]{$\psi 1_{\sigma_a}$} (B)
(B) edge node[right]{$\sigma_{1_a}$} (C)
(A) edge node [left] {$\ell^\prime$} (D)
(D) edge node [above] {${\rho^\prime}^{-1}$} (E)
(E) edge node [right] {$1_{\sigma_a} \phi$} (C);
\end{tikzpicture}
\]
\end{defn}

Let $[\bold{A},\bold{Cat}]_{\textnormal{pseudo}}$ denote the 2-category which has:
\begin{enumerate}
\item{objects as pseudofunctors $(F,\phi) \colon \bold{A} \to \bold{Cat}$,}
\item{a morphism from a pseudofunctor $(F,\phi) \colon \bold{A} \to \bold{Cat}$ to another $(G,\psi) \colon \bold{A} \to \bold{Cat}$ is a pseudonatural transformation $\sigma \colon (F,\phi) \to (G,\psi)$, and}
\item{2-morphisms as modifications.}
\end{enumerate}

%\begin{defn}
%Let $P \colon \bold{D} \to \mathrm{A}$ be a functor. A morphism $f \colon d_{1} \to d_{2}$ in the category $\bold{D}$ is \textbf{cartesian (with respect to the functor P)} if for any object $d^\prime$ in $\bold{D}$ and morphism $g \colon d^\prime \to d_{2}$ and every $p \colon P(d^\prime) \to P(d_{1})$ such that $P(g)=P(f)p$, there exists a unique $h \colon d^\prime \to d_{1}$ such that $g=fh$ and $p=P(h)$.
%\[
%\begin{tikzpicture}[scale=1.5]
%\node (A) at (0,0) {$d^\prime$};
%\node (B) at (0,-1.5) {$d_{1}$};
%\node (C) at (1.5,-1.5) {$d_{2}$};
%\node (H) at (2,-0.5) {$P$};
%\node (D) at (2,-.75) {$\mapsto$};
%\node (E) at (3,0) {$P(d^\prime)$};
%\node (F) at (3,-1.5) {$P(d_{1})$};
%\node (G) at (4.5,-1.5) {$P(d_{2})$};
%\path[->,font=\scriptsize,>=angle 90]
%(A) edge[dashed] node[above,left]{$\exists ! h$} (B)
%(A) edge node[above]{$g$} (C)
%(E) edge node[above,left]{$p=P(h)$} (F)
%(F) edge node[above]{$P(f)$} (G)
%(E) edge node[above,right]{$P(g)$} (G)
%(B)edge node[above]{$f$}(C);
%\end{tikzpicture}
%\]
%\end{defn}
\begin{defn}
Let $P \colon \mathrm{D} \to \mathrm{A}$ be a functor. A morphism $f \colon d_{1} \to d_{2}$ in the category $\mathrm{D}$ is \textbf{cocartesian (with respect to the functor P)} if for any object $d^\prime$ in $\mathrm{D}$ and morphism $g \colon d_{1} \to d^\prime$ and every $p \colon P(d_{2}) \to P(d^\prime)$ such that $P(g)=p P(f)$, there exists a unique $h \colon d_{2} \to d^\prime$ such that $g=hf$ and $p=P(h)$.
\[
\begin{tikzpicture}[scale=1.5]
\node (A) at (0,0) {$d^\prime$};
\node (B) at (0,-1) {$d_{2}$};
\node (C) at (1.5,-1) {$d_{1}$};
\node (H) at (2,-0.25) {$P$};
\node (D) at (2,-.5) {$\mapsto$};
\node (E) at (3,0) {$P(d^\prime)$};
\node (F) at (3,-1) {$P(d_{2})$};
\node (G) at (4.5,-1) {$P(d_{1})$};
\path[->,font=\scriptsize,>=angle 90]
(B) edge[dashed] node[above,left]{$\exists ! h$} (A)
(C) edge node[above,right]{$g$} (A)
(F) edge node[above,left]{$p=P(h)$} (E)
(G) edge node[above]{$P(f)$} (F)
(G) edge node[above,right]{$P(g)$} (E)
(C)edge node[above]{$f$}(B);
\end{tikzpicture}
\]
\end{defn}
\begin{defn}
A functor $P \colon \mathrm{D} \to \mathrm{A}$ is a \textbf{(Grothendieck) opfibration} if for any object $d$ in $\bold{D}$ and morphism $f \colon c \to P(d)$ there exists a cocartesian morphism $\phi \colon d^\prime \to d$ such that $P(\phi)=f$.
\end{defn}

\begin{defn}
Let $P \colon \mathrm{D} \to \mathrm{A}$ and $P^\prime \colon \mathrm{D^\prime} \to \mathrm{A}$ be opfibrations over a category $\mathrm{A}$. A \define{morphism of opfibrations} is a functor $F \colon \mathrm{D} \to \mathrm{D}^\prime$ such that $F(p)$ is cocartesian with respect to $P^\prime$ if $p$ is cocartesian with respect to $P$ and $P=P^\prime F$.
\end{defn}

Let Opfib$(\mathrm{A})$ denote the 2-category of opfibrations over $\mathrm{A}$ which has:
\begin{enumerate}
\item{objects as opfibrations $P \colon \mathrm{D} \to \mathrm{A}$,}
\item{morphisms as morphisms of opfibrations, and}
\item{a 2-morphism from one morphism of opfibrations $F \colon P \to P^\prime$ to another $F^\prime \colon P \to P^\prime$ is a natural transformation $\alpha \colon F \to F^\prime$ such that the left whiskering $P^\prime \alpha$ is trivial.}
\end{enumerate}

The 2-category Opfib$(\mathrm{A})$ is a 2-subcategory of $\bold{Cat}/ \mathrm{A}$ where $\bold{Cat} / \mathrm{A}$ is the 2-category of functors over $\mathrm{A}$. The Grothendieck construction is a 2-functor $\int F \colon [ \mathrm{A},\bold{Cat}] \to \bold{Cat}/ \mathrm{A}$ which factors through the image of the embedding $\textnormal{Opfib}(\mathrm{A}) \hookrightarrow \bold{Cat} / \mathrm{A}$: $$\int F \colon [ \mathrm{A}, \bold{Cat} ] \to \textnormal{Opfib}(\mathrm{A}) \hookrightarrow \bold{Cat} / \mathrm{A}.$$

Under certain conditions, the functor resulting from the Grothendieck construction $p \colon \int{F} \to \mathrm{A}$ will be right adjoint to the desired left adjoint $L \colon \mathrm{A} \to \mathrm{X}$.













\section{An equivalence of symmetric monoidal double categories}
In this section we prove that two different frameworks which utilize symmetric monoidal double categories, namely decorated cospans and `structured cospans', are equivalent under the conditions required from the previous section that allow for a left adjoint $L \colon \mathrm{A} \to \mathrm{X}$ to be obtained from a symmetric lax monoidal pseudofunctor $F \colon \mathrm{A} \to \bold{Cat}$. The double category version of decorataed cospans is given in Section \ref{DecCospansDoublecat}. The other framework also utilizing symmetric monoidal double categories due to the first two authors and which goes by the name of `structured cospans', is explained below. Once again following the notation of Shulman \cite{Shul2}, given a double category $\mathbb{A}$, we write $_f \mathbb{A}_g(M,N)$ for the set of 2-morphisms in $\mathbb{A}$ of the form:
\[
  \xymatrix@-.5pc{
    A \ar[r]|{|}^{M}  \ar[d]_f \ar@{}[dr]|{\Downarrow a}&
    B\ar[d]^g\\
    C \ar[r]|{|}_N & D
  }
\]
We call $M$ and $N$ the \define{horizontal source and target} of the 2-morphism $a$, respectively, and likewise we call $f$ and $g$ the \define{vertical source and target} of the 2-morphism $a$, respectively. Thus $_f \mathbb{A}_g(M,N)$ denotes the set of 2-morphisms in $\mathbb{A}$ with horizontal source and target $M$ and $N$ and vertical source and target $f$ and $g$.
\begin{defn}
A (possibly lax or oplax) double functor $\mathbb{F} \colon \mathbb{A} \to \mathbb{X}$ is \define{full} (respectively, \define{faithful}) if $\mathbb{F}_0 \colon \mathbb{A}_0 \to \mathbb{X}_0$ is full (respectively, faithful) and each map $$\mathbb{F}_1 \colon _f \mathbb{A}_g(M,N) \to _{\mathbb{F}(f)} \mathbb{X}_{\mathbb{F}(g)}(\mathbb{F}(M),\mathbb{F}(N))$$ is surjective (respectively, injective).
\end{defn}
\begin{defn}
A (possibly lax or oplax) double functor $\mathbb{F} \colon \mathbb{A} \to \mathbb{X}$ is \define{essentially surjective} if we can simultaneously make the following choices:
\begin{enumerate}
\item{For each object $x \in \mathbb{X}$, we can find an object $a \in \mathbb{A}$ together with a vertical 1-isomorphism $\alpha_x \colon \mathbb{F}(a) \to x$, and}
\item{For each horizontal 1-cell $N \colon x_1 \tobar x_2$  of $\mathbb{X}$, we can find a horizontal 1-cell $M \colon a_1 \tobar a_2$ of $\mathbb{A}$ and a 2-isomorphism $a_{N}$ of $\mathbb{X}$ as in the following diagram:
\[
  \xymatrix@-.5pc{
    \mathbb{F}(a_1) \ar[r]|{|}^{\mathbb{F}(M)}  \ar[d]_{\alpha_{x_1}} \ar@{}[dr]|{\Downarrow a_N}&
    \mathbb{F}(a_1) \ar[d]^{\alpha_{x_2}}\\
    x_1 \ar[r]|{|}_N & x_2
  }
\]
}
\end{enumerate}
\end{defn}
\begin{defn}
A double functor $\mathbb{F} \colon \mathbb{A} \to \mathbb{X}$ is \define{strong} if the comparison and unit constraints are globular isomorphisms, meaning that for each composable pair of horizontal 1-cells $M$ and $N$ we have a natural isomorphism $$\mathbb{F}_{M,N} \colon \mathbb{F}(M) \odot \mathbb{F}(N) \xrightarrow{\sim} \mathbb{F}(M \odot N)$$and for each object $a \in \mathbb{A}$ a natural isomorphism $$\mathbb{F}_a \colon \hat{U}_{\mathbb{F}(a)} \xrightarrow{\sim} \mathbb{F}(U_a).$$
\end{defn}
\begin{thm}[Shulman,7.8]\label{ShulDubEquiv}
Given a strong double functor $\mathbb{F} \colon \mathbb{A} \to \mathbb{X}$, $\mathbb{F}$ is part of a double equivalence if and only if $\mathbb{F}$ is full, faithful and essentially surjective.
\end{thm}
\begin{prop}
Let $\mathbb{A}$ and $\mathbb{X}$ be symmetric monoidal double categories and let $\mathbb{F} \colon \mathbb{A} \to \mathbb{X}$ be a symmetric monoidal strong double functor. If $\mathbb{F}$ is part of a double equivalence, then $\mathbb{F}$ is in fact part of a symmetric monoidal double equivalence, and $\mathbb{A}$ and $\mathbb{X}$ are equivalent as symmetric monoidal double categories.
\end{prop}
Our next goal is to show that the symmetric monoidal double category $\mathbb{F}\textnormal{Cospan}(\mathrm{A})$ of Section \ref{DecCospansDoublecat} is equivalent as a symmetric monoidal double category to the symmetric monoidal double category $_L \mathbb{C}\textnormal{sp}(\mathrm{X})$ obtained using structured cospans.

\begin{thm}\label{SC}
Given a category $\mathrm{X}$ with finite colimits and a category $\mathrm{A}$ with finite coproducts and a finite coproduct preserving functor $L \colon \mathrm{A} \to \mathrm{X}$ with $\mathrm{A}$ and $\mathrm{X}$ regarded as cocartesian monoidal categories, there exists a symmetric monoidal double category $_L \mathbb{C}\textnormal{sp}(\mathrm{X})$ which has:
\begin{enumerate}
\item{objects given by objects of $\mathrm{A}$,}
\item{vertical 1-morphisms given by morphisms of $\mathrm{A}$,}
\item{horizontal 1-cells given by cospans of $\mathrm{X}$ of the form:
\[
\begin{tikzpicture}[scale=1.5]
\node (A) at (0,0) {$L(c)$};
\node (B) at (1,0) {$x$};
\node (C) at (2,0) {$L(c^\prime)$};
\path[->,font=\scriptsize,>=angle 90]
(A) edge node[above]{$$} (B)
(C) edge node[above]{$$} (B);
\end{tikzpicture}
\]
and}
\item{2-morphisms given by maps of cospans of $\mathrm{X}$ of the form:
\[
\begin{tikzpicture}[scale=1.5]
\node (A) at (0,0) {$L(c_1)$};
\node (B) at (1,0) {$x$};
\node (C) at (2,0) {$L(c_2)$};
\node (A') at (0,-1) {$L(c_1^\prime)$};
\node (B') at (1,-1) {$x^\prime$};
\node (C') at (2,-1) {$L(c_2^\prime)$};
\path[->,font=\scriptsize,>=angle 90]
(A) edge node[above]{$$} (B)
(C) edge node[above]{$$} (B)
(A') edge node[above]{$$} (B')
(C') edge node[above]{$$} (B')
(A) edge node [left]{$L(f)$} (A')
(B) edge node [left]{$\alpha$} (B')
(C) edge node [left]{$L(g)$} (C');
\end{tikzpicture}
\]
Composition of horizontal 1-cells and 2-morphisms is given by pushouts in $\mathrm{X}$ and tensoring of objects is under binary coproducts in $\mathrm{A}$.
}
\end{enumerate}
\end{thm}
\begin{proof}
See the first two authors' work on structured cospans \cite{BC2}.
\end{proof}
In the previous section, we started with a symmetric lax monoidal pseudofunctor $F \colon \mathrm{A} \to \bold{Cat}$ and created a symmetric monoidal double category $\mathbb{F}\textnormal{Cospan}(\mathrm{A})$ which has:
\begin{enumerate}
\item{objects of $\mathrm{A}$ as objects,}
\item{morphisms of $\mathrm{A}$ as vertical 1-morphisms,}
\item{horizontal 1-cells are given by $F$-decorated cospans, which are pairs:
\[
\begin{tikzpicture}[scale=1.5]
\node (A) at (0,0) {$c_1$};
\node (B) at (1,0) {$c$};
\node (C) at (2,0) {$c_2$};
\node (D) at (3,0) {$x \in F(c)$};
\path[->,font=\scriptsize,>=angle 90]
(A) edge node[above]{$i$} (B)
(C) edge node[above]{$o$} (B);
\end{tikzpicture}
\]
and}
\item{2-morphisms are given by maps of cospans in $\mathrm{A}$:
\[
\begin{tikzpicture}[scale=1.5]
\node (A) at (0,0) {$c_1$};
\node (B) at (1,0) {$c$};
\node (C) at (2,0) {$c_2$};
\node (A') at (0,-1) {$c_1^\prime$};
\node (B') at (1,-1) {$c^\prime$};
\node (C') at (2,-1) {$c_2^\prime$};
\node (D) at (3,0) {$x \in F(c)$};
\node (D') at (3,-1) {$x^\prime \in F(c^\prime)$};
\path[->,font=\scriptsize,>=angle 90]
(A) edge node[above]{$i$} (B)
(C) edge node[above]{$o$} (B)
(A') edge node[above]{$i^\prime$} (B')
(C') edge node[above]{$o^\prime$} (B')
(A) edge node [left]{$f$} (A')
(B) edge node [left]{$h$} (B')
(C) edge node [left]{$g$} (C');
\end{tikzpicture}
\]
together with a morphism $\iota \colon F(h)(d) \to d^\prime$ in $F(c^\prime)$.}
\end{enumerate}
We now show that these two symmetric monoidal double categories are equivalent. 
\begin{thm}\label{Equiv}
Let $\mathrm{A}$ and $\mathrm{X}$ be categories with finite colimits and $F \colon \mathrm{A} \to \bold{Cat}$ a symmetric lax monoidal pseudofunctor. Let $L \colon \mathrm{A} \to \int{F} \cong \mathrm{X}$ be the left adjoint of the Grothendieck construction $R \colon \int{F} \to \mathrm{A}$ of $F$. Then the symmetric monoidal double category $_L \mathbb{C}\textnormal{sp}(\mathrm{X})$ utilizing structured cospans and the symmetric monoidal double category $\mathbb{F}\textnormal{Cospan}(\mathrm{A})$ utilizing decorated cospans are equivalent as symmetric monoidal double categories.
\end{thm}
\begin{proof}
To prove this, we define a double functor $\mathbb{E} \colon _L \mathbb{C}\textnormal{sp}(\mathrm{X}) \to \mathbb{F}\textnormal{Cospan}(\mathrm{A})$ as follows: the object component of the double functor $\mathbb{E}$ is given by $\mathbb{E}_0 = \id_{\mathrm{A}}$ as both double categories $_L \mathbb{C}\textnormal{sp}(\mathrm{X})$ and $\mathbb{F}\textnormal{Cospan}(\mathrm{A})$ have objects and morphisms of $\mathrm{A}$ as objects and vertical 1-morphisms, respectively. The functor $\mathbb{E}_0$ is trivially an equivalence of categories.

Given a horizontal 1-cell of $_L \mathbb{C}\textnormal{sp}(\mathrm{X})$, which is a cospan in $\mathrm{X}$ of the form:
\[
\begin{tikzpicture}[scale=1.5]
\node (A) at (0,0) {$L(c)$};
\node (B) at (1,0) {$x$};
\node (C) at (2,0) {$L(c^\prime)$};
\path[->,font=\scriptsize,>=angle 90]
(A) edge node[above]{$i$} (B)
(C) edge node[above]{$o$} (B);
\end{tikzpicture}
\]
the image of this horizontal 1-cell under the arrow component $\mathbb{E}_1$ is the pair:
\[
\begin{tikzpicture}[scale=1.5]
\node (A) at (0,0) {$c$};
\node (B) at (1,0) {$R(x)$};
\node (C) at (2,0) {$c^\prime$};
\node (D) at (3,0) {$x \in F(R(x))$};
\path[->,font=\scriptsize,>=angle 90]
(A) edge node[above]{$R(i) {\eta_c}$} (B)
(C) edge node[above]{$R(o) {\eta_{c^\prime}}$} (B);
\end{tikzpicture}
\]
where $R\colon \mathrm{X} \to \mathrm{A}$ is the right adjoint to the functor $L \colon \mathrm{A} \to \mathrm{X}$ and $\eta \colon 1_{\mathrm{A}} \to RL$ is the unit of the adjunction $L \dashv R$ which is an isomorphism since $L$ is fully faithful. \textbf{Hopefully...actually, it's a `lari'!} Similarly, the image of a 2-morphism in $_L \mathbb{C}\textnormal{sp}(\mathrm{X})$:
\[
\begin{tikzpicture}[scale=1.5]
\node (A) at (0,0) {$L(c_1)$};
\node (B) at (1,0) {$x$};
\node (C) at (2,0) {$L(c_2)$};
\node (A') at (0,-1) {$L(c_1^\prime)$};
\node (B') at (1,-1) {$x^\prime$};
\node (C') at (2,-1) {$L(c_2^\prime)$};
\path[->,font=\scriptsize,>=angle 90]
(A) edge node[above]{$i$} (B)
(C) edge node[above]{$o$} (B)
(A') edge node[above]{$i^\prime$} (B')
(C') edge node[above]{$o^\prime$} (B')
(A) edge node [left]{$L(f)$} (A')
(B) edge node [left]{$\alpha$} (B')
(C) edge node [left]{$L(g)$} (C');
\end{tikzpicture}
\]
is the 2-morphism in $\mathbb{F}\textnormal{Cospan}(\mathrm{A})$ given by:
\[
\begin{tikzpicture}[scale=1.5]
\node (A) at (0,0) {$c_1$};
\node (B) at (1,0) {$R(x)$};
\node (C) at (2,0) {$c_2$};
\node (A') at (0,-1) {$c_1^\prime$};
\node (B') at (1,-1) {$R(x^\prime)$};
\node (C') at (2,-1) {$c_2^\prime$};
\node (D) at (3,0) {$x \in F(R(x))$};
\node (D') at (3,-1) {$x^\prime \in F(R(x^\prime))$};
\path[->,font=\scriptsize,>=angle 90]
(A) edge node[above]{$R(i) \eta_{c_1}$} (B)
(C) edge node[above]{$R(o) \eta_{c_2}$} (B)
(A') edge node[above]{$R(i^\prime) \eta_{c_1^\prime}$} (B')
(C') edge node[above]{$R(o^\prime) \eta_{c_2^\prime}$} (B')
(A) edge node [left]{$f$} (A')
(B) edge node [left]{$R(\alpha)$} (B')
(C) edge node [left]{$g$} (C');
\end{tikzpicture}
\]
together with a morphism $\iota \colon F(R(h))(x) \to x^\prime$ in $F(R(x^\prime)) \subseteq \mathrm{X}$ which comes from the Grothendieck construction of the pseudofunctor $F \colon \mathrm{A} \to \bold{Cat}$. First, to see that this functor is essentially surjective, given a horizontal 1-cell in $\mathbb{F}\textnormal{Cospan}(\mathrm{A})$:
\[
\begin{tikzpicture}[scale=1.5]
\node (A) at (0,0) {$c_1$};
\node (B) at (1,0) {$c$};
\node (C) at (2,0) {$c_2$};
\node (D) at (3,0) {$x \in F(c)$};
\path[->,font=\scriptsize,>=angle 90]
(A) edge node[above]{$i$} (B)
(C) edge node[above]{$o$} (B);
\end{tikzpicture}
\]
we can find a 2-isomorphism in $\mathbb{F}\textnormal{Cospan}(\mathrm{A})$ whose codomain is the above horizontal 1-cell and whose domain is the image of the following horizontal 1-cell in $_L \mathbb{C}\textnormal{sp}(\bold{D})$:
\[
\begin{tikzpicture}[scale=1.5]
\node (A) at (0,0) {$L(c_1)$};
\node (B) at (1,0) {$x$};
\node (C) at (2,0) {$L(c_2)$};
\path[->,font=\scriptsize,>=angle 90]
(A) edge node[above]{$i^\prime$} (B)
(C) edge node[above]{$o^\prime$} (B);
\end{tikzpicture}
\]
with the 2-isomorphism in $\mathbb{F}\textnormal{Cospan}(\mathrm{A})$ given by:
\[
\begin{tikzpicture}[scale=1.5]
\node (A) at (0,0) {$c_1$};
\node (B) at (1,0) {$R(x)$};
\node (C) at (2,0) {$c_2$};
\node (A') at (0,-1) {$c_1$};
\node (B') at (1,-1) {$c$};
\node (C') at (2,-1) {$c_2$};
\node (D) at (3,0) {$x \in F(R(x))$};
\node (D') at (3,-1) {$x \in F(c)$};
\path[->,font=\scriptsize,>=angle 90]
(A) edge node[above]{$R(i^\prime) \eta_{c_1}$} (B)
(C) edge node[above]{$R(o^\prime) \eta_{c_2}$} (B)
(A') edge node[above]{$i$} (B')
(C') edge node[above]{$o$} (B')
(A) edge node [left]{$1$} (A')
(B) edge node [left]{${(R(e) \eta_c)}^{-1}$} (B')
(C) edge node [left]{$1$} (C');
\end{tikzpicture}
\]
$$\iota \colon F({(R(e)\eta_c)}^{-1})(x) \to x$$
where $e \colon L(c) \to x$ is given by the map from the trivial decoration on $c$ to $x \in F(c)$. The object and arrow components $\mathbb{E}_0$ and $\mathbb{E}_1$ satisfy the equations $S \mathbb{E}_1 = \mathbb{E}_0 S$ and $T \mathbb{E}_1 = \mathbb{E}_0 T$.

To show that the double functor $\mathbb{E}$ is fully faithful, we need to show that the map  $$\mathbb{E}_1 \colon _f { _L \mathbb{C}\textnormal{sp}(\mathrm{X})}_g(M,N) \to _{\mathbb{E}(f)} {\mathbb{F}\textnormal{Cospan}(\mathrm{A})}_{\mathbb{E}(g)}(\mathbb{E}(M),\mathbb{E}(N))$$ is bijective for arbitrary vertical 1-morphisms $f$ and $g$ and horizontal 1-cells $M$ and $N$ of $_L \mathbb{C}\textnormal{sp}(\bold{D})$. Consider a 2-morphism in $_L \mathbb{C}\textnormal{sp}(\mathrm{X})$:
\[
\begin{tikzpicture}[scale=1.5]
\node (A) at (0,0) {$L(c_1)$};
\node (B) at (1,0) {$x$};
\node (C) at (2,0) {$L(c_2)$};
\node (A') at (0,-1) {$L(c_1^\prime)$};
\node (B') at (1,-1) {$x^\prime$};
\node (C') at (2,-1) {$L(c_2^\prime)$};
\node (D) at (1,0.5) {$M$};
\node (E) at (-1,-0.5) {$f$};
\node (F) at (1,-1.5) {$N$};
\node (G) at (3,-0.5) {$g$};
\path[->,font=\scriptsize,>=angle 90]
(A) edge node[above]{$i$} (B)
(C) edge node[above]{$o$} (B)
(A') edge node[above]{$i^\prime$} (B')
(C') edge node[above]{$o^\prime$} (B')
(A) edge node [left]{$L(f)$} (A')
(B) edge node [left]{$\alpha$} (B')
(C) edge node [left]{$L(g)$} (C');
\end{tikzpicture}
\]
The set $$_f { _L \mathbb{C}\textnormal{sp}(\mathrm{X})}_g(M,N)$$ consists of triples $$(f,\alpha,g)$$ where $f$ and $g$ are morphisms of $\mathrm{A}$ and $\alpha$ is a morphism of $\mathrm{X}$. The image of the above 2-morphism under the double functor $\mathbb{E}$ is given by:
\[
\begin{tikzpicture}[scale=1.5]
\node (A) at (0,0) {$c_1$};
\node (B) at (1,0) {$R(x)$};
\node (C) at (2,0) {$c_2$};
\node (A') at (0,-1) {$c_1^\prime$};
\node (B') at (1,-1) {$R(x^\prime)$};
\node (C') at (2,-1) {$c_2^\prime$};
\node (D) at (1,0.5) {$x \in F(R(x))$};
\node (D') at (1,-1.5) {$x^\prime \in F(R(x^\prime))$};
\node (E) at (1,1) {$\mathbb{E}(M)$};
\node (F) at (-1,-0.5) {$\mathbb{E}(f)$};
\node (G) at (1,-2) {$\mathbb{E}(N)$};
\node (H) at (3,-0.5) {$\mathbb{E}(g)$};
\path[->,font=\scriptsize,>=angle 90]
(A) edge node[above]{$R(i)\eta_{c_1}$} (B)
(C) edge node[above]{$R(o)\eta_{c_2}$} (B)
(A') edge node[above]{$R(i^\prime)\eta_{c_1^\prime}$} (B')
(C') edge node[above]{$R(o^\prime)\eta_{c_2^\prime}$} (B')
(A) edge node [left]{$f$} (A')
(B) edge node [left]{$R(\alpha)$} (B')
(C) edge node [left]{$g$} (C');
\end{tikzpicture}
\]
together with a morphism $\iota \colon F(R(\alpha))(x) \to x^\prime$ of $F(R(x^\prime))$.
Thus the set $$_{\mathbb{E}(f)} {\mathbb{F}\textnormal{Cospan}(\mathrm{A})}_{\mathbb{E}(g)}(\mathbb{E}(M),\mathbb{E}(N))$$ consists of 4-tuples $$(f,R(\alpha),g,\iota).$$ The morphisms $R(\alpha) \colon R(x) \to R(x^\prime)$ and $\iota \colon F(R(\alpha))(x) \to x^\prime$ together carry all of the information of the morphism $\alpha \colon x \to x^\prime$ in $\mathrm{X}$ and conversely; given two objects $x=(c,x \in F(c))$ and $x^\prime=(c^\prime,x^\prime \in F(c^\prime))$ of $\mathrm{X}=\int{F}$, a morphism from $\alpha \colon x \to x^\prime$ is a pair $$(h \colon c \to c^\prime, \iota \colon F(h)(x) \to x^\prime)$$ where $h \colon c \to c^\prime$ is given by $R(\alpha) \colon R(x) \to R(x^\prime)$. This shows that $\mathbb{E}$ is fully faithful. \textbf{At least in my favorite example...}

Next we show that the double functor $\mathbb{E}$ is strong by exhibiting natural isomorphisms $$\mathbb{E}_{M,N} \colon \mathbb{E}(M) \odot \mathbb{E}(N) \xrightarrow{\sim} \mathbb{E}(M \odot N)$$ for every pair of composable horizontal 1-cells $M$ and $N$ of $_L \mathbb{C}\textnormal{sp}(\mathrm{X})$ and for each object $c \in { _L \mathbb{C}\textnormal{sp}(\mathrm{X})}$ a natural isomorphism $$\mathbb{E}_c \colon \hat{U}_{\mathbb{E}(c)} \xrightarrow{\sim} \mathbb{E}(U_c)$$ where $U$ and $\hat{U}$ are the unit structure functors of $_L \mathbb{C} \textnormal{sp}(\mathrm{X})$ and $\mathbb{F}\textnormal{Cospan}(\mathrm{A})$, respectively. For any object $c$, the horizontal 1-cell $\hat{U}_{\mathbb{E}(c)}$ is given by $\hat{U}_c$ which is given by the pair:
\[
\begin{tikzpicture}[scale=1.5]
\node (A) at (0,0) {$c$};
\node (B) at (1,0) {$c$};
\node (C) at (2,0) {$c$};
\node (D) at (3,0) {$!_c \in F(c)$};
\path[->,font=\scriptsize,>=angle 90]
(A) edge node[above]{$1$} (B)
(C) edge node[above]{$1$} (B);
\end{tikzpicture}
\]
The horizontal 1-cell $U_c$ is given by
\[
\begin{tikzpicture}[scale=1.5]
\node (A) at (0,0) {$L(c)$};
\node (B) at (1,0) {$L(c)$};
\node (C) at (2,0) {$L(c)$};
%\node (D) at (3,0.5) {$!_c \in F(c)$};
\path[->,font=\scriptsize,>=angle 90]
(A) edge node[above]{$1$} (B)
(C) edge node[above]{$1$} (B);
\end{tikzpicture}
\]
and so $\mathbb{E}(U_c)$ is given by the pair:
\[
\begin{tikzpicture}[scale=1.5]
\node (A) at (0,0) {$c$};
\node (B) at (1,0) {$R(L(c))$};
\node (C) at (2,0) {$c$};
\node (D) at (3.25,0) {$!_c \in F(R(L(c)))$};
\path[->,font=\scriptsize,>=angle 90]
(A) edge node[above]{$\eta_c$} (B)
(C) edge node[above]{$\eta_c$} (B);
\end{tikzpicture}
\]
Then we can obtain the natural isomorphism $\mathbb{E}_c$ as the 2-morphism
\[
\begin{tikzpicture}[scale=1.5]
\node (A) at (0,0) {$c$};
\node (B) at (1,0) {$c$};
\node (C) at (2,0) {$c$};
\node (A') at (0,-1) {$c$};
\node (B') at (1,-1) {$R(L(c))$};
\node (C') at (2,-1) {$c$};
\node (D) at (3,0) {$!_c \in F(c)$};
\node (D') at (3.25,-1) {$!_c \in F(R(L(c)))$};
\path[->,font=\scriptsize,>=angle 90]
(A) edge node[above]{$1$} (B)
(C) edge node[above]{$1$} (B)
(A') edge node[above]{$\eta_c$} (B')
(C') edge node[above]{$\eta_c$} (B')
(A) edge node [left]{$1$} (A')
(B) edge node [left]{$\eta_c$} (B')
(C) edge node [left]{$1$} (C');
\end{tikzpicture}
\]
$$\iota \colon F(\eta_c)(!_c) \xrightarrow{!} L(c)$$
of $\mathbb{F}\textnormal{Cospan}(\mathrm{A})$. Seems reasonable...

Next, given composable horizontal 1-cells $M$ and $N$ in $_L \mathbb{C}\textnormal{sp}(\mathrm{X})$:
\[
\begin{tikzpicture}[scale=1.5]
\node (A) at (0,0) {$L(c_1)$};
\node (B) at (1,0) {$x$};
\node (C) at (2,0) {$L(c_2)$};
\node (D) at (3,0) {$L(c_2)$};
\node (E) at (4,0) {$x^\prime$};
\node (F) at (5,0) {$L(c_3)$};
%\node (D) at (3,0.5) {$!_c \in F(c)$};
\path[->,font=\scriptsize,>=angle 90]
(A) edge node[above]{$i$} (B)
(C) edge node[above]{$o$} (B)
(D) edge node[above]{$i^\prime$} (E)
(F) edge node[above]{$o^\prime$} (E);
\end{tikzpicture}
\]
their images $\mathbb{E}(M)$ and $\mathbb{E}(N)$ are given by:
\[
\begin{tikzpicture}[scale=1.5]
\node (A) at (0,0) {$c_1$};
\node (B) at (1,0) {$R(x)$};
\node (C) at (2,0) {$c_2$};
\node (D) at (3,0) {$c_2$};
\node (E) at (4,0) {$R(x^\prime)$};
\node (F) at (5,0) {$c_3$};
\node (G) at (1,-0.5) {$x \in F(R(x))$};
\node (H) at (4,-0.5) {$x^\prime \in F(R(x^\prime))$};
%\node (D) at (3,0.5) {$!_c \in F(c)$};
\path[->,font=\scriptsize,>=angle 90]
(A) edge node[above]{$R(i) \eta_{c_1}$} (B)
(C) edge node[above]{$R(o) \eta_{c_2}$} (B)
(D) edge node[above]{$R(i^\prime) \eta_{c_2}$} (E)
(F) edge node[above]{$R(o^\prime) \eta_{c_3}$} (E);
\end{tikzpicture}
\]
and so $\mathbb{E}(M) \odot \mathbb{E}(N)$ is given by:
\[
\begin{tikzpicture}[scale=1.5]
\node (A) at (0,0) {$c_1$};
\node (B) at (1.5,0) {$R(x)+_{c_2}R(x^\prime)$};
\node (C) at (3,0) {$c_3$};
\node (G) at (1.5,-0.5) {$\hat{x} \in F(R(x)+_{c_2}R(x^\prime))$};
%\node (D) at (3,0.5) {$!_c \in F(c)$};
\path[->,font=\scriptsize,>=angle 90]
(A) edge node[above]{$j \psi R(i) \eta_{c_1}$} (B)
(C) edge node[above]{$j \psi R(o^\prime) \eta_{c_3}$} (B);
\end{tikzpicture}
\]
$$\hat{x} \colon 1 \xrightarrow{\lambda^{-1}} 1 \times 1 \xrightarrow{x \times x^\prime} F(R(x)) \times F(R(x^\prime)) \xrightarrow{\phi_{R(x),R(x^\prime)}} F(R(x)+R(x^\prime)) \xrightarrow{F(j_{R(x),R(x^\prime)})} F(R(x)+_{c_2}R(x^\prime))$$where $\psi$ denotes each natural map into the coproduct and $j$ denotes the natural map from the coproduct to the pushout. On the other hand, $M \odot N$ is given by
\[
\begin{tikzpicture}[scale=1.5]
\node (A) at (0,0) {$L(c_1)$};
\node (B) at (1.25,0) {$x+_{L(c_2)}x^\prime$};
\node (C) at (2.5,0) {$L(c_3)$};
%\node (G) at (1,-0.5) {$\hat{d} \in F(R(d)+_{c_2}R(d^\prime))$};
%\node (D) at (3,0.5) {$!_c \in F(c)$};
\path[->,font=\scriptsize,>=angle 90]
(A) edge node[above]{$J \zeta i$} (B)
(C) edge node[above]{$J \zeta o^\prime$} (B);
\end{tikzpicture}
\]
where $\zeta$ is each inclusion into the coproduct and $J$ is the natural map from the coproduct to the pushout. Then $E(M \odot N)$ is given by
\[
\begin{tikzpicture}[scale=1.5]
\node (A) at (0,0) {$c_1$};
\node (B) at (1.5,0) {$R(x+_{L(c_2)}x^\prime)$};
\node (C) at (3,0) {$c_3$};
\node (G) at (1.5,-0.5) {$x+_{L(c_2)} x^\prime \in F(R(x+_{L(c_2)}x^\prime))$};
%\node (D) at (3,0.5) {$!_c \in F(c)$};
\path[->,font=\scriptsize,>=angle 90]
(A) edge node[above]{$R(J \zeta i) \eta_{c_1}$} (B)
(C) edge node[above]{$R(J \zeta o^\prime) \eta_{c_3}$} (B);
\end{tikzpicture}
\]
and so $\mathbb{E}_{M,N} \colon \mathbb{E}(M) \odot \mathbb{E}(N) \xrightarrow{\sim} \mathbb{E}(M \odot N)$ is given by the 2-morphism:
\[
\begin{tikzpicture}[scale=1.5]
\node (A) at (0,0) {$c_1$};
\node (B) at (1.5,0) {$R(x)+_{c_2}R(x^\prime)$};
\node (C) at (3,0) {$c_3$};
\node (A') at (0,-1) {$c_1$};
\node (B') at (1.5,-1) {$R(x+_{L(c_2)}x^\prime)$};
\node (C') at (3,-1) {$c_3$};
\node (D) at (5.5 ,0) {$\hat{x} \in F(R(x)+_{c_2}R(x^\prime))$};
\node (D') at (5.5,-1) {$x+_{L(c_2)}x^\prime \in F(R(x+_{L(c_2)}x^\prime))$};
\path[->,font=\scriptsize,>=angle 90]
(A) edge node[above]{$j \psi R(i) \eta_{c_1}$} (B)
(C) edge node[above]{$j \psi R(o^\prime) \eta_{c_3}$} (B)
(A') edge node[above]{$R(J \zeta i) \eta_{c_1}$} (B')
(C') edge node[above]{$R(J \zeta o^\prime) \eta_{c_3}$} (B')
(A) edge node [left]{$1$} (A')
(B) edge node [left]{$\sigma$} (B')
(C) edge node [left]{$1$} (C');
\end{tikzpicture}
\]
First, if the right adjoint $R \colon \mathrm{X} \to \mathrm{A}$ is also a left adjoint, then $R$ also preserves all colimits and we have an isomorphism $$\kappa \colon R(x) +_{R(L(c_2))} R(x^\prime) \to R(x+_{L(c_2)}x^\prime).$$ Also, since the left adjoint $L \colon \mathrm{A} \to \mathrm{X}$ is fully faithful, the unit of the adjunction $L \dashv R$ at the object $c_2$ gives an isomorphism $\eta_{c_2} \colon c_2 \to R(L(c_2))$ which results in an isomorphism $$j_{\eta_{c_2}} \colon R(x) +_{c_2} R(x^\prime) \to R(x) +_{R(L(c_2))} R(x^\prime).$$ Composing these two results in an isomorphism $$\sigma \coloneqq \kappa j_{\eta_{c_2}} \colon R(x) +_{c_2} R(x^\prime) \to R(x+_{L(c_2)}x^\prime).$$
Next, to see that the above diagram commutes, it suffices to show that for the object $c_1 \in \mathrm{A}$, $$R(J \zeta i)\eta_{c_1}(c_1) = R(J)R(\zeta)R(i)\eta_{c_1}(c_1) \stackrel{!}{=} \sigma j \psi R(i)\eta_{c_1}(c_1).$$ This follows as $R(i) \eta_{c_1} \colon c_1 \to R(x)$ and the following diagram commutes:
\[
\begin{tikzpicture}[scale=1.5]
\node (B) at (0,0) {$R(x)$};
\node (C) at (2,0) {$R(x)+R(x^\prime)$};
\node (A') at (4,0) {$R(x)+_{c_2}R(x^\prime)$};
\node (B') at (4,-2) {$R(x+_{L(c_2)}x^\prime)$};
\node (D) at (0,-2) {$R(x+x^\prime)$};
\node (D') at (4,-1) {$R(x)+_{R(L(c_2))} R(x^\prime)$};
\path[->,font=\scriptsize,>=angle 90]
(C) edge node[above]{$j$} (A')
(B) edge node[above]{$\psi$} (C)
(D) edge node[above]{$R(J)$} (B')
(B) edge node [left]{$R(\zeta)$} (D)
(A') edge node [right]{$j_{\eta_{c_2}}$} (D')
(A') edge [out=345,in=15] node [right]{$\sigma$} (B')
(D') edge node [right]{$\kappa$} (B');
\end{tikzpicture}
\]
Lastly, this map of cospans comes with an isomorphism $\iota \colon F(\sigma)(\hat{x}) \to (x+_{L(c_2)}x^\prime)$ in $F(R(x+_{L(c_2)}x^\prime))$. This shows that $\mathbb{E}$ is strong, and so $\mathbb{E} \colon _L \mathbb{C}\textnormal{sp}(\mathrm{X}) \xrightarrow{\sim} \mathbb{F}\textnormal{Cospan}(\mathrm{A})$ is part of a double equivalence by Theorem \ref{ShulDubEquiv}.

Next, if both double categories $_L \mathbb{C}\textnormal{sp}(\mathrm{X})$ and $\mathbb{F}\textnormal{Cospan}(\mathrm{A})$ are symmetric monoidal, as they are if both $\mathrm{A}$ and $\mathrm{X}$ have finite colimits, then this equivalence of double categories $\mathbb{E} \colon _L \mathbb{C}\textnormal{sp}(\mathrm{X}) \to \mathbb{F} \textnormal{Cospan}(\mathrm{A})$ will be symmetric monoidal. First note that we have an isomorphism $\epsilon \colon 1_{\mathbb{F}\textnormal{Cospan}(\mathrm{A})} \to \mathbb{E}(1_{_L \mathbb{C}\textnormal{sp}(\mathrm{X})})$ and natural isomorphisms $\mu_{c_1,c_2} \colon \mathbb{E}(c_1) \otimes \mathbb{E}(c_2) \to \mathbb{E}(c_1 \otimes c_2)$ for every pair of objects $c_1,c_2 \in {_L \mathbb{C} \textnormal{sp}(\mathrm{X})}$ both of which are given by identities since both double categories $_L \mathbb{C}\textnormal{sp}(\mathrm{X})$ and $\mathbb{F}\textnormal{Cospan}(\mathrm{A})$ have $\mathrm{A}$ as their category of objects and $\mathbb{E}_0=\id_{\mathrm{A}}$. The diagrams containing these morphisms that are required to commute do so trivially.

For the arrow component $\mathbb{E}_1$, we have an isomorphism $\delta \colon U_{1_{\mathbb{F}\textnormal{Cospan}(\mathrm{A})}} \to \mathbb{E}(U_{1_{_L \mathbb{C}\textnormal{sp}(\mathrm{X})}})$ where the horizontal 1-cell $U_{1_{\mathbb{F}\textnormal{Cospan}(\mathrm{A})}}$ is given by:
\[
\begin{tikzpicture}[scale=1.5]
\node (A) at (0,0) {$1_\mathrm{A}$};
\node (B) at (1,0) {$1_\mathrm{A}$};
\node (C) at (2,0) {$1_\mathrm{A}$};
\node (D) at (3,0) {$!_{1_{\mathrm{A}}} \in F(1_\mathrm{A})$};
\path[->,font=\scriptsize,>=angle 90]
(A) edge node[above]{$1$} (B)
(C) edge node[above]{$1$} (B);
\end{tikzpicture}
\]
where $!_{1_{\mathrm{A}}} = \phi \colon 1 \to F(1_\mathrm{A})$ is the trivial decoration which comes from the structure of the symmetric lax monoidal pseudofunctor $F \colon \mathrm{A} \to \bold{Cat}$. The horizontal 1-cell $U_{1_{_L \mathbb{C}\textnormal{sp}(\mathrm{X})}}$ is given by:
\[
\begin{tikzpicture}[scale=1.5]
\node (A) at (0,0) {$L(1_\mathrm{A})$};
\node (B) at (1,0) {$L(1_\mathrm{A})$};
\node (C) at (2,0) {$L(1_\mathrm{A})$};
%\node (D) at (3,0.5) {$I \in F(1_\mathrm{A})$};
\path[->,font=\scriptsize,>=angle 90]
(A) edge node[above]{$1$} (B)
(C) edge node[above]{$1$} (B);
\end{tikzpicture}
\]
where here we make use of the fact that the left adjoint $L \colon (\mathrm{A},+,1_\mathrm{A}) \to (\mathrm{X},+,1_\mathrm{X})$ preserves all colimits and thus $L(1_\mathrm{A}) \cong 1_\mathrm{X}$. The horizontal 1-cell $\mathbb{E}(U_{1_{_L \mathbb{C} \textnormal{sp}(\mathrm{X})}})$ is given by the pair:
\[
\begin{tikzpicture}[scale=1.5]
\node (A) at (0,0) {$1_\mathrm{A}$};
\node (B) at (1.25,0) {$R(L(1_\mathrm{A}))$};
\node (C) at (2.5,0) {$1_\mathrm{A}$};
\node (D) at (4,0) {$!_{1_\mathrm{A}} \in F(R(L(1_\mathrm{A}))) \cong F(1_\mathrm{A})$};
\path[->,font=\scriptsize,>=angle 90]
(A) edge node[above]{${\eta_{1_\mathrm{A}}}$} (B)
(C) edge node[above]{${\eta_{1_\mathrm{A}}}$} (B);
\end{tikzpicture}
\]
The isomorphism $\delta$ is then given by the 2-morphism:
\[
\begin{tikzpicture}[scale=1.5]
\node (A) at (0,0) {$1_\mathrm{A}$};
\node (B) at (1,0) {$1_{\mathrm{A}}$};
\node (C) at (2,0) {$1_\mathrm{A}$};
\node (A') at (0,-1) {$1_\mathrm{A}$};
\node (B') at (1,-1) {$R(L(1_\mathrm{A}))$};
\node (C') at (2,-1) {$1_\mathrm{A}$};
\node (D) at (3,0) {$!_{1_\mathrm{A}} \in F(1_\mathrm{A})$};
\node (D') at (3.75,-1) {$!_{1_\mathrm{A}} \in F(R(L(1_\mathrm{A}))) \cong F(1_\mathrm{A})$};
\path[->,font=\scriptsize,>=angle 90]
(A) edge node[above]{$1$} (B)
(C) edge node[above]{$1$} (B)
(A') edge node[above]{${\eta_{1_\mathrm{A}}}$} (B')
(C') edge node[above]{${\eta_{1_\mathrm{A}}}$} (B')
(A) edge node [left]{$1$} (A')
(B) edge node [left]{${\eta_{1_\mathrm{A}}}$} (B')
(C) edge node [left]{$1$} (C');
\end{tikzpicture}
\]
$$\iota_{{\eta_{1_\mathrm{A}}}} \colon F({\eta_{1_\mathrm{A}}}(!_{1_\mathrm{A}})) \to !_{1_\mathrm{A}}$$
of $\mathbb{F}\textnormal{Cospan}(\mathrm{A})$.

Given two horizontal 1-cells $M$ and $N$ of $_L \mathbb{C}\textnormal{sp}(\mathrm{X})$:
\[
\begin{tikzpicture}[scale=1.5]
\node (A) at (0,0) {$L(c_1)$};
\node (B) at (1,0) {$x$};
\node (C) at (2,0) {$L(c_2)$};
\node (D) at (3,0) {$L(c_1^\prime)$};
\node (E) at (4,0) {$x^\prime$};
\node (F) at (5,0) {$L(c_2^\prime)$};
%\node (D) at (3,0.5) {$!_c \in F(c)$};
\path[->,font=\scriptsize,>=angle 90]
(A) edge node[above]{$i$} (B)
(C) edge node[above]{$o$} (B)
(D) edge node[above]{$i^\prime$} (E)
(F) edge node[above]{$o^\prime$} (E);
\end{tikzpicture}
\]
their images $\mathbb{E}(M)$ and $\mathbb{E}(N)$ are given by:
\[
\begin{tikzpicture}[scale=1.5]
\node (A) at (0,0) {$c_1$};
\node (B) at (1,0) {$R(x)$};
\node (C) at (2,0) {$c_2$};
\node (D) at (3,0) {$c_1^\prime$};
\node (E) at (4,0) {$R(x^\prime)$};
\node (F) at (5,0) {$c_2^\prime$};
\node (G) at (1,-0.5) {$x \in F(R(x))$};
\node (H) at (4,-0.5) {$x^\prime \in F(R(x^\prime))$};
%\node (D) at (3,0.5) {$!_c \in F(c)$};
\path[->,font=\scriptsize,>=angle 90]
(A) edge node[above]{$R(i) \eta_{c_1}$} (B)
(C) edge node[above]{$R(o) \eta_{c_2}$} (B)
(D) edge node[above]{$R(i^\prime) \eta_{c_1^\prime}$} (E)
(F) edge node[above]{$R(o^\prime) \eta_{c_2^\prime}$} (E);
\end{tikzpicture}
\]
and so $\mathbb{E}(M) \otimes \mathbb{E}(N)$ is given by:
\[
\begin{tikzpicture}[scale=1.5]
\node (A) at (0,0) {$c_1+c_1^\prime$};
\node (B) at (2.25,0) {$R(x)+R(x^\prime)$};
\node (C) at (4.5,0) {$c_2+c_2^\prime$};
\node (D) at (2.25,-0.5) {$\hat{x} \in F(R(x)+R(x^\prime))$}; 
%\node (D) at (3,0.5) {$!_c \in F(c)$};
\path[->,font=\scriptsize,>=angle 90]
(A) edge node[above]{$R(i) \eta_{c_1} +R(i^\prime) \eta_{c_1^\prime}$} (B)
(C) edge node[above]{$R(o) \eta_{c_2} + R(o) \eta_{c_2^\prime}$} (B);
\end{tikzpicture}
\]
where $$\hat{x} \colon 1 \xrightarrow{\lambda^{-1}} 1 \times 1 \xrightarrow{x \times x^\prime} F(R(x)) \times F(R(x^\prime)) \xrightarrow{\phi_{R(x),R(x^\prime)}} F(R(x)+R(x^\prime)).$$ On the other hand, $M \otimes N$ is given by
\[
\begin{tikzpicture}[scale=1.5]
\node (A) at (0,0) {$L(c_1+c_1^\prime)$};
\node (B) at (1.5,0) {$x+x^\prime$};
\node (C) at (3,0) {$L(c_2+c_2^\prime)$};
%\node (D) at (3,0.5) {$!_c \in F(c)$};
\path[->,font=\scriptsize,>=angle 90]
(A) edge node[above]{$i+i^\prime$} (B)
(C) edge node[above]{$o+o^\prime$} (B);
\end{tikzpicture}
\]
and $\mathbb{E}(M \otimes N)$ is given by:
\[
\begin{tikzpicture}[scale=1.5]
\node (A) at (0,0) {$c_1+c_1^\prime$};
\node (B) at (2,0) {$R(x+x^\prime)$};
\node (C) at (4,0) {$c_2+c_2^\prime$};
\node (D) at (2,-0.5) {$x+x^\prime \in F(R(x+x^\prime))$};
%\node (D) at (3,0.5) {$!_c \in F(c)$};
\path[->,font=\scriptsize,>=angle 90]
(A) edge node[above]{$R(i+i^\prime) \eta_{c_1+c_1^\prime}$} (B)
(C) edge node[above]{$R(o+o^\prime) \eta_{c_2+c_2^\prime}$} (B);
\end{tikzpicture}
\]
We then have a 2-isomorphism $\mu_{M,N} \colon E(M) \otimes E(N) \xrightarrow{\sim} E(M \otimes N)$ in $\mathbb{F}\textnormal{Cospan}(\mathrm{A})$ given by:
\[
\begin{tikzpicture}[scale=1.5]
\node (A) at (0,0) {$c_1+c_1^\prime$};
\node (B) at (2,0) {$R(x)+R(x^\prime)$};
\node (C) at (4,0) {$c_2+c_2^\prime$};
\node (A') at (0,-1) {$c_1+c_1^\prime$};
\node (B') at (2,-1) {$R(x+x^\prime)$};
\node (C') at (4,-1) {$c_2+c_2^\prime$};
\node (D) at (5.75,0) {$\hat{x} \in F(R(x)+R(x^\prime))$};
\node (D') at (5.75,-1) {$x+x^\prime \in F(R(x+x^\prime))$};
\node (E) at (2,-1.5) {$\iota_\mu \colon F(\kappa)(\hat{x}) \to x+x^\prime$};
\path[->,font=\scriptsize,>=angle 90]
(A) edge node[above]{$R(i)\eta_{c_1} + R(i^\prime)\eta_{c_1^\prime}$} (B)
(C) edge node[above]{$R(o)\eta_{c_2} + R(o^\prime)\eta_{c_2^\prime}$} (B)
(A') edge node[above]{$R(i+i^\prime)\eta_{c_1+c_1^\prime}$} (B')
(C') edge node[above]{$R(o+o^\prime)\eta_{c_2+c_2^\prime}$} (B')
(A) edge node [left]{$1$} (A')
(B) edge node [left]{$\kappa$} (B')
(C) edge node [left]{$1$} (C');
\end{tikzpicture}
\]
The isomorphisms $\delta$ and $\mu$ satisfy the left and right unitality squares, associativity hexagon and braiding square. Let $M_1,M_2$ and $M_3$ be horizontal 1-cells in $_L \mathbb{C}\textnormal{sp}(\mathrm{X})$ given by:
\[
\begin{tikzpicture}[scale=1.5]
\node (A) at (0,0) {$L(c_1)$};
\node (B) at (1,0) {$x_1$};
\node (C) at (2,0) {$L(c_1^\prime)$};
\node (D) at (3,0) {$L(c_2)$};
\node (E) at (4,0) {$x_2$};
\node (F) at (5,0) {$L(c_2^\prime)$};
\node (G) at (6,0) {$L(c_3)$};
\node (H) at (7,0) {$x_3$};
\node (I) at (8,0) {$L(c_3^\prime)$};
%\node (D) at (3,0.5) {$!_c \in F(c)$};
\path[->,font=\scriptsize,>=angle 90]
(A) edge node[above]{$i_1$} (B)
(C) edge node[above]{$o_1$} (B)
(D) edge node[above]{$i_2$} (E)
(F) edge node[above]{$o_2$} (E)
(G) edge node[above]{$i_3$} (H)
(I) edge node[above]{$o_3$} (H);
\end{tikzpicture}
\]
The left unitality square:
\[
\begin{tikzpicture}[scale=1.5]
\node (A) at (0,0) {$1_{\mathbb{F}\textnormal{Cospan}(\mathrm{A})} \otimes \mathbb{E}(M_1)$};
\node (B) at (3,0) {$\mathbb{E}(1_{ _L \mathbb{C}\textnormal{sp}(\mathrm{X})}) \otimes \mathbb{E}(M_1)$};
\node (C) at (0,-1) {$\mathbb{E}(M_1)$};
\node (D) at (3,-1) {$\mathbb{E}(1_{ _L \mathbb{C}\textnormal{sp}(\mathrm{X})} \otimes M_1)$};
\path[->,font=\scriptsize,>=angle 90]
(A) edge node[above]{$\delta \otimes 1$} (B)
(B) edge node[right]{$\mu_{1,M_1}$} (D)
(A) edge node[left]{$\lambda$} (C)
(D) edge node[above]{$\mathbb{E}(\lambda)$} (C);
\end{tikzpicture}
\]
has underlying maps of cospans given by:
\[
		\begin{tikzpicture}
			\node (d) at (7.5,0) {$\mathbb{E}(1_{ _L \mathbb{C}\textnormal{sp}(\mathrm{X})}) \otimes \mathbb{E}(M_1)$};
			\node (a) at (-4,0) {$1_\mathrm{A}+c_1$};
			\node (b) at (0.5,0) {$R(L(1_\mathrm{A})) + R(x_1)$};
			\node (c) at (5,0) {$1_\mathrm{A}+c_1^\prime$};
			\node (d2) at (7.5,1) {$1_{\mathbb{F}\textnormal{Cospan}(\mathrm{A})} \otimes \mathbb{E}(M_1)$};
			\node (a2) at (-4,1) {$1_\mathrm{A}+c_1$};
			\node (b2) at (0.5,1) {$1_\mathrm{A}+R(x_1)$};
			\node (c2) at (5,1) {$1_\mathrm{A}+c_1^\prime$};
			\node (d3) at (7.5,2) {$\mathbb{E}(M_1)$};
                                \node (a3) at (-4,2) {$c_1$};
			\node (b3) at (0.5,2) {$R(x_1)$};
			\node (c3) at (5,2) {$c_1^\prime$};
			\node (d4) at (7.5,-1) {$\mathbb{E}(1_{ _L \mathbb{C}\textnormal{sp}(\mathrm{X})} \otimes M_1)$};
                                \node (a5) at (-4,-1) {$1_\mathrm{A}+c_1^\prime$};
			\node (b5) at (0.5,-1) {$R(L(1_\mathrm{A})+x_1)$};
			\node (c5) at (5,-1) {$1_\mathrm{A}+c_1^\prime$};
			\node (d5) at (7.5,-2) {$\mathbb{E}(M_1)$};
                                \node (a6) at (-4,-2) {$c_1$};
			\node (b6) at (0.5,-2) {$R(x_1)$};
			\node (c6) at (5,-2) {$c_1^\prime$};
			\path[->,font=\scriptsize,>=angle 90]
			(d2) edge node [left]{$\lambda$} (d3)
			(d2) edge node [left] {$\delta \otimes 1$} (d)
			(d) edge node [left] {$\mu_{1,M_1}$} (d4)
			(d4) edge node [left] {$\mathbb{E}(\lambda)$} (d5)
			(a) edge node[above]{$\eta_{1_\mathrm{A}}+R(i_1)\eta_{c_1}$} (b)
			(c) edge node[above]{$\eta_{1_\mathrm{A}}+R(o_1)\eta_{c_1^\prime}$} (b)
                                (a2) edge node[above]{$1+R(i_1)\eta_{c_1}$} (b2)
			(c2) edge node[above]{$1+R(o_1)\eta_{c_1^\prime}$} (b2)
                                (a2) edge node[left]{$1$} (a)
                                (b2) edge node[left]{$\eta_{1_\mathrm{A}}+1$} (b)
(b2) edge node[right]{$\iota_2$} (b)
			(c2) edge node[left]{$1$} (c)
                                (a3) edge node[above]{$R(i_1)\eta_{c_1}$} (b3)
			(c3) edge node[above]{$R(o_1)\eta_{c_1^\prime}$} (b3)
                                (a2) edge node[left]{$\lambda_{\mathrm{A}}$} (a3)
                                (b2) edge node[left]{$\lambda_{\mathrm{A}}$} (b3)
(b2) edge node[right]{$\iota_1$} (b3)
			(c2) edge node[left]{$\lambda_{\mathrm{A}}$} (c3)
                                (a5) edge node[above]{$(\mu_{L(1_\mathrm{A}),d_1})(\eta_{1_\mathrm{A}}+R(i_1)\eta_{c_1})$} (b5)
			(c5) edge node[above]{$(\mu_{L(1_\mathrm{A}),d_1})(\eta_{1_\mathrm{A}}+R(o_1)\eta_{c_1^\prime})$} (b5)
                                (a) edge node[left]{$1$} (a5)
                                (b) edge node[left]{$\mu_{L(1_\mathrm{A}),d_1}$} (b5)
(b) edge node[right]{$\iota_3$} (b5)
			(c) edge node[left]{$1$} (c5)
                                (a6) edge node[above]{$R(i_1)\eta_{c_1}$} (b6)
			(c6) edge node[above]{$R(o_1)\eta_{c_1^\prime}$} (b6)
                                (a5) edge node[left]{$\lambda_{\mathrm{A}}$} (a6)
                                (b5) edge node[left]{$R(\lambda_{\mathrm{X}})$} (b6)
 (b5) edge node[right]{$\iota_4$} (b6)
			(c5) edge node[left]{$\lambda_{\mathrm{A}}$} (c6);
		\end{tikzpicture}
	\]
with the corresponding maps of decorations amounting to the following commutative diagram in $F(R(x_1))$:
\[
\begin{tikzpicture}[scale=1.5]
\node (A) at (-.5,0) {$F(\lambda_{\mathrm{A}})(!_{1_\mathrm{A}}+x_1)$};
\node (C) at (4,0) {$F(R(\lambda_{\mathrm{X}})(\mu_{L(1_\mathrm{A}),x_1}))(!_{R(L(1_\mathrm{A}))} +x_1$)};
\node (D) at (-.5,-1) {$x_1$};
\node (E) at (4,-1) {$F(R(\lambda_\mathrm{X}))(x_{!+1})$};
\path[->,font=\scriptsize,>=angle 90]
(A) edge node[above]{$F(R(\lambda_{\mathrm{X}})(\mu_{L(1_\mathrm{A}),x_1}))(\iota_2)$} (C)
(A) edge node[left]{$\iota_1$} (D)
(E) edge node[above]{$\iota_4$} (D)
(C) edge node[right]{$F(R(\lambda_\mathrm{X}))(\iota_3)$} (E);
\end{tikzpicture}
\]
where $x_{!+1}$ is the decoration $x_1$ on the element $R(L(1_\mathrm{A})+x_1) \in \mathrm{A}$. The above square commutes because $$F(\lambda_\mathrm{A})(!_{1_\mathrm{A}}+x_1) = F(R(\lambda_\mathrm{X})(\mu_{L(1_\mathrm{A}),x_1})(\eta_{1_\mathrm{A}}+1))(!_{1_\mathrm{A}}+x_1)$$ as the above diagram of maps of cospans commutes. The right unitality square is similar. The associator hexagon:
\[
\begin{tikzpicture}[scale=1.5]
\node (A) at (0,0) {$(\mathbb{E}(M_1) \otimes \mathbb{E}(M_2)) \otimes \mathbb{E}(M_3)$};
\node (B) at (3.5,0) {$\mathbb{E}(M_1 \otimes M_2) \otimes \mathbb{E}(M_3)$};
\node (C) at (7,0) {$\mathbb{E}((M_1 \otimes M_2) \otimes M_3)$};
\node (A') at (0,-1) {$\mathbb{E}(M_1) \otimes (\mathbb{E}(M_2) \otimes \mathbb{E}(M_3))$};
\node (B') at (3.5,-1) {$\mathbb{E}(M_1) \otimes \mathbb{E}(M_2 \otimes M_3)$};
\node (C') at (7,-1) {$\mathbb{E}(M_1 \otimes (M_2 \otimes M_3))$};
\path[->,font=\scriptsize,>=angle 90]
(A) edge node[above]{$\mu_{M_1,M_2} \otimes 1$} (B)
(B) edge node[above]{$\mu_{M_1 \otimes M_2,M_3}$} (C)
(A') edge node[above]{$1 \otimes \mu_{M_2,M_3}$} (B')
(B') edge node[above]{$\mu_{M_1,M_2 \otimes M_3}$} (C')
(A) edge node [left]{$a^\prime$} (A')
(C) edge node [left]{$\mathbb{E}(a)$} (C');
\end{tikzpicture}
\]
has underlying maps of cospans given by:
\[
		\begin{tikzpicture}
\node(d) at (1.5,7) {$(\mathbb{E}(M_1) \otimes \mathbb{E}(M_2)) \otimes \mathbb{E}(M_3)$};
\node (d2) at (1.5,8) {$\mathbb{E}(M_1 \otimes M_2) \otimes \mathbb{E}(M_3)$};
\node (d3) at (1.5,9) {$\mathbb{E}((M_1 \otimes M_2) \otimes M_3)$};
\node (d4) at (1.5,10) {$\mathbb{E}(M_1 \otimes (M_2 \otimes M_3))$};
\node (d5) at (1.5,6) {$\mathbb{E}(M_1) \otimes (\mathbb{E}(M_2) \otimes \mathbb{E}(M_3))$};
\node (d6) at (1.5,5) {$\mathbb{E}(M_1) \otimes \mathbb{E}(M_2 \otimes M_3)$};
\node (d7) at (1.5,4) {$\mathbb{E}(M_1 \otimes (M_2 \otimes M_3))$};
			\node (a) at (-4.25,0) {$(c_1+c_2)+c_3$};
			\node (b) at (1.5,0) {$(R(x_1)+R(x_2))+R(x_3)$};
			\node (c) at (7.25,0) {$(c_1^\prime+c_2^\prime)+c_3^\prime$};
			\node (a2) at (-4.25,1) {$(c_1+c_2)+c_3$};
			\node (b2) at (1.5,1) {$R(x_1+x_2)+R(x_3)$};
			\node (c2) at (7.25,1) {$(c_1^\prime+c_2^\prime)+c_3^\prime$};
                                \node (a3) at (-4.25,2) {$(c_1+c_2)+c_3$};
			\node (b3) at (1.5,2) {$R((x_1+x_2)+x_3)$};
			\node (c3) at (7.25,2) {$(c_1^\prime+c_2^\prime)+c_3^\prime$};
                                \node (a4) at (-4.25,3) {$c_1+(c_2+c_3)$};
			\node (b4) at (1.5,3) {$R(x_1+(x_2+x_3))$};
			\node (c4) at (7.25,3) {$c_1^\prime + (c_2^\prime + c_3^\prime)$};
                                \node (a5) at (-4.25,-1) {$c_1+(c_2+c_3)$};
			\node (b5) at (1.5,-1) {$R(x_1)+(R(x_2)+R(x_3))$};
			\node (c5) at (7.25,-1) {$c_1^\prime + (c_2^\prime + c_3^\prime)$};
                                \node (a6) at (-4.25,-2) {$c_1+(c_2+c_3)$};
			\node (b6) at (1.5,-2) {$R(x_1)+R(x_2+x_3)$};
			\node (c6) at (7.25,-2) {$c_1^\prime + (c_2^\prime + c_3^\prime)$};
                                \node (a7) at (-4.25,-3) {$c_1+(c_2+c_3)$};
			\node (b7) at (1.5,-3) {$R(x_1+(x_2+x_3))$};
			\node (c7) at (7.25,-3) {$c_1^\prime + (c_2^\prime + c_3^\prime)$};
			\path[->,font=\scriptsize,>=angle 90]
(d) edge node[left] {$\mu_{M_1,M_2} \otimes 1$} (d2)
(d2) edge node[left] {$\mu_{M_1 \otimes M_2,M_3}$} (d3)
(d3) edge node[left] {$\mathbb{E}(a)$} (d4)
(d) edge node[left] {$a^\prime$} (d5)
(d5) edge node[left] {$1 \otimes \mu_{M_2,M_3}$} (d6)
(d6) edge node[left] {$\mu_{M_1,M_2 \otimes M_3}$} (d7)
			(a) edge node[above]{$(R(i_1)\eta_{c_1}+R(i_2)\eta_{c_2})+R(i_3)\eta_{c_3}$} (b)
			(c) edge node[above]{$(R(o_1)\eta_{c_1^\prime}+R(o_2)\eta_{c_2^\prime})+R(o_3)\eta_{c_3^\prime}$} (b)
                                (a2) edge node[above]{$R(i_1+i_2)\eta_{c_1+c_2}+R(i_3)\eta_{c_3}$} (b2)
			(c2) edge node[above]{$R(o_1+o_2)\eta_{c_1^\prime+c_2^\prime}+R(o_3)\eta_{c_3^\prime}$} (b2)
                                (a) edge node[left]{$1$} (a2)
                                (b) edge node[left]{$\kappa + 1$} (b2)
(b) edge node[right]{$\iota_1$} (b2)
			(c) edge node[left]{$1$} (c2)
                                (a3) edge node[above]{$R((i_1+i_2)+i_3)\eta_{(c_1+c_2)+c_3}$} (b3)
			(c3) edge node[above]{$R((o_1+o_2)+o_3)\eta_{(c_1^\prime+c_2^\prime)+c_3^\prime}$} (b3)
                                (a2) edge node[left]{$1$} (a3)
                                (b2) edge node[left]{$\kappa$} (b3)
(b2) edge node[right]{$\iota_2$} (b3)
			(c2) edge node[left]{$1$} (c3)
                                (a4) edge node[above]{$R(i_1+(i_2+i_3))\eta_{c_1+(c_2+c_3)}$} (b4)
			(c4) edge node[above]{$R(o_1+(o_2+o_3))\eta_{c_1^\prime+(c_2^\prime+c_3^\prime)}$} (b4)
                                (a3) edge node[left]{$a_\mathrm{A}$} (a4)
                                (b3) edge node[left]{$R(a_\mathrm{X})$} (b4)
(b3) edge node[right]{$\iota_3$} (b4)
			(c3) edge node[left]{$a_\mathrm{A}$} (c4)
                                (a5) edge node[above]{$R(i_1)\eta_{c_1}+(R(i_2)\eta_{c_2}+R(i_3)\eta_{c_3})$} (b5)
			(c5) edge node[above]{$R(o_1)\eta_{c_1^\prime}+(R(o_2)\eta_{c_2^\prime}+R(o_3)\eta_{c_3^\prime})$} (b5)
                                (a) edge node[left]{$a_\mathrm{A}$} (a5)
                                (b) edge node[left]{$a_\mathrm{A}$} (b5)
(b) edge node[right]{$\iota_4$} (b5)
			(c) edge node[left]{$a_\mathrm{A}$} (c5)
                                (a6) edge node[above]{$R(i_1)\eta_{c_1}+R(i_2+i_3)\eta_{c_2+c_3}$} (b6)
			(c6) edge node[above]{$R(o_1)\eta_{c_1^\prime}+R(o_2+o_3)\eta_{c_2^\prime+c_3^\prime}$} (b6)
                                (a5) edge node[left]{$1$} (a6)
                                (b5) edge node[left]{$1+\kappa$} (b6)
 (b5) edge node[right]{$\iota_5$} (b6)
			(c5) edge node[left]{$1$} (c6)
                                (a7) edge node[above]{$R(i_1+(i_2+i_3))\eta_{c_1+(c_2+c_3)}$} (b7)
			(c7) edge node[above]{$R(o_1+(o_2+o_3))\eta_{c_1^\prime+(c_2^\prime+c_3^\prime)}$} (b7)
                                (a6) edge node[left]{$1$} (a7)
                                (b6) edge node[left]{$\kappa$} (b7)
(b6) edge node[right]{$\iota_6$} (b7)
			(c6) edge node[left]{$1$} (c7);
		\end{tikzpicture}
	\]
with the corresponding maps of decorations amounting to the following commutative diagram in $F(R(x_1+(x_2+x_3)))$:
\[
\begin{tikzpicture}[scale=1.5]
\node (A) at (0,0) {$F((\kappa)(1+\kappa)(a_\mathrm{A}))((x_1+x_2)+x_3)$};
\node (B) at (5,0) {$F((R(a_\mathrm{X}))(\kappa))((x_1+x_2)+x_3)$};
\node (C) at (5,-1) {$F(R(a_\mathrm{X}))((x_1+x_2)+x_3)$};
\node (D) at (5,-2) {$x_1+(x_2+x_3)$};
\node (E) at (0,-1) {$F((\kappa)(1+\kappa))((x_1+x_2)+x_3)$};
\node (F) at (0,-2) {$F(\kappa)(x_1+(x_2+x_3))$};
\path[->,font=\scriptsize,>=angle 90]
(A) edge node[above]{$F((R(a_\mathrm{X}))(\kappa))(\iota_1)$} (B)
(B) edge node[right]{$F(R(a_\mathrm{X}))(\iota_2)$} (C)
(C) edge node[right]{$\iota_3$} (D)
(A) edge node[left]{$F((\kappa)(1+\kappa))(\iota_4)$} (E)
(E) edge node[left]{$F(\kappa)(\iota_5)$} (F)
(F) edge node[above]{$\iota_6$} (D);
\end{tikzpicture}
\]
The above square commutes because $$F((\kappa)(1+\kappa)(a_\mathrm{A}))((x_1+x_2)+x_3) = F((R(a_\mathrm{X}))(\kappa)(\kappa+1))((x_1+x_2)+x_3)$$ as the above diagram of maps of cospans commutes. Lastly, the braiding square:
\[
\begin{tikzpicture}[scale=1.5]
\node (A) at (0,0) {$\mathbb{E}(M_1) \otimes \mathbb{E}(M_2)$};
\node (B) at (3,0) {$\mathbb{E}(M_2) \otimes \mathbb{E}(M_1)$};
\node (C) at (0,-1) {$\mathbb{E}(M_1 \otimes M_2)$};
\node (D) at (3,-1) {$\mathbb{E}(M_2 \otimes M_1)$};
\path[->,font=\scriptsize,>=angle 90]
(A) edge node[above]{$\beta$} (B)
(B) edge node[right]{$\mu_{M_2,M_1}$} (D)
(A) edge node[left]{$\mu_{M_1,M_2}$} (C)
(C) edge node[above]{$\mathbb{E}(\beta)$} (D);
\end{tikzpicture}
\]
has underlying map of cospans given by:
\[
		\begin{tikzpicture}
\node (d) at (7,0) {$\mathbb{E}(M_1) \otimes \mathbb{E}(M_2)$};
\node (d2) at (7,1) {$\mathbb{E}(M_2) \otimes \mathbb{E}(M_1)$};
\node (d3) at (7,2) {$\mathbb{E}(M_2 \otimes M_1)$};
\node (d4) at (7,-1) {$\mathbb{E}(M_1 \otimes M_2)$};
\node (d5) at (7,-2) {$\mathbb{E}(M_2 \otimes M_1)$};
			\node (a) at (-4,0) {$c_1+c_2$};
			\node (b) at (0.5,0) {$R(x_1)+R(x_2)$};
			\node (c) at (5,0) {$c_1^\prime+c_2^\prime$};
			\node (a2) at (-4,1) {$c_2+c_1$};
			\node (b2) at (0.5,1) {$R(x_2)+R(x_1)$};
			\node (c2) at (5,1) {$c_2^\prime+c_1^\prime$};
                                \node (a3) at (-4,2) {$c_2+c_1$};
			\node (b3) at (0.5,2) {$R(x_2+x_1)$};
			\node (c3) at (5,2) {$c_2^\prime + c_1^\prime$};
                                \node (a5) at (-4,-1) {$c_1+c_2$};
			\node (b5) at (0.5,-1) {$R(x_1+x_2)$};
			\node (c5) at (5,-1) {$c_1^\prime+c_2^\prime$};
                                \node (a6) at (-4,-2) {$c_2+c_1$};
			\node (b6) at (0.5,-2) {$R(x_2+x_1)$};
			\node (c6) at (5,-2) {$c_2^\prime + c_1^\prime$};
			\path[->,font=\scriptsize,>=angle 90]
(d) edge node[left]{$\beta$} (d2)
(d2) edge node[left]{$\mu_{M_2,M_1}$}(d3)
(d) edge node[left] {$\mu_{M_1,M_2}$}(d4)
(d4)edge node[left]{$\mathbb{E}(\beta)$}(d5)
			(a) edge node[above]{$R(i_1) \eta_{c_1} + R(i_2)\eta_{c_2}$} (b)
			(c) edge node[above]{$R(o_1) \eta_{c_1^\prime} + R(o_2) \eta_{c_2^\prime}$} (b)
                                (a2) edge node[above]{$R(i_2) \eta_{c_2} + R(i_1) \eta_{c_1}$} (b2)
			(c2) edge node[above]{$R(o_2) \eta_{c_2^\prime} + R(o_1) \eta_{c_1^\prime}$} (b2)
                                (a) edge node[left]{$\beta_\mathrm{A}$} (a2)
                                (b) edge node[left]{$\beta_\mathrm{A}$} (b2)
(b) edge node[right]{$\iota_1$} (b2)
			(c) edge node[left]{$\beta_\mathrm{A}$} (c2)
                                (a3) edge node[above]{$R(i_2+i_1)\eta_{c_2+c_1}$} (b3)
			(c3) edge node[above]{$R(o_2+o_1)\eta_{c_2^\prime+c_1^\prime}$} (b3)
                                (a2) edge node[left]{$1$} (a3)
                                (b2) edge node[left]{$\kappa$} (b3)
(b2) edge node[right]{$\iota_2$} (b3)
			(c2) edge node[left]{$1$} (c3)
                                (a5) edge node[above]{$R(i_1+i_2)\eta_{c_1+c_2}$} (b5)
			(c5) edge node[above]{$R(o_1+o_2)\eta_{c_1^\prime+c_2^\prime}$} (b5)
                                (a) edge node[left]{$1$} (a5)
                                (b) edge node[left]{$\kappa$} (b5)
(b) edge node[right]{$\iota_3$} (b5)
			(c) edge node[left]{$1$} (c5)
                                (a6) edge node[above]{$R(i_2+i_1)\eta_{c_2+c_1}$} (b6)
			(c6) edge node[above]{$R(o_2+o_1)\eta_{c_2^\prime+c_1^\prime}$} (b6)
                                (a5) edge node[left]{$\beta_\mathrm{A}$} (a6)
                                (b5) edge node[left]{$R(\beta_\mathrm{X})$} (b6)
 (b5) edge node[right]{$\iota_4$} (b6)
			(c5) edge node[left]{$\beta_\mathrm{A}$} (c6);
		\end{tikzpicture}
	\]
with the corresponding maps of decorations amounting to the following commutative diagram in $F(R(x_2+x_1))$:
\[
\begin{tikzpicture}[scale=1.5]
\node (A) at (-.5,0) {$F((\kappa)(\beta_\mathrm{A}))(x_1+x_2)$};
\node (C) at (4,0) {$F(\kappa)(x_2+x_1)$};
\node (D) at (-.5,-1) {$F(R(\beta_\mathrm{X}))(x_1+x_2)$};
\node (E) at (4,-1) {$x_2+x_1$};
\path[->,font=\scriptsize,>=angle 90]
(A) edge node[above]{$F(\kappa)(\iota_1)$} (C)
(A) edge node[left]{$F(R(\beta_\mathrm{X}))(\iota_3)$} (D)
(D) edge node[above]{$\iota_4$} (E)
(C) edge node[right]{$\iota_2$} (E);
\end{tikzpicture}
\]
The above square commutes because $$F((\kappa)(\beta_\mathrm{A}))(x_1+x_2) = F((R(\beta_\mathrm{X}))(\kappa))(x_1+x_2)$$ as the above diagram of maps of cospans commutes. Thus the double functor $\mathbb{E} \colon _L\mathbb{C}\textnormal{sp}(\mathrm{X}) \to \mathbb{F}\textnormal{Cospan}(\mathrm{A})$ is symmetric monoidal.
\end{proof}
Using a result of Shulman \cite{Shul}, each of the isofibrant symmetric monoidal double categories $\mathbb{F}\textnormal{Cospan}(\mathrm{A})$ and $_L \mathbb{C}\textnormal{sp}(\mathrm{X})$ give rise to underlying symmetric monoidal bicategories, namely $\mathbb{F}\textnormal{Cospan}(\mathrm{A})$ induces a symmetric monoidal bicategory $H(\mathbb{F}\textnormal{Cospan}(\mathrm{A}))$ which has:
\begin{enumerate}
\item{objects as those of $\mathrm{A}$,}
\item{morphisms as horizontal 1-cells of $\mathbb{F}\textnormal{Cospan}(\mathrm{A})$, and}
\item{2-morphisms as globular 2-morphisms of $\mathbb{F}\textnormal{Cospan}(\mathrm{A})$.}
\end{enumerate} 
Likewise, $_L \mathbb{C}\textnormal{sp}(\mathrm{X})$ induces a symmetric monoidal bicategory $H( {_L \mathbb{C}\textnormal{sp}(\mathrm{X})})$ which has:
\begin{enumerate}
\item{objects as those of $\mathrm{A}$,}
\item{morphisms as horizontal 1-cells of $_L \mathbb{C}\textnormal{sp}(\mathrm{X})$, and}
\item{2-morphisms as globular 2-morphisms of $_L \mathbb{C}\textnormal{sp}(\mathrm{X})$.}
\end{enumerate}
Another result of Shulman \cite{Shul2} is the following:
\begin{prop}[Shulman, Prop. B.3]
An equivalence of fibrant double categories induces a biequivalence of horizontal bicategories.
\end{prop}
\begin{cor}
The bicategories $H(\mathbb{F}\textnormal{Cospan}(\mathrm{A}))$ and $H( {_L \mathbb{C}\textnormal{sp}(\mathrm{X})})$ are biequivalent.
\end{cor}
\textbf{Is this biequivalnce symmetric monoidal? Probably}

We can also define the part of the double equivalence $\mathbb{G} \colon \mathbb{F}\textnormal{Cospan}(\mathrm{A}) \to _L \mathbb{C} \textnormal{sp}(\mathrm{X})$ which goes in the other direction: again, the object component of this double functor will be $\mathbb{G}_0 = \id_{\mathrm{A}}$.

Given a horizontal 1-cell $M$ of $\mathbb{F} \textnormal{Cospan}(\mathrm{A})$:
\[
\begin{tikzpicture}[scale=1.5]
\node (A) at (0,0) {$c_1$};
\node (B) at (1,0) {$c$};
\node (C) at (2,0) {$c_2$};
\node (D) at (1,-0.5) {$x \in F(c)$};
\path[->,font=\scriptsize,>=angle 90]
(A) edge node[above]{$I$} (B)
(C) edge node[above]{$O$} (B);
\end{tikzpicture}
\]
the image $\mathbb{G}(M)$ is the horizontal 1-cell in $_L \mathbb{C}\textnormal{sp}(\mathrm{X})$ given by:
\[
\begin{tikzpicture}[scale=1.5]
\node (A) at (0,0) {$L(c_1)$};
\node (B) at (1,0) {$x$};
\node (C) at (2,0) {$L(c_2)$};
\path[->,font=\scriptsize,>=angle 90]
(A) edge node[above]{$\tau_c L(I)$} (B)
(C) edge node[above]{$\tau_ c L(O)$} (B);
\end{tikzpicture}
\]
where $\tau_c \colon L(c) \to x$ is the unique morphism from the trival decoration on $c$ given by $$!_c \coloneqq 1 \xrightarrow{\phi} F(0) \xrightarrow{F(!)} F(c)$$ to $x \in F(c)$. In other words, the trivial decoraction $!_c$ is initial in $F(c)$ and $\tau_c \colon !_c \to x$ is the unique morphism to $x$. Similarly, given a 2-morphism $(f,h,g,\iota) \colon M \to M^\prime$ of $\mathbb{F} \textnormal{Cospan}(\mathrm{A})$:
\[
\begin{tikzpicture}[scale=1.5]
\node (D) at (1,0.5) {$x \in F(c)$};
\node (D') at (1,-1.5) {$x^\prime \in F(c^\prime)$};
\node (A) at (0,0) {$c_1$};
\node (B) at (1,0) {$c$};
\node (C) at (2,0) {$c_2$};
\node (A') at (0,-1) {$c_1^\prime$};
\node (B') at (1,-1) {$c^\prime$};
\node (C') at (2,-1) {$c_2^\prime$};
\path[->,font=\scriptsize,>=angle 90]
(A) edge node[above]{$I$} (B)
(C) edge node[above]{$O$} (B)
(A') edge node[above]{$I^\prime$} (B')
(C') edge node[above]{$O^\prime$} (B')
(A) edge node [left]{$f$} (A')
(B) edge node [left]{$h$} (B')
(C) edge node [left]{$g$} (C');
\end{tikzpicture}
\]
$$\iota \colon F(h)(x) \to x^\prime$$
the image in $_L \mathbb{C}\textnormal{sp}(\mathrm{X})$ is given by the 2-morphism:
\[
\begin{tikzpicture}[scale=1.5]
\node (A) at (0,0) {$L(c_1)$};
\node (B) at (1,0) {$x$};
\node (C) at (2,0) {$L(c_2)$};
\node (A') at (0,-1) {$L(c_1^\prime)$};
\node (B') at (1,-1) {$x^\prime$};
\node (C') at (2,-1) {$L(c_2^\prime)$};
\path[->,font=\scriptsize,>=angle 90]
(A) edge node[above]{$\tau_c L(I)$} (B)
(C) edge node[above]{$\tau_c L(O)$} (B)
(A') edge node[above]{$\tau_{c^\prime} L(I^\prime)$} (B')
(C') edge node[above]{$\tau_{c^\prime} L(O^\prime)$} (B')
(A) edge node [left]{$L(f)$} (A')
(B) edge node [left]{$\alpha$} (B')
(C) edge node [left]{$L(g)$} (C');
\end{tikzpicture}
\]
where $\alpha \colon x \to x^\prime$ is a morphism in the Grothendieck construction of $F$ given by $\alpha = (h \colon c \to c^\prime, \iota \colon F(h)(x) \to x^\prime)$. 

Next, we exhibit natural isomorphisms $\eta \colon \id_{ _L \mathbb{C}\textnormal{sp}(\mathrm{X})} \cong \mathbb{G} \mathbb{E}$ and $\epsilon \colon \mathbb{E} \mathbb{G} \cong \id_{\mathbb{F}\textnormal{Cospan}(\mathrm{A})}$. Specifically, these are double natural isomorphisms given by double transformations $\eta$ and $\epsilon$ whose components are isomorphisms.

First we compute the composites $\mathbb{G} \mathbb{E}$ and $\mathbb{E} \mathbb{G}$. On the object categories, both composites are $\id_\mathrm{A}$ and we have natural isomorphisms $\eta \colon \id_{\mathrm{A}} \cong \mathbb{G}_0 \mathbb{E}_0$ and $\epsilon \colon \mathbb{E}_0 \mathbb{G}_0 \cong \id_{\mathrm{A}}$.

Given a horizontal 1-cell $M$ in $_L \mathbb{C} \textnormal{sp}(\mathrm{X})$:
\[
\begin{tikzpicture}[scale=1.5]
\node (A) at (0,0) {$L(c_1)$};
\node (B) at (1,0) {$x$};
\node (C) at (2,0) {$L(c_2)$};
\path[->,font=\scriptsize,>=angle 90]
(A) edge node[above]{$i$} (B)
(C) edge node[above]{$o$} (B);
\end{tikzpicture}
\]
the horizontal 1-cell $\mathbb{E}(M)$ is given by:
\[
\begin{tikzpicture}[scale=1.5]
\node (A) at (0,0) {$c_1$};
\node (B) at (1,0) {$R(x)$};
\node (C) at (2,0) {$c_2$};
\node (D) at (1,-0.5) {$x \in F(R(x))$};
\path[->,font=\scriptsize,>=angle 90]
(A) edge node[above]{$R(i)$} (B)
(C) edge node[above]{$R(o)$} (B);
\end{tikzpicture}
\]
and then the horizontal 1-cell $\mathbb{G} \mathbb{E}(M)$ is given by:
\[
\begin{tikzpicture}[scale=1.5]
\node (A) at (0,0) {$L(c_1)$};
\node (B) at (1.5,0) {$x$};
\node (C) at (3,0) {$L(c_2)$};
\path[->,font=\scriptsize,>=angle 90]
(A) edge node[above]{$\tau_{R(x)}L(R(i))$} (B)
(C) edge node[above]{$\tau_{R(x)}L(R(o))$} (B);
\end{tikzpicture}
\]
\textbf{What can we say about LR? Find a 2-iso between these in $_L \mathbb{C}\textnormal{sp}(\mathrm{X})$.}
Then we can find a 2-isomorphism $\eta_M \colon M \xrightarrow{\sim} \mathbb{G}\mathbb{E}(M) $ in $_L \mathbb{C} \textnormal{sp}(\mathrm{X})$ given by:
\[
\begin{tikzpicture}[scale=1.5]
\node (A) at (0,0) {$L(c_1)$};
\node (B) at (1.25,0) {$x$};
\node (C) at (2.5,0) {$L(c_2)$};
\node (A') at (0,-1) {$L(c_1)$};
\node (B') at (1.25,-1) {$x$};
\node (C') at (2.5,-1) {$L(c_2)$};
\path[->,font=\scriptsize,>=angle 90]
(A) edge node[above]{$i$} (B)
(C) edge node[above]{$o$} (B)
(A') edge node[above]{$\tau_{R(x)}L(R(i))$} (B')
(C') edge node[above]{$\tau_{R(x)}L(R(o))$} (B')
(A) edge node [left]{$1$} (A')
(B) edge node [left]{$1$} (B')
(C) edge node [left]{$1$} (C');
\end{tikzpicture}
\]
On the other hand, given a horizontal 1-cell $N$ in $\mathbb{F}\textnormal{Cospan}(\mathrm{A})$:
\[
\begin{tikzpicture}[scale=1.5]
\node (A) at (0,0) {$c_1$};
\node (B) at (1,0) {$c$};
\node (C) at (2,0) {$c_2$};
\node (D) at (1,-0.5) {$x \in F(c)$};
\path[->,font=\scriptsize,>=angle 90]
(A) edge node[above]{$I$} (B)
(C) edge node[above]{$O$} (B);
\end{tikzpicture}
\]
the horizontal 1-cell $\mathbb{G}(N)$ is given by:
\[
\begin{tikzpicture}[scale=1.5]
\node (A) at (0,0) {$L(c_1)$};
\node (B) at (1,0) {$x$};
\node (C) at (2,0) {$L(c_2)$};
\path[->,font=\scriptsize,>=angle 90]
(A) edge node[above]{$\tau_c L(I)$} (B)
(C) edge node[above]{$\tau_c L(O)$} (B);
\end{tikzpicture}
\]
and then the horizontal 1-cell $\mathbb{E} \mathbb{G}(N)$ is given by:
\[
\begin{tikzpicture}[scale=1.5]
\node (A) at (0,0) {$c_1$};
\node (B) at (1,0) {$R(x)$};
\node (C) at (2,0) {$c_2$};
\node (D) at (1,-0.5) {$x \in F(R(x))$};
\path[->,font=\scriptsize,>=angle 90]
(A) edge node[above]{$R(\tau_c)I$} (B)
(C) edge node[above]{$R(\tau_c)O$} (B);
\end{tikzpicture}
\]
Then we can find a 2-isomorphism $\epsilon_N \colon \mathbb{E} \mathbb{G} (N) \xrightarrow{\sim} N$ in $\mathbb{F}\textnormal{Cospan}(\mathrm{A})$ given by:
\[
\begin{tikzpicture}[scale=1.5]
\node (D) at (1,0.5) {$x \in F(R(x))$};
\node (D') at (1,-1.5) {$x \in F(c)$};
\node (A) at (0,0) {$c_1$};
\node (B) at (1,0) {$R(x)$};
\node (C) at (2,0) {$c_2$};
\node (A') at (0,-1) {$c_1$};
\node (B') at (1,-1) {$c$};
\node (C') at (2,-1) {$c_2$};
\path[->,font=\scriptsize,>=angle 90]
(A) edge node[above]{$R(\tau_c)I$} (B)
(C) edge node[above]{$R(\tau_c)O$} (B)
(A') edge node[above]{$I$} (B')
(C') edge node[above]{$O$} (B')
(A) edge node [left]{$1$} (A')
(B) edge node [left]{$e$} (B')
(C) edge node [left]{$1$} (C');
\end{tikzpicture}
\]
$$\iota \colon F(e)(x) \to x$$
\section{Applications}
In this section we present several examples each of which may be realized in the context of decorated cospans or in the context of structured cospans. The first example regarding graphs was mentioned in the introduction and used as a reoccurring theme throughout the paper. The next three examples which take on more of an applied flavor, consists of electrical circuits, Markov processes and Petri nets. Each of these has been studied extensively by the first author and others by way of `black-boxing'. Black-boxing is a way of interpreting the behavior of an open system, that is, a system with prescribed inputs and outputs such as the terminals of an electrical circuit, by observing the activity at the inputs and the outputs. The semantics of the activity at an open system's inputs and outputs is typically described in a category such as $\mathrm{LinRel}$ of finite dimensonal vector spaces and linear relations. Thus, in each case, black-boxing results in functors such as: $$\blacksquare_1 \colon \mathrm{Circ} \to \mathrm{LinRel}$$ $$\blacksquare_2 \colon \mathrm{Mark} \to \mathrm{LinRel}$$ $$\blacksquare_3 \colon \mathrm{Petri} \to \mathrm{LinRel}.$$ Each of these black-boxing functors also possess other convenient properties such as being symmetric monoidal. The first two of these were first done using Fong's theory of decorated cospans and then extended by the first two authors using structured cospans. The last two of these were also extended by being realized as double functors between double categories. \textbf{Make this better...}
\subsection{Graphs}
As a first example that was also mentioned in the introduction, let $L \colon \mathrm{Set} \to \mathrm{Graph}$ be the functor that assigns to a set $N$ the \emph{discrete graph} on $N$ which is the edgeless graph $L(N)$ with no edges and $N$ as its set of vertices. Both $\mathrm{Set}$ and $\mathrm{Graph}$ are cocartesian monoidal and the functor $L \colon \mathrm{Set} \to \mathrm{Graph}$ is left adjoint to the forgetful functor $R \colon \mathrm{Graph} \to \mathrm{Set}$ which assigns to a graph $G$ its underlying set of vertices $U(G)$. Using structured cospans and appealing to Theorem \ref{SC}, we get a symmetric monoidal double category $_L \mathbb{C}\textnormal{sp}(\mathrm{Graph})$ which has:
\begin{enumerate}
\item{sets as objects,}
\item{functions as vertical 1-morphisms,}
\item{cospans of graphs, or, \emph{open} graphs of the form
\[
\begin{tikzpicture}[scale=1.5]
\node (A) at (0,0) {$L(N)$};
\node (B) at (1,0) {$G$};
\node (C) at (2,0) {$L(M)$};
\path[->,font=\scriptsize,>=angle 90]
(A) edge node[above]{$I$} (B)
(C) edge node[above]{$O$} (B);
\end{tikzpicture}
\]
as horizontal 1-cells, where $L(N)$ and $L(M)$ are discrete graphs on the sets $N$ and $M$, respectively, $G$ is a graph and $I$ and $O$ are graph morphisms, and}
\item{maps of cospans of graphs of the form
\[
\begin{tikzpicture}[scale=1.5]
\node (A) at (0,0) {$L(N_1)$};
\node (B) at (1,0) {$G_1$};
\node (C) at (2,0) {$L(M_1)$};
\node (A') at (0,-1) {$L(N_2)$};
\node (B') at (1,-1) {$G_2$};
\node (C') at (2,-1) {$L(M_2)$};
\path[->,font=\scriptsize,>=angle 90]
(A) edge node[above]{$I_1$} (B)
(C) edge node[above]{$O_1$} (B)
(A') edge node[above]{$I_2$} (B')
(C') edge node[above]{$O_2$} (B')
(A) edge node [left]{$L(f)$} (A')
(B) edge node [left]{$\alpha$} (B')
(C) edge node [left]{$L(g)$} (C');
\end{tikzpicture}
\]
as 2-morphisms, where $L(f)$ and $L(g)$ are maps of discrete graphs induced by the underlying functions $f$ and $g$, respectively, and $\alpha \colon G_1 \to G_2$ is a graph morphism.
}
\end{enumerate}

We can obtain a similar symmetric monoidal double category using decorated cospans. Let $F \colon \mathrm{Set} \to \mathrm{Cat}$ be the symmetric lax monoidal pseudofunctor that assigns to a set $N$ the \emph{category} of all graph structures whose underlying set of vertices is $N$. Using Theorem \ref{DC}, we then obtain a symmetric monoidal double category $\mathbb{F}\textnormal{Cospan}(\mathrm{Set})$ which has:
\begin{enumerate}
\item{sets as objects,}
\item{functions as vertical 1-morphisms,}
\item{horizontal 1-cells as pairs:
\[
\begin{tikzpicture}[scale=1.5]
\node (A) at (0,0) {$N$};
\node (B) at (1,0) {$P$};
\node (C) at (2,0) {$M$};
\node (D) at (3.25,0) {$G \in F(P)$};
\path[->,font=\scriptsize,>=angle 90]
(A) edge node[above]{$i$} (B)
(C) edge node[above]{$o$} (B);
\end{tikzpicture}
\]
which can also be thought of as open graphs, and}
\item{2-morphisms as maps of cospans of sets
\[
\begin{tikzpicture}[scale=1.5]
\node (A) at (0,0) {$N_1$};
\node (A') at (0,-1) {$N_2$};
\node (C') at (2,-1) {$M_2$};
\node (B) at (1,0) {$P_1$};
\node (C) at (2,0) {$M_1$};
\node (D) at (1,-1) {$P_2$};
\node (E) at (3,0) {$G_1 \in F(P_1)$};
\node (F) at (3,-1) {$G_2 \in F(P_2)$};
\path[->,font=\scriptsize,>=angle 90]
(A) edge node[above]{$i_1$} (B)
(C) edge node[above]{$o_1$} (B)
(A) edge node[left]{$f$} (A')
(C) edge node[right]{$g$} (C')
(C') edge node [above] {$o_2$} (D)
(A') edge node [above] {$i_2$} (D)
(B) edge node [left] {$h$} (D);
\end{tikzpicture}
\]
together with a graph morphism $\iota \colon F(h)(G_1) \to G_2$ in $F(P_2)$.}
\end{enumerate}
We thus have two symmetric monoidal double categories: $_L \mathbb{C}\textnormal{sp}(\mathrm{Graph})$ obtained from structured cospans and $\mathbb{F}\textnormal{Cospan}(\mathrm{Set})$ obtained from decorated cospans. Both of these double categories have $\mathrm{Set}$ as their categories of objects, open graphs as horizontal 1-cells and maps of open graphs as 2-morphisms, and by Theorem \ref{Equiv}, we have an equivalence of symmetric monoidal double categories $$_L \mathbb{C}\textnormal{sp}(\mathrm{Graph}) \sim \mathbb{F}\textnormal{Cospan}(\mathrm{Set}).$$

\subsection{Electrial circuits}

\subsection{Markov processes}

\subsection{Petri nets}

\section{Appendix}
Before formally defining `pseudo double category', it is helpful to have the following picture in mind. A pseudo double category has 2-morphisms shaped like:

\[
\begin{tikzpicture}[scale=1]
\node (D) at (-4,0.5) {$A$};
\node (E) at (-2,0.5) {$B$};
\node (F) at (-4,-1) {$C$};
\node (A) at (-2,-1) {$D$};
\node (B) at (-3,-0.25) {$\Downarrow a$};
\path[->,font=\scriptsize,>=angle 90]
(D) edge node [above]{$M$}(E)
(E) edge node [right]{$g$}(A)
(D) edge node [left]{$f$}(F)
(F) edge node [above]{$N$} (A);
\end{tikzpicture}
\]

We call $A, B, C$ and $D$ \textbf{objects} or \textbf{0-cells}, $f$ and $g$ \textbf{vertical 1-morphisms}, $M$ and $N$ \textbf{horizontal 1-cells} and $a$ a \textbf{2-morphism}. Note that a vertical 1-morphism is a morphism between 0-cells and a 2-morphism is a morphism between horizontal 1-cells. We will denote both kinds of morphisms and horizontal 1-cells as a single arrow, namely `$\to$'. We follow the notation of Shulman \cite{Shul} with the following definitions.

\begin{defn}
A \textbf{pseudo double category} $\lD$, or $\textbf{double category}$ for short, consists of a category of objects $\bold{D_{0}}$ and a category of arrows $\bold{D_{1}}$ with the following functors
\begin{center}
$U\colon \bold{D_{0}} \to \bold{D_{1}}$\\
$S,T \colon \bold{D_{1}} \rightrightarrows \bold{D_{0}}$\\
$\odot \colon \bold{D_{1}} \times_{\bold{D_{0}}} \bold{D_{1}} \to \bold{D_{1}}$ (where the pullback is taken over $\bold{D_{1}} \xrightarrow[]{T} \bold{D_{0}} \xleftarrow[]{S} \bold{D_{1}}$) \\
\end{center}
 such that \\
\begin{center}
$S(U_{A})=A=T(U_{A})$\\
$S(M \odot N)=SN$\\
$T(M \odot N)=TM$\\
\end{center}
equipped with natural isomorphisms
\begin{center}

$\alpha \colon (M \odot N) \odot P \xrightarrow{\sim} M \odot (N \odot P)$\\
$\lambda \colon U_{B} \odot M \xrightarrow{\sim} M$\\
$\rho \colon M \odot U_{A} \xrightarrow{\sim} M$

\end{center}
such that $S(\alpha), S(\lambda), S(\rho), T(\alpha), T(\lambda)$ and $T(\rho)$ are all identities and that the coherence axioms of a monoidal category are satisfied. Following the notation of Shulman, objects of $\bold{D_{0}}$ are called $\textbf{0-cells}$ and morphisms of $\bold{D_{0}}$ are called $\textbf{vertical 1-morphisms}$. Objects of $\bold{D_{1}}$ are called $\textbf{horizontal 1-cells}$ and morphisms of $\bold{D_{1}}$ are called $\textbf{2-morphisms}$. The morphisms of $\bold{D_{0}}$, which are vertical 1-morphisms, will be denoted $f \colon A \to C$ and we denote a 1-cell $M$ with $S(M)=A,T(M)=B$ by $M \colon A \to B$. Then a 2-morphism $a \colon M \to N$ of $\bold{D_{1}}$ with $S(a)=f,T(a)=g$ would look like:
\[
\begin{tikzpicture}[scale=1]
\node (D) at (-4,0.5) {$A$};
\node (E) at (-2,0.5) {$B$};
\node (F) at (-4,-1) {$C$};
\node (A) at (-2,-1) {$D$};
\node (B) at (-3,-0.25) {$\Downarrow a$};
\path[->,font=\scriptsize,>=angle 90]
(D) edge node [above]{$M$}(E)
(E) edge node [right]{$g$}(A)
(D) edge node [left]{$f$}(F)
(F) edge node [above]{$N$} (A);
\end{tikzpicture}
\]
\end{defn}

The key difference between a `strict' double category and a pseudo double category is that in a pseudo double category, horizontal composition is associative and unital only up to natural isomorphism. Equivalently, as a double category can be viewed as a category internal to $\bold{Cat}$, we can view a pseudo double category as a category `weakly' internal to $\bold{Cat}$. We will sometimes omit the word pseudo and simply say double category.

\begin{defn}
A 2-morphism where $f$ and $g$ are identities is called a \textbf{globular 2-morphism}.
\end{defn}

\begin{defn}
Let $\lD$ be a pseudo double category. Then the $\textbf{horizontal bicategory}$ of $\lD$, which we denote as $H(\lD)$, is the bicategory consisting of objects of $\lD$, morphisms that are horizontal 1-cells of $\lD$ and 2-morphisms that are globular 2-morphisms of $\lD$.
\end{defn}

\begin{defn}
  A \textbf{monoidal double category} is a double category equipped the following
structure.
\begin{enumerate}
\item $\bold{D_{0}}$ and $\bold{D_{1}}$ are both monoidal categories.
\item If $I$ is the monoidal unit of $\bold{D_{0}}$, then $U_I$ is the
  monoidal unit of $\bold{D_{1}}$.
\item The functors $S$ and $T$ are strict monoidal, i.e.\ $S(M\ten N)
  = SM\ten SN$ and $T(M\ten N)=TM\ten TN$ and $S$ and $T$ also
  preserve the associativity and unit constraints.
\item We have globular isomorphisms
  \[\chi \maps (M_1\ten N_1)\odot (M_2\ten N_2)\too[\sim] (M_1\odot M_2)\ten (N_1\odot N_2)\]
  and
  \[\mu\maps U_{A\ten B} \too[\sim] (U_A \ten U_B)\]
  such that the following diagrams commute:
		\item \label{diag:MonDblCat}
			The following diagrams commute expressing the constraint data for the double functor $\otimes$.
			\[
			\begin{tikzpicture}
				\node (A) at (0,3) {\footnotesize{
							$((M_1\otimes N_1)\odot (M_2\otimes N_2)) \odot (M_3\otimes N_3)$}
				};
				\node (B) at (7,3) {\footnotesize{
						$((M_1\odot M_2)\otimes (N_1\odot N_2)) \odot (M_3\otimes N_3) $}
				};
				\node (A') at (0,1.5) {\footnotesize{
						$(M_1\otimes N_1)\odot ((M_2\otimes N_2) \odot (M_3\otimes N_3)) $}
				};
				\node (B') at (7,1.5) {\footnotesize{
						$((M_1\odot M_2)\odot M_3) \otimes ((N_1\odot N_2)\odot N_3)$}
				};
				\node (A'') at (0,0) {\footnotesize{
						$(M_1\otimes N_1) \odot ((M_2\odot M_3) \otimes (N_2\odot N_3))$}
				};
				\node (B'') at (7,0) {\footnotesize{
						$(M_1\odot (M_2\odot M_3)) \otimes (N_1\odot (N_2\odot N_3))$}
				};
			%
			\path[->,font=\scriptsize]
				(A) edge node[left]{$\alpha$} (A')
				(A') edge node[left]{$1 \odot \chi$} (A'')
				(B) edge node[right]{$\chi$} (B')
				(B') edge node[right]{$\alpha \otimes \alpha$} (B'')
				(A) edge node[above]{$\chi \odot 1$} (B)
				(A'') edge node[above]{$\chi$} (B'');
		\end{tikzpicture}
		\]
		\[
		\begin{tikzpicture}
			\node (UL) at (0,1.5) {\footnotesize{
					$(M\otimes N) \odot U_{C\otimes D}$}
			};
			\node (LL) at (0,0) {\footnotesize{
					$M\otimes N$}
			};
			\node (UR) at (3.5,1.5) {\footnotesize{
					$(M\otimes N)\odot (U_C\otimes U_D)$}
			};
			\node (LR) at (3.5,0) {\footnotesize{
					$(M\odot U_C) \otimes (N\odot U_D)$}
			};
			%
			\path[->,font=\scriptsize]
				(UL) edge node[above]{$1 \odot \mu$} (UR) 
				(UL) edge node[left]{$\rho$} (LL)
				(LR) edge node[above]{$\rho \otimes \rho$} (LL)
				(UR) edge node[right]{$\chi$} (LR);
		\end{tikzpicture}
		%
		\quad
		%
		\begin{tikzpicture}
			\node (UL) at (0,1.5) {\scriptsize{$U_{A\otimes B}\odot (M\otimes N)$}};
			\node (LL) at (0,0) {\scriptsize{$M\otimes N$}};
			\node (UR) at (3.5,1.5) {\scriptsize{$(U_A\otimes U_B)\odot (M\otimes N)$}};
			\node (LR) at (3.5,0) {\scriptsize{$(U_A \odot M) \otimes (U_B\odot N)$}};
			%
			\path[->,font=\scriptsize]
				(UL) edge node[above]{$\chi \odot 1$} (UR) 
				(UL) edge node[left]{$\lambda$} (LL)
				(LR) edge node[above]{$\lambda \otimes \lambda$} (LL)
				(UR) edge node[right]{$\chi$} (LR);
		\end{tikzpicture}
		\]
		%
		\item The following diagrams commute expressing 
		the associativity isomorphism for $\otimes$ is a transformation of double categories.
		\[
		\begin{tikzpicture}
			\node (A) at (0,3) {\footnotesize{
					$((M_1\otimes N_1)\otimes P_1) \odot ((M_2\otimes N_2)\otimes P_2)$}
			};
			\node (B) at (7,3) {\footnotesize{
					$(M_1\otimes (N_1\otimes P_1)) \odot (M_2\otimes (N_2\otimes P_2))$}
			};
			\node (A') at (0,1.5) {\footnotesize{
					$((M_1\otimes N_1) \odot (M_2\otimes N_2)) \otimes (P_1\odot P_2)$}
			};
			\node (B') at (7,1.5) {\footnotesize{
					$(M_1\odot M_2) \otimes ((N_1\otimes P_1)\odot (N_2\otimes P_2))$}
			};
			\node (A'') at (0,0) {\footnotesize{
					$((M_1\odot M_2) \otimes(N_1\odot N_2)) \otimes (P_1\odot P_2)$}
			};
			\node (B'') at (7,0) {\footnotesize{
					$(M_1\odot M_2) \otimes ((N_1\odot N_2)\otimes (P_1\odot P_2))$}
			};
			%
			\path[->,font=\scriptsize]
				(A) edge node[left]{$\chi$} (A')
				(A') edge node[left]{$\chi \otimes 1$} (A'')
				(B) edge node[right]{$\chi$} (B')
				(B') edge node[right]{$1 \otimes \chi$} (B'')
				(A) edge node[above]{$\alpha \odot \alpha$} (B)
				(A'') edge node[above]{$\alpha$} (B'');
		\end{tikzpicture}
		\]
		\[
		\begin{tikzpicture}
			\node (A) at (0,3) {\footnotesize{$U_{(A\otimes B)\otimes C}$}};
			\node (B) at (4,3) {\footnotesize{$U_{A\otimes (B\otimes C)} $}};
			\node (A') at (0,1.5) {\footnotesize{$U_{A\otimes B} \otimes U_C $}};
			\node (B') at (4,1.5) {\footnotesize{$U_A\otimes U_{B\otimes C}$}};
			\node (A'') at (0,0) {\footnotesize{$(U_A\otimes U_B)\otimes U_C$}};
			\node (B'') at (4,0) {\footnotesize{$U_A\otimes (U_B\otimes U_C) $}};
			%
			\path[->,font=\scriptsize]
				(A) edge node[left]{$\mu$} (A')
				(A') edge node[left]{$\mu \otimes 1$} (A'')
				(B) edge node[right]{$\mu$} (B')
				(B') edge node[right]{$1 \otimes \mu$} (B'')
				(A) edge node[above]{$U_{\alpha}$} (B)
				(A'') edge node[above]{$\alpha$} (B'');
		\end{tikzpicture}
		\]
		\item The following diagrams commute expressing that 
		the unit isomorphisms for $\otimes$ are transformations of double categories. 
		\[
		\begin{tikzpicture}
			\node (A) at (0,1.5) {\footnotesize{$(M\otimes U_I)\odot (N\otimes U_I)$}};
			\node (A') at (0,0) {\footnotesize{$M\odot N $}};
			\node (B) at (4,1.5) {\footnotesize{$(M\odot N)\otimes (U_I \odot U_I) $}};
			\node (B') at (4,0) {\footnotesize{$(M\odot N)\otimes U_I $}};
			%
			\path[->,font=\scriptsize]
				(A) edge node[left]{$r \odot r$} (A')
				(A) edge node[above]{$\chi$} (B)
				(B) edge node[right]{$1 \otimes \rho$} (B')
				(B') edge node[above]{$r$} (A');
		\end{tikzpicture}
		%
		\quad
		%
		\begin{tikzpicture}
			\node (A) at (0,0.75) {\footnotesize{$U_{A\otimes I} $}};
			\node (B) at (1.5,1.5) {\footnotesize{$U_A\otimes U_I $}};
			\node (B') at (1.5,0) {\footnotesize{$U_A$}};
			%
			\path[->,font=\scriptsize]
				(A) edge node[above]{$\mu$} (B)
				(A) edge node[below]{$U_{r}$} (B')
				(B) edge node[right]{$r$} (B');
		\end{tikzpicture}
		\]
		%
		%
		%
		%
		\[
		\begin{tikzpicture}
			\node (A) at (0,1.5) {\footnotesize{$(U_I\otimes M)\odot (U_I\otimes N)$}};
			\node (A') at (0,0) {\footnotesize{$M\odot N$}};
			\node (B) at (4,1.5) {\footnotesize{$(U_I \odot U_I) \otimes (M\odot N)$}};
			\node (B') at (4,0) {\footnotesize{$U_I\otimes (M\odot N) $}};
			%
			\path[->,font=\scriptsize]
				(A) edge node[left]{$\ell \odot \ell$} (A')
				(A) edge node[above]{$\chi$} (B)
				(B) edge node[right]{$\lambda \otimes 1$} (B')
				(B') edge node[above]{$\ell$} (A');
		\end{tikzpicture}
		%
		\quad
		\begin{tikzpicture}
			\node (A) at (0,0.75) {\footnotesize{$U_{I\otimes A}$}};
			\node (B) at (1.5,1.5) {\footnotesize{$U_I\otimes U_A$}};
			\node (B') at (1.5,0) {\footnotesize{$U_A$}};
			%
			\path[->,font=\scriptsize]
				(A) edge node[above]{$\mu$} (B)
				(A) edge node[below]{$U_{\ell}$} (B')
				(B) edge node[right]{$\ell$} (B');
		\end{tikzpicture}
		\]
		\newcounter{mondbl}
		\setcounter{mondbl}{\value{enumi}}
	\end{enumerate}
	A \textbf{braided monoidal double category} 
	is a monoidal double category 
	such that:
	\begin{enumerate}
		\setcounter{enumi}{\value{mondbl}}
		\item $\dblcat{D}_{0}$ and $\dblcat{D}_{1}$ are braided monoidal categories.
		\item The functors $S$ and $T$ are strict braided monoidal functors.
		\item The following diagrams commute expressing that the braiding is a transformation of double categories.
		\[
		\begin{tikzpicture}
			\node (A) at (0,1.5) {\footnotesize{$(M_1 \odot M_2) \otimes (N_1 \odot N_2)$}};
			\node (A') at (0,0) {\footnotesize{$(M_1\otimes N_1) \odot (M_2\otimes N_2)$}};
			\node (B) at (5,1.5) {\footnotesize{$(N_1\odot N_2) \otimes (M_1 \odot M_2)$}};
			\node (B') at (5,0) {\footnotesize{$(N_1 \otimes M_1) \odot (N_2 \otimes M_2)$}};
			%
			\path[->,font=\scriptsize]
				(A) edge node[left]{$\chi$} (A')
				(A) edge node[above]{$\beta$} (B)
				(B) edge node[right]{$\chi$} (B')
				(A') edge node[above]{$\beta \odot \beta$} (B');
		\end{tikzpicture}
		%
		\quad
		%
		\begin{tikzpicture}
			\node (A) at (0,1.5) {\footnotesize{$U_A \otimes U_B$}};
			\node (A') at (0,0) {\footnotesize{$U_B\otimes U_A$}};
			\node (B) at (2,1.5) {\footnotesize{$U_{A\otimes B} $}};
			\node (B') at (2,0) {\footnotesize{$U_{B\otimes A}$}};
			%
			\path[->,font=\scriptsize]
				(A) edge node[left]{$\beta$} (A')
				(B) edge node[above]{$\mu$} (A)
				(B) edge node[right]{$U_\beta$} (B')
				(B') edge node[above]{$\mu$} (A');
		\end{tikzpicture}
		\]
		\setcounter{mondbl}{\value{enumi}}
	\end{enumerate}
	Finally, a \textbf{symmetric monoidal double category} 
	is a braided monoidal double category $\mathbb{D}$ such that:
	\begin{enumerate}
		\setcounter{enumi}{\value{mondbl}}
		\item $\dblcat{D}_{0}$ and $\dblcat{D}_{1}$ are symmetric monoidal.
	\end{enumerate}
\end{defn}


\begin{defn}\label{def:companion}
  Let \lD\ be a double category and $f\maps A\to B$ a vertical
  1-morphism.  A \textbf{companion} of $f$ is a horizontal 1-cell
  $\fhat\maps A\to B$ together with 2-morphisms
	\[
	\raisebox{-0.5\height}{
	\begin{tikzpicture}
		\node (A) at (0,1) {$A$};
		\node (B) at (1,1) {$B$};
		\node (A') at (0,0) {$B$};
		\node (B') at (1,0) {$B$};
		%
		\path[->,font=\scriptsize,>=angle 90]
			(A) edge node[above]{$\widehat{f}$} (B)
			(A) edge node[left]{$f$} (A')
			(B) edge node[right]{$1$} (B')
			(A') edge node[below]{$U_B$} (B');
		%
	%	\draw (0.5,.925) -- (0.5,1.075);
	%	\draw (0.5,-.075) -- (0.5,.075);
		\node () at (0.5,0.5) {\scriptsize{$\Downarrow$}};
	\end{tikzpicture}
	}
	%
	\quad \text{ and } \quad
	%
	\raisebox{-0.5\height}{
	\begin{tikzpicture}
		\node (A) at (0,1) {$A$};
		\node (B) at (1,1) {$A$};
		\node (A') at (0,0) {$A$};
		\node (B') at (1,0) {$B$};
		%
		\path[->,font=\scriptsize,>=angle 90]
			(A) edge node[above]{$U_A$} (B)
			(A) edge node[left]{$1$} (A')
			(B) edge node[right]{$f$} (B')
			(A') edge node[below]{ $\widehat{f}$} (B');
		%
	%	\draw (0.5,.925) -- (0.5,1.075);
	%	\draw (0.5,-.075) -- (0.5,.075);
		\node () at (0.5,0.5) {\scriptsize{$\Downarrow$}};
	\end{tikzpicture}
	}
	\]
  such that the following equations hold.
	\begin{equation}
	\label{eq:CompanionEq}
	\raisebox{-0.5\height}{
	\begin{tikzpicture}
		\node (A) at (0,2) {$A$};
		\node (B) at (1.1,2) {$A$};
		\node (A') at (0,1) {$A$};
		\node (B') at (1.1,1) {$B$};
		\node (A'') at (0,0) {$B$};
		\node (B'') at (1.1,0) {$B$};
		%
		\path[->,font=\scriptsize,>=angle 90]
			(A) edge node[left]{$1$} (A')
			(A') edge node[left]{$f$} (A'')
			(B) edge node[right]{$f$} (B')
			(B') edge node[right]{$1$} (B'')
			(A) edge node[above]{$U_A$} (B)
			(A') edge  (B')
			(A'') edge node[below]{$U_B$} (B'');
		%
	%	\draw (0.5,1.925) -- (0.5,2.075);
		\draw[line width=2mm,white] (0.5,.925) -- (0.5,1.075);
	%	\draw (0.5,-.075) -- (0.5,.075);
		\node () at (0.5,0.5) {\scriptsize{$\Downarrow$}};
		\node () at (0.5,1.5) {\scriptsize{$\Downarrow$}};
		\node () at (0.5,1) {\scriptsize $\widehat{f}$};
	\end{tikzpicture}
	}
	%
	\raisebox{-0.5\height}{=}
	%
	\raisebox{-0.5\height}{
	\begin{tikzpicture}
		\node (A) at (0,1) {$A$};
		\node (B) at (1,1) {$A$};
		\node (A') at (0,0) {$B$};
		\node (B') at (1,0) {$B$};
		%
		\path[->,font=\scriptsize,>=angle 90]
		(A) edge node[left]{$f$} (A')
		(B) edge node[right]{$f$} (B')
		(A) edge node[above]{$U_A$} (B)
		(A') edge node[below]{$U_B$} (B');
		%
		%\draw (0.5,.925) -- (0.5,1.075);
		%\draw (0.5,-.075) -- (0.5,.075);
		\node () at (0.5,0.5) {\scriptsize{$\Downarrow U_f$}};
	\end{tikzpicture}
	}
	%
	\raisebox{-0.5\height}{\text{   and   }}
	%
	\raisebox{-0.5\height}{
	\begin{tikzpicture}
		\node (A) at (0,1) {$A$};
		\node (A') at (0,0) {$A$};
		\node (B) at (1,1) {$A$};
		\node (B') at (1,0) {$B$};
		\node (C) at (2,1) {$B$};
		\node (C') at (2,0) {$B$};
		%
		\path[->,font=\scriptsize,>=angle 90]
			(A) edge node[left]{$1$} (A')
			(B) edge node[left]{$f$} (B')
			(C) edge node[right]{$1$} (C')
			(A) edge node[above]{$U_A$} (B)
			(B) edge node[above]{$\widehat{f}$} (C)
			(A') edge node[below]{$\widehat{f}$} (B')
			(B') edge node[below]{$U_B$} (C');
		%
	%	\draw (1.5,0.925) -- (1.5,1.075);
	%	\draw (1.5,0.925) -- (1.5,1.075);
	%	\draw (0.5,.925) -- (0.5,1.075);
	%	\draw (0.5,-.075) -- (0.5,.075);
		\node () at (0.5,0.5) {\scriptsize{$\Downarrow$}};
		\node () at (1.5,0.5) {\scriptsize{$\Downarrow$}};
	\end{tikzpicture}
	}
	%
	\raisebox{-0.5\height}{=}
	%
	\raisebox{-0.5\height}{
	\begin{tikzpicture}
		\node (A) at (0,1) {$A$};
		\node (B) at (1,1) {$B$};
		\node (A') at (0,0) {$A$};
		\node (B') at (1,0) {$B$};
		%
		\path[->,font=\scriptsize,>=angle 90]
			(A) edge node[left]{$1$} (A')
			(B) edge node[right]{$1$} (B')
			(A) edge node[above]{$\widehat{f}$} (B)
			(A') edge node[below]{$\widehat{f}$} (B');
		%
	%	\draw (0.5,.925) -- (0.5,1.075);
	%	\draw (0.5,-.075) -- (0.5,.075);
		\node () at (0.5,0.5) {\scriptsize{$\Downarrow \id_{\widehat{f}}$}};
	\end{tikzpicture}
	}
	\end{equation}
  A \textbf{conjoint} of $f$, denoted $\fchk \maps B\to A$, is a
  companion of $f$ in the double category $\lD^{h\cdot\mathrm{op}}$
  obtained by reversing the horizontal 1-cells, but not the vertical
  1-morphisms, of \lD.
\end{defn}
\noindent
In a pseudo double category, the second equation above requires an insertion of unit isomorphisms to make sense due to horizontal composition only holding up to isomorphism.
\begin{defn}
  We say that a double category is \textbf{fibrant} if every vertical
  1-morphism has both a companion and a conjoint and \define{isofibrant} if every vertical 1-isomorphism has both a companion and a conjoint.
\end{defn}
\section{Acknowledgements}
Daniel Cicala, person that Christina emailed about lax slice category?
\begin{thebibliography}{100}

\bibitem{BC} J.\ C.\ Baez and K.\ Courser, Coarse-graining open Markov processes. Available as \href{https://arxiv.org/abs/1710.11343}{arXiv:1710.11343}.

\bibitem{BC2} J.\ C.\ Baez and K.\ Courser, Structured cospans. In preparation.

\bibitem{BCR} J.\ C.\ Baez, B.\ Coya and F.\ Rebro, Props in circuit theory. In preparation. THIS IS FINISHED!!!

\bibitem{BF} J.\ C.\ Baez and B.\ Fong, A compositional framework for passive linear networks. Available as \href{http://arxiv.org/abs/1504.05625}{arXiv:1504.05625}.

\bibitem{BFP} J.\ C.\ Baez, B.\ Fong and B.\ Pollard, A compositional framework for Markov processes, \textsl{Jour. Math. Phys.} \textbf{57} (2016), 033301. Available as \href{http://arxiv.org/abs/1508.06448}{arXiv:1508.06448}.

\bibitem{BP} J.\ C.\ Baez and B.\ Pollard, A compositional framework for chemical reaction networks, in preparation.

\bibitem{Brown1} R.\ Brown and C.\ B.\ Spencer, Double groupoids and crossed modules, 
\textsl{Cah.\ Top.\ G\'eom.\ Diff.} \textbf{17} (1976), 343--362.

\bibitem{Brown2} R.\ Brown, K.\ Hardie, H.\ Kamps and T.\ Porter, The homotopy double groupoid of a Hausdorff space, \textsl{Th.\ Appl.\ Categ.} \textbf{10} (2002), 71--93.

%\bibitem{Be} J.\ B\'enabou, Introduction to bicategories, in {\sl Reports
%of the Midwest Category Seminar}, Lecture Notes in Mathematics, vol.\ \textbf{47}, Springer, Berlin, 1967, pp.\ 1--77.

%\bibitem{Brown1} R.\ Brown and C.\ B.\ Spencer, Double groupoids and crossed modules, 
%\textsl{Cah.\ Top.\ G\'eom.\ Diff.} \textbf{17} (1976), 343--362.

%\bibitem{Brown2} R.\ Brown, K.\ Hardie, H.\ Kamps and T.\ Porter, The homotopy double groupoid of a Hausdorff space, \textsl{Th.\ Appl.\ Categ.} \textbf{10} (2002), 71--93. 

%\bibitem{CC} D.\ Cicala and K.\ Courser, Spans of cospans in a topos. Available as \href{https://arxiv.org/abs/1707.02098}{arXiv:1707.02098}.

\bibitem{Cour} K.\ Courser, A bicategory of decorated cospans, \emph{Theor.\ Appl.\ Cat.} \textbf{32} (2017), 995--1027. Also available as \href{https://arxiv.org/abs/1605.08100}{arXiv:1605.08100}.

%\bibitem{Ehresmann63} C.\ Ehresmann, Cat\'egories structur\'ees III: Quintettes et applications covariantes,  \textsl{Cah.\ Top.\ G\'eom.\ Diff.} \textbf{5} (1963), 1--22.

%\bibitem{Ehresmann65} C.\ Ehresmann, {\sl Cat\'egories et Structures,} Dunod, Paris, 1965.

\bibitem{Ehresmann63} C.\ Ehresmann, Cat\'egories structur\'ees III: Quintettes et applications covariantes,  \textsl{Cah.\ Top.\ G\'eom.\ Diff.} \textbf{5} (1963), 1--22.

\bibitem{Ehresmann65} C.\ Ehresmann, {\sl Cat\'egories et Structures,} Dunod, Paris, 1965.

\bibitem{Fong} B.\ Fong, Decorated cospans, \emph{Theor.\ Appl.\ Cat.} \textbf{30} (2015), 1096--1120. Also available as \href{http://arxiv.org/abs/1502.00872}{arXiv:1502.00872}.

%\bibitem{GP1} M.\ Grandis and R.\ Par\'e, Limits in double categories, \textsl{Cah.\ Top.\ G\'eom.\ Diff.} \textbf{40} (1999), 162--220.

%\bibitem{GP2} M.\ Grandis and R.\ Par\'e, Adjoints for double categories, 
% \textsl{Cah.\ Top.\ G\'eom.\ Diff.} \textbf{45} (2004), 193--240.

%\bibitem{Haug} R.\ Haugseng, Iterated spans and ``classical" topological field theories. Available as \href{https://arxiv.org/abs/1409.0837}{arXiv:1409.0837}.

%\bibitem{Hoff} A.\ Hoffnung, Spans in 2-categories: a monoidal tricategory. Available as \href{http://arxiv.org/abs/1112.0560}{arXiv:1112.0560}.

\bibitem{JM} J.\ C.\ Baez and J.\ Master, A compositional framework for Petri nets. In preparation.

\bibitem{LS} E.\ Lerman and D.\ Spivak, An algebra of open continuous time dynamical systems and networks. Available as \href{http://arxiv.org/abs/1602.01017}{arXiv:1602.01017}.

%\bibitem{Lack} S.\ Lack, Limits for lax morphisms, \emph{Applied Categorical Structures} $\bold{30}$ (2005), 189--203. Available %at \href{http://maths.mq.edu.au/~slack/papers/talgl.pdf}{http://maths.mq.edu.au/$\sim$slack/papers/talgl.pdf}.

   %\bibitem{Lerm} E.\ Lerman and D.\ Spivak, An algebra of open continuous time dynamical systems and networks. Available as %%\href{http://arxiv.org/abs/1602.01017}{arXiv:1602.01017}.

%   \bibitem{ML} S.\ Mac Lane, {\sl Categories for the Working Mathematician},
%     Springer, Berlin, 1998.

%\bibitem{Pol} B.\ Pollard, Open Markov processes: A compositional perspective on non-equilibrium steady states in biology, %%%\emph{Entropy} $\bold{18}$ (2016), 140. Available as \href{http://arxiv.org/abs/1601.00711}{arXiv:1601.00711}.

%\bibitem{Nie}
%S.~Niefield,
%Span, cospan, and other double categories.
%\textsl{Theory Appl.\ Categ.}
%\textbf{26} (2012), 729--742.
%Available as \href{https://arxiv.org/abs/1201.3789}{arXiv:1201.3789}.

%\bibitem{Panan} F.\ Clerc, H.\ Humphrey and P.\ Panangaden, Bicategories of Markov processes,  to appear.

%\bibitem{RSW} R.\ Rosebrugh, N.\ Sabadini and R.\ F.\ C.\ Walters, Generic commutative separable algebras and cospans of graphs, \textsl{Theory Appl. Categ.} \textbf{15} (2005), 164--177. Available at \href{http:/.www.tac.mta.ca/tac/volumes/15/6/15-06.pdf}{http:/.www.tac.mta.ca/tac/volumes/15/6/15-06.pdf}.

%\bibitem{Reb} F.\ Rebro, Constructing the bicategory Span$_{2}(\mathrm{A})$. Available as \href{https://arxiv.org/abs/1501.00792}{arXiv:1501.00792}.

\bibitem{Shul} M.\ Shulman, Constructing symmetric monoidal bicategories. Available as \href{http://arxiv.org/abs/1004.0993}{arXiv:1004.0993}.

\bibitem{Shul2} M.\ Shulman, Framed bicategories and monoidal fibrations. Available as \href{https://arxiv.org/abs/0706.1286}{arXiv:0706.1286}.

%\bibitem{Stay} M.\ Stay, Compact closed bicategories.  Available as \href{http://arxiv.org/abs/1301.1053}{arXiv:1301.1053}.

\bibitem{Yass} A.\ Yassine, Open Systems in Classical Mechanics. Available as \href{https://arxiv.org/abs/1710.11392}{arXiv:1710.11392}.

\end{thebibliography}
\end{document}